

\documentclass{amsart}


\usepackage[applemac]{inputenc}




\addtolength{\hoffset}{-12mm}
\addtolength{\voffset}{-10mm}
\addtolength{\textheight}{20mm}

\begin{document}


\centerline{\bf{Hyperbolic PDE-equations.}}

\medskip

\noindent
{\bf{Introduction.}}
We shall expose
a   result from Friedrichs' article \emph{xxx}
dealing with boundary value problems
for linear 
symmetric hyperbolic systems.
The  proof of our main result in Theorem 0.1
teaches the usefulness of regarding 
densely defined but unbounded linear operators on
Hilbert spaces. 
Theorem 0.1
is announced for a symmetric hyperbolic first order system
in any number of variables.
The proof involves quite  technical
steps.
are  when one restricts  to
the case of two variables and a single scalar function.
For the reader's convenience we therefore include  a self-contained
proof  in the restricted case  for systems in 
two variables and a single scalar function.
Here
the
Hilbert space methods are  transparent, and yet
the methods which are used in this restricted situation
are crucial, i.e. the general case is  verbatim the same except for
various technical points.
Before we announce the result by Firedrichs
we recall some background about hyperbolic equations.

\medskip


\noindent
A classic result due to Hadamard 
gives a  vanishing principle for well-posed boundary
value problems.
With coordinates
$(x,s)= (x_1,\ldots,x_n,s)$ in
${\bf{R}}^{n+1}$
we consider a differential operator of the form
\[
Q(x,s,\partial_x,\partial_s)=
\partial_s^p+\sum_{\nu=0}^{p-1}\,
P_\nu(x,s,\partial_x)\cdot \partial_s^\nu
\]
where $\{P_\nu(x,s,\partial_x)$ are differential operators which
are independent of $\partial_s$ and  coefficients in
$C^\infty({\bf{R}}^{n+1})$ 
which in general are complex-valued.
The Cauchy problem is well posed in Hadamard's sense
if there
to
every $f(x)\in C^\infty({\bf{R}}^n)$
exists a unique $C^\infty$-function $g(x,s)$
in the half-space $\{s\geq 0\}$ such that
$Q(g)(x,s)=0$ when $s>0$
and
on $s=0$ one has
\[
\partial_s^\nu(g)(x,0)=0\,\colon 0\leq \nu\leq p-1
\quad\colon\quad 
\partial_s^p(g)(x,0)=f(x)
\]

\medskip

\noindent 
Under the hypothesis that Cauchy's problem is well-posed one has:
\medskip

\noindent
{\bf{Hadamard's Theorem.}}
\emph{If $K$ is a compact subset in ${\bf{R}}^{n+1}$
there exists a compact set $K$ in the $x$-space such that
thre unique solution $f$ whose Cauchy data on $s=0$ vanishes on $k$
must vanish on $K$.}
\medskip

\noindent This result 
follows easily from Baire's category theorem. The reader may consult
[D-S: Volume 2: page 1649-1652] for details.
\medskip

\noindent
We shall not enter an extensive discussion about
conditions for a PDE-operator to be hyperbolic.
The reader may consult the text-book
by Petrowsky for an account where  many examples of
hyperbolic, elliptic and parabolic equations
appear.

\medskip


\noindent
{\bf{An ill-posed equation.}} 
Let $n=1$ and consider the $2\times 2$-matrices
\[ 
A_1(x,s)=
\begin{pmatrix} -e^{-x}&0\\0&1 \end{pmatrix}\quad\colon
B(x,s)= \begin{pmatrix} 0&1\\0&0 \end{pmatrix}\quad\colon
\]
One seeks pairs of functions $(f_1(x,s),f_2(x,s)$
which satisfy the first order system:
\[
\frac{\partial f_1}{\partial s}=
-e^{-x}\cdot \frac{\partial f_1}{\partial x}+f_2
\quad\colon\,
\frac{\partial f_2}{\partial s}=
\frac{\partial f_2}{\partial x}
\]
For any function $C^\infty$-function $h$ of a single variable 
we see that
\[ 
f_1(x,s)= h(e^x-s)\quad\colon f_2=0
\]
solves the system above and here $f_1(x,0)= h(x)$.
Consider the singleton set $\{0,1\}$ in ${\bf{R}}^2$.
Then $f(A)=1$ for all $h$-functions
such that $h(0)=0$. Fix a test-function $\phi(t)$ on the real $t$-line
where $\phi(0)=1$ while $\phi(t)=0$ if $|t|\geq 1/2$.
For every positive integer $N$
we take $h(t)= \phi(Nt)$.
If the Cauchy problem is well posed we get the unique solution
with $f_1= h(s-e^x)$ and now
\[ 
f_1(x,0)= \phi (e^{Nx})
\] 
Since the support of $\phi$ is contained in $[-1/2,1/2]$
we see that $f_1(x,0)\neq 0$ entails that
\[
e^{Nx}\leq 1/2\implies x\leq -N\cdot \log 2
\]
Since $N$ can be arbitrary large this violates Hadamard's
vanishing principle and hence the Cauchy problem for this system is
not well-posed.






\medskip


\centerline{\bf{The  wave operator.}}
\medskip

\noindent
With two real variables we consider the PDE-operator
\[ 
P=\frac{\partial^2}{\partial x_1^2}-
\frac{\partial^2}{\partial x_2^2}
\]
To each pair of functions $g(x_1),h(x_1)$ one seeks
$f(x_1,x_2)$ such that $P(f)=0$ and
Cauchy's boundary value conditions:
\[
f(x_1,0)= g(x_1)\quad\colon
\frac{\partial f}{\partial x_2}(x_1,0)= h(x_1)\tag{i}
\]


\medskip


\noindent
This boundary value problem corresponds to
a first order system where one seeks a pair
$f_1(x_1,x_2)$ and $f_2(x_1,x_2)$ such that
\[
\frac{\partial f_1}{\partial x_1}=-
\frac{\partial f_1}{\partial x_2}+f_2
\quad\colon\quad 
\frac{\partial f_2}{\partial x_1}=
\frac{\partial f_2}{\partial x_2}\tag{ii}
\]
with boundary values 
\[
f_1(x_1,0)= g(x_1)\quad\colon\quad f_2(x_1,0)=g'(x_1)+h(x_1)\tag{iii}
\]



\noindent
{\bf{Exercise.}}
Show that if $f$ solves (i)
then the boundary value
system is solved by the pair
\[
 f_1=f\quad\colon f_2= \frac{\partial f}{\partial x_1}+
\frac{\partial f}{\partial x_2}
\]
Show also that if $f_1,f_2$ solves the system then
$f=f_1$ solves
the original equation.
Next, consider the matrices
\[
A=
\begin{pmatrix}  -1&0\\0&1\end{pmatrix}\quad\colon\quad
B=
\begin{pmatrix}  0&0\\1&0
\end{pmatrix}
\]
Then the system can be written in matrix form as
\[
\frac{\partial}{\partial x_1}
\begin{pmatrix} f_1\\f_2\end{pmatrix}
=A\,\frac{\partial}{\partial x_2}
\begin{pmatrix}  f_1\\f_2\end{pmatrix}+ 
B\begin{pmatrix}  f_1\\f_2\end{pmatrix}
\]

\medskip

\noindent
This clarifies that the  orginal equation is a first order symmetric
system to be dsecribed in � xx,
\medskip

\noindent
{\bf{Higher order systems.}}
Consider systems
with  $n+1$ many  variables
$x_1,\ldots,x_n,s$ where the variable $s$ is distinguished.
In a first order scalar system 
one seeks a function $f(x,s)$ such that
\[
\frac{\partial f}{\partial s}=
\sum_{j=1}^{j=n}\, a_j(x,s)\cdot 
\frac{\partial f}{\partial x_j}+b(x,s)
\]
satisfying  the boundary condition
\[ f(x,0)= g(x)
\] 
where the $g$ function is given  in the $n$-dimensional $x$-space.
In a vector-valued system of order $m\geq 2$
the $a$-functions are replaced by $m\times m$-matrices and $b$ by
some $m\times m$-matrix. Here 
one seeks a vector-valued function 
$f=(f_1,\ldots,f_m)$ such that
\[
\frac{\partial}{\partial s}
\begin{pmatrix} f_1\\...\\f_m\end{pmatrix}=
\sum_{j=1}^{j=n}\, A_j(x,s)\cdot 
\frac{\partial }{\partial x_j}\begin{pmatrix} f_1\\...\\f_m\end{pmatrix}+
B(x,s)\begin{pmatrix} f_1\\...\\f_m\end{pmatrix}\tag{*}
\]
The boundary conditions are expressed by an $m$-tuple of functions
$\{g_\nu(x)\}$ such that $f_\nu(x,0)= g_\nu(x)$ hold for each $\nu$.
The $A$-matrices and the $B$-matrix are in general complex valued.
A famous  example is:
\medskip

\noindent
{\bf{Maxwell's equations of electrodynamics}}.
Let $n=m=3$ where  $\{A_\nu\}$ are constant $3\times 3$-matrices
\[
A_1=\begin{pmatrix} 0&0&0\\ 0&0&-i\\ 0&i&0 \end{pmatrix}
\quad\colon\, 
A_2=\begin{pmatrix} 0&0&i\\ 0&0&0\\ -i&0&0 \end{pmatrix}\quad\colon
A_3=\begin{pmatrix} 0&-i&0\\ i&0&0\\ 0&0&0 \end{pmatrix}
\]
Moreover $B=0$ and 
one seeks a vector-valued function $f_1,f_2,f_3$
which satisfies (*) and boundary value conditions $f_\nu(x,0)= g_\nu(x)$
\medskip



\centerline{\bf{The symmetric case.}}
\medskip


\noindent
The system (*) is  symmetric if
the matrices $\{A_\nu(x,s)\}$ are Hermitian for each
$1\leq\nu\leq n$.
No special condition is imposed on $B$, i.e. it can be
an arbitrary complex $m\times m$-matrix.
Notice that  the $A$-matrices
in Maxwell's equations are hermitian.
The following result is due to  Friedrichs:
\bigskip

\noindent
{\bf{0.1 Theorem.}}
\emph {Assume that the $A$-matrices are hermitian and that
the matrix elements of $A_1,\ldots,A_n$ and  $B$ are
bounded functions in ${\bf{R}}^{n+1}$.
Then Cauchy's boundary value problem has a unique $C^\infty$solution
$f=(f_1,\ldots,f_m)$ for every $m$ tuple
$g=(g_1,\ldots,g_m)$ of $C^\infty$-functions in the $n$-dimensional $x$-space.}

\medskip


\noindent
{\bf{Remark.}}
The example after Hadamard's result above 
shows that this boundedness is needed in order that
the  Cauchy problem  is well-posed.

\medskip

\noindent
{\bf{0.2 On the uniqueness.}}
Let us illustrate why the condition that the
$A$-matrices are hermitian
gives a certain vanishing principle.
Let $\Omega$ be a bounded open set in the $n$-dimensional $x$-space
and $-s_*<s<s_*$
is some open $s$-interval.
Let $\{A_j(x,s)\}$ be Hermitian $m\times m$-matrices whose elements as
well as their first order partial
derivatives are 
bounded $C^\infty$-functions
in $\Omega\times (-s_*,s_*)$. Similarly, assume
that $B(x,s)$ is  an $m\times m$-matrix whose elements
are bounded
$C^\infty-$-functions in $\Omega$.
Let $f=(f_1,\ldots,f_m)$ be a vector-valued solution to the system (*)
where 
\[
f(x,0)=0\quad\colon x\in\Omega\tag{0.2.1}
\]
and assume 
there is a compact subset $K$ of $\Omega$ and
$f=0$ in $(\Omega\setminus K)\times (-s_*,s_*)$.
Then (0.2.1) entails that
$f=0$ in 
$\Omega\times (-s_*,s_*)$.
To prove this we introduce the function
\[ 
J(s)=
\int_\Omega\, |f(x,s)|^2\,dx
\]
where $|f|^2=\sum\, f_\nu\cdot \bar f_\nu$.
Taking the derivative with respect to $s$ we get
\[
\frac{dJ}{ds}=2\cdot \mathfrak{Re}\,
\sum_{j=1}^{j_m}\int\, \partial_s(f_\nu)\cdot \bar f_\nu\, dx\tag{i}
\]
Since $f$ satisfies (*) we have
\[
\sum_{j=1}^{j_m}\int\, \partial_s(f_\nu)\cdot \bar f_\nu=
\sum_{j=1}^{j=n}\, \langle A_j(\frac{\partial f}{\partial x_j}),f\rangle+
\langle B(f),f\rangle
\]
Since $f_\nu(x,s)$
vanish when $x\in\Omega\setminus K$  
Stokes theorem gives
\[
0=\int\ \partial_{x_j}(\langle A_j(f),f\rangle)\, dx\quad\colon 1\leq j\leq n \tag{ii}
\]
Rules for differentiation identifies for each $j$
the integrand with
\[
\langle  \frac{\partial A_j}{\partial x_j}(f),f\rangle
+
\langle A_j(\partial_{x_j}(f),f\rangle+
\langle A_j(f), \partial_{x_j}(f)\rangle
\]
Since  $A_j$ is hermitian we have
\[
\mathfrak{Re}\ \langle A_j(\partial_{x_j}(f),f\rangle=
\mathfrak{Re}\, 
\langle A_j(f), \partial_{x_j}(f)\rangle\tag{iii}
\]
Hence (ii) and (iii) give
\[
\mathfrak{Re} \int\,
\langle A_j(\frac{\partial f}{\partial x_j}),f\rangle\, dx=-
\frac{1}{2}\int
\langle  \frac{\partial A_j}{\partial x_j}(f),f\rangle\,dx
\]
\medskip

\noindent
Introduce the matrix-valued function
\[
\text{div}(A)=
\sum_{j=1}^{j=n}\,\frac{\partial A_j}{\partial x_j}
\]
From  the above we get
the equation
\[ 
\frac{dJ}{ds}=\mathfrak{Re}\, \int\, \bigl[-\frac{1}{2}
\langle 
\text{div}(A)(f),f\rangle+
\langle 
B(f),f\rangle\,\bigr]\, dx\tag{iv}
\]
By assumption the elements of the matrices
$\{A_j\}$  and of $B$ as well as $\{\frac{\partial A_j}{\partial x_j}\}$.
are bounded $C^\infty$-functions.
Hence  (iv) and the  Cauchy-Schwarz inequality gives a constant $C$
such that the absolute
value in
the right hand side in (iv) is estimated above by 
\[
C\cdot \int\, |f(x,s)|^2\ dx\quad\colon -s_*\leq s< s^*
\]
It follows that 
\[
|\frac{dJ}{ds}|\leq C\cdot J(s)\quad\colon 0\leq s\leq s^*
\]
At the same time (i) means that $J(0)=0$. Hence
Picard's uniquneness theorem to be exposed  in � XX
implies that 
if $J(s)=0$ when $-s_*<s<s_*$, i.e.
\[
J(x,s)=0\quad
\colon (x,s)\in \{|x|\leq r\}\times [0,s^*]\tag{0.3.1}
\]
So we have  
\[ 
\int_\Omega\, |f(x,s)|^2\, dx=0
\]
and  $\{f_\nu\}$ are continuous functions 
they vanish identically in
$\Omega\times(-s_*,s_*)$ as requested.



\medskip

\noindent
{\bf{0.3 A semi-global uniqueness result.}}
Let $\{A_j(x,s)\}$ be Hermitian matrices defined in the whole of
${\bf{R}}^{n+1}$
whose elements are bounded $C^\infty$-functions. 
Similarly $B(x,s)$ is defined in the whole of
${\bf{R}}^{n+1}$. 

\medskip

\noindent
{\bf{0.3.1 Proposition.}}
\emph{There exists a positive number $\rho$ which only depends on
the
matrices above such that if $R>0$ and
a vector valued $C^\infty$-function $f(x,s)$
is defined in the ball $\{|x|^2+s^2<R^2\}$
where it is a solution to (*) and satisfies}
\[ 
f(x,0)=0\quad\colon |x|<R
\]
\emph{Then it follows that}
\[ 
f(x,s)= 0\quad\colon x^2+s^2<\rho\cdot R^2
\]

\medskip


\noindent
\emph{Proof.}
Choose a test-function $\psi(x)$ in ${\bf{R}}^n$
such that $\psi=1$ when $|x|\leq 1$ and
vanishes when $|x|>3/2$ while the values stay in $[0,1]$.
Set
\[
\mu =\max_{1\leq j\leq m}\, \sup_{(x,s)}\, ||A_j(x,s)||\tag{i}
\]
where the supremum
is taken over all $(x,s)$ in ${\bf{R}}^{n+1}$
and we have taken the Hibert-Schmidt norms of
the $A$-matrices.
With $\epsilon>0$
we set $\phi(x)=\psi(\epsilon\cdot x)$
and construct the following $m\times m$-matrices
\[
H(x,s)= \sum_{j=1}^{j=n}\ \frac{\partial \phi}{\partial x_j}
(x)\cdot
A_j(x,s)
\]
\[ 
\widehat{A_j}(x,s)= \phi(x)\cdot
A_j(x,\phi(x)s)\quad\colon\quad
\widehat{B}(x,s)= \phi(x)\cdot
B(x,\phi(x)s)
\]
Put
\[ 
F(x,s)= f(x,\phi(x)s)
\]
Since $0\leq \phi(x)\leq 1$
hold for all $x$ it follows that
$F$ is defined
in 
$\{|x|^2+s^2<R^2\}$ and
the construction of $\phi$  gives
\[ 
F(x,s)= f(x,s)\quad\colon |x|<\epsilon^{-1}\tag{ii}
\]

\noindent
Rules for differentiation  show that
$F$ satisfies the system
\[
(E-H(x,s))\partial_s(F)=\sum_{j=1}^{j=n} \widehat{A_j}(x,s)\partial_{x_j}(F)+
\widehat{B}(x,s)F\quad\colon\, x^2+s^2<R^2
\]
where $E$ is the identity operator.
\medskip

\noindent
{\bf{A choice of $\epsilon$.}}
The partial derivatives of the test-function $\psi$ are bounded
by some constant $C$ 
and we set
\[ 
\epsilon^*=\frac{1}{2n\cdot C\mu}
\]
It follows that 
\[
|\frac{\partial \phi}{\partial x_j}|= \epsilon\cdot |\frac{\partial \psi}{\partial x_j}|
\leq \frac{1}{2n\mu}
\]
By (i) this entails that  
\[ 
\sup_{(x,s)}||H(x,s]|\leq \frac{1}{2}\tag{iii}
\]
Next, the support of $\phi$ is contained in the ball 
$\{|x|\leq \frac{3}{2\epsilon^*}$ so
the vanishing
in (xx)  entails that
$F(x,s)=0$  when
\[
\frac{3}{2\epsilon^*}\leq |x|<R
\]
By the condition (xx) $R\geq R_*$ entails that
\[
\frac{3}{2\epsilon^*}=3nC\mu= R_*/2\leq R/2
\]
So $R\geq R_*$  implies that
\[ 
F_\epsilon(x,s)=0\quad \colon
\frac{R}{2}\leq |x|<\sqrt{R^2-s^2}\tag{v}
\]
Now (iv) implies that
the hermitian matrix
$E-H(x,s)$ is invertible and
by (iii) $F$ satisfies  system as in (0.2).
The vanishing in (v)
therefore implies that
$F_\epsilon (x,s)=0$ hold when
$x^2+s^2<R^2$.

\medskip

FINISH


\newpage

\centerline{\bf{� 1: Symmetric hyperbolic systems.}}


\bigskip

\noindent
The main result in this section appears in
Theorem 1.xx.
Before it can be announced
we need
several preliminaries.
We will  study periodic functions. 
Let $n$ be a positive integer and 
consider the $(n+1)$-dimensional torus $T^{n+1}$
with variables $(x,s)= (x_1,\dots,x_n,s)$.
Denote by $C^\infty(T^{n+1})$
the space of complex-valued $C^\infty$-functions which
are $2\pi$-periodic in all the variables.
Passing to the $L^2$-norm
the closure of these functions give
the complex  Hilbert space $L^2(T^{n+1})$
whose vectors are complex-valued functions
$f(x,s)$ which are square integrable on the $(n+1)$-dimensional
$2\pi$-periodic torus. 
Next,
to each multi-index $\alpha=(\alpha_1,\ldots,\alpha_{n+1})$
one associates the differential operator
\[ 
\partial^\alpha=\partial_{x_1}^{\alpha_1}\cdots
\partial_{x_n}^{\alpha_n}\cdot \partial_{s}^{\alpha_{n+1}}
\]
If $k$ is a positive integer
an inner product is defined on
$C^\infty(T^{n+1})$
by
\[ 
\langle f,g\rangle_{(k)}=
\sum_{|\alpha|\leq k}\, \int\, \partial^\alpha(f)\cdot \overline{\partial^\alpha(g)}\, dxds\tag{1.1}
\]
Passing to the closure we obtain a Hibert space denoted by
$H^{(k)}$ whose 
elements are $L^2$-functions 
$g(x,s)$ such that the distribution derivatives
$\partial^\alpha(g)$ are square integrable when
$|\alpha|\leq k$.
From � XX we recall:

\medskip

\noindent
{\bf{The Fourier-Sobolev  Lemma.}} \emph{If $k\geq xx$
 every $g\in H^{(k)}$
is a periodic function of class $C^1$ at least on
${\bf{T}}^{n+1}$.}


\medskip

\noindent
More generally, if $m\geq 2$ we consider vector-valued functions
$f=(f_1,\ldots,f_m)$
and get the Hilbert space $H^{(k)}[m]$
whose vectors are $m$-tuples of functions in
$H^{(k)}$.
With $f=(f_1,\ldots,f_m)$ and $g=(g_1,\ldots,g_m)$
the inner product is defined as in
(1,1):
\[
\langle f,g\rangle_{(k)}=\sum_{\nu=1}^{\nu=m}
\sum_{|\alpha|\leq k}\, \int\, \partial^\alpha(f_\nu)\cdot \overline{\partial^\alpha(g_\nu)}\, dxds
\tag{1.2}
\]
With $m\geq 1$ we consider
a matrix-valued functions
$\{A_j(x,s)\ldots A_n(x,s)\}$
where each $A_j(x,s)$ is an $m\times m$-matrix whose elements are
periodic complex-valued $C^\infty$-functions on
$T^{n+1}$.
Let $B(x,s)$ be another matrix-valued functions whose elements also are
periodic and of class $C^\infty$.
Set 
\[
P(x,s,\partial_x,\partial_s)=
E_m\cdot \partial_s-\sum_{j=1}^{j=n}\ A_j(x,s)\cdot \partial_{x_j}+ B(x,s)\tag{1.3}
\]
This differential operator acts on
vector-valued functions $f$.
Identifying
the space of vector-valued and periodic $C^\infty$-functions
with a subspace of $H^{(k)}[m]$
one has the linear map
\[ 
P\colon f\to P(f)
\] 
from $C^\infty[m]$
into
$H^{(k)}[m]$.
Keeping $k$ and $m$ fixed
we denote this linear map by $T_0$. It means that
$T_0$ is densely defined linear operator
on
$H^{[k)}[m]$
whose  domain  of definition $\mathcal D(T_0)=
C^\infty_[m]$.
We have also
the densely defined linear operator $T_1$
where
\[
 \mathcal D(T_1)= \{f\in H^{(k)}[m]\,\colon P(f)\in  H^{[k)}[m]\}
\]
By the general result in � XX the graph of $T_1$
taken in
the product 
$H^{(k)}[m]\times H^{(k)}[m]$ is closed.
Next, since $T_0$ is densely defined there exists
the adjoint operator
$T_0^*$. By definition 
$\mathcal D(T_0^*)$ consists of 
vectors $g\in  H^{(k)}[m]$ for which there exists a constant $C(g)$ such that
\[
|\langle T_0(f),g\rangle|\leq C(g)\dot ||f||_k
\quad\colon f\in\mathcal D(T_0)
\]
and for such $g$-vectors we get
a unique vector $T_0^*(g)$ such that
\[
\langle T_0(f),g\rangle=\langle f,T_0^*(g)\rangle
\]
\medskip

\noindent
{\bf{1.4 The case when
$\{A_j\}$ are hermitian.}}
An $m\times m$-matrix $A(x,s)=\{a_{\nu,\mu}(x,s)\}$
whose elements are periodic $C^\infty$-functions
is hermitian if
\[
a_{\mu,\nu}(x,s)=\overline {a_{\nu,\mu}(x,s)}
\]
hold for all pairs $1\leq \nu,\mu\leq m$.
\medskip

\noindent
{\bf{1.5 Proposition.}}
\emph{If  $A_1,\ldots,A_n$ are hermitian it follows that
$\mathcal D(T_0^*)= \mathcal D(T_1)$
and there exists a bounded and self-adjoint linear operator
$\mathcal B$ on $H^{(k)}[m]$ such that}
\[
 T_0^*+T_1=\mathcal B\tag{*}
 \]
 \medskip
 
 \noindent
 Before we enter the proof we need
 some constructions.
Repeated use of Stokes Theorem gives the equality
below for every pair of functions $f,g$ in $C^\infty(T^{n+1})$
and every multi-index $\alpha$:
\[
(-1)^{|\alpha|}\cdot \int\,  \partial^{2\alpha}(f)\cdot \overline{g}\ dxds=
\int\,  \partial^{\alpha}(f)\cdot \overline{\partial^\alpha(g)}\ dxds
\]
More generally, let
$Q= Q(x,s,\partial_x,\partial_s)$ be a differential operator.
With
$Q$ given as
\[ 
Q= \sum\, q_\alpha(x,s)\cdot \partial^\alpha
\]
where  $q_\alpha\in C^\infty(T^{n+1})$
one gets the differential operator
\[ 
Q^*=\sum\,(-1)^\alpha\cdot \partial^\alpha\circ q_\alpha(x,s)
\]
where  $\partial^\alpha\circ q_\alpha(x,s)$ is the product taken in the ring of
differential operators with
$\{q_\alpha(x,s)\}$ regarded as zero-order differential operators.
Stokes theorem  gives
\[
\int Q(f)\cdot \bar g\, dxds=
\int f\cdot Q^*(\bar g)\,dxds\tag{1.6}
\]
Let us write out  
\[ 
Q^*=\sum r_\alpha(x,s)\cdot \partial^\alpha
\]
We take the complex conjugates of the $r$-functions and put
\[ 
\overline{Q^*}=\sum \bar r_\alpha(x,s)\cdot \partial^\alpha
\]
Using the hermitian inner product on
$L^2(T^{n+1})$ we can express (1.6) by the equation
\[ 
\langle Q(f),g\rangle=
\langle f,\overline{Q^*}(g)\rangle\tag{1.7}
\]
\medskip

\noindent
{\bf{1.8 The $\Gamma$-operator.}}
Let us  introduce the differential operator
\[
\Gamma=
\sum_{|\alpha|\leq k} 
(-1)^{|\alpha|}\cdot  \partial^{2\alpha}
\]
If $m\geq 2$ we denote by $\Gamma_m$
the operator given by the diagonal $m\times$-matrix where
whose diagonal elements are $\Gamma$.
Stokes theorem
entails
that if $f$ and $g$ is a pair of
vector-valued functions in
$C^\infty(T^{n+1})$ then 
\[ 
\langle f,g\rangle_{(k)}=
\int\,\Gamma_m(f)\cdot \overline{g}\, dxds\tag{1.9}
\]
\medskip

\noindent
{\bf{1.10 Exercise.}}
Both sides in (1.9 ) are defined under the relaxed condition that 
$g\in H^{(k)}[m]$
while $f\in C^\infty[m]$.
Show by continuity that
(1.9) remains valid for such pairs.
\medskip

\noindent
Next, in the algebra 
of $m\times m$-matrices whose elements
are
differential operators
we consider the product $\Gamma_m\cdot P$.
\medskip

\noindent
{\bf{Exercise.}}
Show that when $P$�is as in (1.3)  where
$\{A_j\}$ are hermitian then
there exists an $m\times m$-matrix
$Q$ whose elements
are differential operators of order
$\leq 2k$ such that the following hold in the algebra above:
\[
\Gamma_m\circ P+\overline{P^*}\circ\Gamma_m=Q\tag{1.11}
\]
 \medskip

\noindent
Now (xx) and (xx) give the equation
\[
\langle  T_0(f),g\rangle_{(k)} =\int  Q(f)\cdot \bar g\, dxds-
\int  \overline{P^*}\circ\Gamma_m(f)\cdot \bar g\, dxds
\]
Apply (xx) to the pair of vector-valued functions
$\Gamma_m(f)$ and $g$ and the differential operator
$\overline{P^*}$.
Notice that
the complex conjugate of  the adjoint
$(\overline{P^*})^*$ is equal to $P$ and from this the reader can check
form the above that 
the last term in (xx) is equal to
\[
-\int  \Gamma_m(f)\cdot \overline{P(g)} \, dxds
\]
Applying (xx) this entails that the following hold for each
pair $f,g$ in $C^\infty(T^{n+1})$.
\[
\langle  T_0(f),g\rangle_{(k)}=-
\langle  f,T_0(g)\rangle_{(k)}+\int  Q(f)\cdot \bar g\, dxds
\]
\medskip

\noindent
{\bf{Exercise.}}
Since the differential operator
$Q$ has degree $\leq 2k$ the reader should verify the
existence a bounded linear operator $\mathcal B_k$ on
the Hilbert space $H^{(k)}[m]$ such that
\[
\int  Q(f)\cdot \bar g\, dxds= 
\langle  \mathcal B_k(f),g\rangle_{(k)}
\]
hold when $f$ is a  vector-valued $C^\infty$-function
and
$g\in H^{(k)}[m]$. In particular we can take a pair $f,g$ in
$C^\infty$ and notice that
\[ 
(f,g)\mapsto
\langle  T_0(f),g\rangle_{(k)}+\langle  T_0(g),f\rangle_{(k)}
\] 
is symmtric in $f$ and $g$. Form this the reader can conclude that
the bounded operator $\mathcal B_k$ is self-adjoint.

\newpage

\centerline{\emph{Proof of Proposition 1.5}}
\bigskip

\noindent
Assume first that
$g\in \mathcal D(T_0)$ which gives
\[
\langle  T_0(f),g\rangle_{(k)}=
\langle  f,T_0^*(g)\rangle_{(k)}
\]
Here $g\in H^{(k)}$
and regarding $g$ as a distribution we get
the vector-valued distribution $P(g)$. Now 
xx hold for all vector-valued periodic $C^\infty$-functions $f$.
From the above the trinsgle inequality and Cauchy-Schwarz gives
\[
\bigl|\langle  f),P(f)g\rangle_{(k)}\bigr|\leq
(||g||_k+ ||\mathcal B_k(g)||_k)\cdot ||f||_k
\]
Since this inequslity hold for all $f$
in the dense subspace $C^\infty[m]$
it follows that the distribution
$P(f)g$ belongs to $H^{(k)}[m]$
so by the construction of $T_1$ one has $g\in\mathcal D(T_1)$.
Hence one has the inclusion
\[
\mathcal D(T_1)\subset \mathcal D(T_0^*)\tag{i}
\]
Conveersley, if
$g\in\mathcal D(T_1)$
the absolute value in the right hand side of (xx) is
majorized by
\[
||T_1(g)||_k+||\mathcal B_k(g)||_k)\cdot ||f||_k
\]
The  construction of $T_0^*$ entails that
$g\in\mathcal D(T^*_0$ and hence equality holds in (i)
Finally it is clear that this equality and
(xxx) gives the operator equation
\[ 
T_0^*= -T_1+\mathcal B_k^*
\]
Since we already proved that
$\mathcal B_k$ is self-adjoint
the proof of Proposition 1.5 is finished.








\bigskip

\noindent
\centerline {\bf{� 2. A study of $T_1$}}.
\medskip

\noindent
In � 1   we  constructed the densely defined and closed operator
$T_1$ on  $H^{(k)}[m]$.
Consider some
$f\in C^\infty[m]$ and a real number $\lambda$.
Now
\[
||T_1(f)+\lambda\cdot f-\frac{1}{2}\mathcal B_k^*(f)||^2_{(k)}=
\]
\[
||T_1(f)-\frac{1}{2}\mathcal B_k^*(f)||^2_{(k)}+
\lambda^2||f||_{(k)}^2+
\lambda\cdot 
\langle T_1(f)-\frac{1}{2}\mathcal B_k^*(f),f\rangle_{(k)}
+\lambda\cdot \langle f, T_1(f)-\frac{1}{2}\mathcal B_k^*(f)\rangle_{(k)}
\]
Since $f$ is $C^\infty$ we have
$T_1(f)= T_0(f)$ and since $\mathcal B_k^*$ is self-adjoint it follows that
the sum of  the last two terms above becomes
\[
\lambda\cdot( \langle\,f,T_0^*(f)-\frac{1}{2}\mathcal B_k^*(f)\rangle_{(k)}
+\langle f, T_0(f)-\frac{1}{2}\mathcal B_k^*(f) \rangle_{(k)})\tag{i}
\]
The operator equation in Proposition 1.5 gives
\[
 T_0(f)+T_0^*(f)=\mathcal B_k^*
 \]
 which proves that (i)  is zero.
 Hence we have proved the equality
 \[
||T_1(f)+\lambda\cdot f-\frac{1}{2}\mathcal B_k^*(f)||^2_{(k)}=
||T_1(f)-\frac{1}{2}\mathcal B_k^*(f)||^2_{(k)}+
\lambda^2||f||_{(k)}^2\tag{2.1}
\]
\medskip

\noindent
Since
$||T_1(f)-\frac{1}{2}\mathcal B_k^*(f)||_{(k)}\geq 0$ the triangle inequality gives
the inequality below for every real number $\lambda$
\[
||T_1(f)+\lambda\cdot f||_{(k)}
\geq |\lambda|\cdot ||f||_{(k)}-\frac{1}{2}\cdot
||\mathcal B_k^*(f)||_{(k)}\quad\colon
f\in C^\infty_{\text{per}}[m]\tag{2.2}
\]
Above the real number $\lambda$ can be both positive or negative
and the inequality is of interest when
$|\lambda|$ exceeds the operator norm
of 
$\frac{\mathcal B_k^*}{2}$. We have for example
\[
||T_1(f)+\lambda\cdot f||_{(k)}\geq \frac{|\lambda|}{2}\cdot ||
||f||_{(k)}\quad\colon  |\lambda|\geq
||\mathcal B_k^*||\tag{2.3}
\]
\medskip

\noindent
{\bf{2.4 The operator $\widehat{T_0}$}}.
Recall that $T_1$ is closed and  extends $T_0$ in the
sense that
its graph contains that of $T_0$.
Taking the closure of $\Gamma(T_0)$ we get the densely  defined and 
closed operator
$\widehat{T_0}$ whose graph is contained in $\Gamma(T_1)$.
When $f$ are $C^\infty$-functions we have 
$T_0(f)=\widehat{T_0}(f)= T_1(f)$
so (2.3)  holds with
$T_1$ replaced by $\widehat{T_0}$.
Since
$C^\infty[m]$
is dense in $H^{(k)}[m]$
the inequality below holds for every $g\in\mathcal D(\widehat{T_0})$:
\[
||\widehat{T_0}(g)+\lambda\cdot g||_{(k)}\geq \frac{|\lambda|}{2}\cdot ||
||g||_{(k)}\quad\colon 
 |\lambda|\geq
||\mathcal B_k^*||\tag{2.4}
\]
Since $\widehat{T_0}$ is closed it follows  that
the range of  
$\widehat{T_0}(g)+\lambda\cdot E$
is closed when 
$ |\lambda|\geq||\mathcal B_k^*||$.
\medskip

\noindent
{\bf{2.5 Density of the range.}}
From now on $ |\lambda|\geq||\mathcal B_k^*||$.
Recall from the general material in � XX that the adjoint of
$\widehat{T_0}$ is equal to $T_0^*$.
If the  range of
$\widehat{T_0}(g)+\lambda\cdot E$
is not dense there exists $0\neq g\in H^{(k)}[m]$
such that
\[
\langle
\widehat{T_0}(f)+\lambda\cdot f,g\rangle_{(k)}=0
\quad\colon f\in C^\infty[m]
\]
Here $\widehat{T_0}(f)= P(f)$ for $C^\infty$-functions
and  (x) gives
\[
|\langle
P(f),g\rangle_{(k)}|\leq |\lambda|\cdot
|\langle
f,g\rangle_{(k)}|
\]
It follows that $g\in\mathcal D(T_0^*)$
so (i) in gives
\[
\langle
f,T_0^*(g)\rangle_{(k)}+
\lambda\cdot 
\langle
f,g\rangle_{(k)}=0
\]
This hold for all $f\in C^\infty[m]$
and since $\lambda$ is real we get
hence
\[
T_0^*(g)+\lambda\cdot g=0
\]
Now the operator equation in Proposition
XX gives
\[
T_1(g)=\lambda\cdot g+\mathcal B_k^*(g)
\]
At this stage we assume that $k$ is so large that the Sobolev inequslity
entails that
$H^{(k)}[m]$
consists of vector-valued functions of class $C^1$ at least
which in addition are peridic on the whole torus 
$T^{n+1}$.
Now (xx) means that
\[
P(g)=\lambda\cdot g+\mathcal B_k^*(g)
\]
From this we shall prove that $g=0$ if
the real number $\lambda$ is sufficiently large.
To attin this we consider the function
\[ 
G(s)=\int_{T^n}\, |g(x,s)|^2\, dx
\]
It follows that
\[
\frac{dG}{ds}=
2\cdot \mathfrak{Re}\, \int_{T^n}\, \partial_s(g)(x,s)
\cdot \overline{g(x,s)}\, \, dx
\]
Then conclude...


\bigskip

\noindent 
{\bf{2.6 Conclusive results.}}
If $k\geq xxx$ we have proved that there exists
a positive constant $\mu_k$
such that if $|\lambda|\geq \mu_l$
then the densely defined operator $\lambda\cdot E-\widehat{T_0}$
is surjective and at the same time
one has the inequality XX.
This means that
there exists the resolvent operator $R(\lambda;\overline{T_0})$
for such real $\lambda$.
Keeping $k$ fixed we get the closed
spectrum of $\widehat{T_0}$ which by the general result in � xx
is  a closed subset of ${\bf{C}}$.
Since $\widehat{T_0}$ is an unbounded
operator one
cannot expect that the spectrum is compact.
Moreover in contrast to the more favourable cases for  elliptic equations
the resolvent operators are in general not compact.




\newpage






























 












\centerline{\bf{� 1.Differential inequalities.}}

\bigskip

\noindent
Let $M(s)$ be a non-negative real-valued continuous function
on a closed interval $[0,s^*]$.
To each $0\leq s<s^*$
we set
\[
d_M^+(s)=\limsup_{\Delta s\to 0}\, \frac{M(s+\Delta s)-M(s)}{\Delta s}
\]
where $\Delta s$ are positive during the limit.
\medskip

\noindent
{\bf{1.1 Proposition.}} \emph{If there exists a real number
$B$ such that
$d_M^+(s)\leq B\cdot M(s)$ holds in $[0,s^*)$ then }
\[ 
M(s)\leq M(0)\cdot e^{Bs}\quad\colon 0<s\leq s^*
\]
\medskip

\noindent
The proof of this result is left as an exercise to the reader.
The hint is to consider the function $N(s)= M(s)e^{-Bs}$
and show that $d^+_N(s)\leq 0$ for all $s$.
Notice that $B$ is an arbitrary real number, i.e. it may also be $<0$.

\medskip

\noindent
More generally, let $k(s)$ be some  non-decreasing continuous function
with $k(0)=0$. 
suppose that
\[
d^+_M(s)\leq B\cdot M(s)+k(s)\quad \colon 0\leq s<s^*
\]
Now the reader may verify that
\[
M(s)\leq M(0)\cdot e^{Bs}+\int_0^s\, k(t)\, dt\tag{1.1.1}
\]


\medskip


\noindent
Next, consider
a product set
$\square=[0,\pi]\times [0,s^*]$
where $0\leq x\leq\pi$ and consider functions
$g(x,s)$ which are periodic in $x$, i.e.
\[ 
g(0,s)= g(\pi,s)\quad\colon 0\leq s\leq s^*
\]
A $C^1$-function
$g$ is  periodic $C^1$-function when 
$g$  and the partial derivatives
$\partial_s(g)$ and $\partial_x(g)$  are  periodic in
$x$.
\medskip

\noindent
{\bf{1.2 Theorem.}}
\emph{Let $g$ be a $C^1$-function which satisfies the PDE-equation}
\[
\partial_s(g)(x,s)= a\cdot \partial_x(g)+ b\cdot g\tag{*}
\]
\emph{in $\square$ where $a$ and $b$ are 
$x$-periodic real-valued continuous functions. Set}
\[
M_g(s)= \max_x\, |g(x,s)|\quad\colon \,B=\max_{x,s}\, |b(x,s)|
\]
\emph{Then one has the inequality}
\[
M_g(s)\leq M_g(0)\cdot e^{Bs}
\]


\noindent
\emph{Proof.}
Consider some $0<s<s^*$ and let $\epsilon>0$.
Put
\[ 
m^*(s)=\{ x\,\colon\, g(x,s)= M_g(s)\}
\]
The  continuity of $g$
entials that the function $M(s)$ is continuous and
the sets $m^*(s)$ are compact.
If $x^*\in m^*(s)$ the periodicity of
the
$C^1$-function $x\mapsto g(x,s)$
entails that
$\partial_x(x^*,s)=0$ and (*) gives
\[
\partial_s(g)(x,s)=b(x,s)g(x,s)\quad\colon x\in m^*(s)
\]
Next, let $\epsilon>0$. We find an open neighborhood $U$
of $m^*(s)$
such that
\[
|\partial_x(g)(x,s)|\leq \epsilon\quad\colon x\in U
\]
Now there exists
$\delta>0$ such that
\[
|g(x,s)|\leq M(s)-2\delta\quad\colon x\in [0,\pi]\setminus U
\]
Continuity gives  some
$\rho>0$ such that
if $0<\Delta s<\rho$ then  the inequalities below hold:
\[
|g(x,s+\Delta s)|\leq M(s)-\delta\quad\colon x\in [0,\pi]\setminus U
\quad\colon\,M(s+\Delta s)>M(s)-\delta\tag{i}
\]
\[
M(s+\Delta s)\leq M(s)+\epsilon\quad \colon\,
|\partial_x(g)(x,s+\Delta s)|\leq 2\epsilon \quad\colon x\in m^*(s)\tag{ii}
\]

\noindent
If $0<\Delta s<\rho$ we see that (i) gives
$x\in m^*(s+\Delta s)\subset U$
and for such $x$-values 
Rolle's mean-value theorem and the PDe-equation give
\[
M_g(x,s+\Delta s)- g(x,s)=\Delta s\cdot \partial_s(g(x,s+\theta\cdot \Delta s)=
\] 
\[
\Delta s\cdot \bigl[a(x,s+\Delta s)\cdot
\partial_x(g)(x+\theta\cdot \Delta s)+
b(x,s+\Delta s)\cdot
g(x,s+\theta\cdot \Delta s)\bigr]\tag{iii}
\]
Let  $A$ be the maximum norm of $|a(x,s)|$ taken over
$\square$.
Since $|g(x,s)|\leq M(s)$
the triangle inequality and (iii) give
\[ 
M(s+\Delta s)\leq M(s)+\Delta s[\cdot A\cdot 2\epsilon+
B\cdot M(s+\theta\cdot \Delta s)]
\]
Since the function $s\mapsto M(s)$ is continuous
it follows that
\[
\limsup_{\Delta s\to 0}\,
\frac{M(s+\Delta s)-M(s)}{\Delta s}\leq
A\cdot 2\epsilon+ BM(s)
\]
Above $\epsilon$ can be arbitrary small
and hence
\[ d^+(s)\leq B\cdot M(s)
\]
Then Proposition 1.1 gives (*) in the theorem.
\medskip


\noindent
{\bf{1.3 Higher order derivatives.}}
Supose that $g$ is a $C^2$-function satisfying the PDE-equation
(*) where  $a$ and $b$ have a continuous partial $x$-derivative.
Set  $h=\partial_x(g)$. Since the differential operators
$\partial_x$ and $\partial_s$ commute
we obtain
\[
\partial_s(h)=\partial_x(a\cdot h)+
\partial_x(b\cdot g)=
a\cdot\partial_xh+(\partial_x(a)+b)h+\partial_x(b)g\tag{1.3.1}
\]


\medskip

\noindent
{\bf{$L^2$-inequalities.}}
Let $g(x,s)$ be a $C^1$-function satisfying (*)
in
Theorem 1.2.
Set
\[ 
J_g(s)=\int_0^\pi\, g^2(x,s)\, dx
\]
Diffeernation with respect to $s$ and (*) give
\[
\frac{dJ_g}{ds}=2\cdot \int_0^\pi\, (a\partial_s(g)\cdot \partial g+ b\cdot g)\,dx
\]
By periodicity $\int_0^\pi\, \partial_x(ag^2)\, dx=0$
which entials that the right hand side becomes
\[
\int_0^\pi\, (-\partial_x(a)+b)\cdot g^2\, dx
\]
So if $K$ is the maximum norm of
$-\partial_x(a)+b$ over $\square$ it follows that
\[
\frac{dJ_g}{ds}(s)\leq K\cdot J(s)
\]
Hence Theorem xx gives
\[
\int_0^\pi\, g^2(x,s)\, dx\leq e^{Ks}\cdot
\int_0^\pi\, g^2(x,0)\, dx\quad\colon 0<s\leq s^*
\]
Integration with respect to $s$ entails that
\[ 
\iint_\square\, g^2(x,s)\, dxds\leq
\int_0^{s^*}\, e^{Ks}\,ds\cdot
\int_0^\pi\, g^2(x,0)\, dx
\]
This, the $L^2$-integral of $x\to g(x,0)$
majorizes both the area integral and each slice integral when
$0<s\leq s^*$.
\medskip

\noindent
{\bf{Higher order derivatives.}}
Same procedure gives majorisations of these
integrals when
higher order $x$-derivatives are inserted.
Simialrly we regard PDE-equations when $g$
is replaced by $h=\partial_s(g)$ and so on.


\newpage

\centerline{\bf{� 2. A boundary value equation}}
\bigskip

\noindent
Let $a(x,s)$ and $b(x,s)$ be real-valued $C^\infty$-functions on
$\square$ which are periodic in $x$. Consider the PDE-operator
\[ 
P=\partial_s-a\cdot \partial_x-b
\]
Given a periodic $C^1$-function $f(x)$ on $[0,\pi]$
we seek a function $g(x,s)$ in $\square$ which satisfies $P(g)=0$ and the
initial condition
\[ 
g(x,0)= f(x)
\]
Above we  require that
$g$ is at least of class $C^1$.
From � xx we see that $g$ is unique if it exists.
There remains to prove existence.
\medskip

\noindent
{\bf{2.1 Theorem.}}
\emph{For every positive integer $p$ and each
periodic $f\in C^p[0,\pi]$
there exists a unique periodic $g\in C^p(\square)$ 
where $P(g)=0$ and $g(x,0)= f(x)$.}

\medskip

\noindent
The proof requires several steps and 
employs Hilbert space methods.
To each non-negtive integer $k$
we get the complex Hilbert space
$\mathcal H^{(k)}$
from � xx, i.e. we complete  the space
of complex-valued
$C^k$-functions on 
$\square$ which are periodic with respect to $x$.
By the Soboloev inequality from � xx
we know that if $k\geq 2$
every function in
$\mathcal H^{(k)}$ is continuous and more generally
one has the inclusion
\[
\mathcal H^{(k)}\subset C^{k-2}(\square)\quad\colon k\geq 3
\]
Moreover, the  first order PDE-operator $P$ maps
$\mathcal H^{(k+1)}$ into
$\mathcal H^{(k)}$.
From now on we only consider $k$-integers which are $\geq 2$.
On the periodic $x$-interval $[0,\pi]$ we get the Hilbert spaces
$\mathcal H^k[0,\pi]$.
The result in � xx shows that
if
$f(x)\in\mathcal H^k[0,\pi]$
is such that there exists some
$F(x,s)\in\mathcal H^{(k)}$
such that $P(F)=0$ and $F(x,0)= f(x)$ then
$F$ is unique and
there exists a constant $C$ which only depends upon the
$C^\infty$-functions $a$ and $b$ such that
\[
||F||_k\leq C\cdot ||f||_k
\]
where we have taken norms in
$\mathcal H^{(k)}$
$\mathcal H^k[0,\pi]$.
Moreover (*) shows that $C$ can be chosen such that
the
function $f^*(x)=F(x,s^*)$ satisfies
\[
||f^*||_k\leq C\cdot ||f||_k
\]
Let $\mathcal D_k(P)$ be the set of all
$f(x)\in\mathcal H^k[0,\pi]$ for which $F(x,s)$ above exists.
\medskip

\noindent
{\bf{2.2 Density Lemma.}}
\emph{If
$\mathcal D_k(P)$  is dense in 
$\mathcal H^k[0,\pi]$ then it is equal to this Hilbert space.}
\medskip

\noindent
\emph{Proof.}
Let $f$ be in
$\mathcal H^k[0,\pi]$ and by density we find a sequence
$\{f_n\}$ in 
$\mathcal D_k(P)$ where $||f_n-g||_k\to 0$.
By xx above we have
\[ 
||F_n-F_m||_k\leq C||f_n-f_m||_k
\]
hence $\{F_n\}$ is a Cauchy sequence in
the Hilbert space $\mathcal H^{(k)}$
and converges to a limit $F$.
Since each $P(F_n)=0$ it follows that
$P(F)=0$
and it is clear that the continuous boundary value function
$F(x,0)=f(x)$ which entails that
$f$ belongs to 
$\mathcal D_k(P)$.
\medskip

\noindent
{\bf{2.3 The operators $S_k$.}}
To each $f\in \mathcal D_k(P)$ we get $F(x,\pi)$
in $\mathcal H^k[0,\pi]$ and  set
\[ 
S_k(f)=F(x,\pi)
\]
So here the domain of definition of $S_k$ is equal to
$\mathcal D_k(P)$ and by (xx) above there exists a conatant
$M_k$ such that
\[
||S_k(f)||\leq M_k\cdot ||f||_k\quad\colon f\in  \mathcal D_k(P)
\]
\medskip

\noindent
{\bf{2.4 Proposition.}}
\emph{For each $k$ there exists some
$\alpha(k)<0$ such that
the range of the operator $E-\alpha\cdot S_k$ contains
all periodic $C^\infty$-functions on
$[0,\pi]$.}
\medskip

\noindent
We prove Propostion 2.4 in � xx below. Let us
now show that it gives the density of
$\mathcal D_k(P)$.
Namely, if $\mathcal D_k(P)$ fails to be dense there exists
a non-zero $f_0\in\mathcal D_k(P)$ which is
$\perp$ to $\mathcal D_k(P)$.
In Proposition 2.4 we choose $0<\alpha\leq \alpha(k)$ so small that
we also have
\[
\alpha<M_k/2\tag{i}
\]
Since periodic $C^\infty$-functions are dense in
$\mathcal H^k[0,\pi]$
it follows from Proposition 2.4 that there exists a sequence
$\{h_n\}$ in $\mathcal D_k(P)$
such that
\[ 
\lim_{n\to \infty}\, ||h_n-\alpha\cdot S_k(h_n)-f_0||_k\to 0\tag{ii}
\]
It follows that
\[
\langle f_0,f_0\rangle=1=\lim\, 
\langle f_0,h_n-\alpha\cdot S_k(h_n)\rangle=
-\alpha\cdot \lim\, \langle f_0,S_k(h_n)\rangle\tag{iii}
\]
Next, the triangle inequality and (ii) give
\[
||h_n||_k\leq 1+\alpha\cdot ||(S_k(h_n)||
\leq 1+1/2\cdot||h_n||\implies
||h_n||_k\leq 2\tag{iv}
\]
Funally, by the Cauchy-Schwarz inequality the absolute va�lue in
the right hand side of (iii) is majorized by
\[
\alpha\cdot M_K\cdot 2<1
\] 
which contradicts (iii).
Hence the orthogonal complement of $\mathcal D_k(P)$ is zero
which proves the requiested density and by the above
we get the following conclusive result:
\medskip

\noindent
{\bf{2.5 Theorem.}}
\emph{If $k\geq 2$
and $f(x)\in \mathcal H^k[0,\pi]$
there exists a unique $F(x,s)\in \mathcal H^{(k)}$
such that $F(x,0)= f(x)$.}





\newpage

\centerline{\bf{� 3. A class of inhomogeneous PDE-equations.}}


\medskip

\noindent
Before Theorem 3.1 is announced we introduce some notations.
Put
\[ 
\square =\{ 0\leq x\leq \pi\}\times
\{0\leq s\leq 2\pi\}
\]
We shall consider doubly periodic functions
$g(x,s)$ on $\square$, i.e. 
\[
g(\pi,s)= g(0,s)\quad\colon g(x,0)= g(x,2\pi)
\]
If $k\geq 0$ we denote by $C^k(\square)$ the space of $k$-times
doubly-periodic continuously differentiable functions. 
If 
$g\in C^k(\square)$
we set
\[ 
||g||^2_{(k)}= \sum_{j,\nu}\, \int_\square
\bigl|\frac{\partial^{j+\nu}g}{\partial x^j\partial s^\nu}(x,s)\bigr|^2\, dxds
\]
with the double sum extended  pairs  $j+\nu\leq k$.
This gives the complex Hilbert space $\mathcal H^{(k)}$
after a completion of $C^k(\square)$ with respect
to the norm  above.
Recall from � xx that every function $g\in\mathcal H^{(2)}$
is automatically continuous and doubly periodic  on
the closed square.
More generally, if $k\geq 3$ 
each
$g\in\mathcal H^{(k)}$ has continuous and doubly periodic derivatives
up to order $k-2$.
Next, consider a first order PDE-operator
\[
P=\partial_s-a(x,s)\partial_x-b(x,s)
\]
where $a$ and�$b$ are real-valued doubly 
periodic $C^\infty$-functions.
It is clear that
$P$ maps
$\mathcal H^{(k)}$ into $\mathcal H^{(k+1)}$
for every $k\geq 2$.
Keeping $k\geq 2$ fixed we set
\[
\mathcal D_k(P)=\{g\in\mathcal H^{(k)}\,\colon\,
P(g)\in\mathcal H^{(k)}\}
\]
Since $C^\infty(\square)$
is dense in
$\mathcal H^{(k)}$ this yields for each $k\geq 2$ a densely defined operator
\[
P\colon \mathcal D_k(P)\to \mathcal H^{(k)}\tag{i}
\]
In
$\mathcal H^{(k)}\times \mathcal H^{(k)}$ we get the graph
\[
\Gamma_k=\{(g,P(g)\colon\, g\in\mathcal D_k(P)\}
\]
Since $P$ is a differential operator the general result in
� xx entails that
$\Gamma_k$ is a closed subspace so the densely defined operator in
(i)  has a closed graph. Thus. for each
$k\geq 2$ we have a densely defined linear operator
and closed operator on $\mathcal H^{(k)}$
denoted by $\mathcal T_k$.
So its domain of definition  $\mathcal D(T_k)= \mathcal D_k$.
Next, we consider the graph
\[
\gamma_k=\{(g,P(g)\colon\, g\in C^\infty(\square)\}
\]
This is a subspace of
$\Gamma_k$ and  the closure $\bar\gamma_k$
yields the graph of another densely defined linear operator
denoted by
$T_k$.
We remark that the inclusion
\[
\mathcal D(T_k)\subset\mathcal D(\mathcal T_k)
\] 
in general is strict.
Let $E$ be the identity operator. With these notations one has

\medskip

\noindent 
{\bf{3.1  Theorem.}}
\emph{For each integer $k\geq 2$ there exists
a positive real number $\rho(k)$ such that
$T_k-\lambda\cdot E$ is surjective on
$\mathcal H^{(k)}$ for every $\lambda>\rho(k)$
and its kernel is zero.}
\medskip


\noindent
{\bf{3.2 Remark.}}
The result is  remarkable  since $T_k$ is only densely defined
while Theorem 3.1 asserts that
\[ 
T_k-\lambda\cdot E\colon \mathcal D_k\to \mathcal H^{(k)}
\] 
is bijective.
Hence 
the closed and densely defined operator $T_k$
has a non-empty resolvent set
so by the general results in �� x there exists
resolvent operators $R_k(\lambda)$ defined outside the
closed specrum $\sigma(T_k)$
where the composed operators 
\[
(\lambda\cdot E-T_k)\circ R_k(\lambda)=E
\]
for all $\lambda$ outside $\sigma(T_k)$.
The determination of these spectra
is unclear and most likely one needs extensive numerical studies
to grasp these closed sets which in addition depend upon $k$.
\medskip

\noindent
Next,R
recall from � xx
that the closed and densely defined operator
$T_k$ has an adjoint $T_k^*$.
A crucial step in the proof of Theorem 3.1 is the following:
\medskip

\noindent
{\bf{3.3  Proposition.}}
\emph{One has the equality $\mathcal D(T_k^*)= \mathcal D_k$
and there exists a bounded self-adjoint operator
$B_k$ on
 $\mathcal H^{(k)}$ such that}
 \[
 T_k^*=-\mathcal T_k+B_k
 \]
 

\medskip




\centerline{\emph{Proof of Proposition 3.3}}
\bigskip


\noindent
Keeping $k\geq 2$ fixed we set $\mathcal H=\mathcal H^{(k)}$.
For each pair $g,f$ in $\mathcal H$
their inner product is defined by
\[
\langle f,g\rangle=
 \sum\, \int_\square
\frac{\partial^{j+\nu}f}{\partial x^j\partial s^\nu}(x,s)
\cdot
\overline{\frac{\partial^{j+\nu}g} {\partial x^j\partial s^\nu}}(x,s)
\, dxds
\]
where the sum is taken when
$j+\nu\leq k$.
Introduce the differential operator
\[
\Gamma= \sum_{j+\nu\leq  k}\, (-1)^{j+\nu}\cdot \partial_x^{2j}\cdot \partial_s^{2\nu}
\]
Partial integration gives
\[
\langle f,g\rangle=\int_\square\, f\cdot \Gamma(\bar g)\, dxds=
\int_\square\, \Gamma(f)\cdot\bar g\, dxds
\quad\colon\, f,g\in C^\infty
\tag{i}
\]
Now we consider the operator
$P=\partial_s-a\cdot\partial_x-b$ and 
get 
\[
\langle P(f),g\rangle=\int_\square\, P(f)\cdot \Gamma(\bar g)\,dxds\tag{ii}
\]
Partial integration identifies (ii) with
\[
-\int_\square\, f\cdot \bigl(\partial_s-\partial_x(a)-a\cdot\partial_x-b)\circ\Gamma(\bar g)\,dxds\tag{iii}
\]


\noindent
{\bf{1.1 Exercise.}}
In (iii)  appears the composed differential operator
\[
\partial_s-\partial_x(a)-a\cdot\partial_x-b)\circ\Gamma
\]
Show that in the ring of differential operators with
$C^\infty$-coefficients this differential operator can be written
in the form
\[
\Gamma\circ(\partial_s-a\cdot\partial _x-b)+Q(x,s,\partial_x,\partial_s)
\]
where $Q$ is a differential of order $\leq 2k$ 
with coefficients in $C^\infty(\square)$.
Conclude from the above that
\[
\langle Pf,g\rangle=-
\langle f,Pg\rangle+\int_\square\ f\cdot Q(\bar g)\,dxds\tag{1.1.1}
\]
\medskip

\noindent
{\bf{1.2 Exercise.}}
With $Q$ as above we have a bilinear form which sends a pair
$f,g$ in $C^\infty(\square)$ to 
\[
\int_\square\ f\cdot Q(\bar g)\,dxds\tag{1.2.1}
\]
Use partial integration and
the
Cauchy-Schwarz inequelity to show that
there exists a conatant $C$ which depends on $Q$ only such that
the absolute value of (1.2.1) is majorized by
$C_Q\cdot ||f||_k\cdot ||g||_k$.
Conclude that
there exists a bounded linear operator 
$B_k$ on $\mathcal H$ such that
\[
\langle f,B_k(g)\rangle=\int_\square\ f\cdot Q(\bar g)\,dxds\tag{1.2.2}
\]

\medskip

\noindent
{\bf{1.3 Proof that $B_k$ is self-adjoint}}
From the above we have
\[
\langle Pf,g\rangle=
-\langle f,Pg\rangle+
\langle f,B_k(g)\rangle\tag{1.3.1}
\]

\noindent
Keeping $f$ in $C^\infty(\square)$ we notice that
$\langle f,B_k(g)\rangle$ is defined for every
$g\in\mathcal H$.
From this the reader can check that (1.3.1) remains valid when
$g$ belongs to $\mathcal D(\mathcal T_k)$
which means that
\[
\langle Pf,g\rangle=
-\langle f,\mathcal T_kg\rangle+
\langle f,B_k(g)\rangle\quad\colon f\in C^\infty(\square)\tag{1.3.2}
\]

\medskip

\noindent
Moreover, when both $f$ and $g$ belong to $C^\infty(\square)$
we can
reverse their positions in (*) which gives
\[
\langle Pg,f\rangle=
-\langle g,Pf\rangle+
\langle g,B_k(f)\rangle\tag{1.3.3}
\]
Since $a$ and $b$ are real-valued it is clear that
\[
\langle Pg,f\rangle=-\langle f,Pg\rangle\tag{1.3.4}
\]
It follows that
\[
\langle f,B_k(g)=\langle g,B_k(f)\quad\colon f,g\in C^\infty(\square)\tag{1.3.5}
\]
Since this hold for all pairs of $C^\infty$-functions and
$B_k$ is a bounded linear operator on
$\mathcal H$ the density of $C^\infty(\square)$ entails that
$B_k$ is a bounded self-adjoint operator
on $\mathcal H$.



\medskip

\noindent
{\bf{1.4  The equality 
$\mathcal D(T_k^*)=
\mathcal D_k$.}}
The density of $C^\infty(\square)$ in $\mathcal H$
entails that  a function $g\in \mathcal H$ belongs to $\mathcal D(T_k^*)$  
if and only if there exists a constant $C$ such that
\[
|\langle Pf,g\rangle|\leq C\cdot ||f||\quad\colon f\in C^\infty(\square)\tag{1.4.1}
\]
Since $B_k$ is a bounded operator,
(1.3.2) gives the inclusion
\[
\mathcal D_k\subset
\mathcal D(T_k^*)\tag{1.3.3}
\]
To prove the opposite inclusion we use that
the $\Gamma$-operator is elliptic.
If $g\in \mathcal D(T_k^*)$
we have from (i) in � 1.1:
\[
\langle Pf,g\rangle=\langle f,T_k^*g\rangle=
\int\, \Gamma(f)\cdot \overline{T_k^*(g)}\, dxds
\quad\colon f\in C^\infty(\square)
\]
Similarly
\[
\langle f,B_k(g)\rangle=
\int\, \Gamma(f)\cdot \overline{B_k(g)}\, dxds
\]
Treating $\mathcal T_k(g)$ as a distribution the equation
(1.3.2)  entails that
the elliptic operator $\Gamma$ annihilates
$T^*_k(g)-\mathcal T_k(g)+B_k(g)$.
Since both
$T^*_k(g)$ and $B_k(g)$ belong to $\mathcal H$
this implies by the general result in � xx that
$\mathcal T_k(g)$  belongs to $\mathcal H$
which proves the 
requested equality (1.4) and at the same time the operator equation
\[
T_k^*=-\mathcal T_k(g)+B_k\tag{1.4.2}
\] 

\medskip

\noindent
{\bf{1.5 An inequality.}}
Let $f\in C^\infty(\square)$
and $\lambda$ is a positive real number.
Then
\[
||\mathcal T_k(f)-\frac{1}{2}B_k(f)-\lambda\cdot f||^2=
\]
\[
||\mathcal T_k(f)-\frac{1}{2}B_k(f)||^2+\lambda^2\cdot ||f||^2-
\lambda\bigl(\langle \mathcal T_k(f)-\frac{1}{2}B_k(f),f\rangle+
\langle f, \mathcal T_k(f)-\frac{1}{2}B_k(f)\rangle\bigr)
\]
The last term is $\lambda$ times
\[
\langle \mathcal T_k(f),f\rangle+
\langle f,\mathcal T_k(f)\rangle-\langle f,B_kf\rangle\tag{i}
\]
where we used that $B_k$ is symmetric.
Now $T_k=\mathcal T_k$ holds
on $C^\infty(\square)$ and the definition of adjoint
operators give
\[
\langle \mathcal T_k(f),f\rangle=
\langle f,T_k^*\rangle\tag{ii}
\]
Then (1.4.2 ) implies that (i) is zero and hence we have
proved 
\[
||T_k(f)-\frac{1}{2}B_k(f)-\lambda\cdot f||^2=
\lambda^2\cdot ||f||^2+||T_k(f)-\frac{1}{2}B_k(f)||^2\geq \lambda^2\cdot ||f||^2\tag{iii}
\]
From (iii)  and the triangle inequality for norms we obtain
\[
||T_k(f)-\lambda\cdot f||\geq \lambda\cdot ||f||-
\frac{1}{2}||B_k(f)||\tag{iv}
\]
Now $B_k$ has a finite operator norm and if
$\lambda\geq ||B_k||$ we see that 
\[
||T_k(f)-\lambda\cdot f||\geq
\frac{\lambda}{2}\cdot ||f||\tag{v}
\]


\noindent
Finally, since $C^\infty(\square)$ is dense in $\mathcal D(T_k)$
it is clear that (v) gives
\[
||T_k(f)-\lambda\cdot f||\geq
\frac{\lambda}{2}\cdot ||f||\quad\colon f\in\mathcal D(T_k)\tag{1.5.1}
\]
\medskip

\centerline{\emph{� 2. Proof of Theorem 3.1}}
\bigskip

\noindent
Suppose we have found some
$\lambda^*\geq \frac{1}{2}\cdot ||B||$
such that
$T_k-\lambda$  has a dense range in $\mathcal H$
for every $\lambda\geq \lambda^*$.
If this is so we fix $\lambda\geq \lambda^*$
 and take some
$g\in \mathcal H$. The hypothesis gives a
sequence $\{f_n\in \mathcal D(T_k)$ such that
\[
\lim_{n\to\infty}\,||T(f_n)-\lambda\cdot f_n-g||=0
\]
In particular $\{||T_k(f_n)-\lambda\cdot f_n\}$ is a Cauchy sequence 
in $\mathcal H$
and (1.5.x )  implies that $\{f_n\}$ is a Cacuhy sequence 
in the Hilbert space $\mathcal H$
and hence converges to
a limit $f_*$. Since the operator $T_k$ is closed we conclude that
$f_*\in\mathcal D(T)$ and we get  the equality
\[
T(_kf_*)-\lambda\cdot f_*=g
\]
Finally, since the graph of $T$ is contained in $T_1$
we have the requested equation 
\[
P(f_*)-\lambda\dot f_*=g
\]
Thus finishes the proof of Theorem 1.6  provided we
have established the existence of $\lambda_*$ above
\medskip

\noindent{\bf{2.1 Density of the range.}}
By the construction of adjoint operators the range of $T_k-\lambda\cdot E$
fails to be dense
if and ony if
$T_k^*-\lambda$ has a non-zero kernel.
So assume that
\[ 
T_k^*(f)-\lambda\cdot f=0\tag{i}
\]
for some $f\in\mathcal D(T_k^*)$ which is not identically zero.
Notice that $T_k$ sends real-va�lued functions into real-valued
functions. So above we can assume that
$f$ is real-valued and 
also  assume that
$f$ is normalised so that
\[
\int_\square f^2(x,s)\, dxds=1
\]
By  (**) the equation (xx) gives
\[
\mathcal T_k(f)+\lambda\cdot f-B(f)=0\tag{ii}
\]
Let us then consider
the function
\[ 
V(s)= \int_0^\pi\, f^2(x,s)\, dx
\]
Recall from � xx that the $\mathcal H$-function $f$ is of class
$C^1$.
Now
\[ 
\frac{1}{2}\cdot V'(s)=  \int_0^\pi\, f\cdot \frac{\partial f}{\partial s}\, dx\tag{iii}
\]
By (ii) we have
\[
 \frac{\partial f}{\partial s}-
a(x)\frac{\partial f}{\partial x}-b\cdot f=B(f)-\lambda\cdot f
\]
Hence the right hand side in (iii) becomes
\[
-\lambda\cdot V(s)+ \int_0^\pi\, f(x,s)\cdot B(f)(x,s)\,dx+
+ \int_0^\pi\,a(x,s)\cdot f(x,s)\cdot \frac{\partial f}{\partial x}(x,s)\,dx
\]
By partial integration the last term is equal to
\[
-\frac{1}{2}\int_0^\pi\, \partial_x(a)(x,s)\cdot f^2(x,s)\,dx
\]
Set
\[
M=\frac{1}{2}\cdot \max_{(x,s )\in\square}\,|\partial_x(a)(x,s)|
\]
Then we get the inequality
\[
\frac{1}{2}\cdot V'(s)\leq
(M-\lambda)\cdot V(s)+\int_0^\pi\, f(x,s)\cdot B(f)(x,s)\,dx
\]
Set
\[
 \Phi(s)=\int_0^\pi\, |f(x,s)|\cdot |B(f)(x,s)|\,dx
\]
Since the $L^2$-norm of $f$ is one
the Cauchy-Schwarz inequality 
gives
\[
\int_{-\pi}^\pi \Phi(s)\,ds \leq
\sqrt{\int_\square\,  |B(f)(x,s)|^2\, dx ds}\leq ||B(f)||
\]
where the last equality follows since
the  squared integral of $B(f)$ is majorized by its squared
norm in $\mathcal H$.
When $\lambda>M$ it follows from  (xx) that
\[
(\lambda-M)\cdot V(s)+
\frac{1}{2}\cdot V'(s)\leq\Phi(s)
\]
Next, since $f$ is double periodic we have  $V(-\pi)= V(\pi)$
so after an integration (xx) gives
\[
(\lambda-M)\cdot \int_\pi^\pi V(s)\, ds=\int_{-\pi}^\pi \Phi(s)\,ds \leq ||B(f)||
\]
By (xx) we have
$\int_\pi^\pi V(s)\, ds=1$ which gives  a contradiction if
$\lambda>M+||B(f)||$.
\medskip


\noindent
{\bf{Remark.}}
Set
\[ 
\tau=\min_f\,||B(f)||
\] 
with the minimum taken over
funtions $f\in\mathcal D(T_0^*)$ whose $L^2-$integral is normalised by
(xx). The proof has shown that
the kernel of $T_0^*-\lambda$ is zero for all $\lambda>M+\tau$.

\newpage


\centerline{\bf{A special solution.}}
\bigskip


\noindent
Let $f(x)$ be a periodic $C^\infty$-function 
on $[0,\pi]$.
Put
\[ 
Q= a(x,s)\cdot \frac{\partial}{\partial x}+ b(x,s)
\]
Let  $\eta(s)$ be a $C^\infty$-function of $s$ and 
$m$ a postive integer
If $\lambda>0 $ is a real number.
we set
\[ 
g_\lambda(x,s)=\eta(s)\cdot f+\eta(s)\cdot\sum_{j=1}^{j=m}\,
\frac{(s-\pi)^j}{j!}\cdot (Q-\lambda)^j(f)\quad\colon 0\leq s\leq \pi\tag{i}
\]
We choose $\eta$ to be a real-valued $C^\infty$-function
such that
$\eta(s)=0$ when $s\leq 1/4$ and -1 if $s\geq 1/2$.
Hence $g_\lambda(x,s)=0$ in (i) when
$0\leq  s\leq 1/4$
and we extend the function to $[-\pi\leq s\leq \pi$ where
$g_\lambda(x,-s)= g_\lambda(x,s)$ if $0\leq s\leq \pi$.
So now $g_\lambda$ is $\pi$-periodic with respect to $s$
and  vanishes when
$|s|\leq 1/4$.
\medskip

\noindent
{\bf{Exercise.}}
If $1/2\leq s\leq \pi$ we have $\eta(s)=1$. Use (i) to show that 
\[
(P+\lambda)(g_\lambda)=
\frac{\partial g_\lambda}{\partial s}
-(Q-\lambda)(g_\lambda)=\frac{(s-\pi)^m}{m!}\cdot (Q-\lambda)^{m+1}(f)
\]
hold when
$1/2\leq s\leq \pi$.
At the same time
$g_\lambda(s)=0$ when $0\leq s\leq 1/4$.
So $(P+\lambda)(g)$
is a function whose derivatives
with respect to $s$ vasnish up to order
$m$ at $s=0$ and $s=\pi$ and is therefore
doubly periodic of class $C^m$ in
$\square$.
Now Theorem 2.2  applies. For a given $k\geq 2$
we choose a sufficently large $m$
and find $h(x,s)$ so that  
\[ 
P(h)+\lambda\cdot h=(P+\lambda)( g_\lambda)(x,s)
\]
where $h$ is $s$-periodic, i.e.
\[ 
h(x,0)= h(x,\pi)
\]
Notice also that
$g_\lambda(x,0)=0$ while $g_\lambda(x,\pi)= f(x)$.
Set
\[ 
g_*(x)=h-g_\lambda
\]
Then  $P(g_*)+\lambda\cdot g_*=0$
and 
\[ 
g_*(x,0)-g_*(x,\pi)=f(x)
\]
\medskip

\noindent
Above we started with the $C^\infty$-function.
Given $k\geq 2$
we can take
$m$ sufficiently large during the constructions above
so that $g_*$ belongs to
$\mathcal H^{(k)}(\square)$.







\end{document}
































 

 























