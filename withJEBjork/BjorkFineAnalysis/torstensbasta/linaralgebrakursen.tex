




\documentclass{amsart}
\usepackage[applemac]{inputenc}
\addtolength{\hoffset}{-12mm}
\addtolength{\textwidth}{22mm}
\addtolength{\voffset}{-10mm}
\addtolength{\textheight}{20mm}

\def\uuu{_}

\def\vvv{-}

\begin{document}


\noindent
The proof requires some preliminary results.
First we  need  inequality due to  Hadamard which
goes as follows:
\medskip

\noindent
{\bf {6.2 Hadamard's inequality.}}
\emph{For every matrix $A$ with a non-zero determinant one has the inequality}
\[
\bigl|\text{det}(A)\bigr|\cdot \text{Norm}(A^{-1})\leq
\frac{||A||^{n-1}}{(n-1)^{n-1)/2}}
\]


\noindent
{\bf{Exercise.}} Prove this  result. 
The hint is to  use
expansions of certain determinants while one
considers 
$\text {det}(A)\cdot \langle A^{-1}(x),y\rangle$ 
for all pairs of unit vectors $x$ and $y$.

\newpage

We are given the simple algebra $A$. For ech non-ero elemebt $x\in A$ we put
\[
\rho(x)=\dim_{{\bf{C}}}\, Ax
\]
where $Ax$ is the left principal ideal generated by $x$.
Choose some $x_*$ such that
\[
\rho(x_*)= \min_{x\in A}\, \rho(x)
\]
if $a\in A$ gives $ax_*\neq 0$ the obvious inclusion
$Aax_*\subset Ax_*$ gives
\[
\rho(ax_*)\leq\ rho(x_*)
\]
and the minimal, choice in 8i) entials that
eqauality holds which implies that
the complex vector apaces $Ax$ and $Aax$ are equal.
This can be expressed by saying that the left ideal $Ax$ is minimal, i.e. it does not contain a
proper left ideal than the zero-ideal.
Next, since the algebra $A$ is since, the 
\emph{two-sided ideal} generated by $x_*$ is equal
to $A$.
This means that the identty $1$ can be wirtten as
\[
1=a_1xb_1+\ldots+a_mxb_m
\]
where $m$ is some positive integer and
$\{a_\nu\}$ and $\{b_\nu\}$ two $m$-tuples of $A$-elements.
Here $m$ can be chosen minimal
Let us then consider one of the left prinicipal ideals
\[
L_\nu= Aa_\nu x_*b_\nu
\]
Firsot, since $Ax_*$ is minimal we see that
\[
L_\nu= Ax_*b_\nu
\]
and (xx) above gives $\rho(x_*b_\nu)\leq \rho(x_*$ so
the minmal choice of $\rho(x_)*$ entials by 8xx) above that
$L_\nu$ is a minimal left ideal.
Put
\[
e_\nu=a_\nu x_*b_\nu\quad\colon 1\leq \nu\leq m
\]
Suppose one has an equaution
\[
y_1e_1+\dots+y_me_m=0
\] 
where $\{y_\nu\}$ is some $m$-tuple in $A$.
If $y_1e_1\neq 0$ the fact that $L_1$ is minimal entails that
there exists an element $u_1$ such that
\[ u_1y_1e_1=e_1
\]
This would give
\[
1= (u_1y_2+1)e_2+\ldots+(u_1y_m+1)e_m
\]
which contradicts the minimality of $m$ in (xx).
Hence $y_1e_=0$ and in a similar way one proves that
$y_\nu e_\nu=0$ for every $\nu$.
So one has a direct sum decomposition
\[
A=\oplus\,Ae_\nu\quad\&\quad 1=e_1+\ldots+e_m
\]
\medskip

\noindent
{\bf{Exercise.}}
Show that (x) implies that
$\{e_\nu\}$ are idempotents and that
\[
e_k\cdot e_\nu=0\quad\colon k\neq \nu
\]
Moreover, from  the above
\[
\rho(e_\nu) =\rho(x_*)=m
\] 
hold for every $\nu$ and hence the diirect sum decompsotion 8xx) gives
\[
\dim A= m\cdot \rho(x_*)
\]





\newpage



\noindent
Let $n\geq 2$ and $A$ is an $n\times n$-matrix whose elements are complex numbers.
Let $\lambda$ be a complex parameter and $E_n$ the identity matrix of order $n$.
The charactwraitic polynomial of $AQ$ is defined by
\[
P(\lambda)= \det (\lambda\cdot E_n-A)\tag{0.1}
\]
The zeros of $P$ is a finite set of comppex numbers denoted by
$\sigma(A)$ and called the spectrum of $A$.
Since matrices with non-zero determintns are inveritble, it follows
that
there exists inverse matrices
\[
R_A(\lambda)=
(\lambda\cdot E_n-A)^{-1}\quad\colon\quad
\lambda\in{\bf{C}}\setminus\sigma(A) \tag{0.2}
\]

\noindent
Moreover, the construction of inverse matrices via Cramer's rule entails
gives an $n$-tuple of matrices
$\{Q_\nu\}$ such that 
\[
R_A(\lambda)=\frac{1}{P_A(\lambda)}\cdot \sum_{\nu=0}^{n-1}\,
\lambda^\nu\cdot Q_\nu\tag{0.3}
\]
Next, when $|\lambda|$�is large we can express $R(\lambda$ by the Neuann series:
\[
R(\lambda)=
(\lambda\cdot E_n-A)^{-1}\quad\colon\quad
\lambda\in{\bf{C}}\setminus\sigma(A) \tag{0.4}
\]
\medskip

\noindent
{\bf{1. Exercise.}}
We can construct
line integrals over circles 
$|\lambda|=w$ where
$w$ is strictly 
larger than
the absolute value of every
root of
$P_A(\lambda)$. Residue calcuus teaches that  
\[
\frac{1}{2\pi i}\int_{|\lambda|=w}\,
\lambda^\nu=0\quad\colon\quad \nu\neq -1
\]
Show that (0.4) gives
\[ 
A^k=\frac{1}{2\pi i}\int_{|\lambda|=w}\,
\lambda^k\cdot R_A(\lambda)\cdot d\lambda
\quad\colon\quad k=1,2,\ldots\quad\&\quad
E_n=\frac{1}{2\pi i}\int_{|\lambda|=w}\,
R_A(\lambda)\cdot d\lambda
\]
More generally, if  $q(\lambda)$ is an arbitrary polynomial then
\[ 
q(A)=\frac{1}{2\pi i}\int_{|\lambda|=w}\,
q(\lambda)\cdot R_A(\lambda)\cdot d\lambda\tag{1.1}
\]
\medskip

\noindent
{\bf{2. Some idempotent matrices.}}
Consider  a zero $\alpha$ of $P_A(\lambda)$
whose multiplicity is  some positive integer $e_\alpha$.
Put
\[
P_\alpha(\lambda)=
\prod_{\beta\neq \alpha}\, (\lambda-\beta)^{e_\beta}
\]
where the profict is taken over the reamining zeros of $P(\lambda)$.
Euclidian divsions give a unique polynomial
$\rho_\alpha(\lambda)$ of degree $\leq e_\alpha$ such that
\[
\rho_\alpha(\lambda)\cdot P_\alpha(\lambda)= 1+(\lambda-\alpha)^{e_\alpha}\cdot S_\alpha(\lambda)
\]
where $S_\alpha$ is another polynomial.
\medskip

\noindent
{\bf{Exercise.}}
Show that
\[
\rho_\alpha(A)\cdot P_\alpha(A=
\rho=\frac{1}{2\pi i}\int_{|\lambda-\alpha|=\delta}\,
R(\lambda)\,d\lambda\tag{1}
\]
hold for all  $\delta>0$ for which the distance from
the root $\alpha$ to other points in $\sigma(A)$�are $>\delta$.
Moroeover, put
\[
E_\alpha=\rho=\frac{1}{2\pi i}\int_{|\lambda-\alpha|=\delta}\,
R(\lambda)\,d\lambda\tag{1}
\]
and show that this matrix is idempotent, i.e.
\[
E_\alpha=E_\alpha^2
\]
Finally, show that
\[
(A-\alpha\cdot E_n)^k\cdot E_\alpha=\frac{1}{2\pi i}\int_{|\lambda-\alpha|=\delta}\,
\lambda^k\cdot R(\lambda)\,d\lambda\tag{1}
\]
hold for every $k\geq 1$.
\bigskip


\noindent
{\bf{The minimal polynomial $p_*$ of $A$.}}
To begin with (1.1) and (0.3) give
\[
p(A)=\sum 
\frac{1}{2\pi i}\int_{|\lambda|=w}\,\lambda^\nu\,d\lambda\cdot Q_\nu
\]
and the right hand side is zero because the complex ine integrals over
$\lambda$-monomials are all zero.
Hence
we have
\[
p(A)=0
\] 
Next, in the rpniciapl ideal domain of pilynomials of one
variable over the complex field, we find a unique monic plynomia $p_*$ of smallest
degree  such that
$p_*(A)=0$.
Now one has a favtorisation
\[
p(\lambda)= \prod\, (\lambda-\alpha)^{e_\alpha}
\]
with the prodict taken over all $\alpha$ in $\sigma(A)$.
Since $p(A)=$ it follows that
\[
p_*(\lambda)= \prod\, (\lambda-\alpha)^{j_\alpha}
\]
where $\{j_\alpha\}$
are integers and $j_\alpha\leq e-\alpha$ for every
$\alpha$.
It turnd out that these $j$-integers acan be found via
the meromorphic matrix-valued function $R(\lambda)$.
\medskip

\noindent
{\bf{Theorem}}.
\emph{For each $\alpha\in \sigma(A)$ the integer $j_\alpha$ is
equal to the order ot the ple of
$R(\lambda)$ at $\alpha$.}



\medskip

\noindent
{\bf{Exercise}} Prove Theorem xx.

\bigskip

\centerline{\bf{Jordan's theorem.}}
\bigskip

\noindent
We are given a finite dimensional complex vector space
$V$ and $A\colon V\to V$ is a linear operator which is
nilptent, i.e. there exists an integer $N$ such that
$A^N$ is the zero operator.
Every non-zero vector $x\in V$ generates an $A$-invariant subspace denoted
by $\mathcal C(x)$.
It menas that $\mathcal C(x)$ is the vector space generated
by $x,Ax,A2x,\ldots$.
Let $j(x)$ be the smallest positive integer such that
\[
A^{j(x)}(x)=0
\]
\medskip

\noindent
{\bf{Exercise.}}
Show  that $\mathcal C(x)$ is has dimensionl $j(x)$ where
the vectors $x,Ax,\ldots,A^{j(x)-1}$ is a basis.
\medskip

\noindent
Now we anounce Jordan's theorem.
\medskip

\noindent
{\bf{Theorem.}}
\emph{The ector sopace $V$ is a direct sum of cyclic subspaces,  i.e. there exists
a finite set of vectors
$\{x_\nu\}$ siuch that}
\[
V=\oplus\, \mathcal C(x_\nu)
\]
\medskip

\noindent
\emph{Proof.})
we shall use an induction over $\dim V$.
To begin with we choose a vector $x_*\in V$ such that
\[
j(x_*)= \max_x\, j(x)
\]
Consider he vector space
\[
W=\frac{W}{\mathcal C(x_*)}
\]
Since $\mathcal C(x)$ is an $A$-invariant subspace of $V$, it follows that
$A$ induces a linear operator
$\bar A$ on $W$ whic hobvousoy also is nilpotent.
By the iunduction over $\dim V$ we can assume that
Jordan's theorem holds for $W$.
Hence
\[
W=\oplus\, \mathcal C(\bar x_\nu)
\]
where $\{\bar x_\nu\}$ are images of $V$-vectors $\{x_\nu\}$.
By the condtriution of the quitoent space $W$, the $j$.numers of
the vectors $\bar x_\nu$ are given by integers $\bar j(\nu)$ 
whosen to be minimal so that
\[
A^{\bar j(\nu)}(x_\nu)\in \mathcal C(x_*)
\]
hold in $V$. So with $\nu$ kept fixed we can wirte
\[
A^{\bar j(\nu)}(x_\nu)=c_1x_*+\ldots+c_{j(x_*)-1}\cdot x_*^{j(x_*)-1}
\]
Here
\[
j(\nu)\leq j(x_\nu)\leq j(x_*)
\]
So with 
$k=j(x_*)-j(\nu)$
we get
\[
0= A^k(A^{\bar j(\nu)}(x_\nu)=
c_1A^k(x_*) +\ldots+c_{j(x_*)-1}\cdot A^{k+j(x_*)-1}(x_*)
\]
Since $\{A^\nu(x_*)\,\colon 0\leq \nu\leq j(x_*)\}$
are inearlu indep+endent it follows that
\[
c_1=\ldots c_{j(\nu)-1}=0\implies
\]
\[
A^{\bar j(\nu)}(x_\nu)= A^{j(\nu)}(\xi_\nu)\quad\colon, \xi_\nu\in \mathcal C(x_*)
\]
Replacin $x_\nu$ by $x_\nu-\xi_\nu$ does not change its image in
$W$ and here
\[ 
j(\nu)= j(x_\nu-\xi_\nu)
\]
So now one has a direct sum decompostion
\[
W=\oplus\, \mathcal C(\bar x_\nu)
\quad\colon \, j(\nu) =j(x_\nu)
\]
At this tage the reade can check that $V$ is the direft sium of the cyuc lic suspaces
$\{\mathcal C(x_\nu)\}$ in (x) together wit
$\mathcal C(x_*)$. This finihses the proof of Jordan's theorem.







\newpage




Resturing to (0.3)
we notice that since $P_A(\lambda)$ is a polynomial of degree $n$
with highest coefficient equal to one,  it follows that
\[
Q_{n-1}= E_n
\]
Next, with $k=1$ in (0.6) one has 
\[
A=Q_{n-2}+ \lim_{R\to\infty}\, \frac{1}{2\pi i}\cdot \int_{|\lambda|=R}\,
\frac{\lambda^n}{P_A(\lambda)}\, d\lambda\tag{0.10}
\]
Let  us write
\[
P_A(\lambda)= \lambda^n+c_{n-1}\cdot \lambda^{n-1}+\ldots+c_0
\]
The  reader can  check that the last term in (0.10) is $c_{n-1}$
and hence
\[
Q_{n-2}= A-c_{n-1}\tag{0.11}
\]
If one   continues in this way it follows that  each $j\geq 2$ gives 
\[ 
Q_{n-j}=q_j(A)
\]
where $q_j(A)$ is a polynomial in $A$ of degree $\leq j-1$.
When $j=n$ the reader can check that Cauchy'sresidue formula gives
\[
Q_0=\frac{1}{2\pi i}\cdot \int_{|\lambda|=w}\,
\frac{P_A(\lambda)\cdot R_A(\lambda)}{\lambda}\, d\lambda=
A^{n-1}+c_{n-1}A^{n-2}+\ldots +c_2A+c_1\cdot E_n
\]





\medskip


\noindent
{\bf{1. The case when $P_A(\lambda)$ has simple roots.}}
Let $\alpha_1,\dots,\alpha_n$ be the simle roots.
To each $1\leq k\leq n $ we put
\[
 \mathcal C_k(A)=
 \frac{1}{\prod_{\nu\neq k}\, (\alpha_k-\alpha_\nu)}
\cdot\prod_{\nu\neq k}\,(A-\alpha_\nu E_n)\tag{1.1}
\]
When $\lambda$ is outside $\sigma(A)$ we get the matrix
\[
\mathcal C(\lambda)= \sum_{k=1}^{k=n}
 \frac{1}{\lambda-\alpha_k}\cdot \mathcal C_k(A)\tag{1.2}
 \]
With these notations one has the equation  below which is due to
Cayley, Hamilton and Sylvester:
\[
\mathcal C(\lambda)=R_A(\lambda)\tag{1.3}
\]
\medskip


\noindent
{\bf{Exercise.}} Prove  (1.3) using residuye calculas and the previous equations.
\medskip

\noindent
{\bf{2. The Cayley-Hamilton polynomial.}}
It is by
definition  the unique monic polynomial $p_*(\lambda]$ in the 
polynomial ring ${\bf{C}}[\lambda]$
of smallest possible degree such that the associated
matrix
$p_*(A)=0$.
It is found as follows:
Let $\alpha_1,\ldots,\alpha_k$
be the distinct roots of $P_A(\lambda)$
so that
\[
P_A(\lambda)=
\prod_{\nu=1}^{\nu=k}\, 
(\lambda-\alpha_\nu)^{e_\nu}
\]
where $e_1+\ldots+e_k=n$.
From (0.3) and (0.7) it is clear that
\[
P_A(A)=0
\]
Hence $p_*(\lambda)$ is a factor of the 
chacteristic polynomial $P_A(\lambda)$.
If $P_A$ has multiple  zeros it can occur that
the degree of $p_*(\lambda)$ is strictly smaller than$ n$.
To get the exact formula for
$p_*\lambda$ one needs Jordan's theorem in � 3
where we also explain how to compiute the minimal
polynomial  $p_*$ attached to our given matrix $A$.


\bigskip

\noindent
{\bf{2.1 Residue matrices.}}
Let $\alpha_1,\ldots,\alpha_k$ be the distinct zeros
of $P_A(\lambda)$.
For a given root, say $\alpha_ 1$ of  multiplicity
$p\geq 1$ we have a local Laurent series expansion
\[
R_A(\alpha_1+\zeta)=
\frac{G_p}{\zeta^p}+\ldots+\frac{G_1}{\zeta}+
B_0+\zeta\cdot B_1+\ldots\tag{i}
\]
which converges in a disc $\{|\zeta|<\delta\}$.
One  refers to $G_1,\ldots,G_p$ as the residue matrices at $\alpha_1$.
Choose a polynomial $q(\lambda)$ in ${\bf{C}}[\lambda]$
which vanishes up to the multiplicity at all the
the remaining roots
$\alpha_2,\ldots,\alpha_k$ while it has a zero of order $p-1$ at $\alpha_1$, i.e.
locally
\[
q(\alpha_1+\zeta)= \zeta^{p-1}(1+q_1\zeta+\ldots)\tag{i}
\]

\noindent
{\bf{2.2 Exercise.}}
Use residue calculus and   show that:
\[ 
q(A)=
\frac{1}{2\pi}\int_{|\lambda-\alpha_1|=\epsilon}\,
q(\lambda)\cdot R_A(\lambda)\cdot d\lambda=
G_p\tag{*}
\]
Hence the matrix $G_p$ is a polynomial of $A$. In a similar way one proves that
every $G$-matrix in the Laurent series (i) is a polynomial in $A$.
\medskip


\noindent
{\bf{2.3 Some idempotent matrices.}}
Consider  a zero $\alpha_j$ and choose a polynomial
$Q_j$ such that
$Q_j(\lambda)-1$ has a zero of order $e(\alpha_j)$ at
$\alpha_j$ while $Q_j$ has a zero of order
$e(\alpha_\nu)$ at the remaining roots.
Set
\[ 
E_A(\alpha_j)=\frac{1}{2\pi i}\int_{|\lambda|=w}\,
Q_j(\lambda)\cdot R_A(\lambda)\cdot d\lambda\tag{1}
\]
where $w$ is large as in 2.5.
Since 
the polynomial 
$S=Q_j-Q_j^2$ vanishes up to the multiplicities
at all the roots of $P_A(\lambda$ we have $S(A)=0$ from (0.9) in
which entails that

\[
E_A(\alpha_j)=E_A(\alpha_j)\cdot E_A(\alpha_j)\tag{2.3.1}
\]
In other words, we have constructed an idempotent matrix.

\bigskip



\noindent
{\bf{2.4 The Cayley-Hamilton decomposition.}}
Recall the equality
\[ 
E_n=\frac{1}{2\pi i}\cdot\int_{|\lambda|=w}\,
R_A(\lambda)\cdot d\lambda
\]
where the radius $w$ is so large that the disc $D_w$ contains the zeros of 
$P_A(\lambda)$.
The previous construction of the $E$-matrices at
the roots of $P_A(\lambda)$  entail that
\[
 E_n=E_A(\alpha_1)+\ldots+E_A(\alpha_k)
\]


\noindent
Identifying $A$ with a ${\bf{C}}$-linear operator on
${\bf{C}}^n$ we obtain a direct sum decomposition
\[ 
{\bf{C}}^n= V_1\oplus\ldots\oplus V_k\tag{*}
\] 
where each $V_\nu$ is an $A$-invariant subspace given by 
the image of $E_A(\alpha_\nu)$.
Here $A-\alpha_\nu$ restricts to a \emph{nilpotent} linear operator on
$V_\nu$ and the dimension of this vector space is equal to the multiplicity
of the root $\alpha_\nu$ of the characteristic polynomial.
One refers to (*) as the \emph{Cayley-Hamilton decomposition} of
${\bf{C}}^n$.
\bigskip

\noindent
{\bf{2.5 About invertible matrices.}}
Consider the characteristic polynomial $P_A(\lambda)$
and let us write
\[
 P_A(\lambda)= 
 \lambda^n+c_{n-1}\lambda^{n-1}+\ldots+ c_1\lambda+c_0
\]

\noindent
Notice that
$c_0=(-1)^n\cdot\text{det}(A)$.
So if the determinant of $A$ is $\neq 0$
the vanhsi ng of $P_A(A)$ gives the equation
\[ 
A\cdot [A^{n-1}+c_{n-1}A^{n-2}+\ldots+ c_1\bigr]=
(-1)^{n-1}
\text{det}(A)
\cdot E_n
\]



\noindent
Hence 
the inverse $A^{-1}$ is  a polynomial in $A$.
\medskip





\noindent
{\bf{2.6 Similarity of matrices.}}
Recall
that
the determinant of  a matrix $A$ does not  change when it is
replaced by $SAS^{-1}$
where $S$ is an arbitrary invertible matrix.
This implies  that
the coefficients of the
characterstic polynomial  $P_A(\lambda)$
are intrinsically defined via the
associated linear operator, i.e. if another basis is chosen in
${\bf{C}}^n$ the given $A$-linear operator is expressed by a matrix
$SAS^{-1}$ where $S$ effects the change of the basis.
Let us now draw an interesting consequence
of the previous operational calculus.
Let us give the following:
\medskip


\noindent {\bf 2.6.1 Definition.}
\emph{A pair of $n\times n$-matrices $A,B$ are
\emph{similar} if there exists some invertible matrix $S$ such that}
\[ 
B=SAS^{-1}
\]


\noindent
Since the product of two invertible matrices is invertible this yield
an equivalence relation on $M_n({\bf{C}})$ and 
$P_A(\lambda)$� depends only on its equivalence class.
The question arises if  to matrices $A$ and $B$ whose characteristic 
polynomials are equal
are similar in the sense of Definition 2.6.
This is not true in general.
More precisely,
\emph{Jordan normal form}
determines the eventual similarity between
a pair of matrices with the same characteristic polynomial.



\newpage







\newpage


\centerline {\bf \large I:C  Complex vector spaces}

\bigskip


\centerline{\emph{Contents}}
\bigskip


\noindent
0. Introduction
\medskip

\noindent
0.A The Sylvester\vvv Franke theorem
\medskip

\noindent
0.B Hankel determinants
\medskip

\noindent
0.C The Gram-Fredhom formula
\medskip

\noindent
0.D Resolvents of integral operators
\medskip

\noindent
0.D.1 Hilbert derterminants


\medskip

\noindent
0.D.2 Some results by Carleman


\medskip






\noindent
1. Wedderburn's Theorem
\medskip

\noindent
2. Resolvents


\medskip

\noindent
3. Jordan's normal form
\medskip

\noindent
4. Hermitian and normal operators
\medskip

\noindent
5. Fundamental solutions to ODE-equations

\medskip

\noindent
6. Carleman's  inequality for resolvents

\medskip

\noindent
7. Hadamad's  radius formula

\medskip

\noindent
8. On Positive definite quadratic forms
\medskip

\noindent
9. The Davies-Smith inequality
\medskip

\noindent
10. An application to integral equations























\newpage


\centerline {\bf Introduction.} 
\bigskip

\noindent
The 
modern era about
matrices and determinants started around 1850
with major contributions by Hamilton, Sylvester  and Cayley.
In � 1 and � 2 we expose  results
which foremost are due to these mathematicians.
Matrices and their determinants of arbitary high order 
were defined around 1810. One 
result from this early period is Cauchy's spectral theorem for
a symmetric $n\times $n-matrix $A$ whose characteristic   polynomials have
simple zeros. More precisely, when the elements of $a$ are real, Cacuhy  proved
that there exists an orthogonal matrix $U$ such that
$U^*AU$ is a a diagonal matrix. His result was later extended 
by Weierstrass to the general case when
multiple zeros appear.
Considerable credit to the whoke subjhect
dealing with linear systems of equations
must also be given to
Cramer. Already in 17xx he  gave the general inversion fornula for
$4\times 4$-matrices.  
More refined results were later established by Laplace
around 1820 , such as his general expansion theorem to calculate
determinants.
Another major achievement is due to Camille Jordan. His theorem from
1850
is exposed in � 3.
\bigskip

\noindent
Some facts will
be taken for granted.
The reader is expected to be familiar with the
construction of determinants of 
$n\times n$-matrices $A=\{a_{pq}\}$.
But let  us recall Cram�r's formula to solve
systems of linear equations
\[ 
\sum_{p=n}^{p=n}\, a_{pq}\cdot x_q=y_p\quad\colon\quad 1\leq p\leq n
\]
Under the hypothesis that
$\det(A)\neq 0$ this system has a unique solution
$x_\bullet$ for every complex $n$-vector $y_\bullet$.
It is obtained as follows:
 For each pair $1\leq p,q\leq n$ one deletes the
$p$:th row and the $q$:th colum from $A$ which gives an
$(n-1)\times (n-1)$-matrix denoted by
$A[p,q]$. Put
\[
C_{p,q}=(-1)^{j+q}\cdot  \det(A[p,q])
\]
Computing the determinant of $A$ via an expansion along the $p$:th row gives   
\[
\sum_{p=1}^{p=n}\, a_{pq}\cdot C_{p,q}=
\det(A)\tag{i}
\]
At the same time one has
\[
\sum_{p=1}^{p=n}\, a_{pq}\cdot C_{j,q}=0\quad\colon\, j\neq p\tag{ii}
\]
which follows because
a matrix with two equal rows has a zero determinant.
Given an $n$-vector $y$ we set
\[
x_q=\frac{1}{\det(A)}\cdot \sum_{j=1}^{p=n}\, C_{j,q}\cdot y_j
\]
Now (i-ii) entail that  $n$-vector $x_\bullet$ solves 
(*).
\medskip

\noindent
{\bf{Example.}} Take $n=2$ and let
\[
A=\begin{pmatrix}
a_{11}&a_{12} \\
a_{21}&a_{22}\\
\end{pmatrix}\implies
\det(A)= a_{11}a_{22}-a_{12}a_{21}
\]
Consider the $2\times 2$-matrix
\[
B=\begin{pmatrix}
C_{11}&C_{21} \\
C_{12}&C_{22}\\
\end{pmatrix}=
\begin{pmatrix}
a_{22}&-a_{12} \\
-a_{21}&a_{11}\\
\end{pmatrix}
\]
The rule for products of matrices show that
$B\cdot A= E_2$, i.e. $B$ is the inverse of $A$.
If $n\geq 3$, Cramer's invesrion formula
for an $n\times n$-matrix $A$
with a non-zero determinant is given by
\[
A^{-1}= \frac{1}{\det A}\cdot
\begin{pmatrix}
C_{11}&C_{21}&\ldots&C_{n1}  \\
C_{12}&C_{22}&\ldots&C_{n2}  \\
\ldots&\ldots&\ldots&\ldots\\
C_{1n}&C_{2n}&\ldots&C_{nn}\\
\end{pmatrix}\tag{*}
\]






\bigskip


\noindent
Another  fundamental fact is the product formula for determinants:
\[
\det(A)\cdot \det(B)= \det(AB)
\]
which hold for every pair of $n\times n$-matrices.
The verification is exposed
in many text-books and  therefore left to the reader.
A consequence is that a determinant can be associated in an intrinsic way to
a linear operator $L$ on an $n$-dimensional complex vector space $V$
where one
from the start has not fixed a basis.
Given a basis $e_1,\ldots,e_n$ one gets a matrix $A=\{a_pq\}$ where
\[
L(e_p)=
\sum_{q=1}^{q=n}\,a_{qp}e_q
\]
i.e. the columns of the $A$-matrix  give  the vectors
$\{L(e_p)\}$.
If $f_1,\ldots,f_n$ is another basis 
then
$L$ is represented by another matrix $B$
and one has the equality
\[
B=SAS^{-1}
\]
where $S$ is an invertible matrix
which interchanges the $e$-basis with the $f$-basis.
The product rule gives $\det(A)=\det(B)$ and this common determinant
is therefore associated to the linear operator $L$.
A perspective on the construction of determinants arises via
exterior products of a given vector space
$V$, i.e. for every $1\leq p\leq n$ one considers 
elements of the form
\[
v_1\wedge\ldots\wedge v_p
\]
with the rule that 
\[
v_{\sigma(1)}\wedge\ldots\wedge v_{\sigma(p)}=
\text{sign}(\sigma)\cdot v_1\wedge\ldots\wedge v_p
\]
where $\text{sign}(\sigma)$ is the signature of the permutation $\sigma$
acting on $1,\ldots,p$.
Now one get a vector space
denoted by $\wedge^p(V)$ whose dimension os $\binom{n}{p}$.
It is called the $p$-fold exterior product of $V$. A linear operator
$L$ yields a linear operator $\wedge^p(L)$ on 
this vector space. When $p=n$  one has a 1-dimensional vector space and
$\det(L)$ corresponds to a scalar multiplication on
$\wedge^n(V)$ by this complex number.
\bigskip


\noindent
Since many text-books expose linear algebra
we shall not dwell  upon general
constructions, but pay attention to   results ehichminvolve \emph{inequalities}
since this is the central issue in mathematics, at least if one is concerned with
analysis.
To attain this one often needs
quite involved results  about determinants.
An example is Hadamard's
theorem in the chapter devoted to series which gives
conditions on order that a  power series with 
a finite positive radius of convergence extends to
a meromorphic function in a larger disc.
Lack of space prevents us from 
a more detailed account about the
positions of zeros of analytic functions vis  their Taylor series.
Let us only remark that if
$P(z)=1+c-1z+\dots+c_nz^n$ is a polynomial whose
zeros are simple, then the sequence of their 
absolute values $\{|\alpha_\nu\}$
can be recaptured via determinants.
This goes back to work by Runge in his article [Acta Mathemaca vol 6]
and is also treated in Hadamard's cited thesis.
One should  add that Bernoulli found the smallest absolute value of
the roots of $P(z)$ by investigating its logarithmic  derivative.
See � xxx. It is therefore no surprise that analytic function theory
has become a very useful tool
to study matrices and their determinants, and conversely
calculations with determinants lead to some
quite remarkable results in analytic function theory.
Recall also that 
the postion of roots to characteristic  polynomials are
used to study  asymptotic properties of solutions to
ordinary differential equations.
An example where analytic function theory was used to achieve
conditions in order tht
the roots  of a charaterstic polynomia,stay in a half-space
$\mathfrak{Re}\,\lambda<0$
was given by Eduard Routh in his famous treatise from 1876.
Since his proof relies on analytic function theory
it is given in my notes on this topic.
\medskip

\noindent
{\bf{An open problem.}}
While we review classic results in this chapter it may be of interest to
give an example of an open problem in the spirit of
the subsequent material.
Let $m\geq 2$ and consider a differential operator
\[
T=q_m(x)\partial^m+\ldots +x\partial
\]
where $\partial=\frac{d}{dx}$ and
$\{q_\nu\}$ are polynomials such that
\[
\deg q_\nu\leq \nu-1\quad\colon\, 2\leq \nu\leq m
\]
It is easily seen that for evety positive integer $n$ there exists a unique monic plynomial $p_n(x)$
of degree $n$ such that
\[
T(p_n)= n\cdot p_n
\]
Denote  by $r^*_n$ the maximum of the absolute values of the zeros of
$p_n$.
A result due to T. Bergquist in the phd-thesis [Berg]
shows that there exists a constant $c_*>0$ such that
\[
r^*_n\geq c_*
\]
hold for every $n\geq 1$.
But the following is an open:
\medskip

\noindent
{\bf{Conjercture.}}
Does there exists a constant $c^*$ which depends on $�t$ nnoly such that
\[
r^*_n\leq c^*\cdot n\tag{*}
\]
hold for every $n\geq 1$.
Let us remark that numerical experiments
indicate that $c^*$ exists. 
But so far no proof has been found, except for special cases.
For example, in situations where the zeros of
the eigenpolynomials above are real then it is easily verified via Sturm 
chains that $c^*$
exists.
It is tempting to try to apply the
studiues by Hadamard and Ringe to settle the conjecture. But
I have bern unable to do this and
perhaps some rader is able to find the answer.
For a proof of the lower boud $c_*$ I refer to a joint
preprint by T. Bergquist and myself.








\medskip

\noindent
{\bf{Integral equations.}}
The calculus with determinants
is   used  to  study  integral equations.
The \emph{Gram-Fredholm formula} given in 
� xx is the starting point to analyze
spectra of linear operators expresse via kernel functions.
Results about matrices of finite order and their determinants
have 
paved the way to operator theory on normed linear spaces
which in general are infinite dimensional.

\bigskip

\noindent
{\bf{An example.}}
As  alreafy said the aim of this   chapter is to
expose precise inequalities.



An important result is the \emph{spectral theorem} for symmetric
and real $n\times n$\vvv matrices,  and its counterpart for
complex Hermitian matrices.
In both these cases   eigenvalues are found
by regarding maxima and minima  of  quadratic forms.
Far-reaching studies  of quadratic forms appear in the colected work by
Weierstrass  which contains a wealth of  results
related to the spectral theorem for hermitian matrices and
their interplay with quadratic forms.
One should also mention later  investigations by Frobenius
about quadratic forms. 
Here is an  example
from Weierstrass' studies which goes as follows:
Let
$N\geq 2$ and $\{c_pq\,\colon 1\leq p,q\leq N\}$  a doubly indexed sequence of
positive numbers which is symmetric, i.e.  $c_{qp}= c_{pq}$
hold for all pairs $1\leq p,q\leq N$.
Suppose that
\[
\sum_{q=1}^{q=N}\, c_{p,q}\leq 1\quad\colon 1\leq p\leq N
\]
Then 
\[
\sum_{p=1}^{p=N}\, \bigr[\sum_{q=1}^{q=N}\,
c_{p,q}\cdot x_q\bigr]^2\leq 
\sum_{p=1}^{p=N}\, x_p^2\tag{0.1}
\]
hold for every $N$-tuple $\{x_p\}$ of non-negative real numbers.
The proof uses
the spectral theorem for symmetric matrices and is given in � xx.
A result with a wide range of applications due to frobebnius goes as follows:
Let $\{a_{pq}\,\colon 1\leq p,q\leq N\}$
be a double indexed family of positive real numbers where symmetry is not assumed.
This double indexed family are elements of an
$N\times N$-matrix $A$ which yields a linear operator on
${\bf{R}}^N$.
We shall learn how to contruct determinants and
the zeros of the polynomial $P_A(\lambda)= \det(\lambda\cdot E_N-A)$
are in general complex numbers.
When the elements of $A$ are positive real numbers
Frobenius proved that
there exists a unique $N$-vector $x^*=(x^*_1,\dots,x^*_N)$
where every $x^*_\nu>0$ and $\sum\, x^*_\nu=1$
which is an eigenvector for $A$, i.e.
\[
A(x^*)= \rho \cdot x^*
\]
holds for a positive real number $\rho$. Moreover, $\rho$�is a simple zero of
$P_A(\lambda)$ and the absolute value of  every other root 
is $<\rho$.
A result where the calcukus  based upon determinants and
solutions to  systems of  linear equations using the rule of Cramer
becomes useful appears in Hadamard's theorem exposed in
�xx
which 
give necessary and sufficiet conditions for in order that 
a complex powere series $\sum\, c_n\cdot z^n$
which from the start have some finite radius of convergence.
$\rho>0$ extends to a meromrphic function in a large disc $|z|<\rho^*$.
Among other important  results one should mention the
theorem due to Camille Jordan
which shows that a linear operator after a suitable
linear transformation is represented by a matirx of special form.
 






\medskip



\noindent
Using Lagrange's interpolation formula 
Sylvester exhibited
extensive classes of matrix-valued functions by residue calculus and
more delicate results were achieved by Frobenius who  treated
the general case when a characteristic
polynomial of a matrix has multiple roots.
Thjs is exposed in � 0.4. 
Passing to
inifinite dimensions, the usefulness of 
matrices and their determinants  was put forward 
by Fredholm in his studies of   integral 
equations. Here  estimates are  needed
to control determinants of matrices of large size  to
study  resolvents of linear operators acting on infinite dimensional vector spaces.
To handle cases where  
singular kernels appear in  an
integral operator, modified Fredholm
determinants were introduced by Hilbert whose  text\vvv book  \emph{Zur
Theorie der Integralgleichungen} from 1904  laid  the foundations
for  spectral theory  of
linear operators on infinite dimensional spaces.
A  systematic study of matrices with infinitely many elements was
done by Hellinger and Toeplitz in their joint article
\emph{Grundlagen f�r eine theorie der undendlichen matrizen} from 1910
and  applied to solve
integral equations of
the Fredholm-Hilbert type.
During these investigations
Carleman's  inequality for norms of resolvents
in � 6 is  a veritable cornerstone.
Let me remark that Carelan's proof ffers a very insructive lesson
in the subject dealing with
matrices and their determinants.



\bigskip



\centerline{\emph{Outline of the content.}} 
\medskip

\noindent
Here  follows 
a brief presentation of basic material.
To each integer $n\geq 2$ we denote by  
$M_n({\bf{C)}}$ the set of $n\times n$\vvv matrices with complex elements.
As a complex vector space 
$M_n({\bf{C)}}$ 
has dimension $n^2$ and it is
an associative ${\bf{C}}$-algebra defined by
the usual matrix product where the
identity  $E_n$ is the matrix whose elements outside
the diagonal are zero while $e_{\nu\nu}=1$ for every
$1\leq\nu\leq n$.
When $n\geq 2$
a pair of $n\times n$-matrices $A$ and $B$
do not commute in general which
means that $M_n({\bf{C}})$ is a non-commutative algebra
over the complex field. 
In � 1 we prove Wedderburn's theorem which 
asserts that the matrix algebras $\{M_n({\bf{C}})\}$ are the sole  finite dimensional
complex algebra with no other  two-sided ideals
than the zero ideal and the whole algebra.
Resovents are studied in
� 2.
They consist of  inverse matrices 
$R_\lambda(A)= \lambda\cdot E_n-A)^{-1}$ 
when $\lambda$ is outside the spectrum $\sigma(A)$ of a matrix $A$
defined as the set of 
zeros of the characteristic polynomial
\[
P_A(\lambda)=\det(\lambda\cdot E_n-A)\tag{0.1}
\]

\noindent
A fundamental fact is that
$P_A(\lambda)$ only depends upon the associated linear operator
defined by the $A$-matrix. More precisely, if $S$ is an invertible matrix 
the product formula for determinants give the equality
\[
P_A(\lambda)=P_{SAS^{-1}}(\lambda)\tag{0.2}
\]
Using some analytic function theory
one gets a certai calculus with resolvents
which was 
carried out by Cayley, Hamilton and Sylvester and was
later extended by Carl Neumann to study inverses of linear operators on normed
vector spaces which in general need not be bounded, but only densely defined.
So inspired by the material in the present chapter which
deals with finite-dimensional situations, the
Neumann calculus which atarted in 1880
has become a corner stone in operator theory
and  exposed in my notes about functional analysis.


\medskip

\noindent
{\bf{Final remark.}}
The subsewuent material os foremost devoted to
establish \emph{inequalities}, i.e. reader who prefer
formal and more abstract theories
may refrain from studying our account about linear operators in 
finite dimensional vector spaces,
while they are of great importance in applications to
various problems in analysis.
An example of the psirit in the sections below is
the followinfg inequality for
iterated functions.








\newpage

\centerline
{\bf{A. Matrices and determinants.}}
\bigskip


\noindent
Let  $A$ be an $n\times n$-matrix whose 
elements $\{a_{pq}\}$ are complex numbers.
The Hilbert-Schmidt norm is defined
by
\[
{\bf{HS}}(A)=\sqrt{ \sum\sum\, |a_{pq}|^2}\tag{*}
\]
where the doube sum extends over all pairs
$1\leq p,q\leq n$.
The operator norm is defined by:
\[
||A||=
\max_{z_1,\ldots z_n}\,\sqrt{
\sum_{p=1}^{p=n}\, \bigl|\sum_{q=1}^{q=n}\, a_{pq}z_q|^2\,}\tag{**}
\]
with the maximum taken over $n$-tuples of complex numbers
such that
$\sum\, |z_p|^2=1$.
Introduce the Hermitian inner product on
${\bf{C}}^n$
and identify  $A$ with the linear operator which
sends a basis vector $e_q$ into
\[ 
A(e_q)= \sum_{p=1}^{p=n}\, a_{pq}\cdot e_p
\]
If $z$ and $w$ is a pair of complex $n$-vectors one gets:
\[ 
\langle Az,w\rangle=
\sum\sum a_{pq}z_q\bar w_p
\]
The Cauchy-Schwarz inequality
gives
\[
\bigl|\langle Az,w\rangle\bigr|^2\leq
\bigl(\sum_{p=1}^{p=n}\,  \bigl|\sum_{q=1}^{q=n}\, a_{pq}z_q|^2\,\bigr)
\cdot 
\sum_{p=1}^{p=n}\, |w_p|^2\tag{1}
\]
So if both $z$ and $w$ have length $\leq 1$
The definition of the operator norm entails that
\[
||A||=\max_{z,w}\, |\langle Az,w\rangle\bigr|\tag{2}
\]
where the maximum is taken over
vectors  $z$ and $w$ of unit length.
Next, another application of the Cauchy-Schwarz inequality
shows that if $z$ has unit length, then
\[
\sum_{p=1}^{p=n}\,  \bigl|\sum_{q=1}^{q=n}\, a_{pq}z_q|^2
\leq 
\sum_{p=1}^{p=n}\sum_{q=1}^{q=n}\, |a_{pq}|^2\tag{3}
\]
Then (1) and (3) give the inequality
\[
||A||\leq {\bf{HS}}(A)\tag{4}
\]

\medskip

\noindent{\bf{5. Example.}}
Consider  a matrix $A$ whose elements are non-negative real numbers.
Then it is clear that the operator norm
is found when
we use complex $n$-vectors
with real and non-negative components.
Thus,
\[
||A||=
\max_{x_1,\ldots x_n}\,\sqrt{
\sum_{p=1}^{p=n}\, \big(\sum_{q=1}^{q=n}\, a_{pq}x_q\bigr )^2\,}\tag{5.1}
\]
taken over real $n$-vectors
for which
$\sum\, x_p^2=1$ with  every $x_p\geq 0$.
The $A$-norm  is found
via Lagrange's multiplier which shows that 
 (5.1) is maximized by a real non-negative $n$-vector $x$
satisfying  the  linear system of equations
\[
\lambda\cdot x^*_j=\sum_{p=1}^{p=n}\, a_{pj}\cdot 
\sum_{q=1}^{q=n}\, a_{pq}x^*_q\tag{i}
\]
Introducing the double indexed numbers
\[
\beta_{jq}= \sum_{p=1}^{p=n}\, a_{pj}a_{pq}\tag{ii}
\]
Lagrange's equations corresponds to the system
\[ 
\lambda\cdot x^*_j=\sum_{q=1}^{q=n}\,\beta_{jq}\cdot x_q^*\tag{iii}
\]
Notice that the $\beta$-mstrix is symmetric, i.e.
$\beta_{jq}=\beta_{qj}$ hold for ech pair.
So (iii) amounts to find an eigenvector
to the symmetric $\beta$-matrix with an eigenvector $x^*$ for which
$x_j^*\geq 0$ hold for each $j$.
In "generic" cases the
$\{\beta_{jq}\}$
are strictly positive numbers, and for such special matrices
methods to attain the largest eigenvalue have been studied extensively by
Frobenius and Perron. 
As a specific   example 
we consider an 
$n\times n$-matrix of the form
\[ 
T_s[n]=
\begin{pmatrix}
1&s&s&\ldots &s \\
0&1&\ldots &s&s\\
\ldots&\ldots&\ldots&\ldots&\ldots\\
\ldots&\ldots&\ldots&\ldots\\
0&0&\ldots &1& s\\
0&0&\ldots &0& 1\\
\end{pmatrix}
\]
In spite of the explicit
expression above the computation of its
operator norm is not so obvious.
For example, with $s=2$ a  result due to   Hankel and Frobenius gives   
\[ 
||T_2[n]||= \cot\frac{\pi}{4n}\tag{*}
\]
Of course, classic formulas of this kind are nowadays
implemented in computer programs where the reader can "check" (*).
But of course it is more instructive to try to prove it.
An excellent text
which offers detailed proofs of many
delicate results about matrices and deterjinants
is
Gerhard Kovalevski's text-book
\emph{Determinantenheorie} from 1909.
Pereonally I have not fonund any more recent text-book
whch contains such
challenging results, and it goes without saying that
theorems like  thos in � A.6 below
cannot be derived by trivial abstract non nsense
�resented in elementary books in linear algebra.
\medskip






\bigskip



\centerline {\bf{A.6  Sylvester's determinant formula.}}
\medskip


\noindent
Let    $A$  be some
$n\times n$\vvv matrix with
elements $\{a\uuu{ik}\}$.
For eacg integer 
$1\leq h\leq n-1$ one constructs the
$(n-h\times (n-h)$-matrix whose elements are

\[ b\uuu{rs}= \det\,
\begin{pmatrix}
a\uuu{11}&a\uuu{12}&\ldots &a\uuu{1h}& a\uuu{1s}\\
a\uuu{21}&a\uuu{22}&\ldots &a\uuu{2h}& a\uuu{2s}\\
\ldots&\ldots&\ldots&\ldots&\ldots\\
\ldots&\ldots&\ldots&\ldots&\ldots\\
a\uuu{h1}&a\uuu{h2}&\ldots &a\uuu{hh}& a\uuu{hs}\\
a\uuu{r1}&a\uuu{r2}&\ldots &a\uuu{rh}& a\uuu{rs}\\
\end{pmatrix}\quad\colon\quad h+1\leq r,s\leq n
\]
\medskip

\noindent
With these notation one has the Sylvester equation:

\[
\det
\begin{pmatrix}
b\uuu{h+1,h+1}&b\uuu{h+1,h+2}&\ldots &b\uuu{h+1,n}\\
b\uuu{h+2,h+1}&b\uuu{h+2,h+2}&\ldots &b\uuu{h+2,n}\\
\ldots&\ldots&\ldots&\ldots\\
\ldots&\ldots&\ldots&\ldots\\
b\uuu{n,h+1}&b\uuu{n,h+2}&\ldots &b\uuu{n,n}\\
\end{pmatrix}=
\bigl[\,\det\begin{pmatrix}
a\uuu{11}&a\uuu{12}&\ldots &a\uuu{1h}\\
a\uuu{21}&a\uuu{22}&\ldots &a\uuu{2h}\\
\ldots&\ldots&\ldots&\ldots\\
\ldots&\ldots&\ldots&\ldots\\
a\uuu{h1}&a\uuu{h2}&\ldots &a\uuu{hh}\\
\end{pmatrix}\,\bigr]^{n\vvv h\vvv 1}\cdot \det(A)\tag{*}
\]
\medskip


\noindent
For a proof of (*) we refer to original work by Sylvester
or 
[Kovalevski: page xx\vvv xx].

\newpage

\noindent
{\bf{A result about symmetric matrices.}}
The next result is also due to Sylvester.
Let 
$n\geq 2$ and consider a symmetric matrix

\[
S=\begin{pmatrix}
s\uuu{11}&s\uuu{12}&\ldots &s\uuu{1n}\\
s\uuu{21}&s\uuu{22}&\ldots &s\uuu{2n}\\
\ldots&\ldots&\ldots&\ldots\\
\ldots&\ldots&\ldots&\ldots\\
s\uuu{n1}&a\uuu{N2}&\ldots &s\uuu{nn}\\
\end{pmatrix}
\]
Now we construct three matrices as follows.
First
we get three $(n-1)\times (n-1)$-matrices

\[
S_1= 
\begin{pmatrix}
s\uuu{22}&s\uuu{23}&\ldots &s\uuu{2n}\\
s\uuu{32}&s\uuu{33}&\ldots &s\uuu{3n}\\
\ldots&\ldots&\ldots&\ldots\\
\ldots&\ldots&\ldots&\ldots\\
s\uuu{n2}&s\uuu{N3}&\ldots &s\uuu{nn}\\
\end{pmatrix}
\quad\colon\quad 
S_2= 
\begin{pmatrix}
s\uuu{12}&s\uuu{13}&\ldots &s\uuu{1n}\\
s\uuu{22}&s\uuu{23}&\ldots &s\uuu{2n}\\
\ldots&\ldots&\ldots&\ldots\\
\ldots&\ldots&\ldots&\ldots\\
s\uuu{n-1,2}&s\uuu{N-1,3}&\ldots &s\uuu{n-1,n}\\
\end{pmatrix}
\]
\medskip
\[
S_3= 
\begin{pmatrix}
s\uuu{11}&s\uuu{12}&\ldots &s\uuu{1,n-1}\\
s\uuu{21}&s\uuu{22}&\ldots &s\uuu{2,n-1}\\
\ldots&\ldots&\ldots&\ldots\\
\ldots&\ldots&\ldots&\ldots\\
s\uuu{n-1,1}&s\uuu{N-1,2}&\ldots &s\uuu{n-1,n-1}\\
\end{pmatrix}
\]
\medskip

\noindent
We have also the $(n-2)\times (n-2)$-matrix
when extremal rows and columns are removed:

\[ S_*=\begin{pmatrix}
s\uuu{22}&s\uuu{23}&\ldots &s\uuu{2,n-1}\\
s\uuu{32}&s\uuu{33}&\ldots &s\uuu{3,n-1}\\
\ldots&\ldots&\ldots&\ldots\\
\ldots&\ldots&\ldots&\ldots\\
s\uuu{2,n-1}&s\uuu{3,n-1}&\ldots &s\uuu{n-1,n-1}\\
\end{pmatrix}
\]
\medskip

\noindent
{\bf{A.6.1  Sylvester's identity.}}
\emph{One has the equation:}
\[
\det S\cdot \det S_*=
\det S_1\cdot \det S_3-\bigl(\det S_2\bigr)^2
\]
\medskip

\noindent
The proof is left as an exercise. If necessay, consult the literature where the most
elegant proofs of course are found in Sylvester's 
original work.



\bigskip


\centerline
{\bf{A.7 The Franke-Sylvester theorem.}}

\medskip

\noindent
Let  $n\geq 2$ and $A=\{a\uuu{ik}\}$ an
$n\times n$\vvv matrix. Each integer 
$1\leq m<n$ gives the family of minors of size $m$.
Thus, one 
one picks $m$ columns and $m$ rows
which give  an $m\times m$\vvv matrix
whose determinant is  a minor of size $m$.
The total number of
such minors is equal to
\[
 \binom{n}{m}^2
\]
Namely, with
$N=\binom{n}{m}$
one  has $N$ many strictly increasing sequences
$1\leq \gamma\uuu1<\ldots\gamma \uuu m\leq n$
where a $\gamma$\vvv sequence corresponds to preserved
columns, and  similarly there exist 
$N$ strictly increasing sequences which correspond to preserved rows.
With this in mind we get  for each pair $1\leq r,s\leq N$
a minor $\mathfrak{M}\uuu {rs}$
where the enumerated $r$:th $\gamma$-sequence preserve columns and 
$s$ corresponds to the enumerated sequence of rows.
Now we obtain the $N\times N$\vvv matrix

\[
\mathcal A\uuu m= \begin{pmatrix}
\mathfrak{M}\uuu{11}&\mathfrak{M}{12}&\ldots &\mathfrak{M}\uuu{1N}\\
\mathfrak{M}\uuu{21}&\mathfrak{M}{22}&\ldots &\mathfrak{M}\uuu{2N}\\
\ldots&\ldots&\ldots&\ldots\\
\ldots&\ldots&\ldots&\ldots\\
\mathfrak{M}\uuu{N1}&\mathfrak{M}{22}&\ldots &\mathfrak{M}\uuu{NN}\\
\end{pmatrix}
\]


\noindent
It is called the
Franke-Sylvester matrix of order
$m$. 

\medskip







\noindent
{\bf{A.7.1 Theorem.}}
\emph{For every $1\leq m<n$ one has
the equality}
\[
\mathcal A\uuu m= \text{det}(A)^{\binom{n\vvv 1}{m\vvv 1}}
\]


\medskip

\noindent
{\bf{Example.}} Consider the diagonal $3\times 3$\vvv matrix:

\[
A=\begin{pmatrix}
1&0&0\\
0&1&0\\
0&0&2\\
\end{pmatrix}
\]

\medskip

\noindent
With $m=2$
we have 9 minors of size 2 and the reader can recognize that
when they are arranged so that we begin to remove
the first column, respectively the first row, then 
the resulting $\mathfrak{M}$\vvv matrix becomes
\[
\begin{pmatrix}
2&0&0\\
0&2&0\\
0&0&1\\
\end{pmatrix}
\]
Its determinant is $4= 2^2$ which is in accordance with the general formula since
$n=3$ and $m=2$ give $\binom{n\vvv 1}{m\vvv 1}=2$.
For a proof of Theorem A.7.1  the reader can consult 
[Kovalevski: page102-105].




\bigskip



\centerline {\bf{A.8  Hankel determinants.}}
\bigskip


\noindent
Let $\{c\uuu 0,c\uuu 1,\ldots\}$
be a sequence of complex numbers.
For each  integer $p\geq 0$ and  every $n\geq 0$
we get the symmetric  
$(p+1)\times (p+1)$-matrix:


\[
\mathcal C\uuu n^{(p)}=
\begin{pmatrix}
c\uuu{n}&c\uuu{n+1}&\ldots&c\uuu{n+p}\\
c\uuu{n+1}&c\uuu{n+2}&\ldots&c\uuu{n+p}\\
\ldots&\ldots&\ldots&\ldots\\
c\uuu{n+p}&c\uuu{n+p+1}&\ldots&c\uuu{n+2p}\\
\end{pmatrix}
\]

\medskip


\noindent
Set
\[
\mathcal D\uuu n^{(p)}=\mathcal C\uuu n^{(p)}\tag{A.8.1}
\]
One
refers to $\{\mathcal D\uuu n^{(p)}\}$
 as the recursive Hankel determinants.
They  are used to describe   properties of the given
$c$\vvv sequence.
To begin with we define  the rank   $r^*$ 
of $\{c_n\}$
as follows:
To every non\vvv negative integer $n$
one has the  infinite vector
\[ 
\xi\uuu n=(c\uuu n,c\uuu{n+1},\ldots)
\]
We say that $\{c\uuu n\}$ has finite rank if
there exists an integer  $N$ such that
$N$ many $\xi$\vvv vectors
are linearly independent and the rest are linear combinations of these.
\medskip

\noindent
{\bf{A.8.2 Taylor series of rational functions.}}
A given  sequence $\{c_n\}$ gives the formal power series
\[
f(x)=\sum\uuu{\nu=0}^\infty\, c\uuu\nu x^\nu \tag{*}
\]
For every $n\geq 1$ we get a new formal power series:

\[
\phi\uuu n(x)= x^{\vvv n}\cdot(
f(x)\vvv \sum\uuu{\nu=0}^{n\vvv 1} c\uuu\nu x^\nu)=
\sum\uuu{\nu=0}^\infty c\uuu{n+\nu} x^\nu\tag{*}
\]
Regariding the right hand side it is
 clear
that the sequence $\{c\uuu \nu\}$ has finite rank if and only if 
$\{\phi_n(x)\}$
generates a finite dimensional complex subspace of the vector space 
${\bf{C}}[[x]]$ whose elements are formal power series.
If this dimension is finite we find a positive integer
$p$ and a 
non\vvv zero $(p+1)$\vvv tuple $(a\uuu 0,\ldots,a\uuu p)$ of complex numbers
such that the power series
\[ 
a\uuu 0\cdot  \phi\uuu 0(x)+\ldots+a\uuu p\cdot \phi\uuu p(x)=0
\]
Multiplying this equation with $x^p$ it follows from the first equaity in (**) that 
\[
(a\uuu p+a\uuu{p\vvv 1} x+\ldots+a\uuu o x^p)\cdot f(x)=q(x)
\]
where $q(x)$ is a polynomial of degree $p-1$ at most.
So when $\{c_n\}$ has finite rank    the power series (*) 
represents a rational function.
\medskip



\centerline{\bf{Exercises.}}
\medskip


\noindent
Assume that
\[
\sum\, c\uuu\nu x^\nu= \frac{q(x)}{g(x)}
\] 
for some pair of polynomials. Show that $\{c_n\}$ has finite rank.
Show also that a sequence 
$\{c\uuu n\}$ has  finite rank if and only if
there exists an integer $p$ such that
\[
\mathcal D\uuu 0^{(p)}\neq 0\quad
\text{and}\quad D\uuu 0^{(q)}=0\quad \colon\quad q>p\tag{4}
\]
and check also that $p$ is equal to the rank $N$ of the given sequence.


\medskip

\noindent
{\bf{A.8.3  specific example.}}
Suppose that the degree of $q$ is strictly less than that of $g$ in Exercise B.1 
and that the rational function $\frac{q}{g}$
is  a sum of simple 
fractions,  i.e. 
\[ 
\sum\, c\uuu\nu x^\nu= \sum\uuu{k=1}^{k=p}\, \frac{d\uuu k}{1\vvv \alpha\uuu k x}
\] 
where $\alpha\uuu 1,\ldots,\alpha\uuu p$ are distinct and every $d\uuu k\neq 0$.
The reader can check that
$c_0=\sum\, d_k$�and  $n\geq 1$ gives:

\[ 
c_n=\sum_{k=1}^{k=p}\, d\uuu k\cdot \alpha\uuu k^n\quad 
\]
\medskip



\noindent
{\bf{A.8.4 The reduced rank.}}
Assume that $\{c\uuu n\}$ has a finite rank
$N$.  To each $k\geq 0$ we denote by $r_k$
the dimension of the vector space generated by
$\xi\uuu k,\xi\uuu{k+1},\ldots$.
It is clear that $\{r\uuu k\}$ decrease as $k$ increases.
Hence there exists
a non-negative integer
$N_*$ such that $r\uuu k=N_*$ for large $k$. One refers to 
$N_*$ as the reduced rank. It is obvious that 
$N_*\leq N$. The relation between $N_*$ and $N$
is related to the representation
\[
 f(x)= \frac{q(x)}{g(x)}
\]
where $q$ and $g$ are polynomials without common factor.
We shall not pursue this study in detail  but refer to the literature.
See for example 
the exercises
in [Polya\vvv Szeg� : Chapter VII:problems 17\vvv 34].




\bigskip



\noindent
{\bf{A.8.5  Hankel's formula for Laurent series.}}
Let $p\geq 2$ and
consider a rational function of the form
\[
R(z)= \frac{z^{p-1}}{z^p\vvv [a_1z^{p-1}+\ldots
+a_{p-1}z+ a_p]}
\]
The rational function $R$ has a simple pole 
at inifinity and when $|z|$ is large it is  expressed via a
Laurent series
\[ 
R(z)= \frac{c\uuu 0}{z}+ 
\frac{c\uuu 1}{z^2}+\ldots
\]
Consider the $p\times p$\vvv matrix
\[
A=\begin{pmatrix}
0&0&&\ldots&0&a\uuu p\\
1&0&0&\ldots&0&a\uuu{p\vvv 1}\\
0&1&0&\ldots&\ldots&c\uuu{a\vvv 2}\\
\ldots&
\ldots&
\ldots&
\ldots&\ldots&\ldots
\\
0&0&0&\ldots&1&a\uuu 1\\
\end{pmatrix}
\]
\medskip

\noindent
{\bf{Exercfise.}}
Prove Hankel's equations
which assert that for every $n\geq 1$ one has:

\[
\mathcal D\uuu n^{(p)}=  \mathcal D^{(p)}\uuu 0\cdot
\bigl[\text{det}(\,A\bigr)\bigr ) ^n
\]

\medskip

\noindent

\noindent
{\bf{A.8.6 The Hadamard-Kronecker identity.}}
For all pairs
of positive integers $p$ and $n$ one has the equality:
\[
\mathcal D\uuu n^{(p+1)}\cdot
\mathcal D\uuu {n+2}^{(p-2)}=
\mathcal D\uuu n^{(p+1)}
\mathcal D\uuu {n+2}^{(p\vvv 1)}-
\bigl[\mathcal D\uuu {n+1}^{(p)}\,\bigr]^2\tag{*}
\]





\newpage




\centerline{\bf{A.9  The Gram\vvv Fredholm formula.}}
\medskip


\noindent
A result whose discrete version is due to Gram  was 
extended to integrals  by
Fredholm and goes as follows:
Let $\phi\uuu 1,\ldots,\phi\uuu p$
and $\psi\uuu 1,\ldots,\psi\uuu p$ be two
$p$\vvv tuples of continuous functions on
the unit interval.
We get the $p\times p$\vvv matrix with elements
\[
a\uuu {\nu k}= \int\uuu 0^1\, \phi\uuu \nu(x)\psi\uuu k(x)\cdot dx
\]
At the same time
we define the following  functions on $[0,1]^p$:

\[ 
\Phi(x\uuu 1,\ldots,x\uuu p)
=\text{det}
\begin{pmatrix}
\phi\uuu 1(x\uuu 1)&\cdots&\phi\uuu 1(x\uuu p)\\
\cdots &\cdots&\cdots \\
\cdots &\cdots&\cdots\\
\phi\uuu p(x\uuu 1)&\cdots &\phi\uuu 1(x\uuu p)\\
\end{pmatrix}\quad\colon\quad
\Psi(x\uuu 1,\ldots,x\uuu p)
=\text{det}
\begin{pmatrix}
\psi\uuu 1(x\uuu 1)&\cdots&\psi\uuu 1(x\uuu p)\\
\cdots &\cdots&\cdots \\
\cdots &\cdots&\cdots\\
\psi\uuu p(x\uuu 1)&\cdots &\psi\uuu 1(x\uuu p)\\
\end{pmatrix}
\]

\medskip

\noindent
Product rules for determinants give the Gram\vvv Fredholm  equation
\[
\text{det}(a\uuu{\nu k})=
\frac{1}{p\,!}\int\uuu{[0,1]^p}\, 
\Phi(x\uuu 1,\ldots,x\uuu p)\cdot
\Psi(x\uuu 1,\ldots,x\uuu p)\cdot
dx\uuu 1\ldots dx\uuu p\tag{A.9.1}
\]
\medskip

\noindent
{\bf{Exercise.}} Prove (A.9.1)
or consult the literature.
See for example  the 
text-book [Bocher]
which contains a detailed account about Fredholm
determinants and
their role for solutions to integral equations.





 
\newpage

\centerline
{\bf{A.9.2  Resolvents of integral operators.}}
\medskip

\noindent
Fredholm studied integral equations of the form
\[
\phi(x)\vvv \lambda\cdot 
\int\uuu\Omega\,
K(x,y)\cdot \phi(y)\cdot dy=
f(x)\tag{*}
\]
where $\Omega$ is a bounded domain in some euclidian space
and the kernel function $K$ is complex\vvv valued. In general 
$K(x,y)\neq K(y,x)$.
Various regularity conditions can be imposed upon the kernel. The simplest is when
$K(x,y)$ is a continuous function in
$\Omega\times\Omega$.
The situation becomes more involved when singularities occur, for example when
$K$ is $+\infty$ on the diagonal, i.e. $|K(x,x)|=+\infty$.
This occurs for example when
$K$�is derived from Green's functions
which yield fundamental solutions to
elliptic PDE\vvv equations
where corresponding boundary value problems are solved
via integral equations. 
To obtain square integrable solutions in (*) for such singular 
kernels,
functions, 
the  determinants used by Fredholm were modified by
Hilbert which avoid the singularities
and are used to  obtain 
resolvents of the integral operator $\mathcal K$ defined by
\[
\mathcal K(\phi)(x)=\int\uuu\Omega\,
K(x,y)\cdot \phi(y)\cdot dy
\]
where $\phi$ belong to the Hilbert space
$L^2(\Omega)$.
Hilbert studied   the case when $K$ is square integrable,  i.e.
when
\[
\iint\uuu{\Omega\times\Omega}\, |K(x,y)|^2\, dxdy<\infty\tag{**}
\]
An eigenvalue to $\mathcal K$ is a complex number
$\lambda\neq 0$ for which there exists a non-zero $L^2$-function $\phi$ such that
\[
\mathcal K(\phi)= \lambda\cdot \phi
\]
It is not difficult to show that (**) entails  that 
the set of eigenvalues form a discrete set $\{\lambda\uuu n\}$
with a sole cluster point at the origin in the complex
$\lambda$-plane.
In a famous article from 1909, Schur proved  that (**) gives 
the inequality
\[
\sum\, \frac{1}{|\lambda\uuu n|^2}\leq
\iint\uuu{\Omega\times\Omega}\, |K(x,y)|^2\, dxdy\tag{***}
\]
It means that the non-increasing sequence
$\{|\lambda_1\geq |\lambda_2|\geq\ldots\}$
miust tend to zero rather rapidly, where these
eigenvalues are repeated when the eigenspaces have dimension
$\geq 2$.

\bigskip

\noindent
{\bf{A.9.3 Hilbert's determinants.}}
Let $K$ be a kernel function whose   integral in (**) is finite.
A typical case is that
$K$ is singular on the diagonal subset
of $\Omega\times\Omega$ while it is contonous outside the diagonal.
To each positive integer $m$ one associates a pair of matrices of size
$(m+1)\times m(+1)$ whose elements depend upon a pair 
$(\xi,\eta)
\in\Omega\times\Omega$ and an $m$\vvv tuple of distinct points
$x\uuu 1,\ldots,x\uuu m$ in $\Omega$: 

\[
C^*\uuu m=
\begin{pmatrix}
0&K(\xi,x\uuu 1)&K(\xi,x\uuu 2)&\ldots&\ldots &K(\xi, x\uuu m)\\
K(x\uuu 1,\eta)&0&K(x\uuu 1,x\uuu 2)&\ldots&\ldots &K(x\uuu 1,x\uuu m)\\
K(x\uuu 2,\eta)&K(x\uuu 2,x\uuu 1)&0
&\ldots&\ldots&0\\
\ldots&
\ldots&
\ldots&
\ldots&\ldots&\ldots
\\
\ldots&
\ldots&
\ldots&
\ldots&\ldots&\ldots
\\
K(x\uuu m,\eta)&K(x\uuu m,x\uuu 1)&K(x\uuu m,x\uuu 2)&\ldots &\ldots&0\\
\end{pmatrix}
\]

\bigskip


\[
C\uuu m=\begin{pmatrix}
0&K(x\uuu 1,x\uuu 2)&&\ldots&0&K(x\uuu 1, x\uuu m)\\
K(x\uuu 2,x\uuu 3)&0&K(x\uuu 2,x\uuu 3)
&\ldots&&K(x\uuu 2,x\uuu m)\\
\ldots&\ldots &\ldots&\ldots&\ldots&\ldots\\
\ldots&
\ldots&
\ldots&
\ldots&\ldots&\ldots
\\
K(x\uuu m,x\uuu 1)&K(x\uuu m,x\uuu 2)
&K(x\uuu m,x\uuu 3) &\ldots&K(x\uuu m,x\uuu{m\vvv 1})&0\\
\end{pmatrix}
\]

\newpage

\noindent
Put:
\[
D^*\uuu m(\xi,\eta)=
\int\uuu{\Omega^m}\, 
C^*\uuu m(\xi,\eta: x\uuu 1,\ldots,x\uuu m)\cdot dx\uuu 1\cdots dx\uuu m\tag{i}
\]
\[
D\uuu m=
\int\uuu{\Omega^m}\, 
C\uuu m(x\uuu 1,\ldots,x\uuu m)\cdot dx\uuu 1\cdots dx\uuu m\tag{ii}
\]
Thus, we take the integral over the $m$\vvv fold product of
$\Omega$.
Next, let $\lambda$ be a new complex parameter and set
\[ 
\mathcal D^*(\xi,\eta,\lambda)=
\sum\uuu{m=1}^\infty\, \frac{(\vvv\lambda)^m}{m!}
\cdot D^{**}\uuu m(\xi,\eta)
\]

\[ 
\mathcal D(\lambda)=1+
\sum\uuu{m=1}^\infty\, \frac{(\vvv\lambda)^m}{m!}
\cdot D_m
\]

\noindent


\medskip

\centerline{\bf{Some results by Carleman}}
\medskip


\noindent
Using  Fredholm-Hilbert determinants 
some
conclusive facts about integral operators
were established
by Carleman in the article \emph{Zur Theorie der Integralgleichungen}
from 1921  when the kernel
$K$ is of the Hilbert-Schmidt type, i.e. (**) holds.
Outside a discrete set
in the complex $\lambda$-plane we have the inverse  operator
\[
R(\lambda)= (\lambda\cdot E-\mathcal K)^{-1}
\]
\medskip

\noindent
{\bf{A.9.4 Theorem.}} \emph{The kernel function of $R(\lambda)$
is for each complex $\lambda$ outside the spectrum 
given by}
\[
\Gamma(\xi,\eta;\lambda)=
K(\xi,\eta)+\frac{\mathcal D^*(\xi,\eta,\lambda)}{
\mathcal D(\lambda)}
\]


\noindent
{\bf{Remark.}}
Let  $\{\lambda\uuu\nu\}$ be the discrete spectrum of $\mathcal K$
where multiple
eigenvalues are repeated when the corresponding 
eigenspaces have dimension $\geq 2$. 
This spectrum constitutes the zeros of 
$\mathcal D(\lambda)$
and out turns out that thid is an entire function, i.e. analytic 
in the whole complex $\lambda$-plane.
When
$\lambda$ is outside the spectrum 
the inverse operator $(\lambda\cdot E-\mathcal K)^{-1}$
is an integral operator
defined by
\[
\phi \mapsto \int_\Omega\, \Gamma(\xi,\eta;\lambda)\cdot \phi(\eta)\, d\eta
\]
where $\xi$ and $\eta$ are  variable points in
$\Omega$.
Using
inequalities of Fredholm\vvv Hadamard type for determinants, 
a first result from Carelan's cited article asserts that 
\[
\int\uuu{\Omega}\, \Gamma (\xi,\xi; \lambda)\cdot d\xi=
\vvv \lambda\cdot \sum\uuu{\nu=1}^\infty\,
\frac{1}{\lambda\uuu\nu(\lambda\vvv\lambda\uuu\nu)}\tag{A.9.5}
\]
where Schur's inequality from (***) in A.9.2 entails that the right hand side indeed
represents a meromorphic function.
A  major result in Carelan'as article is about 
the entire function $\mathcal D(\lambda)$.

\medskip

\noindent
{\bf{A.9.5 Theorem.}}
\emph{$\mathcal D(\lambda)$ is an entire function of the complex parameter
$\lambda$ given by a Hadamard product}
\[
\mathcal D(\lambda)=
\prod\,(1-\frac{\lambda}{\lambda_n})\cdot e^{\frac{\lambda}{\lambda_n}}\tag{1}
\]
\emph{where $\{\lambda_n\}$ satisfy}
\[
\sum\, |\lambda_n|^{-2}<\infty\tag{2}
\]
\medskip


\noindent
{\bf{Remark.}}
Prior to this Schur  proved a
representation as in (1) 
adding
a factor $e^{b\lambda^2}$ in the right hand side.
So the novelty
in Carleman's
work is that
$b=0$ always hold.
A
crucial
step in Carleman's  proof of (1) 
was to use  an  inequality for
determinants which goes as follows:
Let $q>p\geq 1$ be a pair of integers
and
$\{a\uuu{k,\nu}\}$ a doubly\vvv indexed sequence of complex numbers
which appear as elements in a $p+q$\vvv matrix of the form:






\[
\begin{pmatrix}
0&\ldots&0&a\uuu{1,p+1}&\ldots &a\uuu{1,p+q}\\
0&\ldots&0&a\uuu{2,p+1}&\ldots &a\uuu{1,p+q}\\
\ldots&
\ldots&\ldots&\ldots&\ldots&\ldots \\
0&\ldots&0&a\uuu{p,p+1}&\ldots &a\uuu{p,p+q}\\
a\uuu{p+1,1}&\ldots &a\uuu{p+1,p}&a\uuu{p+1,p+1}&\ldots&
a\uuu{p+1,p+q}\\
\ldots&
\ldots&\ldots&\ldots&\ldots&\ldots
\\
a\uuu{p+q,1}&\ldots &a\uuu{p+q,p}&a\uuu{p+q,p+1}&\ldots
&a\uuu{p+q,p+q}
\\
\end{pmatrix}\tag{*}
\]
\medskip



\noindent
For each pair $1\leq m\leq p$ we put
\[
L\uuu m=\sum\uuu{\nu=1}^{\nu=q}\, |a\uuu{m,p+\nu}|^2
\quad\colon\quad 
S\uuu m=\sum\uuu{\nu=1}^{\nu=q}\, |a\uuu{p+\nu,m}|^2
\quad\colon\quad 
N=\sum\uuu{j=1}^{j=q} \sum\uuu{\nu=1}^{\nu=q}\, |a\uuu{p+j,p+\nu}|^2
\]





\noindent
{\bf{A.9.6 Theorem.}}\emph{
Let $\Delta$ be the determinant of the matrix  (*). Then}
\[
|\Delta|\leq (L\uuu 1\cdots L\uuu p) ^ {\frac{1}{p}}\cdot
\sqrt{M\uuu 1\cdots M\uuu p}\cdot
\frac{N^{\frac{q\vvv p}{2}}}{(q\vvv p)^{\frac{q\vvv p}{2}}}
\]


\noindent
\emph{Proof.}
After unitary transformations of the last $q$ rows and 
the last $q$ columns respectively, the proof is reduced to the case
when $a\uuu{jk}=0$ for pairs $(j,k)$ with  $j\leq p$ and $k>p+j$ or with
$k\leq p$ and $j>p+k$.
Here $L\uuu m, S\uuu m$ and $N$  are unchanged and we get
\[ 
\Delta=(\vvv 1)^p\cdot \prod\uuu{j=1}^{j=p}\,
a\uuu{j,p+j}\cdot \prod\uuu{k=1}^{k=p}\,a\uuu {p+k,k}
\cdot 
\det \begin{pmatrix}
a\uuu{p+1,2p+1}&\ldots &a\uuu{p+1,p+q}\\
\ldots&
\ldots&\ldots
\\
a\uuu{p+q,p+1}&\ldots &a\uuu{p+q,p+q}\\
\end{pmatrix}
\]
\medskip


\noindent
The absolute value of the last determinant is
majorized by   Hadamard's inequality in � F.XX   and 
the requested inequality in Theorem A.9.6 follows.




\newpage





\centerline{\bf{� 1. Wedderburn's theorem.}}

\bigskip


\noindent
A finite dimensional ${\bf{C}}$-algebra $\mathcal A$
is  an associative  ring
which contain
${\bf{C}}$ as a central subfield,  i.e.
$\lambda\cdot a=a\cdot \lambda$ for pairs $a\in\mathcal A$
and complex numbers $\lambda$.
The ring product gives  the family of
left ideals. They consist of complex subspaces $L$
which are stable under left multiplication, i.e.
$aL\subset L$ hold for every element $a$ in 
$\mathcal  A$.
One may also regard two-sided ideals $J$  
where one requires that both $aJ$ and $Ja$ are contained in $J$
for every $a\in\mathcal A$.
One says that  $\mathcal A$ is called a simple algebra if the sole 2-sided ideals are 
$\mathcal A$ and the trivial zero ideal.
Examples of  finite dimensional ${\bf{C}}$-algebras
are  matrix-algebras $\{M_n({\bf{C}}\,\colon\, n\geq 1$.
It turns out that they give the sole simple algebras.
\bigskip



\noindent
{\bf{1.1 Theorem.}}
\emph{Let $A$ be a finite dimensional and simple 
${\bf{C}}$-algebra.
Then there exists an integer $n$ such that}
\[
A\simeq M_n({\bf{C}})
\]
\medskip


\noindent
The proof requires several steps.
Let us first show that
the matrix algebras are simple.
With $n\geq 2$ we  put 
$\mathcal A=M_n({\bf{C}})$ and identify every  matrix
with a linear operator
on ${\bf{C}}^n$.
Conioder the special matrices
$e_1,\ldots,e_n$ where the elements of $e_p$ are zero except for the diagonal
element 
$a_{pp}=1$.
Product rukes for matrices show that
\[
e_pe_q=0\quad\colon\, p\neq q
\]
At thes same time we notice that
the identity element is $e_1+\ldots+e_n$ and that
$e_p=e_p^2$. This means  that
$\{e_p\}$ are pairwise  orthogonal idempotent elements.
Wirh $p$ fixed we have the left ideal 
\[
L_p=Ae_p
\]
The reader can  check that it consists of all matrices whose columns of degree $q\neq p$
are zero.
The left ideal $L_p$ is minimal. For let
$\xi$ be a non-zero matrix in $L_p$ which means that there exists
at least some $\nu$ where the matrix element $a_{\nu p}\neq 0$.
mutlplying with a sclar we can assume that
$a_{\nu p}=1$ and then we see that 
\[
e_\nu\cdot \xi=e_p
\]
Hence the principal left ideal generated by $\xi$ is equal to $L_p$, i.e.
every non-zero element in $L_p$ generates $L_p$ which means  that
this left ideal is minimal.
Next, let $\xi= \{a_{qp}\}$
be a non-zero matrix.
and choose $p$ so that $a_{qp}\neq 0$ for at least one $q$.
Then $\xi\cdot e_p$ is a non-zero element in $L_p$ so the
2-sided ideal generated by $\xi$ contains the minimal left ideal $L_p$.
If we consider another integer $q$
we take the matrix $\xi$ 
with a single non-zero element  placed at $(p,q)$
which is equal to one. Then we see that $e_p\cdot \xi=e_q$
and hence the 2-sided ideal contains $L_q=Le_q$, Since this hold for every
$1\leq q\leq n$
the reader mat conclude that the 2-sided ideal generated by $\xi$ is the whole ring
$A$. This proves that $A$ is simple.
\medskip

\noindent
Next, let $\xi$ be a non-zero matrix which commutes with all other matrices.
To prove that $\xi$ is a complex multiple of the identity matrix
one argues as follows:
The matrix elements of $\xi$ are $\{a_{\nu k}\}$. For a given $p$
the product $\xi\cdot e_p $ gives a matrix with a single
non-zero column put in place $p$ with elements $\{a_{\nu p}\}$.
At the same time $e_p\cdot \xi$ is a matrix with a single non-zero row
placed in degree $p$. So the equality $e_p\xi=\xi e_p$ entails that
\[
a_{qp}=0\quad\colon q\neq p
\]
So when  $\xi$ commnutes with
all the $e$-matrices we concude that
$\xi$ is a diagonal matrix,  i.e. the elements outside
the diagonal are all zero.
There remain to see that the diagonal elements  are all equal.
Suppose for example that $a_{11}\neq a_{22}$.
Now there exists the matrix $\beta $ where $b_{12}=b_{21}=1$
and all other elements are zero.
Then we see that
\[
\beta\cdot |xi= a_{11}e_2+a_{22}e_1\quad\colon\quad
\xi\cdot \beta= a_{11}\cdot e_1+a_{22}\cdot e_2
\]
Hence the equality $\xi\beta=\beta\xi $ entails that
$a_{11}=a_{22}$. In  the same way one proves that
all diagoal elements are equal. This proves that  the center of the matrix algebra is reduced to
complex multiples of the identity.

\medskip








\noindent
{\bf{1.1. Exercise.}} Set 
$\mathcal A=M_n({\bf{C}})$ and identify every  matrix
with a ${\bf{C}}$-linear operator on
${\bf{C}}^n$.
To each left ideal $L$ we assign the null space
\[
L^\perp=\{ v\in {\bf{C}}^n\,\colon\, L(v)=0\} 
\]
Thus one takes the intersection of the null spaces of operators from $L$.
Show that $L^\perp$ determines $L$ in the sense
that a matrix $Q$ belongs to $L$ if and only if its null-space contains
$L^\perp$. 
So by 
\[
L\mapsto L^\perp
\]
one has  a bijective map
between
the family of left ideals in the matrix algebra and
subspaces of ${\bf{C}}^n$.
\medskip


\noindent
Next, let 
$p$ is the dimension  of the vector space
$L^\perp$.
Show that 
dimension of $L$ regared as a complex vector space is equal to
$n(n-p)$. 





\medskip


\noindent
For each $1\leq p\leq n-1$ we get the family
$\mathcal L_p$ of left ideals $L$ for which
$L^\perp$ has dimension $n-p$.
The family of $(n-p)$-dimensional subspaces of
${\bf{C}}^n$ is denoted by $\mathcal G(n-p;n)$ and called the Grassmannian
of degree $n-p$.
From the above one has the set-theoretic equality
\[
\mathcal G(n-p;n)=\mathcal L_p\tag{1.1.2}
\]
\medskip

\noindent
Next, show that for every left ideal $L$ there exists
an idempotent matrix $\Pi$ such that
\[
L= \mathcal A\cdot \Pi\tag{i}
\]
Thus, every left ideal is generated by a single matrix
and (i) means that
\[
L^\perp=\text{Ker}(\Pi)
\]
When $L\in \mathcal L_p$
the $\Pi$-kernel has dimension $n-p$ and this implies that the range
of $\pi$ is $p$-dimensional. Moreover, since
$\Pi^2=\Pi$ one has a direct sum decomposition
\[
{\bf{C}}^n= L^\perp\oplus \Pi({\bf{C}}^n)\tag{ii}
\]








\newpage







\medskip

\noindent
\centerline {\bf{Proof of 1.1 Theorem.}}
\medskip

\noindent
Denote by $\mathcal L_*$ the family of non-zero
left ideals $L$ in $A$ which are minimal in the sense that
every
non-zero left ideal $L_1\subset L$ is equal to $L$.
Since $A$ is a finite dimensional vector space  it is clear that
there exists at least one minimal left ideal $L$.
Identifying $L$ with a complex vector space it has  some dimension 
$k$, and we  get the ${\bf{C}}$-algebra
\[
\mathcal M=\text{Hom}_{{\bf{C}}}(L,L)
\]
Choosing a basis in the complex vector space $L$
one has 
\[
\mathcal M\simeq M_k({\bf{C}})
\]
Wedderburn's theorem follows if we prove that
\[
\mathcal M\simeq A\tag{*}
\]
To get (*)  we first consider some $a\in A$ which by left multiplication
gives a map
\[
a^*\colon x\mapsto ax\quad\colon\quad x\in L
\]
Since ${\bf{C}}$ by assumption is a central subfield  of
$A$ these maps are complex linear and hence
$a^*$ is an element in $\mathcal M$.
If $b$ is another element in $A$
we get the composed linear operator
$b^*\circ a^*$ defined by
\[
x\mapsto bax=(ba)^*(x)
\]
Hence 
\[
a\mapsto a^*\tag{i}
\]
is an algebra homomorphism from
$A$ into $\mathcal M$.
We claim  that this map is injective.
For if $a^*$ is the zero map we use that $L$ is a left ideal which gives
\[
ax\xi=0
\]
for all $x\in A$ and $\xi\in L$.
This gives $a^*\circ x^*=0$
and since it is obvious that $x^*\circ a^*=0$ also holds, we conclude that
the kernel of  (i) is a 2-sided ideal in $A$. 
Since $A$ is simple 
this kernel is zero which proves that
(i) is injective.
\medskip

\noindent
\emph{Proof of surjectivity.}
If $x\in A$ is such that
$Lx\neq 0$ then this is a non-zero left ideal and since
$L$ is minimal the reader can check that we also have $Lx\in \mathcal L_*$
and the simple left $A$-modues $L$ and $Lx$ are isomorphic, i.e.
\[
L\simeq Lx\tag{ii}
\]
hold whenever $Lx\neq 0$.
Next, by  assumption  the 2-sided ideal generated by  $L$
is the whole ring
$A$. Hence there exists a finite set of $A$-elements $\{x_\nu\}$ such that
\[
A=Lx_1+\ldots+Lx_m\tag{iii}
\]
Above we can choose $m$ minimal which 
gives 
a direct sum
\[
A=Lx_1\oplus\ldots\oplus Lx_m\tag{iv}
\]
For suppose that 
\[
\xi_1x_1+\ldots+\xi_mx_m=0\quad\colon\, \xi_\nu\in L
\]
where $\xi_kx_k\neq 0$ for some $k$.
Since $Lx_k$ is minimal the resder can chek that
$Lx_k$ now can be deleted in (iii) which contradicts the minmal chose of $m$.
Hence one has the direct sum in (iv). By (ii)
the vector spaces
$L$ and $Lx_k$ are isomorphic
for every $1\leq k\leq m$.
Counting dimensions we conclude that
\[
\dim_{{\bf{C}}}A= m\cdot k\tag{v}
\]
Since the map from $A$ into $\mathcal M$  was injective
and $\mathcal M$ is a matrix algebra of  dimension $k^2$, thr injectivity
entails that  
$m\leq k$, and there
remains  to prove the opposite inequality
\[
k\leq m\tag{*}
\]
To get (*)  we take the identity element $1$ in $A$ and (iv) gives
an $m$-tuple $\{\xi_\nu\}$ in $L$ so that
\[
1=\xi_1x_1+\dots+\xi_kx_m\tag{1}
\]
Put $e_\nu=\xi_\nu\cdot x_\nu$.
Multiplying to the left by some $e_k$ in (1) we get
\[
e_k= e_ke_1+\ldots+e_ke_m
\]
The direct sum in (iv) entials that
\[
e_ke_\nu=0\,\colon \nu\neq k\quad \&\quad e_ke_k=e_k
\]
Hence $\{e_\nu\}$
are pairwise  orthogonal idempotent elements  in $A$.
For a fixed $k$ the equality $e_k=e_k^2$ entails  that
$e_k\cdot A\cdot e_k$ is a subalgebra of $A$.
If $x=e_k\cdot x\cdot e_k$ is an element in this subalgebra
then right multiplication
by $x$ on the left ideal $Ae_k$ is left $A$-linear, i.e.
one has a map
\[
e_k\cdot A\cdot e_k\to \text{Hom}_A(Ae_k,Ae_k)\tag{2}
\]
Now we use that $Ae_k$ is a minimal left ideal, i.e. as a left $A$-module it is simple.
This implies that
the right hand side in (2) is a division ring, i.e. every non-zero element is invertible,
Since the complex field is algebraically closed
this division ring is equal to ${\bf{C}}$. 
Moreover, if $\xi=e_kxe_k$ is such that
its image in (2) is zero, then
\[
e_k\xi= e_k^2xe_k=e_kxe_k=\xi=0
\]
So (2) is injective and hence
\[
e_kAe_k={\bf{C}}\tag{3}
\]
\medskip


\noindent
Let us now take some $j\neq k$ and consider the
space
\[
\text{Hom}_A(Ae_j,Ae_k)\tag{4}
\]
Every left $A$-linear map from
$Ae_j$ into $Ae_k$ is induced by
right multiication with an element $\xi$ and since
$e_j$ and $e_k$ are idempotens one has
\[
\xi=e_j\xi e_k
\]
Conversly,  every $x\in A$ gives
gets a $\xi$-element $a_kxe_j$. Hence the vector space (4)  can be identfied with
the subset of $A$ given by
\[
e_jAe_k\tag{5}
\]
We have already seen that the left $A$-modukes generated by $e_k$ and $e_j$ are isomorphic
and then (3) entails that
\[
\dim_{\bf{C}}(e_jAe_k)=1\tag{6}
\]
Now (*) follows because with $L=Ae_1$
one has
\[
L=\sum_{j=1}^{j=m}\, e_jAe_1
\]
which proves thst the $k$-dimensional vector space $L$ has dimension
$m$ at most which gives the inequality (*) and finishes the proof
of Wedderburn's theorem.
\bigskip


\noindent
{\bf{Remark.}} For readers interested in ring theory we
recall that Wedderburn's theorem
is more general, i.e. it describes every
simple associative ring $A$ which in addition is
\emph{artinian}, i.e. the descendinig chain condition holds for the family of left ideals.
Every  simple and artinian ring is isomorphic to
a matrix ring
$M_k(D)$ where $D$ in general is a division ring.
The proof is verbatim the same as above where
the sole difference is that
we now get the division ring $D$ via (3) in the previous proof.
A more extensive family of rings
is found as follows: An associative ring $B$ with a unit elementis called
\emph{semi-primsry}
if it contains a tweo-sided ifeal $J(B)$ which is nilpotent
while
the quotent ring
\[
\frac{B}{J(B)}
\]
is a semi-simle artinian ring which via Wdderburin's theorem means thst
this ring contains a finite family of central and orthogonal idempotents
$e_1,\ldots,e_k$ such that the identity 
$1=e_1+\ldots+e_k$ and the subrings
$\{e_\nu\cdot \frac{B}{J(B)}\cdot e_\nu\}$
are simple artinian rings.
One reason why this extended family of rings is interesting is that
they srise via simle artinian rings
while one regards invariant subrings.
A general result of this kind which is  due to the present author  goes as follows:

\medskip

\noindent
{\bf{Theorem.}}
\emph{Let $A$ be a simple artinian ring and
$\mathcal E$ a family of left $A$-linear maps on
$A$ into itself. Then the invariant subring below is semi-priumary.}�
\[
A_\mathcal E=\{ x\in A\,\colon E(x)=x\,\colon\, E\in \mathcal E\}
\]











\newpage


\centerline{\bf{2. Resolvents}}
\bigskip

\noindent
Let $A$ be some matrix in $M_n({\bf{C}})$.
Its characteristic polynomial is defined by
\[ 
P_A(\lambda)=
\text{det}(\lambda\cdot E_n-A)\tag{*}
\]
By the fundamental theorem of
algebra $P\uuu A$
has $n$ roots
$\alpha_1,\ldots,\alpha_n$ where eventual
multiple roots are repeated.
The 
union of  distinct roots is denoted by
$\sigma(A)$ and called the spectrum of $A$.
Since matrices with non-zero determinants are invertible
we obtain a matrix valued function defined in
${\bf{C}}\setminus\sigma(A)$ by:
\bigskip
\[ 
R_A(\lambda)=
(\lambda\cdot E_n-A)^{-1}\quad\colon\quad
\lambda\in{\bf{C}}\setminus\sigma(A) \tag{**}
\]


\noindent
One refers to $R_A(\lambda)$ as the resolvent of $A$.
The map
\[ 
\lambda\mapsto R_A(\lambda)
\] 
yields a matrix-valued analytic function defined in
${\bf{C}}\setminus \sigma(A)$.
To see this we take some
$\lambda_*\in {\bf{C}}\setminus \sigma(A)$ and set
\[
R_*=(\lambda_*\cdot E_n-A)^{-1}
\]
Since
$R_*$ is a 2-sided inverse we have
the equality
\[
E_n=R_*(\lambda_*\cdot E_n-A)=
(\lambda_*\cdot E_n-A)\cdot R_*\implies
R_*A=AR_*
\] 
Hence the resolvent $R_*$ commutes with $A$.
Next,
construct the matrix-valued power series
\[
\sum_{\nu=1}^\infty (-1)^\nu\cdot \zeta^\nu\cdot (R_*A)^\nu\tag{1}
\]
which is convergent when $|\zeta|$ are small enough.
\medskip


\noindent
{\bf{2.1 Exercise.}}
Prove  the equality
\[
R_A(\lambda_*+\zeta)=R_*+\sum_{\nu=1}^\infty
(-1)^\nu\cdot \zeta^\nu\cdot R_*\cdot (R_*A)^\nu
\]
The local series expansion () above therefore  shows that
the resolvents yield a matrix-valued analytic function
in ${\bf{C}}\setminus\sigma(A)$.

\medskip

\medskip

\noindent
We are going to use 
analytic function theory 
to establish results which after can be extended
to
an operational calculus for  linear operators on infinite dimensional 
vector spaces.
The  analytic constructions are     also useful to investigate
dependence  upon parameters. Here is 
an example.
Let
$A$ be an $n\times n$-matrix whose
characteristic polynomial $P_A(\lambda)$ has $n$ simple roots
$\alpha_1,\ldots,\alpha_n$. When
$\lambda$ is outside the spectrum $\sigma(A)$.
residue calculus  gives
the following   expression for
the resolvents:
\[
(\lambda\cdot E_n-A)^{-1}=
\sum_{k=1}^{k=n}
 \frac{1}{\lambda-\alpha_k}\cdot \mathcal C_k(A)\tag{*}
 \]
where each matrix $\mathcal C_k(A)$ is a polynomial in $A$ given by:
\[
 \mathcal C_k(A)=
 \frac{1}{\prod_{\nu\neq k}\, (\alpha_k-\alpha_\nu)}
\cdot\prod_{\nu\neq k}\,(A-\alpha_\nu E_n)
\]
The  formula (*)   goes back to work
by
Sylvester, Hamilton and Cayley.
The resolvent $R_A(\lambda)$
is also used to construct  the Cayley-Hamilton polynomial of $A$
which 
by definition this is the unique monic polynomial $P\uuu *(\lambda]$ in the 
polynomial ring ${\bf{C}}[\lambda]$
of smallest possible degree such that the associated
matrix
$p_*(A)=0$.
It is found as follows:
Let $\alpha_1,\ldots,\alpha_k$
be the distinct roots of $P_A(\lambda)$
so that
\[
P_A(\lambda)=
\prod_{\nu=1}^{\nu=k}\, 
(\lambda-\alpha_\nu)^{e_\nu}
\]
where $e_1+\ldots+e_k=n$.
Now the meromorphic and matrix-valued resolvent
$R_A(\lambda)$
has  poles at $\alpha_1,\ldots,\alpha_k$. If
the order of a pole at root $\alpha_j$ is denoted by
$\rho_j$ one has the inequality
$\rho_j\leq e(\alpha_j)$
which in general can be strict. The Cayley\vvv Hamilton polynomial
becomes:
\[
P_*(\lambda)=\prod_{\nu=1}^{\nu=k}\, 
(\lambda-\alpha_\nu)^{\rho_\nu}\tag{**}
\]
\medskip




\noindent
Now we begin to prove results in more detail.
To begin with one has the Neumann series expansion:

\medskip

\noindent
{\bf{Exercise.}}
Show that if $|\lambda|$ is 
strictly larger than the absolute values of the roots
of $P_A(\lambda)$, then the resolvent is given by the  series
\[ 
R_A(\lambda)=\frac{E_n}{\lambda}+ \sum_{\nu=1}^\infty
\,\lambda^{-\nu-1}\cdot A^\nu\tag{*}
\]

\noindent
{\bf{A differential equation.}}
Taking the complex derivative of $\lambda\cdot R_A(\lambda)$ 
in (*) we get
\[ \frac{d}{d\lambda}(\lambda R_A(\lambda))
=-\sum_{\nu=1}^\infty\,\nu\cdot\lambda^{-\nu-1}\cdot A^\nu\tag{1}
\]


\noindent
{\bf{Exercise.}}
Use (1) to prove that
if $|\lambda|$ is large then
$R_A(\lambda)$ satisfies the differential equation:
\[ 
\frac{d}{d\lambda}(\lambda R_A(\lambda))
+A[\lambda^2R_A(\lambda)-E_n-\lambda A]=0\tag{2}
\]


\noindent
Now
(2) and  the analyticity of the resolvent 
outside the spectrum of $A$ give:


\medskip

\noindent
{\bf 2.3 Theorem} \emph{Outside the spectrum
$\sigma(A)$
$R(\lambda)$
satisfies the differential equation}
\[
\lambda\cdot R_A'(\lambda)+R_A(\lambda)+\lambda^2\cdot
A\cdot R_A(\lambda)=
A+\lambda\cdot A^2
\]

\bigskip

\noindent
{\bf{2.4 Residue formulas.}}
Since the resolvent is analytic we can construct complex line integrals and
apply results in complex residue calculus.
Start from the Neumann series (*) above 
and  perform integrals over circles 
$|\lambda|=w$ where $w$ is large.
\medskip

\noindent
{\bf{2.5 Exercise.}}
Show that when $w$ is strictly 
larger than
the absolute value of every
root of
$P_A(\lambda)$ then
\[ A^k=\frac{1}{2\pi i}\int_{|\lambda|=w}\,
\lambda^k\cdot R_A(\lambda)\cdot d\lambda
\quad\colon\quad k=1,2,\ldots
\]
It follows that when $Q(\lambda)$ is an arbitrary polynomial then
\[ Q(A)=\frac{1}{2\pi i}\int_{|\lambda|=w}\,
Q(\lambda)\cdot R_A(\lambda)\cdot d\lambda\tag{*}
\]
In particular we take the identity $Q(\lambda)=1$ and obtain

\[ E_n=\frac{1}{2\pi i}\cdot\int_{|\lambda|=w}\,
R_A(\lambda)\cdot d\lambda\tag{**}
\]
Finally, show  that if $Q(\lambda)$ is a  polynomial
which has a zero of order
$\geq e(\alpha_\nu)$ at every root then
\[ 
Q(A)=0\tag{***}
\]

\bigskip

\noindent
{\bf{2.6 Residue matrices.}}
Let $\alpha_1,\ldots,\alpha_k$ be the distinct zeros
of $P_A(\lambda)$.
For a given root, say $\alpha_ 1$ of  multiplicity
$p\geq 1$ we have a local Laurent series expansion
\[
R_A(\alpha_1+\zeta)=
\frac{G_p}{\zeta^p}+\ldots+\frac{G_1}{\zeta}+
B_0+\zeta\cdot B_1+\ldots\tag{i}
\]
We refer to $G_1,\ldots,G_p$ as the residue matrices at $\alpha_1$.
Choose a polynomial $Q(\lambda)$ in ${\bf{C}}[\lambda]$
which vanishes up to the multiplicity at all the
the remaining roots
$\alpha_2,\ldots,\alpha_k$ while it has a zero of order $p-1$ at $\alpha_1$, i.e.
locally
\[
Q(\alpha_1+\zeta)= \zeta^{p-1}(1+q_1\zeta+\ldots)\tag{i}
\]

\noindent
{\bf{2.7 Exercise.}}
Use residue calculus and  (*) from Exercise 2.5 to show that:
\[ 
Q(A)=
\frac{1}{2\pi}\int_{|\lambda-\alpha_1|=\epsilon}\,
Q(\lambda)\cdot R_A(\lambda)\cdot d\lambda=
G_p\tag{*}
\]
Hence the matrix $G_p$ is a polynomial of $A$. In a similar way one proves that
every $G$-matrix in the Laurent series (i) is a polynomial in $A$.
\medskip


\noindent
{\bf{2.7 Some idempotent matrices.}}
Consider  a zero $\alpha_j$ and choose a polynomial
$Q_j$ such that
$Q_j(\lambda)-1$ has a zero of order $e(\alpha_j)$ at
$\alpha_j$ while $Q_j$ has a zero of order
$e(\alpha_\nu)$ at the remaining roots.
Set
\[ 
E_A(\alpha_j)=\frac{1}{2\pi i}\int_{|\lambda|=w}\,
Q_j(\lambda)\cdot R_A(\lambda)\cdot d\lambda\tag{1}
\]
where $w$ is large as in 2.5.
Since 
the polynomial 
$S=Q_j-Q_j^2$ vanishes up to the multiplicities
at all the roots of $P_A(\lambda$ we have $S(A)=0$ from (***) in 2.5
which entails that

\[
E_A(\alpha_j)=E_A(\alpha_j)\cdot E_A(\alpha_j)\tag{*}
\]
In other words, we have constructed an idempotent matrix.



 














\bigskip



\noindent
{\bf{2.8 The Cayley-Hamilton decomposition.}}
Recall the equality


\[ E_n=\frac{1}{2\pi i}\cdot\int_{|\lambda|=w}\,
R_A(\lambda)\cdot d\lambda
\]
where the radius $w$ is so large that the disc $D_w$ contains the zeros of 
$P_A(\lambda)$.
The previous construction of the $E$-matrices at
the roots of $P_A(\lambda)$  entail that
\[
 E_n=E_A(\alpha_1)+\ldots+E_A(\alpha_k)
\]


\noindent
Identifying $A$ with a ${\bf{C}}$-linear operator on
${\bf{C}}^n$ we obtain a direct sum decomposition
\[ 
{\bf{C}}^n= V_1\oplus\ldots\oplus V_k\tag{*}
\] 
where each $V_\nu$ is an $A$-invariant subspace given by 
the image of $E_A(\alpha_\nu)$.
Here $A-\alpha_\nu$ restricts to a \emph{nilpotent} linear operator on
$V_\nu$ and the dimension of this vector space is equal to the multiplicity
of the root $\alpha_\nu$ of the characteristic polynomial.
One refers to (*) as the \emph{Cayley-Hamilton decomposition} of
${\bf{C}}^n$.
\bigskip

\noindent
{\bf{2.9 The vanishing of $P_A(A)$}}.
Consider the characteristic polynomial $P_A(\lambda)$. By definition it vanishes up to the order
of multiplicity at every point in $\sigma(A)$ and hence (***) in 2.5
gives
$P_A(A)=0$.
Let us write:
\[
 P_A(\lambda)= 
 \lambda^n+c_{n-1}\lambda^{n-1}+\ldots+ c_1\lambda+c_0
\]

\noindent
Notice that
$c_0=(-1)^n\cdot\text{det}(A)$.
So if the determinant of $A$ is $\neq 0$
we get
\[ 
A\cdot [A^{n-1}+c_{n-1}A^{n-2}+\ldots+ c_1\bigr]=
(-1)^{n-1}
\text{det}(A)
\cdot E_n
\]



\noindent
Hence 
the inverse $A^{-1}$ is  expressed as a polynomial in $A$.
Concerning the equation
\[ 
P_A(A)=0
\]
it is in general not the minimal 
equation for $A$, i.e. it can occur that $A$ satisfies an equation of degree $<n$.
More precisely , if $\alpha_\nu$ is a  root of some multiplicity
$k\geq 2$ there exists a  Jordan decomposition which gives an 
integer $k_*(\alpha_\nu)$ for the largest Jordan block
attached to the nilpotent operator $A-\alpha_\nu$ on $V_{\alpha_\nu}$.
The \emph{reduced} polynomial $P_*(\lambda)$ is  the product
where the factor $(\lambda-\alpha_\nu)^{k_\nu}$ is replaced by
$(\lambda-\alpha_\nu)^{k_*(\alpha_\nu)}$
for every  $\alpha_\nu$ where $k_\nu<k_*(\alpha_\nu)$ occurs. 
Then $P_*$ is the polynomial of smallest possible degree such that
$P_*(A)=0$.
One  refers to $P_*$ as the \emph{Hamilton polynomial} attached to $A$.
This   result relies upon Jordan's result in � 3.

\medskip





\noindent
{\bf{2.10 Similarity of matrices.}}
Recall
that
the determinant of  a matrix $A$ does not  change when it is
replaced by $SAS^{-1}$
where $S$ is an arbitrary invertible matrix.
This implies  that
the coefficients of the
characterstic polynomial  $P_A(\lambda)$
are intrinsically defined via the
associated linear operator, i.e. if another basis is chosen in
${\bf{C}}^n$ the given $A$-linear operator is expressed by a matrix
$SAS^{-1}$ where $S$ effects the change of the basis.
Let us now draw an interesting consequence
of the previous operational calculus.
Let us give the following:
\medskip


\noindent {\bf 2.11 Definition.}
\emph{A pair of $n\times n$-matrices $A,B$ are
\emph{similar} if there exists some invertible matrix $S$ such that}
\[ 
B=SAS^{-1}
\]


\noindent
Since the product of two invertible matrices is invertible this yield
an equivalence relation on $M_n({\bf{C}})$ and from 2.2 above we conclude that
$P_A(\lambda)$�only depends on its equivalence class.
The question arises if  to matrices $A$ and $B$ whose characteristic 
polynomials are equal
also are similar in the sense of Definition 2.6.
This is not true in general.
More precisely,
\emph{Jordan normal form}
determines the eventual similarity between
a pair of matrices with the same characteristic polynomial.

\bigskip




\centerline {\bf\large 3.  Jordan's normal form}
\bigskip

\noindent
{\bf Introduction.}
Theorem 3.1 below is due to 
Camille Jordan. It plays
an important role when we discuss multi-valued analytic functions in punctured discs
and is also  used  in ODE-theory.
Jordan's theorem   says
that every equivalence class in $M_n({\bf{C}})$
contains a 
matrix which is built up by Jordan blocks which are defined below.
The proof employs 
the Cayley-Hamilton decomposition from 2.7.
which shows that an arbitrary $n\times n$-matrix $A$
has a similar matrix
$B=S^{-1}AS$  represented in a block form. More precisely, to every root
$\alpha_\nu$ of $P_A(\lambda)$
of some multiplicity
$e_\nu$
there occurs a square matrix $B_\nu$
of size
$e_\nu$ and $\alpha_\nu$ is the only root of
$P_{B_\nu}(\lambda)$.
It follows that for every fixed $\nu$ one has
\[
B_\nu= \alpha\cdot E_{k\uuu \nu}+
S\uuu\nu
\]
where $E\uuu {k\uuu \nu}$ is an identity matrix of size
$k\uuu\nu$ and $S\uuu\nu$ is nilpotent, i.e. there exists an integer $m$ such that
$S\uuu\nu^m=0$.
Jordan's theorem gives a further decomposition
of these nilpotent $S$-matrices.
\medskip

\noindent
{\bf{3.0 Jordan blocks.}}
An \emph{elementary} Jordan matrix of size $4$
is  matrix of the form
\[
\begin{pmatrix}
\lambda&0&0&0\\
1&\lambda&0&0\\
0&1&\lambda&0\\
0&0&1&\lambda\\
\end{pmatrix}
\]
where $\lambda$ is the eigenvalue. For $k\geq 5$ one has similar 
expressions. In general several elementary Jordan block matrices
build up a matrix which is said to be in Jordan's normal form.

\medskip

\noindent {\bf 3.1 Theorem}. \emph{For every matrix $A$ there
exists an invertible matrix $u$ such that
$UAU^{\vvv 1}$ is in Jordan's normal form.}



\bigskip

\noindent 
\emph{Proof.} By the remark after Proposition 2.12
it suffices to prove Jordan's result when $A$ has a single eigenvalue 
$\alpha$. Replacing
$A$ by $A-\alpha$
there remains only to consider the nilpotent case, i.e when
$P_A(\lambda)=\lambda^n$ so that $A^n=0$ and then
we must find a basis
where $A$ is represented in Jordan's normal form.

\bigskip



\noindent {\bf 3.2 Nilpotent operators.}
Let $S$ be a nilpotent ${\bf{C}}$-linear operator on some $n$-dimensional 
complex vector space $V$. 
So for  each non-zero vector in $v\in V$ there exists a unique integer
$m$ such that
\[ 
S^m(v)=0\quad\&\quad
S^{m-1}(v)\neq 0
\]
The unique integer $m$ is denoted by
$\text{ord}(S,v)$. The case $m=1$ occurs if  
$S(v)=0$. If $m\geq 2$ the reader can check  
that the vectors $v,S(v),\ldots,S^{m-1}(v)$ are linearly independent
The vector space generated by this $m$-tuple is denoted by
$\mathcal C_S(v)$ and  called the  \emph{cyclic} subspace of $V$
generated by $v$.
With these notations the reader cabn check that
Jordan's theorem amounts to prove the following:
\bigskip

\noindent {\bf 3.3 Proposition}
\emph{Let $S$ be a nilpotent linear operator. Then $V$ is a direct sum of
cyclic subspaces.}
\medskip

\noindent\emph
{Proof.}
Set
\[ m^*=
\max_{v\in V}\, \text{ord}(S,v)
\]
and choose a vector  $v^*\in V$ such that
$\text{ord}(S,v^*)=m^*$. On the
quotient space
$W=\frac{V}{\mathcal C_S(v^*)}$ we notice that   $S$ induces a linear operator
which we denote  by $\bar S$.
By  induction over  $\text{dim}(V)$ we may assume that
$W$ is a direct sum of cyclic subspaces. Hence
we can pick
a finite set of vectors $\{v_\alpha\}$ in $V$ such that if
$\{\bar v_\alpha\}$ are the images in $W$, then
\[
W=\oplus\,\mathcal C_{\bar S}(\bar v_\alpha)\tag{1}
\]
For each
$v_\alpha$
we put
\[
k_\alpha=\text{ord}(\bar S,\bar v_\alpha)
\]
The construction of a quotient space  means that
\[
S^{k\uuu\alpha}(v_\alpha)\in\mathcal C_S(v^*)\tag{2}
\]
Hence there exists  a unique
$m^*$-tuple $c_0,\ldots,c_{m-1}$�in ${\bf{C}}$
such that 

\[ 
S^{k\uuu\alpha}(v_\alpha)=c_0\cdot v^*+c_1\cdot S(v^*)+ \ldots+
c_{m^*-1}\cdot S^{m^*-1}(v^*)\tag{3}
\]
Put
\[
k^*\uuu\alpha= 
\text{ord}(S,v_\alpha)\tag{4}
\]
It is obvious that
$k^*\uuu\alpha\geq k\uuu \alpha$ and (3) gives
\[ 
0=S^{k^*\uuu\alpha}(v_\alpha)=
\sum\, 
c\uuu\nu\cdot
S^{k^*\uuu\alpha\vvv k\uuu\alpha+\nu}(v^*)\tag{5}
\]
The maximal choice of $m^*$ entails that $k^*\uuu\alpha\leq m^*$.
Since the vectors
$v^*,S(v^*),\ldots S^{m^*-1}(v^*)$ are linearly independent
we see that (5) gives 
\[
c\uuu 0=\ldots=c\uuu{k_\alpha-1}=0\tag{5}
\]
Hence (3) gives a vector 
$w\uuu\alpha\in \mathcal C_S(v^*)$
such that
\[
S^{k\uuu\alpha}(v\uuu \alpha)= S^{k\uuu\alpha}(w\uuu \alpha)\tag{6}
\]
The images of $v_\alpha$ and $v_\alpha-w_\alpha$
are equal in $\mathcal C(v^*)$. Replacing 
$\{v_\alpha\}$ by the vectors $\{\xi_\alpha= v_\alpha- w_\alpha\}$
one still has
\[
W=\oplus\,\mathcal C_{\bar S}(\bar \xi_\alpha)\tag{7}
\]
and the construction of the $\xi$-vectors give
\[
\text{ord}(\bar S,\bar\xi_\alpha)=\text{ord}(S,v_\alpha)\tag{8}
\]
for each $\alpha$. At this stage
an obvious counting of dimensions give the requested
direct sum decomposition
\[
V=\mathcal C_S(v^*)\,\oplus \mathcal C_S(\xi_\alpha)
\]

\noindent
{\bf Remark.}
The proof was  bit cumbersome. The reason  is
that the direct sum decomposition  
in Jordan's Theorem is not unique.
Only the   individual \emph{dimensions}
of the cyclic subspaces which appear in a direct sum decomposition are unique.
It is instructive to perform Jordan decompositions 
of specific matrices using
an implemented program which for example
can be found in
\emph{Mathematica}.


\newpage

\centerline{\bf {4. Hermitian  operators.}}

\bigskip

\noindent
The $n$-dimensional vector space
${\bf{C}}^n$ is equipped with
the hermitian inner product:
\[ 
\langle x,y\rangle= x_1\bar y_1+\ldots+x_n\bar y_n
\]
A basis $e_1,\ldots,e_n$ is orthonormal if
$\langle e_i,e_k\rangle=\text{Kronecker's delta function}$.
To each linear operator $A$
the adjoint $A^*$ is the linear opeartor for which
\[
\langle A(x),y\rangle=
\langle x,A^*(y)\rangle
\]

\noindent
{\bf{4.0. Hermitian operators.}}
A linear operator $A$ is called Hermitian if
\[
\langle A(x),y\rangle=
\langle x,A(y)\rangle
\]
holds for all $x$ and $y$.
An equivalent condition is that $A$ is equal to its adjoint $A^*$. 
Therefore one also
refers to a self-adjoint operator, i.e the notion of a hermitian respectively
self-adjoint  matrix  is the same.
\medskip

\noindent
{\bf{4.1 Unitary operators.}}
A linear operator
$U$ is  unitary if
it preserves the inner product:
\[
\langle U(x),U(y)\rangle=
\langle x,y\rangle
\]
for all $x$ and $y$.
It is clear that a unitary operator $U$ 
sends an orthonormal basis to another 
orthonormal basis and the reader may verify
that a linear operator $U$ is unitary if and only if
\[
U^{-1}=U^*
\]

\medskip


\noindent
{\bf{4.3 Self-adjoint projections.}}
Let $V$ be a subspace of ${\bf{C}}^n$ of some dimension
$1\leq k\leq n-1$.
Its orthogonal complement is denoted by $V^\perp$ and we have 
the direct sum decomposition
\[
{\bf{C}}^n=V\oplus V^\perp
\]
To $V$ we associate the linear operator $E$ whose kernel is
$V^\perp$ while it restricts to the identity on $V$.
Here
\[
E=E^2\quad\text{and}\quad E=E^*
\]
One refers to $E$ as a self-adjoint projection.
\medskip



\noindent
{\bf{4.4 Orthonormal bases.}}
Let $V_1\subset V_2\subset \ldots V_n={\bf{C}}^n$
be a strictly increasing sequence of subspaces. So here each $V_k$
has dimension $k$.
The \emph{Gram-Schmidt orthogonalisation}
yields  an orthonormal basis $\xi_1,\ldots,\xi_n$
such that
\[
V_k={\bf{C}}\cdot\xi_1+\ldots+
{\bf{C}}\cdot\xi_k
\] 
hold for every $k$.
The verification of this wellknown construction is left to the reader.
Next, if $A$ is an arbitrary $n\times n$-matrix the fundamental theorem of algebra implies
that there exists a sequence
$\{V_k\}$ as above such that
every $V_k$ is $A$-invariant, i.e.
\[
 A(V_k)\subset V_k
\] 
hold for each $k$.
We find the orthonormal basis $\{\xi_k\}$
and construct the unitary operator $U$ which sends the standard basis in
${\bf{C}}^n$ onto this $\xi$-basis.
In this $\xi$-basis we see  that the linear operator $A$ is represented by an upper
triangular matrix. Hence we have

\medskip

\noindent
{\bf{4.5 Theorem.}}
\emph{For every $n\times n$-matrix $A$ there exists a unitary matrix
$U$ such that
$U^*AU$ is upper triangular.}


\newpage








\centerline {\bf{The spectral theorem.}}
\medskip

\noindent
It asserts the following:

\medskip

\noindent
{\bf{4.6 Theorem.}}
\emph{If $A$ is Hermitian  there exists
an orthonormal basis $e_1,\dots,e_n$ 
in ${\bf{C}}^n$ where each $e_k$ is an eigenvector 
to $A$ whose eigenvalue is a real number. Thus,
$A$ can be diagonalised in an
orthonormal basis and  expressed  by matrices this means that 
there exists
a unitary matrix $U$ such that}
\[ 
U^*AU= S\tag{*}
\] 
\emph{where $S$ is a diagonal  matrix and  every $s_{ii}$
is a real number. In particular the roots of the characteristic polynomial
$\text{det}(P_A(\lambda))$ are all real.}

\medskip

\noindent
\emph{Proof.}
Since $A$ is self-adjoint we have
a real-valued function on ${\bf{C}}^n$ defined by
\[ 
x\mapsto \langle Ax,x\rangle\tag{1}
\]
Let $m^*$ be the  maximum 
of (1) as $x$ varies over the compact unit sphere of unit vectors in
${\bf{C}}^n$.
The maximum is attained by some
complex vector $x_*$ of unit length. 
Suppose $y$ is a unit vector where
 that $y\perp x_*$ and let $\lambda$ be a complex number.
Since $A$ is self-adjoint we have:
\[
 \langle A(x_*+\lambda y),x_*+\lambda y\rangle=
 m^*+2\cdot\mathfrak{Re}
 \bigl(\lambda\cdot
\langle Ax_*,y\rangle\bigr)+|\lambda|^2\cdot
\langle Ay,y\rangle\tag{2}
\]
Now $x+\lambda y$ has norm $\sqrt{1+\lambda|^2}$ 
 and the maximality gives:
 \[
 m^*+2\cdot\mathfrak{Re}
 \bigl(\lambda\cdot
\langle Ax_*,y\rangle\bigr)+|\lambda|^2\cdot
\langle Ay,y\rangle\leq\sqrt{1+|\lambda|^2}\cdot m^*\tag{3}
\]
Suppose now that
$\langle Ax_*,y\rangle\neq 0$
and set
\[
\langle Ax_*,y\rangle= s\cdot e^{i\theta}\quad\colon\quad   s>0
\]
With $\delta>0$ we take
 $\lambda= \delta\cdot e^{-i\theta}$ and (3) entails
that
\[
2s\cdot\delta\leq (\sqrt{1+\delta^2}-1)\cdot m^*-\langle Ay,y\rangle\cdot\delta^2\tag{4}
\]
Next, by calculus one has
$2\cdot \sqrt{1+\delta^2}-1)\leq\delta^2$ so after division with $\delta$
we get
\[
2s\leq\delta\cdot\bigl(\frac{m_*}{2}- \langle Ay,y\rangle\bigr)\tag{5}
\]
But this is impossible for arbitrary small $\delta$
and hence we have proved that
\[ 
y\perp x_*\implies \langle Ax_*,y\rangle=0\tag{6}
\]

\noindent
This means that  $x_*^\perp$
is an invariant subspace for $A$
and the restricted operator remains self-adjoint. At this stage
the reader can finish the proof
to get a unitary matrix $U$ such that (*) holds.



\bigskip


\centerline {\bf{5.  Normal operators.}}
\medskip

\noindent
An $n\times n$-matrix
$A$ is  normal if it commutes with its adjoint, i.e.  
$A^*A=AA^*$  holds in $M_n({\bf{C}})$.


\medskip

\noindent
{\bf{5.1 Exercise.}}
Let $A$ be a normal matrix. Show that every
equivalent
matrix is normal, i.e. if $S$ is invertible then
$SAS^{-1}$ is also normal. The hint is to use that
\[ 
(S^{-1})^*=(S^*)^{-1}
\] 
holds
for every invertible matrix.
Conclude  that one  can refer to normal linear operators
the hermitian vector space 
${\bf{C}}^n$.
\medskip

\noindent
{\bf{5.2 Exercise.}}
Let $A$ and $B$ be two Hermitian matrices which commute, 
i.e. $AB=BA$. Show that the matrix 
$A+iB$ is normal.





\medskip

\noindent
Next, let $R$ be normal and assume that
the its characteristic polynomial has simple roots.
This means that there exists  a basis $\xi_1,\ldots,\xi_n$
formed by eigenvectors to $R$ with eigenvalues
$\lambda_1,\ldots,\lambda_n$.
Thus:
\[ 
R(\xi_\nu)=\lambda_\nu\cdot \xi_\nu\quad\colon\quad 1\leq\nu\leq n\tag{*}
\] 
Notice that $R$ is invertible if and only if
al the eigenvalues are $\neq 0$. It turns out that the normality
gives a stronger conclusion.

\medskip

\noindent
{\bf{5.3 Proposition.}} \emph{Assume that  the eigenvalues
are $\neq 0$. Then
the $\xi$-vectors in (*) are orthogonal.}
\medskip

\noindent
\emph{Proof.}
Consider some  eigenvector, say $\xi_1$.
Now we get
\[ 
R(R^*(\xi_1))=
R^*(R(\xi_1))=\lambda_1\cdot 
R^*(\xi_1)\tag{i}
\]
Hence $R^*(\xi_1)$ is an eigenvector to $R$ with eigenvalue $\lambda_1$. By 
hypothesis this eigenspace is 1-dimensional which gives
\[ 
R^*(\xi_1)=\mu\cdot \xi_1\implies
\] 
\[
\lambda_1\cdot \langle \xi_1,\xi_1\rangle=
\langle R(\xi_1),\xi_1\rangle=
\langle \xi_1),R^*(\xi_1)\rangle=\bar\mu\cdot
\langle \xi_1,\xi_1\rangle
\]


\noindent
Hence  $\mu=\bar\lambda_1$ which shows that
the eigenvalues of $R^*$ are the complex conjugates of
the eigenvalues  of $R$. There remains to show that
the $\xi$-vectors are orthogonal.
Consider two eigenvectors, say $\xi_1,\xi_2$. Then
we obtain:
\[ 
\bar\lambda_2\lambda_1\cdot\langle \xi_1,\xi_2\rangle=
\langle R\xi_1,R\xi_2\rangle
=\langle \xi_1,R^*R\xi_2\rangle
\langle \xi_1,RR^*\xi_2\rangle=
\]
\[
\langle R^*\xi_1,R^*\xi_2\rangle=\bar\lambda_1\cdot \lambda_2\cdot
\langle \xi_1,\xi_2\rangle\implies
(\bar\lambda_2\lambda_1-\lambda_2\bar\lambda_1)\cdot\langle \xi_1,\xi_2\rangle=0\tag{ii}
\]

\noindent
By assumption   $\lambda_1\neq\lambda_2$ and both are $\neq 0$. It follows
that 
$\bar\lambda_2\lambda_1-\lambda_2\bar\lambda_1\neq 0$ and then (ii) gives
$\langle \xi_1,\xi_2\rangle=0$ as required.
\medskip

\noindent
{\bf{5.4 Remark.}}
Proposition 5.3  shows that if $R$ is an invertible normal operator with
$n$ distinct eigenvalues then
there exists a unitary matrix
$U$ such that
$U^*RU$ is a diagonal matrix. But in contrast to the Hermitian
case the eigenvalues can be  complex.
\medskip


\noindent
{\bf{5.5 Exercise.}}
Let $R$  be an invertible normal operator
with distinct eigenvalues.
Show that $R$ is a Hermitian matrix if and only if
the eigenvalues are real numbers.

\medskip


\noindent
{\bf{5.6 A consequence.}} Let $R$  as above be an invertible normal operator
with distinct eigenvalues. 
From the above there exists a unitary
matrix $U$ so that $U^*RU$ is a diagonal matrix.
Let $\{a_k+ib_k\}$ be the diagonal elements where
the $a$- and the $b$-mumners are real.
Now we get the diagonalmatrix $A$ with
$a_{kk}=a_k$ and the diagonal matrix $B$ with
$b_{kk}=b_k$.
it follows that
\[
U^*RU= A+iB
\]
\medskip

\noindent
{\bf{5.7 Exercise.}}
Conclude from the above that if 
$R$ is an invertible  normal operator  woith distinct eigenvalues
then there exists
\emph{a unique pair of  pairwise  commuting 
Hermitian operators} $A,B$ such that
\[
 R=A+iB
\]


\medskip

\noindent
{\bf{5.8 The operator $R^*R$}}.
Let $R$  be  an invertible  normal operator. Here we do not assume that
the charactersitic polynomial $P_R(\lambda)$ has simple roots.
Since $R$ commutes with $R^*$ the reader can check that
$R^*R$ is a Hermitian operator whose eigenvalues are all real and positive.

\medskip

\noindent{\bf{5.9 The normal operator
$(A+iE_n)^{-1}$}}.
Let $A$ be a arbitrary  Hermitian $n\times n$-matrix. 
We have already seen that
its eigenvalues are real and denote them
by $r_1,\ldots,r_n$. 
It follows that the matrix
$A+iE_n$ is invertible.
The reader should check that the
inverse
\[
R=
(A+iE_N)^{-1}
\]
is a normal operator with eigenvalues $\{\frac{1}{r_\nu+i}\}$.

\bigskip

\noindent{\bf{4.10 The case of multiple roots}}
The assumption that the eigenvalues of a normal operator are all distinct can be 
relaxed. Thus, for every normal and invertible operator $R$ there  exists
a unitary operator $U$ such that $U^*RU$ is diagonal.


\medskip

\noindent{\bf{5.11 Exercise.}}
Prove the assertion above.
The hint is to show that if $R$ is normal and nilpotent,  i.e.$R^k=0$ for some $k\geq 1$, then
$R$ must be the zero operator.
To prove this one employs Jordan's theorem and the reader should verify that
normality of an operator prevents Jordan blocks of size $\geq 2$.
For example, with I$n=2$ we take 
\[
S=
\begin{matrix}
0&1\\0&0\end{matrix}\implies S^*=
\begin{matrix}
0&0\\1&0\end{matrix}
\]
and the reader can observe that
$S^*S\neq SS^*$, i.e. the Jordan matrix $S$ is not normal.




\newpage 



\bigskip

\centerline{\bf {5. Fundamental solutions to ODE:s.}}

\bigskip

\noindent Recall from
Calculus that every ordinary
differential equation can be expressed as a system of first
order equations. The fundamental issue is therefore to consider
a matrix valued function $A(t)$,  i.e. an $n\times n$-matrix whose
elements $\{a_{ik}(t)\}$ are functions of $t$.
Given $A(t)$ there exists  at least locally close to $t=0$, a unique
$n\times n$-matrix $\Phi(t)$ such that
\[ 
\frac{d\Phi}{dt}=A(t)\cdot \Phi(t)
\]
with the initial condition  $\Phi(0)=E_n$.
One refers to $\Phi$ as a fundamental solution.
The columns of the $\Phi$-matrix give
solutions to the homogenous system defined by $A(t)$.
Moreover, the determinant of
$\Phi(t)$ is $\neq 0$ for every $t$.
In fact his follows from the equality (*) below:
\medskip

\noindent
{\bf{Exercise.}}
The trace function of $A$ is defined by:
\[ 
\text{Tr}(A)(t)= a_{11}(t)+\ldots+a_{nn}(t)
\]
Show that  the function 
$t\mapsto \text{det}(\,\Phi(t))$ satisfies the ODE.equation
\[
\frac{d}{dt}( \text{det}\,\Phi(t))=
\text{det}\,\Phi(t)\cdot 
\text{Tr}(A)(t)
\]
Hence we have the formula
\[ 
\text{det}\,\Phi(t)=e^{\int_0^t\, \text{Tr}(A)(s)\cdot ds}\quad\colon t\geq 0\tag{*}
\]
For example, if the trace function  is identically zero then
$\text{det}\,\Phi(t)=1$ for all $t$.
\bigskip


\noindent
{\bf{5.1 Inhomogeneous equations.}}
From (*) it follows that the matrix $\Phi(t)$ is invertible for all $t$.
This gives a formula to solve
a inhomogeneous
equation:
\[
\frac{d{\bf{x}}}{dt}=
A(t)({\bf{x}}(t))+{\bf{u}}(t)\tag{1}
\] 
Here
${\bf{u}}(t)=(u_1(t),\ldots,u_n(t)$
is a given vector-valued function and one seeks
a vector-valued function ${\bf{x}}(t)=(x_1(t),\ldots,x_n(t)$
such that (1) holds and in addition
satisfies the initial condition:
\[
{\bf{x}}(0)={\bf{b}}\quad\text{where} \,\,\,
{\bf{b}}\,\,\,\text{is some vector }\tag{2}
\] 


\noindent
{\bf{Exercise.}}
Show that the unique solution to (1) is given by
\[ 
{\bf{x}}(t)=\Phi(t)({\bf{b}})+
\Phi(t)\bigr(\int_0^t\, \Phi^{-1}(s)\bigl({\bf{u}}(s)\bigr)\cdot ds\bigl)\tag{**}
\]
\medskip

\noindent
In other words, for every $t$ we first evaluate
the matrix $\Phi(t)$ on the $n$-vector ${\bf{b}}$
which gives the first time dependent vector in the right
hand side.
In the second term the inverse matrix
$\Phi^{-1}(s)$ is applied to ${\bf{u}}(s)$ for every
$0\leq s\leq t$. After integration over
$[0,t]$ we get a time-dependent  $n$-vector on which
$\Phi(t)$ is applied.









 
 
 


\newpage

\centerline{\bf\large{6. Carleman's inequality}}
\medskip

\noindent

\noindent
{\bf{Introduction }} Theorem 6.1 below was proved 1917 by
Carleman in the article
\emph{Sur le genre du denominateur $D(\lambda)$
de Fredholm}
from 1917. At that time the result was used 
to study non-singular integral equations of the Fredholm type.
For more recent applications of Theorem 6.1
we refer to Chapter XI
in [Dunford-Schwartz]. 
The Hilbert-Schmidt norm, respectively the operator norm
 were defined in � xx.
 


\noindent
{\bf{6.1 Theorem.}}
\emph{Let $A$ be an $n\times n$-matrix and 
$\lambda_1,\ldots,\lambda_n$ are the roots of 
$P_A(\lambda)$. For every   $\lambda\neq 0$ is outside $\sigma(A)$
one has the inequality:}
\[ 
\bigl|\prod_{i=1}^{i=n}\, 
\bigl[1-\frac{\lambda_i}{\lambda}\bigr ]
e^{\lambda_i/\lambda}\bigr|\cdot ||R_A(\lambda)||
\leq |\lambda|\cdot \text{exp}
\bigl(\frac{1}{2}+ \frac{{\bf{HS}}(A)^2}{2\cdot |\lambda|^2}\bigr)
\]




\bigskip

\noindent
The proof requires several steps.First we shall need
some preliminsries about traces of matrices.
\medskip

\noindent
{\bf{6.2 Traceless matrices.}}
Let $A$ be an  $n\times n$-matrix. The trace is given by:
\[
\text{Tr}(A)= b_{11}+\ldots+b_{nn}\tag{i}
\]
Recall  that 
$-\text{Tr}(A)$ is
equal to the sum of the roots of $P_A(\lambda)$.
In particular the trace of two equivalent matrices are equal.
This will be used to prove the following:
\medskip

\noindent
{\bf{6.3 Proposition.}}
\emph{Let $A$ be an $n\times n$-matrix whose trace is zero. Then there
exists a unitary matrix
$U$ such that  the diagonal elements of $U^*AU$ all are zero.}
\medskip


\noindent
\emph{Proof}. Consider
first consider the case $n=2$. By Theorem 4.0.7
it suffices to consider the case when the $2\times 2$-matrix $A$
is upper diagonal and since the trace is zero it has the form
\[ 
A=
\bigl(\,\begin{matrix} a&b\\0&-a
\end{matrix}\,\bigr)
\]
where $a,b$ is a pair of complex numbers.
If $a=0$ then the two diagonal elements are zero and wee can take $U=E_2$ to be the identity in Lemma 6.5. If $a\neq 0$ we consider a vector $\phi=(1,z)$ in ${\bf{C}}^2$.
Then $A(\phi)$ is the vector $(a+bz,-az)$ and hence the inner product becomes:
\[
\langle A(\phi),\phi\rangle=a+bz-a|z|^2\tag{i}
\]
We can write
\[
\frac{b}{a}= re^{i\theta}
\]
where $r>0$ and then (i) is zero if
\[
|z|^2=1+se^{i\theta}\cdot z\tag{ii}
\]
With $z=se^{-i\theta}$
it amounts to find a positive real number $s$ such that
$s^2=1+s$ which clearly exists.
Now we get the vector
\[ 
\phi_*=\frac{1}{1+s^2}(1,se^{-i\theta})
\]
which has unit length and 
\[
\langle A(\phi_*),\phi_*\rangle=0\tag{ii}
\]
By 4.0.6 we find another unit vector $\psi_*$ so that
$\phi_*,\psi_*$ is an orthonormal base in
${\bf{C}}^2$ and hence there exists a unitary matrix
$U$ such that $U(e_1)=\phi_*$ and $U(e_2)= \psi_*$.
If $B=U^*AB$ the vanishing in (ii) gives $b_{11}=0$. At the same time
the trace is unchanged, i.e. $\text{tr}(B)=0$ holds and hence
we also get
$b_{22}=0$. This means  that  the diagonal elements of $U^*AU$ 
are both zero as required.
\medskip

\noindent{\bf{The case $n\geq 3$}}.
For the
induction
the following is needed:
\medskip

\noindent
{\bf{Sublemma.}} \emph{Let $n\geq 3$
and assume as above that
$\text{Tr}(A)=0$. Then there exists some
non-zero vector $\phi\in{\bf{C}}^n$ such that}
\[
\langle A(\phi),\phi\rangle=0\tag{*}
\]

\noindent
\emph{Proof.}
If (*) does not hold we
get the positive number
\[
m_*=\min_\phi\, \bigl|\langle A(\phi),\phi\rangle\bigr|
\]
where the minimum is taken over unit vectors in
${\bf{C}}^n$.
The minimum is achieved by some unit vector $\phi_*$. Let
$\phi_*^\perp$ be its orthonormal complement
and $E$ the self-adjoint projection from
${\bf{C}}^n$ onto $\phi_*^\perp$.
On the $(n-1)$-dimensional inner product space
$\phi_*^\perp$ we get the linear operator
$B=EA$, i.e. 
\[ 
B(\xi)= E(A(\xi))\quad\colon\quad \xi\in \phi_*^\perp\tag{i}
\]
If $\psi_1,\ldots,\psi_{n-1}$ is an orthonormal basis in
$\phi_*^\perp$ then the $n$-tuple $\phi_*,\psi_1,\ldots,\psi_{n-1}$
is an orthonormal basis in ${\bf{C}}^n$ and since the trace of $A$ is zero we get
\[
0=\langle A(\phi_*),\phi_*\rangle+
\sum_{\nu=1}^{\nu=n-1}\, \langle A(\psi_\nu),\psi_\nu\rangle=
m+\sum_{\nu=1}^{\nu=n-1}\, \langle B(\psi_\nu),\psi_\nu\rangle\tag{ii}
\]
where we used that $E(\psi_\nu)=\psi_\nu$ for each $\nu$
and that $E$ is self-adjoint so that

\[
\langle A(\psi_\nu),\psi_\nu\rangle=
\langle A(\psi_\nu),E(\psi_\nu)\rangle=\langle E(A(\psi_\nu)),\psi_\nu\rangle
=\langle B(\psi_\nu),\psi_\nu\rangle
\]
Now (ii)  gives
\[ 
\text{Tr}(B)=-m
\]
Hence the $(n-1)\times(n-1)$-matrix which represents
$B+\frac{m}{n-1}\cdot E$
has trace zero. By an induction over $n$ we find a unit vector
$\psi\in \phi_*^\perp$
such that
\[
\langle B(\psi_*),\psi_*\rangle=-\frac{m}{n-1}
\]
Finally, since $E$ is self-adjoint we have already seen that
\[
\langle A(\psi_*),\psi_*\rangle=\langle B(\psi_*),\psi_*\rangle\implies
\bigl|\langle A(\psi_*),\psi_*\rangle\bigr |=\bigl|\frac{m}{n-1}\bigr |=
\frac{m_*}{n-1}
\]
Since $n\geq 3$ the last number is $<m_*$ which contradicts the minimal choice 
of $m_*$.
Hence we must have $m_*=0$ which proves the sublemma.
\bigskip

\noindent
{\emph{Final part of the proof.}
Let $n\geq 3$. The Sublemma  gives unit vector $\phi$
such that
$\langle A(\phi),\phi\rangle=0$.
Consider the hyperplane
$\phi^\perp$ and the operator $B$ from the Sublemma  which now has trace
zero on this $(n-1)$-dimensional space. So by an induction over
$n$
there exists an orthonormal basis $\psi_1,\ldots,\psi_{n-1}$ in
$\phi^\perp$ such that
$\langle B(\psi_\nu),\psi_nu\rangle=0$ for every $\nu$.
Now $\phi,\psi_1,\ldots,\psi_{n-1}$
is an orthonormal basis in ${\bf{C}}^n$ and if $U$
is the unitary matrix which has this $n$-tuple as column vectors
it follows that the diagonal elements of $U^*AU$ all vanish.
This finishes the proof of Proposition 6.3.


 
\newpage

\centerline{\bf{Proof Theorem 6.1}}.



\noindent
Set $B=\lambda^{-1}A$ so that $\sigma(B)=\{ \lambda_i/\lambda\}$ and
$\text{Tr}(B)=\sum\,\frac{\lambda_i}{\lambda}$.
We also have 
\[
{\bf{HS}}(B)^2=\frac{{\bf{HS}}(A)^2||}{|\lambda|^2}\quad\&\quad
\bigl |\lambda\bigr |\cdot ||R_A(\lambda)||=||(E-B)^{-1}||
\]


\noindent Hence Theorem 6.1 follows if we prove the inequality
\[
\bigl |e^{\text{Tr}(B)}\bigr|\cdot
\bigl|\prod_{i=1}^{i=n}\, 
\bigl[1-\frac{\lambda}{\lambda_i}\bigr ]
\cdot 
||(E-B)^{-1}||
\leq \text{exp}\bigl[\frac{1+ {\bf{HS}}(B)^2}{2}\bigr]\tag{*}
\]


\noindent
To prove (*) we choose an arbitrary integer $N$ such that
$N>\bigl |\text{Tr}(B)\bigr|$ and for each such $N$ we define the linear operator
$B_N$ on the $n+N$-dimensional complex space with points 
denoted by $(x,y)$ with  $y\in{\bf{C}}^N$
as follows:
\[
B_N(x,y)= (Bx\, , \, -\frac{\text{Tr}(B)}{N}\cdot y)\tag{**}
\]


\noindent
The
eigenvalues of the linear operator $E-B_N$ is the union of the $n$-tuple 
$\{1-\frac{\lambda_i}{\lambda}\}$ and 
the $N$-tuple of equal eigenvalues given by 
$1+\frac{\text{Tr}(B)}{N}$.
This gives the determinant formula
\[
\text{det}(E-B_N)=
\bigl(1+\frac{\text{Tr}(B)}{N}\bigr)^N
\cdot \prod_{i=1}^{i=n}\, (1-\frac{\lambda_i}{\lambda}\bigr)\tag{1}
\]
The choice of $N$ implies that (1) is $\neq 0$ so 
the inverse $(E-B_N)^{-1}$ exists.
Moreover, the construction of $B_N$ gives
for any pair $(x,y)$ in ${\bf{C}}^{N+n}$:
\medskip
\[
(E-B_N)^{-1}(x,y)=
\bigl (E-B)^{-1}(x), \frac{y}{
1+\frac{1}{N}\cdot \text{Tr}(B)}\bigr)\implies
\]
\[
||(E-B)^{-1}||\leq
||(E-B_N)^{-1}||
\]
Multiply both soide with
$\det(E-B_N)$ which gives
\[
\bigl|\det(E-B_N)\bigr|\cdot||(E-B)^{-1}||\leq
\bigl|\text{det}(E-B_N)\bigr|\cdot
||(E-B_N)^{-1}||\tag{2}
\]


\noindent 
Hadarmard's inequality from � xx majorises
the right 
hand side in (2) by:
\[
\frac{{\bf{HS}}(E-B_N)^{N+n-1}}{(N+n-1)^{N+n-1)/2}}\tag{3}
\]
\medskip

\noindent
Next, the construction of $B_N$ implies that its trace is zero.
So  by the result in 6.3 we can find
an orthonormal basis $\xi_1,\ldots,\xi_{n+N}$
in ${\bf{C}}^{n+N}$ such that
\[ 
\langle B_N(\xi_k),\xi_k\rangle=0\quad\colon 1\leq k\leq n+N
\]


\noindent
Relative to this basis the matrix of $E-B_N$ 
has 1 along the diagonal and the negative of the
elements of $B_N$ elsewhere. It follows that the Hilbert-Schmidt norm
satisfies the equality:
\[
{\bf{HS}}(E-B_N)^2=
N+n+{\bf{HS}}(B_N)^2=N+n+{\bf{HS}}(B)^2+ N^{-1}\cdot
|\text{Tr}(B)|^2\tag{4}
\]

\medskip

\noindent 
To simplify notations we set $\mathfrak{B}= {\bf{HS}}(B)$
Hence, (1) and the inequalities from (2-3) give:
\medskip

\[
\bigl(1+\frac{\text{Tr}(B)}{N}\bigr)^N
\cdot \prod_{i=1}^{i=n}\, (1-\frac{\lambda_i}{\lambda}\bigr)\cdot
||(E-B)^{-1}||\leq
\] 

\[
\frac{\bigl(N+n+\mathfrak{B}^2+
N^{-1}\cdot
|\text{Tr}(B)|^2\bigr)^{(N+n-1)(2}}{
\bigl(N+n-1\bigr)^{N+n-1/2}}=
\frac{\bigl(1+\frac{\mathfrak{B}^2}{N+n}+
\frac{|\text{Tr}(B)|^2}{N(N+n)}\bigr)^{(N+n-1)/2}}{
(1-\frac{1}{N+n}\bigr)^{N+n-1/2}}
\]
\bigskip

\noindent
This inequality holds for
arbitrary large $N$.
Passing to the limit as $N\to\infty$  the definition of Neper's constant $e$
shpws that the last term above converges to
$\text{exp}\bigl[\frac{1+ \mathfrak{B}^2}{2}\bigr]$ which gives
(*) above.

\newpage

\centerline {\bf{0.C.2 Hadamard's inequality.}}
\medskip

\noindent
The following result is due
Hadamard whose proof is left as an exercise.

\medskip

\noindent
{\bf{0.C.3  Theorem.}}
\emph{Let $A=\{a\uuu{\nu k}\}$ be some $p\times p$\vvv matrix whose elements are
complex numbers. To each $1\leq k\leq p$ we set}
\[
\ell\uuu p= \sqrt{|a\uuu{1k}|^2+\ldots+|a\uuu{p k}|^2}
\]
\emph{Then}
\[ 
\bigl |\text{det}(A)\bigr |\leq
\ell\uuu 1\cdots \ell\uuu p
\]
\newpage


\centerline{\bf{7. Hadamard's radius theorem.}}

\bigskip

\noindent
Hadamard's thesis \emph{Essais sur l'�tudes des fonctions donn�s par leur
d�velopment d Taylor} contains many interesting results.
Here we expose material from Section 2 in [ibid].
Consider a power series
\[
 f(z)=\sum\, c\uuu nz^n
\]
whose radius is a positive  number
$\rho$.
So $f$ is analytic in the open disc $\{|z|<\rho\}$
and has at least one singular point on the circle
$\{|z|=\rho\}$.
Hadamard found a condition in order that
these singularities consists of a finite set of poles only so that
$f$ extends to be meromorphic in some disc $\{|z|<\rho\uuu *\}$ with
$\rho\uuu * >\rho$. The condition is expressed via properties of
the  Hankel determinants 
$\{\mathcal D\uuu n^{(p)}\}$ from � 0.B.
For each $p\geq 1$ we set 
\[
\delta(p)=\, 
\limsup\uuu{n\to \infty}\, 
[\mathcal D\uuu n^{(p)}]^{\frac{1}{n}}
\]

\noindent
In the special case $p=0$ we have $\{\mathcal D\uuu n^{(0)}\}=\{c\uuu n\}$
and hence 
\[
\delta(0)= \frac{1}{\rho}=\limsup\uuu{n\to \infty}\, |c\uuu n|^{\frac{1}{n}}
\]
This entails that for every  $\epsilon>0$  there exists a constant $C\uuu\epsilon$ 
such that
\[ 
|c\uuu n|\leq C\cdot (\rho \vvv  \epsilon)^{\vvv n}\quad\text{
hold for every}\quad  n
\]
It follows trivially that
\[
|\mathcal D\uuu n^{(p)}|\leq (p+1) !\cdot C^{p+1}(\rho\vvv \epsilon)^{\vvv (p+1)n}
\]
Passing to limes superior where  high $n$:th roots are taken
we conclude that:
\[
\delta(p)= \limsup\uuu{n\to \infty}\, 
\bigl[\mathcal D\uuu n^{(p)}\bigr]^{\frac{1}{n}}\leq \rho^{\vvv (p+1)}\tag{1}
\]

\medskip


\noindent
Suppose   there exists some $p\geq 1$ 
where a strict inequality occurs:
\[
\delta(p)<\rho^{\vvv(p+1)}\tag{2}
\]
Let $p$ be the smallest integer $\geq 1$ where the strict
inequality holds. This gives  
a number $\rho\uuu *>\rho$ such that
\[
\delta(p)=\rho\uuu *^{\vvv 1}\cdot
\rho ^{\vvv p}\tag{3}
\]


\medskip

\noindent
{\bf{7.1 Theorem.}} \emph{With $p$ chosen  minimal as above,
it follows that $f(z)$ extends to a meromorphic function in the disc
of radius $\rho\uuu *$ where the number of poles counted with multiplicity
is at most  $p$.}
\bigskip


\noindent
The proof requires several steps. To begin with one has

\medskip

\noindent
{\bf{7.2 Lemma. }}\emph{When $p$ as above is minimal one has
the unrestricted limit formula:}
\[
\lim\uuu{n\to \infty}\, 
\bigl[\mathcal D\uuu n^{(p\vvv 1)}\bigr]^{\frac{1}{n}}=
\rho ^{\vvv p}\tag{*}
\]

\bigskip

TO BE GIVEN: Exercise power series+ Sylvesters equation.

\bigskip


\noindent
{\bf{7.3 The meromorphic extension
to $\{|z|<\rho\uuu *\}$.}} Lemma 7.2 entails that if $n$ is large
$\{\mathcal D\uuu n^{(p\vvv 1)}\}$
are  $\neq 0$.
So there exists some $n_*$ such that every  $n\geq n_*$
gives a    unique $p$\vvv vector
$(A\uuu n^{(1)},\ldots, A\uuu n^{(p)})$
which solves the inhomogeneous system
\[
\sum\uuu{k=0}^{k=p\vvv 1}
\, c\uuu {n+k+j}\cdot A\uuu n^{(p\vvv k)}
=\vvv c\uuu {n+p+j}\quad\colon\quad 0\leq j\leq p\vvv 1
\]
Or expressed in matrix notation:
\[
\begin{pmatrix}
c\uuu{n}&c\uuu{n+1}&\ldots&c\uuu{n+p-1}\\
c\uuu{n+1}&c\uuu{n+2}&\ldots&c\uuu{n+p}\\
\ldots&\ldots&\ldots&\ldots\\
c\uuu{n+p-1}&c\uuu{n+p}&\ldots&c\uuu{n+2p-2}\\
\end{pmatrix}\,
\begin{pmatrix}A_n^{(p)}\\\ldots\\\ldots\\\ldots\\
A_n^{(1)}\end{pmatrix}=-
\begin{pmatrix}c_{n+p} \\\ldots\\\ldots\\\ldots\\
c_{n+2p-1}\end{pmatrix}\tag{*}
\]

\medskip


\noindent
{\bf{7.4 Exercise.}}
Put
\[
H\uuu n=
c\uuu{n+2p}+ A\uuu n^{(1)}\cdot  c\uuu{n+2p\vvv 1}+
\ldots+ A\uuu n^{[(p)} \cdot c\uuu{n+p}
\]
Show that the evaluation of  $\mathcal D\uuu n^{(p)}$ 
via an expansion of the last column gives the equality:
\[ 
H\uuu n=
\frac{\mathcal D\uuu n^{(p)}}{\mathcal D\uuu n^{(p\vvv 1)}}\tag{i}
\]
\medskip

\noindent

\noindent
Next,  the  limit formula   (3) above Theorem 7.1 together with
Lemma 7.2  
give for every $\epsilon>0$
a constant $C\uuu\epsilon$ 
such that the following hold for all sufficiently large $n$:
\[ 
|H\uuu n|\leq C\uuu\epsilon \cdot 
\bigl(\frac{\rho+\epsilon}{\rho\uuu *\vvv \epsilon}\bigr)^n\tag{ii}
\]

\noindent
Next, 
put
\[ 
\delta\uuu n^{k}=A\uuu {n+1}^{(k)}\vvv A\uuu n^{(k)}
\quad\colon\quad 1\leq k\leq p\tag{iii}
\]

\medskip

\noindent
Solving (*) above for $n$ and $n+1$ a computation shows that
the $\delta$\vvv numbers satisfy the system
\[
\sum\uuu{k=0}^{k=p\vvv 1}
\, 
c\uuu {n+j+k+1}\cdot \delta \uuu n^{(p\vvv k)}=0
\quad\colon\quad 0\leq j\leq p\vvv 2
\]
\[
\sum\uuu{k=0}^{k=p\vvv 1}
\, 
c\uuu {n+p+k}\cdot \delta \uuu n^{(p\vvv k)}=
\vvv (c\uuu{n+2p}+ A\uuu n^{(1)}\cdot  c\uuu{n+2p\vvv 1}+
\ldots+ A\uuu n^{[(p)} \cdot c\uuu{n+p})
\tag{iv}
\]
\medskip



\noindent
The $\delta$\vvv numbers in the linear system ( iv)
are found  via Cramer's rule. 
The minors of degree $p\vvv 1$ in the Hankel matrices 
$\mathcal C\uuu {n+1}^{(p\vvv 1)}$ have elements from
the given
$c$\vvv sequence and  (7.0)  implies
that every such minor has an absolute value majorized by
\[
C\cdot (\rho\vvv \epsilon)^{\vvv (p\vvv 1)n}
\] 
where $C$ is a constant 
which is independent of $n$.
We conclude that the $\delta$\vvv numbers satisfy
\[
|\delta \uuu n^{(k)}|\leq |\mathcal D\uuu n^{(p\vvv 1)}|^{\vvv 1}\cdot 
C\cdot (\rho\vvv \epsilon)^{\vvv (p\vvv 1)n}\cdot |H\uuu n|\tag{v}
\]


\noindent
The unrestricted limit in Lemma 7.2
give  upper bounds for
 $|\mathcal D\uuu n^{(p\vvv 1)}|^{\vvv 1}$ so that  (iii) and (v) give:
 
\medskip
 
 \noindent
{\bf{7.5 Lemma}}
 \emph{To each $\epsilon>0$ there is a constant
 $C\uuu\epsilon$ such that}
 \[
|\delta \uuu n^{(k)}|\leq 
C\uuu\epsilon\cdot
 \bigl(\frac{\rho+\epsilon}{\rho\uuu *\vvv \epsilon}\bigr)^n
 \quad\colon\quad 1\leq k\leq p
\]
\medskip


\noindent
{\bf{7.6 The polynomial $Q(z)$}}.
Lemma 7.5  and (iii) entail that
the sequence $\{A\uuu n^{(k)}\,\colon\, n=1,2,\ldots\}$
converges for every $k$ and  we set
\[
A\uuu *^{(k)}=\lim_{n\to\infty}\, A\uuu n^{(k)}\,
\]
 Notice   that Lemma 7.5 after summations of geometric series gives
a constant $C_1$ such that
\[
|A\uuu *^{(k)}\vvv A\uuu n^{(k)}|\leq C_1\cdot 
\bigl(\frac{\rho+\epsilon}{\rho\uuu *\vvv \epsilon}\bigr)^n\tag{7.6.i}
\]
hold for every $1\leq k\leq p$ and every $n$.

\noindent
Now we consider the sequence
\[
b\uuu n=
 c\uuu{n+p}+ A\uuu *^{(1)}\cdot c\uuu {n+p\vvv 1}+
\ldots A\uuu *^{(p)}\cdot c\uuu n\tag{7.6.ii}
 \]
Equation (*) applied to   $j=0$  gives
\[
b\uuu n= 
(A\uuu *^{(1)}\vvv A\uuu n ^{(1)})
\cdot c\uuu {n+p\vvv 1}+
\ldots +(A\uuu *^{(p)}\vvv A\uuu n^{(p)})\cdot c\uuu n\tag{7.6.iii}
\]

\medskip

\noindent
Next, we have already seen that $|c\uuu n|\leq C\cdot(\rho\vvv \epsilon)^{\vvv n}$
hold for some constant $C$ which
together with (7.6.i)  gives:
\medskip

\noindent
{\bf{7.7 Lemma.}}
\emph{For every $\epsilon>0$ there exists a constant $C$ such that}
\[
|b\uuu n|\leq C\cdot \bigl(\frac{1+\epsilon}{\rho\uuu *}\bigr)^n
\]
\medskip

\noindent
Finally, consider the polynomial
\[
Q(z)=  1+ A\uuu *^{(1)}\cdot z+
\ldots A\uuu *^{(p)}\cdot z^p
\]

\noindent
Set $g(z)= Q(z)f(z)$ which has a power series
$\sum\, d_\nu z^\nu$
where 
\[
b_n=
c_n\cdot   A_*^{(p)}+\ldots
c_{n+p-1}A_*^{(1)} +c_{n+p}=d_{n+p}
\]
\medskip


\noindent
Above $p$ is fixed so Lemma 7.7 and the trivial spectral radius formula 
show that
$g(z)$ is analytic in the disc $|z|<\rho_*$. This
proves that $f$ extends and the poles are contained in
the zeros of the polynomial $Q$ which occur  in the annulus
$\rho\leq |z|<\rho_*$.







 
 














\newpage


\centerline{\bf{8. On positive definite quadratic forms}}
\bigskip


\noindent
In many situations one is asking 
when  a given a bi\vvv linear form is positive definite.
We  prove a result 
which has a geometric interpretation.
Let $m\geq 2$ and denote $m$\vvv vectors in
${\bf{R}}^m$
with  capital letters, i.e. $X=(x\uuu 1,\ldots,x\uuu m)$.
Let $N\geq 2 $ be some positive integer 
and $X\uuu 1,\ldots,X\uuu N$
an $N$\vvv tuple of real $m$\vvv vectors.
To each pair $j\neq k$
we set
\[
b\uuu{ij}= ||X\uuu j||+X\uuu k||\vvv 
||X\uuu j\vvv X\uuu k||
\]
where $||\cdot ||$ is 
the usual euclidian length in ${\bf{R}}^m$.
We get the symmetric $N\times N$\vvv matrix with elements
$\{b\uuu{ij}\}$ and the associated
quadratic form

\[ 
H(\xi\uuu 1,\ldots,\xi\uuu N)=
\sum\sum\, b\uuu{ij}\cdot \xi\uuu i\cdot \xi\uuu j
\]
\medskip

\noindent
{\bf{8.1 Theorem.}}
\emph{If the
$X$\vvv vectors are all different then
$H$ is positive definite.}
\medskip

\noindent
The proof relies upon a useful formula to express
the length of a vector in ${\bf{R}}^m$.
\medskip

\noindent
{\bf{8.2 Lemma }}There exists a constant
$C\uuu m$
such that
for every $m$\vvv vector $X$  one has
\[
||X||=C\uuu m\cdot  
\int\uuu{{\bf{R}}^m}\,
\frac{1\vvv \cos\,\langle X,Y\rangle }{||Y||^{m+1}}\cdot dY\tag{*}
\]
\medskip

\noindent
\emph{Proof.}
We use polar
coordinates and denote by $dA$  the area measure on the unit sphere
$S^{m\vvv 1}$ and
$\omega=(\omega\uuu 1,\ldots,\omega\uuu m)$
denote points on the unit sphere
$S^{m\vvv 1}$.
Notice that the integrals 
\[
\int\uuu{S^{m\vvv 1}}\, 
(1\vvv \cos\,\langle X,\omega\rangle)\cdot dA
\]
only depend upon $||X||$. Hence it suffices to prove Lemma 8.2 when
$X=(R,\ldots,0)$ where $R=||X||$ and here  the integral in (*) becomes:
\[
\int\uuu 0^\infty\, \bigl[\,
\int\uuu{S^{m\vvv 2}}\, (1\vvv\cos Rr\omega\uuu 1)\cdot dA\uuu{m\vvv 1}
\,\bigr]\cdot \frac{dr}{r^2}
\] 
where $dA\uuu{m\vvv 1}$ is the area measure on $S^{m\vvv 2}$.
Set 
\[
B(R,\omega\uuu 1)=
\int\uuu 0^\infty\, (1\vvv\cos Rr\omega\uuu 1)\cdot \frac{dr}{r^2}
\]
for each   $\vvv 1<\omega\uuu 1<1$. The variable substitution $r\to s/R$
gives
\[
B(R,\omega\uuu 1)=
R\cdot 
\int\uuu 0^\infty\, \frac{1\vvv\cos s\omega\uuu 1}{s^2}\cdot ds
=R\cdot B\uuu *(\omega\uuu 1)
\]
With these notations the integral in (*) becomes
\[ 
R\cdot\int\uuu{S^{m\vvv 2}}\,B\uuu *(\omega\uuu 1)\cdot dA\uuu{m\vvv 2}\tag{1}
\]
Hence Lemma 8.2 follows where $C\uuu m^{\vvv 1}$ is equal to (1) above.



\medskip

\noindent
\emph{Proof of Theorem 8.1.}
For a given pair $i,j$ the addition formula for the cosine\vvv function gives:
\[
1\vvv\cos\, \langle X\uuu i,Y\rangle+
1\vvv\cos\, \langle X\uuu j,Y\rangle+
\cos\, \langle (X\uuu i\vvv X\uuu j ),Y\rangle=
\]
\[
(1\vvv\cos\, \langle X\uuu i,Y\rangle)\cdot 
(1\vvv\cos\, \langle X\uuu i,Y\rangle)+\sin\,\langle X\uuu i,Y\rangle\cdot
\sin\,\langle X\uuu j,Y\rangle\tag{1}
\]
It follows that the matrix element $b\uuu{ij}$
is given by

\[
C\uuu m\cdot \int\uuu{{\bf{R}}^m}\,
\frac{(1\vvv\cos\, \langle X\uuu i,Y\rangle)\cdot 
(1\vvv\cos\, \langle X\uuu j,Y\rangle)+\sin\,\langle X\uuu i,Y\rangle\cdot
\sin\,\langle X\uuu j,Y\rangle}{||Y||^{m+1}}\cdot dY
\]
From this we see that

\[
H(\xi)= C\uuu m\cdot
\int\uuu{{\bf{R}}^m}\,\bigl([\sum\, (\xi\uuu k\cdot(1\vvv \cos\langle X\uuu k,Y\rangle) \,]^2
+[\sum\, (\xi\uuu k\cdot(\sin\langle X\uuu k,Y\rangle) \,]^2\bigr) \cdot 
\frac{dY}{||Y||^{m+1}}
\]
This shows that $H$ is positive definite as requested.

\bigskip

\noindent
{\bf{8.3 Exercise.}}
Prove more generally that for every $1<p<2$
a similar result as above holds when the elements of the matrix are:
\[
b\uuu{ij}= ||X\uuu j||^p+X\uuu k||^p\vvv 
||X\uuu j\vvv X\uuu k||^p
\]
\emph{Hint.} Employ a similar formula as in (*) where a
new constant $C\uuu{p,m}$ appears and $||Y||^{m+1}$
is replaced by $||Y||^{m+p}$.
\bigskip

\medskip

\noindent
{\bf{8.4  A class of Hermitian matrices.}}
\emph{Let $z\uuu 1,\ldots,z\uuu N$ be an $n$\vvv tuple of distinct and non\vvv
zero complex numbers. Set}
\[
b\uuu {ij}= \{\frac{z\uuu i}{z\uuu j}\}
\]
\emph{Then the matrix $B=\{b\uuu{ij}\}$ is Hermitian and positive definite.}
\medskip

\noindent
Again the proof is left as an exercise to the reader.
\medskip

\noindent
{\bf{8.5  Remark.}}
Theorem 8.1 has several applications. For example, Beurling used it to
prove the existence of certain spectral measures which arise in
ergodic processes.
Another application  from [Beurling: Notes  Uppsala
1935] goes as follows: Let $f$ and $g$
be a pair of continuous and absolutely integrable functions on
the real line. Define the function on the real  $t$\vvv line by

\[
\phi(t)=\int\uuu{\vvv \infty}^\infty\, 
[f(t+s)\vvv g(s)|\cdot ds
\]

\medskip

\noindent
{\bf{8.6 Theorem.}}
\emph{There exists a measure $\mu$ 
on the $\xi$\vvv line of total variation 
$\leq 2\sqrt{||f||\uuu 1\cdot ||g||\uuu1}$
such that}

\[ 
\phi(t)=||f||\uuu1+||g||\uuu 1+\int\uuu{\vvv \infty}^\infty\,
e^{i\xi t}\cdot d\mu(\xi)
\]
\bigskip


\noindent
The reader is invited to try to prove this theorem 
using Theorem 8.1 and the observation that
the a similar result as above holds for $L^2$\vvv functions
$f$ and $g$, i.e. this time  we set

\[
\psi(t)=\int\uuu{\vvv \infty}^\infty\, 
[f(t+s)\vvv g(s)|^2\cdot ds
\]
and one shows that there exists a measure $\gamma$ whose total variation is
$\leq 2\sqrt{||f||\uuu 2\cdot ||g||\uuu2}$ and
\[
\psi(t)=||f||\uuu2+||g||\uuu 2+\int\uuu{\vvv \infty}^\infty\,
e^{i\xi t}\cdot d\gamma(\xi)
\]
\bigskip






\newpage


\centerline{\bf{9. The Davies-Simon inequality.}}
\bigskip

\noindent
{\bf{Introduction.}}
Every
$n\times n$-matrix $A$ can be regared as a ${\bf{C}}$-linear operator on
the hermitian complex $n$-space which yields
the  operator norm $\text{Norm}(A)$.
Just  as in Theorem 6.1 we shall exhibit an inequality for the
operator norm but this time  another feature appears.
Namely, Theorem 9.1  yields
an upper bound
expressed by the euclidian  distance from $\lambda$ to $\sigma(A)$
which is better than the product which appears in the left hand side of
Theorem 6.1. On the other hand, the inequality below is restricted
to special $\lambda$-values whose absolute values are  larger than
the operator norm of $A$. Hence 
the results in 6.1 and 9.1 supplement each other.

\medskip



\noindent
{\bf{9.1 Theorem.}}
\emph{For every $n\times n$-matrix $A$
whose operator norm is $\leq 1$
the inequality below holds for every
$0\leq\theta\leq 2\pi$ outside $\sigma(A)$ }
\[
 \text{Norm}(R_A(e^{i\theta}))\leq \cot\,\frac{\pi}{4n}\cdot
 \text{dist}(e^{i\theta},\sigma(A))^{-1}
 \]



\noindent
\emph{Proof.} Schur's result in Theorem
4.0.7 reduces the proof to the case when
$A$ is upper triangular and
replacing $A$ by $e^{i\theta}A$ we may take $\theta=0$.
Set
$B= (E-A)^{-1}$ and let $B^*$ be the adjoint  operator.
The equations  $B-BA=E$ and $A^*B^*-B^*=-E$ give
\[
B(E-AA^*)B^*=BB^*-(B-E)A^*B^*
=BB^*-(B-E)(B^*-E)=B+B^*-E
\]
Set $C=B+B^*-E$ and notice that the diagonal elements
\[
c_{kk}=\frac{1}{1-\lambda_k i}+\frac{1}{1-\bar \lambda_k i}-1
= \frac{1-|\lambda_k|^2}{|1-\lambda_k|^2}\tag{1}
\]
where
$\{\lambda_k\}$ are the diagonal elements of $A$
which give points in $\sigma(A)$.
Now we shall we prove the inequality:
\[ 
|b_{ij}|^2\leq\frac{(1-|\lambda_i|^2)\cdot (1-|\lambda_j|^2)}
{(1-\lambda_i|^2\cdot |1-\lambda_j|^2}\tag{2}
\]
To get (2) we consider a vector $x$ and obtain
\[ 
\langle Cx,x\rangle=\langle B(E-AA^*)B^*x,x\rangle=
 \langle (E-AA^*)B^*x,B^*x\rangle\geq 0\tag{3}
 \] 
 where the last equality holds 
 since
 the self-adjoint
 matrix $E-AA^*$ is non-negative
 because $A$ by assumption has operator norm
 $\leq 1$.
 From (3) and the Cauchy-Schwarz inequality applied to the symmetric matrix
 we get
 \[
 |c_{ij}|^2\leq |c_{ii}|\cdot |c_{jj}|\quad\colon\quad i<j\tag{4}
\] 
for each pair $i\neq j$.
Since $c_{ij}= b_{ij}$ when $i<j$ we get (2).
Next, put $\delta=\text{dist}(1,\sigma(A))$ which means that
$|1-\lambda_i|\geq\delta$ for every $i$. From this it is clear that
(2) and the triangle inequality give
\[
 |b_{ij}|^2\leq \frac{4}{\delta^2}\quad\colon\quad i<j\tag{5}
\]
At the same time the diagonal elements satisfy:
\[
|b_{ii}|^2= \frac{1}{|1-\lambda_i|^2}\leq \frac{1}{\delta^2}\tag{6}
\]
Let $T$ be the upper triangular matrix where
$t_{ij}=2$ when $i<j$ and $t_{ii}=1$ for each $i$.
Then the elements in 
$\frac{1}{\delta}\cdot T$  majorize the absolute values of the
$B$-matrix.
The observation from � xx implies that 
\[ 
\text{Norm}(B)\leq \frac{1}{\delta}\cdot\text{Norm}(T)
\]
Now Theorem 9.1 follows from
the formula in � xx for the operator norm of $T$.


\newpage



\centerline{\bf{10. An equality by Schur.}}

\bigskip

\noindent
Let $A$ be an $n\times n$-matrix with operator norm
$\leq 1$, i.e., $A$ is a contraction.
For each
polynomisl $p(z)= a_0+a_1z+\ldots+a_Nz^N$
with complex coefficients we get the matrix $p(A)$.

\medskip

\noindent
{\bf{10.1 Theorem.}} \emph{One has the inequality}
\[
||p(A)||\leq \max_{z\in D}|, |p(z)|
\]
\medskip

\noindent
To prove this we first
consider an analytic function
$g(z)$
in the unit disc which extends continuously to the boundary
and 
an  $n\times n$-matrix $A$ whose spectrum is 
contained in the open unit disc.
Here we do not assume that $A$ is a contraction, i.e. the sole assumption is that
$\sigma(A)$ is a compact subset of the open unit disc.
Now  $g(z)$ has a  series expansion $\sum\, c_kz^k$
we know from � xx that the matrix-valued series
$\sum\, c_kA^k$ converges and gives a matrix $g(A)$. Hence there also 
exists  the exponential matrix
\[
B= e^{g(A)}
\]
The adjoint $B^*$ is found as follows.
Consider the analytic function  $g^*(z)=\sum\, \overline{c_k}z^k$. The reader can check that
\[
B^*=e^{g^*(A^*)}
\]
Put
\[
C=e^{g^*(A^*)+g(A)}=B^*B
\]
The result in � xx gives
\medskip

\noindent
{\bf{10.2 The Schur-Weierstrass inequality.}}
\emph{For each pair $A$ and $g$ as above one has}
\[
||e^{g(A)}||=
\max_{\lambda\in\sigma(A)}\, e^{\mathfrak{Re}(g(\lambda))}\tag{*}
\]
\medskip



\noindent
{\bf{10.3.  Another norm inequality.}}
Let $\alpha$ be a point in the open unit disc and suppose now that
$A$ is a contraction. It follows that
\[
(1-|\alpha|^2)\cdot \langle (E-A^*A)(y),y\rangle\geq 0\tag{i}
\]
hold for every vector $y$. 
The reader can check that (i) gives
\[
||\alpha E-A)(y)||^2\leq ||y-\bar\alpha\cdot A(y)||^2\tag{ii}
\]
Next, there
exists the matrix
\[
g_\alpha(A)= (\alpha E-A)\cdot (E-\bar\alpha A)^{-1}
\]
Consider a vector $x$ of unit norm and set
\[
y=(E-\bar\alpha A)^{-1}(x)
\]
Then
\[ 
||g_\alpha(A)(x)||^2=||\alpha E-A)(y)||^2\leq ||y-\bar\alpha\cdot A(y)||^2\tag{iii}
\]
where the last inequality used (ii).
Now
$y-\bar\alpha\cdot A(y)=x$ and hence (iii) gives
\[
||g_\alpha(A)(x)||^2||\leq||x||^2
\]
Since the vector $x$ was arbitrary we conclude that
\[ 
||g_\alpha(A)||\leq 1\tag{iv}
\]
\bigskip

\centerline {\emph{Proof of Theorem 10.1.}}
\medskip


\noindent
By scaling we can assume that the maximum norm
$|p|_D=1$.
Construct the Blaschke product $B(z9$ taken over the zeros of $p$ in the open unit disc
which gives a factorisation
\[
p(z)=B(z)\cdot e^{g(z)}
\]
where the zero-free analytic function $e^{g(z)}$ has maximum norm one, and
hence
\[
\mathfrak{Re}(g)(z)\leq 0\quad\colon\, z\in D
\]
Now
\[
p(A)= B(A)\cdot e^{g(A)}
\]
Here $B(A)$ is the product of operators of the form $-g_\alpha(A)$
where $\alpha$ are zeros of $p$ in $D$. By (iv) from (10.3)
each of these operators have norm $\leq 1$ and (*) in (*) in (10.2)
entails that
the same holds for $e^{g(A)}$.
So $p(A)$ is the product of operators of norm $\leq 1$ and
Theorem 10.1 follows.
\medskip


\noindent
{\bf{Remark.}}
The interested
reader should consult the text-book
[Davies] for  further extensions 
of Theorem 10.1 which
appear in [ibid: Chapter 10].


































\newpage



\centerline{\bf{11. An application to integral equations.}}



\bigskip


\noindent
Let $k(x,y)$ be a complex-valued continuous function on the unit square
$\{0\leq x,y\leq 1\}$. We do not assume that $k$ is symmetric,  i.e,
in general $k(x,y)\neq k(y,x)$.
 Let $f(x)$ be another  continuousfunction  on $[0,1]$.
Assume that the maximum norms of $k$ and $f$ both are $<1$.
By induction over $n$ starting with $f\uuu 0(x)= f(x)$
we get a sequence $\{f\uuu n\}$ where
\[
f\uuu n(x)=\int\uuu 0^1\, k(x,y)\cdot f\uuu{n\vvv 1}(y)\cdot dy
\quad \colon\quad n\geq 1
\]
The hypothesis entails that each $f\uuu n$ has maximum norm
$<1$ and hence there exists  a power series:
\[
u\uuu\lambda(x)= \sum\uuu{n=0}^\infty\, f\uuu n(x)\cdot \lambda^n
\] 
which converges for every  $|\lambda|<1$ and yields a continuous function
$u\uuu\lambda(x)$  on $[0,1]$.

\medskip

\noindent
{\bf{11.1 Theorem.}}
\emph{The function $\lambda\mapsto u\uuu\lambda(x)$ with values in the Banach space
$B=C^0[0,1]$ extends to a meromorphic $B$\vvv valued 
function in the whole
$\lambda$\vvv plane.}
\bigskip

\noindent
To prove this we introduce the recursive Hankel determinants for
each $0\leq x\leq 1$:
\[
\mathcal D_n^{(p)}(x)=
\det
\begin{pmatrix}
f_{n+1}(x)
&f_{n+2}(x)
&\ldots&\ldots
& f_{n+p}(x)\\
f_{n+2}(x)
&f_{n+3}(x)
&\ldots&\ldots
& f_{n+p+1}(x)\\

\ldots &\ldots &\ldots&\ldots \\
\ldots &\ldots&\ldots&\ldots\\
f_{n+p}(x)
&f_{n+p+1}(x)
&\ldots&\ldots
& f_{n+2p-1}(x)\\
\end{pmatrix}
\]
\medskip


\noindent
{\bf{Proposition 11.2}} \emph{For every $p\geq 2$ and $0\leq x\leq 1$ one has
the inequality}

\[
\bigl |\,  \mathcal D\uuu n^{(p)}(x))\,\bigr|\leq 
(p\, !)^{\vvv n}\cdot \bigl( p^{\frac{p}{2}}) ^n\cdot \frac{p^p}{p\,!}
\]
\medskip

\noindent
{\bf{11.3 Conclusion.}}
The inequality above  entails that
\[ 
\limsup\uuu{n\to \infty}\, \bigl| \mathcal D\uuu n^{(p)}(x))\,\bigr |^{1/n}
\leq 
\frac{p^{p/2}}{p\,!}
\]
Next, Stirling's formula gives:
\[
\lim\uuu{p\to \infty}\bigl[\frac{p^{1/2}}{p\,!}\,\bigr]^{\vvv 1/p}=0
\]
Hence  Hadamard's theorem gives
Theorem 11.1



\bigskip

\centerline{\emph{Proof of Proposition 11.2}}
\bigskip


\noindent
The proof requires several steps. 
First, 
we get the sequence $\{k^{(m)}(x)\}$
which starts with $k=k^{(1)}$ and:
\[
k^{(m)}(x)= \int\uuu 0^1\, k^{(m\vvv 1)}(x,s)\ddot k(s)\cdot ds
\quad\colon\quad m\geq 2
\]
It is easily seen that
\[
f\uuu{n+m}(x)= \int\uuu 0^1\, k^{m)}(x,s)\cdot f\uuu n(s)\cdot ds
\]
hold for all pairs $m\geq 1$ and $n\geq 0$.
\medskip

\noindent
{\bf{11.4 Determinant formulas.}}
Let $\phi\uuu 1(x),\ldots,\phi\uuu p(x)$ and
 $\psi\uuu x),\ldots,\psi\uuu p(x)$
be a pair of $p$\vvv tuples of continuous functions on
$[0,1]$.
For each point $(x\uuu 1,\ldots,x\uuu p)$ in
$[0,1]^p$ we put

\[ 
D_{\phi_1,\ldots,\phi_p}(x\uuu 1,\ldots,x\uuu p)
=\text{det}
\begin{pmatrix}
\phi\uuu 1(x\uuu 1)&\cdots&\phi\uuu 1(x\uuu p)\\
\cdots &\cdots&\cdots \\
\cdots &\cdots&\cdots\\
\phi\uuu p(x\uuu 1)&\cdots &\phi\uuu 1(x\uuu p)\\
\end{pmatrix}\quad\colon\quad
\]
In the same way we define $D_{\psi_1,\ldots,\psi_p}(x\uuu 1,\ldots,x\uuu p)$.
Next, define the  $p\times p$\vvv matrix with elements

\[
a\uuu{jk}= \int\uuu 0^1\, \phi\uuu j(s)\cdot \psi\uuu k(s)\, ds
\]

\medskip

\noindent {\bf{11.5 Lemma.}} \emph{One has the equality}
\[
\text{det}(a\uuu{jk})=
\frac{1}{p\,!}\int\uuu {[0,1] ^p}\,
\Phi(s\uuu 1,\ldots,s\uuu p)\cdot 
\Psi(s\uuu 1,\ldots,s\uuu p)\cdot ds\uuu 1\cdots ds\uuu p
\]

\medskip

\noindent
{\bf{11.6 Exercise.}} Prove this result using standard formulas for
determinants.

\medskip

\noindent
Next, for each $0\leq x\leq 1$ and every pair $n,p$ of
positive integers we consider the $p\times p$-matrix

\[
\begin{pmatrix}
\int_0^1\, k(x,s)f_n(s)\,ds
&\int_0^1\, k^{(2)}(x,s)f_n(s)\,ds
&\ldots&\ldots
&\int_0^1\, k^{(p)}(x,s)f_n(s)\\

\int_0^1\, k^{(2)}(x,s)f_{n+1}(s)\,ds
&\int_0^1\, k^{(2)}(x,s)f_{n+1}(s)\,ds
&\ldots&\ldots
&\int_0^1\, k^{(2)}(x,s)f_{n+1}(s)\\
\ldots &\ldots &\ldots&\ldots \\
\ldots &\ldots&\ldots&\ldots\\
\int_0^1\, k^{(p)}(x,s)f_{n+p-1}(s)\,ds
&\int_0^1\, k^{(p)}(x,s)f_{n+p-1}(s)\,ds
&\ldots&\ldots
&\int_0^1\, k^{(p)}(x,s)f_{n+p-1}(s)\\
\end{pmatrix}\quad\colon\quad
\]
\medskip

\noindent 
We also get the two determinant functions
\[
\mathcal K^{(p)}(x,s_1,\ldots,s_p)=
\det
\begin{pmatrix}
k^{(1)}(x,s_1)
&k^{(1)}(x,s_2)
&\ldots&\ldots
& k^{(1)}(x,s_p)\\
k^{(2)}(x,s_1)
&k^{(2)}(x,s_2)
&\ldots&\ldots
& k^{(2)}(x,s_p)\\
\ldots &\ldots &\ldots&\ldots \\
\ldots &\ldots&\ldots&\ldots\\
k^{(p)}(x,s_1)
&k^{(p)}(x,s_2)
&\ldots&\ldots
& k^{(p)}(x,s_p)\\
\end{pmatrix}
\]

\bigskip

\[
\mathcal F_n^{(p)}(s_1,\ldots,s_p)=
\det
\begin{pmatrix}
f_n(s_1)
&f_n(s_2)
&\ldots&\ldots
& f_n(s_p)\\
f_{n+1}(s_1)
&f_{n+1}(s_2)
&\ldots&\ldots
& f_{n+1}(s_p)\\

\ldots &\ldots &\ldots&\ldots \\
\ldots &\ldots&\ldots&\ldots\\
f_{n+p-1}(s_1)
&f_{n+p-1}(s_2)
&\ldots&\ldots
& f_{n+p-1}(s_p)\\
\end{pmatrix}
\]

\medskip


\noindent{\bf{11.7 Lemma}}.
Let 
$\mathcal D_n^{(p)}(x)$
denote the determinant of the matrix (x). Then
one has the equation





\[
\mathcal D_n^{(p)}(x)=
\frac{1}{p !}\cdot\int_{[0,1]^p}\,
\mathcal K^{(p)}(x,s_1\ldots,s_p)\cdot \mathcal F^{(p)}_n(s_1,\ldots,s_p)
\, ds_1\cdots ds_p
\]


PROOF: Apply previous lemma ....


\bigskip

\noindent
Next, using (xx) we have the equality





\bigskip

\noindent
{\bf{Exercise.}}
Use the formulas above to conclude that
the requested intequality in Proposition 11.2 holds.











 
 
 
 
 
 
 
 
 
 
 
 
 
 
 
 
 
 
 
 

\newpage







\end{document}







