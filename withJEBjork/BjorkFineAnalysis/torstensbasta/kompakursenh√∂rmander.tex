\documentclass{amsart}
\usepackage[applemac]{inputenc}


\addtolength{\hoffset}{-12mm}
\addtolength{\textwidth}{22mm}
\addtolength{\voffset}{-10mm}
\addtolength{\textheight}{20mm}
\def\uuu{_}


\def\vvv{-}

\begin{document}


\centerline{\bf{H�rmander's  $L^2$-estimate in dimension one}}

\bigskip

\noindent
Let $\Omega$ be an open  set in  ${\bf{C}}$.
Every real-valued continuous and non-negative function  $\phi$ in
$\Omega$ gives
the Hilbert space 
$\mathcal H_\phi$
whose elements are Lebesgue measurable functions $f$ in $\Omega$ such that
\[ 
\int_\Omega\, |f|^2\cdot e^{-\phi}\, dxdy<\infty\tag{0.0}
\]
and equipped with a norm defined by the square root above.
Notice that
$\mathcal H_\phi$ contain the space of test-functions
wtih compact support in $\Omega$ as a dense subspace.
Let $\psi$ be another
continuous and non-negative function 
which gives the Hilbert space 
$\mathcal H_\psi $ where the norm of an element $g$ is denoted by $||g||_{\psi}$.
The 
$\bar\partial$-operator
sends a
test-function $f$   to
$\bar\partial(f)$.
Consider 
the equation
\[
\bar\partial(f)= w\quad\colon w\in H_\psi\tag{5.0}
\]
where $f$ belongs to $\mathcal H_\phi$ with the additional condition that
its $\bar\partial$-derivative belongs to
$\mathcal H_\psi$, i.e this puts a constraint on $f$. Notice  that
$f$ is not unique   since
every holomorphic function in
$\Omega$ with a finite norm in (0.0)
belongs to the $\bar\partial$-kernel. conversely, since 
$\bar\partial$ is an elliptic differential operator
this means 
the kernel of the densely defined linear operator
\[
\bar\partial\colon \mathcal H_\phi\to \mathcal H_\psi\tag{0.1}
\]
consists of holomorphic functions for which (0,0) is finite.
We shall find condions on the pair
$\phi,\psi$ such that there exists
a constant $C$ where
(5.0) has a solution $f$ with
\[
 ||f||_\phi\leq C\cdot ||w||_\psi\quad\colon \forall\, w\in H_\psi\tag{*}
\]
A sufficent condtion to obtain solutions in (*) with a constant
$C$ is that the \emph{adjoint} operator in (0.1)
has  special properties.
Denote this adjoint  by $\bar\partial^*$. Notice that it is 
densely defined since
it
contains the space of test-funtions in
$\Omega$.
Suppose there exists a  positive contant $c_0$ such that
\[
||\bar\partial^*(g)||_\phi\geq c_0\cdot ||g||_\psi
\quad\colon\quad g\in \mathcal D(\bar\partial^*)\tag{0.2}
\]
where $\mathcal D(\bar\partial^*$ is the domain of defintion for
the adjoint operator.
Then standard  Hilbert space theory gives
(*) where onr can take $C=c_0^{-1}$.
To ensure that (0.2) we give:

\medskip

\noindent
{\bf{1.1 Definition.}}
\emph{The pair  $\phi,\psi$ satisfies the H�rmander condition if there exists some 
positive constant $c_*$ such that }
\[ 
\Delta(\psi)\vvv 2\cdot(\psi_x^2+\psi_y^2)+
\psi_x\phi_x+\psi\uuu y\phi_y\geq
2\cdot c_*\cdot e^{\psi-\phi}
\tag{1.1.1}
\]
\medskip


\noindent
{\bf{1.2 Theorem.}}
\emph{If  (1.1.1) holds then (*) has solutions with}
\[
C\leq \frac{1}{\sqrt{c_*}}\quad\colon\forall\, w\in\mathcal H_\psi
\]


\medskip


\noindent
\emph{Proof.}
Let $w$ be in the domain of definition for
the adjoint operator 
$\bar\partial^*$. If $f\in C_0^\infty(\Omega)$
one has 
\[ 
\langle \bar\partial(f),w\rangle=\int \bar\partial(f)\cdot \bar w\cdot e^{-\psi}\,dxdy=
-\int\, f\cdot 
\bigl[\bar\partial(\bar w)-\bar w\cdot \bar\partial(\psi)\bigr]\cdot
 e^{-\psi}\,dxdy\tag{i}
\]
where Stokes theorem gives the last equality.
Since $\psi$ is real-valued, 
$\bar\partial(\bar w)-\bar w\cdot \bar\partial(\psi)$
is equal to the complex conjugate of
$\partial(w)-w\cdot\partial(\psi)$. Hence (i) and the construction of adjoint operators give
\[
\bar\partial ^*(w)=-\bigl[\partial(w)-w\cdot\partial(\psi)\bigr]\cdot
 e^{\phi-\psi}\tag{ii}
\]
Taking the squared $L^2$-norm in $\mathcal H_\phi$
we obtain
\[ 
||\bar\partial^*(w)||^2_\phi=
\int\, |\partial(w)-w\cdot\partial(\psi)|^2\cdot
e^{\phi-2\psi}=
\]
\[
\int\, \bigl(|\partial(w)|^2+|w|^2\cdot|\partial(\psi)|^2\bigr)\cdot
e^{\phi-2\psi}-2\cdot\mathfrak{Re}\bigl (
\int\, \partial(w)\cdot\bar w\cdot\bar \partial(\psi)\cdot
e^{\phi-2\psi}\bigr)\tag{iii}
\]
By partial integration
the last integral in (iii) is equal
\[
2\cdot\mathfrak{Re}\bigl(\,  \int\,w\cdot[\partial(\bar w)\cdot\bar \partial(\psi)+
\bar w\cdot \partial\bar\partial(\psi)\vvv
2\bar w\cdot \bar \partial(\psi)\cdot \partial (\psi)+
\bar w\cdot \bar\partial(\psi)\cdot \partial(\phi)]\cdot e^{\phi\vvv 2\psi}\,\bigr)\tag{iv}
\]


\noindent
Next, the Cauchy-Schwarz inequality gives
\[
\bigl|2\cdot 
\int \, w\cdot  \partial(\bar w)\cdot \bar \partial(\psi)\cdot e^{\phi\vvv 2\psi}\,\bigr|\leq
\int\, \bigl(|\partial(w)|^2+|w|^2\cdot|\partial(\psi)|^2\bigr)\cdot
e^{\phi-2\psi}\tag{v}
\] 
Together (iii-v) give
\[
||\bar\partial^*(w)||^2_\phi\geq 
2\cdot \mathfrak{Re}\,  \int\, |w|^2\cdot \bigl[\,\partial\bar\partial(\psi)\vvv
2\cdot \bar \partial(\psi)\cdot \partial (\psi)+
\bar\partial(\psi)\cdot \partial(\phi)\,\bigr ]\cdot e^{\phi\vvv 2\psi}\tag{iv}
\]
Now we recall that
\[
\partial\bar\partial(\psi)=\frac{1}{4}\Delta(\psi)\quad\&\quad
\bar \partial(\psi)\cdot \partial (\psi)=\frac{1}{4}\cdot ( \psi_x^2+\psi_y^2)
\]
It follows that
(iv) is equal to
\[
2\cdot \mathfrak{Re}\,  \int\, |w|^2\cdot
\frac{1}{4}\bigl[\Delta(\psi)\vvv 2\cdot |\nabla(\psi)|^2+
\psi\uuu x\phi\uuu x+\psi\uuu y\phi\uuu y\,\bigr]\, \cdot e^{\phi\vvv 2\psi}
\tag{vi}
\]
Hence (1.1.1) gives
\[
|\bar\partial^*(w)||^2_\phi\geq c\uuu 0^2\cdot \int\, |w|^2\cdot
e^{\psi\vvv \phi}\cdot e^{\phi\vvv 2\psi}=
c_*\cdot ||w||^2\uuu\psi\tag{vi}
\]
This lower bound gives
solutions to (*) by general facts about densely defined operators on
Hilbert spaces.
and Theorem 5.2 follows.




\bigskip

\noindent {\bf{5.3 Remark.}}
The full strength of  $L^2$-estimate
appears in dimension $n\geq 2$ where one works with
\emph{plurisubharmonic functions}  and 
impose the condition that $\Omega$ is a
strictly pesudo-convex set in ${\bf{C}}^n$ and
solve inhomogeneous $\bar\partial$-equations for 
differential forms of bi-degree $(p,q)$. 
In addition to  H�rmander's original article [H�rmander] we refer to his text-book
[H�rmander] and  Chapter XX in [H�mander XX-Vol 2]
where   bounds for $\bar\partial$-equations are 
established with certain  relaxed assumptions which are used 
to settle the fundamental principle 
for over-determined systems of PDE-equations in the smooth case.





\centerline{\bf{The case $n=2$}}

\bigskip

\noindent
The special case below  may help the reader to
pursue details from
H�rmander's work, where I personally recomend his 
original article from 1962 in Acta matematica.
Take $n$=2 and let 
$D^2$ be the 2-dimensional polydisc in ${\bf{C}}^2$ with
coordinates $z=(z_1,z_2)$. Here $\bar\partial_1$ and $\bar\partial_2$
are pairwise commuting operators. Let 
$\phi(z)$  be a real-valued function in $D^2$
which is at least of class $C^2$.
We get the
Hilbert space
$\mathcal H$
of locally square integrab�e functions with
finite norm:
\[
||a||_\phi =\sqrt{\int_{D^2}\, |a(z)|^2\cdot e^{-\phi(z)}\,d\lambda(z)}
\]
where $d\lambda(z)$ is the 4-dimensional Lebesgue meaure.
Now we consider  the  densely defined
linear operator $T$ from
$\mathcal H$ into $\mathcal H\oplus\mathcal H$ defined by
\[ 
T(a)=\bar\partial_1(a)\oplus\,\bar\partial_2(a)
\]

\noindent
{\bf{A. Exercise.}}
Let
$T^*$ be the adjoint of $T$ which sends a pair $(f,g)\in\mathcal H\oplus \mathcal H$
to $\mathcal H$. Show that 
\[
T^*(f,g)=
-(\partial_1(f)-f\cdot\partial_1(\phi)+\partial_1(g)-g\cdot\partial_1(\phi)]\tag{A}
\]
\medskip

\noindent
{\bf{B. Exercise.}}
Put
\[
 \delta_1(f)= \partial_1(f)-f\cdot \partial_1(\phi)\quad\colon\quad 
  \delta_2(g)= \partial_2(g)-g\cdot \partial_2(\phi)
 \]
Use (A) to
show that
\[
||T^*(f,g)||^2=
||\delta_1(f)||^2+||\delta_2(g)||^2+2\cdot \mathfrak{Re}\,
\int\, \delta_1(f)\cdot \overline{ \delta_2(g)}\cdot e^{-\phi}\, d\lambda\tag{B.1}
\]
Next, use Stokes theorem to show that
\[
\int\, \delta_1(f)\cdot  \overline{ \delta_2(g)}
\cdot e^{-\phi}\, d\lambda
=
- \int f\cdot\overline{\partial_1(\delta_2(g))}\cdot
e^{-\phi}\,d\lambda\tag{B.2}
\]
\medskip

\noindent
{\bf{C. Exercise.}}
Put
\[ 
H(z)=
\frac{\partial^2\phi}{\partial z_1\bar\partial z_2}\tag{C.0}
\]
and by multiplication one identifies $H$ with a zero-order
differential operator. Show
the following
equality  in  the ring of differential  operators in
${\bf{C}}^2$:
\[ 
\partial_1\circ \delta_2=\delta_2\circ\partial_1-H\tag{C.1}
\]
Conclude that
(B.2) becomes
\[
\int\, f\cdot \bar g\cdot \bar H\cdot e^{-\phi}\,d\lambda
- \int f\cdot\overline{\delta_2\partial_1((g))}\cdot
e^{-\phi}\,d\lambda\tag{C.2}
\]
\medskip


\noindent
{\bf{D. The case $\bar\partial_1(g)= \bar\partial_2(f)$}}.
Use the above to show that this equality gives
\[
-2\cdot \mathfrak{Re}\, \int \bar f\cdot\delta_2(\bar\partial_2((f))\cdot
e^{-\phi}\,d\lambda=
-2\cdot \mathfrak{Re}\, \int\, |\partial_2(f)|^2\cdot e^{-\phi}\,d\lambda\tag{D.1}
\]
\medskip

\noindent
{\bf{E. Conclusion.}}
Show that (A), (B.1-2) and (D.1) and  the equality
$\bar\partial_1(g)= \bar\partial_1(f)$
give:
\[
||T^*(f,g)||^2=
||\delta_1(f)||^2+||\delta_2(g)||^2+2\cdot \mathfrak{Re}\,
\int\, f\cdot\bar  g\cdot H(z)\cdot e^{-\phi}\, d\lambda+
||\partial_2(f)||^2\tag{E.1}
\]


\noindent
Above the last term is always $\geq 0$.
To ensure that there exists a constant $c_0$ such that
\[
||f||^2+||g||^2\leq c_0^2\cdot ||T^*(f,g)||^2\tag{E.2}
\]

\noindent
one must impose  suitable conditions upon $\phi$ 
Above the mixed derivative which defines
the
$H$-function appears while 
norms $||\delta_1(f)||^2$ and $||\delta_2(g)||^2$
can be estimated as in the case $n=1$ applied 
with  $\phi=\psi$. The reader is invited to contemplate upon conditions 
which give a constant $c_0$ in (E.2) where 
H�rmander's work can be consulted.
further dertails. Above we treated a  special case
since we used the same weight  function $\phi$ and not a pair as in
the case $n=1$.



\newpage








 


\end{document}










\newpage


\centerline{\bf VI. The Corona Theorem.}

\bigskip
\noindent
{\bf Introduction.} In  the unit disc $D$ we 
have the Banach algebra $H^\infty(D)$ of bounded analytic functions.
Let $\mathfrak{M}_\infty(D)$ denote its maximal ideal space.
Via point evaluations in $D$ we get a map