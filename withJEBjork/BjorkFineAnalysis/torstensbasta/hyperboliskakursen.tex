

\documentclass{amsart}


\usepackage[applemac]{inputenc}

\addtolength{\hoffset}{-12mm}
\addtolength{\voffset}{-10mm}
\addtolength{\textheight}{20mm}


\begin{document}


\centerline{\bf{A hyperbolic boundary value equation,.}}

\medskip


\noindent
{\bf{Introduction.}}
Let $x,s$ be coordinates in
${\bf{R}}^2$ and consider the rectangle
\[
\square=\{(x,y)\colon\, 0\leq x\leq \pi\colon\, 0\leq s\leq s^*\}
\]
for some  $s^*>0$.
A continuous and real-valued function $g(x,s)$ in $\square$ is $x$-periodic if
\[
g(0,s)=g(\pi,s)\quad\colon 0\leq s\leq s^*\
\]
More generally, if $k\geq 1$ and $g(x,s)$ 
belongs to $C^k(\square)$ then
it is $x$-periodic if
\[
\partial_x^\nu(g(0,s))=
\partial_x^\nu(g(\pi,s)\quad\colon 0\leq\nu\leq k \tag{i}
\] 
In particular we can consider real-valued $C^\infty$-functions on
$\square$ for which (i) hold for every $\nu\geq 0$.
Let $a(x,s)$ and $b(x,s)$ be a pair real-valued $C^\infty$-functions on
$\square$ which are periodic in $x$. They give
the PDE-operator
\[ 
P=\partial_s-a\cdot \partial_x-b\tag{*}
\]
\medskip

\noindent
{\bf{A boundary value problem.}}
Let $p\geq 1$
and $f(x)$ is a periodic function on
$[0,\pi]$ which is $p$-times continuously differentiable.
Now we seek $F(x,s)\in C^p(\square)$
which is $x$-periodic and satisfies
$P(F)=0$ in $\square$ and the 
initial condition
\[ 
F(x,0)= f(x)
\]



\noindent
We are going to prove that this
boundary value equation has a unique solution for every $f\in C^p[0,\pi]$.
The proof  requires several steps and is not finished until
� 4.
We shall use  Hilbert space methods.
If $k\geq 2$ there exists the Hilbert space
$\mathcal H^{(k)}$
which arises via the completetion of $C^k(\square)$
with respect to the sum of $L^2$-norms
of derivatives up to order $k$ of $x$-periodic
$C^\infty$-functions in $\square$.
Sobolev's inquality gives
\[
\mathcal H^{(k)}\subset C^{k-2}(\square)\quad\colon\, k\geq 2
\]
Staying in the interval $\{0\leq x\leq \pi\}$
we also have  the Hilbert space $H^k[0,\pi]$
which is the  completion of periodic $C^\infty$-functions $f(x)$.
For a fixed $k\geq 2$
we denote by $\mathcal D_k(P)$ the set of
$f\in H^k[0,\pi]$ such that
there exists $F\in\mathcal H^{[k)}$ where
$P(F)=0$ and $F(x,0)=f(x)$ on $[0,\pi]$.
\medskip

\noindent
In 1� xx we prove the
following Hilbert space version
of the boundary problem.
\medskip


\noindent
{\bf{0.1 Theorem.}}
\emph{For each $k\geq 2$  the equality
$\mathcal D_k(P)=H^k[0,\pi]$ holds and the map $f\to F$ 
from $H^k[0,\pi]$  to
$\mathcal H^{(k)}$ is bijective.}
\medskip


\noindent
\emph{About the proof.} The material in � 1 is
used to prove that $P\colon \mathcal D_k(P)\to  H^k[0,\pi]$
is injective. The next step is to show that
$\mathcal D_k(P)$ is a dense subspace of
$H^k[0,\pi]$, and once this has been achieved we can finish the proof
rather easily.
To   prove the density of $\mathcal D_k(P)$ 
we shall consider the the linear operator $S_k$
which
for each $f\in\mathcal D_k(P)$ associates the  function
$x\mapsto F(x,s^*)$ on $[0,\pi]$.
So  here the domain of definition
$\mathcal D(S_k)=\mathcal D(P_k)$.
Material from  � 1 will be  used to prove that $S_k$ is a bounded operator, i.e.
there exists a constant $C$ such that
\[ 
||S_k(f)||_k\leq C\cdot ||f||_k\quad\colon f\in\mathcal D(S_k)
\]
Armed with this we   prove that in � xx that the requested
density of $\mathcal D_k(P)$
follows from
the following:
\medskip

\noindent
{\bf{0.1.1 Proposition.}}
\emph{For each $k\geq 2$ there exists
a positive number
$\alpha(k)$ such that
the range of 
$E-\alpha\cdot S_k$
contains all periodic $C^\infty$-functions on $[0,\pi]$ when
$\alpha<\alpha(k)$.}




\newpage


\centerline{\bf{0. A periodic equation.}}
\medskip



\noindent
To prove of Proposition 0.1.1  
we shall work with 
doubly periodic functions
$g(x,s)$ defined in the rectangle $\{0\leq x\leq \pi\}\times
\{0\leq s\leq 2\pi\}$.
When $k\geq 2$ we get the Hilbert space
$\mathcal H^{(k)}$
after the completion of doubly periodic $C^\infty$-functions with 
$L^2$-norms of derivatives up to order $k$.
This time we are given a   differential operator
\[
P=\partial _s-a(x,s)\partial_x-b(x,s)
\]
where $a$ and $b$ are doubly periodic $C^\infty$-functions.
Set
\[ 
\mathcal D_k(P)= \{ g\in \mathcal H^{(k)}\,\colon
P(g)\in \mathcal H^{(k)}\}
\]
In the product
space
$\mathcal H^{(k)}\times \mathcal H^{(k)}$
we have the graphic set
\[
\gamma_k=\{(g,P(g)\,\colon\, g\in C^\infty\}
\] 
where we always refer to doubly periodic $C^\infty$-functions as 
above.
The closure of $\gamma_k$
is the graph of a closed and densely defined 
linear operator  on
$\mathcal H^{(k)}$ denoted by $T_k$.
With these notations the following holds,  which apart from
its use during the proof of Theorem 0.1  has independent interest:

\medskip

\noindent
{\bf{1.1 Theorem.}}
\emph{There exists a positive number
$\lambda(k)$ such that}
\[
 \lambda\cdot E-T_k\colon\,\mathcal D(T_k)\to \mathcal H^{(k)}
\] 
\emph{are surjective fo every $\lambda>\lambda(k)$.}
\medskip

\noindent
To prove this theorem  we shall 
consider the  closed and densely defined operator
$\mathcal T_k$
on
$\mathcal H^{(k)}$ where  
\[
\Gamma(\mathcal T_k)=\{(g,P(g)\,\colon\, g\in \mathcal D_k(P)\}
\]
Since doubly periodic $C^\infty$-functions belong to
$\mathcal D_k(P)$ we have 
$\Gamma(T_k)\subset
\Gamma(\mathcal T_k)$, i.e. 
$\mathcal T_k$ is an extension
of
$T_k$.
Since $T_k$  is densely defined this entails that the adjoint operators
$T_k^*$ and $\mathcal T_k^*$ are equal.
A crucial step in the proof of
Theorem 1.1
is the following:

\medskip

\noindent
{\bf{1.2 Theorem.}}
\emph{One has the equality  $\mathcal D_k(P)=\mathcal D(T_k^*)$
and there exists a densely defined self-adjoint operator $B_k$ such that}
\[
T_k^*= -\mathcal T_k+B_k
\]





























 











\newpage

\centerline{\bf{� 1. Differential inequalities
and energy integrals.}}

\bigskip

\noindent
Let $M(s)$ be a non-negative real-valued continuous function
on a closed interval $[0,s^*]$.
To each $0\leq s<s^*$
we set
\[
d_M^+(s)=\limsup_{\Delta s\to 0}\, \frac{M(s+\Delta s)-M(s)}{\Delta s}
\]
where $\Delta s$ are positive during the limit.
\medskip

\noindent
{\bf{1.1 Proposition.}} \emph{Let $B$ be a real number such that
$d_M^+(s)\leq B\cdot M(s)$ holds in $[0,s^*)$. Then }
\[ 
M(s)\leq M(0)\cdot e^{Bs}\quad\colon 0<s\leq s^*
\]
\medskip

\noindent
The proof of this result is left as an exercise.
The hint is to consider the function $N(s)= M(s)e^{-Bs}$
and show that $d^+_N(s)\leq 0$ for all $s$.
Notice that $B$ is an arbitrary real number, i.e. it may also be $<0$.
More generally, let $k(s)$ be a  non-decreasing continuous function
with $k(0)=0$ and 
suppose that
\[
d^+_M(s)\leq B\cdot M(s)+k(s)\quad \colon 0\leq s<s^*
\]
Now the reader may verify that
\[
M(s)\leq M(0)\cdot e^{Bs}+\int_0^s\, k(t)\, dt\tag{1.1.1}
\]


\medskip


\noindent
Next, consider
the  set
$\square=[0,\pi]\times [0,s^*]$ as above.
A $C^1$-function
$g$ is  
periodic  with respect to $x$ if
$g$  and the partial derivatives
$\partial_s(g)$,$\partial_x(g)$  are  periodic in
$x$, i.e.
\[ 
g(0,s)= g(\pi,s)\quad\colon 0\leq s\leq s^*
\]
and similarly  for $\partial_x(g)$ and $\partial_s(g)$.



\medskip

\noindent
{\bf{1.2 Theorem.}}
\emph{Let $g$ be a periodic $C^1$-function which satisfies the PDE-equation}
\[
\partial_s(g)= a\cdot \partial_x(g)+ b\cdot g\tag{*}
\]
\emph{in $\square$ where $a$ and $b$ are 
$x$-periodic real-valued continuous functions
on $\square$. Set}
\[
M_g(s)= \max_x\, |g(x,s)|\quad\colon \,B=\max_{x,s}\, |b(x,s)|
\]
\emph{Then one has the inequality}
\[
M_g(s)\leq M_g(0)\cdot e^{Bs}
\]
\medskip


\noindent
\emph{Proof.}
Consider some $0<s<s^*$ and let $\epsilon>0$.
Put
\[ 
m^*(s)=\{ x\,\colon\, g(x,s)= M_g(s)\}
\]
The  continuity of $g$
entials that the function $M(s)$ is continuous and
the sets $m^*(s)$ are compact.
If $x^*\in m^*(s)$ the periodicity of
the
$C^1$-function $x\mapsto g(x,s)$
entails that
$\partial_x(x^*,s)=0$ and (*) gives
\[
\partial_s(g)(x,s)=b(x,s)g(x,s)\quad\colon x\in m^*(s)
\]
Next, let $\epsilon>0$. We find an open neighborhood $U$
of $m^*(s)$
such that
\[
|\partial_x(g)(x,s)|\leq \epsilon\quad\colon x\in U
\]
Now there exists
$\delta>0$ such that
\[
|g(x,s)|\leq M(s)-2\delta\quad\colon x\in [0,\pi]\setminus U
\]
Continuity gives  some
$\rho>0$ such that
if $0<\Delta s<\rho$ then  the inequalities below hold:
\[
|g(x,s+\Delta s)|\leq M(s)-\delta\quad\colon x\in [0,\pi]\setminus U
\quad\colon\,M(s+\Delta s)>M(s)-\delta\tag{i}
\]
\[
M(s+\Delta s)\leq M(s)+\epsilon\quad \colon\,
|\partial_x(g)(x,s+\Delta s)|\leq 2\epsilon \quad\colon x\in m^*(s)\tag{ii}
\]

\noindent
If $0<\Delta s<\rho$ we see that (i) gives
$x\in m^*(s+\Delta s)\subset U$
and for such $x$-values 
Rolle's mean-value theorem and the PDE-equation give
\[
M_g(x,s+\Delta s)- g(x,s)=\Delta s\cdot \partial_s(g(x,s+\theta\cdot \Delta s)=
\] 
\[
\Delta s\cdot \bigl[a(x,s+\Delta s)\cdot
\partial_x(g)(x+\theta\cdot \Delta s)+
b(x,s+\Delta s)\cdot
g(x,s+\theta\cdot \Delta s)\bigr]\tag{iii}
\]
Let  $A$ be the maximum norm of $|a(x,s)|$ taken over
$\square$.
Since $|g(x,s)|\leq M(s)$
the triangle inequality and (iii) give
\[ 
M(s+\Delta s)\leq M(s)+\Delta s[\cdot A\cdot 2\epsilon+
B\cdot M(s+\theta\cdot \Delta s)]
\]
Since the function $s\mapsto M(s)$ is continuous
it follows that
\[
\limsup_{\Delta s\to 0}\,
\frac{M(s+\Delta s)-M(s)}{\Delta s}\leq
A\cdot 2\epsilon+ BM(s)
\]
Above $\epsilon$ can be arbitrary small
and hence
\[ d^+(s)\leq B\cdot M(s)
\]
Then Proposition 1.1 gives (*) in the theorem.


\medskip

\noindent
{\bf{1.3 $L^2$-inequalities.}}
Let $g(x,s)$ be a $C^1$-function satisfying (*)
in
Theorem 1.2.
Set
\[ 
J_g(s)=\int_0^\pi\, g^2(x,s)\, dx
\]
Taking the $s$-derivative we obtain
 with respect to $s$ and (*) give
\[
\frac{dJ_g}{ds}= 2\cdot \int_0^\pi\, g\cdot \partial_s(g)\,ds=
2\cdot \int_0^\pi\, (a\partial_x(g)\cdot \partial g+ b\cdot g)\,dx
\]
The periodicity of $g$ with respect to $x$ gives
$\int_0^\pi\, \partial_x(ag^2)\, dx=0$. This
entails that the right hand side becomes
\[
\int_0^\pi\, (-\partial_x(a)+b)\cdot g^2\, dx
\]
So if $K$ is the maximum norm of
$-\partial_x(a)+b$ over $\square$ it follows that
\[
\frac{dJ_g}{ds}(s)\leq K\cdot J_g(s)
\]
Hence Theorem 1.2 gives
\[
\int_0^\pi\, g^2(x,s)\, dx\leq e^{Ks}\cdot
\int_0^\pi\, g^2(x,0)\, dx\quad\colon 0<s\leq s^*\tag{1.3.1}
\]
Integration with respect to $s$ entails that
\[ 
\iint_\square\, g^2(x,s)\, dxds\leq
\int_0^{s^*}\, e^{Ks}\,ds\cdot
\int_0^\pi\, g^2(x,0)\, dx\tag{1.3.2}
\]
Thus, the $L^2$-integral of $x\to g(x,0)$
majorizes both the area integral and each slice integral when
$0<s\leq s^*$.


\bigskip

\centerline{\bf{� 2. A boundary value equation}}
\bigskip

\noindent
Let $a(x,s)$ and $b(x,s)$ be real-valued $C^\infty$-functions on
$\square$ which are periodic in $x$ and consider the PDE-operator
\[ 
P=\partial_s-a\cdot \partial_x-b
\]
\medskip

\noindent
{\bf{2.1 Theorem.}}
\emph{For every positive integer $p$ and each
periodic $f\in C^p[0,\pi]$
there exists a unique periodic $g\in C^p(\square)$ 
where $P(g)=0$ and $g(x,0)= f(x)$.}

\bigskip

\noindent
The uniqueness follows from the results in � 1.
For if $g$ and $h$ are solutions in Theorem 2.1
then $\phi=g-h$   satisfies $P(\phi)=0$. Here
$\phi(x,0)=0$ which gives $\phi=0$ in $\square$
via
(1.3.2).
The proof of existence requires several steps and 
employs Hilbert space methods. So first we
introduce
certain Hilbert spaces.

\medskip

\noindent
{\bf{2.2 The space $\mathcal H^{(k)}$}}.
To each  integer $k\geq 2 $ 
the complex Hilbert space
$\mathcal H^{(k)}$ is the
completion  
of complex-valued
$C^k$-functions on 
$\square$ which are periodic with respect to $x$.
A trivial Sobolev  inequality  entails that 
every function in
$\mathcal H^{(2)}$ is continuous, and more generally
\[
\mathcal H^{(k)}\subset C^{k-2}(\square)\quad\colon k\geq 3
\]
and it  clear that the  first order PDE-operator $P$ maps
$\mathcal H^{(k+1)}$ into
$\mathcal H^{(k)}$.
Next,
on the periodic $x$-interval $[0,\pi]$ we have  the Hilbert spaces
$H^k[0,\pi]$ for each $k\geq 2$.
\medskip

\noindent
{\bf{2.3 Definition.}}
\emph{For each integer $k\geq 2$ we denote by 
$\mathcal D_k(P)$   the  family of all
$f(x)\in H^k[0,\pi]$ for which 
there exists some
$F(x,s)\in\mathcal H^{(k)}$ such that}
\[
P(F)=0\quad\colon\, F(x,0)=f(x)\tag{*}
\]
The results in � 1 show that $F$ is uniquely determined by (*). Moreover,
there exists a constant $C_k$ which only depends upon the
$C^\infty$-functions $a$ and $b$ and the given integer $k$
such that
\[
||F||_k\leq C_k\cdot ||f||_k\tag{2.3.1}
\]
where we have taken norms in
$\mathcal H^{(k)}$ and 
$H^k[0,\pi]$ respectively.
Next, the last inequality in (1.3.2)
shows that
$C_k$ can be chosen such that 
\[
||f^*||_k\leq C_k\cdot ||f||_k\tag{2.3.3}
\]
where $f^*(x)= F(x,s^*)$ belongs to $H^k[0,\pi]$.


\medskip

\noindent
{\bf{2.4 A density principle}}
Above we introduced the space
$\mathcal D_k(P)$. Now the following hold:
\medskip


\noindent
{\bf{2.4.1 Proposition.}}
\emph{If $\mathcal D_k(P)$ is dense in 
$\mathcal H^k[0,\pi]$, then one has the equality}
\[
\mathcal D_k(P)=\mathcal H^k[0,\pi]\tag{2.4.1}
\]

\medskip

\noindent
\emph{Proof.}
Suppose that
$\mathcal D_k(P)$ is dense.
So if $f\in\mathcal H^k[0,\pi]$ there exists a sequence
$\{f_n\}$ in 
$\mathcal D_k(P)$ where $||f_n-g||_k\to 0$.
By (2.2.2)   we have
\[ 
||F_n-F_m||_k\leq C||f_n-f_m||_k
\]
Hence $\{F_n\}$ is a Cauchy sequence in
the Hilbert space $\mathcal H^{(k)}$
and converges to a limit $F$.
Since each $P(F_n)=0$ it follows that
$P(F)=0$
and it is clear that the continuous boundary value function
$F(x,0)$�is equal to
$f(x)$ which entails that
$f$ belongs to 
$\mathcal D_k(P)$.
\medskip

\noindent
{\bf{2.5 The operators $S_k$.}}
Each $f\in \mathcal D_k(P)$ gives the function  $f^*(x)= F(x,s^*)$
in $\mathcal H^k[0,\pi]$ and  set
\[ 
S_k(f)=f^*(x)
\]
So  the domain of definition of $S_k$ is equal to
$\mathcal D_k(P)$ and  (2.3.3) gives  a constant
$M_k$ such that
\[
||S_k(f)||\leq M_k\cdot ||f||_k\quad\colon f\in  \mathcal D_k(P)
\]
where $M_k$ only  depends on the integer $k$ and the given
PDE-operator $P$. The next result constitutes  a crucial point
to attain Theorem 2.1.




\bigskip

\noindent
{\bf{2.6 Proposition.}}
\emph{For each $k\geq 2 $ there exists some
$\alpha(k)<0$ such that for every $0<\alpha<\alpha(k)$
the range of the operator $E-\alpha\cdot S_k$ contains
all periodic $C^\infty$-functions on
$[0,\pi]$.}
\bigskip

\noindent
{\bf{2.7 The density of $\mathcal D_k(P)$.}}
We prove Proposition 2.6 in � xx  and proceed to show 
that it gives the density of
$\mathcal D_k(P)$.
For if $\mathcal D_k(P)$ fails to be dense there exists
a non-zero $f_0\in\mathcal D_k(P)$ which is
$\perp$ to $\mathcal D_k(P)$.
In Proposition 2.6 we choose $0<\alpha\leq \alpha(k)$ so small that
\[
\alpha<M_k/2\tag{i}
\]
Since periodic $C^\infty$-functions are dense in
$\mathcal H^k[0,\pi]$,
 Proposition 2.6 gives  a sequence
$\{h_n\}$ in $\mathcal D_k(P)$
such that
\[ 
\lim_{n\to \infty}\, ||h_n-\alpha\cdot S_k(h_n)-f_0||_k\to 0\tag{ii}
\]
It follows that
\[
\langle f_0,f_0\rangle=1=\lim\, 
\langle f_0,h_n-\alpha\cdot S_k(h_n)\rangle=
-\alpha\cdot \lim\, \langle f_0,S_k(h_n)\rangle\tag{iii}
\]
Next, the triangle inequality and (ii) give
\[
||h_n||_k\leq 1+\alpha\cdot ||(S_k(h_n)||
\leq 1+1/2\cdot||h_n||\implies
||h_n||_k\leq 2\tag{iv}
\]
Finally, by the Cauchy-Schwarz inequality the absolute value in
the right hand side of (iii) is majorized by
\[
\alpha\cdot M_K\cdot 2<1
\] 
which contradicts (iii).
Hence the orthogonal complement of $\mathcal D_k(P)$ is zero
which proves the requested density.
\medskip


\noindent
Together with Propostion 2.4.1  we get
the following conclusive result:
\medskip

\noindent
{\bf{2.8 Theorem.}}
\emph{For each  $k\geq 2$
and $f(x)\in \mathcal H^k[0,\pi]$
there exists a unique function
$F(x,s)\in \mathcal H^{(k)}$ such that (*) holds in
Definition 2.3.}
\medskip


\noindent
{\bf{2.9 Remark.}}
The result above soplves the requested boundsry valued problem in
$\mathcal H^{(k)}$ -spaces.
Using Sobolev  inequalities
oner easily derives Theorem  2.1.





\newpage

\centerline{\bf{� 3. A doubly periodic class of inhomogeneous PDE-equations.}}


\medskip

\noindent
Before Theorem 3.2 is announced we introduce some notations.
Put
\[ 
\square =\{ 0\leq x\leq \pi\}\times
\{0\leq s\leq 2\pi\}
\]
In this section we  shall consider doubly periodic functions
$g(x,s)$ on $\square$, i.e. 
\[
g(\pi,s)= g(0,s)\quad\colon g(x,0)= g(x,2\pi)
\]
For each non-negative integer
$k$
we denote by $C^k(\square)$ the space of $k$-times
doubly periodic continuously differentiable functions. 
If 
$g\in C^k(\square)$
we set
\[ 
||g||^2_{(k)}= \sum_{j,\nu}\, \int_\square
\bigl|\frac{\partial^{j+\nu}g}{\partial x^j\partial s^\nu}(x,s)\bigr|^2\, dxds
\]
with the double sum extended  pairs  $j+\nu\leq k$.
This gives the complex Hilbert space $\mathcal H^{(k)}$
after a completion of $C^k(\square)$ with respect
to the norm  above.
Recall  that a Sobolev inequality entails that 
a function $g\in\mathcal H^{(2)}$
is automatically continuous and doubly periodic  on
the closed square.
More generally, if $k\geq 3$ 
each
$g\in\mathcal H^{(k)}$ has continuous and doubly periodic derivatives
up to order $k-2$.
Next, consider a first order PDE-operator
\[
P=\partial_s-a(x,s)\partial_x-b(x,s)\tag{3.1}
\]
where $a$ and�$b$ are real-valued doubly 
periodic $C^\infty$-functions.
It is clear that
$P$ maps
$\mathcal H^{(k)}$ into $\mathcal H^{(k+1)}$
for every $k\geq 2$.
Keeping $k\geq 2$ fixed we set

\[
\mathcal D_k(P)=\{g\in\mathcal H^{(k)}\,\colon\,
P(g)\in\mathcal H^{(k)}\}
\]
Since $C^\infty(\square)$
is dense in
$\mathcal H^{(k)}$ this yields  a densely defined operator
\[
P\colon \mathcal D_k(P)\to \mathcal H^{(k)}\tag{i}
\]
In
$\mathcal H^{(k)}\times \mathcal H^{(k)}$ we get the graph
\[
\Gamma_k=\{(g,P(g)\colon\, g\in\mathcal D_k(P)\}
\]
Since $P$ is a differential operator we know from  general results 
that
$\Gamma_k$ is a closed subspace. Hence there exists
a densely defined linear operator
and closed operator on $\mathcal H^{(k)}$ which we 
denote by $\mathcal T_k$.
So here   $\mathcal D(\mathcal T_k)= \mathcal D_k$. Set
\[
\gamma_k=\{(g,P(g)\colon\, g\in C^\infty(\square)\}\tag{ii}
\]
This is a subspace of
$\Gamma_k$ and 
denote by
$\overline{\gamma}_k$
its closure taken in  $\mathcal H^{(k)}\times \mathcal H^{(k)}$.
Since $\Gamma_k$ is closed we have
\[
\overline{\gamma}_k\subset\Gamma_k
\]
We get
the densely defined linear operator
$T_k$ whose graph is
$\overline{\gamma}_k$. By this construction
$\mathcal T_k$ is an extension of
$T_k$ which in particular gives
the
inclusion
\[
\mathcal D(T_k)\subset\mathcal D(\mathcal T_k)\tag{iii}
\] 
Next, let $E$ be the identity operator
on
$\mathcal H^{(k)}$.
 With these notations we shall prove:

\bigskip

\noindent 
{\bf{3.2  Theorem.}}
\emph{For each integer $k\geq 2$ there exists
a positive real number $\rho(k)$ such that
the map}
\[
T_k-\lambda\cdot E\colon\mathcal H^{(k)} \to  \mathcal H^{(k)}  
\] 
\emph{is 
bijective
for every $\lambda>\rho(k)$.}
\medskip



\noindent
The proof requires several steps and is not finished until
� 3.x. First we shall study the adjoint operator $T_k^*$
and establish the following:
\medskip

\noindent
{\bf{3.3  Proposition.}}
\emph{One has the equality $\mathcal D(T_k^*)= \mathcal D_k(P)$
and there exists a bounded self-adjoint operator
$B_k$ on
 $\mathcal H^{(k)}$ such that}
 \[
 T_k^*=-\mathcal T_k+B_k
 \]
 

\medskip




\noindent{\emph{Proof of Proposition 3.3}}
Keeping $k\geq 2$ fixed we set $\mathcal H=\mathcal H^{(k)}$.
For each pair $g,f$ in $\mathcal H$
their inner product is defined by
\[
\langle f,g\rangle=
 \sum\, \int_\square
\frac{\partial^{j+\nu}f}{\partial x^j\partial s^\nu}(x,s)
\cdot
\overline{\frac{\partial^{j+\nu}g} {\partial x^j\partial s^\nu}}(x,s)
\, dxds
\]
where the sum is taken when
$j+\nu\leq k$.
Introduce the differential operator
\[
\Gamma= \sum_{j+\nu\leq  k}\, (-1)^{j+\nu}\cdot \partial_x^{2j}\cdot \partial_s^{2\nu}
\]
Partial integration gives
\[
\langle f,g\rangle=\int_\square\, f\cdot \Gamma(\bar g)\, dxds=
\int_\square\, \Gamma(f)\cdot\bar g\, dxds
\quad\colon\, f,g\in C^\infty
\tag{i}
\]
Now we consider the operator
$P=\partial_s-a\cdot\partial_x-b$ and 
get 
\[
\langle P(f),g\rangle=\int_\square\, P(f)\cdot \Gamma(\bar g)\,dxds\tag{ii}
\]
Partial integration identifies (ii) with
\[
-\int_\square\, f\cdot \bigl(\partial_s-\partial_x(a)-a\cdot\partial_x-b)\circ\Gamma(\bar g)\,dxds\tag{iii}
\]


\noindent
{\bf{1.1 Exercise.}}
In (iii)  appears the composed differential operator
\[
\partial_s-\partial_x(a)-a\cdot\partial_x-b)\circ\Gamma
\]
Show that in the ring of differential operators with
$C^\infty$-coefficients this differential operator can be written
in the form
\[
\Gamma\circ(\partial_s-a\cdot\partial _x-b)+Q(x,s,\partial_x,\partial_s)
\]
where $Q$ is a differential of order $\leq 2k$ 
with coefficients in $C^\infty(\square)$.
Conclude from the above that
\[
\langle Pf,g\rangle=-
\langle f,Pg\rangle+\int_\square\ f\cdot Q(\bar g)\,dxds\tag{1.1.1}
\]
\medskip

\noindent
{\bf{1.2 Exercise.}}
With $Q$ as above we have a bilinear form which sends a pair
$f,g$ in $C^\infty(\square)$ to 
\[
\int_\square\ f\cdot Q(\bar g)\,dxds\tag{1.2.1}
\]
Use partial integration and
the
Cauchy-Schwarz inequelity to show that
there exists a conatant $C$ which depends on $Q$ only such that
the absolute value of (1.2.1) is majorized by
$C_Q\cdot ||f||_k\cdot ||g||_k$.
Conclude that
there exists a bounded linear operator 
$B_k$ on $\mathcal H$ such that
\[
\langle f,B_k(g)\rangle=\int_\square\ f\cdot Q(\bar g)\,dxds\tag{1.2.2}
\]

\medskip

\noindent
{\bf{1.3 Proof that $B_k$ is self-adjoint}}
From the above we have
\[
\langle Pf,g\rangle=
-\langle f,Pg\rangle+
\langle f,B_k(g)\rangle\tag{1.3.1}
\]

\noindent
Keeping $f$ in $C^\infty(\square)$ we notice that
$\langle f,B_k(g)\rangle$ is defined for every
$g\in\mathcal H$.
From this the reader can check that (1.3.1) remains valid when
$g$ belongs to $\mathcal D(\mathcal T_k)$
which means that
\[
\langle Pf,g\rangle=
-\langle f,\mathcal T_kg\rangle+
\langle f,B_k(g)\rangle\quad\colon f\in C^\infty(\square)\tag{1.3.2}
\]

\medskip

\noindent
Moreover, when both $f$ and $g$ belong to $C^\infty(\square)$
we can
reverse their positions in (*) which gives
\[
\langle Pg,f\rangle=
-\langle g,Pf\rangle+
\langle g,B_k(f)\rangle\tag{1.3.3}
\]
Since $a$ and $b$ are real-valued it is clear that
\[
\langle Pg,f\rangle=-\langle f,Pg\rangle\tag{1.3.4}
\]
It follows that
\[
\langle f,B_k(g)=\langle g,B_k(f)\quad\colon f,g\in C^\infty(\square)\tag{1.3.5}
\]
Since this hold for all pairs of $C^\infty$-functions and
$B_k$ is a bounded linear operator on
$\mathcal H$ the density of $C^\infty(\square)$ entails that
$B_k$ is a bounded self-adjoint operator
on $\mathcal H$.



\medskip

\noindent
{\bf{ 1.4  The equality 
$\mathcal D(T_k^*)=
\mathcal D_k(P)$.}}
The density of $C^\infty(\square)$ in $\mathcal H$
entails that  a function $g\in \mathcal H$ belongs to $\mathcal D(T_k^*)$  
if and only if there exists a constant $C$ such that
\[
|\langle Pf,g\rangle|\leq C\cdot ||f||\quad\colon f\in C^\infty(\square)\tag{1.4.1}
\]
Since $B_k$ is a bounded operator,
(1.3.2) gives the inclusion
\[
\mathcal D_k(P)\subset
\mathcal D(T_k^*)\tag{1.4.2}
\]
To prove the opposite inclusion we use that
the $\Gamma$-operator is elliptic.
If $g\in \mathcal D(T_k^*)$
we have from (i) in � 1.1:
\[
\langle Pf,g\rangle=\langle f,T_k^*g\rangle=
\int\, \Gamma(f)\cdot \overline{T_k^*(g)}\, dxds
\quad\colon f\in C^\infty(\square)
\]
Similarly
\[
\langle f,B_k(g)\rangle=
\int\, \Gamma(f)\cdot \overline{B_k(g)}\, dxds
\]
Treating $\mathcal T_k(g)$ as a distribution the equation
(1.3.2)  entails that
the elliptic operator $\Gamma$ annihilates
$T^*_k(g)-\mathcal T_k(g)+B_k(g)$.
Since both
$T^*_k(g)$ and $B_k(g)$ belong to $\mathcal H$
this implies by the general result in � xx that
$\mathcal T_k(g)$  belongs to $\mathcal H$
which proves the 
requested equality (1.4) and at the same time the operator equation
\[
T_k^*=-\mathcal T_k(g)+B_k\tag{1.4.3}
\] 

\bigskip


\centerline {\bf{3.4  An inequality.}}
\medskip

\noindent
Let $f\in C^\infty(\square)$
and $\lambda$ is a positive real number.
Then
\[
||\mathcal T_k(f)-\frac{1}{2}B_k(f)-\lambda\cdot f||^2=
\]
\[
||\mathcal T_k(f)-\frac{1}{2}B_k(f)||^2+\lambda^2\cdot ||f||^2-
\lambda\bigl(\langle \mathcal T_k(f)-\frac{1}{2}B_k(f),f\rangle+
\langle f, \mathcal T_k(f)-\frac{1}{2}B_k(f)\rangle\bigr)
\]
The last term is $\lambda$ times
\[
\langle \mathcal T_k(f),f\rangle+
\langle f,\mathcal T_k(f)\rangle-\langle f,B_kf\rangle\tag{i}
\]
where we used that $B_k$ is symmetric.
Now $T_k=\mathcal T_k$ holds
on $C^\infty(\square)$ and the definition of adjoint
operators give
\[
\langle \mathcal T_k(f),f\rangle=
\langle f,T_k^*\rangle\tag{ii}
\]
Then (1.4.3) implies that (i) is zero and hence we have
proved 
\[
||T_k(f)-\frac{1}{2}B_k(f)-\lambda\cdot f||^2=
\lambda^2\cdot ||f||^2+||T_k(f)-\frac{1}{2}B_k(f)||^2\geq \lambda^2\cdot ||f||^2\tag{iii}
\]
From (iii)  and the triangle inequality for norms we obtain
\[
||T_k(f)-\lambda\cdot f||\geq \lambda\cdot ||f||-
\frac{1}{2}||B_k(f)||\tag{iv}
\]
Now $B_k$ has a finite operator norm and if
$\lambda\geq ||B_k||$ we see that 
\[
||T_k(f)-\lambda\cdot f||\geq
\frac{\lambda}{2}\cdot ||f||\tag{v}
\]


\noindent
Finally, since $C^\infty(\square)$ is dense in $\mathcal D(T_k)$
it is clear that (v) gives
\medskip


\noindent
{\bf{3.41 Proposition.}}
\emph{One has the inequality}
\[
||T_k(f)-\lambda\cdot f||\geq
\frac{\lambda}{2}\cdot ||f||\quad\colon f\in\mathcal D(T_k)\tag{3.4.1}
\]
\medskip

\centerline{\emph{� 3.5. Proof of Theorem 3.2}}
\bigskip

\noindent
Suppose we have found some
$\lambda^*\geq \frac{1}{2}\cdot ||B||$
such that
$T_k-\lambda$  has a dense range in $\mathcal H$
for every $\lambda\geq \lambda^*$.
If this is so we fix $\lambda\geq \lambda^*$
 and take some
$g\in \mathcal H$. The hypothesis gives a
sequence $\{f_n\in \mathcal D(T_k)\}$ such that
\[
\lim_{n\to\infty}\,||T_k(f_n)-\lambda\cdot f_n-g||=0
\]
In particular $\{T_k(f_n)-\lambda\cdot f_n\}$ is a Cauchy sequence 
in $\mathcal H$
and (1.5.x )  implies that $\{f_n\}$ is a Cauchy sequence 
in the Hilbert space $\mathcal H$
and hence converges to
a limit $f_*$. Since the operator $T_k$ is closed we conclude that
$f_*\in\mathcal D(T_k)$ and we get  the equality
\[
T_k(f_*)-\lambda\cdot f_*=g
\]
Since $g\in\mathcal H$ was arbitrary we have proved Theorem 3.2.

\bigskip

\noindent{\bf{3.5.1 Density of the range.}}
There remains to find $\lambda^*$ as above.
By the construction of adjoint operators, the range of $T_k-\lambda\cdot E$
fails to be dense
if and ony if
$T_k^*-\lambda$ has a non-zero kernel.
So assume that
\[ 
T_k^*(f)-\lambda\cdot f=0
\]
for some $f\in\mathcal D(T_k^*)$ which is not identically zero.
Notice that $T_k$ sends real-valued functions into real-valued
functions. So above we can assume that
$f$ is real-valued and 
normalised so that
\[
\int_\square f^2(x,s)\, dxds=1\tag{i}
\]
From (i) and Proposition 3.3 we have
\[
\mathcal T_k(f)+\lambda\cdot f-B(f)=0\tag{ii}
\]
Let us  consider
the function
\[ 
V(s)= \int_0^\pi\, f^2(x,s)\, dx
\]
Since $k\geq 2$ is assumed we recall  that
the $\mathcal H$-function $f$ is of class
$C^1$ at least. The $s$-derivative of $V(s)$ becomes:
\[ 
\frac{1}{2}\cdot V'(s)=  \int_0^\pi\, f\cdot \frac{\partial f}{\partial s}\, dx\tag{iii}
\]
By (ii) we have
\[
 \frac{\partial f}{\partial s}-
a(x)\frac{\partial f}{\partial x}-b\cdot f=B(f)-\lambda\cdot f
\]
\medskip

\noindent
Hence the right hand side in (iii) becomes
\[
-\lambda\cdot V(s)+ \int_0^\pi\, f(x,s)\cdot B(f)(x,s)\,dx
+ \int_0^\pi\,a(x,s)\cdot f(x,s)\cdot \frac{\partial f}{\partial x}(x,s)\,dx
\]
By partial integration the last term is equal to
\[
-\frac{1}{2}\int_0^\pi\, \partial_x(a)(x,s)\cdot f^2(x,s)\,dx\tag{iv}
\]
Set
\[
M=\frac{1}{2}\cdot \max_{(x,s )\in\square}\,|\partial_x(a)(x,s)|
\]
From the above  we get the inequality
\[
\frac{1}{2}\cdot V'(s)\leq
(M-\lambda)\cdot V(s)+\int_0^\pi\, f(x,s)\cdot B(f)(x,s)\,dx\tag{v}
\]
Set
\[
 \Phi(s)=\int_0^\pi\, |f(x,s)|\cdot |B(f)(x,s)|\,dx
\]
Since the $L^2$-norm of $f$ is one
the Cauchy-Schwarz inequality 
gives
\[
\int_{-\pi}^\pi \Phi(s)\,ds \leq
\sqrt{\int_\square\,  |B(f)(x,s)|^2\, dx ds}\leq ||B(f)||
\]
where the last equality follows since
the  squared integral of $B(f)$ is majorized by its squared
norm in $\mathcal H$.
When $\lambda>M$ it follows from  (v) that
\[
(\lambda-M)\cdot V(s)+
\frac{1}{2}\cdot V'(s)\leq\Phi(s)\tag{vi}
\]
Next, since $f$ is double periodic we have  $V(-\pi)= V(\pi)$
so after an integration (vi) gives
\[
(\lambda-M)\cdot \int_\pi^\pi V(s)\, ds=\int_{-\pi}^\pi \Phi(s)\,ds \leq ||B(f)||\tag{vii}
\]
Finally, thr normalisation (i) gives
$\int_\pi^\pi V(s)\, ds=1$ and then (vii) cannot hold
if
\[
\lambda>M+||B(f)||
\]

\medskip


\noindent
{\bf{Remark.}}
Set
\[ 
\tau=\min_f\,||B(f)||
\] 
with the minimum taken over
funtions $f\in\mathcal D(T_0^*)$ whose $L^2-$integral is normalised by
(i) above.  The proof has shown that
the kernel of $T_0^*-\lambda$ is zero for all $\lambda>M+\tau$.

\newpage


\centerline{\bf{A special solution.}}
\bigskip


\noindent
Let $f(x)$ be a periodic $C^\infty$-function 
on $[0,\pi]$.
Put
\[ 
Q= a(x,s)\cdot \frac{\partial}{\partial x}+ b(x,s)
\]
Let  $\eta(s)$ be a $C^\infty$-function of $s$ and 
$m$ some  positive integer
If $\lambda>0 $ is a real number.
we set
\[ 
g_\lambda(x,s)=\eta(s)\cdot f+\eta(s)\cdot\sum_{j=1}^{j=m}\,
\frac{(s-\pi)^j}{j!}\cdot (Q-\lambda)^j(f)\quad\colon 0\leq s\leq \pi\tag{i}
\]
We choose $\eta$ to be a real-valued $C^\infty$-function
such that
$\eta(s)=0$ when $s\leq 1/4$ and -1 if $s\geq 1/2$.
Hence $g_\lambda(x,s)=0$ in (i) when
$0\leq  s\leq 1/4$
and we extend the function to $[-\pi\leq s\leq \pi$ where
$g_\lambda(x,-s)= g_\lambda(x,s)$ if $0\leq s\leq \pi$.
So now $g_\lambda$ is $\pi$-periodic with respect to $s$
and  vanishes when
$|s|\leq 1/4$.
\medskip

\noindent
{\bf{Exercise.}}
If $1/2\leq s\leq \pi$ we have $\eta(s)=1$. Use (i) to show that 
\[
(P+\lambda)(g_\lambda)=
\frac{\partial g_\lambda}{\partial s}
-(Q-\lambda)(g_\lambda)=\frac{(s-\pi)^m}{m!}\cdot (Q-\lambda)^{m+1}(f)
\]
hold when
$1/2\leq s\leq \pi$.
At the same time
$g_\lambda(s)=0$ when $0\leq s\leq 1/4$.
So $(P+\lambda)(g)$
is a function whose derivatives
with respect to $s$ vasnish up to order
$m$ at $s=0$ and $s=\pi$ and is therefore
doubly periodic of class $C^m$ in
$\square$.
Now Theorem 2.2  applies. For a given $k\geq 2$
we choose a sufficently large $m$
and find $h(x,s)$ so that  
\[ 
P(h)+\lambda\cdot h=(P+\lambda)( g_\lambda)(x,s)
\]
where $h$ is $s$-periodic, i.e.
\[ 
h(x,0)= h(x,\pi)
\]
Notice also that
$g_\lambda(x,0)=0$ while $g_\lambda(x,\pi)= f(x)$.
Set
\[ 
g_*(x)=h-g_\lambda
\]
Then  $P(g_*)+\lambda\cdot g_*=0$
and 
\[ 
g_*(x,0)-g_*(x,\pi)=f(x)
\]
\medskip

\noindent
Above we started with the $C^\infty$-function.
Given $k\geq 2$
we can take
$m$ sufficiently large during the constructions above
so that $g_*$ belongs to
$\mathcal H^{(k)}(\square)$.






\end{document}










\documentclass{amsart}


\usepackage[applemac]{inputenc}

\addtolength{\hoffset}{-12mm}
\addtolength{\voffset}{-10mm}
\addtolength{\textheight}{20mm}


\begin{document}


\noindent
{\bf{Introduction.}}
We shall expose a result due to Friedrichs from the article \emph{xxx}.
To begin with we shall consider hyperbolic equations of
one space variable while the general case is postponed until
� xx. Here is the crucial
boundary value equation in dimension one.
Let $x,s$ be coordinstes in
${\bf{R}}^2$ and consider the rectangle
\[
\square=\{(x,y)\colon\, 0\leq x\leq \pi\colon\, 0\leq s\leq s^*\}
\]
where $s^*>0$.
A continuous and real-valued function $g(x,s)$ in $\square$ is $x$-periodic if
\[
g(0,s)=g(\pi,s)\quad\colon 0\leq s\leq s^*\
\]
More generally, if $k\geq 1$ and $g(x,s)$ 
belongs to $C^k(\square)$ then
it is $x$-periodic if
\[
\partial_x^\nu(g(0,s))=
\partial_x^\nu(g(\pi,s)
\] 
hold for each $0\leq \nu\leq k$.
In particular we can consider real-valued $C^\infty$-functions on
$\square$ for which (xx) hold for every $\nu\geq 0$.
Let $a(x,s)$ and $b(x,s)$ be a pair real-valued $C^\infty$-functions on
$\square$ which are periodic in $x$. Consider the PDE-operator
\[ 
P=\partial_s-a\cdot \partial_x-b
\]
\medskip

\noindent
{\bf{A boundary value problem.}}
Let $p\geq 1$
and $f(x)$ is a periodic function on
$[0,\pi]$ which is $p$-times continuously differentiable.
Now we seek $g(x,s)\in C^p(\square)$
which is $x$-periodic and satisfies
$P(g)=0$ in $\square$ and 
initial condition
\[ 
g(x,0)= f(x)
\]
\medskip


\noindent
We are going to prove that this
boundary value equation has a unique solution $g$�for every
$f$.
Notice that the regularity is expressed by $p$, i.e
one has a specific boundary value problem for each positive integer $p$.
The proof  requires several steps and is not finished unitl
� 4.
A crucial result of independent interest occurs ion
� 3 where
we encounter certain densely defined linear operators
on Hilbert spsces of the Sobolev type.

\bigskip

\centerline{\bf{� 1. Differential inequalities.}}

\bigskip

\noindent
Let $M(s)$ be a non-negative real-valued continuous function
on a closed interval $[0,s^*]$.
To each $0\leq s<s^*$
we set
\[
d_M^+(s)=\limsup_{\Delta s\to 0}\, \frac{M(s+\Delta s)-M(s)}{\Delta s}
\]
where $\Delta s$ are positive during the limit.
\medskip

\noindent
{\bf{1.1 Proposition.}} \emph{Let $B$ be a real number such that
$d_M^+(s)\leq B\cdot M(s)$ holds in $[0,s^*)$. Then }
\[ 
M(s)\leq M(0)\cdot e^{Bs}\quad\colon 0<s\leq s^*
\]
\medskip

\noindent
The proof of this result is left as an exercise.
The hint is to consider the function $N(s)= M(s)e^{-Bs}$
and show that $d^+_N(s)\leq 0$ for all $s$.
Notice that $B$ is an arbitrary real number, i.e. it may also be $<0$.
More generally, let $k(s)$ be a  non-decreasing continuous function
with $k(0)=0$. 
suppose that
\[
d^+_M(s)\leq B\cdot M(s)+k(s)\quad \colon 0\leq s<s^*
\]
Now the reader may verify that
\[
M(s)\leq M(0)\cdot e^{Bs}+\int_0^s\, k(t)\, dt\tag{1.1.1}
\]


\medskip


\noindent
Next, consider
a product set
$\square=[0,\pi]\times [0,s^*]$
where $0\leq x\leq\pi$.
A $C^1$-function
$g$ is  
periodic  with respect to $x$ if
$g$  and the partial derivatives
$\partial_s(g)$,$\partial_x(g)$  are  periodic in
$x$, i.e.
\[ 
g(0,s)= g(\pi,s)\quad\colon 0\leq s\leq s^*
\]
and similarly  for $\partial_x(g)$ and $\partial_s(g)$.



\medskip

\noindent
{\bf{1.2 Theorem.}}
\emph{Let $g$ be a periodic $C^1$-function which satisfies the PDE-equation}
\[
\partial_s(g)= a\cdot \partial_x(g)+ b\cdot g\tag{*}
\]
\emph{in $\square$ where $a$ and $b$ are 
$x$-periodic real-valued continuous functions
on $\square$.. Set}
\[
M_g(s)= \max_x\, |g(x,s)|\quad\colon \,B=\max_{x,s}\, |b(x,s)|
\]
\emph{Then one has the inequality}
\[
M_g(s)\leq M_g(0)\cdot e^{Bs}
\]


\noindent
\emph{Proof.}
Consider some $0<s<s^*$ and let $\epsilon>0$.
Put
\[ 
m^*(s)=\{ x\,\colon\, g(x,s)= M_g(s)\}
\]
The  continuity of $g$
entials that the function $M(s)$ is continuous and
the sets $m^*(s)$ are compact.
If $x^*\in m^*(s)$ the periodicity of
the
$C^1$-function $x\mapsto g(x,s)$
entails that
$\partial_x(x^*,s)=0$ and (*) gives
\[
\partial_s(g)(x,s)=b(x,s)g(x,s)\quad\colon x\in m^*(s)
\]
Next, let $\epsilon>0$. We find an open neighborhood $U$
of $m^*(s)$
such that
\[
|\partial_x(g)(x,s)|\leq \epsilon\quad\colon x\in U
\]
Now there exists
$\delta>0$ such that
\[
|g(x,s)|\leq M(s)-2\delta\quad\colon x\in [0,\pi]\setminus U
\]
Continuity gives  some
$\rho>0$ such that
if $0<\Delta s<\rho$ then  the inequalities below hold:
\[
|g(x,s+\Delta s)|\leq M(s)-\delta\quad\colon x\in [0,\pi]\setminus U
\quad\colon\,M(s+\Delta s)>M(s)-\delta\tag{i}
\]
\[
M(s+\Delta s)\leq M(s)+\epsilon\quad \colon\,
|\partial_x(g)(x,s+\Delta s)|\leq 2\epsilon \quad\colon x\in m^*(s)\tag{ii}
\]

\noindent
If $0<\Delta s<\rho$ we see that (i) gives
$x\in m^*(s+\Delta s)\subset U$
and for such $x$-values 
Rolle's mean-value theorem and the PDe-equation give
\[
M_g(x,s+\Delta s)- g(x,s)=\Delta s\cdot \partial_s(g(x,s+\theta\cdot \Delta s)=
\] 
\[
\Delta s\cdot \bigl[a(x,s+\Delta s)\cdot
\partial_x(g)(x+\theta\cdot \Delta s)+
b(x,s+\Delta s)\cdot
g(x,s+\theta\cdot \Delta s)\bigr]\tag{iii}
\]
Let  $A$ be the maximum norm of $|a(x,s)|$ taken over
$\square$.
Since $|g(x,s)|\leq M(s)$
the triangle inequality and (iii) give
\[ 
M(s+\Delta s)\leq M(s)+\Delta s[\cdot A\cdot 2\epsilon+
B\cdot M(s+\theta\cdot \Delta s)]
\]
Since the function $s\mapsto M(s)$ is continuous
it follows that
\[
\limsup_{\Delta s\to 0}\,
\frac{M(s+\Delta s)-M(s)}{\Delta s}\leq
A\cdot 2\epsilon+ BM(s)
\]
Above $\epsilon$ can be arbitrary small
and hence
\[ d^+(s)\leq B\cdot M(s)
\]
Then Proposition 1.1 gives (*) in the theorem.


\medskip

\noindent
{\bf{1.3 $L^2$-inequalities.}}
Let $g(x,s)$ be a $C^1$-function satisfying (*)
in
Theorem 1.2.
Set
\[ 
J_g(s)=\int_0^\pi\, g^2(x,s)\, dx
\]
Taking the $s$-derivative we obtain
 with respect to $s$ and (*) give
\[
\frac{dJ_g}{ds}= 2\cdot \int_0^\pi\, g\cdot \partial_s(g)\,ds=
2\cdot \int_0^\pi\, (a\partial_x(g)\cdot \partial g+ b\cdot g)\,dx
\]
The periodicity of $g$ with respect to $x$ gives
$\int_0^\pi\, \partial_x(ag^2)\, dx=0$. This
entails that the right hand side becomes
\[
\int_0^\pi\, (-\partial_x(a)+b)\cdot g^2\, dx
\]
So if $K$ is the maximum norm of
$-\partial_x(a)+b$ over $\square$ it follows that
\[
\frac{dJ_g}{ds}(s)\leq K\cdot J_g(s)
\]
Hence Theorem 1.2 gives
\[
\int_0^\pi\, g^2(x,s)\, dx\leq e^{Ks}\cdot
\int_0^\pi\, g^2(x,0)\, dx\quad\colon 0<s\leq s^*\tag{1.3.1}
\]
Integration with respect to $s$ entails that
\[ 
\iint_\square\, g^2(x,s)\, dxds\leq
\int_0^{s^*}\, e^{Ks}\,ds\cdot
\int_0^\pi\, g^2(x,0)\, dx\tag{1.3.2}
\]
Thus, the $L^2$-integral of $x\to g(x,0)$
majorizes both the area integral and each slice integral when
$0<s\leq s^*$.


\bigskip

\centerline{\bf{� 2. A boundary value equation}}
\bigskip

\noindent
Let $a(x,s)$ and $b(x,s)$ be real-valued $C^\infty$-functions on
$\square$ which are periodic in $x$. Consider the PDE-operator
\[ 
P=\partial_s-a\cdot \partial_x-b
\]
Given a periodic $C^1$-function $f(x)$ on $[0,\pi]$
we seek a periodic $C^1$-function $g(x,s)$ in $\square$ which satisfies $P(g)=0$ and the
initial condition
\[ 
g(x,0)= f(x)
\]
\medskip

\noindent
{\bf{2.1 Theorem.}}
\emph{For every positive integer $p$ and each
periodic $f\in C^p[0,\pi]$
there exists a unique periodic $g\in C^p(\square)$ 
where $P(g)=0$ and $g(x,0)= f(x)$.}

\medskip

\noindent
The uniqueness follows from the results in � 1.
For if $g$ and $h$ are solutions in Theorem 2.1
then $\phi=g-h$   satisfies $P(\phi)=0$. Here
$\phi(x,0)=0$ which gives $\phi=0$ in $\square$
via
(1.3.2).
The proof of existence requires several steps and 
employs Hilbert space methods. So first we
introduce
certain Hilbert spaces.

\medskip

\noindent
{\bf{2.2 The space $\mathcal H^{(k)}$}}.
To each  integer $k\geq 2 $ 
the complex Hilbert space
$\mathcal H^{(k)}$ defined is the
completion  
of complex-valued
$C^k$-functions on 
$\square$ which are periodic with respect to $x$.
Recall  the Soboloev inequality from � xx entails that 
every function in
$\mathcal H^{(2)}$ is continuous, and more generally
one has the inclusion
\[
\mathcal H^{(k)}\subset C^{k-2}(\square)\quad\colon k\geq 3
\]
It is also clear that the  first order PDE-operator $P$ maps
$\mathcal H^{(k+1)}$ into
$\mathcal H^{(k)}$.
\medskip

\noindent
Next,
on the periodic $x$-interval $[0,\pi]$ we get the Hilbert spaces
$\mathcal H^k[0,\pi]$ from � xx. 
\medskip

\noindent
{\bf{2.3 Definition.}}
\emph{For each integer $k\geq 2$
$\mathcal D_k(P)$  denotes the  family of all
$f(x)\in\mathcal H^k[0,\pi]$ for which 
there exists some
$F(x,s)\in\mathcal H^{(k)}$ such that}
\[
P(F)=0\quad\colon\, F(x,0)=f(x)\tag{*}
\]
The results in � 1 show that $F$ is uniquely determined by (*). Moreover.
there exists a constant $C$ which only depends upon the
$C^\infty$-functions $a$ and $b$ and the given integer $k$
such that
\[
||F||_k\leq C\cdot ||f||_k\tag{2.3.1}
\]
where we have taken norms in
$\mathcal H^{(k)}$ and 
$\mathcal H^k[0,\pi]$ respectively.
Moreover, the last inequality in (1.3.2)
shows that
$C$ can be chosen such that we also have
\[
||f^*||_k\leq C\cdot ||f||_k\tag{2.3.3}
\]
where $f^*(x)= F(x,s^*)$.

\medskip

\noindent
{\bf{2.4 A density principle}}
Above we introduced the space
$\mathcal D_k(P)$.
It turns out that if it is dense in
$\mathcal H^k[0,\pi]$ then one has the equality
\[
\mathcal D_k(P)=\mathcal H^k[0,\pi]\tag{2.4.1}
\]

\medskip

\noindent
\emph{Proof.}
Suppose that
$\mathcal D_k(P)$ is dense.
So if $f\in\mathcal H^k[0,\pi]$ there exists a sequence
$\{f_n\}$ in 
$\mathcal D_k(P)$ where $||f_n-g||_k\to 0$.
By (2.2.2)   we have
\[ 
||F_n-F_m||_k\leq C||f_n-f_m||_k
\]
Hence $\{F_n\}$ is a Cauchy sequence in
the Hilbert space $\mathcal H^{(k)}$
and converges to a limit $F$.
Since each $P(F_n)=0$ it follows that
$P(F)=0$
and it is clear that the continuous boundary value function
$F(x,0)$�is equal to
$f(x)$ which entails that
$f$ belongs to 
$\mathcal D_k(P)$.
\medskip

\noindent
{\bf{2.5 The operators $S_k$.}}
Each $f\in \mathcal D_k(P)$ gives the function  $f^*(x)= F(x,s^*)$
in $\mathcal H^k[0,\pi]$ and  set
\[ 
S_k(f)=f^*(x)
\]
So  the domain of definition of $S_k$ is equal to
$\mathcal D_k(P)$ and  (2.3.3) gives  a constant
$M_k$ such that
\[
||S_k(f)||\leq M_k\cdot ||f||_k\quad\colon f\in  \mathcal D_k(P)
\]
where $M_k$ by the above depends on the integer $k$ and the given
PDE-operator $P$.
\medskip

\noindent
{\bf{2.6 Proposition.}}
\emph{For each $k$ there exists some
$\alpha(k)<0$ such that for every $0<\alpha<\alpha(k)$
the range of the operator $E-\alpha\cdot S_k$ contains
all periodic $C^\infty$-functions on
$[0,\pi]$.}
\medskip

\noindent
{\bf{2.7 The density of $\mathcal D_k(P)$.}}
We prove Proposition 2.6 in � xx  and proceed to 
that it gives the density of
$\mathcal D_k(P)$.
For if $\mathcal D_k(P)$ fails to be dense there exists
a non-zero $f_0\in\mathcal D_k(P)$ which is
$\perp$ to $\mathcal D_k(P)$.
In Proposition 2.6 we choose $0<\alpha\leq \alpha(k)$ so small that
\[
\alpha<M_k/2\tag{i}
\]
Since periodic $C^\infty$-functions are dense in
$\mathcal H^k[0,\pi]$,
 Proposition 2.6 gives  a sequence
$\{h_n\}$ in $\mathcal D_k(P)$
such that
\[ 
\lim_{n\to \infty}\, ||h_n-\alpha\cdot S_k(h_n)-f_0||_k\to 0\tag{ii}
\]
It follows that
\[
\langle f_0,f_0\rangle=1=\lim\, 
\langle f_0,h_n-\alpha\cdot S_k(h_n)\rangle=
-\alpha\cdot \lim\, \langle f_0,S_k(h_n)\rangle\tag{iii}
\]
Next, the triangle inequality and (ii) give
\[
||h_n||_k\leq 1+\alpha\cdot ||(S_k(h_n)||
\leq 1+1/2\cdot||h_n||\implies
||h_n||_k\leq 2\tag{iv}
\]
Funally, by the Cauchy-Schwarz inequality the absolute value in
the right hand side of (iii) is majorized by
\[
\alpha\cdot M_K\cdot 2<1
\] 
which contradicts (iii).
Hence the orthogonal complement of $\mathcal D_k(P)$ is zero
which proves the requested density.
\medskip


\noindent
Together with (2.4) we get
the following conclusive result:
\medskip

\noindent
{\bf{2.8 Theorem.}}
\emph{For each  $k\geq 2$
and $f(x)\in \mathcal H^k[0,\pi]$
there exists a unique function
$F(x,s)\in \mathcal H^{(k)}$ such that (*) holds in
Definition 2.3.}





\bigskip

\centerline{\bf{� 3. A class of inhomogeneous PDE-equations.}}


\medskip

\noindent
Before Theorem 3.1 is announced we introduce some notations.
Put
\[ 
\square =\{ 0\leq x\leq \pi\}\times
\{0\leq s\leq 2\pi\}
\]
In this section we  shall consider doubly periodic functions
$g(x,s)$ on $\square$, i.e. 
\[
g(\pi,s)= g(0,s)\quad\colon g(x,0)= g(x,2\pi)
\]
For each non-negative integer
$k$
we denote by $C^k(\square)$ the space of $k$-times
doubly periodic continuously differentiable functions. 
If 
$g\in C^k(\square)$
we set
\[ 
||g||^2_{(k)}= \sum_{j,\nu}\, \int_\square
\bigl|\frac{\partial^{j+\nu}g}{\partial x^j\partial s^\nu}(x,s)\bigr|^2\, dxds
\]
with the double sum extended  pairs  $j+\nu\leq k$.
This gives the complex Hilbert space $\mathcal H^{(k)}$
after a completion of $C^k(\square)$ with respect
to the norm  above.
Recall from � xx that every function $g\in\mathcal H^{(2)}$
is automatically continuous and doubly periodic  on
the closed square.
More generally, if $k\geq 3$ 
each
$g\in\mathcal H^{(k)}$ has continuous and doubly periodic derivatives
up to order $k-2$.
Next, consider a first order PDE-operator
\[
P=\partial_s-a(x,s)\partial_x-b(x,s)
\]
where $a$ and�$b$ are real-valued doubly 
periodic $C^\infty$-functions.
It is clear that
$P$ maps
$\mathcal H^{(k)}$ into $\mathcal H^{(k+1)}$
for every $k\geq 2$.
Keeping $k\geq 2$ fixed we set

\[
\mathcal D_k(P)=\{g\in\mathcal H^{(k)}\,\colon\,
P(g)\in\mathcal H^{(k)}\}\tag{3.1}
\]
Since $C^\infty(\square)$
is dense in
$\mathcal H^{(k)}$ this yields for each $k\geq 2$ a densely defined operator
\[
P\colon \mathcal D_k(P)\to \mathcal H^{(k)}\tag{i}
\]
In
$\mathcal H^{(k)}\times \mathcal H^{(k)}$ we get the graph
\[
\Gamma_k=\{(g,P(g)\colon\, g\in\mathcal D_k(P)\}
\]
Since $P$ is a differential operator the general result in
� xx entails that
$\Gamma_k$ is a closed subspace so the densely defined operator in
(i)  has a closed graph. Thus. for each
$k\geq 2$ we have a densely defined linear operator
and closed operator on $\mathcal H^{(k)}$
denoted by $\mathcal T_k$.
So its domain of definition  $\mathcal D(T_k)= \mathcal D_k$.
Next, we consider the graph
\[
\gamma_*=\{(g,P(g)\colon\, g\in C^\infty(\square)\}\tag{ii}
\]
This is a subspace of
$\Gamma_k$ and 
denote by
$\overline{\gamma}_k$
its closure taken in  $\mathcal H^{(k)}\times \mathcal H^{(k)}$.
So here
\[
\overline{\gamma}_k\subset\Gamma_k
\]
and this inclusion yields 
another densely defined linear operator
denoted by
$T_k$ whose graph is
$\overline{\gamma}_k$. So here
$\mathcal T_k$ is an extension of
$T_k$ and we have an inclusion
\[
\mathcal D(T_k)\subset\mathcal D(\mathcal T_k)\tag{iii}
\] 
in general is strict.
Let $E$ be the identity operator
on
$\mathcal H^{(k)}$.
 With these notations one has

\medskip

\noindent 
{\bf{3.2  Theorem.}}
\emph{For each integer $k\geq 2$ there exists
a positive real number $\rho(k)$ such that
$T_k-\lambda\cdot E$ is surjective on
$\mathcal H^{(k)}$ for every $\lambda>\rho(k)$
and its kernel is zero.}
\medskip



\noindent
The proof requires several steps and is not finished until
� 3.x. First we establish the following:
\medskip

\noindent
{\bf{3.3  Proposition.}}
\emph{One has the equality $\mathcal D(T_k^*)= \mathcal D_k$
and there exists a bounded self-adjoint operator
$B_k$ on
 $\mathcal H^{(k)}$ such that}
 \[
 T_k^*=-\mathcal T_k+B_k
 \]
 

\medskip




\noindent{\emph{Proof of Proposition 3.3}}
Keeping $k\geq 2$ fixed we set $\mathcal H=\mathcal H^{(k)}$.
For each pair $g,f$ in $\mathcal H$
their inner product is defined by
\[
\langle f,g\rangle=
 \sum\, \int_\square
\frac{\partial^{j+\nu}f}{\partial x^j\partial s^\nu}(x,s)
\cdot
\overline{\frac{\partial^{j+\nu}g} {\partial x^j\partial s^\nu}}(x,s)
\, dxds
\]
where the sum is taken when
$j+\nu\leq k$.
Introduce the differential operator
\[
\Gamma= \sum_{j+\nu\leq  k}\, (-1)^{j+\nu}\cdot \partial_x^{2j}\cdot \partial_s^{2\nu}
\]
Partial integration gives
\[
\langle f,g\rangle=\int_\square\, f\cdot \Gamma(\bar g)\, dxds=
\int_\square\, \Gamma(f)\cdot\bar g\, dxds
\quad\colon\, f,g\in C^\infty
\tag{i}
\]
Now we consider the operator
$P=\partial_s-a\cdot\partial_x-b$ and 
get 
\[
\langle P(f),g\rangle=\int_\square\, P(f)\cdot \Gamma(\bar g)\,dxds\tag{ii}
\]
Partial integration identifies (ii) with
\[
-\int_\square\, f\cdot \bigl(\partial_s-\partial_x(a)-a\cdot\partial_x-b)\circ\Gamma(\bar g)\,dxds\tag{iii}
\]


\noindent
{\bf{1.1 Exercise.}}
In (iii)  appears the composed differential operator
\[
\partial_s-\partial_x(a)-a\cdot\partial_x-b)\circ\Gamma
\]
Show that in the ring of differential operators with
$C^\infty$-coefficients this differential operator can be written
in the form
\[
\Gamma\circ(\partial_s-a\cdot\partial _x-b)+Q(x,s,\partial_x,\partial_s)
\]
where $Q$ is a differential of order $\leq 2k$ 
with coefficients in $C^\infty(\square)$.
Conclude from the above that
\[
\langle Pf,g\rangle=-
\langle f,Pg\rangle+\int_\square\ f\cdot Q(\bar g)\,dxds\tag{1.1.1}
\]
\medskip

\noindent
{\bf{1.2 Exercise.}}
With $Q$ as above we have a bilinear form which sends a pair
$f,g$ in $C^\infty(\square)$ to 
\[
\int_\square\ f\cdot Q(\bar g)\,dxds\tag{1.2.1}
\]
Use partial integration and
the
Cauchy-Schwarz inequelity to show that
there exists a conatant $C$ which depends on $Q$ only such that
the absolute value of (1.2.1) is majorized by
$C_Q\cdot ||f||_k\cdot ||g||_k$.
Conclude that
there exists a bounded linear operator 
$B_k$ on $\mathcal H$ such that
\[
\langle f,B_k(g)\rangle=\int_\square\ f\cdot Q(\bar g)\,dxds\tag{1.2.2}
\]

\medskip

\noindent
{\bf{1.3 Proof that $B_k$ is self-adjoint}}
From the above we have
\[
\langle Pf,g\rangle=
-\langle f,Pg\rangle+
\langle f,B_k(g)\rangle\tag{1.3.1}
\]

\noindent
Keeping $f$ in $C^\infty(\square)$ we notice that
$\langle f,B_k(g)\rangle$ is defined for every
$g\in\mathcal H$.
From this the reader can check that (1.3.1) remains valid when
$g$ belongs to $\mathcal D(\mathcal T_k)$
which means that
\[
\langle Pf,g\rangle=
-\langle f,\mathcal T_kg\rangle+
\langle f,B_k(g)\rangle\quad\colon f\in C^\infty(\square)\tag{1.3.2}
\]

\medskip

\noindent
Moreover, when both $f$ and $g$ belong to $C^\infty(\square)$
we can
reverse their positions in (*) which gives
\[
\langle Pg,f\rangle=
-\langle g,Pf\rangle+
\langle g,B_k(f)\rangle\tag{1.3.3}
\]
Since $a$ and $b$ are real-valued it is clear that
\[
\langle Pg,f\rangle=-\langle f,Pg\rangle\tag{1.3.4}
\]
It follows that
\[
\langle f,B_k(g)=\langle g,B_k(f)\quad\colon f,g\in C^\infty(\square)\tag{1.3.5}
\]
Since this hold for all pairs of $C^\infty$-functions and
$B_k$ is a bounded linear operator on
$\mathcal H$ the density of $C^\infty(\square)$ entails that
$B_k$ is a bounded self-adjoint operator
on $\mathcal H$.



\medskip

\noindent
{\bf{1.4  The equality 
$\mathcal D(T_k^*)=
\mathcal D_k$.}}
The density of $C^\infty(\square)$ in $\mathcal H$
entails that  a function $g\in \mathcal H$ belongs to $\mathcal D(T_k^*)$  
if and only if there exists a constant $C$ such that
\[
|\langle Pf,g\rangle|\leq C\cdot ||f||\quad\colon f\in C^\infty(\square)\tag{1.4.1}
\]
Since $B_k$ is a bounded operator,
(1.3.2) gives the inclusion
\[
\mathcal D_k\subset
\mathcal D(T_k^*)\tag{1.3.3}
\]
To prove the opposite inclusion we use that
the $\Gamma$-operator is elliptic.
If $g\in \mathcal D(T_k^*)$
we have from (i) in � 1.1:
\[
\langle Pf,g\rangle=\langle f,T_k^*g\rangle=
\int\, \Gamma(f)\cdot \overline{T_k^*(g)}\, dxds
\quad\colon f\in C^\infty(\square)
\]
Similarly
\[
\langle f,B_k(g)\rangle=
\int\, \Gamma(f)\cdot \overline{B_k(g)}\, dxds
\]
Treating $\mathcal T_k(g)$ as a distribution the equation
(1.3.2)  entails that
the elliptic operator $\Gamma$ annihilates
$T^*_k(g)-\mathcal T_k(g)+B_k(g)$.
Since both
$T^*_k(g)$ and $B_k(g)$ belong to $\mathcal H$
this implies by the general result in � xx that
$\mathcal T_k(g)$  belongs to $\mathcal H$
which proves the 
requested equality (1.4) and at the same time the operator equation
\[
T_k^*=-\mathcal T_k(g)+B_k\tag{1.4.2}
\] 

\bigskip


\centerline {\bf{3.4  An inequality.}}
\medskip

\noindent
Let $f\in C^\infty(\square)$
and $\lambda$ is a positive real number.
Then
\[
||\mathcal T_k(f)-\frac{1}{2}B_k(f)-\lambda\cdot f||^2=
\]
\[
||\mathcal T_k(f)-\frac{1}{2}B_k(f)||^2+\lambda^2\cdot ||f||^2-
\lambda\bigl(\langle \mathcal T_k(f)-\frac{1}{2}B_k(f),f\rangle+
\langle f, \mathcal T_k(f)-\frac{1}{2}B_k(f)\rangle\bigr)
\]
The last term is $\lambda$ times
\[
\langle \mathcal T_k(f),f\rangle+
\langle f,\mathcal T_k(f)\rangle-\langle f,B_kf\rangle\tag{i}
\]
where we used that $B_k$ is symmetric.
Now $T_k=\mathcal T_k$ holds
on $C^\infty(\square)$ and the definition of adjoint
operators give
\[
\langle \mathcal T_k(f),f\rangle=
\langle f,T_k^*\rangle\tag{ii}
\]
Then (1.4.2 ) implies that (i) is zero and hence we have
proved 
\[
||T_k(f)-\frac{1}{2}B_k(f)-\lambda\cdot f||^2=
\lambda^2\cdot ||f||^2+||T_k(f)-\frac{1}{2}B_k(f)||^2\geq \lambda^2\cdot ||f||^2\tag{iii}
\]
From (iii)  and the triangle inequality for norms we obtain
\[
||T_k(f)-\lambda\cdot f||\geq \lambda\cdot ||f||-
\frac{1}{2}||B_k(f)||\tag{iv}
\]
Now $B_k$ has a finite operator norm and if
$\lambda\geq ||B_k||$ we see that 
\[
||T_k(f)-\lambda\cdot f||\geq
\frac{\lambda}{2}\cdot ||f||\tag{v}
\]


\noindent
Finally, since $C^\infty(\square)$ is dense in $\mathcal D(T_k)$
it is clear that (v) gives
\[
||T_k(f)-\lambda\cdot f||\geq
\frac{\lambda}{2}\cdot ||f||\quad\colon f\in\mathcal D(T_k)\tag{3.4.1}
\]
\medskip

\centerline{\emph{� 3.5. Proof of Theorem 3.2}}
\bigskip

\noindent
Suppose we have found some
$\lambda^*\geq \frac{1}{2}\cdot ||B||$
such that
$T_k-\lambda$  has a dense range in $\mathcal H$
for every $\lambda\geq \lambda^*$.
If this is so we fix $\lambda\geq \lambda^*$
 and take some
$g\in \mathcal H$. The hypothesis gives a
sequence $\{f_n\in \mathcal D(T_k)$ such that
\[
\lim_{n\to\infty}\,||T(f_n)-\lambda\cdot f_n-g||=0
\]
In particular $\{||T_k(f_n)-\lambda\cdot f_n\}$ is a Cauchy sequence 
in $\mathcal H$
and (1.5.x )  implies that $\{f_n\}$ is a Cauchy sequence 
in the Hilbert space $\mathcal H$
and hence converges to
a limit $f_*$. Since the operator $T_k$ is closed we conclude that
$f_*\in\mathcal D(T)$ and we get  the equality
\[
T(_kf_*)-\lambda\cdot f_*=g
\]
Finally, since the graph of $T$ is contained in $T_1$
we have the requested equation 
\[
P(f_*)-\lambda\dot f_*=g
\]
Thus finishes the proof of Theorem 3.2  provided we
have established the existence of $\lambda_*$ above.
\medskip

\noindent{\bf{3.5.1 Density of the range.}}
By the construction of adjoint operators the range of $T_k-\lambda\cdot E$
fails to be dense
if and ony if
$T_k^*-\lambda$ has a non-zero kernel.
So assume that
\[ 
T_k^*(f)-\lambda\cdot f=0\tag{i}
\]
for some $f\in\mathcal D(T_k^*)$ which is not identically zero.
Notice that $T_k$ sends real-valued functions into real-valued
functions. So above we can assume that
$f$ is real-valued and 
also  assume that
$f$ is normalised so that
\[
\int_\square f^2(x,s)\, dxds=1
\]
By  (**) the equation (xx) gives
\[
\mathcal T_k(f)+\lambda\cdot f-B(f)=0\tag{ii}
\]
Let us then consider
the function
\[ 
V(s)= \int_0^\pi\, f^2(x,s)\, dx
\]
Recall from � xx that the $\mathcal H$-function $f$ is of class
$C^1$.
Now
\[ 
\frac{1}{2}\cdot V'(s)=  \int_0^\pi\, f\cdot \frac{\partial f}{\partial s}\, dx\tag{iii}
\]
By (ii) we have
\[
 \frac{\partial f}{\partial s}-
a(x)\frac{\partial f}{\partial x}-b\cdot f=B(f)-\lambda\cdot f
\]
Hence the right hand side in (iii) becomes
\[
-\lambda\cdot V(s)+ \int_0^\pi\, f(x,s)\cdot B(f)(x,s)\,dx+
+ \int_0^\pi\,a(x,s)\cdot f(x,s)\cdot \frac{\partial f}{\partial x}(x,s)\,dx
\]
By partial integration the last term is equal to
\[
-\frac{1}{2}\int_0^\pi\, \partial_x(a)(x,s)\cdot f^2(x,s)\,dx
\]
Set
\[
M=\frac{1}{2}\cdot \max_{(x,s )\in\square}\,|\partial_x(a)(x,s)|
\]
Then we get the inequality
\[
\frac{1}{2}\cdot V'(s)\leq
(M-\lambda)\cdot V(s)+\int_0^\pi\, f(x,s)\cdot B(f)(x,s)\,dx
\]
Set
\[
 \Phi(s)=\int_0^\pi\, |f(x,s)|\cdot |B(f)(x,s)|\,dx
\]
Since the $L^2$-norm of $f$ is one
the Cauchy-Schwarz inequality 
gives
\[
\int_{-\pi}^\pi \Phi(s)\,ds \leq
\sqrt{\int_\square\,  |B(f)(x,s)|^2\, dx ds}\leq ||B(f)||
\]
where the last equality follows since
the  squared integral of $B(f)$ is majorized by its squared
norm in $\mathcal H$.
When $\lambda>M$ it follows from  (xx) that
\[
(\lambda-M)\cdot V(s)+
\frac{1}{2}\cdot V'(s)\leq\Phi(s)
\]
Next, since $f$ is double periodic we have  $V(-\pi)= V(\pi)$
so after an integration (xx) gives
\[
(\lambda-M)\cdot \int_\pi^\pi V(s)\, ds=\int_{-\pi}^\pi \Phi(s)\,ds \leq ||B(f)||
\]
By (xx) we have
$\int_\pi^\pi V(s)\, ds=1$ which gives  a contradiction if
$\lambda>M+||B(f)||$.
\medskip


\noindent
{\bf{Remark.}}
Set
\[ 
\tau=\min_f\,||B(f)||
\] 
with the minimum taken over
funtions $f\in\mathcal D(T_0^*)$ whose $L^2-$integral is normalised by
(xx). The proof has shown that
the kernel of $T_0^*-\lambda$ is zero for all $\lambda>M+\tau$.

\newpage


\centerline{\bf{A special solution.}}
\bigskip


\noindent
Let $f(x)$ be a periodic $C^\infty$-function 
on $[0,\pi]$.
Put
\[ 
Q= a(x,s)\cdot \frac{\partial}{\partial x}+ b(x,s)
\]
Let  $\eta(s)$ be a $C^\infty$-function of $s$ and 
$m$ some  positive integer
If $\lambda>0 $ is a real number.
we set
\[ 
g_\lambda(x,s)=\eta(s)\cdot f+\eta(s)\cdot\sum_{j=1}^{j=m}\,
\frac{(s-\pi)^j}{j!}\cdot (Q-\lambda)^j(f)\quad\colon 0\leq s\leq \pi\tag{i}
\]
We choose $\eta$ to be a real-valued $C^\infty$-function
such that
$\eta(s)=0$ when $s\leq 1/4$ and -1 if $s\geq 1/2$.
Hence $g_\lambda(x,s)=0$ in (i) when
$0\leq  s\leq 1/4$
and we extend the function to $[-\pi\leq s\leq \pi$ where
$g_\lambda(x,-s)= g_\lambda(x,s)$ if $0\leq s\leq \pi$.
So now $g_\lambda$ is $\pi$-periodic with respect to $s$
and  vanishes when
$|s|\leq 1/4$.
\medskip

\noindent
{\bf{Exercise.}}
If $1/2\leq s\leq \pi$ we have $\eta(s)=1$. Use (i) to show that 
\[
(P+\lambda)(g_\lambda)=
\frac{\partial g_\lambda}{\partial s}
-(Q-\lambda)(g_\lambda)=\frac{(s-\pi)^m}{m!}\cdot (Q-\lambda)^{m+1}(f)
\]
hold when
$1/2\leq s\leq \pi$.
At the same time
$g_\lambda(s)=0$ when $0\leq s\leq 1/4$.
So $(P+\lambda)(g)$
is a function whose derivatives
with respect to $s$ vasnish up to order
$m$ at $s=0$ and $s=\pi$ and is therefore
doubly periodic of class $C^m$ in
$\square$.
Now Theorem 2.2  applies. For a given $k\geq 2$
we choose a sufficently large $m$
and find $h(x,s)$ so that  
\[ 
P(h)+\lambda\cdot h=(P+\lambda)( g_\lambda)(x,s)
\]
where $h$ is $s$-periodic, i.e.
\[ 
h(x,0)= h(x,\pi)
\]
Notice also that
$g_\lambda(x,0)=0$ while $g_\lambda(x,\pi)= f(x)$.
Set
\[ 
g_*(x)=h-g_\lambda
\]
Then  $P(g_*)+\lambda\cdot g_*=0$
and 
\[ 
g_*(x,0)-g_*(x,\pi)=f(x)
\]
\medskip

\noindent
Above we started with the $C^\infty$-function.
Given $k\geq 2$
we can take
$m$ sufficiently large during the constructions above
so that $g_*$ belongs to
$\mathcal H^{(k)}(\square)$.






\end{document}
































 

 















































 

 























