\documentclass{amsart}

\usepackage[applemac]{inputenc}

\addtolength{\hoffset}{-12mm}
\addtolength{\textwidth}{22mm}
\addtolength{\voffset}{-10mm}
\addtolength{\textheight}{20mm}

\def\uuu{_}

\def\vvv{-}

\begin{document}







\newpage

\centerline{\bf {Hilbert spaces.}}
\bigskip

\noindent
{\bf Introduction.}
Euclidian geometry teaches   that if
$A$ is some  invertible $n\times n$-matrix whose elements are real
numbers and $A$ is regarded  as a linear map from
${\bf{R}}^n$ into itself, then the image of the euclidian unit sphere
$S^{n-1}$ is an ellipsoid $\mathcal E_A$. Conversely  if
$\mathcal {E}$ is an ellipsoid   there exists an invertible matrix
$A$ such that $\mathcal{E}=\mathcal{E_A}$.
\medskip

\noindent
{\bf {The case $n=2$}}. 
Let $(x,y)$ be the coordinates in
${\bf{R}}^2$ and  $A$  the linear map
\[ 
(x,y)\mapsto  (x+y,y)\tag{0.1}
\]
To get the image of the unit circle  $x^2+y^2=1$ 
we use polar coordinates and write
$x=\text{cos}\,\phi$ and $y=\text{sin}\,\phi$.
This gives the closed image curve
\[ 
\phi\mapsto (\text{cos}\phi+\text{sin}\phi\,;\,\text{sin}\phi)
\quad\colon|\quad 0\leq\phi\leq 2\pi\tag{i}
\]
It is not obvious  how to  determine the principal axes of
this ellipse. The gateway is to consider the
\emph{symmetric} $2\times 2$-matrix $B=A^*A$.
If $u,v$ is a pair of vectors in ${\bf{R}}^2$ we have
\[
\langle Bu,v\rangle=\langle Au,Av\rangle\tag{ii}
\]
It follows that $\langle Bu,u\rangle>0$ for all $u\neq 0$. By
a wellknown result in elementary geometry it means that the symmetric matrix
$B$ is positive, i.e. the eigenvalues arising from zeros of
the characteristic polynomial $\text{det}(\lambda E_2-B)$
are both positive.
Moreover, the {\emph{spectral theorem} for symmetric matrices
shows that there exists an orthonormal basis in
${\bf{R}}^2$ given by a pair of eigenvectors for $B$ denoted by
$u_*$ and $v_*$. So here

\[ 
B(u_*)=\lambda_1\cdot u_*\quad\colon\quad
B(v_*)=\lambda_2\cdot v_*
\]
Next, since $(u_*,v_*)$ is an orthonormal basis in
${\bf{R}}^2$ points on the unit circle are
of the form
\[
\xi=
\text{cos}\phi\cdot u_*+
\text{sin}\phi\cdot v_*
\]
Then we get
\[
|A(\xi)|^2=
\langle A(\xi).A(\xi)\rangle=\langle B(\xi),\xi)=
\text{cos}^2\phi\cdot \lambda_1+
\text{sin}^2\phi\cdot \lambda_2
\]
From this we see that the ellipse $\mathcal E_A$ has $u_*$ and
$v_*$ as principal axes. It is a circle if and only if
$\lambda_1=\lambda_2$. If $\lambda_1>\lambda_2$
the largest principal axis has length $2\sqrt{\lambda_1}$ 
and the smallest has length $2\sqrt{\lambda_2}$. The reader should
now compute the specific example (*) and find
$\mathcal E_A$. 
\medskip

\noindent
{\bf 0.2 A Historic Remark.}
The fact that $\mathcal E_A$ is an ellipsoid was  wellknown in
the Ancient Greek mathematics when $n=2$ and $n=3$.
 After  general matrices and their determinants were introduced,
the spectral theorem for symmetric matrices was 
established by A. Cauchy in
1810 under the assumption that the eigenvalues are different. Later
Weierstrass gave the proof in the general case
where multiple eigenvalues appear,
and independently Gram
and Weierstrass  found a method to
produce an  
orthonormal basis of eigenvectors for
a  symmetric $n\times n$-matrix $B$. To find an  eigenvector with
largest eigenvalue one studies the extremal problem
\[ 
\max_x\,\langle Bx,x\rangle\quad\colon\quad||x||=1\tag{1}
\]
If a unit vector $x_*$ maximises
(1) then it is an eigenvector, i.e.
\[ 
Bx_*=a_1x_*
\] 
holds for a real number $a$.
In the next stage one takes the orthogonal complement $x_*^\perp$
and proceed to  the restricted extremal problem
where $x$ say in this orthogonal complement which gives  an eigenvector
whose eigenvalue $a_2\leq a_1$. After $n$ steps we obtain an
$n$-tuple of pairwise orthogonal eigenvectors to $B$.
In the orthonormal basis given by this $n$-tuple the linear operator of $B$ is
represented by a diagonal matrix.
One often refers to this as the Gram-Schmidt construction of an orthonormal basis
with respect to $B$.
\medskip

\noindent
{\bf{Singular values.}}
\emph{Mathematica} has implemented programs which for every
invertible $n\times n$-matrix $A$ determines the
ellipsoid $\mathcal E_A$ numerically. This is presented under the headline
\emph{singular values for matrices}. In general
the $A$-matrix  is not symmetric but  the spectral theorem
is applied to the symmetric matrix $A^*A$ 
which  determines
the ellipsoid $\mathcal E_A$ and whose principal axis
are pairwise disjoint.


\bigskip

\noindent{\bf 0.3 Rotating bodies.}
The spectral theorem in dimension $n=3$ is
best illustrated by regarding a rotating body.
Consider a bounded
3-dimensional body
$K$ in which some distribution of mass is given.
The body is placed i ${\bf{R}}^3$ where
$(x_1,x_2,x_3)$ are the coordinates and the distribution of mass
is expressed by a positive function
$\rho(x,y,z)$ defined in $K$.
The \emph{center of gravity} in $K$ is the point
$(\bar x_1,\bar x_2,\bar x_3)$ where
\[ 
\bar x_\nu=\iiint_K\,x_\nu\cdot \rho(x_1,x_2,x_3)\cdot dx_1dx_2dx_3\tag{i}
\quad\colon\, 1\leq\nu\leq 3
\]
After a translation we may assume that 
the center of mass is the
origin.
Now we imagine that  a rigid bar
which stays on a  line $\ell$ is attached to $K$ with its two endpoints 
$p$ and $q$, i.e. if $\gamma$ is the unit vector
in ${\bf{R}}^3$ which determines the line
then
\[
p= A\cdot\gamma\quad\colon\quad q=-A\cdot\gamma
\]
where $A$ is so large that
$p$ and $q$ are outside $K$.
The mechanical experiment is to rotate
around $\ell$ with some constant
angular velocity $\omega$ while
the two points $p$ and $q$ are kept fixed.
The question arises if such an imposed rotation
of $K$ around $\ell$
implies that  external forces
at $p$ and $q$ are needed to prevent these to points from moving.
It turns out that there exist
so called free axes where no such forces are needed, i.e. for certain directions of
$\ell$ the body rotates nicely around
the axis with constant angular velocity.
The free axes are found from the spectral theorem.
More precisely, one introduces the symmetric 
$3\times 3$-matrix $A$ whose elements are
\[
a_{pq}=
\bar x_\nu=\iiint_K\,x_p\cdot x_q\cdot \rho(x_1,x_2,x_3)\cdot dx_1dx_2dx_3\tag{i}
\]
Using the expression for the centrifugal force
by C. Huyghen's one has the  \emph{Law of Momentum}
which in the present case shows that
the body has a free rotation along the
lines which correspond to eigenvectors of
the symmetric matrix $A$ above.
In view of the historic importance of this example we present the proof of this
in a separate section even though 
some readers may
refer to this as a subject in classical mechanics rather
than linear algebra.
Hence   the spectral theorem becomes evident by
this mechanical experiment, i.e. just as Stokes Theorem 
the spectral theorem  for symmetric matrices
is rather a Law
of Nature than a mathematical discovery.



\bigskip
\centerline {\bf 0.4 Inner product norms}
\medskip

\noindent
Let $A$ be an invertible $n\times n$-matrix.
The ellipsoid $\mathcal{E}_A$ defines a norm on
${\bf{R}}^n$ by the general construction in XX.
This norm has a special property. For if $B=A^*A$
and $x,y$ is a pair of $n$-vectors, then
\[
||x+y||^2=\langle B(x+y),B(x+y)\rangle
=||x||^2+||y||^2+2\cdot B(x,y)\tag{i}
\]
It means that the map
\[ (x,y)\mapsto 
||x+y||^2-||x||^2-||y||^2\tag{ii}
\]
is linear both with respect to $x$ and to $y$, i.e. it is a
bilinear map given by
\[
(x,y)\mapsto 2\cdot B(x,y)\tag{iii}
\]
We leave as an exercise for the reader to prove that if
$K$ is a symmetric convex set in
${\bf{R}}^n$
defining the $\rho_K$-norm as in xx, then this norm
satisfies the bi-linearity (ii) if and only if
$K$ is an ellipsoid and therefore  equal to  $\mathcal{E}_A$
for an invertible $n\times n$-matrix $A$.
Following Hilbert we refer to a norm
defined by some bilinear form $B(x,y)$ as an
\emph{inner product norm.}
The spectral theorem asserts that there exists
an orthonormal basis in ${\bf{R}}^n$ with respect to this norm.
\medskip

\noindent
{\bf 0.5 The complex case.}
Consider a  Hermitian matrix
$A$,  i.e an $n\times n$-matrix with complex elements
satisfying
\[ 
a_{qp}=\bar a_{pq}\quad\colon\quad 1\leq p,q\leq n\tag{*}
\]
Consider the $n$-dimensional complex vector space
${\bf{C}}^n$
with the basis $e_1,\ldots,e_n$.
An inner product is defined by
\[
\langle x,y\rangle=x_1\bar y_1+\ldots+x_n\bar y_n\tag{**}
\]
where 
$x_\bullet=\sum x_\nu\cdot e_\nu$
and
$y_\bullet=\sum y_\nu\cdot e_\nu$
is a pair of complex $n$-vectors.
If $A$ as above is a Hermitian matrix we obtain
\[
\langle Ax,y\rangle=\sum \sum a_{pq}x_q\cdot \bar y_p
\sum \sum \, x_p\cdot \bar a_{qp}\bar y_q=
\langle x,Ay\rangle\tag{***}
\]
\medskip

\noindent
Let us consider the characteristic polynomial
$\text{det}(\lambda\cdot E_n-A)$. If
$\lambda$ is a root there exists a non-zero eigenvector
$x$ such that
$Ax=\lambda\cdot x$. Now (***) entails that
\[
\lambda\cdot ||x||^2=
\langle Ax,x\rangle=
\langle x, Ax\rangle=\bar \lambda\cdot ||x||^2
\]
It follows that $\lambda$ is \emph{real}, i.e. the roots of the
characteristic polynomial of a Hermitian matrix are always
real numbers.
If all  roots are $>0$ one say that
the Hermitiain matrix is \emph{positive}.
\medskip

\noindent
\emph {0.6 Unitary matrices.}
An $n\times n$-matrix $U$ is called unitary if
\[
\langle Ux,Ux\rangle=\langle x,x\rangle
\]
hold for all $x\in{\bf{C}}^n$.
The spectral theorem for Hermitian  matrices asserts that if
$A$ is Hermitian then there exists a unitary
matrix $U$ such that
\[ 
UAU^*=\Lambda
\] 
where $\Lambda$ is a diagonal matrix whose elements are real.




\bigskip

\centerline{\bf  0.7 The passage to infinite dimension.}
\medskip

\noindent
Around 1900 the need for a spectral theorem in infinite dimensions
became  urgent.
In his article \emph{Sur une nouvelle m�thode pour la resolution
du probl�me de Dirichlet} from 1900,
Ivar Fredholm extended earlier construction by Volterra and
studied  
systems of linear equations in an infinite number
of variables with certain bounds.
In Fredholm's investiagations one starts
with a sequence of matrices $A_1,A_2,\ldots$ where
$A_n$ is an $n\times n$-matrix and an infinite dimensional vector space
\[ 
V=
{\bf{R}}e_1+
{\bf{R}}e_2+\ldots
\]
To each $N\geq 1$ we get the finite dimensional subspace
$V_N={\bf{R}}e_1+\ldots
{\bf{R}}e_N$.
Now $A_N$ is regarded as a linear operator on $V_N$ and we assume that
the $A$-sequence is matching , i.e. if $M>N$ then
the restriction of $A_M$ to $V_N$ is equal to $A_N$.
This means   that we take  any infinite matrix $A_\infty$ with elements 
$\{a_{ik}\}$ and here $A_N$ is the $N\times N$-matrix
which appears as an upper block with $N^2$-elements
$a_{ik}\,\colon\,1\leq i,k\leq N$.
To each $N$ 
the ellipsoid $\mathcal {E}_N=\mathcal{E}_{A_N}$ on $V_N$ where 
defines a norm.
As $N$ increases the norms are matching
and hence  $V$ is equipped with a norm
which for every $N\geq 1$ restricts to the norm defined by $\mathcal E_N$
on the finite dimensional subspace $V_N$. Notice that
the norm of any vector $\xi\in V$
is finite  since $\xi$ belongs to $V_N$ for some $N$, i.e. by
definition any vector in $V$ is a finite
${\bf{R}}$-linear combination of the basis vectors
$\{e_\nu\}$.
Moreover, the norm on $V$ satisfies the
bilinear rule from (0.3), i.e. on $V\times V$
there exists a bilinear form $B$ such that
\[ 
||x+y||^2-||x||^2-||y||^2=2B(x,y)\quad\colon\quad
x,y\in V\tag{*}
\]
\medskip

\noindent
{\bf Remark.}
Inequalities for determinants
due to Hadamard play an important role in
Fredholm's work and since the Hadamard inequalities are used
in many other situations we 
announce some of his results, leaving proofs as an exercise or
consult the literature where an  excellent source is the introduction to integral equations by
the former professor at Harvard University
Maxime Bochner [Cambridge University Press: 1914):
\medskip

\noindent
{\bf 0.8 Two inequalities.} Let
$n\geq 2$ and $A=\{a_{ij}\}$  some
$n\times n$-matrix whose elements are real numbers.
Show that if
\[ a_{i1}^2+\ldots+a_{in}^2=1\quad\colon\quad 1\leq i\leq n
\]
then the determinant of $A$ has absolute value $\leq 1$.
Next, assume that there is a constant $M$ such that the absolute values
$|a_{ij|}|\leq M$ 
hold for all pairs $i,j$. Show that this gives
\[ 
\bigl|\text{det}(A)\bigr |\leq \sqrt{n^n}\cdot M^n
\]
 



\medskip

\noindent
{\bf 0.9 The Hilbert space $\mathcal{H}_V$}.
This is the completition of the normed space $V$.
That is, exactly as when the field of rational numbers is
completed to the real number system one regards Cauchy sequences for the norm
of vectors in $V$ and in this way we get  a  normed
vector space denoted by $\mathcal{H}_V$ where the norm topology is complete.
Under this process the bi-linearity is preserved, i.e.
on $\mathcal{H}_V$ there exists  a bilinear form $B_{\mathcal{H}}$ such that
(*) above holds for pairs $x,y\in\mathcal {H}_V$.
Following Hilbert we refer to
$B_{\mathcal{H}}$ as the  \emph{inner product}
attached to the norm.
Having performed this construction starting from any
infinite matrix $A_\infty$ it is tempting to make a further abstraction. 
This is precisely
what Hilbert did, i.e. he ignored the "source" of a matrix $A_\infty$
and defined a complete normed vector space over
${\bf{R}}$ to be  a real Hilbert space if the there exists a bilinear form
$B$ on $V\times V$ such that (*) holds. 
\medskip

\noindent
{\bf {Remark.}}
If $V$ is a "abstract"  Hilbert space the restriction of the norm to
any finite dimensional subspace $W$ is determined by
an ellipsoid and  exactly as in linear algebra  one constructs an orthonormal basis
on $W$.
By the Gram-Schmidt 
construction there exists
an orthonormal sequence $\{e_n\}$ in $V$.
However, in order to be sure that it suffices to take a
\emph{denumerable} orthonormal basis it is necessary and sufficient that
the normed space $V$ is \emph{separable}.
Assuming this  every $v\in V$ has a unique
representation
\[ 
v=\sum\, c_n\cdot e_n\quad\colon\, \sum\, |c_n|^2=||v||^2\tag{i}
\]
\medskip

\noindent
The existence of this orthonormal family  means that every
separable Hilbert space is isomorphic to
the standard space $\ell^2$ whose vectors are infinite sequences
$\{c_n\}$ where the square sum $\sum\, c_n^2<\infty$.
So in order to prove general results about separable Hilbert spaces
it is  sufficient to regard
$\ell^2$. However,  the abstract notion of a Hilbert space
is useful since inner products on 
specific linear spaces
appear in many different situations. 
For example, in complex analysis  one
regards the space
of analytic functions which are square 
integrable in a bounded domain or
whose boundary values are square integrable.
Here  the inner product
is given in advance but it can be  a highly non-trivial affair to exhibit
an orthonormal basis.

\medskip

\noindent
{\bf 0.10 Linear operators on $\ell^2$.}
A bounded linear operator
$T$ from the complex Hilbert space $\ell^2$ into itself is
described by an infinite matrix
$\{a_{p,q}\}$ whose elements are complex numbers. Namely, 
for each $p\geq 1$ we set
\[ T(e_p)=\sum_{q=1}^\infty\, a_{pq}\cdot e_q\tag{i}
\]
For each fixed $p$ we get
\[
||T(e_p)||^2=\sum_{q=1}^\infty\, |a_{pq}[^2\tag{ii}
\]


\noindent
Next,
let $x=\sum\,\alpha_\nu\cdot e_\nu$
and $y=\sum\,\beta_\nu\cdot e_\nu$ be two vectors in
$\ell^2$. Then we get
\[
||x+y||^2=\sum\, |\alpha_\nu+\beta_\nu|^2\cdot e_\nu
\]
For each $\nu$ we have the pair of complex numbers
$\alpha_\nu,\beta_\nu$ and here
we have the inequality
\[
|\alpha_\nu+\beta_\nu|^2\leq  2\cdot |\alpha_\nu|^2+2\cdot |\beta_\nu|^2
\]
It follows that
\[
||x+y||^2\leq 2\cdot ||x||^2+2\cdot||y||^2\tag{iii}
\]


\noindent
In (iii) equality holds if and only if the
two vectors $x$ and $y$ are linearly dependent, i.e. if there
exists some complex number $\lambda$ such that
$y=\lambda\cdot x$.
Let us now return to the linear operator $T$.
In (ii) we get an expression for
the norm of the $T$-images of the orthonormal basis
vectors. So when $T$ is bounded with  operator norm
$M$ then the sum of the squared absolute values in each row
of the matrix $A=\{a_{p,q}\}$ is $\leq M^2$.
However, this condition along is not sufficient to
guarantee that
$T$ is a bounded linear operator. For example, suppose that
the row vectors in $T$ are all equal to a given
vector in $\ell^2$, i.e. $a_{p,q}=\alpha_q$ hold for all pairs where
$\sum,|\alpha_q|^2=1$.
Then
\[ 
T(e_1+\ldots+e_N)=N\cdot v\quad\colon\quad
v=\sum\,\alpha_q\cdot e_q
\]
The norm in the right hand side is $N$ while the norm
of $e-1+\ldots+e_n$ is $\sqrt{N}$. Since
$N>>\sqrt{N}$ when $N$ increases this shows that
$T$ cannot be bounded. So the condition on
the matrix $A$ in order that
$T$ is bounded is more subtle.
In fact, given a vector $x=\sum\,\alpha_\nu\cdot e_\nu$ as above with
$||x||=1$ we have
\[ 
||T(x)||^2=\sum_{p=1}^\infty\, 
\sum_q\sum_k\, a_{p,q}\cdot\alpha_q\cdot\bar a_{pk}\cdot\bar \alpha_k\tag{*}
\]


\noindent
So we encounter an involved triple sum.
Notice also that for each fixed $p$ we get a \emph{non-negative}
term 
\[ 
\rho_p=
\sum_q\sum_k\, a_{p,q}\cdot\alpha_q\cdot\bar a_{pk}\cdot\bar \alpha_k=
\bigl| \sum_{q=1}^\infty\, a_{pq}\cdot\alpha_q\,\bigr|^2
\]
\medskip

\noindent
{\bf Final remark.}
Thus, the description of the Banach space
$L(\ell^2,\ell^2)$ of all bounded
linear operators on
$\ell^2$ is not easy to grasp. In fact, no
"comprehensible" description exists of this space.


\newpage



\bigskip

\centerline {\bf { 1.  General  results about Hilbert spaces.}}
\medskip

\noindent
Let $\mathcal H$  be  a real Hilbert space.
The construction of the inner product norm entails  that
\[
||x+y||^2+||x-y||^2=
2\cdot ||x||^2+2\cdot ||y||^2\tag{*}
\]
for every pair $x,y$ in $\mathcal H$.
Using this one solves an extremal problem.
For every  closed convex subset $K$ of
$\mathcal H$ and if $\xi\in \mathcal H\setminus K$  there exists a unique
$k_*\in K$ such that
\[
\min_{k\in K}\,||\xi-k||=||\xi-k_*||\tag{**}
\]
To prove (**) 
we let
$\rho$ denote  the minimal distance.
We find a sequence $\{k_n\}$ in $K$ such that
$||\xi-k_n||\to\rho$. Now we show that 
$\{k_n\}$ is a Cauchy sequence. For let 
$\epsilon>0$ which gives some integer  $N_*$ such that
\[
||\xi-k_n||<\rho+\epsilon\quad\colon\quad n\geq N_*\tag{i}
\]
The convexity of $K$ implies that
if $n,m\geq N_*$ then $\frac{k_n+k_m}{2}\in K$. 
Hence 
\[
\rho^2\leq ||\xi-\frac{k_n+k_m}{2}||^2\implies
4\rho^2\leq ||(\xi-k_n)+(\xi-k_m]||^2\tag{ii}
\]
By the identity (*) the right hand side is
\[
2||\xi-k_n||^2+2||\xi-k_m||^2-||k_n-k_m||^2\tag{iii}
\]
It follows from (i-iii) that
\[
||k_n-k_m||^2\leq 4(\rho+\epsilon)^2-4\rho^2=
8\rho\cdot\epsilon+4\epsilon^2
\]
Since $\epsilon$ can be made arbitrary small
$\{k_n\}$ is a Cauchy sequence and hence 
there exists a limit $k_n\to k_*$ where $k_*\in K$
since $K$ is closed.
Finally, the  uniqueness of $k_*$ 
follows from the equality
\[
||\xi-k_1||^2+||\xi-k_2||^2=
2\cdot ||\xi-\frac{k_1+k_2}{2}||^2+\frac{1}{2}\cdot ||k_1-k_2||^2
\]
for every pair $k_1,k_2$ in $K$. In fact, this equality entails that if
$\epsilon>0$ and 
$k_1,k_2$ is a pair such that
\[
||\xi-k_\nu||\leq \rho^2+\epsilon\quad\colon\, \nu=1,2
\]
then we have
\[
||k_1-k_2||^2\leq 4\epsilon
\]
from which the uniquness of $k_*$ follows.

\bigskip

\noindent
{\bf 1.1 The decomposition theorem.}
Let $V$ be a closed subspace of $H$.
Its orthogonal complement is defined by
\[ 
V^\perp=\{x\in H\quad\colon\,\langle x,V\rangle=0\}\tag{i}
\]
It is obvious that $V^\perp$ is a closed subspace of $H$
and that $V\cap V^\perp=0$.
There remains to prove the equality
\[ 
H=V\oplus\, V^\perp\tag{ii}
\]
To see this we take some $\xi\in H\setminus V$. 
Now $V$ is a closed convex set so we find $v_*$ such that
\[ 
\rho=||\xi-v_*||=\min_{v\in V}\,||\xi-v||\tag{iii}
\]
If we prove that $\xi-v_*\in V^\perp$ we get (ii).
To show this we consider  some $\eta\in V$. If
$\epsilon>0$ we have
\[
\rho^2\leq ||\xi-v_*+\epsilon\cdot\eta||^2=
||\xi-v_*||^2+\epsilon^2 \cdot||\eta||^2+
\epsilon\langle \xi-v_*,\eta\rangle
\]
Since $||\xi-v_*||^2=\rho^2$ and $\epsilon>0$ it follows that
\[
\langle \xi-v_*,\eta\rangle+\epsilon \cdot||\eta||^2\geq 0
\]
here $\epsilon$ can be arbitrary small and we conclude that
$\langle \xi-v_*,\eta\rangle\geq 0$. Using 
$-\eta$ instead we get the 
opposed inequality and hence
$\langle \xi-v_*,\eta\rangle=0$ as required.

\newpage

\noindent
{\bf {1.2 Complex Hilbert spaces.}}
On a complex vector space  similar results as above hold
provided that 
we regard convex sets which
are ${\bf{C}}$-invariant. 
Here the inner product is hermitian. It means that
\[
\langle x,y\rangle=
\overline{\langle y,x\rangle}
\] hold for every pair of vectors.
When $y$ is fixed
\[
x\mapsto \langle x,y\rangle
\]
is ${\bf{C}}$-linear. By the construction of the norm and
the Cacuhy-Schweartz inequality one gets
\[
|\langle x,y\rangle|\leq ||x||\cdot ||y||
\]
In particular (xx) gives a linear funtional on $\mathcal H$
of norm
$\leq ||y||$ and by taking $x=y$ we see that the norm is equal to
$||y||$.
\medskip


\noindent
{\bf{1.4 Hilbert spsces are self-dual.}}
A fundamental fact is that
(x) exhibit vectors in the dual space
$\mathcal H^*$.
To see this we consider some
$\phi\in \mathcal H^*$ whose null-space is a complex hyperplane 
$\text{ker}(\phi$.
The complex verion of (1.1) whose verification is left as an exercise gives
\[
\mathcal H= \text{ker}(\phi)\oplus\, {\bf{C}}\cdot \xi
\]
for some vector $\xi$ which belongs to
$\text{ker}(\phi)^\perp$.
Now the reader can check that
there exists a complex number $c$ such that
\[
\phi(x)= c\cdot \langle x,\xi\rangle
\]
for every $x\in\mathcal H$.
The conclusion is that
one has a bijective  map from $\mathcal H$ to $\mathcal H^*$
where every $\in\in \mathcal H$ gives the linear functional
\[
x\mapsto \langle x,\xi\rangle
\]
\medskip


\noindent
In particular  Hilbert spaces are reflexive Banach spaces.


\medskip

\noindent
{\bf{1.6 Self-adjoint prjections.}}
They consist of linear operator $\Pi$ on the complex Hlbert space
which satisfy
\[
\Pi^2=\Pi\quad\&\quad \Pi=\Pi^*
\]
For every such operator we get a decompsition
\[
\mathcal H=\Pi(\mathcal H)\oplus \text{Ker}(\Pi)
\]
So the range and the kenrle of $\Pi$ are orthogonal subspace of
$\mathcal H$ snd thier direct sum is the whole space.
More genwerally one can consider a finite
srt
ot self-adjoint projections
$\{\Pi_\nu\}$ where
$\Pi_\nu\circ \Pi_k=0$ when $k\neq \nu$
while $\sum\, Pi_\nu(\mathcal H)= \mathcal H$.
This gives a decompdiion
\[
\mathcal H=\Pi_1(\mathcal H)\oplus\ldots\oplus
\Pi_k(\mathcal H)
\]



\bigskip



\noindent {\bf{1.x Hilbert-Schmidt operators.}}
We restrict the study to separable Hilbert spaces and 
assume that
$\mathcal H=\ell^2$.
A linear operator $S$ on $\ell^2$ 
gives a doubly indexed sequence of complex numbers
$\{a_{pq}\}$
where
\[
S(e_p)= \sum_{q=1}^\infty\ a_{pq}\cdot e_q\tag{i}
\]
Impose the condition
\[
\sum\sum\, |a_{pq}|^2<\infty\tag{*}
\]
If (*) holds one says that $S$ is an operator of the Hilbert Schmidt
type.
Every such operator is bounded. For let
$x=\sum\, x_p\cdot e_p$ be a vector of unit norm in
$\ell^2$, i.e. $\sum\, |x_p|^2=1$.
For each integer $q\geq 1$ we set
\[
\rho_q= \sum\, a_{qp}\cdot x_p
\]
Then (i) gives
\[
Sx=\sum\, \rho_q\cdot e_q\implies
||Sx||^2= \sum_q\, |\rho_q|^2\tag{ii}
\]
Apply the 
Cauchy-Schwarz inequality
and sibnce $x$ is a unit vector the last term 
in (ii) is majorised by (*).
In partocuoar the operator norm of $S$ satisfies
\[
||S||\leq \sqrt{\sum\sum\, |a_{pq}|^2}\tag{iii}
\]
One refers to the right hand side as the Hilbert-Schmidt norm of
$S$. Examples show that the inequality above in generla is strict.




\bigskip

\noindent
{\bf{Exercise.}}
Let $N$ be a positive integer and consider
the opertor $S_N$ respresented by the truncated matrix where
we keep $a_{pq}$ for pairs $1\leq p,q\leq N$ and otherwise
all matrix elements are zero.
Sow that (*) entails that
\[
\{lim_{N\to \infty}\, ||S-S_N||=0
\]
Hence the Hilbert-Scmhidt operator $S$ can be 
approximated in the operator norm
by linear operators with a finite-dimensional range.
in particular $S$ is a compact operator.
The converse is not true,  i.e. there exist compact operators on
$\ell^2$ which are not of the Hilbert-Schmidt type.
An example is when we take a sequence of complex numbers
$\{c_n\}$ whixh convefges to zero while
$\sum\, |c_n|^2=\infty$.
Let $S$ be the operator defined via the diagonsl matrix
where $a_{p,p}=c_p$ for every $p$.
Then $S$ is not a Hilbert-Schmidt operator while the 
reader should verify that it is a compact operator
on $\ell^2$.
\bigskip



\centerline{\bf{Carleman's inequality.}}
\bigskip


\noindent
We announce a result for matrices and remark that Theorem xx below
can be used to derive important results about spectea for Hilbert-Schmidt operators on infinite dimensional Hilbert spaces.
See � for an account.

\bigskip

BLABLA ....







\newpage



\bigskip

\centerline{\bf{Hilbert's spectral theorem.}}
\bigskip





\noindent
In his  book \emph{Integralgleichungen} from 1904
Hilbert established a fundamental  result  which extends the spectral
theorem for hermitian matrices.
A bounded linear operator $A$ on a complex  Hilbert space is  self-adjoint if
\[
\langle Ax,y\rangle =\langle x,Ay\rangle
\]
hold for every pair of vectors. It can be 
expressed by the equality $A=A^*$.
Let us now  regard a complex number
$\lambda=a+ib$ where $b\neq 0$.
For a vector $x$ the reader can check that
\[
||\lambda\cdot x+Ax||^2=
(a^2+b^2)||x||^2+||Ax||^2+
(a+ib)\langle x,Ax\rangle +
(a-ib)\langle Ax,x\rangle\tag{i}
\]
Since $A$ is self-adjoint two terms cancel and (i) becomes
\[
(a^2+b^2)||x||^2+||Ax||^2+
2\cdot \langle ax,Ax\rangle
\]
Next, the Cauchy-Schwarz inequality gives
\[
2\cdot \bigl|\langle ax,Ax\rangle\bigr|\leq 2\cdot ||ax||\cdot ||Ax||
\]
Since $r^2+s^2-2rs=(r-s)^2\geq 0$
for every pair ofm non-negative numbers we conclude  that
\[
||(a+ib)\cdot x+Ax||^2\geq b^2\cdot ||x||^2
\]
It follows that the linear operator
$S=\lambda\cdot E+A$ is injective
and has closed range.
Moreover it is surjective for if $y\perp S(\mathcal H)$ we have
\[
0=\langle \lambda\cdot x+Ax,y\rangle=
\langle x, \bar \lambda\cdot y+Ay\rangle
\]
for every $x$ and then
\[
 \bar \lambda\cdot y+Ay=0
 \] 
 Since the complex conjugate $\bar\lambda$ also has a non-zero imaginary part
 it follows that  $y=0$ and the requested surjectivity follows.
Hence 
 $S$ in invertible anbd since $\lambda$ was an arbitrary non-real complex number
 the  spectrum
 $\sigma(A)$ consists  of real numbers.
\medskip
 
\noindent
{\bf{The equality $||A||^2=||A^2//$.}}
Let $x$ be a unit vector. Now we get
\[
||Ax||^2=\langle Ax,Ax\rangle=
\langle A^2x,x\rangle
\]
By the Cauchy-Schwarz inequslity the absolute vaue of the right hand side above is
b ounded by
$||A^2x||$ and hence
\[
||Ax||^2||\leq ||A^2x||
\] 
hold for every $x$.
We can take $x$ so that $||Ax||$
approxiumates the operator norm
of $A$ and conclude that
\[
||A||^2||\leq ||A^2||
\]
Equality holds since
the submultiplicative inequality for operator norms
of composed linear operators entail that the right hand side
is majorised by
$||A||^2||$.
We cabn repat ths using the self-adjoint operator $A^2$ and so on.
It follows that
\[
||A||^{2m}||\leq ||A^{2m}||
\]
hold for every positive integer $m$.
Taking $m$:th roots and passing to the limit we get
the equality
\[
||A||= \rho(A)= \limsup\, ||A^n||^{\frac{1}{n}}
\]
where the right hand side is the spectral radius of $A$.
Now we can apply reults from the general theoru about
commutative Banach algebras, applied to
the closed subalgebra of $L(\mathcal H)$ generted by
$A$.
The spectral radius formula in Banach algebras together with
(xx) entails that
$\sigma(A)$ is a compact interval on the real line which
contains at least one of the points
$||A||$ or $- ||A||$.
\medskip


\noindent
At this stage Hilbert adopted  
 Carl Neumann's calculus together with
 measure theory which at this time
 had been sufficiently devloped by Stieltjes, Borel and Lebesgue. The great 
 merit was of course
 how Hilbert
 applied this to a situation where one regards operators on
 infinite dimensional spaces.
 The crucial point is the following:
 
 \medskip
 
 \noindent
 {\bf{1. Proposition.}}
 \emph{For every polynomial $p(t)$ with real coefficients
 one has the equality}
 \[
 ||p(A)||=\max_{t\in\sigma(A)}\, |p(t)|\tag{1.1}
 \]
 \medskip
 
 \noindent
 \emph{Proof.}
 By the general formula for spectra in � xx one has
 \[
 \sigma(p(A))=p(\sigma(A))
 \]
next, since $p$ has real coefficients it is clear that
$p(A)$ is self-adjoint  and from the above
$\sigma(p(A))$ contains at least on of the points
$||p(A)||$ or $-||p(A)||$ and (1.1) follows.
 
 

\bigskip

\noindent
{\bf{2. Spectral measures.}}
Let $A$ be self-adjoint.
Following Hilbert we consider a
pair of vectors $x,y$ in $\mathcal H$.
Then
\[
p(t)\mapsto \langle p(A)x,y\rangle\tag{2.1}
\]
is a linear functional on
the space of real-valued polynomials on the $t$-line.
The right hand side in (2.1) has absolute value bounded by
\[
||x||\cdot ||y||\cdot ||p(A)||\tag{2.2}
\]
From (1.1 ) this is equal to
$||x||\cdot ||y||$ times the maximum norm of
$P(t)$ on $\sigma(A)$.
Now we  use a wellkonown  result due to Weierstrass which 
assetts that the space iof real-valued polynomials is dense in
the normed space $C^0(\sigma(A)$ of real-valued and continuous functions on
the compact set $\sigma(A)$. Together with
the Riesz' represention theoren for the dual space
we find  a unique
meausre $\mu_{x,y}$ oin $\sigma(A)$ whose  total variation
is at most $||x||\cdot |y||$ and 
\[
 \langle p(A)x,y\rangle=\int_{\sigma(A)}\, p(t)\cdot
 d\mu_{x,y}(t)\tag{2.3}
\]
hold for every real-valued polynomial.
Next, let $g$ be some  real-valued continuous function on
$\sigma(A)$. Weierstrass theorem gives a sequence of polynomials
$\{p_\nu\}$ whixh converge uniformly to $g$ on
$\sigma(A)$. In particular the maxium norms
\[
\lim\, |p_n-p_m|_{\sigma(A)}= 0
\]
when $n$ and $m$ tend to $\infty$.
From (1.1) it follows that
\[
\lim\, |p_n(A)-p_m(A)||=0
 \]
 Now we use that the space of bounded liunear operators on
 $\mathcal H$ is a Banach space, a fact which of course was wellknown already in 1904.
 We conckude that the given continuous function $g$ 
 yields  a unique bounded linear operator
 $g(A)$ where
 \[
 \lim\, |p_n(A)-g(A)||=0
\]
and  
\[
 \langle g(A)x,y\rangle=\int_{\sigma(A)}\, g(t)\cdot
 d\mu_{x,y}(t)
\]
hold for every pair $x,y$ in $\mathcal H$.

\medskip

\noindent
{\bf{3. The spectral resolution.}}
Consider the algebra $\mathcal B^\infty(\sigma(A)$
of  real-valued bounded Borel functions on  the spectrum of $A$. As explained in
[Meausre) one can integrate  evety Borel function with
resopect to
the Riesz measures $\{\mu_{x,y}\}$
when  $x,y$ vary in $\mathcal H$.
Let us fix a bounded Borel function $\phi$.
Keeping $y$ fixed in $\mathcal H$
we get a linear functional on the Hilbert space defined by
\[
x\mapsto \int_{\sigma(A)}\, \phi(t)\cdot d\mu_{x,y}(t)
\]
whose norm is majorised by
$||y||$ times the sup-norm of $\phi$.
Since the Hilbert space is self-dual we find a unique
vector  $\Phi(y)$ such that
\[
\langle x,\Phi(y)\rangle=  \int_{\sigma(A)}\, \phi(t)\cdot d\mu_{x,y}(t)\tag{3.1}
\]
hold for every $x$.

\newpage

\noindent
{\bf{Exericse.}}
Verify that the map
\[
y\to \Phi(y)
\] 
is linear and conclude that $\phi$ yields a bounded linear oeprator
$\Phi$ for which (3.1)  hold for all pairs $x,y$.
Show also that the operator  norm
\[
||\Phi||=|\phi|_{\sigma(A)}
\]
and that 
\[
\phi\mapsto \Phi
\]
is an algebra homorphism from the Borel algebra to
a commutative subalgebra of bounded linear operators on
the Hilbert space.
Ths, if $\phi$ and $\psi$ is a pair of bounded Borel functions then
$\Phi\circ \Psi$ is the operator assigned to
$\phi\cdot \psi$.
\medskip

\noindent
{\bf{4. The operators $E(\delta)$.}}
For every Borel set $\delta$ in $ \sigma(A)$
its characterstic function is a 
Borel function which gives an operator denoted by
$E(\delta)$.
The exercise shows that
\[
E(\delta_1)\circ E(\delta_2)= E(\delta_1\cap \delta_2)\tag{4.1}
\]
hold for every pair of Borel sets.
In particular
\[
E(\delta)\circ E(\delta)=E(\delta)
\] 
for every single Borel set.
These 
$E$-operators commute with $A$
and are  self-adjoint which means that
every $E(\delta)$�is an orthogonal  projection from
$\mathcal H$ onto its range. Moreover,
if $\delta_1,\ldots,\delta_N$ is a finite family of disjoint
Brel sets whose union is $\sigma(A)$, then
$\{E(\delta_\nu)\}$ are pairwise orthogonal and
their sum is the identy operator in $\mathcal H$. It can also 
be expressed by the direct sum decomposition
\[
\mathcal H=
\oplus\, 
E(\delta_\nu)(\mathcal H)
\]
\medskip

\noindent
{\bf{5. A more refined study.}}
For each vector  $x$ and every Borel set $\delta$ one has
\[
||E(\delta)(x)||^2=
\langle E(\delta)(x),x\rangle=  \int_{\delta }\, d\mu_{x,x}(t)\tag{5.1}
\]
Since the left hand side   is non-negative 
and the Borel set was arbitrary, we conclude that
the measure
$d\mu_{x,x}$ is non-negative. For every real number $s$ we put
\[
\sigma(A)(s)= \sigma(A)\cap\,(-\infty,s]
\]
Keeping $x$ fixed we obtain a non-derasing function
\[
s\mapsto \int_{\sigma(A)(s)}\, d\mu_{x,x}(t)\tag{5.2}
\]
This  function  can  have jumps
at some
ponts, which by general facts is at most denumerable. or it
is a continuous function, but not necessarily absolutely contonous
in the sense of Lebesgue.
For example, absolute continuity fails for every $x$ if 
$\sigma(A)$ is a null set on the real line.
The behaviour of (5.2) depends on the specific vector $x$.
So
the overall picture is quite involved.

\medskip

\noindent
{\bf{6. Decomposition of $A$.}}
To every Borel set $\delta$ contained in $\sigma(A)$
we get the operator
$A(\delta)= A\cdot E(\delta)$ and fior its spectrum one has
the inclusion
\[
\sigma(A(\delta))\subset \overline{\delta}
\]
In this way Hilbert decomposed  $A$ into a  sum of 
self-adjoint operators whose spectra
are confined to  small Borel sets in
$\sigma(A)$.
This constitutes Hilbert's extension of  the spectral theorem for
matrices where the spectrum is finite,
To fully grasp  - and also to appreciate - Hilbert's theorem one should of course
regard specific cases of self-adjoint operators whose  spectra 
are not reduced to a finite set, and in general  contains  whole  intervals.

\newpage

\centerline{\bf{The spectral theorem for normal operators.}}
\bigskip


\noindent
A bounded linear operator $R$ on the complex Hilbert space is normal if it
commutes with its adjont $R^*$.
When this holds and $x$ is a unit vector one gets
\[
||R^2x||^2=
\langle R^2x,R^2x\rangle=
\langle Rx,R^*R^2x\rangle=
\langle Rx,RR^*R2x\rangle=
\langle R^*Rx,R^*Rx\rangle
\]
The last term is equal to $||R^*Rx||^2$.
We have also
\[
||Rx||^2=\langle Rx,Rx\rangle= \langle R^*Rx,x\rangle
\]
 By the Cauchy-Schwarz inequality the right hand side is majorised by
$||R^*Rx||$ when $x$ is a unit vector.
We conclude that
\[
||R^2x||^2\geq ||Rx||^4
\]
Exactly as in � xx it follows that
\[
||R^2||=||R||^2
\]
and taking higher powers one gets the equality
\[
||R||=\rho(R)
\]
\medskip

\noindent
{\bf{Exercise.}}
Show that if $p(z)= c_0+c_1z+\ldots+c_mz^m$ is a polynomials
with complex coefficents, then
\[
||p(R)||= \max_{z\in \sigma(R)}\, |p(z)|
\]
\medskip

\noindent
Above $\sigma(R)$ is a compact
subset of the complex $z$-plane which
in general  contains non-real points.
More geenrally, consider a polynomial in $z$ and $\bar z$:
\[
q(z,\bar z)= \sum\sum\ c_{jk}\cdot z^j\cdot \bar z^k
\]
where a doubly-indexed
family of complex coefficients appear.
To $q$ one associates the operator
\[
q(R,R^*)=  \sum\sum\ c_{jk}\cdot R^j\cdot (R^*)^k
\]
It turns out that one  has  the equality
\[
||q(R,R^*)||=  \max_{z\in \sigma(R)}\, |q(z,\bar z)|
\]
To prove this we consider the conmutative subalgebra
$L_*(R,R^*)$   of $L(\mathcal H)$ generated by $R$ and $R^*$.
We can take its closure in
thr normed space $L(\mathcal H)$ which gives a commutative Banach algebra
denoted by $B$.
Now we apply Gelfand's general theory from � xx and find
the maximal ideal space of this Banach algebra  which we denote by
$\mathcal M(B)$.
Every operator $S$ in
$B$ is normal and hence the equality in xxx holds.
It follows that $B$ is a sup-norm algebra, i.e. the Gelfand transform
is norm-preserving.
In oteher words, for every operator $S$ in
$B$ one has the equality
\[
||S||= \max_{p\in \mathcal M(B)}\, |\widehat{S}(p)|
\]
where $\widehat{S}$ is the Gelfand transform.
Among the $B$-elements we have $R$ and $R^*$
and hence also $R^*R$.
The biduality equation $R=R^{**}$ entails that
$R^*R$ is self-adjoint and therefore has a real spectrum.
This impes that the funvtion
\[
p\mapsto \widehat{R^*}(p)\cdot \widehat{R}(p)
\]
is real-valued on $\mathcal M(B)$.
From this the reader can ckeck that
the Gelfand transform of $R^*$ is the complex conjugate of that of
$R$.
Next, tghe spectrum
\[
\sigma(R)= \widehat{R}(\mathcal M(B))
\]
It turnd out that 
\[
p\mapsto \widehat{R}(p)
\]
is bijective and therefore identifies $\mathcal M(B)$ with
the spectrum of $R$.
To see this we notice that since $R$ and $R^*$ generate $B$, it follows
that 
when$p$ and $q$ are teo distinc points in 
$\mathcal (B)$
then
we cannot have the tewo equalities
\[
\widehat{R}(p)= \widehat{R}(q)\quad\&\quad
\widehat{R^*}(p)= \widehat{R}(^*q)
\]
Since $\widehat{R^*}$ is the compex cojnugte of
$\widehat{R}$ this means that
\[
\widehat{R}(p)\neq \widehat{R}(q)
\]
whoch proves that the map in (xx) is bijective.
\medskip


\noindent
{\bf{Exercise.}}
Confimrm from the above that
the eaquality in (xx) holds for every $q$-polynomial of $z$ and $\bar z$.
\bigskip


\noindent
{\bf{Spectral resolutions.}}
Exactly as for self-adjoint operatos we find 
spectral measures starting from a normal operator $R$.
More precisely, one employs Weiertrass's
theorem for self-adjoint algebra of continuous funtions, i.e. that
every $g\in C^0(\sigma(R)$ can be uniformly approximated by
$q$-polynomiasl as above.
This leads to
an spectral calcluus where every bounded Borel function
$\phi$ on $\sigma(R)$ gives a bounded linear operator
$\Phi$ and for every pair $x,y,$ in the Hilbert space
one has a unique measure $\mu_{x,y}$ on
$\sigma(R)$ and 
\[
\langle \Phi(x),y\rangle=
\int_{\sigma(R)}\, \phi(z)\cdot d\mu_{x,y}(z)
\]
In partocular we can take charactersitic functions of
Borel sets in $\sigma(R)$ and obtain operators
$E(\delta)$, and exaclty as for self-adjoint oeprstors they
are orthogonal projections.
We also get the normal operators
$R(\delta=R\cdot E(\delta)$ where the spectrum is confined to
the clisure of $\delta$ for every Borel set.

\medskip

\noindent
From the above one has in particular
\[
\langle R(x),y\rangle=
\int_{\sigma(R)}\, z\cdot d\mu_{x,y}(z)\quad\&\quad 
\langle R^*(x),y\rangle=
\int_{\sigma(R)}\, \bar z\cdot d\mu_{x,y}(z)
\]
It follows that the normal operator $R$�is self-adjoint if and only
if the spectrum is confined to the real line.
If $\phi$ is a real-va�ued and bounded Borel function on
$\sigma(R)$, it follows that $\Phi$ is self-adjoint, i.e. starting form a norma operator
$R$ one can produce a quite extensive family of 
self-adjoint operators with the property that they  commute
with each other, as well as with $R$.
\medskip


\noindent


\newpage













     





































\newpage

\centerline{\bf{4:B. Eigenvalues of matrices.}}

\bigskip

\noindent
Using  the Hermitian inner product on ${\bf{C}}^n$
we study
eigenvalues of an $n\times n$-matrices $A$ with complex elements.
The  spectrum $\sigma(A)$ is the $n$-tuple of roots
$\lambda_1,\ldots,\lambda_n$ of the characteristic polynomial
$P_A(\lambda)=\text{det}(\lambda\cdot E_n-A)$,
where eventual multiple eigenvalues are
repeated.
\medskip


\noindent
{\bf{4:B.1 Polarisation.}}
Let $A$ be an arbitrary $n\times n$-matrix.
Then there exists a unitary matrix
$U$ such that the matrix 
$U^*AU$ is upper triangular. To prove this
we first use the wellknown fact that there exists
a basis
$\xi_1,\ldots,\xi_n$ in ${\bf{C}}^n$
in which $A$ is upper triangular, i.e. 
\[ 
A(\xi_k)= a_{1k}\xi_1+\ldots a_{kk}\xi_k\quad\colon\,,1\leq k\leq n
\]
The \emph{Gram-Schmidt orthogonalisation}
gives
an orthonormal basis $e_1,\ldots,e_n$ where
\[ 
\xi_k=c_{1k}\cdot e_1+\ldots c_{kk}\cdot e_k\quad\text{for each}
\quad 1\leq k\leq n
\]
Let $U$ be  the unitary matrix which sends the standard basis in
${\bf{C}}^n$ to the $\xi$-basis. Now the
reader can  verify that the linear operator 
$U^*AU$ is  represented by an upper triangular matrix in the $\xi$-basis.





\medskip


\noindent
\emph{A theorem by H. Weyl.}
Let $\{\lambda_k\}$ be the spectrum of $A$ where the $\lambda$-sequence is
chosen with non-increasing absolute values, i.e. 
$|\lambda_1|\geq \ldots\geq |\lambda_n|$. 
We  have also the Hermitian matrix $A^*A$ which is non-negative so that
$\sigma(A^*A)$ consists of non-negative real numbers
$\mu_1\geq \mu_2\geq\ldots\geq \mu_n$. 
In particular one has 
\[ 
\mu_1=\max_{|x|=1}\, \langle Ax,Ax\rangle\tag{1}
\]
\medskip



\noindent
{\bf{4:B.2 Theorem.}}
\emph{For every $1\leq p\leq n$ one has the inequality}
\[
|\lambda_1\cdots\lambda_p|\leq \sqrt{\mu_1\cdots \mu_p}
\]
\emph{where $\{\mu_k\}$ are the eigenvalues of $A^*A$}


\medskip


\noindent
First
we consider the  case $p=1$ and prove the inequality
\[ 
|\lambda_1| \leq \sqrt{\mu_1}\tag{i}
\]


\noindent
Since $\lambda_1$ is an eigenvalue there exists
a vector $x_*$ with $|x_*|=1$ so that
$A(x_*)=\lambda_1\cdot x_*$. It follows from (1) above
that
\[
\mu_1\geq \langle A(x_*),A(x_*)\rangle=|\lambda_1|^2
\]

\medskip

\noindent
{\bf{Remark.}}
The inequality is in general strict.
Consider the $2\times 2$-matrix
\[ 
A= 
\begin{pmatrix}
1& a\\ 0&b
\end{pmatrix}
\]
where $0<b<1$  and $a\neq 0$ some complex number
which gives
\[ 
A^*A= \begin{pmatrix}
1& a\\ a&a^2+b^2
\end{pmatrix}
\]
Here $\lambda\uuu 1=1$ and the eigenvector $x\uuu *=e\uuu
1$ and we see that
$\langle A(x_*),A(x_*)\rangle=1+|a|^2$.


\bigskip

\noindent
\emph{Proof when $p\geq 2$}
We employ a construction of independent interest.
Let $e_1,\ldots,e_n$ be some orthonormal basis in
${\bf{C}}^n$. For   every $p\geq 2$ we get 
the inner product space $V^{p}$ whose vectors are
\[ 
v=\sum\, c_{i_1,\ldots,i_p}\cdot e_{i_1}\wedge\ldots\wedge e_{i_p}
\]
where the sum extends over $p$-tuples $1\leq i_1<\ldots<i_p$.
This is an inner product   space of dimension $\binom{n}{p}$
where $\{e_{i_1}\wedge\ldots\wedge e_{i_p}\}$ is an orthonormal basis.
Consider a linear operator $A$ on ${\bf{C}}^n$
which in the $e$-basis is represented by a matrix
with elements
\[ 
a_{ik}=\langle Ae_i,e_k\rangle
\]
If $p\geq 1$ we define the linear operaror $A^{(p)}$
on $V^{(p)}$ by
\[ 
A^{(p)}(e_{i_1}\wedge\ldots\wedge e_{i_p})=
A(e_{i_1})\wedge\ldots\wedge A(e_{i_p})=
\sum\, a_{j_1i_1}\cdots a_{j_pi_p}\cdot
e_{j_1}\wedge\ldots\wedge e_{j_p}
\]
with the sum extended over all $1\leq j_1<\ldots<j_p$.
\medskip

\noindent
\emph{Sublemma.} \emph{The eigenvalues of $A^{(p)}$
consists of the $\binom{n}{p}$-tuple given by the products}
\[ 
\lambda_{i_1}\cdots \lambda_{i_m}\quad\colon\quad
1\leq i_1<\ldots<i_p\leq n\tag{*}
\]

\medskip

\noindent
\emph{Proof.} The eigenvalues above are independent of the chosen
orthonormal basis $e_1,\ldots,e_n$ since a change of this basis gives
another orthonormal basis in $V^{(p)}$ which does not affect
the eigenvalues of $A^{(p)}$.
Using  a   polarisation from 4:B.1 we may  assume 
from the start that
$A$ is an upper  triangular matrix and   then reader can
verify (*) in the sublemma.
\bigskip

\noindent
\emph{Final part of the proof.}
If  $p\geq 2$ it is clear that  one has the equality

\[
(A^{(p)})^*\cdot A^{(p)}=(A^*\cdot A)^{(p)}\tag{i}
\]
If $\lambda_1,\ldots\lambda_p$ is the
large $p$-tuple in Weyl's Theorem the product appears
as an eigenvalue of
$A^{(p)}$ and using the case $p=1$
one gets Weyl's inequality
since
the product $\mu_1\cdots\mu_p$
appears as an eigenvalue of
$(A^*\cdot A)^{(p)}$.

\bigskip


\centerline {\emph{4.B.3 An inequality by Pick.}}
\bigskip

\noindent
Let $C=\{c_{ik}\}$ be a skew-symmetric $n\times n$-matrix, i.e. 
$c_{ik}= -c_{ki}$ hold for all pairs $i,k$.
Denote  by $g$  the maximum of the absolute values of
the matrix  elements of $C$. 

\medskip

\noindent
{\bf{4:B.4 Theorem.}} \emph{One
has the inequality}
\[
\max_{|x|=1}\, \bigl|\langle Cx,x\rangle\bigr|
\leq g\cdot \text{cot}(\frac{\pi}{2n})\cdot 
\sqrt{n(n-1)/2}\tag{*}
\]



\noindent
\emph{Proof.}
Since $g$ is unchanged if we permute the columns of the given
$C$-matrix it suffices to prove (*) for a vector $x$ of unit length such that
\[
\mathfrak{Im}(x_k\bar x_i-x_i\bar x_k)\geq 0
\quad\colon\quad 1\leq i<k\leq n\tag{1}
\]



\noindent
Now one has
\[
\langle Cx,x\rangle=
\sum\sum\, c_{ik}x_k\bar x_i=
\sum_{i<k}\, c_{ik}x_k\bar x_i+
\sum_{i>k}\,c_{ik} x_k\bar x_i=
\sum_{i<k}\, c_{ik}(x_k\bar x_i-\bar x_k x_i)
\tag{3}
\]
where the last equality follows since
$C$ is skew-symmetric. Put
\[
\gamma_{ik}=\mathfrak{Im}\,( x_k\bar x_i-\bar x_k x_i)
\]

\noindent
Then (1) and the triangle
inequality give
\[
|\langle Cx,x\rangle|\leq \sum_{i<k}\, |c_{ik}|\cdot \gamma_{ik}
\leq g\cdot \sum_{i<k}\,  \gamma_{ik}
\]
Hence there only  remains to show that
\[
\sum_{i<k}\,  \gamma_{ik}\leq 
\text{cot}(\frac{\pi}{2n})\cdot 
\sqrt{n(n-1)/2}\tag{4}
\]
To prove this we write $x_k= \alpha_k+i\beta_k$
and the reader can verify that (4) follows from the inequality
\[
\sum_{i\neq k}\, a_kb_i\leq \text{cot}(\frac{\pi}{2n})\cdot 
\sqrt{n(n-1)/2}\tag{5}
\] 
whenever $\{a_k\}$ and $\{b_i\}$ are $n$-tuples of non-negative real numbers 
for which
$\sum_{k=1}^{k=n}\, a_k^2+b_k^2=1$.
Finally, 
(4) follows when one
applies Lagrange's multiplier for extremals of a quadratic form.





\newpage


\centerline {\emph{4.B.4 Results  by A. Brauer.}}
\bigskip

\noindent
Let $A$ be an  $n\times n$-matrix. To each
$1\leq k\leq n$ we set

\[ r_k=\text{min}\,[\,\sum_{j\neq k}\, |a_{jk}|\,\colon\,
\sum_{j\neq k}\, |a_{kj}\,\bigr]
\]

\medskip
\noindent
{\bf{4:B.5 Theorem.}}
\emph{Denote by $C_k$ the closed  disc of 
of radius $r_k$ centered at the diagonal element $a_{kk}$.
Then one has the inclusion:}
\[
\sigma(A)\subset C_1\cup\ldots\cup C_n\tag{*}
\]


\noindent 
\emph{Proof.}  Consider some eigenvalue $\lambda$ so that
$Ax=\lambda\cdot x$ for a non-zero eigenvector.
It means that
\[
\sum_{j=1}^
 {j=n}\, a_{j\nu}\cdot x_\nu=\,\lambda\cdot x_j
 \quad\colon\quad 1\leq j\leq n
 \]
Choose $k$ so that $|x_k|\geq |x_j|$ for all $j$.
Now we have
\[ 
(\lambda-a_{kk})\cdot x_k=  \sum_{j\neq k}\, a_{j\nu}\cdot x_\nu\implies
|\lambda-a_{kk}|\leq \sum_{j\neq k}\, |a_{kj}|\tag{1}
\]
At the same time the adjoint $A^*$ satisfies
$A^*(x)=\bar\lambda\cdot x$ which gives
\[
\sum_{j=1}^
 {j=n}\, \bar a_{\nu,j}\cdot x_\nu=\,\bar \lambda\cdot x_j
 \quad\colon\quad 1\leq j\leq n
\] 
Exactly as above we get
\[
|\lambda- a_{kk}|=
|\bar \lambda-\bar a_{kk}|\leq \sum_{j\neq k}\, |a_{jk}|\tag{2}
\]
Hence (1-2) give the inclusion
$\lambda\in C_k$.
\medskip


\noindent
{\bf{4:B.6 Theorem.}} \emph{Assume that the closed discs $C_1,\ldots,C_n$ are disjoint.
Then the eigenvalues of $A$ are simple and for every
$k$ there is a unique $\lambda_k\in C_k$.}
\medskip


\noindent
\emph{Proof.} 
Let $D$ be the diagonal matrix where $d_{kk}=a_{kk}$.
For ever $0<s<1$ we consider the matrix

\[ 
B_s=sA+(1-s)D
\]
Here $b_{kk}=a_{kk}$ for every $k$ and the associated discs
of the $B$-matrix are $C_1(s),\ldots,C_b(s)$ where
$C_k(s)$ is again centered at $a_{kk}$ while the radius is $s\cdot r_k$.
When $s\simeq 0$ the matrix $B\simeq D$ and then it is clear that
the previous theorem implies that
$B_s$ has simple eigenvalues $\{\lambda_k(s)\}$ where
$\lambda_k(s)\in C_k(s)$  for every $k$.
Next, since the "large discs" $C_1,\ldots,C_n$ are disjoint, it follows by
continuity that these inclusions  holds for every $s$ and with $s=1$ we get
the theorem. 
\medskip

\noindent
{\bf{Exercise.}} Assume that the elements of $A$ are all real and
the discs above are disjoint. Show that the eigenvalues of $A$ are all real.

\bigskip


\centerline {\emph{ Results  by Perron and Frobenius}}
\medskip

\noindent
Let $A=\{a_{pq}\}$ be a matrix where all elements are 
real and positive.
Denote by $\Delta^n_+$ the standard simplex of $n$-tuples
$(x_1,\ldots,x_n)$ where
$x_1+\ldots+x_n=1$ and every $x_k\geq 0$.
The following result was  established by Perron in [xx]:

\medskip

\noindent
{\bf{4:B.7  Theorem.}} \emph{There exists a unique ${\bf{x}}^*\in\Delta_+^n$
which is an eigenvector for $A$ with an eigenvalue $s^*$. Moreover.
$|\lambda|<s^*$ holds for every other eigenvalue}.
\medskip


\noindent
We leave the proof as an exercise to the reader.
In [Frob] 
the following addendum to
Theorem 4:B.7 is proved.
\medskip

\noindent
{\bf{4:B.8  Theorem.}} \emph{Let $A$ as above be a positive matrix
which gives the eigenvalue $s^*$. For every complex
$n\times n$-matrix
$B=\{b_{pq}\}$ such that $|b_{pq}|\leq a_{pq}$
hold for all pairs $p,q$, it follows that
every root of $P_B(\lambda)$ has absolute value $\leq s^*$
and equality holds if and only if $B=A$.}

\bigskip


\noindent
{\bf{4:B.9 The case of probability matrices.}}
Let $A$ have positive elements and assume that the sum in every column 
is one. In this case $s^*=1$ for with ${\bf{x}}^*=(x_1^*,\ldots,x_n^*)$ we have
\[
s^*= s^*\cdot \sum\, x^*_p= \sum\sum a_{pq}\cdot x^*_q=
\sum\, x^*_q=1
\]
\medskip

\noindent
The components of the
Perron vector ${\bf{x}}^*$
yields the probabilities to arrive at a station $q$ after many
independent motions in an associated stationary Markov chain where
the $A$-matrix defines the transition probabilities.
\medskip


\noindent
{\bf{Example.}} Let $n=2$ and 
take  $a_{11}=3/4$ and $a_{21}= 1/4$, while $a_{12}=a_{22}= 1/2$.
A computation gives $s^*=2/3$ which in probabilistic terms means that the
asymptotic probability  to arrive
at station 1 after many steps is $2/3$ while that of station 2 is $1/3$.
Here we notice that the second eigenvalue is
$s_*=1/4$ and an associated eigenvector is
$(1,-2)$.
\bigskip

\noindent
{\bf{4.B.10  Extension to infinite dimensions.}}
The Perron-Forbenius result was
extended to positive operstors on Hilbert spaces by
Pietsch in the article [1912].
We refer to � xx the proof.


\newpage

\centerline{\bf{Unitary operators.}}

\bigskip

\noindent
An operator $U$ on the complex Hilbert space
$\mathcal H$ is unitary if $U$ is invertible and its inverse is equal
to the adjoint $U^*$. It follows that
\[
\langle Ux,Ux\rangle=
\langle x,U^*Ux\rangle=||x||^2
\]
Hence $||Ux||=||x|$| for every $x$, i.e. the operator is norm-preserving.
Moreover, since every invertible operator commutes with its invrse, it follows that
$U$ and $U^*$ commute, i.r.  unitary operators are normal.
From (xx) the reader can chekc that
$\sigma(U)$ is confined to the unit circle.
We can apply Hilbert's spectral theorem for bounded normal operators.
So to every Brel set $\delta$ in $T$
we get a self-adjoint projection  $E(\delta)$.
A class of unitary operatoies aries when
we
take a finite family of pairwise disjoint Borel sets
$\{\delta_k\}$ whosr union is the unit circle $T$ and define
\[
U_*= \sum \, e^{i\theta_k}\cdot E(\delta_k)
\]
The  given unitry operator $U$
can be approximated in the operator norm by
unitary ooerators in (xx).
Notice that $U_*$
has the finite spectrum $\{e^{i\theta_k}\}$.
\medskip

\noindent
The approximatins of $U$  lead to a useful  inequality.
Namely, let $P$ be some self-adjoint prpjection from
$\mathcal H$ onto a subspace $P(\mathcal H)$.
Then one has
\[
||P\circ U(x)||\leq ||x||
\]
for every vector $x\in \mathcal H$. In other words,
$P\circ U$ is a contraction.
To prove this it suffices to regard
a unitary operator  with a finte spectrum . given
by $U_*$ in (1). Iif $x\in \mathcal H$ we have
\[
x=\sum\, E_U(\delta_k)(x)
\]
where the vectors $\{x_k=E_U(\delta_k)(x)$ are 
orthogonal and 
\[
||x||^2=\sum\, ||x_k||^2
\]
Now
\[
P\circ U(x)= \sum e^{i\theta_k}\cdot P(x_k)
\]
where the vectors $\{P(x_k)\}$ are orthogonal.  It follows that
\[
 ||P\circ U(x)||^2= \sum\, || P(x_k)||^2\leq ||x||^2
 \] 
 where the last inequality holds since
 $||Py||\leq ||y||$ for every $y\in \mathcal H$.
 Hence, for every self-adjoint  projection $P$ the composed operator
 $P\circ U$ is a contraction.
 \medskip
 
 \noindent
 {\bf{A remarkable inequality.}}
 Let $p(z)= c_0+c-1z+\ldots+c_mz^m$ be a polynomial with
 complex coefficents.
Given a unitsry operator $U$ and a self-adjoint prjection $P$
we put
\[
S= \sum\, c_\nu\cdot (P\circ U)^\nu
\]
Let $x$ be a unit vector in
$\mathcal H$.
Approximating $U$ by a unitsry operator with s finite spectrum
wehave with rthe notstions form xxxx:
\[
S(x)= \sum\sum\, c_\nu\cdot e^{i\nu \theta_k}\cdot P(x_k)
\]
it follows that
\[
||S(x)||^2= \sum\, \sum_{\nu=1}^{\nu=m}\, | c_\nu\cdot e^{i\nu \theta_k}|
\cdot ||P(x_k)||^2
\]
Put
\[
|p|_T= \max_\theta\,|p(e^{i\theta}|
\]
Then the right hsnd side in (xx) is majorised by
\[
|p[_T\cdot \sum ||P(x_k)||^2\leq
|p|_T\cdot \sum\,||x_k||^2=|p|_T\cdot ||x||^2=
|p|_T
\]
where the last equality follows since $x$ has unit norm.
Since $x$ was arbitary we get
the inequality below for the operator norm:
\[
||S||\leq |p|_T
\]
Notice that we can take $P=E$ aobve and hence one has in particular
\[
||p(U)||\leq |p|_T
\] 
for every polynomial $p(z9$.
Let us remakr that this inequality is an immediate consequence of
the result in � xx which shows thst when $x$ is a vecotr of unit length
and $p(z)$ aa polynomial, then
\[
||p(U)x||^2= \int_T\, |p(e^{i\theta})|^2\cdot d\mu_{x,x}(\theta)
\]
where $\mu_{x,x}$ is a probability measure.
Then (xx follows since
the $L^2$integral in the right hand side above is majorised by
$|p|_T^2$.


 
 

\newpage









\centerline{\bf{� 11. Contractions and the Nagy-Szeg� theorem}}
\bigskip


\noindent
A linear operator $A$ on the Hilbert space $\mathcal H$ is  a contraction if
its operator norm is $\leq 1$, i.e.
\[
 ||Ax||\leq ||x||\quad\colon\quad x\in\mathcal H\tag{1}
\] 
Let $E$ be the identity operator on $\mathcal H$.
Now $E-A^*A$ is a bounded self-adjoint operator and  (1) gives:
\[
\langle x-A^*Ax,x\rangle=||x||^2-||Ax||^2\geq 0
\]
The result in � 8.xx shows that 
this  non-negative self-adjoint operator has a square root:
\[ 
B_1=\sqrt{E-A^*A}
\]
Next,  the operator norms of $A$ and  $A^*$ are equal so
$A^*$ is  also a  contraction and the equation
$A^{**}=A$ gives exact�y as above the 
self-adjoint operator
\[
B_2=\sqrt{E-AA^*}
\]
Since  $AA^*=A^*A$ is not assumed
the  self-adjoint operators
$B_1,B_2$ need not be equal.
However, the following hold:

\medskip

\noindent
{\bf{11.3.1 Propostion.}}
\emph{One has the equations}
\[ 
AB_1=B_2A\quad\text{and}\quad A^*B_2= B_1A^*
\]



\noindent
\emph{Proof.}
If $n$ is a positive integer we notice that
\[
A(A^*A)^n=(AA^*)^nA\tag{i}
\]
Now $A^*A$ is a self-adjoint operator whose compact spectrum is confined to
the closed unit interval $[0,1]$.
If $f\in C^0[0,1]$
is a real-valued continuous function it can be approximated uniformly
by a sequence of polynomials $\{p_n\}$
and the operational calculus from � XX yields an operator
$f(A^*A)$ where
\[
\lim_{n\to\infty}\, ||p_n(A^*A)-f(A^*A)||=0
\]
Since the spectrum of $AA^*$ also is confined to $[0,1]$,
the same polynomial sequence
$\{p_n\}$ gives 
an operator $f(AA^*)$ where
\[
\lim\, ||p_n(AA^*)-f(AA^*)||=0
\]
Now (i) and the two limit formulas above give:
\[
A\circ f(A^*A)= f(AA^*)\circ A\tag{ii}
\]
In particular we can take 
$f(t)= \sqrt{1-t}$ which gives the left hand side
in Proposition 11.3.1. By a similar reasoning one proves the 
equality in the right hand side.
\bigskip

\noindent{\bf{11.2 The unitary operator $U_A$}}.
On the Hilbert space $\mathcal H\times \mathcal H$
we define a linear operator $U_A$ represented by the block matrix
\[
U_A=
\begin{pmatrix} A& B_2\\
B_1&-A^*
\end{pmatrix}\tag{*}
\]


\noindent
{\bf{11.3 Proposition.}}
\emph{$U_A$ is a unitary operator on
 $\mathcal H\times \mathcal H$}.
 
 \medskip
 
 \noindent
 \emph{Proof.}
 For a pair of vectors $x,y$ in $\mathcal H$
 we must prove the equality
\[
|| U_A(x\oplus y)||^2=||x||^2+||y||^2\tag{i}
\]
To get (i)  we notice that for every vector $h\in\mathcal H$ 
the self-adjointness of $B_1$ gives
\[
||B_1h||^2=\langle B_1h,B_1h\rangle=
\langle B_1^2h,h\rangle=\langle h-A^*Ah,h\rangle=
||h||^2-||Ah||^2\tag{ii}
\]
where  the last equality holds since we have
$\langle A^*Ah,h\rangle =\langle Ah,A^{**}h\rangle =||Ah||^2$
and the biduality formula $A=A^{**}$.
In the same way one has:
\[
||B_2h||^2=||h||^2-||A^*h||^2\tag{iii}
\]


\noindent
Next, by the construction of $U_A$ the left hand side in (i) becomes
\[
||Ax+B_2y||^2+||B_1x-A^*y||^2\tag{iv}
\]
Using (iii) we have
\[
||Ax+B_2y||^2=||Ax||^2+||y||^2-||A^*y||^2+
\langle Ax,B_2y\rangle+ \langle B_2 y,Ax\rangle
\]
Similarly,  (ii) gives
\[
||B_1x-A^*y||^2=||x||^2-||Ax||^2+||A^*y||^2-
\langle B_1x,A^*y\rangle- \langle A^*y,B_ x\rangle
\]
Adding these two equations  we 
conclude that (i) follows from the equality
\[
\langle Ax,B_2y\rangle+ 
\langle B_2 y,Ax\rangle=
\langle B_1x,A^*y\rangle+ \langle A^*y,B_ x\rangle\tag{v}
\]
To get (v) we use Proposition 11.5.1 which  gives
\[
\langle Ax,B_2y\rangle=
\langle x,A^*B_2y\rangle=
\langle x,B_1A^*y\rangle=
\langle B_1x,A^*y\rangle
\]
where the last equality used that $B_1$ is self-adjoint. In the same way one verifies that
\[
\langle B_2 y,Ax\rangle=
 \langle A^*y,B_1 x\rangle
\]
and (v) follows.

\bigskip



\centerline{\bf{11.4 The Nagy-Szeg� theorem.}}
\bigskip


\noindent
The constructions above were applied 
by 
Nagy and Szeg� to give:
\medskip

\noindent
{\bf{11.4.1 Theorem}}
\emph{For every bounded linear operator $A$ on a Hilbert space
$\mathcal H$ there exists a Hilbert space
$\mathcal H^*$ which contains $\mathcal H$ and a unitary operator $U_A$
on $\mathcal H^*$ such that}
\[
A^n=\mathcal (P\cdot U_A)^n\quad\colon\quad n=1,2,\ldots
\]
\emph{where $\mathcal P\colon\mathcal H^* \to \mathcal H$
is the orthogonal  projection.}
\medskip


\noindent
\emph{Proof.}
On the product  $\mathcal H_1=\mathcal H\times\mathcal H$ we have the unitary
operator $U_A$ from (*) in 11.3.2 and notice that
\[
U(x\oplus 0)= Ax
\]
This gives the equations in (xx) above.

\medskip

\noindent
{\bf{Application.}}
The Nagy-Szeg� result has
an interesting consequence. 
Let $A$  be a contraction. If
$p(z)=
c_0+c_1<+\ldots+c_nz^n$ is a  polynomial with
complex coefficients
we get the 
operator $p(A)=\sum\,c_\nu A^\nu$ and 
with these notations one has:

\medskip

\noindent
{\bf{11.4.2 Theorem}}
\emph{For every pair $A,p(z)$ as above one has}
\[
||p(A)||\leq \max_{z\in D}\, |p(z)|
\] 
\emph{where the the maximum in the right hand side is taken on the unit disc.}
\medskip

\noindent
\emph{Proof.} 
Theorem 11.4.1  gives $p(A)= p(P\circ U_A)$.
Since the orthogonal $\mathcal P$-projection is norm decreasing we get
\[
||p(A)(\xi)||^2\leq ||p(U_A)(\xi,0)||^2
\]
Let $\xi$ be a unit vector such that
$||p(A)(\xi)||= ||p(A)||$.
The operational calculus in � 7 XX  applied to the unitary operator
$U_A$  yields a
probablity measure $\mu_\xi$ on
the unit circle such that
\[
 ||p(U_A)(\xi,0)||^2=
 \int_0^{2\pi}\, |p(e^{i\theta})|^2\cdot d\mu_\xi(\theta)
 \]





The right hand side is majorized by $|p|_D^2$ and Theorem 11.4.2 follows.

\bigskip

\noindent
{\bf{11.4.3 An application.}}
Let $A(D)$ be the disc algebra. Since each $f\in A(D)$ can be 
uniformly approximated by analytic 
polynomials, Theorem 11.4.2  entails that if a linear operator $A$ 
on the Hilbert space $\mathcal H$ is a contraction
then each $f\in A(D)$
gives a 
bounded linear operator $f(A)$, i.e. we have norm-preserving map  from
the supnorm algebra $A(D)$ into the space of bounded linear operators on
$\mathcal H$.


\newpage

\centerline
{\bf{� 12 Miscellanous results}}
\bigskip


\noindent
Before Theorem 12.x is announced
we recall that the product formula for matrices in � X  asserts the following.
Let $N\geq 2$  and $T$  is some $N\times N$-matrix whose
elements are complex numbers
which as usual is regarded as a linear operator on
the Hermitian space ${\bf{C}}^N$.
Then there 
exists the 
self-adjoint matrix $\sqrt{T^*T}$ whose eigenvalues are non-negative.
Notice that for every vector $x$ one has 
\[
||T^*T(x)||^||Tx||^2\implies ||\sqrt{T^*T}(x)||= ||Tx||\tag{i}
\] 


\noindent
and since
$\sqrt{T^*T}$ is self-adjoint
we have an orthogonal decomposition 
\[
\sqrt{T^*T}({\bf{C}}^N)\oplus
\text{Ker}(\sqrt{T^*T})= {\bf{C}}^N\tag{ii}
\]
where the self-adjointness gives the equality
\[
\text{Ker}(\sqrt{T^*T})=\sqrt{T^*T}({\bf{C}}^N)^\perp\tag{iii}
\]
\medskip

\noindent
{\bf{The partial isometry operator.}}
Show that 
there exists a unique linear operator $P$ such that
\[ 
T= P\cdot \sqrt{T^*T}\tag{*}
\]
where the $P$-kernel is
the orthogonal complement of the range of  $\sqrt{T^*T}$.
Moreover, from  (i) it follows that
\[
||P(y)||=||y||
\] 
for each vector in the range of
$\sqrt{T^*T}$.
One  refers to $P$ as a partial isometry attached to $T$.
\bigskip

\noindent
{\bf{Extension to operators on
Hilbert spaces.}}.
Let $T$ be a bounded opertor on
the Hilbert space
$\mathcal H$.
The spectral theorem for bounded
and self-adjoint operators
gives a similar equation as in (*) above
using
the non-negative and self-adjoint
operator $\sqrt{T^*T}$. More generally,
let $T$ be densely defined and closed.
From  � XX  there exists the
densely defined self-adjoint operator $T^*T$
and we can also take its square root.

\medskip

\noindent
{\bf{12.1 Theorem.}}
\emph{There exists  a bounded partial isometry
$P$ such that}
\[ 
T= P\cdot \sqrt{T^*T}
\]



\medskip


\noindent
\emph{Proof.}
Since $T$ has closed graph we have the Hilbert space
$\Gamma(T)$.
For each $x\in\mathcal D(T)$ we get the vector
$x_*=(x,Tx)$ in $\Gamma(T)$.
Now
\[
(x_*.y_*)\mapsto \langle x,y\rangle
\] 
is a bounded Hermitiain bi-linear form on
the Hilbert space $\Gamma(T)$.
The self-duality of Hilbert spaces
gives  bounded and self-adjoint operator
$A$ on
$\Gamma(T)$ such that
\[
\langle x,y\rangle=\{ Ax_*,y_*\}
\]
where the right hand side is the inner product between vectors in
$\Gamma(T)$.
Let 
\[ 
j\colon (x,Tx)\mapsto x
\]
be  the projection from
$\Gamma(T)$ onto $\mathcal D(T)$ and for each
$x\in\mathcal D(T)$ we put
\[ 
Bx= j(Ax_*)
\]
Then $B$ is a linear operator from
$\mathcal D(T)$ into itself where
\[
\langle Bx,y\rangle =\{Ax_*,y_*\}= \{ x_*,Ay_*\}
=\langle x,By\rangle
\quad\colon\quad x,y\in
\mathcal D(T)
\tag{i}
\]
We have also
\[
\langle Bx,x\rangle=\{A^2x_*,x_*\}=\{Ax_*,Ax_*\}=
\langle Bx,Bx\rangle+\langle TBx,TBx\rangle\implies
\]
\[
||Bx||^2= \langle Bx,Bx\rangle\leq \langle Bx,x\rangle
\leq||Bx||\cdot ||x||
\] 

\noindent
where the Cauchy-Schwarz inequality was used in the last step.
Hence
\[
||Bx||\leq ||x||\quad\colon\quad x\in
\mathcal D(T)
\] 
This entails that that the densey defined operator $B$ 
extends uniquely to
$\mathcal H$ as  a bounded operator
of norm $\leq 1$.
 Moreover, since
(i) hold for pairs $x,y$ in the dense subspace
$\mathcal D(T)$, it follows that
$B$  is self-adjoint.
Next, consider a pair $x,y$ in $\mathcal D(T)$ which gives
\[ 
\langle x,y\rangle= \{Ax_*,y_*\}=
\{x_*,Ay_*\}=\langle x, By\rangle+\langle Tx,TBy\rangle
\]
Keeping $y$ fixed the linear functiional
\[ 
x\mapsto \langle Tx,TBy\rangle=\langle x,y\rangle-\langle x, By\rangle
\]
is bounded on $\mathcal D(T)$. By the construction of
$T^*$  it follows that
$TBy\in \mathcal D(T^*)$
and we also get the equality
\[
\langle x,y\rangle=
\langle x, By\rangle+\langle x,T^*TBy\rangle\tag{ii}
\]
\medskip


\noindent
Since (ii)  holds for all $x$ in the dense subspace $\mathcal D(T)$ we conclude that
\[
y=By+T^*TBy= (E+T^*T)(By)\quad\colon\quad y\in \mathcal D(T)\tag{iii}
\]
\medskip

\noindent
{\bf{Conclusion.}}
From the above we have 
the inclusion
\[ 
TB(\mathcal D(T))\subset
\mathcal D(T^*)
\]
Hence   $\mathcal D(T^*T)$
 contains
$B(\mathcal D(T))$ and (iii) 
means that
$B$ is a right inverse of $E+T^*T$
provided that the $y$-vectors are restricted to
$\mathcal D(T)$.
\bigskip


FINISH ..



\bigskip





\centerline{\bf{12.2 Positive operators on $C^0(S)$}}

\bigskip

\noindent
Let $S$ be a compact Hausdorff
space and $X$  the Banach space of continuous and
complex-valued functions on $S$.
A linear operator $T$ on $X$ is  positive if
it sends every non-negative and real-valued function $f$ to another  real-valued and
non-negative function.
Denote by $\mathcal F^+$ the family of positive operators
$T$ which satisfy the following: First
\[
\lim_{n\to\infty}\, \frac{1}{n}\cdot x^*(T^nx)=0\tag{1}
\] 
hold for all pairs $x\in X$ and $x^*\in X^*$.
The second condition is that
$\sigma(T)$ is the union of a compact set in a disc
$\{|\lambda|\leq r$ for some $r<1 $, and a 
finite set of points on the unit circle. 
The  final  condition is that
$R_T(\lambda)$
is meromorphic in the exterior disc
$\{|\lambda>r\}$, i.e. it has poles at
the spectral points on the unit circle.








\medskip

\noindent
{\bf{12.2.1. Theorem.}}
\emph{If $T\in\mathcal F^+$ 
then
each spectral value $e^{i\theta}\in \sigma(T)$ is a root of unity.}

\medskip


\noindent
\emph{Proof.}
Frist we prove that $R_T(\lambda)$  has a simple pole at
each  $e^{i\theta}\in \sigma(T)$.
Replacing 
$T$ by  $e^{-i\theta}\cdot T$
it suffices to prove this when
$e^{i\theta}=1$.
If $R_T(\lambda)$ has a pole of order $\geq 2$
at $\lambda=1 $ we know from � XX that there exists $x\in X$ such that
\[ 
Tx\neq  x\quad\text{and}\quad
(E-T)^2\,x=0\tag{i}
\]
This gives  $T^2+x=2Tx$ and by an induction
\[
\frac{1}{n} \cdot T^n x=\frac{1}{n}\cdot x+(E-T)x\quad\colon n=1,2,\ldots\tag{ii}
\]
Condition (1) and (ii) give for each $x^*\in X^*$:
\[ 
0=\lim_{n\to\infty}\, \frac{1}{n}\cdot x^*( T^n x)=\lim_{n\to\infty}\,
x^*(\frac{1}{n}\cdot x+(E-T)x)
\] 
It follows that $x^*(E-T)(x)=0$ and since 
$x^*$ is arbitrary we get   $Tx=x$ which contradicts (i).
Hence
the pole must be simple.
\medskip


\noindent
Next, with $e^{i\theta}\in\sigma(T) $
we have seen that
$R_T$  has a simple pole.
By the general result in � xx there exists
some    $f\in C^0(S)$ which is not identically zero and
\[ 
T(f)=e^{i\theta}\cdot f
\]
Multiplying $f$ with a complex scalar we may assume that
its  maximum norm on $S$ is one and there exists a point $s_0\in S$ such that
\[ 
f(s_0)=1
\]
For each $n\geq 1$ we have a linear functional on
$X$ defined by $g\mapsto T^n(g)(s_0)$ whuch gives a
Riesz measure $\mu_n$ such that
\[
\int_S\, g\cdot d\mu_n=T^n g(s_0)\quad\colon g\in C^0(S)
\]
Since $T^n$ is positive
the integrals in the left hand side are
$\geq 0$ when $g$ are real-valued and non-negative which entails that
the measures $\{\mu_n\}$ are real-valued and non-negative.
For each $n\geq 1 $ we put
\[ 
A_n=\{x\,\colon\, e^{-in\theta}\cdot f(x)\neq 1\}
\]
Since the sup-norm of $f$ is one we notice that
\[ 
A_n=
\{x\,\colon\, \mathfrak{Re}(e^{-in\theta}f(x))<1\}\tag{iii}
\]
Now
\[
0=f(s_0)- e^{-in\theta}\cdot T^n f(s_0)
=\int_S\,[1-e^{-in\theta}f(s)]\cdot  d\mu_n(s)\tag{iv}
\]
Taking real parts we get
\[
0=\int_S\,[1-\mathfrak{Re}(e^{-in\theta}f(s))]\cdot  d\mu_n(s)\tag{v}
\]
By (iii)  the integrand in (v)
is non-negative and since the whole 
integral is zero it follows that
\[
\mu_n(A_n)= \mu_n(\{\mathfrak{Re}(e^{-in\theta}<1\}=0\tag{vi}
\]
Suppose now that
there exists a pair $n\neq m$ such that
\[
(S\setminus A_n)\cap (S_m\setminus A_m)\neq \emptyset\tag{vii}
\] 
A point $s_*$ in this non-empty intersection
gives
\[
1=e^{in\theta}f(s_*)=e^{im\theta}\cdot f(s_*)\implies
e^{in\theta}=e^{im\theta}
\] 
and hence  $e^{i\theta}$ is a root of unity.
$m-n\neq 0$.
So the proof of Theorem 6.1  is finished if we have established the following

\medskip

\noindent
\emph{Sublemma.}
\emph{The sets $\{S\setminus A_n\}$ cannot be pairwise disjoint.}
\medskip

\noindent
\emph{Proof.}
First,  $f$ has maximum norm
and by the above:
\[ 
\int_S\, f\cdot d\mu_n=e^{in\theta}
\]
Hence the total mass $\mu_n(S)$ is at least one.
Next, 
for each $n\geq 2$ we set
\[
\pi_n=\frac{1}{n}\cdot (\mu_1+\ldots+\mu_n)
\]
Since 
$\mu_n(S)\geq 1$  for each $n$ we get 
$\pi_n(S)\geq 1$. Put
\[
\mathcal A=\bigcap\, A_n
\]
Above we proved that $\mu_n(A_n)=0$ hold for every $n$ which gives
\[ 
\pi_n(\mathcal A)=0\quad\colon\, n=1,2,\ldots\tag{*}
\]
Next, when
the sets $\{S\setminus A_k\}$ 
are pairwise disjoint one has 
the inclusions
\[
S\setminus A_k\subset A_\nu\quad\forall\,\nu\neq k
\]
Keeping $k$ fixed it follows that
$\pi_\nu(S\setminus A_k)=0$ for every $\nu\geq 0$.
So when $n$ is large while $k$ is kept fixed we obtain
\[ 
\pi_n(S\setminus A_k))= \frac{1}{n}\cdot \mu_k(S\setminus A_k))\implies
\lim_{n\to\infty}\, \pi_n(S\setminus A_k))=0
\quad\colon k=1,2,\ldots\tag{**}
\]
At this stage we use Lemma xx which shows that 
$R_T(\lambda)$ has at most a simple pole at $\lambda=1$.
With $\epsilon>0$
the Neumann series expansion gives
\[
E+\sum_{k=1}^\infty\,
\frac {T^k}{(1+\epsilon)^k}=
R_T(1+\epsilon))
=\frac{1}{\epsilon} \cdot Q+W(1+\epsilon))
\]
where $W(\lambda)$ is an operator-valued analytic function
in an open disc centered at $\lambda=1$ while $Q$ is a bounded
linear operator on $C^0(S)$.
Keeping $\epsilon>0$ fixed we apply both sides to the identity function
$1_S$ on $S$ and the construction of
the measures $\{\mu_n\}$ gives
\[
1+
\sum_{k=1}^\infty\,
\frac {\mu_k(S)}{(1+\epsilon)^k}=\frac{1}{\epsilon} \cdot Q(1_S)(s_0)+
W(1+\epsilon))(1_S)(s_0)
\]
If $n\geq 2$ is an integer and $\epsilon=\frac{1}{n}$
one gets the inequality
\[
\sum_{k=1}^{k=n}\,
\frac {\mu_k(S)}{(1+\frac{1}{n})^k}\leq
n\cdot |Q(1_S)(s_0)|+|W(1+1/n))(1_S)(s_0)|\leq
n\dot ||Q||+ ||W(1+1/n)||\implies
\]
\[
\frac{1}{n}\cdot\sum_{k=1}^{k=n}\,\mu_k(S)\leq
(1+\frac{1}{n})^n\cdot (||Q||+\frac{ ||W(1+1/n)||}{n}
\]
Since
Neper's constant $e\geq (1+\frac{1}{n})^n$ for every $n$
we find
a constant $C$ which is
independent of $n$ such that
\[
\frac{1}{n}\cdot\sum_{k=1}^{k=n}\,\mu_k(S)\leq C
\]
Hence the sequence $\{\pi_n(S)\}$ is  bounded
and we can pass to a subsequence
which converges weakly to a limit measure $\mu_*$.
For this $\sigma$-additive measure the limit formula in (**) above entails that
\[ 
\mu_*(S\setminus A_k)=0\quad\colon\, k=1,2,\ldots\tag{i}
\]
Moreover, by (*) we also have
\[
\pi_*(\mathcal A)=0\tag{ii}
\]
Now
$S= \mathcal A\cup\, A_k$ so (i-ii) give:
\[ 
\mu_*(S)=0
\]
But this is impossible for at the same time
we have already seen that
$\pi_n(S)\geq 1$ for each $n$ and hence also $\mu_*(S)\geq 1$.









\bigskip
Compact pertubations to finish Kakutani-Yosida !!!



\bigskip



\noindent 
In general, consider some complex Banch space
$X$ be a Banach space and
denote by $\mathcal F(X)$
the family of bounded liner operators  $T$ on $X$  
such that

\[ 
\lim_{n\to\infty}\, \frac{1}{n}\cdot x^*(T^nx)=0\tag{*}
\] 
hold for all pairs $x\in X$ and $x^*\in X^*$.
\medskip


\noindent
{\bf{1. Exercise.}} Apply the Banach-Steinhaus theorem to show that
if $T\in\mathcal F(X)$ then
there exists a constant $M$ such that the operator norms satisfy
\[ 
||T^n||\leq M\cdot n\quad\colon\, n=1,2,\ldots
\]


\noindent
Since the $n$:th root of $M\cdot n$ tends to one as $n\to+\infty$, 
the spectral radius formula entails that
the spectrum $\sigma(T)$ is contained in the closed unit disc
of the complex $\lambda$-plane.
So in the exterior disc
$\{|\lambda|>1\}$ there exists the
the resolvent 
\[ 
R_T(\lambda)= (\lambda\cdot E-T)^{-1}
\]



\noindent{\bf{2. The class $\mathcal F_*$}}.
It consists of those $T$ in $\mathcal F(X)$
for which there exists
some $\alpha<1$ such that
$R_T(\lambda)$ extends to a meromorphic function in
the exterior disc
$\{|\lambda|>\alpha\}$.
Since $\sigma(T)\subset
\{|\lambda|\leq 1\}$
it follows that when
$T\in\mathcal F_*$ then
the set of points in $\sigma(T)$ which belongs to the unit circle
in the complex $\lambda$-plane is empty or finite and
after we can always choose $\alpha<1$ such that
\[ 
\sigma(T)\cap \{\alpha<|\lambda|<1\}= \emptyset
\]

\medskip

\noindent
{\bf{3. Proposition.}}
\emph{If $T\in \mathcal F_*$ and�$e^{i\theta}\in \sigma(T)$
for some $\theta$, then Neumann's resolvent
$R_T(\lambda)$ 
has a simple pole at $e^{i\theta}$.}
 \medskip

\noindent
\emph{Proof.}
Replacing
$T$ by  $e^{-i\theta}\cdot T$
it suffices to prove the result when
$e^{i\theta}=1$.
If $R_T(\lambda)$ has a pole of order $\geq 2$
at $\lambda=1 $ we know from � XX that there exists $x\in X$ such that
\[ 
Tx\neq  x\quad\text{and}\quad
(E-T)^2\,x=0\tag{i}
\]
The last equation means that $T^2+x=2Tx$ and an induction
over $n$ gives
\[
\frac{1}{n} \cdot T^n x=\frac{1}{n}\cdot x+(E-T)x\tag{ii}
\]
Since $T\in\mathcal F$ we have
\[ 
\lim_{n\to\infty}\, \frac{1}{n}\cdot x^*( T^n x)=0
\quad\colon\, \forall\\, x^*\in X^*\tag{iii}
\] 
Then (ii) entails that $x^*(E-T)(x)=0$. Since
$x^*$ is arbitrary we get  $Tx=x$ which contradicts (i) and hence
the pole is simple.
\medskip

\noindent
{\bf{4. Theorem.}}
\emph{Let $T\in \mathcal F(X)$ be such that   there exists a compact
operator $K$ where 
$||T+K||<1$. Then $T\in \mathcal F_*$ and
for every $e^{i\theta}\in\sigma(T)$
the eigenspace
$E_T(\theta)= \{ x\in X\,\colon\, Tx= e^{i\theta} x\}$ is finite dimensional.}

\medskip

\noindent
\emph{Proof.}
Set $S=T+K$ and for a complex number
$\lambda$ we write 
$\lambda\cdot E-T= \lambda\cdot E-T-K+K$. Outside $\sigma(S)$ we get
\[
R_S(\lambda)(\lambda\cdot E-T)=
E+R_S(\lambda)\cdot K\tag{i}
\]
The Neumann series for large absolute values
$|\lambda|$
applied to $R_S(\lambda)$ gives some $\rho>0$ and
\[
( E+R_S(\lambda)\cdot K)^{-1}= E+R_S(\lambda)\cdot K\dot
( E+R_S(\lambda)\cdot K)^{-1}\quad\colon\,|\lambda|>\rho\tag{ii}
\]

\noindent
Next, when
$|\lambda|$ is large
we notice that
(i) gives
\[
R_T(\lambda)=  ( E+R_S(\lambda)\cdot K)^{-1}\cdot R_S(\lambda)\tag{iii}
\]
Together with (ii) we obtain
\[
R_T(\lambda)=R_S(\lambda)+ 
R_S(\lambda)\cdot 
( E+R_S(\lambda)\cdot K)^{-1}\cdot R_S(\lambda)\tag{iv}
\]
Set $\alpha=||S||$ which by assumption is $<1$.
Now
$R_S(\lambda)$ is analytic in the exterior disc
$\{\lambda|>\alpha\}$ so in this exterior disc
$R_\lambda(T)$ differs from  the analytic function $R_\lambda(S)$ by
\[
\lambda\mapsto R_S(\lambda)\cdot 
( E+R_S(\lambda)\cdot K)^{-1}\cdot R_S(\lambda)\tag{v}
\]
Here $K$ is a compact operator so 
the
result in � XX entails that this function extends to
be meromorphic  in $\{|\lambda|>\alpha\}$.
There remains to prove that
eigenspaces at spectral points on the unit circle are finite dimensional.
To prove this we use (iv). 
Let $e^{i\theta}\in\sigma(T)$. By Proposition 3 it is a simple pole
so we have a Laurent series expansion
\[
R_T(e^{i\theta}+z)= \frac{A_{-1}}{z}+ A_0+ A_1z+\ldots
\]
By the general results from �� there remains to show that 
$A_{-1}$ has finite dimensional range.
To see this we apply (iv) which gives the equation
\[
R_S(e^{i\theta}+z)+
R_S(e^{i\theta}+z)\cdot(E+
R_S(e^{i\theta}+z)\cdot K)^{-1}\cdot R_S(e^{i\theta}+z)
\]
To simplify notations we set $B(z)=R_S(e^{i\theta}+z)$
which by assumption is analytic in a neighborhood of $z=0$.
Moreover, the operator $B(0)$ is invertible.
So now one has
\[
 \frac{A_{-1}}{z}+ A_0+ A_1z+\ldots=
 B(z)+B(z)(E+B(z)\cdot K)^{-1}B(z)
 \]
Since $B(0)$ is invertible  we  have a Laurent series expansion
\[
(E+B(z)\cdot K)^{-1}= \frac{A^*_{-1}}{z}+A_0^*+A_1^*z+\ldots
\]
and identying the coefficient of $z^{-1}$ gives
\[
A_{-1}= B(0)A^*_{-1}B(0)
\]
Next, from (xx) one has

\[
E=(E+B(z)\cdot K)(\frac{A^*_{-1}}{z}+A_0^*+A_1^*z+\ldots)\implies
(E +B(0)\cdot K)A^*_{-1}=0
\]
Here $B(0)\cdot K$ is a compact operator and hence  Fredholm theory 
implies that $A^*_{-1}$ has a finite dimensional range. Since
$B(0)$ is invertible the same is true for $A_{-1}$ which finishes the proof
of 
Theorem 4.




\medskip

\noindent
{\bf{5. Proposition}}.
\emph{
If
$T\in\mathcal F$ is  such that
$T^N\in\mathcal F_*$ for some integer $N\geq 2$.
Then 
$T\in\mathcal F_*$.}
\medskip

\noindent
\emph{Proof.}
We have the algebraic equation
\[ 
\lambda^N\cdot E -T^N=(\lambda\cdot E-T)(
\lambda^{N-1}\cdot E+
\lambda^{N-2}\cdot T+\ldots+T^{N_1})
\]
It follows that
\[ R_T(\lambda)=
(\lambda^{N-1}\cdot E+
\lambda^{N-2}\cdot T+\ldots+T^{N_1})\cdot R_{T^N}(\lambda^N)
\]
Since $T^NB\in\mathcal F_*$ there exists
$\alpha<1$ such that
\[
\lambda\mapsto  R_{T^N}(\lambda^N)
\]
extends to be meromorphic in
$\{|\lambda|>\alpha\}$. At the same time
$(\lambda^{N-1}\cdot E+
\lambda^{N-2}\cdot T+\ldots+T^{N_1})$
is  a polynomial
and hence
$R_T(\lambda)$ also extends to be meromorphic
in this
exterior disc so that 
$T\in\mathcal F_*$.




\newpage




\centerline{\bf{12.3 Factorizations of non-symmetric kernels.}}


\bigskip

\noindent
Recall that
the Neumann-Poincar� kernel $K(p,q)$ of  a
plane $C^1$-curve $\mathcal C$
is given by
\[ 
K(p,q)= \frac{\langle p-q,{\bf{n}}_i(p)\rangle}{|p-q|^2}
\]
This kernel function gives the integral operator $\mathcal K$
defined on $C^0(\mathcal C)$ by
\[
\mathcal K_g(p)=\int_C\, K(p,q)\cdot g(q)\, ds(q)
\]
where $ds$ is the arc-length measure on $C$.
Let $M$ be a positive number which exceeds the diameter of 
$\mathcal C$
so that
$|p-q|<M\,\colon\, p,q\in\mathcal C$.
Set
\[
N(p,q)= \int_\mathcal C\, 
K(p,\xi )\cdot \log\,\frac{M}{|q-\xi|}\cdot ds(\xi)
\]



\noindent
{\bf{Exercise.}} Verify that
$N$ is symmetric, i.e. $N(p,q)= N(q,p)$
hold for all pairs $p,q$ in $\mathcal C$.
Moreover, 
\[ 
S(p,q)=\log\, \frac{M}{|p-q|}
\] 
is a symmetric and positive kernel function
and since $\mathcal C$ is of class $C^1$ the reader should verify that it gives a
Hilbert-Schmidt kernel, i.e.
\[ 
\iint_{\mathcal C\times\mathcal C}\,
S(p,q)^2\, ds(p)ds(q)<\infty
\]



\noindent
Hence the
Neuman-Poincar� operator  $\mathcal K$ appears in an
equation
\[ 
\mathcal N= \mathcal K\circ \mathcal S\tag{*}
\] 
where $\mathcal S$ is defined via a positive  symmetric Hilbert-Schmidt kernel and
$\mathcal N$ is symmetric.
Following  
[Carleman: � 11 ] we give a procedure to determine 
the spectrum of $\mathcal K$.


\newpage

\centerline{\bf{12.3.1 Spectral properties of  non-symmetric kernels.}}
\bigskip


\noindent
Let  $K(x,y)$ be a continuous real-valued function on the closed unit square
$\square=\{0\leq x,y\leq 1\}$. We do not assume that  
$K$ is symmetric but 
there exists a positive definite
Hilbert-Schmidt kernel $S(x,y)$ such that
\[ 
N(x,y)=\int_0^1\, S(x,t) K(t,y)\, dy
\] 
yields a symmetric kernel function, i,e, $N(x,y)= N(y,x)$.
The Hilbert-Schmidt theory
gives
an orthonormal basis $\{\phi_n\}$ in
$L^2[0,1]$ formed by eigenfunctions to $\mathcal S$ where
\[ 
\mathcal S\phi_n=\kappa_n \phi_n\tag{1}
\] 
where the positive $\kappa$-numbers  tend to zero, and 
each $u\in L^2[0,1]$
has a Fourier-Hilbert  expansion
\[ 
 u= \sum\,\alpha_n\cdot \phi_n\tag{2}
\]
We  seek 
eigenfunctions of  the integral operator $\mathcal K$.
Let  $u$ be  a function in $L^2[0,1]$
such that:
\[ 
u=\lambda\cdot \mathcal Ku\tag{3}
\]
where
$\lambda$ in general is a complex number. It follows that
\[
\lambda\cdot \int\, N(x,y)u(y)\, dy
=
\lambda\iint S(x,t)K(t,y)u(y)\, dtdy=
\int S(x,t)u(t)\,dt\tag{4}
\]
Take some $p$ and
multiply both sides with
$\phi_p(x)$ and then an integration gives
\[
\lambda\cdot \int\, \phi_p(x)N(x,y)u(y)\, dx dy=
\iint \phi_p(x) S(x,t)u(t)\,dxdt=
\kappa_p\int \phi_p(t)u(t) \, dt\tag{5}
\]


Set
\[
c_{qp}=\iint  \phi_q(x)\phi_p(x)N(x,y)\, dx dy
\]
Since $N(x,y)= N(y,x)$ the
doubly indexed $c$-sequence is symmetric.
Next, the expnasion of $u$ in (2)
shows that the left hand side in (5) becomes
\[
\lambda\cdot \sum\, \alpha_q\cdot c_{pq}
\]
At the same time we nortice that since
the $\phi$-functions is an orthonormal basis, the right hand side in
(5) is $\kappa_p\cdot \alpha_p$.
Hence the $\alpha$-sequence satisfies the system of equations
\[
\kappa_p\cdot \alpha_p= 
\lambda\cdot \sum\, \alpha_q\cdot c_{pq}
\]






Next, the expansion of $u$ from (2) gives
the equations:
\[
\sum_{q=1}^\infty \alpha_q\cdot 
\iint  \phi_q(x)\phi_p(x)N(x,y)\, dx dy=\kappa_p\alpha_p
\quad\colon\, p=1,2,\ldots\tag{6}
\]
\[ 
\beta_p= \sqrt{\kappa_p}\cdot \alpha_p
\implies
\beta_p=\lambda\cdot \sum_{q=1}^\infty
\frac{c_{pq}}{\sqrt{\kappa_p}\cdot \sqrt{\kappa_q}}
\cdot \beta_q\tag{8}
\]



\noindent
Next, 
put
\[
k_{p,q}=\iint K(x,y)\phi_p(x)\phi_q(y)\, dxdy\tag{9}
\]
From  the above  the following hold for each pair $p,q$:

\[ c_{pq}= \iiint\,
\phi_q(x)\phi_p(y) S(x,t)K(t,y)\, dxdydt=
\kappa_qk_{p,q}=\kappa_pk_{q,p}\implies
\]
\[ 
\frac{c_{p,q}^2}{ \kappa_p\cdot \kappa_q}\leq
|k_{p,q}\cdot  k_{q,p}|
\leq \frac{1}{2}(k_{p,q}^2+k_{q,p}^2)\tag{10}
\]
Here $\{k_{p,q}\}$ are the Fourier-Hilbert coefficients of $K(x,y)$
which entails that
\[ 
\sum\sum\, k_{p,q}^2\leq
\iint K(x,y)^2\, dxdy
\]
Hence the symmetric and doubly indexed sequence
\[
\frac{c_{p,q}}{\sqrt{\kappa_p\cdot \kappa_q}}\tag{11}
\] 
is of Hilbert-Schmidt type.
\bigskip

\noindent
{\bf{11.6.2 Conclusion.}}
The eigenfunctions $u$ in $L^2[0,1]$
associated to the $\mathcal K$-kernel have Fourier-Hilbert expansions via
the $\{\phi_n\}$-basis  which are  determined by 
$\alpha$-sequences  satisfying the system (7)


\medskip

\noindent
{\bf{11.6.3 Remark.}}
When a plane curve
$\mathcal C$ has corner points the
Neumann-Poincar� kernel is
unbounded. Here
the reduction to the symmetric case is  more involved  and leads 
to
quite intricate results which appear in Part II from
[Carleman].
The interplay  between singularities
on boundaries in 
the Neumann-Poincar� equation
and the corresponding unbounded kernel functions 
illustrates  the 
general theory
densely defined self-adjoint operators.
Much analysis remains to be done and
open problems about
the Neumann-Poincar� equation 
remains to be settled in dimension three. So far  it appears that
only the 2-dimensional case 
is properly understood via 
results  in [Car:1916].
See also � xx for a studiy of Neumann's boundary value problem
both in the plane and  ${\bf{R}}^3$.



















\end{document}




