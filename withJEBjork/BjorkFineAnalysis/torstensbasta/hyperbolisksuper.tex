

\documentclass{amsart}


\usepackage[applemac]{inputenc}

\addtolength{\hoffset}{-12mm}
\addtolength{\voffset}{-10mm}
\addtolength{\textheight}{20mm}


\begin{document}


\centerline{\bf{Stochastic  processes in contionous time
and their parabolic PDE:s}}



We consider a stiochastic  differential equation of the form

\[
dX_t= b(t,X_t)\cdot dW\tag{*}
\]
where $b(t,x)$ is a real-valued  functions defined when
$t\geq 0$ and $x$ is real.
For each $t>0$ we denote by $f(t,x)$ be the frequency function of
the stochastic variable $X_t$, and when $\Delta>0$ 
$g_\Delta$ is the Gaussian kernel which gives the frequency function of
a normal variable with mean-value zero and variance $\Delta$.
Now (*) means that
\[
X_{t+\Delta}= X_t+ b(X_t,t)\cdot \mathcal N_\Delta+
\text{small ordo}(\Delta)\tag{0.1}
\]
More precisely, the small ordo term means that it is a stochastic variable whose
mean value and variance both are  small ordo of $\Delta$.
Introduce the characteristic function
\[
\widehat{f_t}(\xi)= \int e^{ix\xi}\cdot f(t,x)\, dx
\]
By assumption 
the stochastic variables $b(X_t,t)$ and $\mathcal N_\Delta$
are independent.
So up to small ordo of $\Delta$
the charactersistic function of the right hand side becomes
\[
\iint\, e^{i\xi(x+b(t,x)\cdot \sqrt{\Delta}y)}
\cdot f(t,x)\cdot g_\Delta(y)\, dxdy\tag{ii}
\]
A Taylor expansion of
$e^{i\xi (x+b(x,t)\cdot \sqrt{\Delta}y)}$ when $\Delta$ is small implies that
(ii) up to small ordo of $\Delta$ is equal to
\[
\iint\, e^{ix\xi}\cdot (1+i\xi b(t,x)\cdot \sqrt{\Delta}y-\xi^2/2
\cdot b(t,x)^2\cdot \Delta\cdot y^2)
\cdot f(t,x)\cdot g_\Delta(y)\, dxdy\tag{iii}
\]
Since $g_\Delta(y)$ has mean value zero and
\[ 
\int\, \Delta\cdot y^2g_\Delta(y)\,dy=1
\]
we see tahat (iii) - as usual up to small ordo $\Delta$ - becomes
\[
\int\, e^{ix\xi}\,f(t,x)\, dx-\Delta\cdot \xi^2/2\cdot \int
\, e^{ix\xi}\,
b(t,x)^2\cdot f(t,x)\, dx
\]
Passing to the limit as $\Delta\to 0$ and using Fourier's inversion formula we obtain
\[
\frac {\partial}{\partial t} (f(t,x))=\frac{1}{2}\cdot 
\frac{\partial^2}{\partial x^2}(b(t,x)^2\cdot f(t,x))\tag{**}
\]
\medskip


\noindent
{\bf{Remark.}}
Above (**) is  a parabolic partial differential equation.
The deduction of (**) from (*)  goes back to work
by Fourier and Laplace. The study of equations such as (**)
were performed by Hadamard, Gevrey and Holmgren around 1900,
and
later work which has led to many interesting and deep facts about
parabolic equations are exposed in many text-books
devoted by PDE-theory. An excellent introduction is
the book by the eminent mathematician Petrovsky and let us remark
that the interplay with random processes was
recognised
and used by him and his collaborators Khintchine and Kolmogorov. As a personal remark I want to add that I cannot see
any novely in the  article by Ito from 1950, except  that it
reproduces Fourier's inversion formula.
As a graduate student I learnt the presentation in text-books
by the eminent Russian mathematicans above, while the so called "Ito-calculus"
even today after my
studies
meany decades ago  
appears for me as 
a nuisance which  tends to make matters more compicated than they are.
\medskip


 









\newpage




\centerline{\bf{Green's function for  domains in ${\bf{C}}$.}}
\bigskip


\noindent
Let $\Omega$ be a connected domain in
${\bf{C}}$ which belongs to the family  $\mathcal D(C^1)$.
Then Dirichlet's problem is solvable and Riesz's representation formula
gives for each $p\in\Omega$ a unique probability measure
$\mu_p$ on $\partial\Omega$ such that
\[
\Phi(p)= \int_{\partial\Omega}\, \phi(q)d\mu_p(q)
\] 
for each $\phi\in C^0(\partial\Omega)$
where $\Phi$ is its unique harmonic extension to
$\Omega$.
In particular, for each $p\in\Omega$
 we solve the Dirichlet problem for
 boundary value functions
\[
q\mapsto \log\,\frac{1}{|p-q|}
\]
This gives the  unique harmonic extension
\[
H_p(q)= \frac{1}{2\pi}\cdot  \int_{\partial\Omega}\, 
\log\, \frac{1}{|p-q|}\cdot d\mu_p(q)\tag{1}
\]
Set
\[
 G(p,q)=\frac{1}{2\pi}\cdot \log\,\frac{1}{|p-q|}-H_p(q)\tag{2}
\]
From the above the $G$-function is defined
for all pairs $(p,q)$ in $\Omega$ and  for each $p\in\Omega$ we have
\[
G(p,q)=0\quad\colon q\in\partial\Omega
\]
We refer to $G(p,q)$ as Green's function.
With $p$ fixed we notice that the function
\[
q\mapsto  \frac{1}{2\pi}\cdot \log\,\frac{1}{|p-q|}
\] 
is super-harmonic in
$\Omega$, i.e. its negative is subharmonic.
The minimum principle for superharmonic functions entails that
$G(p,q)>0$ for all pairs $p,q$ in $\Omega$.
Next,  recall that when $p$ is fixed then
the locally integrable function
$q\mapsto  \frac{1}{2\pi}\cdot \log\,|p-q|$ 
 is a fundamental solution to the Laplcae operator.
 \medskip
 
 \noindent
 {\bf{Exercise.}}
 Use the above to show that when
 $g$ is a function in $C^2(\bar\Omega)$, then
 \[
 g(p)=\int_{\partial\Omega}\, \frac{\partial G(p,q)}{\partial n_*}\cdot
 g(q)\cdot ds(q)+ \iint_\Omega\, G(p,q)\cdot \Delta(g)(q)\, dq\tag{*}
 \]
 Here we have taken the inner normal derivative of
 $q\mapsto G(p,q)$ along
 $\partial\Omega$
 and $ds$ refers to the arc-length measure.
 Finally, the second
 term is an area integral over
 $\Omega$.
 In the specia. case when
 $g$�is harmonic  in
 $\Omega$ we obtain
 \[
g(p)=\int_{\partial\Omega}\, \frac{\partial G(p,q)}{\partial n_*}\cdot 
 g(q)\cdot ds(q)\quad\colon\, p\in\Omega
 \]
 Since (*)  hold for every such $g$-function
  and $C^2$-functions are dense in the space of continuous functions, 
 we conclude:
  
  \medskip
  
  \noindent
  {\bf{0.1 Theorem.}} \emph{For each $p\in\Omega$ one has the equality}
  \[
  \mu_p=\frac{\partial G(p,q)}{\partial n_*}\cdot ds(q)
 \]
  \medskip
  
 \noindent
{\bf{Remark.}} In particular $\mu_p$ is absolutely continuous with respect to
 the arc-length measure.
   \medskip
  
 \noindent
{\bf{0.2 The $\Delta$-operator on $L^2(\Omega)$.}}
We have the complex Holbert space of square-integrable functions
in $\Omega$. Here $C_0^\infty(\Omega)$ appears as a dense subspace
and Stokes theorem gives
\[
\iint_\Omega\, \Delta(g)\cdot \overline{f}\, dq=
\iint_\Omega\, g\cdot \overline{\Delta(f)}\, dq
\] 
for all pairs of test-functions.
It means that $\Delta$ yields a densely defined and symmetric linear operator on
$L^(\Omega)$.
By definition the adjoint operator
$\Delta^*$ has a domain  of definition which consists of
$L^2$-functions $\phi$ for which there exists a constant $C(\phi)$ such that
\[
|\iint_\Omega\, \Delta(g)\cdot \overline{\phi}\, dq|\leq
C(\phi)\cdot ||g||_2\tag{0.2.1}
\]
hold for all test-functions $g$, where the last term is the $L^2$-norm of $g$ in
$L^2(\Omega)$.
Concerning the densely defined operatopr
$\Delta$ its domain of definition consists of 
$L^2$�functions $\phi$ in $\Omega$ such that
$\Delta(\phi)$ in the sense of distribution theory is locally integrable, and
moreover its $L^2$-integral taken over $\Omega$ is finite.
By general facts in distribution theory the graph of
$\Delta$ is closed.
Less obvious is the following:




\medskip

\noindent
{\bf{0.2.2 Theorem.}}
\emph{The closed and densely defined operator
$\Delta$�is self-adjoint, i.e. we have the equality}
\[
\Delta=\Delta^*
\]
\medskip

\noindent
{\bf{Remark.}} The  proof  amounts to show that
if $\phi\in L^2(\Omega)$ satisfies (0.2.1)  
then it belongs to $\mathcal D(\Delta)$, i.e.
(0.2.1 ) implies that
$\Delta(\phi)$ taken in the distribution sense belongs to $L^2(\Omega)$.
\medskip

\noindent
{\bf{0.2.3 How to prove Theorem 0.2.2.}}
The crucial point is to regard the linear operator
on $L^2(\Omega)$ defined by
\[
\mathcal G_\phi(p)=
\iint_\Omega\, G(p,q)\cdot \phi(q)\, dq
\]
\medskip

\noindent
{\bf{Exercise.}} Show that $\mathcal G$ is a bounded
operator and its range is equal 
$\mathcal D(\Delta)$. More precisely 
\[
\Delta\circ \mathcal G_\phi=-\phi\quad\colon
\phi\in L^2(\Omega)\tag{0.2.4}
\]
Finally, use this to prove Theorem 0.2.2

\medskip
 
\noindent
 The equation (0.2.4 ) means that
 $\mathcal G$ is Neumann's resolvent of $-\Delta$. So if
 $\phi\in \mathcal D(\Delta)$ we also have
 \[
 \Delta(\mathcal G_\phi)= -\phi
 \]
\medskip
 
\noindent
{\bf{0.2.4 Eigenfunctions.}}
The densely defined self-adjoint operator
$\delta$ on $L^2(\Omega)$ has a discrete spectrum
which consists of a non-decreasing sequence
of positive real numbers
$\{0<\lambda_1\leq \lambda_2\leq \ldots\}$.
The associate eigenfunctions
$\{\phi_n\}$ can be chosen so that they give an
orthonormal basis for $L^2(\Omega)$. So here
\[
\Delta(\phi_n)+\lambda\cdot \phi_n=0\tag{i}
 \]
 This entails that
 \[
 \phi_n(p)= \lambda_n\cdot \ \int\, G(p,q)\cdot \phi_n(q)\, dq
\]
The symmetry of $G$ entails
that
each $\phi_n$  extends to a continuous function on
the closurse of $\Omega$  where $\phi_n=0$ on the boundary.
Next, we have an expansion
\[
G(p,q)= \sum\, c_n\phi_n(p)\cdot \phi_n(q)
\]
From (i) we see that
\[
 c_n=\frac{1}{\lambda_n}\tag{ii}
\]



 
 
 








  
  
  
 
 
 
 
 
 








\newpage 

\centerline{\bf{A hyperbolic equation.}}

\bigskip

\noindent
{\bf{Introduction.}}
We shall expose a result due to Friedrichs from the article \emph{xxx}.
Here we shall consider hyperbolic equations of
one space variable while the general case is treated in
� xx. The 
boundary value equation in dimension one is as follows:
Let $x,s$ be coordinates in
${\bf{R}}^2$ and consider the rectangle
\[
\square=\{(x,y)\colon\, 0\leq x\leq \pi\colon\, 0\leq s\leq s^*\}
\]
where $s^*>0$.
A continuous and real-valued function $g(x,s)$ in $\square$ is $x$-periodic if
\[
g(0,s)=g(\pi,s)\quad\colon 0\leq s\leq s^*\
\]
More generally, if $k\geq 1$ and $g(x,s)$ 
belongs to $C^k(\square)$ then
it is $x$-periodic if
\[
\partial_x^\nu(g(0,s))=
\partial_x^\nu(g(\pi,s)\tag{i}
\] 
hold for each $0\leq \nu\leq k$.
In particular we can consider real-valued $C^\infty$-functions on
$\square$ for which (i) hold for every $\nu\geq 0$.
Let $a(x,s)$ and $b(x,s)$ be a pair real-valued $C^\infty$-functions on
$\square$ which are periodic in $x$. Consider the PDE-operator
\[ 
P=\partial_s-a\cdot \partial_x-b
\]
\medskip

\noindent
{\bf{The  boundary value problem.}}
Let $p\geq 1$
and $f(x)$ is a periodic function on
$[0,\pi]$ which is $p$-times continuously differentiable.
We seek $F(x,s)\in C^p(\square)$
which is $x$-periodic and satisfies
$P(F)=0$ in $\square$ and the 
initial condition
\[ 
F(x,0)= f(x)
\]
\medskip


\noindent
We are going to prove that this
boundary value equation has a unique solution $F$�for every
$f$.
Notice that the regularity is expressed by $p$, i.e
one has a specific boundary value problem for each positive integer $p$.
The proof  requires several steps and is not finished unitl
� 4.
A crucial result of independent interest occurs ion
� 3 where
we encounter certain densely defined linear operators
on Hilbert spsces of the Sobolev type.

\bigskip

\centerline{\bf{� 1. Differential inequalities.}}

\bigskip

\noindent
Let $M(s)$ be a non-negative real-valued continuous function
on a closed interval $[0,s^*]$.
To each $0\leq s<s^*$
we set
\[
d_M^+(s)=\limsup_{\Delta s\to 0}\, \frac{M(s+\Delta s)-M(s)}{\Delta s}
\]
where $\Delta s$ are positive during the limit.
\medskip

\noindent
{\bf{1.1 Proposition.}} \emph{Let $B$ be a real number such that
$d_M^+(s)\leq B\cdot M(s)$ holds in $[0,s^*)$. Then }
\[ 
M(s)\leq M(0)\cdot e^{Bs}\quad\colon 0<s\leq s^*
\]
\medskip

\noindent
The proof of this result is left as an exercise.
The hint is to consider the function $N(s)= M(s)e^{-Bs}$
and show that $d^+_N(s)\leq 0$ for all $s$.
Notice that $B$ is an arbitrary real number, i.e. it may also be $<0$.
More generally, let $k(s)$ be a  non-decreasing continuous function
with $k(0)=0$. 
suppose that
\[
d^+_M(s)\leq B\cdot M(s)+k(s)\quad \colon 0\leq s<s^*
\]
Now the reader may verify that
\[
M(s)\leq M(0)\cdot e^{Bs}+\int_0^s\, k(t)\, dt\tag{1.1.1}
\]


\medskip


\noindent
Next, consider
a product set
$\square=[0,\pi]\times [0,s^*]$
where $0\leq x\leq\pi$.
A $C^1$-function
$g$ is  
periodic  with respect to $x$ if
$g$  and the partial derivatives
$\partial_s(g)$ and $\partial_x(g)$  are  periodic in
$x$, i.e.
\[ 
g(0,s)= g(\pi,s)\quad\colon 0\leq s\leq s^*
\]
and similarly  for $\partial_x(g)$ and $\partial_s(g)$.



\medskip

\noindent
{\bf{1.2 Theorem.}}
\emph{Let $g$ be a periodic $C^1$-function which satisfies the PDE-equation}
\[
\partial_s(g)= a\cdot \partial_x(g)+ b\cdot g\tag{*}
\]
\emph{in $\square$ where $a$ and $b$ are 
$x$-periodic real-valued continuous functions
on $\square$.. Set}
\[
M_g(s)= \max_x\, |g(x,s)|\quad\colon \,B=\max_{x,s}\, |b(x,s)|
\]
\emph{Then one has the inequality}
\[
M_g(s)\leq M_g(0)\cdot e^{Bs}
\]

\bigskip

\noindent
\emph{Proof.}
Consider some $0<s<s^*$ and let $\epsilon>0$.
Put
\[ 
m^*(s)=\{ x\,\colon\, g(x,s)= M_g(s)\}
\]
The  continuity of $g$
entails that the function $M_g(s)$ is continuous and
the sets $m^*(s)$ are compact.
If $x^*\in m^*(s)$ the periodicity of
the
$C^1$-function $x\mapsto g(x,s)$
entails that
$\partial_x(x^*,s)=0$ and (*) gives
\[
\partial_s(g)(x,s)=b(x,s)g(x,s)\quad\colon x\in m^*(s)
\]
Next, let $\epsilon>0$. We find an open neighborhood $U$
of $m^*(s)$
such that
\[
|\partial_x(g)(x,s)|\leq \epsilon\quad\colon x\in U
\]
Now there exists
$\delta>0$ such that
\[
|g(x,s)|\leq M_g(s)-2\delta\quad\colon x\in [0,\pi]\setminus U
\]
Continuity gives  some
$\rho>0$ such that
if $0<\Delta s<\rho$ then  the inequalities below hold:
\[
|g(x,s+\Delta s)|\leq M_g(s)-\delta\quad\colon x\in [0,\pi]\setminus U
\quad\colon\,M_g(s+\Delta s)>M_g(s)-\delta\tag{i}
\]
\[
M_g(s+\Delta s)\leq M_g(s)+\epsilon\quad \colon\,
|\partial_x(g)(x,s+\Delta s)|\leq 2\epsilon \quad\colon x\in m^*(s)\tag{ii}
\]

\noindent
If $0<\Delta s<\rho$ we see that (i) gives
$x\in m^*(s+\Delta s)\subset U$
and for such $x$-values 
Rolle's mean-value theorem and the PDE-equation give
\[
M_g(x,s+\Delta s)- g(x,s)=\Delta s\cdot \partial_s(g(x,s+\theta\cdot \Delta s)=
\] 
\[
\Delta s\cdot \bigl[a(x,s+\Delta s)\cdot
\partial_x(g)(x+\theta\cdot \Delta s)+
b(x,s+\Delta s)\cdot
g(x,s+\theta\cdot \Delta s)\bigr]\tag{iii}
\]
Let  $A$ be the maximum norm of $|a(x,s)|$ taken over
$\square$.
Since $|g(x,s)|\leq M_g(s)$
the triangle inequality and (iii) give
\[ 
M_g(s+\Delta s)\leq M_g(s)+\Delta s[\cdot A\cdot 2\epsilon+
B\cdot M(s+\theta\cdot \Delta s)]
\]
Since the function $s\mapsto M_g(s)$ is continuous
it follows that
\[
\limsup_{\Delta s\to 0}\,
\frac{M_g(s+\Delta s)-M_g(s)}{\Delta s}\leq
A\cdot 2\epsilon+ BM_g(s)
\]
Above $\epsilon$ can be arbitrary small
and hence
\[ 
d^+(s)\leq B\cdot M_g(s)
\]
Then Proposition 1.1 gives (*) in the theorem.


\bigskip

\noindent
{\bf{1.3 $L^2$-inequalities.}}
Let $g(x,s)$ be a $C^1$-function satisfying (*)
in
Theorem 1.2.
Set
\[ 
J_g(s)=\int_0^\pi\, g^2(x,s)\, dx
\]
Taking the $s$-derivative we obtain
 with respect to $s$ and (*) give
\[
\frac{dJ_g}{ds}= 2\cdot \int_0^\pi\, g\cdot \partial_s(g)\,ds=
2\cdot \int_0^\pi\, (a\partial_x(g)\cdot \partial g+ b\cdot g)\,dx
\]
The periodicity of $g$ with respect to $x$ gives
$\int_0^\pi\, \partial_x(ag^2)\, dx=0$. This
entails that the right hand side becomes
\[
\int_0^\pi\, (-\partial_x(a)+b)\cdot g^2\, dx
\]
So if $K$ is the maximum norm of
$-\partial_x(a)+b$ over $\square$ it follows that
\[
\frac{dJ_g}{ds}(s)\leq K\cdot J_g(s)
\]
Hence Theorem 1.2 gives
\[
\int_0^\pi\, g^2(x,s)\, dx\leq e^{Ks}\cdot
\int_0^\pi\, g^2(x,0)\, dx\quad\colon 0<s\leq s^*\tag{1.3.1}
\]
Integration with respect to $s$ entails that
\[ 
\iint_\square\, g^2(x,s)\, dxds\leq
\int_0^{s^*}\, e^{Ks}\,ds\cdot
\int_0^\pi\, g^2(x,0)\, dx\tag{1.3.2}
\]
Thus, the $L^2$-integral of $x\to g(x,0)$
majorizes both the area integral and each slice integral when
$0<s\leq s^*$.


\bigskip

\centerline{\bf{� 2. A boundary value equation}}
\bigskip

\noindent
Let $a(x,s)$ and $b(x,s)$ be real-valued $C^\infty$-functions on
$\square$ which are periodic in $x$. Consider the PDE-operator
\[ 
P=\partial_s-a\cdot \partial_x-b
\]
Given a periodic $C^1$-function $f(x)$ on $[0,\pi]$
we seek a periodic $C^1$-function $g(x,s)$ in $\square$ which satisfies $P(g)=0$ and the
initial condition
\[ 
g(x,0)= f(x)
\]
\medskip

\noindent
{\bf{2.1 Theorem.}}
\emph{For every positive integer $p$ and each
periodic $f\in C^p[0,\pi]$
there exists a unique periodic $g\in C^p(\square)$ 
where $P(g)=0$ and $g(x,0)= f(x)$.}

\medskip

\noindent
The uniqueness follows from the results in � 1.
For if $g$ and $h$ are solutions in Theorem 2.1
then $\phi=g-h$   satisfies $P(\phi)=0$. Here
$\phi(x,0)=0$ which gives $\phi=0$ in $\square$
via
(1.3.2).
The proof of existence requires several steps and 
employs Hilbert space methods. So first we
introduce
certain Hilbert spaces.

\medskip

\noindent
{\bf{2.2 The space $\mathcal H^{(k)}$}}.
To each  integer $k\geq 2 $ 
the complex Hilbert space
$\mathcal H^{(k)}$ defined is the
completion  
of complex-valued
$C^k$-functions on 
$\square$ which are periodic with respect to $x$.
A wellknown  Sobolev inequality  entails that 
every function in
$\mathcal H^{(2)}$ is continuous, and more generally
one has the inclusion
\[
\mathcal H^{(k)}\subset C^{k-2}(\square)\quad\colon k\geq 3
\]
It is also clear that the  first order PDE-operator $P$ maps
$\mathcal H^{(k+1)}$ into
$\mathcal H^{(k)}$.
\medskip

\noindent
Next,
on the periodic $x$-interval $[0,\pi]$ we have  the Hilbert space
$\mathcal H^k[0,\pi]$.
\medskip

\noindent
{\bf{2.3 Definition.}}
\emph{For each  $k\geq 2$ we denote by
$\mathcal D_k(P)$   the  family of functions 
$f(x)\in\mathcal H^k[0,\pi]$ for which 
there exists some
$F(x,s)\in\mathcal H^{(k)}$ such that}
\[
P(F)=0\quad\colon\, F(x,0)=f(x)\tag{*}
\]
The results in � 1 show that $F$ is uniquely determined by (*). Moreover.
there exists a constant $C$ which only depends upon the
$C^\infty$-functions $a$ and $b$ and the given integer $k$
such that
\[
||F||_k\leq C\cdot ||f||_k\tag{2.3.1}
\]
where we have taken norms in
$\mathcal H^{(k)}$ and 
$\mathcal H^k[0,\pi]$ respectively.
Moreover, the last inequality in (1.3.2)
shows that
$C$ can be chosen such that we also have
\[
||f^*||_k\leq C\cdot ||f||_k\tag{2.3.3}
\]
where $f^*(x)= F(x,s^*)$.

\medskip

\noindent
{\bf{2.4 A density principle}}
Above we introduced the space
$\mathcal D_k(P)$.
It turns out that if it is dense in
$\mathcal H^k[0,\pi]$ then one has the equality
\[
\mathcal D_k(P)=\mathcal H^k[0,\pi]\tag{2.4.1}
\]

\medskip

\noindent
\emph{Proof.}
Suppose that
$\mathcal D_k(P)$ is dense.
So if $f\in\mathcal H^k[0,\pi]$ there exists a sequence
$\{f_n\}$ in 
$\mathcal D_k(P)$ where $||f_n-g||_k\to 0$.
By (2.2.2)   we have
\[ 
||F_n-F_m||_k\leq C||f_n-f_m||_k
\]
Hence $\{F_n\}$ is a Cauchy sequence in
the Hilbert space $\mathcal H^{(k)}$
and converges to a limit $F$.
Since each $P(F_n)=0$ it follows that
$P(F)=0$
and it is clear that the continuous boundary value function
$F(x,0)$�is equal to
$f(x)$ which entails that
$f$ belongs to 
$\mathcal D_k(P)$.
\medskip

\noindent
{\bf{2.5 The operators $S_k$.}}
Each $f\in \mathcal D_k(P)$ gives the function  $f^*(x)= F(x,s^*)$
in $\mathcal H^k[0,\pi]$ and  set
\[ 
S_k(f)=f^*(x)
\]
So  the domain of definition of $S_k$ is equal to
$\mathcal D_k(P)$ and  (2.3.3) gives  a constant
$M_k$ such that
\[
||S_k(f)||\leq M_k\cdot ||f||_k\quad\colon f\in  \mathcal D_k(P)
\]
where $M_k$ by the above depends on the integer $k$ and the given
PDE-operator $P$.
\medskip

\noindent
{\bf{2.6 Proposition.}}
\emph{For each $k$ there exists some
$\alpha(k)<0$ such that for every $0<\alpha<\alpha(k)$
the range of the operator $E-\alpha\cdot S_k$ contains
all periodic $C^\infty$-functions on
$[0,\pi]$.}
\medskip

\noindent
{\bf{2.7 The density of $\mathcal D_k(P)$.}}
We prove Proposition 2.6 in � xx  and proceed to 
that it gives the density of
$\mathcal D_k(P)$.
For if $\mathcal D_k(P)$ fails to be dense there exists
a non-zero $f_0\in\mathcal D_k(P)$ which is
$\perp$ to $\mathcal D_k(P)$.
In Proposition 2.6 we choose $0<\alpha\leq \alpha(k)$ so small that
\[
\alpha<M_k/2\tag{i}
\]
Since periodic $C^\infty$-functions are dense in
$\mathcal H^k[0,\pi]$,
 Proposition 2.6 gives  a sequence
$\{h_n\}$ in $\mathcal D_k(P)$
such that
\[ 
\lim_{n\to \infty}\, ||h_n-\alpha\cdot S_k(h_n)-f_0||_k\to 0\tag{ii}
\]
It follows that
\[
\langle f_0,f_0\rangle=1=\lim\, 
\langle f_0,h_n-\alpha\cdot S_k(h_n)\rangle=
-\alpha\cdot \lim\, \langle f_0,S_k(h_n)\rangle\tag{iii}
\]
Next, the triangle inequality and (ii) give
\[
||h_n||_k\leq 1+\alpha\cdot ||(S_k(h_n)||
\leq 1+1/2\cdot||h_n||\implies
||h_n||_k\leq 2\tag{iv}
\]
Funally, by the Cauchy-Schwarz inequality the absolute value in
the right hand side of (iii) is majorized by
\[
\alpha\cdot M_K\cdot 2<1
\] 
which contradicts (iii).
Hence the orthogonal complement of $\mathcal D_k(P)$ is zero
which proves the requested density.
\medskip


\noindent
Together with (2.4) we get
the following conclusive result:
\medskip

\noindent
{\bf{2.8 Theorem.}}
\emph{For each  $k\geq 2$
and $f(x)\in \mathcal H^k[0,\pi]$
there exists a unique function
$F(x,s)\in \mathcal H^{(k)}$ such that (*) holds in
Definition 2.3.}





\bigskip

\centerline{\bf{� 3. A class of inhomogeneous PDE-equations.}}


\medskip

\noindent
Before Theorem 3.1 is announced we introduce some notations.
Put
\[ 
\square =\{ 0\leq x\leq \pi\}\times
\{0\leq s\leq 2\pi\}
\]
In this section we  shall consider doubly periodic functions
$g(x,s)$ on $\square$, i.e. 
\[
g(\pi,s)= g(0,s)\quad\colon g(x,0)= g(x,2\pi)
\]
For each non-negative integer
$k$
we denote by $C^k(\square)$ the space of $k$-times
doubly periodic continuously differentiable functions. 
If 
$g\in C^k(\square)$
we set
\[ 
||g||^2_{(k)}= \sum_{j,\nu}\, \int_\square
\bigl|\frac{\partial^{j+\nu}g}{\partial x^j\partial s^\nu}(x,s)\bigr|^2\, dxds
\]
with the double sum extended  pairs  $j+\nu\leq k$.
This gives the complex Hilbert space $\mathcal H^{(k)}$
after a completion of $C^k(\square)$ with respect
to the norm  above.
Recall from � xx that every function $g\in\mathcal H^{(2)}$
is automatically continuous and doubly periodic  on
the closed square.
More generally, if $k\geq 3$ 
each
$g\in\mathcal H^{(k)}$ has continuous and doubly periodic derivatives
up to order $k-2$.
Next, consider a first order PDE-operator
\[
P=\partial_s-a(x,s)\partial_x-b(x,s)
\]
where $a$ and�$b$ are real-valued doubly 
periodic $C^\infty$-functions.
It is clear that
$P$ maps
$\mathcal H^{(k)}$ into $\mathcal H^{(k+1)}$
for every $k\geq 2$.
Keeping $k\geq 2$ fixed we set

\[
\mathcal D_k(P)=\{g\in\mathcal H^{(k)}\,\colon\,
P(g)\in\mathcal H^{(k)}\}\tag{3.1}
\]
Since $C^\infty(\square)$
is dense in
$\mathcal H^{(k)}$ this yields for each $k\geq 2$ a densely defined operator
\[
P\colon \mathcal D_k(P)\to \mathcal H^{(k)}\tag{i}
\]
In
$\mathcal H^{(k)}\times \mathcal H^{(k)}$ we get the graph
\[
\Gamma_k=\{(g,P(g)\colon\, g\in\mathcal D_k(P)\}
\]
Since $P$ is a differential operator the general result in
� xx entails that
$\Gamma_k$ is a closed subspace so the densely defined operator in
(i)  has a closed graph. Thus. for each
$k\geq 2$ we have a densely defined linear operator
and closed operator on $\mathcal H^{(k)}$
denoted by $\mathcal T_k$.
So its domain of definition  $\mathcal D(T_k)= \mathcal D_k$.
Next, we consider the graph
\[
\gamma_*=\{(g,P(g)\colon\, g\in C^\infty(\square)\}\tag{ii}
\]
This is a subspace of
$\Gamma_k$ and 
denote by
$\overline{\gamma}_k$
its closure taken in  $\mathcal H^{(k)}\times \mathcal H^{(k)}$.
So here
\[
\overline{\gamma}_k\subset\Gamma_k
\]
and this inclusion yields 
another densely defined linear operator
denoted by
$T_k$ whose graph is
$\overline{\gamma}_k$. So here
$\mathcal T_k$ is an extension of
$T_k$ and we have an inclusion
\[
\mathcal D(T_k)\subset\mathcal D(\mathcal T_k)\tag{iii}
\] 
in general is strict.
Let $E$ be the identity operator
on
$\mathcal H^{(k)}$.
 With these notations one has

\medskip

\noindent 
{\bf{3.2  Theorem.}}
\emph{For each integer $k\geq 2$ there exists
a positive real number $\rho(k)$ such that
$T_k-\lambda\cdot E$ is surjective on
$\mathcal H^{(k)}$ for every $\lambda>\rho(k)$
and its kernel is zero.}
\medskip



\noindent
The proof requires several steps and is not finished until
� 3.x. First we establish the following:
\medskip

\noindent
{\bf{3.3  Proposition.}}
\emph{One has the equality $\mathcal D(T_k^*)= \mathcal D_k$
and there exists a bounded self-adjoint operator
$B_k$ on
 $\mathcal H^{(k)}$ such that}
 \[
 T_k^*=-\mathcal T_k+B_k
 \]
 

\medskip




\noindent{\emph{Proof of Proposition 3.3}}
Keeping $k\geq 2$ fixed we set $\mathcal H=\mathcal H^{(k)}$.
For each pair $g,f$ in $\mathcal H$
their inner product is defined by
\[
\langle f,g\rangle=
 \sum\, \int_\square
\frac{\partial^{j+\nu}f}{\partial x^j\partial s^\nu}(x,s)
\cdot
\overline{\frac{\partial^{j+\nu}g} {\partial x^j\partial s^\nu}}(x,s)
\, dxds
\]
where the sum is taken when
$j+\nu\leq k$.
Introduce the differential operator
\[
\Gamma= \sum_{j+\nu\leq  k}\, (-1)^{j+\nu}\cdot \partial_x^{2j}\cdot \partial_s^{2\nu}
\]
Partial integration gives
\[
\langle f,g\rangle=\int_\square\, f\cdot \Gamma(\bar g)\, dxds=
\int_\square\, \Gamma(f)\cdot\bar g\, dxds
\quad\colon\, f,g\in C^\infty
\tag{i}
\]
Now we consider the operator
$P=\partial_s-a\cdot\partial_x-b$ and 
get 
\[
\langle P(f),g\rangle=\int_\square\, P(f)\cdot \Gamma(\bar g)\,dxds\tag{ii}
\]
Partial integration identifies (ii) with
\[
-\int_\square\, f\cdot \bigl(\partial_s-\partial_x(a)-a\cdot\partial_x-b)\circ\Gamma(\bar g)\,dxds\tag{iii}
\]


\noindent
{\bf{1.1 Exercise.}}
In (iii)  appears the composed differential operator
\[
\partial_s-\partial_x(a)-a\cdot\partial_x-b)\circ\Gamma
\]
Show that in the ring of differential operators with
$C^\infty$-coefficients this differential operator can be written
in the form
\[
\Gamma\circ(\partial_s-a\cdot\partial _x-b)+Q(x,s,\partial_x,\partial_s)
\]
where $Q$ is a differential of order $\leq 2k$ 
with coefficients in $C^\infty(\square)$.
Conclude from the above that
\[
\langle Pf,g\rangle=-
\langle f,Pg\rangle+\int_\square\ f\cdot Q(\bar g)\,dxds\tag{1.1.1}
\]
\medskip

\noindent
{\bf{1.2 Exercise.}}
With $Q$ as above we have a bilinear form which sends a pair
$f,g$ in $C^\infty(\square)$ to 
\[
\int_\square\ f\cdot Q(\bar g)\,dxds\tag{1.2.1}
\]
Use partial integration and
the
Cauchy-Schwarz inequelity to show that
there exists a conatant $C$ which depends on $Q$ only such that
the absolute value of (1.2.1) is majorized by
$C_Q\cdot ||f||_k\cdot ||g||_k$.
Conclude that
there exists a bounded linear operator 
$B_k$ on $\mathcal H$ such that
\[
\langle f,B_k(g)\rangle=\int_\square\ f\cdot Q(\bar g)\,dxds\tag{1.2.2}
\]

\medskip

\noindent
{\bf{1.3 Proof that $B_k$ is self-adjoint}}
From the above we have
\[
\langle Pf,g\rangle=
-\langle f,Pg\rangle+
\langle f,B_k(g)\rangle\tag{1.3.1}
\]

\noindent
Keeping $f$ in $C^\infty(\square)$ we notice that
$\langle f,B_k(g)\rangle$ is defined for every
$g\in\mathcal H$.
From this the reader can check that (1.3.1) remains valid when
$g$ belongs to $\mathcal D(\mathcal T_k)$
which means that
\[
\langle Pf,g\rangle=
-\langle f,\mathcal T_kg\rangle+
\langle f,B_k(g)\rangle\quad\colon f\in C^\infty(\square)\tag{1.3.2}
\]

\medskip

\noindent
Moreover, when both $f$ and $g$ belong to $C^\infty(\square)$
we can
reverse their positions in (*) which gives
\[
\langle Pg,f\rangle=
-\langle g,Pf\rangle+
\langle g,B_k(f)\rangle\tag{1.3.3}
\]
Since $a$ and $b$ are real-valued it is clear that
\[
\langle Pg,f\rangle=-\langle f,Pg\rangle\tag{1.3.4}
\]
It follows that
\[
\langle f,B_k(g)=\langle g,B_k(f)\quad\colon f,g\in C^\infty(\square)\tag{1.3.5}
\]
Since this hold for all pairs of $C^\infty$-functions and
$B_k$ is a bounded linear operator on
$\mathcal H$ the density of $C^\infty(\square)$ entails that
$B_k$ is a bounded self-adjoint operator
on $\mathcal H$.



\medskip

\noindent
{\bf{1.4  The equality 
$\mathcal D(T_k^*)=
\mathcal D_k$.}}
The density of $C^\infty(\square)$ in $\mathcal H$
entails that  a function $g\in \mathcal H$ belongs to $\mathcal D(T_k^*)$  
if and only if there exists a constant $C$ such that
\[
|\langle Pf,g\rangle|\leq C\cdot ||f||\quad\colon f\in C^\infty(\square)\tag{1.4.1}
\]
Since $B_k$ is a bounded operator,
(1.3.2) gives the inclusion
\[
\mathcal D_k\subset
\mathcal D(T_k^*)\tag{1.3.3}
\]
To prove the opposite inclusion we use that
the $\Gamma$-operator is elliptic.
If $g\in \mathcal D(T_k^*)$
we have from (i) in � 1.1:
\[
\langle Pf,g\rangle=\langle f,T_k^*g\rangle=
\int\, \Gamma(f)\cdot \overline{T_k^*(g)}\, dxds
\quad\colon f\in C^\infty(\square)
\]
Similarly
\[
\langle f,B_k(g)\rangle=
\int\, \Gamma(f)\cdot \overline{B_k(g)}\, dxds
\]
Treating $\mathcal T_k(g)$ as a distribution the equation
(1.3.2)  entails that
the elliptic operator $\Gamma$ annihilates
$T^*_k(g)-\mathcal T_k(g)+B_k(g)$.
Since both
$T^*_k(g)$ and $B_k(g)$ belong to $\mathcal H$
this implies by the general result in � xx that
$\mathcal T_k(g)$  belongs to $\mathcal H$
which proves the 
requested equality (1.4) and at the same time the operator equation
\[
T_k^*=-\mathcal T_k(g)+B_k\tag{1.4.2}
\] 

\bigskip


\centerline {\bf{3.4  An inequality.}}
\medskip

\noindent
Let $f\in C^\infty(\square)$
and $\lambda$ is a positive real number.
Then
\[
||\mathcal T_k(f)-\frac{1}{2}B_k(f)-\lambda\cdot f||^2=
\]
\[
||\mathcal T_k(f)-\frac{1}{2}B_k(f)||^2+\lambda^2\cdot ||f||^2-
\lambda\bigl(\langle \mathcal T_k(f)-\frac{1}{2}B_k(f),f\rangle+
\langle f, \mathcal T_k(f)-\frac{1}{2}B_k(f)\rangle\bigr)
\]
The last term is $\lambda$ times
\[
\langle \mathcal T_k(f),f\rangle+
\langle f,\mathcal T_k(f)\rangle-\langle f,B_kf\rangle\tag{i}
\]
where we used that $B_k$ is symmetric.
Now $T_k=\mathcal T_k$ holds
on $C^\infty(\square)$ and the definition of adjoint
operators give
\[
\langle \mathcal T_k(f),f\rangle=
\langle f,T_k^*\rangle\tag{ii}
\]
Then (1.4.2 ) implies that (i) is zero and hence we have
proved 
\[
||T_k(f)-\frac{1}{2}B_k(f)-\lambda\cdot f||^2=
\lambda^2\cdot ||f||^2+||T_k(f)-\frac{1}{2}B_k(f)||^2\geq \lambda^2\cdot ||f||^2\tag{iii}
\]
From (iii)  and the triangle inequality for norms we obtain
\[
||T_k(f)-\lambda\cdot f||\geq \lambda\cdot ||f||-
\frac{1}{2}||B_k(f)||\tag{iv}
\]
Now $B_k$ has a finite operator norm and if
$\lambda\geq ||B_k||$ we see that 
\[
||T_k(f)-\lambda\cdot f||\geq
\frac{\lambda}{2}\cdot ||f||\tag{v}
\]


\noindent
Finally, since $C^\infty(\square)$ is dense in $\mathcal D(T_k)$
it is clear that (v) gives
\[
||T_k(f)-\lambda\cdot f||\geq
\frac{\lambda}{2}\cdot ||f||\quad\colon f\in\mathcal D(T_k)\tag{3.4.1}
\]
\medskip

\centerline{\emph{� 3.5. Proof of Theorem 3.2}}
\bigskip

\noindent
Suppose we have found some
$\lambda^*\geq \frac{1}{2}\cdot ||B||$
such that
$T_k-\lambda$  has a dense range in $\mathcal H$
for every $\lambda\geq \lambda^*$.
If this is so we fix $\lambda\geq \lambda^*$
 and take some
$g\in \mathcal H$. The hypothesis gives a
sequence $\{f_n\in \mathcal D(T_k)$ such that
\[
\lim_{n\to\infty}\,||T(f_n)-\lambda\cdot f_n-g||=0
\]
In particular $\{||T_k(f_n)-\lambda\cdot f_n\}$ is a Cauchy sequence 
in $\mathcal H$
and (1.5.x )  implies that $\{f_n\}$ is a Cauchy sequence 
in the Hilbert space $\mathcal H$
and hence converges to
a limit $f_*$. Since the operator $T_k$ is closed we conclude that
$f_*\in\mathcal D(T)$ and we get  the equality
\[
T(_kf_*)-\lambda\cdot f_*=g
\]
Finally, since the graph of $T$ is contained in $T_1$
we have the requested equation 
\[
P(f_*)-\lambda\dot f_*=g
\]
Thus finishes the proof of Theorem 3.2  provided we
have established the existence of $\lambda_*$ above.
\medskip

\noindent{\bf{3.5.1 Density of the range.}}
By the construction of adjoint operators the range of $T_k-\lambda\cdot E$
fails to be dense
if and ony if
$T_k^*-\lambda$ has a non-zero kernel.
So assume that
\[ 
T_k^*(f)-\lambda\cdot f=0\tag{i}
\]
for some $f\in\mathcal D(T_k^*)$ which is not identically zero.
Notice that $T_k$ sends real-valued functions into real-valued
functions. So above we can assume that
$f$ is real-valued and 
also  assume that
$f$ is normalised so that
\[
\int_\square f^2(x,s)\, dxds=1
\]
By  (**) the equation (xx) gives
\[
\mathcal T_k(f)+\lambda\cdot f-B(f)=0\tag{ii}
\]
Let us then consider
the function
\[ 
V(s)= \int_0^\pi\, f^2(x,s)\, dx
\]
Recall from � xx that the $\mathcal H$-function $f$ is of class
$C^1$.
Now
\[ 
\frac{1}{2}\cdot V'(s)=  \int_0^\pi\, f\cdot \frac{\partial f}{\partial s}\, dx\tag{iii}
\]
By (ii) we have
\[
 \frac{\partial f}{\partial s}-
a(x)\frac{\partial f}{\partial x}-b\cdot f=B(f)-\lambda\cdot f
\]
Hence the right hand side in (iii) becomes
\[
-\lambda\cdot V(s)+ \int_0^\pi\, f(x,s)\cdot B(f)(x,s)\,dx+
+ \int_0^\pi\,a(x,s)\cdot f(x,s)\cdot \frac{\partial f}{\partial x}(x,s)\,dx
\]
By partial integration the last term is equal to
\[
-\frac{1}{2}\int_0^\pi\, \partial_x(a)(x,s)\cdot f^2(x,s)\,dx
\]
Set
\[
M=\frac{1}{2}\cdot \max_{(x,s )\in\square}\,|\partial_x(a)(x,s)|
\]
Then we get the inequality
\[
\frac{1}{2}\cdot V'(s)\leq
(M-\lambda)\cdot V(s)+\int_0^\pi\, f(x,s)\cdot B(f)(x,s)\,dx
\]
Set
\[
 \Phi(s)=\int_0^\pi\, |f(x,s)|\cdot |B(f)(x,s)|\,dx
\]
Since the $L^2$-norm of $f$ is one
the Cauchy-Schwarz inequality 
gives
\[
\int_{-\pi}^\pi \Phi(s)\,ds \leq
\sqrt{\int_\square\,  |B(f)(x,s)|^2\, dx ds}\leq ||B(f)||
\]
where the last equality follows since
the  squared integral of $B(f)$ is majorized by its squared
norm in $\mathcal H$.
When $\lambda>M$ it follows from  (xx) that
\[
(\lambda-M)\cdot V(s)+
\frac{1}{2}\cdot V'(s)\leq\Phi(s)
\]
Next, since $f$ is double periodic we have  $V(-\pi)= V(\pi)$
so after an integration (xx) gives
\[
(\lambda-M)\cdot \int_\pi^\pi V(s)\, ds=\int_{-\pi}^\pi \Phi(s)\,ds \leq ||B(f)||
\]
By (xx) we have
$\int_\pi^\pi V(s)\, ds=1$ which gives  a contradiction if
$\lambda>M+||B(f)||$.
\medskip


\noindent
{\bf{Remark.}}
Set
\[ 
\tau=\min_f\,||B(f)||
\] 
with the minimum taken over
funtions $f\in\mathcal D(T_0^*)$ whose $L^2-$integral is normalised by
(xx). The proof has shown that
the kernel of $T_0^*-\lambda$ is zero for all $\lambda>M+\tau$.

\newpage


\centerline{\bf{A special solution.}}
\bigskip


\noindent
Let $f(x)$ be a periodic $C^\infty$-function 
on $[0,\pi]$.
Put
\[ 
Q= a(x,s)\cdot \frac{\partial}{\partial x}+ b(x,s)
\]
Let  $\eta(s)$ be a $C^\infty$-function of $s$ and 
$m$ some  positive integer
If $\lambda>0 $ is a real number.
we set
\[ 
g_\lambda(x,s)=\eta(s)\cdot f+\eta(s)\cdot\sum_{j=1}^{j=m}\,
\frac{(s-\pi)^j}{j!}\cdot (Q-\lambda)^j(f)\quad\colon 0\leq s\leq \pi\tag{i}
\]
We choose $\eta$ to be a real-valued $C^\infty$-function
such that
$\eta(s)=0$ when $s\leq 1/4$ and -1 if $s\geq 1/2$.
Hence $g_\lambda(x,s)=0$ in (i) when
$0\leq  s\leq 1/4$
and we extend the function to $[-\pi\leq s\leq \pi$ where
$g_\lambda(x,-s)= g_\lambda(x,s)$ if $0\leq s\leq \pi$.
So now $g_\lambda$ is $\pi$-periodic with respect to $s$
and  vanishes when
$|s|\leq 1/4$.
\medskip

\noindent
{\bf{Exercise.}}
If $1/2\leq s\leq \pi$ we have $\eta(s)=1$. Use (i) to show that 
\[
(P+\lambda)(g_\lambda)=
\frac{\partial g_\lambda}{\partial s}
-(Q-\lambda)(g_\lambda)=\frac{(s-\pi)^m}{m!}\cdot (Q-\lambda)^{m+1}(f)
\]
hold when
$1/2\leq s\leq \pi$.
At the same time
$g_\lambda(s)=0$ when $0\leq s\leq 1/4$.
So $(P+\lambda)(g)$
is a function whose derivatives
with respect to $s$ vasnish up to order
$m$ at $s=0$ and $s=\pi$ and is therefore
doubly periodic of class $C^m$ in
$\square$.
Now Theorem 2.2  applies. For a given $k\geq 2$
we choose a sufficently large $m$
and find $h(x,s)$ so that  
\[ 
P(h)+\lambda\cdot h=(P+\lambda)( g_\lambda)(x,s)
\]
where $h$ is $s$-periodic, i.e.
\[ 
h(x,0)= h(x,\pi)
\]
Notice also that
$g_\lambda(x,0)=0$ while $g_\lambda(x,\pi)= f(x)$.
Set
\[ 
g_*(x)=h-g_\lambda
\]
Then  $P(g_*)+\lambda\cdot g_*=0$
and 
\[ 
g_*(x,0)-g_*(x,\pi)=f(x)
\]
\medskip

\noindent
Above we started with the $C^\infty$-function.
Given $k\geq 2$
we can take
$m$ sufficiently large during the constructions above
so that $g_*$ belongs to
$\mathcal H^{(k)}(\square)$.






\end{document}
































 

 























