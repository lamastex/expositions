

\documentclass{amsart}

\usepackage[applemac]{inputenc}


\addtolength{\hoffset}{-12mm}
\addtolength{\textwidth}{22mm}
\addtolength{\voffset}{-10mm}
\addtolength{\textheight}{20mm}

\def\uuu{_}

\def\vvv{-}


 \begin{document}


\newcounter{chapter}











\centerline{\bf{B. The Kovalevsky gyroscope}}

\bigskip

\noindent
The article \emph{Sur le probleme de la rotatiation
d'un corps solide autotur d'un point fixe} 
was printed in Acta Mathematica
1889.
A work for which Kovalevsky  received the \emph{Bordin Prize} on Christmas Eve 1888
in Paris.
Here follows  the first section of the article 
in English translation.
\bigskip

\noindent 
\emph{The problem of the rotation of a rigid body around a  fixed point
with gravity force $g$ reduces, as is wellknown, to solve the following system of
differential equations}:

\begin{align*}
&A\frac{dp}{dt}=(B-C)qr+Mg(y_0\gamma''-z_0\gamma')
\quad\colon\quad
\frac{d\gamma}{dt}= r\gamma'-q\gamma''
\\
&
B\frac{dq}{dt}=(C-A)rp+Mg(z_0\gamma-x_0\gamma'')\quad\colon\quad
\frac{d\gamma'}{dt}= p\gamma''-r\gamma\\
&
C\frac{dr}{dt}=(A-B)rpq+Mg(x_0\gamma'-y_0\gamma)
\quad\colon\quad
\frac{d\gamma''}{dt}= q\gamma-p\gamma'
\\
\end{align*}
\medskip

\noindent
\emph{The constants $A,B,C,M,g,x_0,y_0,z_0$ denote the following: 
$A,B,C$ are the
positive eigenvalues of the intertia operator of the body, i.e. they correspond to lengths of the principal axis when the body is regarded as an ellipsoid.
$M$ is the mass of the body, $g$ the force of gravity and 
$x_0,y_0,z_0$
the coordinates of the center of mass in the coordinate system 
whose origin is the fixed point with 
principal axis  determined as above.
Fuinally, the six quantities $p,q,r,\gamma,\gamma',\gamma''$ are time-dependent
functions 
where $\omega=(p,q,r)$ is Euler's angular velocity
and $(\gamma,\gamma',\gamma'')$
are the coordinates of the
vector $e_z$ taken in the body space.
which extend to complex analytic functions whose
singularities are at most poles.}
\medskip

\noindent
\emph{Until now one has only found two cases where the equations of motion
can be integrated and thus solved by quadrature}:
\bigskip

\noindent
\emph{The case of Poisson (or Euler):}  $\quad x_0=y_0=y_0$.
\bigskip

\noindent
\emph{The case of Lagrange:}  $\, A=B\colon\quad x_0=y_0=0$.
\bigskip

\noindent 
\emph{In these two cases the solution is found after integration by 
theta functions where the
six quantities $p,q,r,\gamma,\gamma',\gamma''$ are time-dependent
functions 
which extend to analytic   functions whose
singularities are at most poles.}

\bigskip


\noindent
Kovalevsky studied the power series solutions to the 
differential system above 
and found constraints on their complex Laurent series expansions which
led  to the conclusion that the constants $A,B,C$ and the position of
the center of mass, must be special  in order find a fourth integral as 
as in the two special  cases above.
\medskip


\noindent
After five pages of   calculations
an  example occurs in  section 2  from [ibid]
which gives  a rigid body whose equations of motion can be solved by quadrature for every
initial position.
More precisely, in the  Kovalevsky gyroscope one has:
\medskip
\[ 
A=B=2\quad C=1\quad\, x_0=1\quad y_0=z_0=0\tag{*}
\]

\noindent 
The equality $A=B$ means that the body has a plane of symmetry
and the center of mass is placed in this plane.
But this center of mass  does not belong to a principal axis so the gyroscope fails to satisfy
the assumption which had been treated   earlier  by Lagrange.
When (*) holds
Kovalevsky found a \emph{fourth integral} of the 
differential system  
expressed by a polynomial of the 4:th degree of 
the variables $p,q,r,\gamma,\gamma',\gamma''$.
Once this is achieved there remains  a considerable
work to express  the solution.
In [ibid] more than ten  pages are devoted to the study of 
\emph{ultra-elliptic functions} which appear in the 
time-dependents solutions using the fourth integral.


\medskip

\noindent
{\bf{The use of complex analysis.}}
To find the fourth integral of the system 
when (*) holds Kovalevsky employed analytic function theory.
Let us describe how one derives the fourth integral.
We prefer to use other notations for $p,q,r$, i.e. Euler's angular velocities are
denoted by
\[ 
\omega\uuu 1=p\quad\colon\quad
\omega\uuu 2=q\quad\colon\quad
\omega\uuu 3=r
\]
Similarly, lower indices are used for the $\gamma$\vvv functions:
\[ 
\gamma\uuu 1=\gamma\quad\colon\quad 
\gamma\uuu 2=\gamma'\quad\colon \quad
\gamma\uuu 3=\gamma''
\]
When $A,B,C$ satisfy (*) and $gM=1$
we obtain a system of six differential equations:





\begin{align*}
&2\dot \omega_1=\omega_2\omega_3
\\
&2\dot \omega_2=\vvv \omega_1\omega_3-\gamma_3
\\
&\dot \omega_3=\gamma_2
\end{align*}

\medskip

\begin{align*}
&\dot \gamma_1=\gamma_2\omega_3-\gamma_3\omega_2
\\
&\dot \gamma_2=-\gamma_1\omega_3+\gamma_3\omega_1
\\
&\dot \gamma_3=-\gamma_1\omega_2-\gamma_2\omega_1
\end{align*}
\medskip
\noindent {\bf Invariant integrals.} 
Since $\gamma$ is a unit vector one has
\[
\gamma_1^2+\gamma_2^2+\gamma_3^2=1\tag{1}
\]
Next, since the sum of kinetic and potential energy is constant one has:
\[
\omega_1^2+\omega_2^2+\frac{\omega_3^3}{2}+
\gamma_1=E\tag{2}
\]
Finally, the time derivative of the angular momentum
$\mathfrak M$ is $\perp$ to $e_z$ which
gives the third algebraic
equation:
\[
2\omega_1\gamma_1+2\omega_2\gamma_2+\omega_3\gamma_3=F\tag{3}
\]
where $F$ is a constant.

\bigskip

\noindent
{\bf Kovalevsky's fourth integral.} There exists one more integral to the 
ODE-system. To obtain it we introduce the imaginary unit. The first two
equations for the $\omega$-system give:
\[
2(\dot \omega_1+i\dot\omega_2)=
\vvv i\omega_3(\omega_1+i\omega_2)-i\gamma_3\tag{i}
\]
Next,  the first two equations for the $\gamma$-system
give:
\[
\dot\gamma_1+i\dot\gamma_2=
i\gamma_3(\omega_1+i\omega_2)-i\omega_3(\gamma_1+i\gamma_2)\tag{ii}
\]
Put 
\[
\phi=(\omega_1+i\omega_2)^2+(\gamma_1+i\gamma_2)
\]


\noindent
It follows from (i-ii) that
\[ 
\dot\phi=\vvv (\omega_1+i\omega_2)\cdot [i\omega_3(\omega_1+i\omega_2)+i\gamma_3]
+i\gamma_3(\omega_1+i\omega_2)-i\omega_3(\gamma_1+i\gamma_2)
\]
A  computation shows that the right hand side becomes
$\vvv i\omega\uuu 3\cdot \phi$ and hence $\phi$ satisfies the differential equation
\[
\dot\phi=\vvv i\omega\uuu 3\cdot \phi\tag{iii}
\]
\medskip

\noindent
Since the function  $\omega_3(t) $ is real-valued
this differential equation implies that
the $t$-derivative of $\text{log}(\phi(t))$ is purely imaginary. The equality
\[
\mathfrak{Re}\,\log(\phi(t))= \log\,|\phi(t)|
\]
implies that the  \emph{absolute value} of $\phi$ is constant.  Hence there exists
a real constant $k$ such that
\[
k^2=|(\omega_1+i\omega_2)^2+\gamma_1+i\gamma_2|^2=
(\omega_1^2-\omega_2^2+\gamma_1)^2+(2\omega\uuu 1\omega\uuu 2+\gamma\uuu 2)^2
\tag{4}
\]
\medskip

\noindent
{\bf Remark} Above
(4)
gives the
fourth integral which is used to solve
the equations of motion  by quadrature.
In contrast to the  case treated  by Euler where
the solution by quadrature
is achieved by an elliptic integral of the first kind, the
solution 
of  the Kovalesvky gyroscope involves 
\emph{hyper-elliptic integrals}.
For details we refer to Kovalevsky's original article which in addition to
the example of a special gyroscope contains 
very interesting
calculations.
Personally I think that every student   should try to 
study
original work. 
The material about hyperelliptic functions and their integrals
 in Kovalevsky's article offers an 
account
which I personally find  more transparent than  more recent  publications on these
topics.
The invariant integrals (1\vvv 3)  will
be established in the section devoted to
rigid bodies.









\medskip

\noindent
Two years after Kovalevsky's disease,
Liapounov proved that the example (*)
is 
\emph{unique}, i.e. it gives the sole example except the classical, where the equations
of motion can be solved by quadrature.
In an article published in Acta Mathematica 1892, K�nigsberg
expressed the motion of Kovalevsky's gyroscope in Eulerian angles where
the  solution   consists of 
generalised theta-functions. K�nigsberg's formula
together with a computer provide a  picture of
the motion of the Kovalevsky gyroscope under arbitrary  initial conditions.
To avoid  confusions we remark that there exist  examples where
the differential system for the motion can be integrated
\emph{provided} that special initial conditions are given.
But these specific  examples  were not  at stake in the 
far more difficult problem treated by  Kovalesvky.






\medskip

\noindent
{\bf Another remark.}
Kovalevsky gave courses 
on many different subjects during her years in Stockholm.
The interplay  between
equations derived from the "real world" of mechanics and
complex integrals which  lead to elliptic functions
was  put forward in these lectures.
Let us illustrate this by an
example
which normally belongs to a
course devoted to analytic function theory.
Consider  Weierstrass' 
$\mathfrak{p}$-function which
is doubly meromorphic with
respect to some lattice in
${\bf{C}}$ generated by two
${\bf{R}}$-linearly independent complex vectors
$\omega_1$ and $\omega_2$. 
Set
\[ 
e_1=\frac{\omega_1}{2}\,,\quad
e_2=\frac{\omega_1}{2}\,,\quad
e_3=\frac{\omega_1+\omega_2}{2}\,.
\]
Now there exists Jacobi's elliptic function
\[
\mathfrak{sn}(z)=
\frac{\sqrt{e_1-e_2}}{\sqrt{\mathfrak{p}(z)-e_3}}
\]
Jacobi's inversion formula asserts that
\[
z=\frac{1}{\sqrt{e_2-e_3}}\cdot 
\int_0^{\mathfrak{sn}(z)}\, \frac{dz}{\sqrt{(1-z^2)(1-\chi^2z^2)}}
\]
This inversion formula is a
consequence of  conservation laws
applied to a two-particle system describing
the motion of two mass-points which
both perform a periodic motion in a simple pendelum.
The addition formula for the
$\mathfrak{p}$-function, and more generally for any
elliptic  function can also be derived via laws of classical mechanics.
So students 
interested in analytic function theory
or algebraic geometry
devoted to 
projective curves
in characteristic zero, should keep in mind that   results
which express  properties of rational functions on such
curves
have a  natural interpretation  via classical mechanics and may  therefore
be considered as a  Law of Nature rather than
a mathematical discovery. 



\newpage




\centerline{\bf\large 2. Rigid Bodies}

\bigskip

\noindent
In a rigid body $K$ distances between points remain constant under motion, even
when a sudden force acts on $K$ by an  impact.
This implies that 
when $K$ is placed in $\bf R^3$,   the position of an arbitrary point
$q\in K$ can be determined
via six cordinates.
To see this we choose a point $p\in K$
and three other points $q\uuu 1,q\uuu 2,q\uuu 3$ in $K$
such that  the euclidian distances satisfy
\[
|q\uuu \nu\vvv p|=1\quad\text{and}\quad
|q\uuu \nu\vvv q\uuu j|=\sqrt{2}\quad\colon\quad
\nu\neq j
\tag{1}
\] 
If necessary we adjoin  such points withous mass  and rigid bars without
mass connecting them to $p$.
Pythagoras' theorem implies this  that the three
unit vectors
$\{q\uuu \nu\vvv  p\}$ are pairwise orthogonal
and we get the body space where $p$ is the origin and
these three vectors is an orthonormal basis.
Denote this orthonormal space by
$\mathcal V_K$.
Next, when $K$ is placed in the euclidian space
${\bf{R}}^3$ the postions of $p$ and
$\{q\uuu \nu\}$ give four points denoted by
$p^*,q\uuu 1^*,
q\uuu 2^*,q\uuu 3^*$. Since distances are preserved 
(1) hold for the (*)-marked points, and Pythagoras's theorem entails that
the vectors $\{\xi^*_\nu=q^*_\nu-p^*\}$
are pairwise orthogonal unit vectors.
Next, let $q\in K$ be an arbitrary
point. In the body space we get coordinates
$a\uuu 1,a\uuu 2,a\uuu 3$
such that
\[
q=\vvv p=a\uuu 1(q\uuu 1\vvv p)+
a\uuu 2(q\uuu 2\vvv  p)+a\uuu 3(q\uuu 3\vvv p)\tag{1}
\]
Put $\xi\uuu k= q\uuu k^*\vvv p^*$ for each $k$.
Passing to the (*) marked points we write
\[
q^*\vvv p^*=b\uuu 1\cdot \xi\uuu 1^*+
b\uuu 2\cdot \xi\uuu 2^*+
b\uuu 3\cdot \xi\uuu 3^*\tag{2}
\]
Now the reader can verify that
the $b$-coordinates are equal to the $a$-coordinates, i.e.
\[ 
b_k= a_k\tag{3}
\] 
hold for each $k$.
\medskip

\noindent
{\bf{Conclusion.}}
For each $q\in K$ the position of   $q^*$
in ${\bf{R}}^3$ is  determined by
the positions of 
$p^*$ and the three orthogonal unit vectors $\{q_\nu^*-p^*\}$.
Together with (3) it follows that there exists an orthogonal $3\times 3$\vvv matrix $S$
such that
\[ 
q^*\vvv p^*= S(q\vvv p)\tag{*}
\] 
hold for every $q\in K$ where
$S$ is regarded as a linear map from the orthogonal  body space into the euclidian 
space ${\bf{R}}^3$.
Recall that the determinant of an orthogonal matrix is +1 or -1. The set of
orthogonal matrices with determinant +1 is a group denoted by
$\text{S0}(3)$.
Under a continuous motion of $K$ in ${\bf{R}}^3$
the sign of the $S$matrix is unchanged and one
can always start with the orthonormal basis in the body space so that
the $S$-matrix in (*) has determinant +1 which means that it
preserves orientation.
At the same time
$p$ can be placed at an arbitrary point $p^*$ in
${\bf{R}}^3$. It follows that the \emph{configuration space}  for $K$
is given  by the product space ${\bf{R}}^3\times \text{SO}(3)$.
We conclude  that a rigid body has \emph{six degrees of freedom}  when it moves
in ${\bf{R}}^3$.
Let us now consider a time\vvv dependent  motion of $K$.
So when $t$ is the time variable the vector valued function $p^*(t)$
describes the motion of $p$ and we also have
the $\text{SO}(3)$\vvv valued function
$t\mapsto S\uuu t$ where

\[
q^*(t)=p^*(t)+S_t(q\vvv p)
\quad\,1\leq\nu\leq 3\quad\colon\quad q\in K\tag{**}
\]
\medskip


\medskip

\noindent
{\bf 2.1 Remark.} 
Above the construction started from a chosen point $p\in K$.
Consider another point $\rho\in K$ which gives
\[
q^*(t)=\rho^*(t)+
(p^*(t)\vvv \rho^*(t))+S\uuu t(q\vvv p)=
\rho^*(t)\vvv S\uuu t(\rho\vvv p)+S\uuu t(q\vvv p)=\rho^*(t)+S\uuu t(q\vvv \rho)
\]

\medskip

\noindent
This shows that the rotation matrix $S\uuu t$
is the same when the body space is centered at $\rho$.


\newpage


\centerline {\bf 2.2 Euler's Angular velocity}
\medskip

\noindent
Consider a motion
of the rigid body whose time dependent
functions are of class $C^2$ at least.
We can take time derivatives of the nine elements 
in the matrix $S_t$. 
This yields for each $t$ 
the $3\times 3$-matrix $\dot S_t$
which again is a linear map from $\mathcal V_K$
into $\bf R^3$.
Let us also regard the \emph{inverse} linear map $S_t^{-1}$. Since $S_t$
is orthogonal this inverse
is the adjoint matrix $S_t^*$.
Now $S_t^*\circ \dot S_t$ is a linear map from $\mathcal V_K$ into itself.
\medskip

\noindent 
{\bf 2.2.1 Proposition.} \emph{The matrices $S_t^*\circ\dot S_t$ are anti-symmetric for all�
$t$}.

\bigskip

\noindent 
{\bf Proof.} 
Since $S_t$ is orthogonal 
$t\mapsto \langle S_t(q),S_t(p)\rangle$ is a constant function of $t$
for all pairs $p,q$ in
$\mathcal V_K$.
Hence the time derivative is zero which gives
\[
0=
\langle \dot S_t(q),S_t(p)\rangle+\langle S_t(q),\dot S_t(p)\rangle
=
\langle S_t^*\circ \dot S_t(q),p\rangle+
\langle q,S_t^*\circ\dot S_t(p)\rangle
\] 
and 
the requested   anti-symmetry follows.
\bigskip

\noindent
Recall from linear algebra 
that 
every anti-symmetric matrix is 
expressed by a vector product.
Thus, for each $t$ there exists a unique vector $\omega(t)\in\mathcal V_K$
such that
\[
S_t^*\circ \dot S_t(q)=\omega(t)\times q\,\quad\colon \,q\in\mathcal V_K
\]


\noindent
From (**) above Remark 2.1 it follows that
the time derivatives of points $q\in K$ satisfy:
\[
\dot q^*(t)=\dot p^*(t)+\dot S_t(q)\tag{2.2.2}
\]
Since $S_t\circ S_t^*$ is the identity we obtain
\[
\dot q^*(t)=\dot p^*(t)+S_t\circ S_t^*\circ \dot S_t(q)=
\dot p^*(t)+S_t(\omega(t)\times q)\tag{2.2.3}
\]

\medskip

\noindent
One refers to $\omega(t)$ as \emph{Euler's angular velocity} which 
by the constructions above is a vector-valued on  the body space.

\bigskip
 


\noindent
\centerline {\bf 2.3 Kinetic energy and momentum}
\medskip

\noindent
Let $K$ be a rigid body
which consists of a finite set of mass points
$p_1,\ldots,p_N$. One can imagine
that they  are joined by rigid bars with zero mass. Put
\[ 
\mathfrak {o}=\frac{1}{M}\sum_{\nu=1}^{\nu=N}\,
m_\nu\cdot p_\nu\quad\text{where}\quad \,M=\sum\,m_\nu
\,\,\text{is the total mass} 
\]
\bigskip

\noindent
One refers to $\mathfrak {o}$ as the
the center of mass, or simply the mass-point of $K$.
When $K$ moves in $\bf R^3$ the velocities 
varies between  individual 
points because of rotation. Therefore it is no sufficient to regard a single point,
such as the center of mass to obtain the kinetic energy.
Let us choose  $\mathfrak o$ as the origin in the   body space
$\mathcal V_K$. Under a motion we get  time dependent  rotation matrices $S_t$ 
and for every mass point $p_\nu\in K$ one has

\[
\dot p\uuu\nu^*(t)=\dot  o^*(t)+S_t(\omega_t\times p_\nu)\tag{2.3.1}
\]

\noindent
It follows that the kinetic energy becomes
\[
T=\frac{1}{2}\cdot \sum\, m_\nu\cdot \langle \dot p_\nu^*,\dot p_\nu^*\rangle=
\]
\[
\frac{1}{2}\cdot \sum\, m_\nu\langle \dot p_\nu^*, \mathfrak o^*\rangle+
\frac{1}{2}\cdot \sum\, m_\nu\cdot \langle \dot p_\nu^*,S_t(\omega_t\times p_\nu)\rangle\tag{2.3.2}
\]
\medskip

\noindent
The first sum above is equal to 
$\frac{1}{2}\cdot \, M\cdot \langle \dot  o^*,\dot o^*\rangle$.
Expanding the second term the reader can verify that it becomes:
\[
\frac{1}{2}\cdot \sum
m_\nu\cdot \langle S_t(\omega_t\times p_\nu),
S_t(\omega_t\times p_\nu)\rangle\tag{2.3.3}
\]

\noindent
Since the orthogonal matrix  $S_t$ preserves   inner products
it follows that (2.3.3) is equal to
\[
\frac{1}{2}\cdot \sum
m_\nu\cdot \langle \omega_t\times p_\nu,
\omega_t\times p_\nu\rangle\tag{2.3.4}
\]
This suggests that we introduce a linear operator on $\mathcal V_K$.
\bigskip

\noindent 
{\bf 2.3.5 Definition} \emph{The central operator of inertia
is the linear operator defined on $\mathcal V_K$ by
\[
q\mapsto \sum m_\nu\cdot p_\nu\times
(q\times p_\nu)
\]
It is denoted by $\mathcal M_{\mathfrak {o}}$.}
\bigskip

\noindent
Recall that the vector product 
satisfies the following for each pair of 
vectors
$u,v$ in  the orthonormal space $\mathcal V_K$:
\[
\langle u,(u\times v)\times u\rangle=
||u\times v||^2
\]
\medskip

\noindent
Applied to (2.3.4) above we 
get:
\bigskip

\noindent 
{\bf 2.3.6  Theorem} \emph{The kinetic energy is expressed by the equation}
\[ 
T=\frac{1}{2}M\cdot|\dot o^*|^2+
\frac{1}{2}\langle \omega_t,\mathcal M_\frak o(\omega_t)\rangle
\]
\emph{One refers to
$\frac{1}{2}\langle \omega_t,\mathcal M_\frak o(\omega_t)\rangle$
as the rotational kinetic energy  and it is denoted by $T_{\text{rot}}$.}
\bigskip

\noindent
{\bf {Change of center.}} It is sometimes 
easier to pursue the motion 
with respect to  another point in $K$ than the mass point. For this purpose we give
\medskip

\noindent 
{\bf 2.3.7 Definition} \emph{
The inertia operator in a body space centered at a
point  $p\in K$
is defined by}
\[
q\mapsto \sum m_\nu\cdot (p_\nu-p)\times
[(q-p)\times (p_\nu-p)]
\]
\emph{It is denoted by
$\mathcal M_p$.}

\bigskip

\noindent
{\bf{Exercise.}}
Show the equality below for each pair of points $p,q$ in $K$:
\[
\mathcal M_p(q)=\mathcal M_\mathfrak o(q-p)
+M\cdot (p\times(q\times p))\tag{2.3.8}
\]
and conclude that ine has
the equation

\[
T=\frac{1}{2}M\cdot||\dot p^*||^2+
\frac{1}{2}\langle \omega_t,\mathcal M_p(\omega_t)\rangle\tag{2.3.9}
\]


\newpage


\centerline {\bf 2.4  Angular momentum}
\bigskip

\noindent
When $K$ is in motion  we define for each time value $t$
the vector

\[
\frak M_\frak o(t)=\sum\, m_\nu\cdot (p^*_\nu(t)-\frak o^*(t))\times
\dot p_\nu^*\tag{2.4.1}
\]
\medskip

\noindent
{\bf{Exercise.}}
Show the equality
\[
\frak M_\frak o(t)=S_t(\mathcal M_o(\omega_t))\tag{2.4.2}
\]
\medskip


\noindent
Show also that the time derivative of the vector valued 
function $\frak M_\frak o$ becomes:
\[
\frac{d}{dt}(\frak M_\frak o)
=S_t\,[\,
\mathcal M_o(\dot \omega_t))+
\omega_t \times\mathcal M_o(\omega_t)\,]\tag{2.4.3}
\]


\noindent
One refers to (**) as
Euler's identity for the time
derivative of angular momentum.

\bigskip


\noindent
{\bf{The time derivative of $\frak{M}_\frak 0$.}}
Since the vector product
is anti-commutative the reader can check that
(2.4.1) gives
the equation
\[
\frac{d}{dt}(\frak M_\frak o)\,=
\sum\, m_\nu\cdot (p_\nu^*(t)-\frak o^*(t))\times
\ddot p_\nu^*\tag{2.4.4}
\]
\bigskip



\centerline
{\bf 2.5  Equations of motion}
\bigskip

\noindent
Now we  study the effect of fores acting on a rigid body during its motion.
At a given time monent we suppose that
force vectors  $\{F_\nu\}$
act on the mass-points  $\{p_\nu\}$.
Then (2.4.4 ) and Newtons formula give:

\[
\frac{d}{dt}(\frak M_\frak o)\,=
\sum\, (p_\nu^*(t)-\frak o^*(t))\times
F_\nu\tag{2.5.1}
\]
\medskip


\noindent {\bf {External and inner forces}}.
During a motion inner forces keep the body rigid. They appear in
pairs $f_{ij}$ and $-f_{ij}$ where $f_{ij}$ acts on 
the mass-point $p_j$
while the opposed force vector $-f_{ij}=f_{ji}$ acts on $p_i$.
Apart from these there exist \emph{external forces} 
denoted by $F_\nu^{\text{ext}}$.
Since the vector product is anti-commutative, the 
reader can check that the total effect
of inner forces disappears in (2.5.1) 
and hence one has:

\[
\frac{d}{dt}(\frak M_\frak o)\,=
\sum\, (p_\nu^*(t)-\frak o^*(t))\times
F_\nu^{\text{ext}}\tag{2.5.2}
\]
\medskip



\noindent
{\bf Example} In $\bf R^3$ we denote the vertical direction by $z$. 
So here gravity yields an external force whose strength is $g$.
At every  mass-point $p_\nu$ we  have
\[
F_\nu^{\text{ext}}=-m_\nu\cdot  g\cdot e_z
\]
So when gravity is the sole external force acting on the rigid body one has:
\[
\frac{d}{dt}(\mathfrak M_\frak o)\,=-g\cdot
\sum\, m_\nu\cdot (p_\nu^*(t)-\mathfrak o^*(t))\times
e_z=0\tag{2.5.3}
\]


\noindent
Notice that the vector in the right hand side
is $\perp$ to $e_z$. It follows that the function
\[
t\mapsto 
\langle \mathfrak M_\mathfrak o(t),e_z\rangle
\]
is constant.
Notice also that (2.4.x) gives
\[
\langle \mathfrak M_\mathfrak o(t),e\uuu z\rangle
=\langle S\uuu t(\mathcal  M_o)(\omega\uuu t),e\uuu z\rangle
=
\langle \mathcal  M_o)(\omega\uuu t),S\uuu t^*(e\uuu z)\rangle
\]



\medskip

\noindent
Hence we have proved the following:
\medskip

\noindent{\bf{2.5.4 Theorem.}}
\emph{When gravity is the sole external force on a body
there exists a constant $C$ such that}
\[
\langle \mathcal  M_\frak o(\omega\uuu t),S\uuu t^*(e\uuu z)\rangle
=C
\]







\bigskip


\centerline{\bf 2.6. Rotation around the mass-point}
\bigskip

\noindent
We shall   study a rigid body $K$ whose  mass-point  
remains fixed during the motion. 
Recall that the  motion
is described by 
matrices  $\{S_t\}$ which map a body space $\mathcal V_K$
centered at $\mathfrak o$ into  $\bf R^3$. 
In $\bf R^3$ the $z$-axis is vertical where  gravity acts in 
the negative $z$-direction and we  assume that this is the sole 
external force acting on $K$.
Since the mass-point is fixed
it means that no external forces appear during the rotation.
Now we shall describe  the equations of motion.
For this purpose we consider the central angular momentum 
which from previous results becomes:
\[
\mathfrak M_\frak o(t)=S_t(\mathcal M(\omega(t))
\]
By 2.5.2  the time-derivative is zero and applied to the right
hand side Leibniz' rule gives

\[
\dot S_t(\mathcal M_\frak o(\omega(t))+
S\uuu t(\mathcal M_\frak o(\dot \omega(t))=0
\]
Applying $S_t^*$ on both sides it follows that

\[
\omega(t)\times\mathcal M_\frak o(\omega(t))+
\mathcal M_\frak o(\dot \omega(t))=0\tag{*}
\]
 
\medskip

\noindent
The system (*) are called the  the \emph{Euler-Lagrange equations} for the body.
To solve (*)
we will choose a suitable \emph{orthonormal basis}
in $\mathcal V_K$   adapted to the linear operator
$\mathcal M$.
\bigskip

\noindent
{\bf 2.6.1  Principal axes.} Recall from � xx  that $\mathcal M_\frak o$ is a
symmetric linear operator on $\mathcal V_K$. The 
spectral theorem for symmetric matrices gives an orthogonal basis 
$e_1,e_2,e_3$ which diagonalizes $\mathcal M_\frak o$ and there exist
constants
$A_1,A_2,A_3$ such that


\[
\mathcal M(e_i)=A_ie_i\quad\,1\leq i\leq 3
\]


\noindent 
Above each $A_i>0$ unless $K$ is a linear body, i.e. where all the mass is concentrated
to a single line.
We 
ignore to discuss this special case and
express the vectors $\omega(t)$  in this basis:

\[
\omega(t)=\omega_1(t)e_1+
\omega_2(t)e_2+
\omega_3(t)e_3
\]
\medskip

\noindent 
The $e-$basis is chosen so that $e_1\times e_2=e_3$ and so on,
i.e. it is positively oriented with respect to the vector product.
Rules for vector products  show that the system (*) becomes:

\medskip

\begin{align*}
&A_1\cdot \dot \omega_1+(A_3-A_2)\omega_2\omega_3=0
\\
&A_2\cdot \dot \omega_2+(A_1-A_3)\omega_1\omega_3=0
\\
&A_3\cdot \dot \omega_3+(A_2-A_1)\omega_1\omega_2=0
\end{align*}
\medskip

\noindent This first order system  
of differential equations
has a unique vector-valued solution $\omega(t)$ when
initial values are   given a time $t=0$.
\medskip

\noindent
{\bf{2.6.2. Exercise}}.
Show  that the differential system above
give positive constants $T$ and $E$ such that

\[
A_1\cdot \dot \omega^2_1+
A_2\cdot \dot \omega_2^2+A_3\cdot \dot \omega_3^2=2T\tag{i}
\]
\[
A^ 2_1\cdot \dot \omega^2_1+
A^2_2\cdot \dot \omega_2^2+A^3_3\cdot \dot \omega_3^2=E\tag{ii}
\]

\medskip

\noindent
{\bf{2.6.3 Solution by quadrature.}}
Consider the case when $A_3<A_2<A_1$.
From (i-ii) we can eliminate $\omega_2^2$ and $\omega_3^2$
and from this the reader can  deduce that
$\omega_1$ satisfies a differential equation of the form

\[
\dot \omega_1=+\,\text{or}\,-\, 
C\cdot 
\sqrt{\alpha^2\vvv\omega\uuu 1^2}\cdot
\sqrt{\beta^2\vvv \omega\uuu 1^2}
\]
where $C,\alpha,\beta$ are constants which in addition to the
body\vvv constants $A\uuu 1,A\uuu 2,A\uuu3$  depend upon $T$ and $E$.
Except for very special case one has $\alpha\neq \beta$
and with $0<\alpha<\beta$ it follows that 
$\omega_1(t)$
is a periodic function of $t$ which oscillates
between $- \alpha$ and $\alpha$.
The time intervals for a full oscillation
is expressed by an elliptic integral.
More precisely, if $T_1$ is the full period
one has the equation

\[
T_1= 4C\cdot \int\uuu 0^\alpha\,
\frac{dw}{\sqrt{\alpha^2\vvv\omega\uuu 1^2}\cdot
\sqrt{\beta^2\vvv \omega\uuu 1^2}}
\]
The  functions $\omega\uuu 3(t)$ and $\omega\uuu 3(t)$
are also periodic with certain periods $T\uuu 2$ and $T\uuu 3$.
Except for special cases the three periods are different
and the whole rotation around the mass-point
 is "chaotic".


\bigskip


\centerline {\bf{2.6.4  Rotation outside the mass-point}}.
\bigskip

\noindent
Consider the case when $K$ rotates around a point $p\neq\mathfrak o$
while gravity is the sole external force.
Herte we let the body space be  centered at $p$.
The linear operator $\mathcal M_p$.
is  symmetric so by the spectral theorem we can choose an orthonormal
basis in
$\mathcal V_K$ where
$\mathcal M_p$ is diagonal. Consider 
the angular momentum

\[ 
\mathfrak M_p(t)=S_t(\mathcal M_p(\omega(t))
\]
\medskip

\noindent
{\bf{2.6.5 Exercise.}}
Show that the time derivative

\[
\frac{d}{dt}(\mathfrak M\uuu p(t)=
M\cdot g\cdot o^*(t)\times e\uuu z\implies \tag{1}
\]
\[
S\uuu t\bigl [
\omega(t)\times\mathcal M(\omega(t))+
\mathcal M(\dot \omega(t))\bigr ]=\vvv M\cdot g\cdot
o^*(t)\times e\uuu z
\]
Applying  $S\uuu t^*$ to both sides one gets

\[
\omega(t)\times\mathcal M_p(\omega(t))+
\mathcal M_p(\dot \omega(t))=-M\cdot g\cdot
\mathfrak o\times S\uuu t^*(e\uuu z)\tag{2}
\]

\medskip

\noindent
In a  basis where $\mathcal M_p(e\uuu 1)=A$,
$\mathcal M_p(e\uuu 2)=B$
and $\mathcal M_p(e\uuu 3)=C$ we
get a system of differential equations which was 
presented in the introduction from Kovalevsky's article.
Notice that the Euler's equations 
no longer are homogenous since
the  function
\[
t\mapsto
S\uuu t^*(e\uuu z)
\] 
with values in the body space 
appears above. 
So here one
encounters a system with six time-dependent functions,
where three of them express
time derivatives of the components of 
the vectors $S_t^*(e_z)$.


\newpage


\centerline{\bf\large 3. Eulerian  angles}
\bigskip

\noindent
Given two angles $0<\theta<\pi$ and $0\leq\phi\leq 2\pi$
we put
\medskip
\[
e_3=\text{cos}\,\theta\cdot e_z+
\text{sin}\,\theta(\text{cos}\,\phi\cdot e_x+
\text{sin}\,\phi\cdot e_y)
\]
To this vector we associate the unit vectors
\medskip
\[
\xi_2=-\text{sin}\,\theta\cdot e_z+\text{cos}\,\theta\,[\text{cos}\,\phi\cdot e_x+
\text{sin}\,\phi\cdot e_y]\quad\colon\quad
\xi_1=\text{sin}\,\phi\cdot e_x-
\text{cos}\,\phi\cdot e_y
\]
\medskip

\noindent
Notice that $\xi_1\times \xi_2=e_3$ and hence 
the  triple $\xi_1,\xi_2,e_3$
is a positively oriented orthonormal frame.
A general ON-frame $(e_1,e_2,e_3)$ arises  when we introduce 
another angular variable $\psi$ and set:
\medskip
\[
e_1=\text{cos}\,\psi\,\xi_1+\text{sin}\,\psi\,\xi_2\quad
\colon\quad e_2=\text{sin}\,\psi\,\xi_1+\text{cos}\,\psi\,\xi_2
\]
\medskip

\noindent
We refer to $\theta,\phi,\psi$ as the Euler angles defining this ON-frame.
Let us now consider a  rigid body
which rotates around the origin and let $e_1,e_2,e_3$ be some 
postively oriented
ON-frame in the body space. We get the time dependent vectors in $\bf R^3$:
\[
e_\nu^*(t)=S_t(e_\nu)
\]
For each $t$ they give an  ON-frame in $\bf R^3$. We assume that 
$e_3^*(t)$ is not parallell to the $z$-axis. Then there exist time dependent functions 
$\theta(t)$ and $\phi(t)$ such that
\[
e^*_3(t)=\text{cos}\,\theta(t)\cdot e_z+
\text{sin}\,\theta(t)(\text{cos}\,\phi(t)\cdot e_x+
\text{sin}\,\phi(t)\cdot e_x)
\]

\noindent
The remaining vectors $e_1^*(t)$ and $e_2^*(t)$
are  determined by the two angular functions
$\theta(t),\phi(t)$ and a $\psi$-function which corresponds to
a rotation of $K$  around $e_3^*$.
The result is that the time dependent rotation matrix $S_t$
is a function of the three angular variables.
Now we can take time derivatives and in this way express 
Euler's angular velocity by the three angle functions and their time derivatives.
A straightforward calculation which is left to the reader gives:


\[
\omega_1=
\dot\phi\cdot
\text{sin}\,\theta\,\text{sin}\,\psi+\dot\theta\,\text{cos}\,\psi
\]
\[
\omega_2=\dot\phi\cdot
\text{sin}\,\theta\,\text{cos}\,\psi-\dot\theta\,\text{sin}\,\psi
\]
\[
\omega_3=\dot\phi\cdot\text{cos}\,\theta+\dot\psi
\]
\medskip

\noindent
{\bf The kinetic energy.} Suppose that $e_1,e_2,e_3$
yield principal axes for the operator of inertia.
Hence
\[ T=\frac{1}{2}[A_1\omega_1^2+A_2\omega_2^2+A_3\omega_3^2]
\]
Inserting the equations above we express $T$ as a function 
of the angle functions and their time derivatives.
\medskip

\noindent
{\bf 3.1 The symmetric case.} Assume that $A_1=A_2$. Then a simple reduction yields:
\[
T=
\frac{1}{2}A_3(\dot\psi+\dot\phi\,\text{cos}\,\theta)^2+
\frac{1}{2}A_1(\dot\theta^2+\dot\phi^2\,\text{sin}^2\,\theta)
\]


\bigskip

\noindent
Above the $\psi$-function only enters via its time derivative.
Using this, Lagrange reduced   the system to a single differential equation for $\theta$.
One refers to the symmetric case above as Lagrange's top.


\newpage


\centerline{\bf 3.2 The Kovalevsky's gyroscope in Eulerian angles}
\medskip

\noindent
Here $A_1=A_2=2$ and $A_3=1$
and the  center of mass $\mathfrak{o}$ is placed at $ae_1$
for some $a>0$.
The kinetic energy becomes
\bigskip
\[
T=
\frac{1}{2}(\dot\psi+\dot\phi\,\text{cos}\,\theta)^2+
(\dot\theta^2+\dot\phi^2\,\text{sin}^2\,\theta)
\]
\medskip

\noindent
To study the effect of the external gravity force we express
the inner product

\[
\langle e_1^*(t),e_z\rangle=
-\text{cos}\,\psi
\langle \xi_1,e_z\rangle+
\text{sin}\,\psi
\langle \xi_2,e_z\rangle=-\text{sin}\,\psi\cdot \text{sin}\,\theta
\]
\medskip

\noindent
Preservation of energy gives a constant $E$ such that

\[
\frac{1}{2}(\dot\psi+\dot\phi\,\text{cos}\,\theta)^2+
(\dot\theta^2+\dot\phi^2\,\text{sin}^2\,\theta)-
ag\cdot \text{sin}\,\psi\cdot \text{sin}\,\theta=E\tag{1}
\]
\medskip

\noindent 
Next, we consider the D'Alembert \vvv Lagrange equations. Here 
$\phi$ is a cyclic variable which gives   a constant $C$ such that
\bigskip
\[
\text{cos}\,\theta\cdot
(\dot\psi+\dot\phi\,\text{cos}\,\theta)+2\dot\phi\cdot \text{sin}^2\,\theta=C\tag{2}
\]
From this equation we can eliminate $\dot\phi$
so that the constant energy integral is a function of
$\psi,\theta$  and its time derivatives.
We have also the Lagrangean equation for the
$\psi$-variable which yields
\[
\frac{d}{dt}(\dot\psi+\dot\phi\,\text{cos}\,\theta)=
-ag\cdot \text{cos}\,\psi\cdot \text{sin}\,\theta\tag{3}
\]
\bigskip

\noindent
Above (1-3) is  a system of ODE-equations which has a unique solution for any
given initial condition. 
In the article [Acta 1892]
K�nigsberg 
implemented Kovalevsky's fourth integral
to express the solutions via
integrals of hyperelliptic functions which means that one encounters
generalised theta-functions which appear in the rich and beautiful theory
which foremost is due to Abel, Jacobi, Hermite and Weierstrass.
So at this point a "pure mathematical study is needed.
Apart from this it is of course valuable to 
find  numerical solutions
to the system above which 
describes the motion of Kovalevsky's gyroscope in
${\bf{R}}^3$.



\newpage








\end{document}










\newpage