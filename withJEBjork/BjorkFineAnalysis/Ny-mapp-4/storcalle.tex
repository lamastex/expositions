\documentclass{amsart}


\usepackage[applemac]{inputenc}

\addtolength{\hoffset}{-12mm}
\addtolength{\textwidth}{22mm}
\addtolength{\voffset}{-10mm}
\addtolength{\textheight}{20mm}

\def\uuu{_}


\def\vvv{-}


\begin{document}


\centerline{\bf\large{20. A Non-Linear PDE-equation}}

\bigskip

\noindent
{\bf{Introduction.}}
We expose  Carleman's article
\emph{�ber eine nichtlineare Randwertaufgabe bei der Gleichung $\Delta u=0$}
(Mathematisches Zeitschrift vol. 9 (1921). 
Here is the equation to be considered: Let
$\Omega$ be a bounded domain in ${\bf{R}}^3$
with $C^1$-boundary and
${\bf{R}}^+$ the non-negative real line where  $u$ is the coordinate.
Let $F(u,p)$
be a real-valued and continuous function
defined on  ${\bf{R}}^+\times\partial\Omega$.
Assume that 
\[
u\mapsto F(u),p)\tag{0,1}
\]
is strictly increasing  for every $p\in\partial\Omega$
and that $F(0,p)\geq 0$. Moreover,  
\[
\lim_{u\to\infty} F(u,p)=+\infty\tag{0.2}
\] 
holds uniformly with respect to $p$.
For a given point $Q_*\in\Omega$ we seek a function $u(x)$
which is harmonic in $\Omega\setminus\{Q_*\}$
and at $Q_*$ it is locally $\frac{1}{|x-Q_*|}$ plus a harmonic function
and on $\partial\Omega$
the  inner normal derivative $\partial u/\partial n$ satisfies the 
equation
\[
\frac{\partial u}{\partial n}(p)= F(u(p),p)\quad \colon p\in\partial\Omega\tag{*}
\]

\bigskip

\noindent 
Finally  $u$
extends to a continuous function on
$\partial\Omega$.
\medskip

\noindent {\bf {Theorem 1.}}
\emph{For each $F$ as above the boundary value problem has a unique solution.}
\medskip

\noindent
{\bf{Remark.}}
The subsequent proof 
teaches how to
handle of non\vvv linear
boundary value problems.
The strategy in Carleman's proof is to consider
the family of boundary value problems where we for each
$0\leq h\leq 1$ seek $u\uuu h$ to satisfy
\[
\frac{\partial u\uuu h}{\partial n}(p)=(1\vvv h)u\uuu h+ h\cdot  F(u\uuu h(p),p)\quad \colon p\in\partial\Omega\tag{**}
\]
where $u\uuu h$ has the same pole as $u$ above.
Starting with $h=0$ one has the  linear Neumann problem
\[
\frac{\partial u\uuu 0}{\partial n}(p)=u_0(p)
\]
This equation has a unique solution
given by
\[ 
u_0=G+\phi
\]
where $G$ is Green's function with a pole at $Q_*$ and $\phi$ is
the harmonic function in
$\Omega$ satisfying the boundary equation
\[
\frac{\partial \phi}{\partial n}(p)+\frac{\partial G}{\partial n}(p)=\phi\tag{i}
\]
Now $G$ is a super-harmonic function in
$\Omega$
and it is welknown that
$\frac{\partial G}{\partial n}$  is a continuous and positive function on
$\partial\Omega$ which gives a pair of positive constants
$0< \gamma_*<\gamma^*$ such that
\[
\gamma_*\leq \frac{\partial G}{\partial n}(p)\leq \gamma^*
\quad\colon\quad p\in\partial\Omega\tag{ii}
\] 
If $\phi$ attains its maximum at some
$p^*\in\partial\Omega$ its inner normal derivative at $p^*$ must be
$\leq 0$ and hence (i-ii) and the maximum principle for harmonic functions
entails that
\[
\max_{p\in\bar\Omega}\, \phi(p)\leq \gamma^*
\]
In a similar fashion one proves that
\[
\min_{p\in\bar\Omega}\, \phi(p)\geq \gamma_*
\]

\noindent
Next, 
one  
reduces the proof of Theorem 1 to the case when $F$ is a real\vvv analytic function
of $u$. This is easy and proved in � below.
If $F$ is real-analytic
 the subsequent proof will show that there exists $\epsilon>0$
 such that if  $0\leq h_0<1$ and a solution
$u\uuu {h\uuu 0}$ to (**) has been found, then
there exist solutions $\{u_h\}$ to (**) for all $h_0<h<h_0+\epsilon$
expressed by a convergent power series

\[ 
u_h= u_{h_0}+
\sum_{\nu01}^\infty\, (h-h_0)^\nu\cdot u_\nu
\]
where $\{u_\nu\}$ is a sequence of harmonic functions 
are found by solving linear boundary value problems.
Starting with the solution $u_0$ it will  follow that there exist
solutions $u_h$ for all $0\leq h\leq 1$ and gives the  requested solution in
Theorem 1   when $h=1$.



\bigskip


\centerline {\bf{A.0. Proof of uniqueness.}}
\medskip

\noindent
Suppose that $u_1$ and $u_2$ are two solutions to the equation (*).
Then   
$u_2-u_1$.
is harmonic in
$\Omega$ and
if $u\uuu 1\neq u\uuu 2$ we may
assume that
the maximum of $u\uuu 2\vvv u\uuu 1$ is $>0$.
The maximum is attained at some $p\uuu *\in\partial\Omega$
and  the strict maximum principle for harmonic functions gives:
\[
u\uuu 2(x)\vvv u\uuu 1(x)<
u\uuu 2(p\uuu *)\vvv u\uuu 1(p\uuu *)\tag{i}
\] 
for all $x\in\Omega$. With $v=u\uuu 2\vvv u\uuu 1$
we have
\[
\frac{\partial v}{\partial n}(p)=F(u_2(p),p)-F(u_1(p),p)
\]
Now (0.1) and (*) entail that
$\frac{\partial v}{\partial n}(p\uuu *)>0$
and since we have taken an inner normal derivative this violates
(i) which proves the uniqueness.


\bigskip



\centerline {\bf{A.1 Montone properties.}}
\bigskip


\noindent
Let $F_1$ and $F_2$ be two functions which both satisfy
(0.1) and (0.2) where  
\[ 
F_1(u,p)\leq F_2(u,p)
\]
hold for all $(u,p)\in{\bf{R}}^+\times\partial\Omega$.
If $u_1$, respectively $u_2$ solve (*) for $F_1$ and $F_2$
it follows that
$u_2(q)\leq u_1(q)$ for all $q\in\Omega$.
To see this we set $v=u_2-u_1$ which is harmonic in
$\Omega$.
If $p\in\partial\Omega$ we get
\[
\frac{\partial v}{\partial n}(p)=F_2(u_2(p),p)-F_1(u_1(p),p)\geq 0\tag{i}
\]
Suppose that the maximum of $v$ is $>0$ and let the maximum be attained at some 
point $p_*$. Since (i) is an inner normal it follows that we must have
$0=\frac{\partial v}{\partial n}(p)$ which would entail that

\[ 
F_2(u_2(p_*)p_*)>F_2(u_1(p_*),p_*)\geq F_1(u_1(p_*),p_*)\implies
\]
and this contradicts the strict inequality
$u\uuu 2(p\uuu *)>
u\uuu 1(p\uuu *)$
since we have an increasing function in (0.1).





\medskip


\noindent
{\bf{A.2. A bound for the maximum norm.}}
Let $u$ be a solution to (*) and $M_u$ denotes
the maximum norm of its restriction to
$\partial\Omega$. Choose
$p^*\in\partial\Omega$
such that
\[
 u(p^*)=M_u\tag{1}
\]
Let $G$ be the Green's function which has a pole
at $Q_*$ while $G=0$ on $\partial\Omega$. Now
\[
h=u-M_u-G
\] 
is a harmonic function in
$\Omega$.
On the boundary we have $h\leq 0$
and $h(p^*)=0$. So $p^*$ is a maximum point for this harmonic function
in the whole closed domain $\bar\Omega$.
It follows that
\[ 
\frac{\partial h}{\partial n}(p^*)\leq 0\implies
\] 
\[
F(u(p^*),p^*)=\frac{\partial u}{\partial n}(p^*)\leq
\frac{\partial G}{\partial n}(p^*)
\]
Set
\[ 
A^*=\max_{p\in\partial\Omega}\, \frac{\partial G}{\partial n}(p)
\]
Then we have
\[ 
F(M_u,p^*)\leq A^*\tag{*}
\]
Hence the assumption (0.2) for $F$ this gives a robust   estimate for the 
maximum norm $M_u$.
Next, let $m_u$ be the minimum of $u$ on $\partial\Omega$ and
consider the harmonic function
\[
h=u-m_u-G
\]
This time $h\geq 0$ on $\partial\Omega$
and if $u(p_*)=m_u$ we have $h(p_*)=0$
so here $p_*$ is a minimum for $h$.
It follows that
\[ 
\frac{\partial h}{\partial n}(p_*)\geq 0\implies
F(u(p_*),p)=\frac{\partial u}{\partial n}(p_*)\geq 
\frac{\partial G}{\partial n}(p_*)
\]
So with
\[
A_*=\min_{p\in\partial\Omega}\,\, \frac{\partial G}{\partial n}(p)
\]
one has the inequality
\[
F(m_u,p^*)\geq A_*\tag{**}
\]



\noindent
{\bf{Remark.}} Above $0<A_*<A^*$ are  constants which are independent of $F$.
Hence the maximum norms of 
solutions $u=u_F$ are controlled if the $F$-functions stay
in a family where  (0.2) holds  uniformly.


\bigskip



\centerline {\bf{B. The  linear equation.}}

\medskip


\noindent
Let $f(p)$ and $W(p)$
be a pair of continuous functions on the boundary
$\partial \Omega$ where  $W$ is positive, i.e. $W(p)>0$ for every
boundary point.
The classical Neumann
theorem
asserts  that there exists a unique function $U$ which is harmonic in
$\Omega$, extends to a continuous function on
the closed domain and its inner normal  derivative satisfies:




\[ 
\partial U/\partial n(p)=W(p)\cdot U(p)+f(p)\quad p\in\partial\Omega\tag{1}
\] 
The uniqueness  is a consequence of Green's formula.
For suppose that $U_1$ and $U_2$ are two solutions to (1) and set $v=U_1-U_2$.
Since $v$ is harmonic in $\Omega$ it follows that:
\[ 
\iiint_\Omega\, |\nabla(v)|^2 dxdydz+
\iint_{\partial\Omega}\, v\cdot \partial v/\partial n\cdot dS=0
\]
Here $\partial v/\partial n= W(p)v$ and since $W(p)>0$ holds on $\partial\Omega$
we conclude that
$v$ must be identically zero.
For the unique   solution to (1)  some  estimates hold.
Namely, set 
\[
M_U=\max_p\, U(p)\quad\text{and}\quad
m_U=\min_p\, U(p)
\] 

\medskip

\noindent
Since $U$ is harmonic in $\Omega$ the
the  maximum and the minimum are taken on the boundary.
If $U(p^*)= M_U$ for some $p^*\in\partial \Omega$
we have $\partial U/\partial n(p^*)\leq 0$.
Set

\[
W_*=\min_p \, W(p)
\]
By assumption $W_*>0$
and we get 
\[
M_U\cdot W(p^*)+f(p^*)=\partial U/\partial n(p^*)\leq 0\implies
M_U\leq  \frac{|f|_{\partial\Omega}}{W_*}
\]
where
$|f|_{\partial\Omega}$ is the maximum norm of $f$ on the boundary.
In the same way one verifies that
\[
m_U\geq -\frac{|f|_{\partial\Omega}}{W_*}
\]
Hence  the following inequality holds for the
 the maximum norm  $|U|_{\partial\Omega}$ :
\[
|U|_{\partial\Omega}\leq
\frac{|f|_{\partial\Omega}}{W_*}\tag{*}
\]


\noindent
{\bf{B.1 Estimates for first order derivatives.}}
Let $p\in\partial \Omega$ and denote by $N$
the inner normal at $p$. Since $\partial\Omega$ is of class $C^1$
a sufficiently small line segment from $p$ along
$N$ stays in $\Omega$. So at points $q=p+\ell \cdot N$
we can take the directional derivative of $U$ along $N_p$
This gives  a function

\[
 \ell\mapsto \partial U/\partial N(p+\ell\cdot N)
\]
Since the boundary is $C^1$
these functions are defined on a fixed interval $0\leq\ell\leq \ell^*$ for all $p$.
With these notations there exists a constant $B$ such
that 
\[
\bigl|\partial U/\partial N(p+\ell\cdot N)\bigr|\leq B\cdot 
||\partial U/\partial n||_{\partial\Omega}\quad
\colon\, p\in\partial\Omega\,\, \colon\,\, 0\leq \ell\leq \ell^*\tag{**}
\]


\noindent
where the size of $B$ is controlled by the maximum norm of $f$
on $\partial\Omega$ and the positive constant $W_*$ above.

\bigskip


\centerline {\bf{C. Proof of Theorem 1}} 

\bigskip

\noindent
It suffices to prove
the theorem
when
$F(u,p)$ is an analytic function with respect to $u$.
For if we have  an arbitrary
$F$-function 
satisfying  (0.1) and (0.2), then $F$ can be  uniformly
approximated by a sequence 
$\{F_n\}$ of analytic
functions. See �� below for an explicit approximation when
a continuous function $F$ is given.
If $\{u_n\}$ are the 
unique solutions  to $\{F_n\}$
the  estimates in (B)  show
that there exists a limit
function $\lim_{n\to\infty}  u_n=u$ where  $u$ solves (*) for the given
$F$-function.
So now we can assume that $u\mapsto F(u,p)$ is a real-analytic function on
the positive real axis for each $p\in\partial\Omega$
where local power series converge uniformly with respect to $p$ and
there remains to prove
the
existence  of  a solution to the PDE in (*) above Theorem 1.


\bigskip

\noindent
{\bf{C.1  The succesive solutions $\{u_h\}$.}}
To each real number $0\leq h\leq 1$ we seek a solution $u_h$ where
\[
 \frac{\partial u_h}{\partial n}(p)= h\cdot F(u_h,p)+(1-h)\cdot u_h(p)\tag{1}
\]
With $h=0$ we get  a solution as explained in  the introduction.
Next, let $0\leq h_0<1$ and suppose we have found the 
solution $u_{h_0}$ in (1) above. Set $u_0=u_{h_0}$
and
with $h=h_0+\alpha$ for some small $\alpha>0$
we shall find  $u_h$ by a series
\[
u_h= u_0+\sum_{\nu=1}^\infty\, \alpha^\nu\cdot u_\nu\tag{3}
 \]

 
\noindent
The pole at $Q_*$  occurs already in $u_0$ so 
$u_1,u_2,\ldots$ are 
harmonic functions in  $\Omega$. There remains to determine
this sequence so that
$u_h$ yields  a solution to (1). We will  show that this can
be achieved when $\alpha $ is sufficiently small.
To begin with the results from
(B) give  positive constants $0<c_1<c_2$
such that
\[
0<c_1\leq u_0(p)\leq c_2\quad\colon p\in\partial\Omega\tag{4}
\]



\noindent
Next,  the analyticity of $F$ with respect to $u$
enables us to write:
\[
 F(u_h(p),p)=
F(u_0(p)+\sum_{k=1}^\infty\, c_k(p)\cdot \bigl[\sum_{\nu=1}^\infty \alpha^\nu u_\nu(p)
 \bigr ]^k\tag{5}
 \]


\noindent
where $\{c_k(p)\}$ are continuous functions on
 $\partial\Omega$ which appear in an expansion
 \[ 
 F(u_0(p)+\xi,p)=F(u_0(p),p)+
 \sum_{k=1}^\infty\, c_k(p)\cdot \xi^k\tag{6}
 \]
 
 
\noindent
Here (4) and the hypothesis on $F$ entail that
the radius of convergence has a uniform bound below, i.e.
there exist positive constants
 $\rho>0$
and $C$ such that which are independent of $h$ such that
\[
\max_{p\in\partial\Omega}\, |c_k(p)|\leq
C\cdot \rho^k\quad\colon\quad k=0,1,\ldots\tag{7}
\]
Moreover, the hypothesis (0.2) from the introduction
gives a posotove constant $C_*$ whgich also is independent of $h$
such that
\[
\min_{p\in\partial\Omega}\, |c_1(p)|\geq C_*\tag{8}
\]

\noindent
Next,  (1) is solved where
the harmonic functions
$\{u_\nu\}$ which are determined inductively
while $\alpha$-powers are identified.
The linear
$\alpha$-term gives  the equation
\[
\frac{\partial u_1}{\partial n}=F(u_0(p),p)-u_0(p)+
(1-h_0)u_1+h_0\cdot c_1(p)\cdot u_1(p)\tag{9}
\]



\noindent
For $u_2$ we find that
\[
\frac{\partial u_2}{\partial n}=\bigl(1-h_0+h_0c_1(p)\bigr) u_2
-u_1+c_1(p)u_1+c_2(p)u_1^2\tag{10}
\]



\noindent
In general, for $\nu\geq 3$ one has
\[
\frac{\partial u_\nu}{\partial n}=(1-h_0+h_0\cdot c_1(p))\cdot u_\nu+
R_\nu(u_0,\ldots,u_{\nu-1},p)\tag{11}
\] 
where $\{R_\nu\}$ are polynomials in the preeceding $u$-functions
whose coefficients are
determined  via the
$c$-functions above.
Notice that (8) gives a positive constant $C_*$
which again is independent of $h$ such that
\[ 
\min_{p\in\partial\Omega}1-h_0+h_0\cdot c_1(p)\geq C_*\tag{12}
\]
Next, the equations in (11) can be expressed as follows:

\[
\frac{\partial u_m}{\partial n}= \bigl(1-h_0+h_0\cdot c_1(p)\bigr)
\cdot u_\nu(p)+
\alpha\cdot \bigl\{\,\sum_{k=1}^\infty c_k(p)\bigl[\sum\, \alpha^\nu u_\nu(p)\bigr ]^k\,\bigr\}_{m-1}\tag{13}
\]
where the index $\nu-1$ indicates that one takes out
the coefficient of $\alpha^{m-1}$
when the double sum inside the bracket is expanded as a series in
$\alpha$.
Next, using (12) and the estimates for the inhomogeneous linear equation in
� B we have a constant $C^*$ which again is independent of $h$ such that

\[
\max_{p\in\Omega}\,  |u_m(p)|\leq
\alpha\cdot \max_{p\in\Omega}\bigl |
\bigl\{\,\sum_{k=1}^\infty c_k(p)\bigl[\sum\, \alpha^\nu u_\nu(p)\bigr ]^k\,\bigr\}_{m-1}
\,\bigr|\leq
\]
\[
C\cdot\alpha\cdot \sum_{k=1}^\infty \,\rho^k\cdot
\max_{p\in\Omega}
\,\bigl |\,\bigl\{ [\,\sum\, \alpha^\nu u_\nu(p)\bigr ]^k
\,\bigr \} _{m-1}\,\bigr|\tag{14}
\]

\bigskip

Finish with a MAJORANT SERIES.

\newpage
























\end{document}