%\documentclass{amsart}


%\usepackage[applemac]{inputenc}


%\addtolength{\hoffset}{-12mm}
%\addtolength{\textwidth}{22mm}
%\addtolength{\voffset}{-10mm}
%\addtolength{\textheight}{20mm}

%\def\uuu{_}


%\def\vvv{-}


%\begin{document}

\centerline{\bf\large{ II. The Jensen-Nevanlinna class and Blaschke products.}}


\bigskip


\noindent

\bigskip


\noindent
\emph{0. Introduction.}
\bigskip


\noindent
\emph{1. The Herglotz integral.}
\bigskip


\noindent
\emph{2. The class JN(D)}

\bigskip


\noindent
\emph{3. Blaschke products}


\bigskip


\noindent
\emph{4.  Invariant subspaces of $H^2(T)$}
\bigskip


\noindent
\emph{5.  Beurling's closure theorem}
\bigskip


\noindent
\emph{6 The Helson\vvv Szeg� theorem.}



\bigskip


\centerline{\bf{Introduction.}}
\bigskip

\noindent
If $\mu$ is a real Riesz measure on the unit circle
there exist the  harmonic function in the disc $D$ defined by
\[
H\uuu\mu(z)=\frac{1}{2\pi}\int_0^{2\pi}\,
\frac{1\vvv |z|^2}{|e^{i\theta}-z|^2}\cdot d\mu(\theta)\tag{0.1}
\]
For each $0<r<1$ one has the inequality
\[
\int_0^{2\pi}\,\bigl | 
H_\mu(re^{i\theta})\bigr |\cdot d\theta\leq
||\mu||\tag{0.2}
\] 
where $||\mu||$ is the total variation of $\mu$.
Moreover, there exists  a weak limit, i.e. 
\[
\lim\uuu{r\to 1}\int_0^{2\pi}\, g(\theta)\cdot H_\mu(re^{i\theta})=
\int_0^{2\pi}\, g(\theta)\cdot d\mu(\theta)\tag{0.3}
\]
holds for every continuous function $g(\theta)$ on $T$.
Conversely we proved in XX that if $H(z)$ is a harmonic function in
$D$ for which there exists a constant $C$ such that
\[
\int_0^{2\pi}\,\bigl | 
H(re^{i\theta})\bigr |\cdot d\theta\leq C\tag{0.4}
\] 
hold for all $r<1$, then there exists a unique Riesz measure
$\mu$ on $T$ where $H=H\uuu \mu$.
Hence there is a 1\vvv 1 correspondence between
the space of harmonic functions in $D$ satisfying (0.4) and the space of real Riesz measures on $T$.
There also exist radial limits almost everywhere.
More precisely, define the
$\mu$\vvv primitive
function
\[
\psi(\theta)= \int\uuu 0^\theta\, d\mu(s)
\]

\noindent
Fatou's Theorem
asserts that for each Riesz measure $\mu$ there exists a radial limit
\[
H\uuu\mu^*(\theta)=\lim\uuu{r\to 1}\, H(re^{i\theta})\tag{0.5}
\]
for each $\theta$ where
$\psi$ has an ordinary derivative.
Since $\psi$ has a bounded variation this holds
almost everywhere by Lebegue's  Theorem in [Measure].
\medskip

\noindent
{\bf{0.6 The case when $\mu$ is singular.}}
If $\mu$ is singular 
the radial limit (0.5) is zero almost everywhere.
If the singular measure $\mu$ is non\vvv negative
with
total mass $2\pi$ we have
$H\uuu\mu(0)=1$ and  the mean\vvv value property for harmonic
functions gives:
\[
\int\uuu 0^{2\pi}\, H\uuu\mu(re^{i\theta})\cdot d\theta=1
\]
for all $0<r<1$.
At the same time the boundary function $H\uuu\mu^*(\theta)$ is almost everywhere zero
which means that no  dominated convergence occurs.
\medskip

\noindent
{\bf{0.7 Exercise.}}
Let $\mu$ be singular with
a Hahn\vvv decomposition
$\mu=\mu\uuu+ \vvv\mu\uuu \vvv$.
Assume that the positive part $\mu\uuu +(T)= a>0$.
Now there exists a closed null set $E$ such that
$\mu\uuu +(E)\geq a\vvv \epsilon$ while $\mu\uuu\vvv(E)=0$.
The last equation
gives a small $\delta>0$
such that if $E\uuu{2\delta}$ is the open
$2\delta$\vvv neighborhood of $E$ then
$\mu\uuu \vvv(E\uuu{2\delta})<\epsilon$.
Set


\[
H\uuu *(z) =
\frac{1}{2\pi}\int_E\,
\frac{1\vvv |z|^2}{|e^{i\theta}-z|^2}\cdot d\mu\uuu +(\theta)
\]
Since $\mu\uuu +(E)\geq a\vvv \epsilon$ we get
\[
\int\uuu 0^{2\pi}\, H\uuu *(re^{i\theta})\cdot d\theta \geq a\vvv \epsilon\tag{ii}
\]
Next, for each pair  $\phi\in E\uuu\delta$ and $e^{i\theta}\in T\setminus E\uuu{2\delta}$
we have:
\[
\frac{1\vvv r^2}{|e^{i\theta}-re^{i\phi}|^2}
\leq \frac{2(1\vvv r)}{1+r^2\vvv 2r\cos(\delta)}
\]
So with

\[
H\uuu\delta(z)=\frac{1}{2\pi}\int_{T\setminus E\uuu{2\delta}}\,
\frac{1\vvv |z|^2}{|e^{i\theta}-z|^2}\cdot d\mu\uuu (\theta)
\]
it follows that
\[
|H\uuu\delta(re^{i\phi})|\leq\frac{1}{2\pi}\cdot 
\frac{2(1\vvv r)}{1+r^2\vvv 2r\cos(\delta)}
\cdot \int\uuu{T\setminus E\uuu{2\delta}}\, |d\mu(\theta)|\tag{iii}
\]
for each $\phi\in E\uuu\delta$.
Since $H\uuu *$ is constructed via the
restriction of $\mu\uuu +$ to $E$, a similar reasoning gives:
\[
|H\uuu*(re^{i\phi})|\leq
\frac{1}{2\pi}\frac{2(1\vvv r)}{1+r^2\vvv 2r\cos(\delta)}
\cdot \mu\uuu +(E)\tag{iv}
\]
when $e^{i\phi}\in T\setminus E\uuu\delta$.
Next, by the constructions above we have

\[ 
H=H\uuu *+H\uuu \delta+H\uuu\nu
\]
where $\nu$ is the measure given by
the restriction of $\mu\uuu +$ to $E\uuu{2\delta}\setminus E$
minus $\mu\uuu \vvv$ restricted to
to $E\uuu{2\delta}$. So by the above the total variation
$||\nu||\leq 2\epsilon$ which gives

\[
\int\uuu 0^{2\pi}\, |H\uuu\nu(re^{i\theta})|\cdot d\theta
\leq 2\epsilon
\]


\medskip

\noindent
Deduce from the above that one has an inequality

\[
\int\uuu {E\uuu\delta}\, H(re^{i\phi})\cdot d\phi\geq
a\vvv \bigl[2\epsilon+\frac{1}{\pi}
\frac{2(1\vvv r)}{1+r^2\vvv 2r\cos(\delta)}
\cdot ||\mu||\bigl]\tag{*}
\]
Since $E$ is a null\vvv set
this shows  that
mean\vvv value integrals of $H$ behave in an "irregular fashion"
when $r\to 1$.









\bigskip



\centerline{\bf{1. The Herglotz integral }}

\bigskip

\noindent
Let $\mu$ be a real  Riesz measure on
the unit circle $T$.
Set
\[
g_\mu(z)=
\frac{1}{2\pi}\int_0^{2\pi}\,
\frac{e^{i\theta}+z}{e^{i\theta}-z}\cdot d\mu(\theta)\tag{*}
\]
This   analytic function is called the Herglotz
extension of the Riesz measure. 
Since $\mu$ is real it follows that
\[
\mathfrak{Re}\,g_\mu(z)=\frac{1}{2\pi}
\int_0^{2\pi}\,
\frac{1-|z[^2}{|e^{i\theta}-z|^2}\cdot d\mu(\theta)=H\uuu\mu(z)
\]
In particular the $L^1$\vvv norms from  (0.2) are uniformly bounded with
respect to $r$ when we integrate the absolute value of
$\mathfrak{Re}\,g\uuu\mu$.
But the conjugate harmonic function representing
$\mathfrak{Im}\, g\uuu\mu$ does not satisfy (0.2) in general.
However the following holds:

\medskip

\noindent
{\bf{1.1 Theorem.}}
\emph{For almost every $\theta$ there exists a radial limit}
\[
\lim\uuu{r\to 1} \,g_\mu(re^{i\theta})
\]


\noindent
To prove Theorem 1.1 we shall use some tricks.
The Hahn\vvv decomposition
$\mu=\mu\uuu +\vvv \mu\uuu \vvv$
enables us to express $g\uuu\mu$ as a difference
$g\uuu 1\vvv g\uuu 2$ where  $g\uuu 1,g\uuu 2 $ both are
Herglotz extensions of
non\vvv negative Riesz measures and hence
$\mathfrak{Re}\,g\uuu\nu(z)>0$ in $D$.
Let us now discuss analytic functions with a positive real part.

\medskip

\noindent
{\bf{1.2 Exercise.}}
Let $f\in\mathcal O(D)$ where $\mathfrak{Re}\,f(z)>0$
and $\mathfrak{Im} f(0)=0$. Set $f=u+iv$ which gives
the analytic function
\[ 
\phi(z)=\log (1+u+iv)
\]
Here
\[ 
\mathfrak{Re}\,\phi=\log\,|1+u+iv|=
\frac{1}{2}\log [(1+u)^2+v^2]
\]
In particular $\mathfrak{Re}\,\phi>0$
so 
this  harmonic function has a radial limit almost everywhere.
We also know that $u$ has a radial limit almost everywhere
and from this the reader may conclude that there
almost everywhere exist
finite radial limits
\[
\lim\uuu{r\to 1}\, v^2(re^{i\theta})\tag{1}
\]
In order to determine the sign of these
radial limits we consider  the analytic function
\[ 
\psi= e^{\vvv u\vvv iv}
\]
Since $u>0$ we have
 $|\psi(z)|= e^{\vvv u(z)}\leq 1$
and hence $\psi(z)$ is a bounded analytic function in
$D$.
The Brothers
Riesz theorem shows that $\psi$ has a radial limit almost everywhere.
Finally, when we have a radial limit
\[
 \lim\uuu {r\to 1}\, e^{\vvv u(re^{i\theta})\vvv i v(re^{i\theta})}
\] 
and  in addition suppose that
$u$ has a radial limit, then it is clear that $v$ has a radial limit too.
\bigskip

\noindent
\emph {Proof of Theorem 1.1}
By the Hahn\vvv decomposition of $\mu$
the proof is reduced to the case $\mu\geq 0$ and 
Exercise 1.2 applies.

\bigskip

\noindent
{\bf{1.3 The case when $\mu$ is singular.}}
When this holds the radial limits of $\mathfrak{Re}\, g\uuu\mu$ are almost
everywhere zero.
With $v=\mathfrak{Im}\, g\uuu\mu$
there remains  to study  the almost 
everywhere defined function
\[ 
v^*(\theta)=\lim\uuu{r\to 1}\, v(re^{i\theta})
\]
It turns out that this  Lebesgue\vvv measurable function 
never is integrable when $\mu$ is singular.
In fact, the Brothers Riesz theorem
shows that if
there exists a constant $C$ such that

\[
\int\uuu 0^{2\pi}\, |v(re^{i\theta})|\cdot d\theta\leq C
\]
hold for all $r<1$, then the analytic function $g\uuu\mu$ belongs
to the Hardy space and its radial limits give an
$L^1$\vvv function $g^*(\theta)$ on the unit circle which
would entail that $\sigma$ is equal to the absolutely continuous
measure defined by $g^*$.
Thus,  for every singular measure $\mu$ one has

\[ 
\lim\uuu {r\to 1}\, \int\uuu 0^{2\pi}\,
\bigl| \,\mathfrak{Im} \,g\uuu\mu(re^{i\theta})\bigr | \cdot d\theta=+\infty\tag{*}
\]

\medskip

\noindent
{\bf{1.4 Example.}} Take the case where $\mu$ is 
$2\pi$ times the Dirac measure at $\theta=0$
which gives
the analytic function
\[ 
g(z)=\frac{1+z}{1\vvv z}
\]
It follows that
\[
v(re^{i\theta})= \vvv 2r\cdot \frac{\sin\theta}{1+r^2\vvv 2 r\cos\,\theta}
\]
and radial limits exist except for $\theta=\pi/2$ or $\vvv \pi/2$, i.e.

\[
v^*(\theta)=\vvv2\cdot \frac{\sin\theta}{2\vvv 2 \cos\,\theta}
\]
when $\theta$ is $\neq \pi/2$ and $\vvv \pi/2$.
At the same time
the reader may verify that $v^*(\theta)$ does not belong to $L^1(T)$
and  that
\[ 
\int\uuu 0^{2\pi}\, |v(re^{i\theta})|\cdot d\theta\simeq
\log\,\frac{1}{1\vvv r}
\] 
as $r\to 1$.
\medskip




\bigskip


\centerline{\bf{2. The Jensen\vvv Nevanlinna class}}



\bigskip

\noindent
Every Riesz measure $\mu$ on $T$
gives the zero\vvv free analytic function
\[ 
G\uuu\mu(z)=e^{g\uuu\mu(z)}\tag{*}
\]
Here 
$\log\,|G\uuu\mu (z)|=\mathfrak{Re}\,g\uuu\mu(z)$ which
gives the inequality

\[
\log^+|G\uuu\mu(z)|\leq \bigl|\mathfrak{Re}\,g\uuu\mu(z)\bigr|
\]
Applying  (0.2) we obtain:
\[
\int_0^{2\pi}\,\log^+|G\uuu\mu (re^{i\theta})|\cdot d\theta\leq ||\mu||\tag{**}
\]
for each $r<1$.
\medskip

\noindent{\bf{2.1 A converse.}}
Let $F(z)$ be a zero\vvv free analytic function in $D$ where
$F(0)=1$. Assume that there exists a constant $C$ such that
\[
\int_0^{2\pi}\,\log^+|F(re^{i\theta})|\cdot d\theta\leq C\tag{i}
\]
hold for each $r<1$.
The mean\vvv value property applied to the harmonic function
$H=\log\, |F|$
gives

\[
\int_0^{2\pi}\,|H(re^{i\theta})|\cdot d\theta=
2\cdot \int_0^{2\pi}\,\log^+|F(re^{i\theta})|\cdot d\theta\tag{ii}
\]
\medskip

\noindent
Hence (i) entails that $H$ satisfies (0.4)
and now the reader can settle the following:

\medskip

\noindent
{\bf{2.2 Exercise.}}
Show that (i) above entails that there exists a Riesz measure $\mu$ such that
$F=G\uuu\mu$ where the normalisation $F(0)=1$ gives
$\mu(T)=2\pi$.




\medskip

\noindent{\bf{2.3 Radial limits.}}
Whenever $g\uuu\mu$ has a radial limit for some
$\theta$ it is clear that $G\uuu\mu$ also has a radial limit in this
direction. So Theorem 1.1 implies that there
exists an almost everywhere defined boundary
function
\[
G^*\uuu\mu(\theta)=\lim\uuu{r\to 1}\,
G\uuu\mu(re^{i\theta})
\]







\noindent
The material above suggests the following:

\newpage


\noindent {\bf 2.4 Definition.}
\emph{An analytic function $f$ in $D$
belongs to the Jensen-Nevanlinna class
if there exists a constant
$C$ such that}
\[
\int_0^{2\pi}\,\log^+|f(re^{i\theta})|\cdot d\theta\leq C
\] 
 \emph{hold for all $r<1$.
The family of Jensen-Nevannlina functions is denoted by
$\text{JN}(D)$.}
\medskip

\noindent
Above we described zero\vvv free functions in 
$\text{JN}(D)$.
Now we shall  study eventual    zeros of
functions in $\text{JN}(D)$.
Recall  that if $f\in\mathcal O(D)$ 
where $f(0)=1$ then  Jensen's formula gives:
\[
\sum_{|\alpha_\nu|<r}\,\text{Log}\,\frac{r}{|\alpha_\nu|}=
\frac{1}{2\pi}\int_0^{2\pi}\,
\log\,|f(re^{i\theta})\bigr|\cdot d\theta\quad\colon\quad 0<r<1\tag{*}
\]
where the left hand side is the sum of zeros of $f$ in the disc 
$D_r$.
\medskip

\noindent
{\bf{A notation.}}
If $f\in\mathcal O(D)$ and $r<1$ we set
\[
\mathcal T\uuu f(r)=\int_0^{2\pi}\,\log^+|f(re^{i\theta})|\cdot d\theta
\] 

\noindent
Since $\log\,|f(re^{i\theta})\bigr|\leq \log^+|f|$ it follows that
\[
\sum_{|\alpha_\nu|<r}\,\text{Log}\,\frac{r}{|\alpha_\nu|}\leq
\mathcal T_f(r)
\]
So if $f\in \text{JN}(D)$ we can pass to the limit 
as $r\to 1$ and conclude that the positive series
\[
\sum\,\text{Log}\,\frac{1}{|\alpha_\nu|}<\infty\tag{**}
\] 
where the sum is taken over all zeros  in $D$.
Next, recall form XX that the positive series  (**) converges
if and only if
\[
\sum\,(1-|\alpha_\nu|<\infty\tag{***}
\]
When (***) holds
we say that
the sequence $\{\alpha\uuu\nu\}$ satisfies the Blaschke condition.
Hence we have proved:

\medskip

\noindent
{\bf 2.5 Theorem.} \emph{Let $f$ be in  JN(D). Then 
its  zero set satisfies the Blaschke condition.}


\bigskip



\centerline
{\bf 3. Blaschke products.}
\bigskip

\noindent
Consider an infinite sequence
$\{\alpha_\nu\}$ in $D$  where $|\alpha_1|\leq |\alpha_2|\leq\ldots$
and the Blaschke condition holds.
For every $N\geq 1$ we put:
\[ 
B\uuu N(z)=
\prod\uuu{\nu=1}^{\nu=N}\,\frac{|\alpha_\nu|}{\alpha_\nu}\cdot
\frac{\alpha_\nu-z}{1-\bar\alpha_\nu z}
\]
We are going to prove that the sequence of analytic function
$\{B\uuu N\}$ converge in $D$ to a limit function
$B(z)$ expressed by the infinite product

\[
B(z)=
\prod\uuu{\nu=1}^\infty\,
\frac{|\alpha_\nu|}{\alpha_\nu}\cdot
\frac{\alpha_\nu-z}{1-\bar\alpha_\nu z}\tag{3.1}
\]

\noindent
To prove this  we first analyze
the individual factors.
For each non\vvv zero $\alpha\in D$
we set
\[ 
B\uuu\alpha(z)=\frac{|\alpha|}{\alpha}\cdot
\frac{\alpha-z}{1-\bar\alpha z}
\]


\noindent
{\bf{Exercise.}}
Show that
\[
B_\alpha(z)=
|\alpha|\cdot\frac{1-z/\alpha}{1-\bar\alpha z}=
|\alpha|+\frac{|\alpha|^2-1}{1-\bar\alpha z}
\cdot\frac{|\alpha|}{\alpha}\cdot z\tag{i}
\]
and conclude that
\[
B_\alpha(z)\vvv 1=(|\alpha|-1)\cdot[1+\frac{|\alpha|+1}{1-\bar\alpha z}
\cdot\frac{|\alpha|}{\alpha}\cdot z]\tag{ii}
\]
Finally, use the triangle inequality to show the inequality
\[
\max_{|z|=r}\,|B_\alpha(z)-1|\leq (1-|\alpha|)\cdot
(1+\frac{2r}{1-r})=\frac{1+r}{1-r}\cdot (1-|\alpha|)\tag{iii}
\]


\noindent
{\bf{The convergence of (3.1)}}
From (iii) and general results about product series
the requested convergence in (3.1) follows from the assumed
Blaschke condition.
In fact, when $|z|\leq r<1$ stays in a compact disc
the Blaschke condition and (iii) entail that
\[
\sum\uuu{\nu=1}^\infty \max_{|z|=r}\,|B_\alpha(z)-1|<\infty
\]
which implies that (3.1) converges uniformly on $|z|\leq r$ to an
analytic function and since
$r<1$ is arbitrary we get a limit function $B(z)\in\mathcal O(D)$.
\medskip

\noindent
{\bf{3.2 Exercise.}} The rate of convergence in $|z|\leq r$
 can be described as follows:
For each  $N\geq 1$ we set
\[ 
G_N(z)=
\prod_{\nu=N+1}^\infty\,
B_{\alpha_\nu}(z)\quad\colon\quad \Gamma_N=
\sum_{\nu=N+1}^\infty\,1-|\alpha_\nu|
\]

\noindent
With $r<1$ kept fixed
we choose $n$ so large that
\[
\frac{1+r}{1-r}\cdot (1-|\alpha_\nu|)\leq \frac{1}{2}\quad\colon\quad \nu>N
\]
Show that this gives:
\[
\max_{|z|=r}\,
|G_N(z)-1|\leq 8\cdot\frac{1+r}{1-r}\cdot \Gamma_N
\]

\medskip

\noindent
Since the Blaschke condition implies that
$\Gamma\uuu N\to 0$ as $N\to\infty$
this gives a control for the rate of convergence in
$|z|\leq r$.



\bigskip

\centerline {\bf 3.3 Radial limits of $B(z)$}
\bigskip

\noindent
When $z=e^{i\theta}$ the absolute value 
$|B\uuu\alpha(e^{i\theta}|=1$.
So $B(z)$ is the product of analytic functions where every
term has absolute value $\leq 1$ and hence
the maximum norm
\[ 
\max\uuu{z\in D}\, |B(z)|\leq 1
\]





\medskip


\noindent
Since the  analytic function $B(z)$ is bounded,
Fatou's Theorem from Section XX  gives
an almost everywhere defined  limit function
\[ 
B^*(e^{i\theta})=
\lim_{r\to 1}\, B(re^{i\theta})\tag{1}
\]
where the radial convergence holds almost everywhere.
Moreover, the Brothers Riesz theorem
gives:
\[
\lim_{r\to 1}\int_0^{2\pi}\,
|B^*(e^{i\theta})-B(re^{i\theta})|d\theta=0\tag{2}
\]

\medskip

\noindent
{\bf{3.4 Theorem.}}\emph{ The equality}
\[
|B^*(e^{i\theta})|=1\quad\text{holds almost everywhere}
\tag{*}
\] 
\medskip

\noindent
\emph{Proof.}
Since $|B^*|\leq 1$ it is clear that (*) follows if we have proved that
\[
\int\uuu 0^{2\pi}\,|B^*(e^{i\theta})|\cdot d\theta=1\tag{i}
\]
Using (2) above and the triangle inequality we get (i) if we prove the limit formula
\[
\lim\uuu {r\to 1}\,
\int\uuu 0^{2\pi}\,|B(re^{i\theta})|\cdot d\theta=1
\tag{ii}
\]
To show (ii) we will apply Jensen's formulas to $B(z)$ in discs
$|z|\leq r$. 
The convergent product which defines $B(z)$ gives

\[
B(0)=\prod\,\log\,|\alpha\uuu\nu|
\]
Next, for $0<r<1$ Jensen's formula gives
\[
\log B(0)=\sum\uuu{\nu=1}^{\rho(r)}\,\log\,\frac{|\alpha\uuu\nu|}{r}
+
\frac{1}{2\pi}\int\, \int\uuu 0^{2\pi}\,\log\,|B(re^{i\theta})|\cdot d\theta
\]
where $\rho(r)$ is the largest $\nu$ for which $|\alpha\uuu\nu|=r$.
It follows that
\[
\frac{1}{2\pi}\int\, \int\uuu 0^{2\pi}\,\log\,|B(re^{i\theta})|\cdot d\theta
\geq \sum\uuu{\nu=1}^{\rho(r)}\,\log\,\frac{r}{|\alpha\uuu\nu|}\vvv
\sum\uuu{\nu=1}^\infty \,\log\,\frac{1}{|\alpha\uuu\nu|}\tag{1}
\]
Next, with $\epsilon>0$ we find an integer $N$ such that
\[
\sum\uuu{\nu=1}^{\nu=N} \,\log\,\frac{1}{|\alpha\uuu\nu|}<\epsilon \tag{2}
\]
Since $|\alpha\uuu\nu|\to 1$
here exists $r\uuu*$ such that
\[ 
r\geq r\uuu *\implies \rho(r)\geq N\tag{3}
\]
When (3) holds it follows from (1\vvv 2) that
\[
\frac{1}{2\pi}\int\, \int\uuu 0^{2\pi}\,\log\,|B(re^{i\theta})|\cdot d\theta
\geq \sum\uuu{\nu=1}^{\rho(r)}\,\log\,\frac{r}{|\alpha\uuu\nu|}\vvv
\sum\uuu{\nu=1}^{\rho(r\uuu *)}\,\log\,\frac{1}{|\alpha\uuu\nu|}\vvv\epsilon\tag{4}
\]
In the first sum every term is $\geq 1$
so we get
a better inequality when the sum is restricted to $\nu\leq\rho(r\uuu *)$, i.e. we have
\[
\frac{1}{2\pi}\int\, \int\uuu 0^{2\pi}\,\log\,|B(re^{i\theta})|\cdot d\theta
\geq \sum\uuu{\nu=1}^{\rho(r\uuu *)}\,\log\,\frac{r}{\alpha\uuu\nu|}\vvv
\sum\uuu{\nu=1}^{\rho(r\uuu *)}\,\log\,\frac{1}{\alpha\uuu\nu|}\vvv\epsilon\tag{5}
\]
Here (5) hold for every $r\uuu *<r<1$ and a passing to the limit 
as $r\to 1$ where we only have a finite sum
$1\leq\nu\leq \rho(r\uuu *)$ above
we conclude that

\[
\lim\uuu{r\to 1}\, \frac{1}{2\pi}\int\, \int\uuu 0^{2\pi}\,\log\,|B(re^{i\theta})|\cdot d\theta
>\vvv \epsilon
\]
Since $\epsilon>0$ is arbitrary we have proved (ii) and hence also Theorem 3.4.





\bigskip


\centerline {\bf 3.5 Division  by Blaschke products.} 


\bigskip

\noindent
Let $F\in\mathcal O(D)$ and assume that  its zero set in
$D$ is a Blaschke sequence $\{\alpha_\nu\}$.
Then we obtain the analytic function
\[ 
G(z)=\frac{F(z)}{B(z)}
\]


\noindent
Here  $G$ has no zeros in $D$
and we can  construct the analytic function
$\text{Log}\,G(z)$.
Set
\[
\mathcal I^+_G(r)=\int_0^{2\pi}\,
\log^+|\,G(re^{i\theta})|\cdot d\theta
\]
Since $\log^+[ab|\leq \log^+|a|+\log^+|b|$ for 
every pair of complex numbers we get:

\[
\mathcal I^+_G(r)\leq
\mathcal I^+_F(r)+
\int_0^{2\pi}\,
\log^+\frac{1}{|\,B(re^{i\theta})|}\cdot d\theta\tag{1}
\]
The last nondecreasing function is
$\leq \log^+\frac{1}{|\,B(0)|}$ for every $r$ which gives 
\[
\mathcal I^+_G(r)\leq
\mathcal I^+_F(r)+\log^+\frac{1}{|\,B(0)|}\tag{2}
\] 
for every $r<1$.
When $F\in\text{JN}(D)$ this implies that
$G$ also belongs to $\text{JN}(D)$. Hence we have proved

\medskip
\noindent{\bf{3.6 Theorem.}}
\emph{For each $f\in\text{JN}(D)$
the function $\frac{f}{B\uuu f}$ also
belongs to $\text{JN}(D)$, where $B\uuu f(z)$ is the 
Blaschke product formed by zeros of $f$�outside the origin.}
\bigskip

\noindent{\bf{3.7 Conclusion.}}
Theorem 3.6 and the material in section 2 about zero\vvv free
Jensen\vvv Nevanlinna functions  give
the following factorisation  formula:

\medskip

\noindent{\bf{3.8 Theorem.}}
\emph{For each $f\in\text{JN}(D)$ there exists a unique
real Riesz measure $\mu$ on $T$ with $\mu(T)=0$
such that}
\[
f(z)=az^k\cdot B\uuu f(z)\cdot
e^{g\uuu\mu(z)}
\]
\emph{where $k\geq 0$ is the order of zero of $f$
at $z=0$ and $a\neq 0$ a constant. Moreover}
\[ 
\mu=\log\, |f(e^{i\theta})|\cdot d\theta+\sigma
\]
\emph{where $\sigma$ is the singular part of $\mu$.}

\medskip



\noindent
{\bf{3.9 Outer factors.}}
In Theorem 3.8 we get the analytic function
\[ 
\mathfrak{O}\uuu f(z)=e^{g\uuu{\log |f|}(z)}
\]
We refer to $\mathfrak O\uuu f$ as the outer part of $f$.
\medskip


\noindent
{\bf{3.10 A division result.}}
Consider a pair $f,h$ in $\text{JN}(D)$
which gives the analytic function in $D$ defined by

\[ 
k(z)= 
\frac{\mathfrak{O}\uuu h(z)}
{\mathfrak{ O}\uuu f(z)}
\]
By (2.3) there exists the almost everywhere defined quotient on $T$
\[ 
k^*(\theta)= \frac{\mathfrak{O}\uuu h^*(\theta)}
{\mathfrak{ O}\uuu f^*(\theta)}
\]

\medskip

\noindent
{\bf{3.11 Theorem.}}
\emph{Assume that $k^*\in L^1(T)$. Then 
$k^*$ belongs to the Hardy space $H^1(T)$.}
\medskip

\noindent
\emph{Proof.}
In $D$ there exists the harmonic  function
\[
k(z)=\log\,|\mathfrak{O}\uuu h(z)|\vvv\log\,\mathfrak |O\uuu f(z)|
\] 
The two harmonic functions in the right hasnd side
have by definition boundary functions in $L^1(T)$ and
Poisson's formula 
gives  for each point $z=re^{i\theta}$:

\[ 
\log |k(re^{i\theta})|=
\frac{1}{2\pi}\int\uuu 0^{2\pi}\, \frac{1\vvv r^2}{1+r^2\vvv
2r\cos(\phi\vvv \theta)}\cdot \log |k^*(\phi)|\cdot d\phi
\]
By the general mean\vvv value inequality
from (xx) the left hand side is majorized by:
\[ 
\leq \log
\bigl[\frac{1}{2\pi}\int\uuu 0^{2\pi}\, \frac{1\vvv r^2}{1+r^2\vvv
2r\cos(\phi\vvv \theta)}\cdot |k^*(\phi)|\cdot d\phi\bigr]
\]
Taking exponentials on both sides we get

\[
|k(re^{i\theta})|\leq 
\frac{1}{2\pi}\int\uuu 0^{2\pi}\, \frac{1\vvv r^2}{1+r^2\vvv
2r\cos(\phi\vvv \theta)}\cdot |k^*(\phi)|\cdot d\phi
\]
Now we integrate both sides with respect to $\theta$. Since
\[
\frac{1}{2\pi}\int\uuu 0^{2\pi}\, \frac{1\vvv r^2}{1+r^2\vvv
2r\cos(\phi\vvv \theta)}\cdot d\theta=1
\] 
for every $\phi$, it follows that
\[
\int\uuu 0^{2\pi}\, |k(re^{i\theta})|\cdot d\theta\leq
\int\uuu 0^{2\pi}\, |k^*(e^{i\phi})|\cdot d\phi
\]
This proves that the $L^1$\vvv norms of $\theta\to k(re^{i\theta})$
are bounded which means that $k$ belongs to $H^1(T)$.
Moreover, by the Brother's Riesz theorem
there exist radial limits almost everywhere
so we have also the equality
\[
\lim\uuu{r\to 1}\, k(re^{i\theta})= k^*(\theta)
\] 
almost everywhere. This proves that $k^*$ is the boundary value function of
the $H^1(T)$\vvv function $k$.
\medskip

\noindent
{\bf{3.12 Exercise.}}
Show by a similar technique that if we instead assume that
$k^*$ is square\vvv integrable, i.e. if $k^*\in L^2(T)$ then
$k(z)$ belongs to the Hardy space $H^2(T)$.
\bigskip

\noindent
{\bf{3.13 Inner functions.}}
If $\sigma$ is a non\vvv negative and singular measure on $T$
we get the bounded analytic function
\[
G\uuu{\vvv\sigma}(z)=
e^{\vvv g\uuu\sigma(z)}\tag{1}
\]
Keeping $\sigma$ fixed we denote this function with $f$.
Here 
\[
\lim\uuu{r\to 1}\, |f(re^{i\theta})|=1
\] 
holds almost everywhere.
So the boundary function $f^*(\theta)$ has absolute value almost
everywhere.
The class of analytic functions obtained via (1) is denoted by
$\mathfrak I\uuu *(D)$ and are called zero\vvv free inner functions.
In general a bounded analytic function $f$ in $D$
whose boundary values have absolute value almost everywhere
is called an inner function and this class is denoted by
$\mathfrak I(D)$.

\medskip

\noindent
{\bf{3.14 Exercise.}}. Use the factorisation in Theorem 3.8
to show that
every $f\in\mathfrak I(D)$ is a product

\[ 
f=B\uuu f\cdot f\uuu *
\] 
where $f\uuu *$ is a zero\vvv free inner function.
\medskip

\noindent
{\bf{3.15 The case of signed singular measures.}}
Let $\mu=\mu\uuu +\vvv \mu\uuu\vvv$ be a signed singular measure
where $\mu\uuu +\neq 0$.
We get the analytic function $G\uuu\mu$ and from the above we know that it
has radial limits almost everywhere and since $\mu$ is singular the boundary
function $G^*\uuu\mu$ has absolute value almost everywhere.
Here the presence of $\mu\uuu +$ implies that the analytic function
$G\uuu\mu$ is unbounded.
In fact, its maximum modules function 
\[
M(r)=\max\uuu{|z|=r|}\, |G\uuu\mu(z)|
\]
has a quite rapid growth as $r\to 1$.
Moreover one always has
\[
\lim\uuu{r\to 1}\, \int\uuu 0^{2\pi}\,
|G\uuu\mu(re^{i\theta})|\cdot d\theta=+\infty\tag{*}
\]
in other words, $G\uuu\mu$\vvv functions constructed by
signed measures with non\vvv zero negative part never belongs to
$H^1(T)$.
\medskip


\noindent
{\bf{3.16 Exercise.}} Prove (*) above using the divergence in (*) from 1.3.











\newpage

\centerline{\bf{4. Invariant subspaces of $H^2(T)$}}
\bigskip


\noindent
The Hilbert space $L^2(T)$ of square integrable functions on
$T$  contains the closed subspace $H^2(T)$
whose elements are boundary values of analytic functions in
$D$.
If f $f\in H^2(T)$
it is expanded as
\[ 
\sum\uuu{n=0}^\infty\, a\uuu n\cdot e^{in\theta}
\]
and Parseval's theorem gives  the equality

\[
\sum\uuu{n=0}^\infty\, |a\uuu n|^2=\frac{1}{2\pi}\cdot
\int\uuu 0^{2\pi}\, 
|f(e^{i\theta})|^2 d\theta
\]
Moreover, in $D$ we get the analytic function
$f(z)=\sum\, a\uuu nz^n$ where
radial limits
\[ 
\lim\uuu {r\to 1}\, f(re^{i\theta})= f(e^{i\theta})
\]
exist almost everywhere.in fact, this follows via
the Brothers Riesz theorem and the inclusion 
$H^2(T)\subset  H^1(T)$.
We shall  study subspaces of $H^2(T)$
which are invariant under multiplication by
$e^{i\theta}$. 

\medskip

\noindent
{\bf{4.2 Definition.}}
\emph{A closed subspace $V$ of $H^2(T)$ is called invariant if
$e^{i\theta}V\subset V$.}
\bigskip




\noindent
{\bf{4.3 Theorem}}
\emph{Let $V$ be an invariant subspace of $H^2(T)$.
Then there exists
$w(\theta)\in H^2(T)$ 
whose absolute value is one almost everywhere
and}
\[ 
V=H^2(T)\cdot w
\]
\medskip


\noindent
\emph{Proof.}
First we show that that $e^{i\theta}V$ is a proper subspace of $V$.
For an equality $e^{i\theta}V=V$  gives
$e^{in\theta}V=V$ for every $n\geq 1$ which entails that if 
$0\neq f\in V$ then 
$f=e^{in\theta}\cdot g\uuu n$ for some $g\uuu n\in H^2(T)$. This  means
that the Taylor series of $f$ at $z=0$ starts with order $\geq n$
which  cannot
hold
for every $n$ unless $f$ is identically zero.
So now $e^{i\theta}V$ is a proper closed subspace of  $V$
which gives some $0\neq w\in V$
which is $\perp$ to $e^{i\theta}V$. It follows that

\[
\langle w,e^{in\theta} \cdot w\rangle
\int\uuu 0^{2\pi}\, w(e^{i\theta})\bar w(e^{i\theta})\cdot e^{\vvv in\theta}
\cdot d\theta=0
\]
hold for every $n\geq 1$.
Since $w\cdot \bar w=|w|^2$ is real\vvv valued we conclude that
this function is constant and we can normalize $w$ so that
$|w(\theta)|=1$ holds almost everywhere.
There remains to prove the equality
\[
V=H^2(T)\cdot w\tag{i}
\]
Since $|w|=1$ almost everywhere the right hand side is a closed subspace of
$V$. If it is proper we find
$0\neq u\in V$ where $u\perp H^2(T)w$ which gives

\[
\int\uuu 0^{2\pi}\, u(e^{i\theta})\bar w(e^{i\theta})\cdot e^{\vvv in\theta}
\cdot d\theta=0\quad\colon\quad n\geq 0\tag{ii}
\]
Taking complex conjugates we get

\[
\int\uuu 0^{2\pi}\, w(e^{i\theta})\bar u(e^{i\theta})\cdot e^{in\theta}
\cdot d\theta=0\quad\colon\quad n\geq 0\tag{iii}
\]
At the same time
$w\perp e^{i\theta}V$ which entails that
\[
\int\uuu 0^{2\pi}\, w(e^{i\theta})\bar u(e^{i\theta})\cdot e^{\vvv in\theta}
\cdot d\theta=0\quad\colon\quad n\geq 1\tag{iv}
\]
Together (iiii\vvv iv) imply that
$w\bar u$ has vanishing Fourier coefficients and is therefore identically zero which gives
$u=0$ and proves that $V=H^2(T)\cdot w$ must hold.
\bigskip

\noindent
{\bf{4.4 Examples.}}
Let $B(z)$ be a non\vvv constant Blaschke product.
Now $|B(e^{i\theta})|=1$  holds almost everywhere and
the presence of zeros of $B(z)$ in $D$ show that
$H^2(T)\cdot B$ is a proper 
and invariant subspace of $H^2(T)$.
Next, let
$\sigma$ be a singular Riesz measure on $T$
which is real and non\vvv negative.
We get the analytic function
\[ 
f(z)= e^{\vvv g\uuu\sigma(z)}
\]
Here
\[ 
|f(z)|= e^{\vvv H\uuu\sigma(z)}
\] 
and since $\sigma\geq 0$ we have $H\uuu \sigma(z)\geq 0$ and hence
$|f(z)|\leq 1$.
So $f$ is a bounded analytic function in $D$ and
in particular it belongs to $H^2(T)$. Moreover
we know from XX that the boundary function $f(e^{i\theta})$ has absolute
value one almost everywhere.
So $H^2(T)\cdot f$ is an invariant subspace of $H^2(T)$ and the
question arises if it is proper or not.
In contrast to the case for Blaschke functions $B$ above this is not obvious since
$f$ has no zeros in $D$.
However it turns out that one has

\medskip

\noindent
{\bf{4.5 Theorem.}} \emph{Let $\sigma$ be a 
singular and non\vvv negative 
Riesz measure which is not
identically zero. Then $H^2(T)\cdot e^{\vvv g\uuu\mu}$
is a proper subspace of $H^2(T)$.}
\medskip

\noindent
\emph{Proof.} 
Set $w(\theta)= e^{\vvv g\uuu\mu(e^{i\theta})}$. For the analytic function
$w(z)$ in the disc  its value at $z=0$ becomes
\[ 
w(0)=e^{\vvv g\uuu\mu(0}= e^{\vvv \sigma(T)/2\pi}
\]
Next, 
if $P(z)$ is a polynomial
we have
\[
\frac{1}{2\pi}\int\uuu 0^{2\pi}\, |P(\theta)w(\theta) \vvv 1|^2d\theta
=\frac{1}{2\pi}\int\uuu 0^{2\pi}\, |P(\theta)|^2\cdot d\theta
+1+2\mathfrak{Re}\,[\int
\frac{1}{2\pi}\int\uuu 0^{2\pi}\, P(\theta)\cdot w(\theta)\cdot d\theta]
\]
By Cauchy's formula the last term becomes
\[
2\mathfrak{Re}(P(0)w(0))= 2w(0)\cdot \mathfrak{Re}(P(0))
\]
By (i) we have $0<w(0)<1$ and if $||P||\uuu 2$ is the $L^2$\vvv norm of $P$
the right hand side majorizes
\[
||P||\uuu 2^2+1\vvv 2w(0)\cdot |P(0)|
\]
We have also the inequality
\[
|P(0)|\leq ||P||\uuu 2
\]
So if we set $\rho=||P||\uuu 2$
then we have shown that

\[
\frac{1}{2\pi}\int\uuu 0^{2\pi}\, |P(\theta)w(\theta) \vvv 1|^2d\theta
\geq \rho^2+1\vvv 2w(0)\cdot \rho
\]
Now we notice that the right hand side is $\geq 1\vvv w(0)^2$
for every $\rho$.
Since $P$ is an arbitrary polynomial we conclude that
the $L^2$\vvv distance of 1 to the subspace
$H^2(T)\cdot e^{\vvv g\uuu\mu}$
is at least
\[
1\vvv w(0)^2= 1\vvv e^{\vvv 2\sigma(T)}\tag{*}
\]








\bigskip

\centerline{\bf{5. Beurling's closure theorem.}}
\medskip

\noindent
A zero\vvv free  function $f\in H^2(T)$ is of outer type when
\[ 
f(z)=G\uuu\mu(z)
\] 
where $\mu$ is the absolutely continuous Riesz measure
$\log|f(e^{i\theta}|$.
The following result is due to Beurling in   [Beur]:
\medskip

\noindent
{\bf{5.1 Theorem.}}
\emph{For every nonzero $f\in H^2(T)$ of outer type 
the closed invariant subspace generated by analytic polynomials $P(z)$ times
$f$ is equal to $H^2(T)$.}
\medskip

\noindent
\emph{Proof.}
If the density fails we find
$0\neq g\in H^2(T)$ such that
\[
\int\uuu 0^{2\pi}\, e^{in\theta} f(e^{i\theta})\cdot \bar g(e^{i\theta}) \cdot d\theta=0\quad\text{for every}\quad  n\geq 0
\tag{i}
\] 
By Cauchy\vvv Schwarz the product $f\cdot \bar g$
belongs to $L^1(T)$ and (i) implies that this function
is of the form $e^{i\theta}\cdot h(\theta)$ where
$h\in H^1(T)$.
So on $T$ we have almost everywhere:
\[ 
\bar g(e^{i\theta})=e^{i\theta}\cdot \frac{h(e^{i\theta})}{f(e^{i\theta})}\tag{ii}
\] 
Now we take the outer factor $\mathfrak{O}\uuu h$
whose absolute value is equal to $|k|$ almost everywhere on $T$. It follows that
\[
|g^*(\theta)|= 
 \frac{\mathfrak{O}\uuu h^*(\theta)}
{\mathfrak{ O}\uuu f^*(\theta)}\tag{iii}
\]
Since $g\in H^2(T)$ 
Exercise 3.12 shows that 
the quotient in (ii) is
the boundary value of an analytic  function in $H^2(T)$ which implies that
the conjugate function $\bar g$ also belongs to $H^2(T)$. But then
$g$ must be a constant and this constant is zero
because the factor $e^{i\theta}$ appears in (ii).
So $g$ must be zero which gives a contradiction and
the requested density is proved.

\bigskip

\centerline{\bf{5.2 Szeg�'s theorem.}}
\bigskip

\noindent
Let $w(\theta)$ be  real\vvv valued and non\vvv negative
function in $L^1(T)$ and denote by $\mathcal P\uuu 0$
the space of analytic polynomials $P(z)$ where $P(0)=0$.
Put
\[ 
\rho(w)=\frac{1}{2\pi}\inf\uuu{P\in\mathcal P\uuu 0}\int\uuu 0^{2\pi}\,
\bigl| 1\vvv P(e^{i\theta})\bigr|\cdot w(\theta)\cdot d\theta
\]
\medskip

\noindent
{\bf{5.3 Theorem.}}
\emph{One has the equality}
\[ 
\rho(w)=\text{exp}\, [\frac{1}{2\pi}\int\uuu 0^{2\pi}\,  \log w(\theta)\cdot d\theta]
\]
\medskip

\noindent
\emph{Proof.} First we consider the case
when
$\log |w|\in L^1(T)$.
Multiplying $w$ with a positive constant we may assume that
\[
\int\uuu 0^{2\pi}\,\log w(\theta)\cdot d\theta=0\tag{i}
\]
Now we must show that
$\rho(w)=1$.
To prove this we use that $\log w\in L^1(T)$ and construct the analytic function
\[
f(z)=G\uuu {\log w(z)}
\]
So $f$ is an outer function where
on $T$ one has
\[ 
|f(e^{i\theta})|=e^{\log|w(\theta)|}=
w(\theta)\tag{ii}
\]
Hence $f\in H^1(T)$
and (1) gives  $f(0)=1$.
Let us now consider some $P(z)\in\mathcal P\uuu 0$
and set
\[
F(z)=(1\vvv P(z))f(z)
\]
Again $F(0)=1$ and $F\in H^1(T)$ which gives the inequality
\[
 1\leq\int\uuu 0^{2\pi}\, |F(e^{i\theta})|\cdot d\theta\tag{iii}
\]
By (ii) this means that
\[ 
1\leq \int\uuu 0^{2\pi}\, |1\vvv  P(e^{i\theta})|\cdot 
w(\theta) \cdot d\theta
\] Since this hold for every $P\in\mathcal P\uuu 0$ we have proved the inequality
\[
\rho(w)\geq 1\tag{iv}
\]
To prove the reverse inequality we
apply Beurling's theorem to the outer function $f$. This gives
a sequence of polynomials $\{Q\uuu n(z)\}$ such that
\[
\lim\uuu{n\to \infty}\, ||Q\uuu n\cdot f\vvv 1||\uuu 1=0\tag{v}
\] 
where we use the norm on $H^1(T)$.
Since $f(0)=1$ it follows that
$Q\uuu n(0)\to 1$ and we can normalize the approximating
sequence so that $Q\uuu n(0)=1$ for every $n$
and write $Q\uuu n=1\vvv P\uuu n$ with $P\uuu n\in\mathcal P\uuu 0$.
Finally using (ii) we get
\[
\lim\uuu{n\to\infty}\int\uuu 0^{2\pi}\, |1\vvv P(e^{i\theta})|\cdot w(\theta)\cdot d\theta=1
\]
This gives $\rho(w)\geq 1$ and Szeg�'s theorem is proved for the case A above.
\medskip

\noindent
\emph{B. The case when 
$\log^+\frac{1}{|w|}$
is not integrable}.
Here we must show 
that
$\rho(w)=0$ and the  proof of this is left as an exercise to the reader.




\bigskip

\centerline{\bf{6. The Helson\vvv Szeg� theorem}}

\bigskip

\noindent
A trigonometric polynomial
on the unit circle is of the form
\[
P(\theta)=\sum\, a\uuu n\cdot e^{in\theta}
\]
where $\{a\uuu n\}$  are complex numbers and only
a finite family is $\neq 0$.
The conjugation operator $\mathcal C$ is defined by

\[
\mathcal C(P)=i\cdot \sum\uuu{n<0}\, a\uuu n\cdot e^{in\theta}\vvv
i\cdot \sum\uuu{n>0}\, a\uuu n\cdot e^{in\theta}\tag{*}
\]
Let $w(\theta)$ be a non\vvv negative function in
$L^1(T)$ and assume also that
$\bigl|\log\,|w|\,\bigr|\in L^1(T)$.
\medskip

\noindent
{\bf{6.1 Definition.}}
\emph{A $w$\vvv function 
as above is of Helson\vvv Szeg� type if there exists a constant $C$ such that}
\[
\int\uuu 0^{2\pi}\, |\mathcal C(P)(e^{i\theta})|^2 \cdot w(\theta)\cdot d\theta
\leq C\cdot \int\uuu 0^{2\pi}\, |P(e^{i\theta})|^2\cdot w(\theta)\cdot d\theta
\tag{*}
\] 
\emph{hold for all trigonometric polynomials}.
\medskip

\noindent
Notice that if (*) holds for some $w$ then it holds for every function of the form 
$\rho\cdot w$ where
$0<c\uuu 0\leq \rho(\theta)\leq c\uuu 1$ for some pair of
positive constants. Or equivalently, with $w$ replaced by
$e^u\cdot w$ for some bounded function $u(\theta)$.
With this in mind we announce the result below which is due to Helson and Szeg� in [HS]:

\medskip

\noindent
{\bf{6.2 Theorem.}}
\emph{A function $w(\theta)$ is of the Helson\vvv Szeg� type if and only if
there exists a bounded function $u$ and 
a function $v(\theta)$ for which the maximum norm of $|v|$ over $T$ is $<1$
and}
\[ 
w(\theta)= e^{u(\theta)+ v^*(\theta)}
\] 
\emph{where $v^*$ is the harmonic conjugate of $v$.}
\bigskip

\noindent
The proof requires several steps. The first part is an exercise on
norms on the Hilbert space $L^2(w)$ which is left to the reader.
\medskip

\noindent
{\bf{Exercise.}}
Show that $w$ is of the Helson\vvv Szeg� type if and only if there exists a constant
$\rho<1$ such that

\[
\bigl |\,\int\uuu 0^{2\pi}\,
P(\theta)\cdot e^{\vvv i\theta}\cdot Q(\theta)\cdot w(\theta)\cdot d\theta\,\bigr|
\leq \rho\cdot ||P||\uuu w\cdot ||Q||\uuu w\tag{*}
\]
hold for all pairs $P,Q$ in $\mathcal P\uuu 0$.
\medskip

\noindent
{\bf{6.3 The outer function $\phi$}}.
We define the analytic function $\phi(z)$ by

\[
\phi(z)=\text{exp}\bigl[\, \frac{1}{2\pi}\int\uuu 0^{2\pi}\,
\frac{e^{i\theta}+z}{e^{i\theta}\vvv z}\cdot 
\log 
(\sqrt{w(\theta)})\cdot d\theta\,\bigr]
\]
Since
$\log \,
\sqrt{w(\theta)}=
\frac{1}{2}\cdot \log\, 
w(\theta)$ is in $L^1(T)$ it means that
$\phi(z)$ is an outer function and
on the unit circle we have  the equality
\[ 
|\phi(\theta)|^2=w(\theta)\tag{1}
\]
Using (1) we find a 
real\vvv valued function $\gamma(\theta)$ such that
\[
w(\theta)= \phi^2(\theta)\cdot e^{i\gamma(\theta)}\tag{2}
\]


\noindent
Next,  (1) implies  that the weighted $L^2$\vvv norm $||P||\uuu w$
is equal to the standard $L^2$\vvv norm of $\phi\cdot P$ on
$T$. Hence (1) holds if and only if
\[
\bigl |\,\int\uuu 0^{2\pi}\,
\phi(\theta)P(\theta)\cdot e^{\vvv i\theta}\cdot 
\phi(\theta)Q(\theta)\cdot e^{i\gamma(\theta)}\cdot d\theta\,\bigr|
\leq \rho\cdot ||\phi\cdot P||\uuu 2 \cdot ||\phi\cdot Q||\uuu 2\tag{3}
\]
hold for all pairs $P,Q$ in $\mathcal P\uuu 0$.
Now we use that $\phi$ is outer which by Beurling's closure
theorem means that $\mathcal P\uuu 0\cdot \phi$ is dense in
$H\uuu 0^2(T)$.
Hence (3) is equivalent to 

\[
\bigl |\,\int\uuu 0^{2\pi}\,
F(\theta)\cdot e^{\vvv i\theta}\cdot G(\theta)
\cdot  e^{i\gamma(\theta)}
\cdot d\theta\,\bigr|
\leq \rho\cdot ||F||\uuu 2 \cdot ||G||\uuu 2\tag{4}
\]
for all pairs $F,G$ in $H^2\uuu 0(T)$.
\medskip

\noindent
Next, in XX we prove that every $f\in H^1\uuu 0(T)$
admits a factorization $f=F\cdot G\cdot e^{\vvv i\theta}$
for a pair $F,G$  where $||f||\uuu 1=||F||\uuu 2\cdot ||G||\uuu 2$.
So (4) is equivalent to 
\[
\bigl |\,\int\uuu 0^{2\pi}\,
f(\theta)\cdot e^{i\gamma(\theta)}
\cdot 
d\theta\,\bigr|
\leq \rho\cdot ||f||\uuu 1\tag{5}
\] 
for each $f\in H^1\uuu 0(T)$.
At this stage we use the duality between
$H^\infty(T)$ and $H^1\uuu 0(T)$ from 
Section XX. It follows that
(5) is equivalent to the following

\medskip

\noindent
{\bf{6.4 Approximation condition.}}
One has
\[ 
\min\uuu h\,||e^{i\gamma(\theta)}\vvv h(\theta)||\uuu\infty=\rho
\]
where the minimum is taken over $h$\vvv functions in $H^\infty(T)$.
\bigskip

\centerline{\emph {Final part of the proof.}}
\medskip


\noindent
Since $w\geq 0$ and $>0$ outside a set of measure zero, 
the approximation condition is equivalent with the existence of some
$h\in H^\infty(T)$ and some $\rho<1$ such that


\[ 
|w(\theta)\vvv \phi^2(\theta)\cdot h(\theta)|\leq \rho\cdot w(\theta)\tag{*}
\]
hold on $T$.
It remains to show that (*) is equivalent to the 
existence of a pair $u,v$ in Theorem 6.2.
Let us begin with

\medskip

\noindent
\emph{Proof that (*) gives the pair $u,v$}.
Since $\log w$ is in $L^1(T)$ we have $w>0$ almost everywhere
and (*) entails that
$\phi^2(\theta)\cdot h(\theta)$
stay in the sector
\[
Z=\{z\colon \vvv\pi/2+ \delta\leq \text{arg}(z)\leq
\pi/2\vvv \delta\}
\]
where we have put $\delta=\arccos(\rho)$.
This inclusion of the range of 
$\phi^2\cdot h$
implies that it is outer. See XX above.
Hence we can find a harmonic function
$V$ such that
\[
\phi^2\cdot h= e^{ia}\cdot e^{V+iV^*}
\]
where $a$ is some real constant.
The inclusion of the range implies that
\[
|a+V^*(\theta)|\leq \pi/2\vvv\delta
\]
Next, define the harmonic function
\[ 
v(\theta)= \vvv (a+ V^*(\theta))
\]
It follows that
\[
\phi^2(\theta)\cdot h(\theta)= e^{v(\theta)+iv^*(\theta)+c}
\]
for some constant $c$.
Finally, since $w=|\phi|^2$ we obtain
\[
w(\theta)= e^{v(\theta)}\cdot \frac{e^a}{|h(\theta)|}
\]
By (xx) above the last factor is  bounded both below and above and hence $e^u$ for some bounded function.
Together with the bound (xx) for the harmonic 
conjugate of $v$
we get the requested form for $w(\theta)$ in Theorem 6.2.
\medskip

\noindent
\emph{Proof that a pair $(u,v)$ gives (*)}.
Consider the special case when $w= e^v$ and 
\[
|v^*(\theta)|\leq 
\pi/2\vvv \epsilon
\] 
holds for 
some $\epsilon>0$.
It is clear that the corresponding $\phi$function obtained via (xx) above
satisfies


\[
\phi^2(\theta)= e^{v(\theta)+iv^*(\theta)}
\] 
This gives
\[
e^{i\gamma(\theta)}= e^{\vvv iv^*(\theta)}
\]
and we notice that if we take the constant function $h(\theta)=
\epsilon$ then the maximum norm
\[
||e^{i\gamma(\theta)}\vvv \epsilon||\uuu \infty<1
\]
which proves that (*) holds.

\newpage

%\end{document}

 







