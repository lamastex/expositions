%\documentclass{amsart}


%\usepackage[applemac]{inputenc}

%\addtolength{\hoffset}{-12mm}
%\addtolength{\textwidth}{22mm}
%\addtolength{\voffset}{-10mm}
%\addtolength{\textheight}{20mm}

%\def\uuu{_}


%\def\vvv{-}

%\begin{document}




\centerline{\bf \large{XVI.. Beurling-Wiener  algebras}}

\bigskip

\noindent
\centerline{\emph{Contents}}
\bigskip

\noindent


\medskip

\noindent
\emph{A: Beurling-Wiener algebras on the real line.}

\medskip

\noindent
\emph{B: A Tauberian theorem}


\medskip

\noindent
\emph{C: Ikehara's theorem}

\medskip
\noindent
\emph{D: The Gelfand space of $L^1({\bf{R}}^+)$.}

\bigskip

\centerline {\bf{Introduction.}}

\bigskip

\noindent
The cornerstone in this section is Wiener's general Tauberian Theorem
which we are going to apply
to the   class of 
Beurling\vvv Wiener algebras where the ordinary convolution algebra
$L^1({\bf{R}}) $ is replaced by various weight algebras
which were introduced by Beurling in the article [Beurling: 1938].
The subsequent material relies upon [ibid] and  on
chapter XX in [Paley\vvv Wiener].
Here follows the set\vvv up in this section.
Consider
the Banach space $L^1({\bf{R}})$
of 
Lebesgue measurable and absolutely integrable functions whose
product is defined by  convolutions:
\[ 
f*g(x)=\int\, f(x-y)g(y)dy
\]

\medskip




\noindent {\bf A.1 The space $\mathcal F_0^\infty$.}
On the $\xi$-line we have the space $C_0^\infty$ 
of infintely differentiable functions with compact support.
Each $g(\xi)\in C_0^\infty$ yields an $L^1$-function on the real 
$x$-line defined by
\[ 
\mathcal F(g)(x)=\frac{1}{2\pi}\int\, e^{ix\xi}\, g(\xi)\cdot d\xi\tag{*}
\]
The resulting subspace of $L^1$ is denoted by
$\mathcal F_0^\infty$.

\medskip


\noindent {\bf A.2  Beurling-Wiener algebras.}
A subalgebra B of $L^1$
is called a Beurling-Wiener algebra - for short a 
$\mathcal B\mathcal W$-algebra - if the
following two conditions hold:
\medskip

\noindent
\emph{Condition 1.}
$B$ is
equipped with a complete norm
denoted by $||\cdot  ||_B$ such that 
\[
||f*g||_B\leq ||f||_B\cdot ||g||_B\quad\colon\quad f,g\in B\quad\text{and}\quad
||f||_1\leq ||f||_B
\]
\medskip

\noindent
\emph{ Condition 2.}
$\mathcal F_0^\infty$ is  a dense subalgebra of
$B$. 
\bigskip


\noindent {\bf A.3 Theorem} \emph{Let $B$ be a $\mathcal B\mathcal W$-algebra.
For each multiplicative and continuous functional $\lambda$ on $B$ 
which is not identically zero
there exists a unique $\xi\in\bf R$ such that}
\[
 \lambda(f)=\widehat f(\xi)\quad\colon\quad f\in B
\]

\medskip

\noindent 
\emph{Proof.}
Suppose that there exists some $\xi$ such that
\[
\lambda(f)=0
 \implies  \widehat f(\xi)=0\tag{i}
\]
This means that the linear
form $f\mapsto \widehat f(\xi)$
has the same kernel as $\lambda$ and hence there exists some
constant $c$ such that
\[
\lambda(f)=c\cdot \hat f(\xi)\quad\text{for all}\,\, f\in B\,.\tag{ii}
\]
Since  $\lambda$  is multiplicative it follows that
$c=c^n$ for every positive integer $n$ and then
$c=1$.
Next, since
$B$ contains $\mathcal F_0^\infty$ and test-functions on
the $\xi$-line separate points, it is clear that
$\xi$  is uniquely determined.
There remains to prove the existence
of some $\xi$ for which (i) holds.
\medskip

\noindent
To prove this we use the
density of $\mathcal F_0^\infty$ in $B$ which by the continuity of
$\lambda$ gives
some
$g\in\mathcal F_0^\infty$ such  that $\lambda(g)\neq 0$.
Let $K$ be the compact support of
the test-function $\widehat g(\xi)$ and
suppose that  
(i)  fails for each point  $\xi\in K$.
The density of
$\mathcal F_0^\infty$ gives
some $f_\xi \in\mathcal F_0^\infty$ such that
\[ 
 \widehat f(\xi)\neq 0\,\quad \text{and}\,\, \lambda(f)=0\tag{iii}
\]

\noindent
Heine-Borel's Lemma yields
a finite set of points $\xi_1,\ldots,\xi_N$ in $K$ such
that
family $\{ \widehat f_{\xi_k}\}$ have no common zero on $K$.
To simplify notations we set
$f_k=f_{\xi_k}$.
The complex conjugates 
of $\{\widehat f_k\}$ are again  test-functions. So for each $k$
we find
$h_k\in B$ such that $\widehat h_k$ is the s complex conjugate
of  $\widehat f_k$.
Set
\[ 
\phi (\xi)=
\sum_{k=1}^{k=N}\,\widehat h_k(\xi)\cdot \widehat f_k(\xi)
\]


\noindent
This test-function is $>0$ on the support of $\widehat g$ and hence there exists
the test-function
\[
Q(\xi)=\frac{\widehat g}{\phi}\tag{iv}
\]
By Condition 2, Q is  the Fourier transform of some $B$-element
$q$. Since $L^1({\bf{R}})$-functions are uniquely determined by their
Fourier transforms, it follows from (iv) that 
\[
\sum_{k=1}^{k=N}\, q*h_k*f_k=g\tag{v}
\]

\noindent
Now we get a contradiction since $\lambda(f_k)=0$ for each  $k$
while $\lambda(g)\neq 0$. 








\bigskip



\centerline {\bf  A.4 The algebra $B_a$.}
\medskip


\noindent 
Let $a>0$ be a positive real number.
Given a Beurling-Wiener algebra $ B$
we set
\[ 
J_a=\{ f\in B\quad\colon\, \widehat f(\xi)=0\,\,\text{for all}\,\,
-a\leq \xi\leq a\}
\]


\noindent
Condition 1 and the continuity of the Fourier transform on
$L^1$-functions imply
that $J_a$ is a closed ideal in $ B$. 
Hence we get the Banach algebra $\frac{B}{J_a}$ which we denote by
$B_a$. Let   $g\in \mathcal F_0^\infty$ be such that
$\widehat g(\xi)=1$ on $[-a,a]$. For every
$f\in B$ it  follows that
$g*f-f$ belongs to $J_a$ which  means that
the image of $f$ in $B_a$ is equal to the image of $g*f$. We conclude that
the $g$-image yields an identity in the algebra $B_a$ and  hence $B_a$
is a Banach algebra with a unit element.
\medskip

\noindent
{\bf A.5 Theorem.} \emph{The Gelfand space of
$B_a$ is equal to the compact interval $[-a,a]$.}
\medskip

\noindent
{\bf{A.6 Exercise.}} Prove this using Theorem A.3


\bigskip

\centerline{\bf {A.7. Examples of $\mathcal BW$-algebras}}
\medskip

\noindent
Let $B$ be the space of all continuous functions $f(x)$ 
on the real $x$-line such that the positive series below is convergent:
\[
\sum_{-\infty}^{\infty}\,||f||_{[\nu,\nu+1]}\tag{*}
\]


\noindent 
where
$||f||_{[\nu,\nu+1]}$ is the maximum norm of $f$ on the closed interval 
$[\nu.\nu+1]$ and the sum extends over all integers. The norm on $B$-elements
is defined by the sum of the series above.
It is obvious that this norm dominates the $L^1$-norm.
Moreover, one easily verifies that
\[
||f*g||_B\leq ||f|||\cdot ||g||_B\tag{i}
\]
for pairs in $B$.
Hence $B$ satisfies Condition 1 from B.
\medskip

\noindent
{\bf{Exercise.}}
Show that 
the Schwartz space $\mathcal S$
of rapidly decreasing functions
on the real $x$-line
is a dense subalgebra
of $B$.

\medskip

\noindent
Next, since  $\mathcal F_0^\infty\subset \mathcal S$ we have the inclusion
\[ 
\mathcal F_0^\infty\subset B\tag{ii}
\]
There remains to see why
$\mathcal F_0^\infty$ is dense in
$B$. To prove this we construct some special functions
on the $x$-line whose Fourier transforms have compact support.
If $b>0$
we set
\[ 
f_b(x)=\frac{1}{2\pi}
\int_{-b}^b\,e^{ix\xi}\cdot (1-\frac{|\xi|}{b})\cdot d\xi
\]
By Fourier's inversion formula this means that
\[ 
\widehat f_b(\xi)=1-\frac{|\xi|}{b}\quad\,-b\leq\xi\leq b\quad
\text{and zero if}\,\,|\xi|>b
\]
A computation which is left to the reader gives
\[
f_b(x)=\frac{1}{\pi}\cdot \frac{1-\text{cos}\, bx}{bx^2}
\]
From this expression it is clear that
$f_b(x)$ belongs to $B$ and we leave it to the reader to verify that
\[ 
\lim_{b\to+\infty}\, ||f_b*g-g||_B=0
\quad\text{for all}\,\, g\in B\tag{iii}
\]
\medskip

\noindent
Next,  the functions
$\widehat f_b(\xi)$ have compact support but they are not smooth, i.e.
they do not belong to $\mathcal F_0^\infty$.
However, we can perform a smoothing of these functions as follows:
Let $\phi(\xi)$  be an even and non-negative
$C_0^\infty$-function with support in
$-1\leq\xi\leq 1$  such that the integral
\[
\int\, \phi(\xi)\cdot d\xi=1
\]


\noindent
With $\delta>0$ we set
$\phi_\delta(\xi)= \frac{1}{\delta}\cdot \phi(\xi/\delta)$ and 
for each
pair $\delta,b$ we get the test-function on the $\xi$-line
defined by
\[
\psi_{\delta,b}(\xi)=
\int_{-b}^b\, \phi_\delta(\xi-\eta)\cdot (1-\frac{|\eta|}{b})\cdot d\eta
\]
The inverse Fourier transforms
\[ 
f_{\delta,b}(x)=\frac{1}{2\pi}\int
e^{ix\xi}\cdot \psi_{\delta,b}(\xi)\cdot d\xi
\]
yield functions in $\mathcal F_0^\infty$ for all pairs $\delta,b$.
Next, if $g\in B$ then 
the Fourier transform of the $B$-element $f_{\delta,b}*g$ is equal to the
\emph{convolution} of
$\phi_\delta(\xi)$ and the Fourier transform of
$f_b*g$.
This implies that
\[ 
f_{\delta,b}* g\in \mathcal F_0^\infty\,.
\]
At this stage we leave it to the reader to verify that
\[
\lim_{(\delta,b)\to (0,0)}\, f_{\delta,b}*g=g
\]
holds for every $g\in B$. Hence the required density of
$\mathcal F_0^\infty$ is proved and $B$ is a Beurling-Winer algebra.



\centerline{\bf A.8 Adding discrete measures}
\medskip

\noindent
Let $M_d({\bf{R}})$ be the Banach algebra of discrete measures
of finite total variation, i.e. measures of the form
\[ 
\mu=\sum\, c_\nu\cdot\delta_{x_\nu}\quad\colon\,
||\mu||=\sum\,|c_\nu|<\infty
\]
\medskip

\noindent
As explained in XX the Gelfand 
space is the compact
Bohr group $\mathfrak{B}$, where the real $\xi$-line via the Fourier
transform
appears as a dense subset.
Now we  adjoin some $\mathcal{BW}$\vvv algebra
$B$ and 
obtain a Banach algebra $B\uuu d$ which 
consists of measures of the form
\[ 
f+\mu\quad\colon\quad f\in B \,\,\text{and}\,\, \mu\in
M_d({\bf{R}})
\]
where the norm of $f+\mu$ is the sum of the $B$-norm of $f$
and the total variation of  $\mu$. Since $B$ is a subspace of
$L^1$ one easuly checks that this yields a complete norm.
next, by condition (2) in A.2 it follows that if
$f\in b$ and
$\mu\in M_d({\bf{R}})$ then the convolution
$f*\mu$ belongs to $B$.
This means that
$B$ appears as a closed ideal in $B\uuu d$.
\medskip

\noindent
{\bf A.9 The Gelfand space
$\mathcal M_{B\uuu d}$}.
Let  $\lambda$ is a multiplicative functional on
$B\uuu d$ which is not identically zero on
$B$. Theorem A.3 gives
a unique
$\xi$ such that
\[
\lambda(f)= \widehat f(\xi)\quad\,\colon\quad 
f\in B\tag{i}
\]
If $a$ is a real number then $\delta_a*f$ has the
Fourier transform
becomes $e^{ia\xi}\cdot \widehat f(\xi)$. It follows that
\[
\lambda(\delta_a)\cdot \widehat f(\xi)=
\lambda(\delta_a*f)= e^{-ia\xi}\cdot \widehat f(\xi)\tag{ii}
\]


\noindent
We conclude that $\lambda(\delta_a)=e^{-ia\xi}$ and hence  the restriction of
$\lambda$ 
is the evaluation of the Fourier transform  at $\xi$ on the whole algebra $B\uuu d$.
In this way the real $\xi$-line is embedded in
$\mathcal M_B$ where a point $\lambda\in\mathcal M_B$
belongs to this subset if and only if
$\lambda(f)\neq 0$ for some
$f\in B$.
The construction of the Gelfand 
topology shows  that
this copy of the real $\xi$-line appears as an \emph{open} subset of
$\mathcal M_{B\uuu d}$  denoted by
${\bf{R}}_\xi$.
\medskip

\noindent
{\bf A.10 The set $\mathcal M_{B\uuu d}\setminus {\bf{R}}_\xi$}.
If $\lambda$ belongs to this closed subset it is identically zero on
the ideal $B$ and its restriction to 
$M_d({\bf{R}})$ 
corresponds to a point $\gamma$ 
in the Bohr group
$\mathfrak{B}$. 
Conversely, every point in  $\mathfrak{B}$
yields a $\lambda\in
\mathcal M_{B\uuu d}\setminus {\bf{R}}_\xi$ 
since the quotient algebra
\[
\frac{B\uuu d}{B}\simeq
 M_d({\bf{R}})
\]
Hence we have the set\vvv theoretic equality
\[
\mathcal M\uuu{B\uuu d}= {\bf{R}}\uuu\xi
\cup\,\mathfrak{B}\tag{*}
\]




\medskip


\noindent
{\bf A.11  Proposition.} \emph{The open subset
${\bf{R}}_\xi$ is dense in $\mathcal M_B$.}
\medskip

\noindent\emph{Proof.}
Let $\lambda$ be a point in 
$\mathcal M_{B\uuu d}\setminus {\bf{R}}_\xi$ which therefore
corresponds to a point $\gamma\in\mathfrak{B}$.
By the result in XX we know that
for every finite set $\mu_1,\ldots,\mu_N$
of discrete measures, there exists a sequence $\{\xi_\nu\}$
such that
\[
\lim_{\nu\to\infty}
\widehat \mu_i(\xi_\nu)=\gamma(\mu_i)\quad\text{and}\,\,
|\xi_\nu|\to \infty
\]

\medskip

\noindent
At the same time
the Riemann-Lebesgue Lemma entails that

\[
\lim_{\nu\to\infty}\, \widehat f(\xi_\nu)=0
\]
for every
$f\in B$. Hence
the construction of the Gelfand topology on
$\mathcal M_{B\uuu d}$ gives the requested density
in Proposition A.11



\medskip

\noindent
{\bf A.12  An inversion formula.}
Let $f\in B $  and $\mu$ is some discrete measure.
Suppose that there exists $\delta>0$ such that
the Fourier transform of $f+\mu$ has absolute value 
$\geq\delta$ for all $\xi$.
Proposition A.11  implies that its Gelfand transform has no zeros and hence
this $B\uuu d$-element is invertible, i.e. there exist
$g\in B$ and a discrete measure $\gamma$ such that
\[
\delta_0=(f+\mu)*(g+\gamma)\tag{i}
\]
Notice  that the right hand side becomes
\[
f*g+ f*\gamma+g*\mu+\mu*\gamma
\]
Here $f*g+ f*\gamma+g*\mu$ belongs to $B$
while
$\mu*\gamma$ is a discrete measure. So (i) implies that
$\gamma$ must be the inverse of $\mu$ in
$M_d({\bf{R}})$
and hence (i) also gives the equality:
\[
f*g+f*\mu^{-1}+g*\mu=0\tag{ii}
\]
\bigskip














\noindent
\centerline {\bf B.  A Tauberian Theorem.}
\medskip

\noindent
Consider the Banach algebra $B$ above.
The dual space $B^*$ consists of  Riesz measures
$\mu$ on the real line for which there exists a constant $A$ such that
\[ 
\int_\nu^{\nu+1}\, |d\mu(x)|\leq A\quad\text{for all integers}\,\,\nu\,.
\] 

\noindent
The smallest $A$ above is  the norm of $\mu$ in $B^*$ and 
duality is expressed by:
\[ 
\mu(f)=\int\, f(x)\cdot d\mu(x)\,\quad\colon\quad f\in B\,\,\text{and}\,\, \mu\in B^*
\]
\noindent 
Let $f\in B$ be such that  $\widehat f(\xi)\neq 0$ for all
$\xi$. For each $a>0$ it follows from
Theorem A.5 that the $f$-image in $B_a$ generates the whole algebra.
Since this hold for every $a>0$
it follows that each
$\phi\in\mathcal F_0^\infty$
belongs to the principal ideal generated by $f$ in $B$, i.e. there exists some
$g\in B$ such that
\[
 \phi=g*f\tag{*}
\]
Since $\mathcal F_0^\infty$ is dense in $B$ we conclude that
$B\cdot f$ is dense in $B$.
Using this density we have:


\bigskip

\noindent 
{\bf B.1  Theorem} \emph{Let 
$\mu\in B^*$ be such that}
\[
\lim_{y\to+\infty}\, \int\, f(y-x)\cdot d\mu(x)=A\,\,\, \text{exists}.
\]
\emph{Then, for each $g\in B$ it follows that}
\[
\lim_{y\to+\infty}\, \int\, g(y-x)\cdot d\mu(x)=B
\quad\text{where}\quad B=A\cdot \frac{ \hat g(0)}{\hat f(0)}
\]

\medskip

\noindent \emph{Proof.} 
Let $g\in B$. If $\epsilon>0$ we find
$h_\epsilon\in B$ such that 
$||g-f*h_\epsilon||_B<\epsilon$. When  $y>0$ we get:
\[
\int\, (f*h_\epsilon)(y-x)\cdot d\mu(x)=\tag{i}
\]
\[
\int\, \bigl[f(y-s-x)h_\epsilon(s)\cdot ds\,\bigr] \cdot  d\mu(x)=
\int\, h_\epsilon(s)\cdot\bigl[\int \,f(y-s-x)\mu(x)\,\bigr]\cdot ds
\]
By the hypothesis the inner integral converges to $A$ when
$y\to+\infty$
every fixed $s$. Since
$h$ belongs to $B$ it follows easily that
the limit of (i) when $y\to+\infty$ is equal to
\[ 
A\cdot \int\, h_\epsilon(s)\cdot ds=A\cdot \widehat h_\epsilon(0)\tag{ii}
\]


\noindent
Next, since the $B$-norm is stronger than the $L^1$-norm it follows that
\[ 
|\widehat g(0)-\hat h_\epsilon(0)\cdot \widehat f(0)|<\epsilon\tag{iii}
\]
Moreover, since the $B$-norm is invariant under translations
we have
\[
\bigr |
 \int\, g(y-x)d\mu(x)-\int\, (f*h_\epsilon)(y-x)\cdot d\mu(x)\bigl|\leq
 \epsilon\cdot ||\mu||\quad\text{for all}\,\,y\tag{iv}
\]
where $||\mu||$ is the norm of $\mu$ in the dual space $B^*$.
Notice also that   (iii) gives:

\[
\lim_{\epsilon\to }\, \hat h_\epsilon(0)=\frac{\hat g(0)}{\hat f(0)}
\]
Finally, since $\epsilon>0$ is arbitrary we see that the limit formula for (i) when
$y\to+\infty$ expressed by
(ii)  and (iv) above together imply that


\[
\lim_{y\to+\infty}\, 
\int\, g(y-x)d\mu(x)=A\cdot \frac{\hat g(0)}{\hat f(0)}
\]
This finishes the proof of 
Theorem A.9

\bigskip




\centerline
{\bf B.2  The multiplicative version}
\medskip



\noindent 
Let ${\bf{ R}}^+$ be the multiplicative group of positive real numbers.
To each function $f(t)$ on ${\bf{R}}^+$
we get the function $E_f(x)= f(e^x)$ on the real $x$-line.
Since $dt=e^xdx$ under the exponential map we have
\[
\int_0^\infty\, f(t)\frac{dt}{t}=
\int_{-\infty}^\infty\, E_f(x)dx
\]
provided that $f$ is integrable.
On $\bf R^+$ we get the convolution algebra $L^1(\bf R^+)$ where
\[ 
f*g(t)=\int_0^\infty\, f(\frac{t}{s})\cdot g(s)\cdot \frac{ds}{s}
\]
This convolution commutes with the $E$ map from
$L^1({\bf {R}}^+$ into $L^1({\bf {R}}^1)$, i.e. 
\[
E_{f*g}= E_f*E_g
\]
Next, recall that the Fourier transform on $L^1({\bf{ R}}^+)$
is defined by
\[ 
\widehat f(\xi)=\int_0^\infty\,t^{-i\xi} \cdot f(t)\cdot \frac{dt}{t}
\]
\medskip









\noindent
{\bf B.3  The Banach algebra $B_m$.} The companion to $B$ on
${\bf {R}}^+$ consists of continuous functions $f(t)$ for which
\[ 
\sum\, ||f||_{[2^\nu,2^{\nu+1}]}<\infty
\]
where the is taken over all integers. Notice  that
with $\nu<0$ one takes small intervals approaching $t=0$.
Just as in Theorem A.9  we obtain a Tauberian Theorem for functions
$f\in B_m$ whose Fourier transform is everywhere $\neq 0$.
Here we  the dual space $B\uuu m^*$
consists of Riesz measures $\mu$ on ${\bf{R}}^+$
for which there exists a constant $C$ such that

\[
\int\uuu{2^m}^{2^{m+1}}\, |d\mu(t)|\leq C
\] 
for all integers $m$.



\bigskip



\bigskip


\centerline{\bf C. Ikehara's theorem.}
\bigskip


\noindent
Let
$\nu$ be a non-negative Riess measure supported on
$[1,+\infty)$ and assume that
\[
\int_1^\infty x^{-1-\delta}\cdot d\nu(x)<\infty\quad
\text{for all}\,\, \delta>0
\]
When this holds we obtain an analytic function
$f(s)$ of the complex variable $s$ defined in
the right half plane
$\mathfrak{Re}(s)>1$ by
\[ 
f(s)=\int_1^\infty x^{-s}\cdot d\nu(x)
\]
\medskip

\noindent
{\bf D.1 Theorem.}
\emph{Assume that there exists a constant
$A$ and a continuous function
$G(u)$ defined on the real $u$-line such that}
\[
\lim_{\epsilon\to 0}\, \bigl[f(1+\epsilon+iu)-\frac{A}{1+\epsilon+iu}\,\bigr]=G(u)\tag{*}
\]
\emph{where the limit holds uniformly   on  every
bounded interval $-b\leq u\leq b$.
Then}
\[ 
\lim_{x\to+\infty}\,\frac{1}{x}\int_1^x\,d\nu(t)=A\tag{**}
\]
\medskip

\noindent
We shall prove a sharper version 
of
Ikehara's result 
where
the assumption on $G(u)$ is relaxed.
Namely, replace  (*)
by the weaker assumption that there
exists a locally integrable function $G(u)$ such that
\[
\lim_{\epsilon\to 0}\, \int_{-b}^b\, \bigl|
f(1+\epsilon+iu)-\frac{A}{1+\epsilon+iu}-G(u)\bigr|\cdot du=0\quad
\text{holds for each}\,\, b>0\tag{***}
\]
\medskip



\noindent
{\emph{Proof that (***) gives (**)}}.
To show this implication
we use  some variable substitutions.
With $x\mapsto e^\xi$
we can write 
\[ 
f(s)=\int_0^\infty\, e^{-\xi s}\cdot d\nu(e^\xi)
\]



\noindent
Next, define the function measure $\mu$ on the non-negative real $\xi$-line by
\[
d\mu(\xi)= e^{-\xi}\cdot d\nu(e^\xi)-A\cdot d\xi\,\quad\colon\,\,\xi\geq 0\tag{1}
\]
Then we see that
\[ 
f(s)-\frac{A}{s-1}=\int_0^\infty e^{(1-s)\xi}d\mu(\xi)\tag{2}
\]
It is clear that  (**) holds if and only if
\[
 \lim_{\eta\to\infty}\int_0^\eta\, e^{-\eta+\xi}\cdot d\mu(\xi)=0\tag{3}
\]
\bigskip

\noindent
\emph{A reformulation of Ikehara's theorem.}
From the observations above we
can restate the sharp version of Ikehara's theorem.
Let
$\nu^*$ be a non-negative measure on $0\leq\xi<+\infty$
such that 
\[ \int_0^\infty\, e^{-\delta\cdot \xi}\cdot d\nu^*(\xi)<\infty
\quad\text{for all}\,\,\delta>0\tag{1}
\]
Next, let $A>0$ be some positive constant
and put $d\mu(\xi)=d\nu^*(\xi)-A\cdot d\xi$.
Then (1) gives
the analytic function
$g(s)$ defined in $\mathfrak{Re}(s)>0$ by
\[
g(s)=\int_0^\infty\, e^{-s\cdot\xi}\cdot d\mu(\xi)
\]
\medskip

\noindent
{\bf D.2. Definition.}
\emph{We say that the  measure $\mu$ is of the
Ikehara type if there exists a locally integrable function
$G(u)$ defined on the real $u$-line such that}
\[
\lim_{\epsilon\to 0}\, \int_{-b}^b\, \bigl|
g(\epsilon+iu)-G(u)\bigr|\cdot du=0\,\quad
\text{holds for each}\,\, b>0
\]
\medskip

\noindent
{\bf D.3. The space $\mathcal W$.}
Let $\mathcal W$ be the space of continuous functions
$\rho(\xi)$ defined on $\xi\geq 0$ which satisfy:
\[
\sum_{n\geq 0}\,||\rho||_n<\infty\,
\quad\text{where}\,\,||\rho||_n=\max_{n\leq u\leq n+1}\, |\rho(u)|
\]
The dual space
$\mathcal W^*$ consists of Riesz measures
$\gamma$ on $[0,+\infty)$ such that
\[ 
\max_{n\geq 0}\,\int_n^{n+1}\, |d\gamma(\xi)|<\infty
\]
With these notations we have
\bigskip

\noindent
{\bf D.4. Theorem.}
\emph{Let $\nu^*$ be a non-negative meausure
on $[0,+\infty)$ and $A\geq 0$ some constant
such that the measure $\mu=\nu^*-A\cdot d\xi$
is of Ikehara type.
Then $\mu\in \mathcal W^*$
and for every
function $\rho\in\mathcal W$ one has}
\[
\lim_{\eta\to+\infty}\, \int_0^\eta
\rho(\eta-\xi)\cdot d\mu(\xi)=0
\]
\medskip

\noindent
{\bf {Exercise.}} 
Use the material above to show that
Theorem D. 4 gives the sharp
version of Ikehara's theorem.
The  hint is to use the function $\rho(s)=e^{-s}$ above.

\medskip

\centerline {\emph{Proof of Theorem D.4.}}
\bigskip

\noindent
Let $b>0$ and define the function $\omega(u)$ by
\[
\omega(u)= 1-\frac{|u|}{b}\,,\quad
-b\leq u\leq b\,\,\quad\text{and}\,\,
\omega(u)=0\,\,\text{outside this interval}\tag{i}
\]


\noindent
Set
\[ J_b(\epsilon,\eta)=
\int_{-b}^b\, e^{i\eta u}\cdot g(\epsilon+iu)\cdot \omega(u)\cdot du\tag{ii}
\]
From Definition 2 we have the $L^1_{\text{loc}}$-function $G(u)$
and since $\omega(u)$ is a continuous function on the compact
interval $[-b,b]$
we have
\[ 
\lim_{\epsilon\to 0}\, J_b(\epsilon,\eta)=J_b(0,\eta)=
\int_{-b}^b\, e^{i\eta u}\cdot G(u)\cdot \omega(u)\cdot du\tag{iii}
\]
With $b$ kept fixed  the right
hand side
is a Fourier transform of an
$L^1$-function. So  the Riemann-Lebesgue theorem gives:
\[
\lim_{\eta\to+\infty}\, J_b(0,\eta)=0\tag{iv}
\]
Moreover, the triangle inequality gives the inequality:
\[
|J_b(0,\eta)|\leq 
\int_{-b}^b\,\bigl| G(u)\bigr|\cdot du\tag{v}
\]
\medskip


\noindent{\bf Some integral formulas.}
From the above
it  is clear that
\[ J_b(\epsilon,\eta)=
\int_0^\infty\,
\bigl[\int_{-b}^b\, e^{i\eta u-i\xi u}\cdot \omega(u)\cdot du\bigr]
\cdot e^{-\epsilon\cdot\xi}\cdot d\mu(\xi)\tag{1}
\]


\noindent
Next, notice that
\[
\int_{-b}^b\, e^{i\eta u-i\xi u}\cdot \omega(u)\cdot du=
2\cdot\frac{1-\text{cos}\,b(\eta-\xi)}{b(\eta-\xi)^2}\tag{2}
\]
Hence we obtain
\[ J_b(\epsilon,\eta)=
2\cdot\int_0^\infty\, 
\frac{1-\text{cos}\,b(\eta-\xi)}{b(\eta-\xi)^2}\cdot e^{-\epsilon\xi}
\cdot d\mu(\xi)\tag{3}
\]

\noindent
From (iii) above it follows that  (3)
has a limit as $\epsilon\to 0$ which is equal to 
the integral in the right hand side in (iii) which is
denoted by $J_b(0,\eta)$. Next, it is 
easily seen that there exists the limit

\[
\lim_{\epsilon\to 0}\, 
2\cdot\int_0^\infty\, 
\frac{1-\text{cos}\,b(\eta-\xi)}{b(\eta-\xi)^2}\cdot e^{-\epsilon\xi}
\cdot Ad\xi=2\pi\cdot A\tag{4}
\]
\medskip

\noindent
Hence (3-4) imply that there exists the limit
\[
\lim_{\epsilon\to 0}\, 
2\cdot\int_0^\infty\, 
\frac{1-\text{cos}\,b(\eta-\xi)}{b(\eta-\xi)^2}\cdot e^{-\epsilon\xi}
\cdot d\nu^*(\xi)=J_b(0,\eta)+2\pi\cdot A\tag{5}
\]
\medskip

\noindent
Next, the measure  $\nu^*\geq 0$ and the function
$\frac{1-\text{cos}\,b(\eta-\xi)}{b(\eta-\xi)^2}\geq 0$
for all $\xi$. So the existence of a finite limit in (5) entails that there exists
the convergent integral 
\[
\int_0^\infty\, 
\frac{1-\text{cos}\,b(\eta-\xi)}{b(\eta-\xi)^2}
\cdot d\nu^*(\xi)=J_b(0,\eta)+2\pi\cdot A\tag{6}
\]



\medskip


\noindent
{\bf Proof that
$\mu\in\mathcal W^*$.}
Since $A\cdot d\xi$ obviously belongs to
$\mathcal W^*$ it suffices to show that
$\nu^*\in\mathcal W^*$.
To prove this we consider some integer $n\geq 0$ and with $b=1$
it is clear that
(6) gives
\[
\bigl|\int_n^{n+1}\, 
\frac{1-\text{cos}\,(\eta-\xi)}{(\eta-\xi)^2}
\cdot d\nu^*(\xi)\bigr|\leq |J_1(0,\eta)|+2\pi=
\int_{-1}^1\, |G(u)|\cdot du+2\pi\cdot A
\]
Apply this with
$\eta=n+1+\pi/2$ and notice that

\[
\frac{1-\text{cos}\,(n+1+\pi/2-\xi)}{(n+1+\pi/2-\xi)^2}
\geq a\quad\,\text{for all}\,\, n\leq\xi\leq n+1
\]
This gives a constant $K$ such that
\[
\int_n^{n+1}\, d\nu^*(\xi)\leq K\quad\, n=0,1,\ldots
\]
\medskip

\noindent
{\emph {Final part of the proof.}}
We have proved that
$\mu\in \mathcal W^*$. Moreover, from
(iv) above and the integral formula (6) we get
\[
\lim_{\eta\to+\infty}\, \int_0^\infty
\frac{1-\text{cos}\,b(\eta-\xi)}{b(\eta-\xi)^2}\cdot d\mu(\xi)=0
\quad\,\text{for all}\,\, b>0\tag{*}
\]
Next, for each fixed 
$b>0$ we notice that the function
\[
\rho_b(\xi)=2\cdot \frac{1-\text{cos}\,(b\xi)}{b\cdot \xi^2}
\]
belongs to $\mathcal W$ and its
Fourier is  $\omega_b(u)$.
Here $\omega_b(u)\neq 0$ when
$-b<u<b$.
So the family of these $\omega$-functions have no common
zero on the real $u$-line. By the Remark in XX this means that
the linear subspace of
$\mathcal W$ generated by
the translates of all $\rho_b$ -functions with
arbitrary large
$b$ is dense in $\mathcal W$. 
Hence (*) above implies
that
we get a zero limit as $\eta\to+\infty$ for every function
$\rho\in\mathcal W$. But this is precisely the
assertion in
Theorem 4.


\bigskip


\centerline{\bf E. The algebra $L^1({\bf{R}}^+)$}

\bigskip

\noindent
Consider the  family
of $L^1$-functions on the real $x$-line which are
supported by the half-line $x\geq 0$.
This yields a closed subalgebra of $L^1({\bf{R}})$
denoted by
$L^1({\bf{R}}^+)$. 
Indeed, if $f(x)$�and $g(x)$ are  two such functions
in
$L^1({\bf{R}}^+)$.
the support of 
the convolution
$g*f$ stays in $[0,+\infty)$. 
Adding the unit point mass $\delta_0$ we obtain a commutative
Banach algebra
\[ 
B={\bf{C}}\cdot \delta_0+L^1({\bf{R}}^+)
\]


\noindent
{\bf E. 1. The Gelfand space
$\mathfrak{M}_B$}.
To obtain this space we consider some
$f(x)\in L^1({\bf{R}}^+)$
and set:
\[
\widehat f(\zeta)=\int_0^\infty\, e^{i\zeta x}\cdot f(x)\cdot dx,\quad\text{where}
\quad \mathfrak{Im}(\zeta)\geq 0
\]
With $\zeta=\xi+i\eta$ we get
\[
|\widehat f(\xi+i\eta)|\leq 
\int_0^\infty\, |e^{i\xi x-\eta x} |\cdot|f(x)|\cdot dx=
\int_0^\infty\, |e^{-\eta x}\cdot |f(x)|\cdot dx\leq ||f||_1
\]
We conclude that
for every point $\zeta=\xi+i\eta$ in the closed upper half-plane 
corresponds to a point
in $\zeta^*\in \mathfrak{M}_B$
defined by
\[ 
\zeta^*(f)=
\widehat f(\zeta)\,\quad\text{and}\quad \,\zeta^*(\delta_0)=1
\]
In addition to this 
 $L^1({\bf{R}}^+)$ is a maximal ideal in
 $B$ and there is the special point
 $\zeta^\infty\in \mathfrak{M}_B$ such that
 \[ 
 \zeta^\infty (f)=0\quad\text{for all}\,\, f\in
L^1({\bf{R}}^+ )
\]
\medskip

\noindent
{\bf E.2. Theorem.} \emph{The Gelfand space
$\mathfrak{M}_B$ can be identified with
the union  of $\zeta^\infty $ and the closed upper half-plane.}
\medskip

\noindent
{\bf Remark.}
The theorem asserts that every multiplicative
functional on $B$ is either
$\zeta^\infty $ 
or determined by a point
$\zeta=\xi+i\eta$ where
$\eta\geq 0$.
Concerning the topological identification
$\zeta^\infty $ corresponds to the 
one point compactification of the closed
upper half-plane. Thus, whenever
$\{\zeta_\nu\}$ is a sequence in
$\mathfrak{Im}(\zeta)\geq 0$ such that
$|\zeta_\nu|\to\infty$ then $\{z_\nu^*\}$
converges to $\zeta^*$ in
$\mathfrak{M}_B$.
In fact, this follows via the Riemann-Lebegue Lemma which gives
\[ 
\lim_{|\zeta|\to\infty}\, \widehat f(\zeta)=0
\quad\text{for all}\,\, f\in L^1({\bf{R}}^+)
\]
\medskip

\noindent
By the general result in XX 
Theorem 2 holds  if we have
proved if   the following:

\medskip

\noindent
{\bf E.3. Proposition.}
\emph{Let $\{g_\nu=\alpha_\nu\cdot\delta_0+f_\nu\}_1^k$
be a finite family in $B$ such that
the $k$-tuple $\{\hat g_\nu\}$
has no common zero in
$\bar U_+\cup\{\infty\}$.
Then the ideal in $B$ generated by this $k$-tuple is equal to
$B$.}

\medskip

\noindent
The proof requires some preliminary constructions.
We  use the conformal map from
the upper half-plane onto the unit disc defined by

\[
w=\frac{\zeta-i}{\zeta+i}
\]
So here $w$ is the complex coordinate in $D$.
Next, consider the disc algebra $A(D)$.
Via  the conformal map
each transform $\widehat f(\zeta)$ of a
function $f\in L^1(\bf{R}^+)$
yields an element of $A(D)$ defined by:

\[ 
F(w)=\widehat f(\frac{i+iw}{1-w})
\]
It is clear that
$F(w)\in A(D)$. Moreover, we notice that
\[
w\to 1\implies\, 
|\frac{i+iw)}{1-w}|\to\infty
\]
It follows that the $A(D)$-function $F(w)$ is zero at
$w=1$ and we can conlcude:
\medskip

\noindent
{\bf E.4. Lemma.} \emph{By 
$f\mapsto F$ we have an algebra homomorphism
from $L^1({\bf{R}}^+)$ to functions in
$A(D)$ which are
zero at $w=1$.}
\medskip

\noindent
Next, let  $\mathcal H$ denote the algebra homomorphism in Lemma 4
and 
consider the function $1-w$ in $A(D)$.
We claim this it belongs to the image under $\mathcal H$.
To see this we consider the function
\[
f(x)= e^{-x}\quad\,x\geq 0\,\quad\text{and}\,\, f(x)=0
\quad\text{when}\,\, x<0
\]
Then we see that
\[
\hat f(\zeta)=\int_0^\infty \, e^{i\zeta x}\cdot e^{-x}\cdot dx=
\frac{1}{1-i\zeta}
\]
It follows that
\[ 
F(w)=\frac{1}{1-i(\frac{iw+i}{1-w})}=\frac{1-w}{1-w+w+1}=
\frac{1-w}{2}
\]
Using $2f$ we conclude that
$1-w$ belongs to the $\mathcal H$-image.
Next, the identity element $\delta_0$ is mapped to the constant function
on $D$. So via $\mathcal H$ we have an algebra homomorphism
from $B$ into a subalgebra of $A(D)$ which contains
$1-w$ and the identity function and hence all $w$-polynomials.
Returning to the special $B$-element $f$ we notice that the convolution

\[ 
f*f(x)= x\cdot e^{-x}
\]
We can continue and conclude that the subalgebra of $B$ 
generated by $f$ and $\delta_0$ contains 
$L^1$-functions of the form $p(x)\cdot e^{-x}$ where $p(x)$ are polynomials.
\medskip


\noindent
{\bf{E.5. Exercise.}}
Prove that the linear space ${\bf{C}}[x]\cdot e^{-x}$
is a dense subspace of $L^1({\bf{R}}^+)$.
\medskip

\noindent
From the result in  the exercise it follows that
the polynomial algebra ${\bf{C}}[w]$ appears as a dense subalgebra
of $\mathcal H(B)$ when it is equipped with the $B$-norm.
At this stage we are prepared to give:

\bigskip

\noindent
{\bf{Proof of Proposition E.3.}}
In $A(D)$ we have the functions $\{G_\nu=\mathcal H(g_\nu)\}$.
By assumption $\{G_\nu\}$ have no common zero in the closed disc $D$.
Since $D$ is the maximal ideal space of the disc algebra and
${\bf{C}}[w]$  a dense subalgebra, it follows that
for every $\epsilon>0$ there exist
polynomials $\{p_\nu(w)\}$ such that
the maximum norm
\[
|p_1\cdot G_1+\ldots+p_k\cdot G_k-1|_D<\epsilon|\tag{1}
\]
where 1 is the identity function.
Now $p_\nu=\mathcal H(\phi_\nu)$ for $B$-elements $\{\phi_\nu\}$.
So in $B$ we get the element
\[
\psi=\phi_1g_1+\ldots+\phi_k\cdot g_k\tag{2}
\] 
Moreover we have $|\mathcal H(\psi)-1|_D<\epsilon$ and here we can choose
$\epsilon<1/4$ and by the previous identifications it follows that
\[ 
|\widehat \psi(\xi)|\geq 1/4\quad\text{for all}\quad -\infty<\xi<\infty\tag{3}
\]
The proof of Proposition E.3 is finished if we can show that (3) entails that
the $B$-element $\psi$ is invertible.
Multiplying $\psi$ with a non-zero scalar we may assume that
\[ 
\psi=\delta_0-g\quad\colon\quad g\in L^1({\bf{R}}^+)
\]
and  the Fourier transform $\widehat\psi(\xi)$
satisfies
\[
|\widehat\psi(\xi)-1|\leq 1/2
\]
for all $\xi$.
It means that $|\widehat g(\xi)|\leq 1/2$.
The spectral radius formula applied to $L^1$-functions shows that if
$N$ is a sufficiently large integer then
\[
||g^{(N)}||_1\leq (3/4)^N\tag{4}
\]
where $g^{(N)}$ is the $N$-fold convolution of $g$.
Now we have
\[
(1+g+\ldots+g^{N_1})\cdot \psi= 1-g^{(N)}\tag{5}
\]
By (4)  the norm of the $B$-element $g^{(N)}$ is strictly less than one
and hence the right hand side is invertible where the inverse is given
by a Neumann series, i.e. with $g_*=g^{(N)}$  the inverse is

\[ 
\delta_0+\sum_{\nu=1}^\infty\, g_*^\nu
\]
Since convolutions of $L^1({\bf{R}}^+)$-functions still are supported by
$x\geq 0$, it follows from the above that $\psi$ is invertible in $B$ and
Proposition E.3 is proved.


\newpage

%\end{document}

