%\documentclass{amsart}
%\usepackage[applemac]{inputenc}


%\addtolength{\hoffset}{-12mm}
%\addtolength{\textwidth}{22mm}
%\addtolength{\voffset}{-10mm}
%\addtolength{\textheight}{20mm}


%\def\uuu{_}


%\def\vvv{-}

%\begin{document}


\centerline{\bf\large{16. Entire functions of exponential type}}

\bigskip

\centerline
{\emph{Contents.}}
\bigskip

\noindent
A. Growth of entire functions
\bigskip

\noindent
B. Hadamard's factorisation for $\mathcal E$.

\bigskip

\noindent
C. The Carleman class $\mathcal N$.

\bigskip

\noindent
D. Tauberian theorems

\bigskip

\noindent
E. Application to measures with compact support.



\bigskip

\centerline
{\bf{Introduction.}}
\bigskip

\noindent
The class $\mathcal E$ of entire functions of exponential type is defined as follows:
\bigskip

\noindent
{\bf{0.1 Definition.}} \emph{An entire function $f$ belongs to
$\mathcal E$ if and only if there exists constants
$A$ and�$C$ such that}
\[
 |f(z)|\leq C\cdot e^{A|z|}\quad\colon\,\, z\in{\bf{C}}\tag{*}
\] 

\medskip

\noindent
We refer to the literature for studies 
of the more extensive class 
of entire functions with arbitrary finite order, i.e.
those $f$ where (*)  is 
replaced by $|z|^\rho$ for some
$\rho>0$.
The results in Sections A\vvv B are foremost due to Hadamard and Lindel�f.
The  class $\mathcal N$ in � 3
was originally used by Carleman
to prove certain approximation theorems related to moment problems.
Our main concern deal with
Tauberian theorems which are established in � D. Here
the material is based upon Chapter V in [Paley\vvv Wiener].
Let us describe some  of the results to be proved in � D
while we refer to sections A and B for
elementary facts  about  $\mathcal E$.
 Consider a non\vvv decreasing sequence $\{\lambda\uuu\nu\}$
of positive real numbers such that the series
\[ 
\sum\, \lambda\uuu \nu^{\vvv 2}<\infty\tag{1}
\]
When this holds there exists the entire function
given by a product series:

\[ 
H(z)=\prod\, (1\vvv \frac{z^2}{\lambda_\nu^2})
\]
Notice that the function $H(z)$ is even and positive
on the  imaginary
axis. Consider   the function defined for real $y$:
\[
y\mapsto \frac{\log H(iy)}{y}=
\frac{1}{y}\, \sum \log\, (1+ \frac{y^2}{\lambda\nu^2})
\]
At the same time we have  the intergals
\[ 
J(R)=\int\uuu{\vvv R}^R\, \frac {\log |H(x)|\cdot dx}{x^2}
\]
With these notations a major result in � D asserts the following:

\medskip
\noindent
{\bf{0.1 Theorem.}} \emph{The statements }
\[
\lim\uuu{y\to \infty}\, \frac{\log H(iy)}{y}= \pi A y\tag{i}
\] 
\emph{and}
\[
\lim\uuu{R\to \infty} J(R)=\vvv \pi^2A\tag{ii}
\]
\emph{hold for some $A\geq 0$ are completely equivalent}
\newpage

\noindent
{\bf{Example.}}
Consider the case when $\{\lambda_\nu\}$ is the set of positive integers.
Here one has the formula
\[
\frac{\sin\, \pi z}{\pi z}=\prod\uuu{n=1}^\infty\, (1\vvv \frac{z^2}{n^2})
\]
Notice that
\[
|\sin\, \pi iy|= \frac{e^{\pi y}\vvv e^{\vvv \pi y}}{2}
\] 
when $y>0$.
From this it follows that  $A=1$ holds in (i) above.
At the same time we proved by residue  calculus shows 
that if $f(z)=\frac{\sin\, \pi z}{\pi z}$ then 
\[
\lim\uuu{R\to \infty}\,
\int\uuu{\vvv R}^R\, \frac {\log |f(x)|\cdot dx}{x^2}=\vvv \pi^2
\]
which is in accordance with (ii) in Theorem 0.1.
A special  case occurs when 
the limit in (ii) is automatically satisfied
by an integrability condition, i.e. when one has 
an absolutely convergent integral
\[
\int\uuu{\vvv\infty}^\infty\,\frac {\bigl|\log |H(x)|\,\bigr |\cdot dx}{x^2}\tag{*}
<\infty
\]
In this case the $J$\vvv integrals converge and as a consequence 
there exists  a limit in (i).
It turns out that further conclusions can be made.
Namely, the convergence of (*) implies that
the
sequence $\{\lambda\uuu\nu\}$
has a regular growth which means  that if $N(r)$ is the counting function
which for every $r>0$ counts the number of $\lambda\nu\leq r$, then
there exists the limit
\[ 
\lim\uuu{R\to\infty}\,
\frac{N(R)}{R}=A
\]
with $A$ determined via Theorem 0.1.
We  prove this in � D and remark that the integrability condition
(*) is related to the study of the Carleman class in
� C.
\medskip

\noindent
{\bf{Remark.}}
The general Tauberian theorems for entire functions 
of exponential type are  due to Wiener. 
See his pioneering article [Wiener].













\bigskip



\centerline{\bf{A. Growth of entire functions.}}


\bigskip

\noindent
Let $f$
be an  arbitrary entire function.
We  shall associate   certain functions
which describe the growth  and the number  of its zeros
in discs of radius $R$ centered at the origin.
We can   write
\[ 
f(z)=az^m\cdot f_*(z)
\] 
where $f_*$ is entire and $f_*(0)=1$.
The  case when $f(0)=1$ is therefore not so special and
several formulas take a simpler form when this holds.





\medskip

\noindent
{\bf{A.1 The functions $T_f(R)$ and $m_f(R)$}}.
They are defined for every $R>0$ by
\[ 
T_f(R)=\frac{1}{2\pi}\cdot   \int_0^{2\pi}\,
\text{Log}^+\,\bigl  |f(R(e^{i\theta})\bigr |\cdot d\theta\tag{i}
\]
\[
m_f(R)=\frac{1}{2\pi}\cdot \int_0^{2\pi}\,
\text{Log}^+\bigl[\frac{1}{\,\bigl  |f(R(e^{i\theta})\bigr |}\bigr]
\cdot d\theta\tag{ii}
\]
\medskip

\noindent
{\bf{A.2 The maximum modulus function.}} It is defined by
\[ M_f(R)= \max_
{0\leq \theta\leq 2\pi}\, |f(Re^{i\theta})|
\]
\medskip

\noindent
{\bf{A.3  The counting function $N_f(R)$}}.
To each $R>0$ we count the number of zeros of
$f$ in the punctured disc $0<|z|<R$. This integer is denoted by
$N_f(R)$, where
multiple zeros are counted according to their
multiplicities.
Jensen's formula shows that if
$f(0)=1$ then
\[
\int_0^R\, \frac{N_f(s)}{s}\cdot ds
=\frac{1}{2\pi}\cdot \int_0^{2\pi}\,
\text{Log}\,\bigl  |f(R(e^{i\theta})\bigr |\cdot d\theta=
T_f(R)-m_f(R)\tag{*}
\]
\medskip

\noindent
Since the left hand side always is
$\geq 0$ the inequality below holds under the hypothesis that
$f(0)=1$:
\[ 
m_f(R)\leq T_f(R)\tag{**}
\]
Next, since $N_f(R)$ is increasing we get
\[
\log 2\cdot N_f(R)\leq \int_R^{2R}\, \frac{N_f(s)}{s}\cdot ds
\leq T_f(2R)\implies
\]
\[
 N_f(R)\leq\frac{T_f(2R)}{\log 2}\tag{***}
 \]





\medskip

\noindent
{\bf{A.4 Harnack's inequality.}}
The function $\text{Log}^+|f|$ is subharmonic
which implies that whenever $0<r<R$ then
one has
\[
\text{Log}^+|f(re^{i\alpha})|\leq
\frac{1}{2\pi}\cdot \int_0^{2\pi}\,
\frac{R+r}{R-r}\cdot 
\text{Log}^+\,\bigl  |f(R(e^{i\theta})\bigr |\cdot d\theta
\]


\noindent
It follows that
\[
M_f(r)\leq \frac{R+r}{R-r}\cdot T_f(R)
\]
In particular we can take $R=2r$ and conclude that
\[
M_f(r)\leq 3\cdot T_f(2r)\quad\text{hold for every}\quad  r>0
\]
The last inequality gives:
\medskip

\noindent
{\bf{A.5 Theorem.}}
\emph{An entire function $f$ belongs to $\mathcal E$ if and only if there
exists a constant $A$ such that}
\[ 
T_f(R)\leq A\cdot R
\]
\emph{holds for every $R$.}
\bigskip

\noindent
{\bf{A.6 A division theorem.}}
Let $f$ and $g$ be in $\mathcal E$ and assume that
$h=\frac{f}{g}$ is entire. Now
\[ 
\log^+|h|\leq \log^+|f|+\log^+|g|\tag{i}
\]
In the case when $g(0)=1$ we apply (**) in A.3 and conclude that
\[ 
T_h(R)\leq T_f(R)+T_g(R)
\]
Hence Theorem A.5 implies that
$h$ belongs to $\mathcal E$.
We leave it to the reader to verify that this conclusion holds
in general, i.e. without any assumption on
$g(0)$.




\bigskip



\medskip

\noindent {\bf{A.7 Hadamard products.}}
Let $\{\alpha_\nu\}$ be a sequence of complex numbers
arranged so that the absolute values are non-decreasing.
The counting function of the sequence is 
denoted by $N_{\alpha(\bullet)}(R)$. Suppose that the counting function
satisfies:
\[
N_{\alpha(\bullet)}(R)\leq A\cdot R\quad\text{for all}\quad R\geq 1\tag{*}
\]

\medskip

\noindent 
{\bf{A.8 Theorem}}
\emph{When (*) holds the infinite product}
\[
\prod\, (1-\frac{z}{\alpha_\nu})\cdot e^{\frac{z}{\alpha_\nu}}
\]
\emph{converges for every $z$ and gives an entire function to be denoted by
$H_{\alpha(\bullet)}$ and  called the Hadamard product of the
$\alpha$-sequence.}

\medskip

\noindent
{\bf{A.9 Exercise.}}
Prove this theorem and show also
that there exists a constant $C$ which is independent of $A$ such that
the Hadamard product satisfies the growth condition:
\[
\bigl|H_{\alpha(\bullet)}(z)\bigr| \leq
C\cdot \text{exp}\bigr[ A\cdot |z|\cdot \log\,|z|\bigl]\quad
\text{for all}\quad |z|\geq e
\]
\medskip


\noindent{\bf{A.10 Lindel�f's condition.}}
For a sequence $\{\alpha\uuu\nu\}$ we define the Lindel�f function
\[ 
L(R)= \sum_{|\alpha_\nu|<R}\, \frac{1}{\alpha_\nu}
\]

\noindent
We say that $\{\alpha\uuu\nu\}$
is of Lindel�f type  if there there exists a constant $L^*$ such that
\[
\bigl|L(R)\bigr|\leq L^*\quad\text{hold  for all}\quad  R.
\tag{**}
\]




\bigskip

\noindent
{\bf{A.11 Theorem.}}
\emph{If the $\alpha$-sequence satisfies (*) in A.8 and is of the Lindel�f type
then
there exists a constant $C$ such that the maximum modules function of
$H\uuu {\alpha(\bullet)}$ satisfies}
\[
M_{H_{\alpha(\bullet)}}(R)\leq C\cdot e^{AR}
\]
\emph{and hence the Hadamard product belongs to $\mathcal E$.}

\medskip

\noindent
{\bf{A.12 Exercise.}}
Prove this result.
A hint is to study the products
\[
\prod_{|\alpha_\nu|<2R}\,(1-\frac{z}{\alpha_\nu})e^{\frac{z}{\alpha_\nu}}
\quad\text{and}\quad 
\prod_{|\alpha_\nu|\geq 2R}\,(1-\frac{z}{\alpha\uuu\nu})e^{\frac{z}{\alpha_\nu}}
\] 
separately for every $R\geq 1$. Try also to find an upper bound for 
$C$ expressed by  $A$ and $L^*$.

\bigskip

\noindent
{\bf{A converse result.}}
Le $f$ belong to $\mathcal L$. Then it turns out that
its set of zeros satisfies (**) in A.10 for a constant $L^*$.
To prove this we shall use:
\medskip


\noindent
{\bf{A.13 An integral formula.}}
With $R>0$ we put
$g(z)=\frac{1}{z}-\frac{\bar z}{R^2}$.
This is a is harmonic function in $\{0<|z|>R\}$ and 
$g=0$ on  $|z|=R$.
Apply Green's formula to $g$ and $\text{Log}\,|f|$
on an annulus $\{\epsilon<|z|<R\}$. Let $f(z)$ be an entire function
with $f(0)=1$
and consider a pair $0<\epsilon<R$ where
$f$ has not zeros in
$|z|\leq\epsilon$.
\bigskip

\noindent
{\bf{A.14 Exercise.}} Show that 
\[
\sum_{|\alpha_\nu|<R}\, \,\bigl[ \frac{1}{\alpha_\nu}-
\frac{\bar\alpha_\nu}{R^2}\bigr]
= \frac{1}{\pi\cdot R}\cdot \int_0^{2\pi}\,
\text{Log}\,|f(Re^{i\theta})|  \cdot e^{-i\theta}\cdot d\theta-f'(0)
\tag{*}
\]
where the sum is taken over zeros of $f$ repeated with multiplicities in
the disc $\{|z|<R\}$.
\bigskip

\noindent
{\bf{A.15 The case $f\in\mathcal E$}}.
Asumme this. From XX we have seen that
the counting function $N\uuu f(R)$ is bounded by
$C\cdot R$ for some constant $C$ and this implies that the series
\[
\sum\,|\alpha\uuu\nu|^{\vvv 2}<\infty
\]
To show that the Lindel�f function $L(R)$
is bounded it therefore sufficices
to show that the function

\[
R\mapsto \frac{1}{\pi\cdot R}\cdot \int_0^{2\pi}\,
\text{Log}\,|f(Re^{i\theta})|  \cdot e^{-i\theta}\cdot d\theta
\] 
is bounded and this follows from A.XX.






 
\bigskip


\centerline {\bf{B. The factorisation theorem for  $\mathcal E$}}

\bigskip

\noindent
Consider some 
$f\in\mathcal E$. 
If $f$ has a  zero at
the origin we can write
\[ 
f(z)= az^m\cdot f_*(z)\quad\text{where}\quad f_*(0)=1
\]
It is clear that $f_*$ again belongs to $\mathcal E$ and in this way
we essentially reduce the study of $\mathcal E$-functions
$f$ to the case when $f(0)=1$. 
Above we proved that the set of zeros satisfies Lindel�f's condition
and therefore the Hadamard product
\[ 
H_f(z)=\prod\,(1-\frac{z}{\alpha_\nu})\cdot e^{\frac{z}{\alpha_\nu}}
\] 
taken over all zeros of $f$ outside the origin belongs to $\mathcal E$.
Now the quotient $f/H\uuu f$ is entire and we shall prove:


\medskip




\noindent{\bf{B.1 Theorem}}
\emph{Let $f\in\mathcal E$ where $f(0)=1$. Then
there exists a complex number $b$ such that}
\[ 
f(z)=e^{bz}\cdot H_f(z)
\]


\noindent
\emph{Proof}.
The division
in
A.6 shows that the function
\[ 
G=\frac{f}{H_f}
\] 
is entire and belongs to $\mathcal E$.
By construction $G$ is zero\vvv free
which gives
the entire
function
$g=\log G$ for we have 
the inequality
\[ 
|g(z)|\leq 1+\log^+|G(z)|\leq 1+C|z|
\]
Since $G\in\mathcal E$ we see that 
$|g|$ increases at most like a linear function so by Liouvile's
theorem it is a polynomial of degree 1. 
Since $f(0)=1$  we have  $g(0)=0$ and hence $g(z)=bz$ for a complex number $b$
and the 
formula in Theorem B.1 follows.




\bigskip



\centerline{\bf\large C. The Carleman class $\mathcal N$}

\medskip

\noindent 
Let $f\in\mathcal E$. On the real $x$-axis we have the non-negative function
$\log^+\,|f(x)|$. If the integral
\[
\int_{-\infty}^\infty\, \frac{\log^+\,|f(x)| \cdot dx}{1+x^2}<\infty\tag{*}
\]



\noindent 
we say that $f$ belongs to the Carleman class 
denoted by $\mathcal N$. To study
$\mathcal N$ the following integral formula
plays an important role.

\bigskip

\noindent 
{\bf C.1 Integral formula in a half-plane.}
Let $g(z)$ be analytic in the half plane $\mathfrak{Im}(z)>0$. Assume that $g$
extends continuously to the boundary $y=0$, i.e. to the real $x$-axis
and that $g(0)=1$. Given a pair $0<\ell<R$
we consider the domain

\[
\Omega_{\ell,R}=\{\ell^2<x^2+y^2<R^2\}\cap\,\{y>0\}
\]


\noindent
With $z=re^{i\theta}$
we have the harmonic function
\[
v(r,\theta)=(\frac{1}{r}-\frac{r}{R^2})\sin\,\theta=\frac{y}{x^2+y^2}-\frac{y}{R^2}
\]
Here $v=0$ on the upper half circle where $ |z|=R$ and $y>0$
and 
the outer normal derivative along the $x$-axis becomes
\[ 
\partial_n(v)=-\partial_y(v)=-\frac{1}{x^2}+\frac{1}{R^2}\quad\colon\,\quad x\neq 0
\]
Let $\{\alpha_\nu\}$ be the zeros of $g$ counted with multiplicites 
in the upper half-plane.Then Green's formula gives:
\medskip

\noindent
{\bf{C.2 Proposition.}} \emph{One has the formula}

\medskip

\[ 
2\pi\cdot\sum\,\frac {\mathfrak{Im}\,\alpha\uuu\nu}{|\alpha\nu|^2}\vvv
\frac{\mathfrak{Im}\, \alpha_\nu}{R^2}
=
\]
\[
\int_\ell^R\,\,\bigl(\frac{1}{R^2}-\frac{1}{x^2}\bigr)\cdot
\text{Log}\,|g(x)\cdot g(-x)|
\cdot dx
\vvv\frac{2}{R}\int_0^\pi\, \text{sin}(\theta)\cdot 
\text{Log}\,|g(Re^{i\theta})|\cdot d\theta+\chi(\ell)\tag{*}
\]
where $\chi(\ell)$ is a contribution from line integrals along
the half circle $|z|=\ell$ with $y>0$.

\bigskip



\noindent
{\bf {C.3 Exercise}}
Prove  via Green's theorem.
Notice that
the term $\chi(\ell)$ is independent of $R$ so
the formula can be used to study  asymptotic behaviour as $R\to+\infty$.
\medskip

\noindent
Next, the family of analytic functions $g(z)$ in the upper half\vvv plane
is identified with $\mathcal O(D)$
using a conformal map, i.e. with a given $g$ we get $g\uuu *\in \mathcal O(D)$
where
\[
g\uuu *(\frac{z\vvv i}{z+i})= g(z)
\] 
holds when $\mathfrak{Im}(z)>0$.
When $g$ extends to a continuous function on
the real $x$\vvv axis we have the equality
As explained in XXX this gives the equality

\[
\int\uuu 0^{2\pi}\, \log^+|g\uuu *(e^{i\theta})|\cdot d\theta=
2\cdot \int_{-\infty}^\infty\, \frac{\log^+\,|g(x)| \cdot dx}{1+x^2}\tag{*}
\]
This means that the last integral is finite if and only if
$g\uuu *$ belongs to the Jensen\vvv Nevanlinna class
and in XX we proved that this entails that
\[
\int\uuu 0^{2\pi}\, \log^+\frac{1}{|g\uuu *(e^{i\theta})|}\cdot d\theta<\infty
\]
In particular we conclude that if an entire function $f$
satisfies (*) above then it follows that
\[
\int_{-\infty}^\infty\, \log^+\frac{1}{|f(x)|} \cdot 
\frac{dx}{1+x^2}\tag{**}
\]
in other words, (*) entails that the absolute value
$|\log\,|f(x)|\,|$ is integrable with respect to
the density $\frac{1}{1+x^2}$.
Using (**) we can prove:



\bigskip

 \noindent {\bf C.4  Theorem} \emph{Let $f\in\mathcal N$. Then}
\[
\sum^*\,\mathfrak{Im}\frac{1}{\alpha_\nu}<\infty
\]
\emph{where the sum is taken over all zeros of $f$ 
which belong to the  upper half-plane.}
\bigskip


\noindent
\emph{Proof.}
Since $f\in\mathcal E$
there exists a constant $C$ such that $N\uuu f(R)\leq C\cdot R$.
If $R\geq 1$ it follows that
\[ 
\bigl|
R^{\vvv 2}\sum\,\bar\alpha_\nu|\leq R^{\vvv 2}\cdot R\cdot N\uuu f(R)\leq C
\]
where  the sum is taken over zeros in $\Omega\uuu{\ell,R}$
Next, since $\mathfrak{Im}\,\alpha\uuu\nu>0$ in this open set
it follows that 
\[
\frac{\mathfrak{Im}\,\alpha_\nu}{|\alpha\uuu \nu|^2}>0
\]
 for every zero in the upper half\vvv plane. In particular this holds for the
 zeros in 
$\Omega\uuu{\ell,R}$ and 
passing to the limit as $R\to \infty$
it suffices to establish an upper bound in the right hand side of
Proposition C.2 with $g=f$.
The integral taken over the half\vvv circle where $|z|=R$
is uniformly  bounded with respect to $R$
since $f\in \mathcal E$
and we have the inequality XX from A.XX.
For the integral on the $x$\vvv axis we therefore only need an upper bound.
Since $R^{\vvv 2}\vvv x^{\vvv 2}\leq 0$
during the integration it suffices to find a constant $C$ such that
\[
\int_\ell^R\,\,\bigl(\frac{1}{x^2}-\frac{1}{R^2}\bigr)\cdot
\log^+\frac{1}{\,|f(x)\cdot f(-x)|}
\cdot dx\leq C\quad\text{hold for all}\quad R\geq 1
\]
The reader may verify that such a constant $C$ since (**) above holds.











\bigskip

\noindent{\bf C.5 A limit for the counting function.}
Using the Tauberian theorem which is proved in
Section D  one has the  following:

\bigskip

\noindent
{\bf C.6 Theorem} \emph{For each $f\in\mathcal N$ there exists the limit:}
\[ 
\lim_{R\to\infty}\,\frac{N_f(R)}{R}
\]


\noindent 
{\bf{C.7 Remark.}}
To prove this we first notice that
if $f\in\mathcal N$ then the product $f(z)\cdot f(\vvv z)$ also 
belongs to $\mathcal N$ and for this even function
the counting function is twice that of $f$. Hence it suffices to prove
Theorem C.6 when $f$ is even.
We may also assume that $f(0)=1$ and since
$f\in\mathcal E$ it is given by a Hadamard product
\[
f(z)=\prod^*\, (1-\frac{z^2}{\alpha_\nu^2})\tag{1}
\]
where $\prod^*$ indicates the we  take the product of zeros whose
real part is $>0$ and if they are purely imaginary 
they are of the form $b\cdot i$ with $b>0$.
We can  replace the zeros by their absolute values and construct
\[
f_*(z)=\prod^*\, (1-\frac{z^2}{|\alpha_\nu|^2})\tag{2}
\]


\noindent
If $x$ is real we see that 
\[
|f_*[x)|\leq |f(x)|\tag{3}
\]
We conclude that if $f$ belongs to
$\mathcal N$ so does $f_*$.
At the same time their counting functions of zeros are equal.
This reduces the proof of Theorem XX to the special case when
$f$ is even and the zeros  are real.
In the next section we study entire and even functions
in $\mathcal E$ whose zeros are real and via a general Tauberian theorem
deduce Theorem C.6 above.

















\bigskip



\centerline{\bf\large D. Tauberian Theorems}


\bigskip

\noindent
To every non-decreasing and discrete sequence of positive real numbers
$\{0<t_1\leq t_2\leq \ldots\}$ we associate the even sequence where we include 
$\{-t_\nu\}$.
Assume that
$\mathcal N_\Lambda(R)\leq C\cdot R$ for some constant.
We get the entire function
\[
f(z)=
\prod\,(1-\frac{z^2}{t_\nu^2})
\]
which by the results in Section A belongs to $\mathcal E$.
If $R>0$ we set:
\[
 J_1(R)=\frac{\text{log}\, f(iR)}{R}
\quad\text{and}\quad 
J_2(R)=\int_{-R}^R\, \frac{\text{Log}\,| f(x)|}{x^2}\cdot dx\tag{*}
\]
\medskip


\noindent
{\bf{D.1 Theorem.}}
\emph{There exists a limit }
\[ 
\lim\uuu {R\to\infty}\, \frac{N\uuu f(R)}{R}=2A
\]
\emph{if and only if at least one of the $J$\vvv functions has
a limit as $R\to\infty$. Moreover, when this holds
one has the equalities:}

\[
\lim_{R\to\infty} J_1(R)=\frac{\pi\cdot A}{2}\quad\text{and}\quad
\lim_{R\to\infty} J_2(R)=\vvv \frac{\pi^2\cdot A}{2}
\]
\bigskip

\noindent
To prove this
we introduce the following:
\medskip


\noindent
{\bf{D.2 The $W$\vvv functions.}}
On the positive real $t$-line we define the following functions:
\[ 
W_0(t)=\frac{1}{t}\quad\colon \,\,t\geq 1\quad
\text{and}\quad W_0(t)=0\quad\text{when}\,\,\, t<1\tag{1}
\]
\[
W_1(t)=\frac{\text{Log}(1+t^2)}{t}\tag{2}
\]
\[
W_2(t)=
\int_0^t
\frac{\text{Log}\,|1-x^2|}{x^2}\cdot dx\tag{3}
\]

\noindent
Next, 
the
real   sequence
$\Lambda=\{t_\nu\}$   gives a discrete measure
on the positive real axis where one  assigns a unit point mass at
every $t_\nu$. If repetitions occur, i.e. if some
$t$-numbers are equal we
add these unit point-masses.
Let $\rho$ denote  the resulting  discrete measure. 
The constructions of the $J$\vvv functions obviously give:
\[ 
\frac{\mathcal N_\Lambda (R)}{R}=
2\cdot \int_0^\infty\, W_0(R/t)\cdot \frac{d\rho(t)}{t}\tag{*}
\]
\[
J_k(R)=
\int_0^\infty\,W_k\bigl(\frac{R}{t}\bigr)\cdot\frac{d\rho(t)}{t}
\quad\colon\, k=1,2\tag{**}
 \]
\medskip

\noindent{\bf{D.3 Exercise.}}
Show that under the assumption that
the function $\frac{\mathcal N_\Lambda (R)}{R}$ is bounded, it follows
the three $\mathcal W$-functions belong to
the $\mathcal {BW}$-algebra defined by the measure $\rho$
as explained in XXX.


\bigskip
\noindent
{\bf  D.4 Fourier transforms.} Recall that on $\{t>0\}$ we have the Haar
measure 
$\frac{dt}{t}$. 
We leave it to the reader to verify that all  the $W$-functions above  belong to $L^1({\bf{R}}^+)$, i.e. 
\[ 
\int_0^\infty\, \bigl|W_k(t)\bigr|\cdot \frac{dt}{t}<\infty\quad\colon
k=0,1,2
\tag{i}
\]
The Fourier
transforms  are defined by
\[ 
\widehat W_k(s)=
\int_0^\infty\,  W_k(t)\cdot t^{-(is+1)}\cdot dt\tag{ii}
\]
We shall prefer to use the functions with reversed sign on $s$, i.e. set
\[
\mathcal F\,W_k(s)=\int_0^\infty\,  W_k(t)\cdot t^{is-1}\cdot dt\tag{iii}
\]

\medskip

\noindent
{\bf{D.5 Proposition}} \emph{One has the formulas}
\[
\mathcal F\,W_0(s)=\frac{1}{1-is}\tag{i}
\]
\[
\mathcal F\,W_1(s)= 
\frac {\pi \cdot e^{\vvv \pi s/2}}{(1\vvv is)\cdot (1+e^{\vvv \pi s})}\tag{ii}
\]
\[
\mathcal F\,W_2(s)=
\frac{2\pi}{(1-is)\cdot (e^{\pi s/2}+e^{-\pi s/2})}\tag{iii}
\]


\bigskip

\noindent
\emph{Proof.}
The equation (i) is easily verified.
To prove (ii) we notice that a partial integration gives
\[
\mathcal F\,W_1(s)=\frac{1}{is\vvv 1}\cdot
\int\uuu 0^\infty\,\frac{2\cdot t^{is}\cdot dt}{1+t^2}
\]
To compute this integral we employ residue calculus where
we
consider the function
\[ 
\phi(z)=\frac{z^{is}}{1+z^2}
\]
We perform  line integrals over 
large half\vvv circles where $z=Re^{i\theta}$ and $0\leq \theta\leq \pi$.
A reside occurs at $z=i$. Notice also that if $t>0$ then
\[
 (\vvv t)^{is}= t^{is}\cdot e^{\vvv \pi s}
 \]
 which gives
\[
 \mathcal FW\uuu 1(s)
 \frac{1}{1\vvv is)\cdot (1+e^{\vvv \pi s})}
 \cdot
 \lim\uuu {R\to\infty}\, \int\uuu{\vvv R}^R\, \phi(t)\cdot dt
\]
Here $\phi$ has a simple pole at $z=1$
so by residue calculus the last integral becomes
\[
\vvv 2\pi i\cdot (i)^{is}\cdot  \frac{1}{2i}
 =\vvv \pi\cdot e^{\vvv \pi s/2}
\]
Taking the minus sign into the account we conclude that

\[
 \mathcal FW\uuu 1(s)=
 \frac {\pi \cdot e^{\vvv \pi s/2}}{(1\vvv is)\cdot (1+e^{\vvv \pi s})}
\]
 


\noindent
For (iii) a partial integration gives
\[
\mathcal F\,W_2(s)=\vvv \frac{1}{is}\cdot
\int\uuu 0^\infty\, \log|1\vvv t^2|\cdot t^{is\vvv 2}\cdot dt
\]
The right hand side is computed in 
[� X: Residue Calculus] which gives  (iii).
\bigskip


\noindent
{\bf{D.6 Evaluations at $s=0$}}
From (i\vvv iii) we find that 
\[ 
\mathcal F W\uuu 2(0)=\frac{\pi}{2}
\quad\text{and}\quad
\mathcal F W\uuu 2(0)=\vvv\frac{\pi^2}{2}
\]
Since we also have $\mathcal F\uuu 1(0)=1$
we apply the  Tauberian Theorem for Beurling-Wiener algebras
in � XX and  read off
the results in Theorem D.1.

\bigskip


\centerline {\emph{ D.8  Proof of Theorem D.1}}
\medskip

\noindent
The formulas for the Fourier transforms in Proposition D.5  show that each
of them is $\neq 0$ on the whole real $s$-line.
Hence we can apply the general result in XX
to the discrete measure $\rho$ since
the
$\mathcal W$-functions belong to
the $\mathcal {BW}$-algebra from XXX. 
This implies that if one of the three
limits in Theorem D.3  above exists, so do the other.
To get the relation between the limit values we 
only have to evaluate the Fourier transform at $s=0$.
From Proposition D.5 we see that
\[
\mathcal FW_0(0)=1\quad\,\colon\quad 
F\mathcal W_1(0)=1\quad\text{and}\,\quad
F\mathcal W_2(0)=\pi\tag{**}
\]



\noindent 
This gives the formulas
in
Theorem D.3 by the general result for $\mathcal{BW}$\vvv algebras in XXX.

\bigskip





\centerline{\bf{E. Application to measures with compact support.}}
\bigskip

\noindent
Le $\mu$ be a Riesz measure on the real $t$\vvv line with
compact support in an interval $[\vvv a,a]$ where
we assume that both end\vvv points belong to the support.
The measure is in general complex\vvv valued.
Now we get the entire function
\[
f(z)= \int\uuu{\vvv a}^a\, e^{\vvv izt}\cdot d\mu(t)
\]
Here $f$ restricts to a bounded function on the real $x$\vvv axis with
maximum norm $\leq ||\mu||$.
Hence $f$ belongs to $\mathcal N$ which means that
Theorem D.x holds.
\medskip

\noindent
{\bf{E.1 Theorem.}}
\emph{One has the equality}
\[
\lim\uuu{R\to \infty}\,\frac{N\uuu f(R)}{R}=\frac{a}{\pi}
\]


\bigskip

















\bigskip





\noindent{\bf {E.2 Tauberian theorems with a remainder term}}
Results  which  contain remainder terms were
established by Beurling in 1936.
An example from Beurling's results
which involve  remainder terms goes as follows:
Let
\[ 
f(z)=\prod\,(1\vvv \frac{z^2}{t\uuu\nu^2})
\]
be an even and entire function of exponential type with real zeros
as in section D. 

\medskip

\noindent
{\bf{E.1 Theorem.}}
\emph{Let $A>0$ and $0<a<1$ and assume that there 
exists a constant $C\uuu 0$ such that}

\[
\bigl|\vvv \frac{1}{\pi^2}\cdot \int\uuu 0^R\,
 \frac{\log |f(x)|}{x^2}\cdot dx\vvv A\,\bigr|\leq
 C\uuu 0\cdot R^{\vvv a}
 \]
\emph{hold for all $R\geq 1$.
Then there is another constant $C$ such that}
\[
\bigl|\, N\uuu f(R)\vvv R\bigr|\leq C\uuu 1\cdot R^{1\vvv a/2}
\]
Beurling's original manuscript which contains  Theorem E.1 as well as other 
results dealing with remainder terms    has remained
unpublished. 
It was   resumed with details of proofs
in  a Master's Thesis at 
Stockholm University 
by F. G�lkan in 1994.
As remarked by Beurling in his article
[Beurling] proofs of 
results with remainders require the full force from the theory of
Fourier integrals in addition to more 
direct use of analytic functions of exponential type.
The interested reader should also consult articles by
Beurlings former Ph.d student
S. Lyttkens which prove various Tauberian theorems with
remainder terms. See also work by T. Ganelius 
for closely related material.





%\end{document}