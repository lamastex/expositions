%\documentclass{amsart}
%\usepackage[applemac]{inputenc}




%\addtolength{\textwidth}{22mm}
%\addtolength{\voffset}{-10mm}
%\addtolength{\textheight}{20mm}
%\def\uuu{_}


%\def\vvv{-}

%\begin{document}





\centerline{\bf\large {XIII. Uniform approximation by Blaschke products}}
\bigskip

\centerline {\emph{Contents.}}
\bigskip

\noindent
\emph {1. Blaschke products and inner functions}.
\bigskip

\noindent
 \emph {2. Proof of Frostman's theorem}
\bigskip

\noindent 
\emph{3. A theorem by Nevanlinna}
\bigskip

\noindent
\emph{4. Marshall's Approximation Lemma}
\bigskip

\noindent 
\emph{5. Density of unimodular functions}

\bigskip

\noindent
\emph{6.  Proof of Marshall's theorem}


\bigskip



\centerline {\bf Introduction}. 
\bigskip


\noindent
A bounded analytic function $f(z)$ in the unit disc $D$
is  an \emph{inner function} if the boundary function
on $T$ has constant absolute value  almost everywhere.
The set of inner functions is denoted by $\mathcal I(D)$.
The first result about inner functions
is due to O. Frostman in [x] :
\medskip
  


\noindent
{\bf 0.1 Theorem}. \emph{Let $w(z)\in\mathcal I(D)$. For every
$\epsilon>0$ there exists a real number $c$ and a Blaschke product
$B(z)$ such that}
\[
\max_{z\in D}\,|w(z)-e^{ic}B(z)|<\epsilon
\]


\noindent
The next result is due to D. Marshall in [Mar] and asserts that
the convex hull of Blaschke products is dense in the unit ball of
$H^\infty(D)$.
\bigskip

\noindent
{\bf 0.2 Theorem.}
\emph{Let $f(z)\in H^\infty(D)$ with $|f|_D\leq 1$. For each $\epsilon>0$ there exists
a finite family  of Blaschke functions
$B_1\,\ldots,B_N$ and
an $N$-tuple of positive real numbers
$a_1,\ldots,a_N$ with $\sum\, a_\nu=1$ such that}
\[
\bigl |f(z)-\sum\,a_\nu\cdot B_\nu(z)\bigr |_D<\epsilon
\]


\medskip

\noindent
{\bf Remark.}
Frostman's theorem reduces Theorem 0.2  to the assertion
that the  convex hull of inner functions is dense in
$H^\infty(T)$. 
The first step in the proof of Theorem 0.2 is to use
an approximation theorem which takes place in
the Banach space $L^\infty(T)$.
Notice that if
$w$ is an inner function then $\frac{1}{w}\in L^\infty(T)$.
We can also take quotients $\frac{w_1}{w_2}$
and get new $L^\infty(T)$-functions.
With this in mind  the following result  is due to
Douglas and Rudin in [xx]
\medskip

\noindent
{\bf 0.3 Theorem} \emph{The convex hull formed by
functions $\frac{w_1}{w_2}$ from pairs in
$\mathcal I(D)$ is dense in $L^\infty(T)$.}

\bigskip

\noindent
We prove this result in � 5.
Now we consider some  $f\in H^\infty(T)$ with
norm $|f|_\infty=1$. 
Theorem 0.3 gives for each
$\epsilon>0$ an
$N$-tuple of unimodular functions
$\{\frac{w_1^\nu}{w_2^\nu}\}$ formed by inner functions 
such that
\[
\bigl |f-
\sum\, a_\nu\cdot \frac{w_1^\nu}{w_2^\nu}\bigr |_\infty
<\epsilon\quad\colon\quad
a_1+\ldots+a_N=1\,\,\colon\,\, a_\nu\geq 0\tag{*}
\]


\noindent
However,  (*) does not give Theorem 0.2
since the individual 
quotients $\frac{w_1^\nu}{w_2^\nu}$ are not analytic in $D$.
To proceed from (*) 
Marshall 
considered the inner function
\[ 
\mathcal J=w_2^1\cdots w_2^N
\]
Then we can write
\[
\sum\, a_\nu\cdot \frac{w_1^\nu}{w_2^\nu}=
\sum\, a_\nu\cdot \frac{W_\nu}{\mathcal J}\quad\colon
W_\nu=w_1^\nu\cdot\prod_{j\neq\nu}\, w_2^j\tag{**}
\]
\medskip

\noindent
The next step in Marshall's proof is to 
apply a theorem by Nevanlinna which implies that
(*)  gives the existence of another inner
function
$W_*$ such that the function 
\[
g=\sum\, a_\nu\cdot \frac{W_\nu}{\mathcal J}+
\epsilon\cdot\frac{W_*}{\mathcal J}\quad\text{belongs to}\,\, H^\infty(T)\tag{***}
\]


\noindent
It follows that
$\bigl|f-g\bigr|<2\epsilon$ and  to finish the proof
of Theorem 0.2 there  remains only to approximate the 
special $H^\infty(T)$-function
$g$-function by a convex combination 
of inner functions. This  step in  the proof of Theorem 0.2 
is given  under the heading \emph{Marshall's Lemma} in section 4.
\medskip

\noindent
{\bf Remark.} Before we enter the proofs of
Theorem 0.1 and 0.2 we  expose some preliminaries
about inner functions and Blaschke products based upon results from
Chapter XX about  the Jensen-Nevanlinna class of
analytic functions in $D$.

\bigskip



\centerline {\bf 1. Blaschke products and inner functions}.
\medskip

\noindent
A sequence $\{z_n\}$ of non-complex numbers
in $D$ arranged so that
$0<|z_1|\leq|z_2|\leq ...$ 
satisfies
Blaschke's condition if
\[
\sum_{n=1}^\infty\, (1-|z_n|)<\infty
\]
Blaschke's theorem  asserts 
that the infinite product
\[
B(z)=\prod_{n=1}^\infty\,
\frac{z-z_n}{1-\bar z_n z}\cdot\frac{\bar z_n}{z_n}\tag{1}
\]
converges in $D$ and yields an analytic function such that
\medskip
\[
\lim_{r\to 1}\,
|B(re^{i\theta})|=1 \quad\text{holds  almost everywhere}\,\,
0\leq\theta\leq 2\pi
\]


\noindent
In particular every Blaschke product belongs to $\mathcal I(D)$.
Next, let
$\mu$ be a singular non-negative Riesz measure on the unit circle, i.e.
$\mu$ carries all mass on a null set in the sense of Lebesgue.
Then  there exists the analytic function in $D$ defined by
\[
G_{\vvv \mu}(z)= 
\text{exp}\bigl[\,-\frac{1}{2\pi}\cdot\int_0^{2\pi}\,
\frac{e^{i\theta}-z}{e^{i\theta}+z}\cdot d\mu(\theta)\,\bigr]\tag{2}
\]


\noindent 
The
Factorisation Theorem from XX gives:
\medskip



\noindent{\bf 1.1. Theorem}
\emph{Every $w\in\mathcal I(D)$ is a unique product}
\[ w(z)=e^{ic}\cdot B(z)\cdot G_{\vvv \mu}(z)\quad\colon\quad  c=\text{a real number}
\]
\emph{where $B(z)$ is the Blascke product formed by the zeros of
$w$ in $D$ and $\mu$  a singular and non-negative measure.}
\medskip

\noindent
We will also need Theorem  XXX from XXX
which characterizes when an inner function is a Blaschke product.
\medskip


\noindent 
{\bf 1.2 Theorem.} \emph{An inner function $w(z)$ is
of the form $e^{c}\cdot B(z)$ if and only if}
\[
\limsup_{r\to 1}
\,\int_0^{2\pi}\,
\log\,\bigl|w(re^{i\theta})\bigr|\cdot d\theta=0
\]
\bigskip







\centerline{ \bf 2. Proof of Frostman's theorem}
\bigskip


\noindent
Let $0<\rho<1$ and  $\gamma$ is a complex number
with $|\gamma|\leq 1$. Then one has the equality
\[
\frac{1}{2\pi}\int_0^{2\pi}\,
\log\,\bigl |\,\frac{\gamma-\rho e^{i\theta}}{1-\rho e^{-i\theta}\gamma}\,\bigr |
\cdot d\theta
=\text{max}\,(\rho,\log\,|\gamma|)\tag{*}
\]


\noindent
The verification is left to the reader.
Now we consider some
$w\in\mathcal I(D)$. Apply (*) with 
$\gamma=w(re^{it})$ for pairs $0<r<1$ and $0\leq t\leq2\pi$.
Integration with respect to $t$  gives the equality:


\[
\frac{1}{2\pi}\int_0^{2\pi}\bigl[\,\int_0^{2\pi}\,
\log|\,\frac{w(re^{it})-\rho e^{i\theta}}{1-\rho e^{-i\theta}w(re^{it})}\,|
\cdot d\theta\,\bigr ] dt
=
\int_0^{2\pi}\,\text{max}\,[\rho,\log\,|w(re^{it})|\,]\cdot dt\tag{*}
\]


\noindent
Since $w\in\mathcal I(D)$ we have
\[
\lim_{r\to 1}\log\,|w(re^{it})|=0
\quad\colon\quad\text{for almost all}\,\,\,t\tag{i}
\]


\noindent
Keeping  $0<\rho<1$  fixed the function
$t\mapsto\text{max}\,\bigl [\rho,\log\,|w(re^{it}|\,\bigr ]$ is bounded
so
by \emph{dominated convergence} under the intergral sign we have:
\[
\lim_{r\to 1}\,\int_0^{2\pi}\,
\text{max}\,[\rho,\log\,|w(re^{it}|)\,]\cdot dt=0\tag{ii}
\]


\noindent
Replace this limit by the double integral which comes from the equality (*)
and apply
\emph{Fubini's theorem} to
interchange the order of integration. Hence (ii)  gives:
\[
\lim_{r\to 1}\,\frac{1}{2\pi}\int_{\theta=0}^{\theta=2\pi}
[\,\int_{t=0}^{t=2\pi}\,
\log\,\bigl |\,\frac{w(re^{it})-\rho e^{i\theta}}{1-\rho e^{-i\theta}w(re^{it})}\,
\bigr |
\cdot dt\,] d\theta=0\tag{iii}
\]
\medskip

\noindent
Now we use that the  integrand
\[
\log\,|\,\frac{w(re^{it})-
\rho e^{i\theta}}{1-\rho e^{-i\theta}w(re^{it})}\,|\leq 0\quad\colon\quad
0\leq \theta,t\leq 2\pi\tag{iv}
\]
\medskip

\noindent
Then (iii) and  \emph{Fatou's theorem} gives:
\[
\limsup_{r\to 1}\,
\int_0^{2\pi}\,\log\,|
\,\frac{w(re^{it})-\rho e^{i\theta}}{1-\rho e^{-i\theta}w(re^{it})}\,|
\cdot dt\,]=0\quad\colon\text{almost everywhere for}\,\,\theta\tag{v}
\]


\noindent
At this stage the proof is almost finished. 
Namely, we notice that the functions
\[ 
F_\theta(z)=\,\frac{w(z)-\rho e^{i\theta}}{1-\rho e^{-i\theta}w(z)}
\]
belong to $\mathcal I(D)$ 
for every $\theta$. Moreover, (v) gives a null set $\mathcal N$
in $T$ such that
\[
\limsup_{r\to 1}\,
\int_0^{2\pi}\,
\log|F_\theta(re^{it})|\, dt=0
\quad\colon\quad \theta\in T\setminus\mathcal N\tag{vi}
\]


\noindent
Finally, for  every $\theta\in T\setminus\mathcal N$, Theorem 1.2 
implies that $F_\theta(z)=e^{ic_\theta}B_\theta(z)$ for
some Blaschke product $B_\theta$ and a constant $c_\theta$.
With such a choice of
$\theta$ we have
\[
|w(z)-e^{ic}\cdot B_\theta(z)|=
|\frac{\rho e^{i\theta}-\rho e^{-i\theta}(w(z))^2}{
1-\rho e^{-i\theta} w(z)}|\leq
\frac{2\rho}{1-\rho}\tag{vii}
\]
Here we can choose any $\rho<1$. So with  $\epsilon>0$
we choose $\rho$ so small that
$\frac{2\rho}{1-\rho}<\epsilon$ and 
Frostman's Theorem follows.


\bigskip


\centerline{\bf 3. A theorem by Nevanlinna}
\bigskip

\noindent
Theorem 3.1 below was proved by
R. Nevanlinna in his article [Nev:xx.] from 1919.
On the unit circle $T$ we have the Banach space
$L^\infty(T)$ which 
contains the 
closed  subspace $H_0^\infty(T)$
of boundary values from
bounded analytic functions $h(z)$ in $D$ which vanish
at $z=0$. 
Recall from XX that a 
bounded Lebesgue measurable function $f(e^{i\theta})$
belongs to $H_0^\infty(T)$ if and only if
\[
\int_0^{2\pi}\, e^{in\theta}\cdot f(e^{i\theta})d\theta=0
\quad\colon\quad n=0,1,2,\ldots\tag{0.1}
\]


\noindent
Since
$H_0^\infty(T)$ is a closed subset of $L^\infty(T)$
there exists the Banach space
\[
\mathcal B=\frac{L^\infty(T)}{H_0^\infty(T)}\tag{0.2}
\]


\noindent
For each $F\in L^\infty (T)$ we set:
\[ 
\mathfrak{nev}(F)=\min\,||F-h||_\infty\quad\colon\, h\in H_0^\infty(D)
\tag{0.3}
\]
\medskip
\noindent
We refer to $\mathfrak{nev}(F)$ as the Nevanlinna norm of $F$.
Since $\mathfrak{nev}(F)$
is the norm in the quotient space $\mathcal B$ it is trivial that
\[
\mathfrak{nev}(F)\leq |F|_\infty\tag{*}
\]


\noindent
When  strict inequality holds in (*) one has the following:
\medskip

\noindent
{\bf 3.1 Theorem}
\emph{Let $F\in L^\infty(T)$ be such that
$\mathfrak{nev}(F)< |F|_\infty$.
Then there exists  $h^*\in H^\infty(T)$ 
such that}


\[
|F(e^{i\theta})-h^*(e^{i\theta})|=|F|_\infty\quad\colon\quad
\text{almost everywhere on}\,\,T
\]
\medskip

\noindent
\emph{Proof.}
By a change of scale, i.e. replacing $F$ by
$F$ times the inverse of $|F|_\infty$ we may  assume that
$|F|_\infty=1$. Set
\[
\mathcal H_F=\{ h\in H^\infty(T)\quad\colon\,\,
|F-h|\leq 1\}\tag{i}
\]


\noindent
The triangle inequality gives $|h|_\infty\leq 2$ for every
$h\in\mathcal H_F$. So we have a uniform bound and hence
$\mathcal H_F$ is a normal family of analytic functions in
the unit disc $D$.
By the general result from Ch. 3-XX
there exists $h^*\in\mathcal H_F$ such that
\[ 
h^*(0)=\max_{h\in\mathcal H_F}\, |h(0|\tag{ii}
\]
\medskip

\noindent
Notice that $h^*(0)>0$. In fact, since
$\mathfrak{nev}(F)<1$ is assumed there exists to begin with
some $g\in H_0^\infty(T)$ with $|F-g|_\infty\leq 1-\delta$.
for some $\delta>0$. Hence $\mathcal H_F$ contains
the functions $g(z)+\delta$ which is $\neq 0$ at the origin.
\medskip


\noindent
\emph{Sublemma.} \emph{One has the equality}
\[
|F-h^*-\phi|_\infty\geq 1\quad\colon\quad \phi\in H_0^\infty(T)
\]


\noindent
\emph{Proof.}
Suppose there exists 
$\phi\in H_0^\infty(T)$
with $|F-h^*-\phi|_\infty=1-\delta$ for some
$\delta>0$. If $a>0$ it follows that
\[ 
|F-(1+a)h^*-\phi|\leq 1-\delta+a|h^*|_\infty\leq 1-\delta+2a
\]
So $h_1=(1+\delta/2)h^*+\phi$ belongs to $\mathcal H_F$.
Here $h_1(0)>h^*(0)$ which contradicts the maximality of
$h^*(0)$ and the Sublemma follows.




\bigskip


\noindent
\emph{Proof continued.} 
Put $G=F-h^*$.
The Sublemma means that
the norm of the $G$-image in the Banach space
$\mathcal B$ is  at least 1. At the same time the $L^\infty(T)$-norm of
$G$ is one since $h^*\in \mathcal H_F$.
It follows that the 
$\mathcal B$-norm of $G$ is 1.
Next, by the duality between
$H_0^\infty(T)$
and $H^1(T)$ from
 XX
$G$ yields a linear functional on the Hardy space $H^1(T)$ of norm
one. Hence there exists  a sequence $\{\phi_n\}$
in $H^1(T)$ with $L^1$-norms equal to one such that:
\[ 
\lim_{n\to\infty}\, \int_0^{2\pi}\,
G\cdot \phi_n\cdot d\theta=1\tag{1}
\]
Put $c_n=\phi_n(0)$.
Since $h^*\in H^\infty(T)$ we notice that
\[
\frac{1}{2\pi}\cdot \int_0^{2\pi}\,
h^*\cdot \phi_n\cdot d\theta=h^*(0)\cdot c_n\tag{2}
\]


\noindent
\emph{Sublemma 2. There exists some positive constant
$a$ such that}
\[ 
|c_n|\geq a\quad\colon\quad n=1,2,\ldots
\]


\noindent
\emph{Proof.}
Assume the contrary. If $c_n\to 0$ then (1) and (2)  give
\medskip
\[
\lim_{n\to\infty}\, \frac{1}{2\pi}\cdot\int_0^{2\pi}\,
F\cdot \phi_n\cdot d\theta=1\tag{3}
\]
But this is impossible since  $\mathfrak{nev}(F)<1$.
Indeed, this gives the existence of some
$\psi\in H^\infty(T)$ with $[F-\psi|_\infty=1-\delta$
for some $\delta>0$.
At the same time $c_n\to 0$ implies that
\[
\lim_{n\to\infty}\, \frac{1}{2\pi}\cdot\int_0^{2\pi}\,
\psi\cdot \phi_n\cdot d\theta=0
\]
Now we get a contradiction, i.e. (3) cannot hold since
\[
\frac{1}{2\pi}\cdot\int_0^{2\pi}\,
|F-\psi|\cdot \phi_n\cdot d\theta\leq |F-\psi|_\infty\cdot |\phi_n|_1=
1-\delta
\] 
\medskip

\noindent
{\emph{Proof continued.}}
Now we may assume that there is some
$a>0$ such that
\[
|\phi_n(0)|\geq a\quad\colon\quad n=1,2,\ldots\tag{*}
\]


\medskip

\noindent
At the same time (1) above holds and
we also know that
$|G|_\infty=1$.
Using  this we will show that 
\[
|G(e^{i\theta})|= 1\quad\colon\,\text{holds almost everywhere}\tag{**}
\]


\noindent
To prove (**) we use Jensen's inequality from XX
which by (*) for every $n$ gives:
\[
2\pi\cdot\text{Log}\,|a|\leq
\int_0^{2\pi}\, \text{Log}\, \bigl |\phi_n(e^{i\theta})\bigr |\cdot d\theta\tag{***}
\]


\noindent 
Suppose  that (**) does not hold. This gives
the existence of some
$\rho<1$ and a  set $E$ of 
positive Lesbesgue measure such that 
the maximum norm $|G|_E \leq \rho$.
Since $|G|_\infty=1$ it follows that
\[
\int_0^{2\pi}\,
G\cdot \phi_n\cdot d\theta
\leq \rho\cdot \int_E\,|\phi_n|\cdot d\theta
+\int_{T\setminus E}\,|\phi_n|\cdot d\theta
\]
By (1) above  the left hand side tends to one  and since
the $L^1$-norms of $\phi_n$ are all equal to one, we conclude that
\[
\lim_{n\to\infty}\,
\int_E\,|\phi_n|\cdot d\theta=0\tag{****}
\]
But this  contradicts  (1) by the Nevanlinna-Jensen theory in
XX. Namely, (****) first entails that
\[
\lim_{n\to\infty}\,
\int_E\,\text{Log}\,|\phi_n|\cdot d\theta=-\infty\tag{1}
\]
At the same time $\phi_n$ have finite $L^1$-norms
which by the material from XXX  gives a constant $C$ which is independent of
$n$ so that
\[
\int_0^{2\pi}\,\text{Log}^+\,|\phi_n|\cdot d\theta\leq C
\]
Then we see that (1) violates (***) and hence 
(**) must hold which finishes the proof of Nevanlinna's theorem.
\medskip

\noindent {\bf{Remark.}} Notice that the proof also shows how to find
$h^*$, i.e. it is  the extremal function in the family $\mathcal H_F$ 
which maximizes the value at $z=0$.




\bigskip






\noindent
{\bf 3.2 A consequence of Nevanlinna's Theorem.}
Let 
$g\in H^\infty(T)$. If 
$\mathcal J$ is an inner function then
$\frac{g}{\mathcal J}$
belongs to $L^\infty(T)$.
But in general this quotient does not belong to
$H^\infty(T)$.
To compensate for this we study its deviation from
$H^\infty(T)$ and obtain:
\bigskip

\noindent
{\bf 3.3 Theorem.} \emph{Let  $0<\epsilon<|g|_\infty$ be such that
there exists $f\in H^\infty(T)$
of norm one for which}
\[
\bigl |f-\frac{g}{\mathcal J}\bigr |_\infty=\epsilon\tag{*}
\]
\emph{Then there exists an inner function $w$ such that}
\[
\frac{\epsilon\cdot w+g}{\mathcal J}\in H^\infty(T)\tag{**}
\]
\medskip

\noindent
\emph{Proof.}
Nevanlinna's Theorem gives some
$h^*\in H^\infty(T)$
such that
\[ 
|\frac{g}{\mathcal J}-h^*|=\epsilon \quad\colon\,\, \text{almost everywhere}
\]
\medskip


\noindent
Since the inner function $\mathcal J$ has absolute value one
almost everywhere, it follows that
the analytic function $g-h^*\mathcal J$
has absolute value $\epsilon$ almost everywhere on
$T$.
This  this gives an inner function $w$ such that
\[
\mathcal J\cdot h^*-g=\epsilon\cdot w\tag{iii}
\]
After division with
$\mathcal J$ we get (**) in Theorem 3.3

\bigskip




\centerline{4. \bf{Marshall's Approximation Lemma}}

\bigskip
Consider  a  function $g\in H^\infty(T)$
expressed as
\[ 
g=\sum_{k=1}^{k=N}\, a_k\cdot \frac{w_k}{J}\tag{*}
\]
where $w_1,\ldots,w_N$ and $J$ 
are inner functions while 
$a_1,\ldots,a_N$ are some real numbers. Here we do not assume that
they are $\geq 0$ or that the sum is one. But we assume that
the maximum norm
$|g|_\infty<1$. 
\medskip

\noindent
{\bf 4.1 Proposition.} \emph{For every
$\epsilon>0$ there exists a convex
sum of inner
functions $V$ such that}
\[ 
\bigl|g-V\bigr|_\infty<\epsilon
\]



\noindent
\emph{Proof.}
For the inner functions $\{w\uuu k\}$ and
$J$
we have the equalities below on $T$:
\[
\frac{1}{w_k}=\bar w_k\quad\colon \frac{1}{J}=\bar J\tag{i}
\]
It follows that almost everywhere on $T$:
\[ 
\bar g(e^{i\theta})=\sum_{k=1}^{k=N}\, a_k\cdot \frac{J(e^{i\theta})}{ w_k
(e^{i\theta})}\tag{ii}
\]
Next, $W=w_1\cdots w_N$ is an inner function and (ii)
implies that the product 
\[
W\cdot \bar g\in H^\infty(T)\tag{iii}
\]


\noindent
Now we shall use (iii) to get
another expression for $g$.
By assumption
$|g|_\infty\leq 1-\delta$  for some
$\delta>0$.
Since $\int_0^{2\pi}\, e^{i\nu t}dt=0$
for every $\nu\geq 1$ we have the  equality below for every
complex number $\lambda$ of absolute value $< 1$:
\[
g(z)= \frac{1}{2\pi}\int_0^{2\pi}\,
\frac{\lambda e^{it}+g(z)}{1+\lambda e^{it}\bar g(z)}\cdot dt\tag{iv}
\]


\noindent
Next, for each $z\in D$ we can take
$\lambda=W(z)$ and obtain
\[
g(z)= \frac{1}{2\pi}\int_0^{2\pi}\,
\frac{W(z) e^{it}+g(z)}{1+e^{it}\cdot W(z)\cdot \bar g(z)}\cdot dt
\quad\colon\quad z\in D
\]
The family of functions
\[
t\mapsto \Phi\uuu t(z)=
\frac{W(z) e^{it}+g(z)}{1+e^{it}\cdot W(z)\cdot \bar g(z)}\quad
\]
is equi-continuous with respect to $t$ since
$|g|_\infty<1-\epsilon$ is assumed.
So we can evaluate the  integral which defines $g(z)$
by Riemann sums in a uniform
manner, i.e. to every $\epsilon>0$
there exists a finite set
$0\leq t_1<\ldots t_M<2\pi$
and positive numbers
$\{b_\nu\}$ with $\sum\, b_\nu=1$ such that
\[
\bigl |g(z)-\sum\, b_\nu\cdot 
\frac{W(z)e^{it_\nu}+g(z)}{1+e^{it_\nu}W(z)\cdot \bar g(z)}\,\bigr|\tag{v}
<\epsilon
\]


\noindent
Next, since $W(z)$ is inner we notice that the functions
\[
\phi_\nu(z)= 
\frac{W(z)e^{it_\nu}+g(z)}{1+e^{it_\nu}W(z)\cdot \bar g(z)}
\quad\colon1\leq\nu\leq M
\]
are all inner.
Since $\epsilon>0$ can be made arbitrary small it follows  that
$g(z)$ can be uniformly approximated by a convex combination of
inner functions.



\bigskip

\centerline{\bf 5. Density of unimodular functions}
\bigskip



\noindent
Let $\mathcal{U}(T)$
be the unimodular functions on $T$, i.e.
$L^\infty$-functions with absolute value one almost
everywhere.
Elementary geometry shows that the convex hull of
$\mathcal U(T)$ is dense in the unit ball of
$L^\infty(T)$, i.e. to every $G$ with
$|G|_\infty\leq 1$ there exist
$g_1,\ldots,g_M$ in $\mathcal{U}(T)$
such that
\[ 
|G-\sum a_\nu\cdot g_\nu|_\infty<\epsilon\quad\colon\, a_1+\ldots+a_M=1\tag{*}
\]
Since products of inner functions are inner, 
Theorem 0.3 follows if we  show that
every $g\in\mathcal{U}(T)$
can be uniformly approximated
by convex combinations of quotients of inner functions.
Again, since products of inner functions again are inner, it suffices to
achieve this approximation for a generating set in the
multiplicative group of  unimodular functions on $T$.
Hence it suffices to consider $g$-functions which only take two values. 
After a rotation
we may assume that they are +1 and -1. So now we have a measurable set
$E$ where $g=1$ on $E$ and $g=-1$ on $T\setminus E$.
Let $K$ be some positive number.
Solving the Dirichlet problem with a real valued function
$u$ which is K on $E$ and and 0 on $T\setminus E$
we take the bounded analytic function
\[ 
f(z)=e^{\vvv (u+iv)}
\]
where $v$ is the harmonic conjugate of $u$.
Here the absolute value $|f|=e^{\vvv K}$ on $E$ and $|f|=1$ on $T\setminus E$.
To be precise, equality holds almost everywhere in the
sense of Lebesgue.
Next, let $\epsilon>0$. Let ${\bf{C}}^e$
denote the extended complex plane. 
By the general result from XX
there exists for a suitable $K$
a conformal map
$\Phi$ from the annulus $1<|z|<e$
to the doubly connected domain
\[ 
\Omega={\bf{C}}^e\setminus [\vvv \epsilon,0]\cup
[\ell,\ell*]
\] 
where 
$\ell$ can be made large and $\ell^*\leq \ell+\epsilon$.
Here $\Phi(z_0)=\infty $  for some
$e^{\vvv K}<|z_0|<1$. Now we put
\[
\Psi=\frac{i+\Phi\circ f}{i\vvv\Phi\circ f}
\]
This yields a  meromorphic in $D$ with poles at the points $a\in D$ for which
$f(a)=z_0$.
On 
$T$ the composed function $\Phi\circ f$
takes values in the union of the real intervals
$[\vvv \epsilon,0]$ and $[\ell,\ell^*]$ which implies that
$|\Psi|=1$ holds almost everywhere on $T$.
With $\epsilon$ small we see that
$\Psi\simeq 1$ holds on
$E$ and when $\ell$ is large we have
$\Psi\simeq \vvv 1$ on $T\setminus E$.
So the $\Psi$\vvv function approximates the given unimodular 
function uniformly up to a small number of order $\epsilon$.
Now $\Psi$ may have some poles and w take the Blaschke product $B$ for this so that
$B\cdot \Psi$ is analytic in $D$ and since $|B|=1$ holds almost
everywhere this yields an inner function denoted by $\psi$.
Now
\[ 
\Psi=\frac{\psi}{B}
\]
is a quotient of inner functions which
gives the requested approximation.


\bigskip

\centerline{\bf{6. Proof of Marshall's theorem}}
\bigskip

\noindent
Consider some
$f\in H^\infty(T)$ of  norm $1$.
Theorem 0.3 gives a convex combination of quotients of
inner functions such that:

\[ 
\bigl|f-\sum_{\nu=1}^{\nu=k} a_\nu\cdot\frac{w_1^\nu}{w_2^\nu}\bigr|_\infty<\epsilon
\]
Set
\[ 
\mathcal J=\prod\, w_\nu^2
\]
This is an inner function and we also get the inner functions

\[ 
W_\nu= \frac{w_1^\nu}{w_2^\nu}\cdot\mathcal J\implies
\bigl|f-\sum_{\nu=1}^{\nu=k} a_\nu\cdot\frac{W_\nu}{\mathcal J}
\bigr|_\infty<\epsilon\tag{1}
\]


\medskip




\noindent
Theorem 3.3 applies to the pair $f$ and $g=\sum\, a_\nu W_\nu$
and gives an inner function $W_*$ such that 
\[ 
\sum_{\nu=1}^{\nu=k} a_\nu\cdot\frac{W_\nu}{\mathcal J}+\epsilon\cdot\frac {W_*}{\mathcal J}\in
H^\infty(T)\tag{2}
\]
Let $g$ be this analytic function. From (1) and the triangle inequality we have
\[
|f-g|_\infty<2\cdot \epsilon\tag{3}
\]

\noindent
Finally, with $N=k+1$ we can apply Marshall's lemma to
the function $\frac{1}{1+\epsilon}\cdot g$
which is expressed by a convex combination
(*) from (4).
This gives a convex sum $V$ formed by inner functions such that
\[ 
|V-g|_\infty<2\cdot\epsilon\tag{4}
\]
Since $\epsilon>0$ is arbitrary
we get Theorem 0.1 via (3) and (4) above.







\newpage

















%\end{document}