%\documentclass{amsart}

%\usepackage[applemac]{inputenc}

%\addtolength{\hoffset}{-12mm}
%\addtolength{\textwidth}{22mm}
%\addtolength{\voffset}{-10mm}
%\addtolength{\textheight}{20mm}
%\def\uuu{_}

%\def\vvv{-}






%\begin{document}

\centerline{\bf\large{8. Series and analytic functions.}}

\bigskip

\centerline{\emph{Contents}}

\bigskip


\noindent
\emph{1. A theorem by Kronecker.}
\bigskip


\noindent
\emph{2. Newton polynomials and the disc algebra.}

\bigskip
\noindent

\noindent
\emph{3. Absolutely convergent Fourier series.}

\bigskip



\noindent
\emph{4. Harald Bohr's inequality}
\bigskip


\noindent
\emph{5. Theorem of Fatou and M. Riesz}


\bigskip

\noindent
\emph{6. On Laplace transforms}
\bigskip


\noindent
\emph{7. The Kepler equation and Lagrange series}

\bigskip

\noindent
\emph{8. An example by Bernstein}


\bigskip

\noindent
\emph{9. Almost periodic functions and additive number theory}



\bigskip







\centerline{\bf 1. A theorem by Kronecker.}

\bigskip

\noindent
{\bf Introduction.} We seek
necessary and sufficient condition in order
that a sequence
$c_0,c_1,c_2,\ldots$ of complex numbers yield the coefficients in
the Taylor series at the origin of a rational function, i.e. that
\[ 
\sum\, c_\nu z^\nu=\frac{a_0+a_1z+\ldots+a_mz^m}
{b_0+b_1z+\ldots+b_nz^n}
\quad\text{where}\,\, b_0\neq 0\,,\tag{*}
\]
Here
$A(z)= a_0+a_1z+\ldots+a_mz^m$
and $B(z)=b_0+b_1z+\ldots+b_nz^n$ are polynomials and we
say that
$\{c_\nu\}$is of \emph{rational type} when
(*) holds.
A necessary condition for $\{c\uuu\nu\}$
to be of rational type follows via euclidian  division.
Namely, let $f(z)=\sum\, c\uuu\nu z^\nu$
and expand the product $f(z)\cdot B(z)$
into a power series. For each integer $M\geq n$
the coefficient of $z^M$ becomes
\[
c_{M\vvv n} b_n+c_{M\vvv n +1}b_{n-1}+\ldots+c_M\cdot b_0=0\tag{1}
\]
When (*) holds it follows that (1) is zero for every
$M\geq m+1$.
It means precisely that if $\lambda$ is an integer which is
$\geq\text{max}(0, n\vvv m+1)$
then
\[
c\uuu\lambda b_n+c_{\lambda +1}b_{n-1}+\ldots+c_{\lambda+n}\cdot b_0=0\tag{2}
\]
Here $(b\uuu n,\ldots, b\uuu 0)$ 
is a fixed non\vvv zero
$(n+1)$\vvv vector and hence (2) implies that the vectors

\[
(c\uuu\lambda +k,\ldots,c\uuu{\lambda+n+k})\quad\colon\quad 0\leq k\leq n
\]
are linearly dependent which entails that the determinant of the
following matrix must be zero:

\[
\begin{pmatrix}c_\lambda&\ldots&c_{\lambda+n}\\
c_{\lambda+1}&\ldots&c_{\lambda+1+n}\\
\\
\\
c_{\lambda+n}&\ldots&c_{\lambda+2n}\end{pmatrix}
\]
\medskip

\noindent
Kronecker proved that the vanishing of similar matrices
also yields a sufficient condition in order that
$\{c\uuu\nu\}$ is of rational type.
More precisely, for each pair of  integers $\lambda\geq 0$
and $\mu\geq 1$ we set

\[
C\uuu\lambda(\mu)=
\text{det}\, 
\begin{pmatrix}c_\lambda&\ldots&c_{\lambda+\mu}\\
c_{\lambda+1}&\ldots&c_{\lambda+1+\mu}\\
\\
\\
c_{\lambda+\mu}&\ldots&c_{\lambda+2\mu}\end{pmatrix}
\]
\bigskip








\noindent
{\bf 1.1  Theorem.} \emph{The sequence  $\{c_\nu\}$ is of rational type if 
if 
there exist a pair of integers $\lambda\uuu *\geq 0$ and
$\mu\uuu *\geq 1$
such that}

\[
C\uuu\lambda(\mu\uuu *)=0\quad
\text{for all}\quad \lambda\geq \lambda\uuu *\tag{*}
\]
\bigskip

\noindent
\emph{Proof.}
For each $\lambda\geq \lambda\uuu *$
we
consider the vectors
\[ 
\xi_\lambda=(c_\lambda,c_{\lambda+1}\ldots,c_{\lambda+\mu_*})\tag{i}
\]
If the family $\{\xi_\lambda\}_{\lambda\uuu *}^\infty$ span
${\bf{C}}^{\mu_*+1}$ we find the smallest
integer $w_*$
for which there exist 
\[
\lambda\uuu *\leq w_0<\ldots<w_{\mu_*-1}<w_{\mu\uuu *}\quad\text{and}\quad 
\xi_{w_0},\ldots,\xi_{w_{\mu_*-1}},\xi_{w_*}\quad
\text{are linearly independent}
\]

\medskip

\noindent
But this gives a contradiction because the vectors
$\{\xi\uuu{w\uuu *\vvv \mu\uuu *},\ldots,\xi\uuu {w\uuu *}\}$
appear as row vectors in the matrix
$C\uuu{w\uuu *\vvv \mu\uuu *}(\mu\uuu *)$
whose determinant by hypothesis is zero because
$w\uuu *\vvv \mu\uuu *\geq \lambda\uuu *$.




Notice that  $w_*\geq M_*$ must hold and
(ii) applied with $\lambda=w_*-M_*$
implies that
$\xi_{w_*}$ belongs to the linear  hull of the
vectors
$\xi_{w-1},\ldots,\xi_{w-M_*}$.
But this contradicts the minimal choice of $w_*$.
Hence the linear hull of the vectors $\{\xi_\lambda\}_0^\infty$
must be a proper subspace of $\neq {\bf{C}}^{M_*+1}$.
This gives a non-zero vector
$(b_0,\ldots,b_{M_*})$
such that

\[
c_\lambda\cdot b_0+\ldots+c_{\lambda+M_*}\cdot b_{M_*}=0
\quad\text{for all}\,\,\lambda\geq 0\,.\tag{iv}
\]
But these relations obviously imply that 
the sequence $\{c_\nu\}$ is of rational type and Kronecker's theorem is proved.
\bigskip


\medskip





\noindent
\emph{Sublemma.}
\emph{For each $\mu\geq 2$ and every $\lambda\geq 0$
one has the equality}
\[ 
C_\lambda(\mu)\cdot C_{\lambda+2}(\mu)-
C_\lambda(\mu+1)\cdot
C_{\lambda+2}(\mu+1)=C_{\lambda+1}(\mu)\cdot 
C_{\lambda+1}(\mu)\,.
\]


\noindent
\emph{Proof continued.}
Notice that the Kronecker matrix 
$\mathcal K_M=\mathcal C_0(M)$.
Assume that there exists $M_*$ such that
\[
\text{det}\,\mathcal K_M=0\quad\text{for all}\,\,M\geq M_*\tag{i}
\]
With the notations above (i) means that
$C_0(\nu)=0$ when
$\nu\geq M_*$.
With $\lambda=0$ in the Sublemma  we conclude that
$C_1(\nu)=0$ for all $\nu\geq M_*$. We can  proceed by an induction
over $\lambda$  which gives:

\[ 
C_\lambda(M_*)=0\quad\text{for all}\,\,\lambda\geq 0\,.\tag{ii}
\]

Let us then consider the $M_*+1$-vectors
\[ 
\xi_\lambda=(c_\lambda,c_{\lambda+1}\ldots,c_{\lambda+M_*})\quad\colon\, \lambda=0,1,\ldots
\]
The vanishing of the determinants in (i)
means that the $(M_*+1)$-tuple of vectors
\[
\xi_\lambda,\xi_{\lambda+1}\,\ldots,\xi_{\lambda+M_*}\tag{iii}
\]
are linearly dependent for every
$\lambda\geq 0$.
Suppose now that the family $\{\xi_\lambda\}_0^\infty$ span
${\bf{C}}^{M_*+1}$.
Choose the \emph{smallest} integer $w_*$
for which there exist 
\[
0\leq w_0<\ldots<w_{M_*-1}<w_*\quad\text{and}\quad 
\xi_{w_0},\ldots,\xi_{w_{M_*-1}},\xi_{w_*}\quad
\text{are linearly independent}
\]
Notice that  $w_*\geq M_*$ must hold and
(ii) applied with $\lambda=w_*-M_*$
implies that
$\xi_{w_*}$ belongs to the linear  hull of the
vectors
$\xi_{w-1},\ldots,\xi_{w-M_*}$.
But this contradicts the minimal choice of $w_*$.
Hence the linear hull of the vectors $\{\xi_\lambda\}_0^\infty$
must be a proper subspace of $\neq {\bf{C}}^{M_*+1}$.
This gives a non-zero vector
$(b_0,\ldots,b_{M_*})$
such that

\[
c_\lambda\cdot b_0+\ldots+c_{\lambda+M_*}\cdot b_{M_*}=0
\quad\text{for all}\,\,\lambda\geq 0\,.\tag{iv}
\]
But these relations obviously imply that 
the sequence $\{c_\nu\}$ is of rational type and Kronecker's theorem is proved.
\bigskip

\noindent
{\bf Remark.}
Kronecker's theorem can be used to establish 
conditions in order that a meromorphic function
is rational.
One has for example the following result which is due to Polya in [Pol]:
\medskip

\noindent
{\bf 1.2 Theorem.}
\emph{Let $\{c_n\}$ be a sequence of integers.
Suppose that the power series}
\[
f(z)=\sum\,c_n\cdot z^n
\]
\emph{converges in some open disc centered at the origin and that
$f(z)$ extends to an analytic function in
a simply connected domain $\Omega$.
whose mapping radius with respect to $z=0$ is
strictly greater than one Then
$f(z)$ is a rational function.}
\medskip

\noindent
{\bf{Remark.}} For the definition and various results about
the \emph{mapping radius} of simply connected domains the
reader may consult Chapter X in [Po-Szeg�] where
other results based upon
Kronecker's theorem appear.


\newpage



\centerline {\bf {II. Newton polynomials and
the disc algebra $A(D)$}}


\bigskip


\noindent
{\bf Introduction.}
Let  $A(D)$ be the disc algebra.
If $f(z)\in A(D)$ then its
Taylor series a $z=0$ give the partial sum polynomials
$\{s_n^f(z)\}$.
Denote by $A^ *(D)$
the unit ball, i.e.
funtions $f$ with maximum norm
$|f|_D\leq 1$ and set
\[
\mathcal M_n=
\max_{f\in A *(D)}\, \,|s^f_n|_D\tag{*}
\]
We are going to determine  these $\mathcal M$-numbers.
In his   text-books from 1666, Isaac Newton
studied  the funcion
$\sqrt{1-z}$ whose
series expansion becomes:
\[
\sqrt{1-z}= q_0+q_1z\ldots\quad\colon\,\quad
q_n=\frac{1\cdot 3\dots(2n-1)}{ 2\cdot 4\cdots 2n}\tag{1}
\]
Notice that
these positive coefficients decrease, i.e.
\[ 
1=q_0>q_1>q_2>\ldots\tag{2}
\]
To each $n\geq 1$ we get the Newton polynomial
\[ 
Q_n(z)= q_0+q_1z+\ldots+q_nz^n\tag{3}
\]
By (2)  and Kakeya's  result from XXX,
$Q_n(z)$ has no zeros in the closed unit disc. Put
\[
\mathcal G_n=1+q_1^2+\ldots+q_n^2\tag{4}
\]


\medskip

\noindent
{\bf 1. Theorem.}
\emph{For each integer $n\geq 1$ one has the equality
$\mathcal M_n=\mathcal G_n$ and the maximum in (1) is attained by
the $A^*(D)$-function}
\[ 
f^*_n(z)=\frac{z^n\cdot Q_n(\frac{1}{z})}{Q_n(z)}\tag{2}
\]
\medskip

\noindent
{\bf 2. Remark.}
Using \emph{Stirling's formula} one can easily show that
\[ 
\lim_{n\to\infty}\,\frac{\mathcal G_n}{\log\,  n}=\frac{1}{\pi}
\]
\medskip


\noindent
Before Theorem 1 is proved we
need some preliminary observations about
partial sum functions.
Let $f\in A_*(D)$.
Cauchy's formula gives


\[ 
s^f_n(1)=\frac{1}{2\pi i}\cdot
\int_{|z|=1}\,
\frac{f(z)}{z^{n+1}}\cdot(1+z+\ldots+z^n)\cdot dz\quad\colon\, n=0,1,\ldots
\tag {i}
\]
Since $\int_{|z|=1}\,f(z)z^kdz=0$ for every $k\geq 0$
we see that if
$Q(z)$ is any polynomial of the form
\[
 Q(z)=1+z+\ldots+z^n+q_{n+1}z^{n+1}+\ldots\implies\tag{ii}
\]
\[
s^f_n(1)=\frac{1}{2\pi i}\cdot
\int_{|z|=1}\,
\frac{f(z)}{z^{n+1}}\cdot Q(z)\cdot dx\tag{iii}
\]
\medskip


 
\noindent
{\emph{ Proof of Theorem 1.}}
For each  $n\geq 1$ 
the squared Newton polynomial $Q_n^2(z)$ satisfies (ii) above.
So if $f\in A*(D)$ we have
\[ 
s^f_n(1)=
\frac{1}{2\pi i}\cdot
\int_{|z|=1}\,
\frac{f(z)}{z^{n+1}}\cdot Q\uuu n^2(z)\cdot dx\tag{iv}
\]
Since the maximum norm of $|f|_D\leq 1$,
the triangle inequality gives:
\[ 
|s^f_n(1)| \leq\frac{1}{2\pi} \cdot\int_0^{2\pi}
\bigl |Q\uuu n(e^{i\theta})\bigr |^2\cdot d\theta\tag{v}
\]

\noindent
By  \emph{Parseval's formula}  the last integral is equal to $\mathcal G_n$.
Hence (v) gives the inequality
\[ 
\mathcal M_n\leq\mathcal G_n\tag{vi}
\]
\medskip
Next, with $n$ kept fixed we consider the function
\[
f^*(z)=\frac{z^n\cdot Q_n(\frac{1}{z})}{Q_n(z)}\implies\tag{vii}
\]
\[ 
s^{f^*}_n(1)=
\frac{1}{2\pi i}\cdot
\int_{|z|=1}\,
\frac{f^*_n(z)}{z^{n+1}}\cdot Q\uuu n^2(z)\cdot dx\tag{viii}
\]
where $\implies$ follows from (iii) above. Notice that
\[
\frac{f^*_n(z)}{z^{n+1}}\cdot Q^2(z)=\frac{1}{z}Q_n(z)\cdot
Q_n(\frac{1}{z})\implies\tag{ix}
\]
\[ 
s^{f^*}_n(1)=
\frac{1}{2\pi}\cdot\int_0^{2\pi}\,
Q_n(e^{i\theta})\cdot Q(e^{-i\theta})\cdot d\theta=
\frac{1}{2\pi}\cdot\int_0^{2\pi}\,
\bigl |Q_n(e^{i\theta})\bigr |^2\cdot d\theta=\mathcal G_n\tag{5}
\]
Since $f^*\in A^ *(D)$
we conclude that (vi) above is an equality and Theorem 1 is proved.




\bigskip

\centerline{\bf 3. Convergence of  Fourier series}

\bigskip

\noindent
Let $f(z)=\sum\, c_nz^n$ be in the disc algebra $A(D)$.
With $c_n=a_n+ib_n$ and $f=u+iv$
we get  series for the real and the imaginary part respectively:
\[ 
u(e^{i\theta})=
\sum_{n=0}^{n=N}\, a_n\cdot\text{cos}\, n\theta-
\sum_{n=0}^N\,  b_n\cdot\text{sin}\, n\theta
\]
\[ 
v(e^{i\theta})=\sum_{n=-N}^{n=N}\, a_n\cdot\text{sin}\, n\theta+
\sum_{n=-N}^N\,  b_n\cdot\text{cos}\, n\theta
\]


\noindent
Continuous boundary values of $f$  certainly exist if
\[ 
\sum\,|a_n|+|b_n|<\infty\tag{*}
\]


\noindent
We shall give a sufficient condition for the validity of
(*) expressed by the modulos of continuity of
$u$:
\[
\omega_u(\delta)=\max\,|u(e^{i\theta})-u(e^{i\phi})|\quad\colon
\,\,\text{maximum over pairs}\,\,\,\,|\theta-\phi|\leq\delta
\]



\noindent
{\bf 1. Theorem.} \emph{The series (*) is convergent if}
\[
\sum_{n=1}^\infty\, \frac{1}{\sqrt{n}}\cdot\omega_u(\frac{1}{n})<\infty
\]
\medskip


\noindent
Theorem 1 is due to S. Bernstein in [1].
One may ask for other convergence criteria.
Denote by
$\mathcal F$ the class of continuous functions
$F(s)$  defined for $s\geq 0$ which are
increasing and concave and $F(0)=0$ while $F(s)>0$ for
every $s>0$. Then  the following result 
is in 
[Salem: P. 39]:
\medskip


\noindent
{\bf 2. Theorem.}
\emph{Let $F\in \mathcal F$ be such that}
\[ \sum_{n=1}^\infty\, F\bigl (\frac{1}{n}\cdot\omega_u^2(\frac{1}{n})\,\bigr )
<\infty\tag{*}
\]
\emph{Then it follows that}
\[
\sum_{n=1}^\infty\, F(a_n^2+b_n^2)<\infty\tag{**}
\]



\medskip

\noindent
Notice that Bernstein's theorem is the case
$F(s)=\sqrt{s}$.
The proof of Theorem 2 relies upon a
result due to La Vall�e Poussin who established a
{\emph{lower bound} for the
modulus of continuity.
Recall first the general $L^2$-equality:
\[ 
2a^2_0+\sum_{n\geq 1}\,a_n^2+b_n^2=\frac{1}{\pi}\int_0^{2\pi}\, u(e^{i\theta})^2
\cdot d\theta\tag{*}
\]
Now we consider the tail sums
\[
V_N=\sum_{n=N+1}^\infty\, (a_n^2+b_n^2)
\]

\medskip


\noindent
{\bf  3. Theorem.}
\emph{For every real valued and continuous function $u$ on the unit circle one has}
\[ 
\omega_u^2(\frac{1}{N})>\frac{1}{72}\cdot V_N
\quad\colon\quad N=1,2,\ldots
\]

\medskip

\noindent
{\bf Remark.} See 
Vall�e Poussin's   text-book
\emph{Lecons sur l'approximation des fonctions
d'une variable r�elle}  for the proof.


\bigskip

\noindent
\emph{Proof of Theorem 2.}
Put $\rho_n^2=a_n^2+b_n^2$.
Since $F(s)$ is concave and increasing we have for every $N\geq 1$:
\[
\frac{1}{N}\sum_{n=N+1}^{n=2N}\,
F(\rho_n^2)\leq F(\frac{1}{N}\sum_{n=N+1}^{n=2N}\rho_n^2)
< 72\cdot F\bigl (\frac{1}{N} \omega^2(\frac{1}{N})\bigr )\tag{i}
\]
where the last inequality follows from Theorem 3 above.
Apply (i) with $N=2^k$ as $k=0,1,\ldots$.
Then (i) obviously gives the implication:
\[
\sum_{k=0}^\infty
2^k\cdot F\bigl (2^{-k}\cdot  \omega^2(2^{-k})\bigr )<\infty\implies
\sum_{n=1}^\infty\, F(\rho_n^2)<\infty\tag{ii}
\]
Now we are almost done. Namely, 
the sequence $2^{-k}\cdot  \omega^2(2^{-k})$ decreases with $k$
and since  $F$ increases we have
\[
\sum_{n=2^{k-1}+1}^{2^k}
F(\frac{1}{n}\omega^2(\frac{1}{n}))\geq
2^k\cdot F(2^{-k}\cdot \omega^2(2^{-k}))\quad\colon\,\,k=1,2,\ldots
\]
Hence Salem's convergence condition (*)
in Theorem 2 gives 
(ii) above and the proof is finished.



\bigskip


\centerline
{\bf  4. Harald Bohr's inequality.}
\bigskip

\noindent
Let $A^*(D)$ be the unit ball in $A(D)$.
When $f\in A^*(D)$ and  $0<r<1$ we set
\[
\mathfrak M_f(r)=\sum\, |a_n|r^n\tag{*}
\]
The question arises for which $r$ it holds that
\[
\mathfrak M_f(r)\leq 1\quad\colon\,\forall\, f\in A^*(D)\tag{**}
\]


\noindent
{\bf 1. Theorem.} \emph{One has (**) if and only if $r\leq \frac{1}{3}$.}
\bigskip

\noindent
\emph{Proof.}
Given $f$ in $A_*(D)$ we set
\[ 
\phi(z)=\frac{f(z)-a_0}{1-\bar a_0\cdot f(z)}\tag{i}
\]
Then the maximum norm 
$|\phi|_D\leq 1$ and its
derivative at $z=0$ becomes
\[
\phi'(0)=\frac{f'(0)}{1-|a_0|^2}\tag{ii}
\]
Since $g(0)=0$ we have $|g'(0)|\leq 1$
by the inequality of Schwarz.
It follows that
\[ 
|a_1|\leq (1-|a_0|^2)
\leq
2\cdot [1-|a_0|)\tag{iii}
\]
where the last inequality holds since
$1+|a_0|\leq 2$.
Next, put  
$\rho= e^{2\pi i/n}$ for every  $n\geq 2$
and regard the function
\[
F_n(z)=\frac{f(z)+f(\rho z)+\ldots+f(\rho^{n-1}z)}{n}\tag{*}
\]


\noindent
Since $1+1+\rho^\nu+\ldots+\rho^{\nu(n-1)}=0$
whenever $\nu$ is not a multiple of $n$ we conclude that
\[
F_n(z)=
a_0+a_nz^n+a_{2n}z^{2n}+\ldots\tag{**}
\]


\noindent
Notice that the maximum norm
$|F_n(z)|_D\leq 1$. 
Now we regard the
analytic function

\[ 
g(z)=a_0+a_n z+a_{2n}z^{2}+\ldots
\]
\medskip

\noindent
Since $|F_n(z)|_D\leq 1$ it is clear that we also have
$|g|_D\leq 1$.
Then (*) applied to $g$ gives
\[
|a_n|\leq 2(1-|a_0|)\quad\colon\, n\geq 1\tag{***}
\]
Armed with this  we  can finish
the proof of Theorem 1.
First we show
\bigskip

\noindent
{\bf 2. The inequality $\mathfrak {M}_f(\frac{1}{3})\leq 1$}.
From (***) we obtain

\[ 
\mathfrak {M}_f(\frac{1}{3})=|a_0|+\sum_{n=1}^\infty\, 3^{-n}|\cdot|a_n|\leq
|a_0|+\sum_{n=1}^\infty\,3^{-n}\cdot 2(1-|a_0|)=1
\]
\medskip

\noindent
There remains to prove that the upper bound
$\frac{1}{3}$ is sharp in Theorem 1.
To see this we take a real number
$0<a<1$ and consider the M�bius function
\[ 
f(z)=\frac{z-a}{1-az}=
-a+(1-a^2)z+(a-a^2)z^2+(a^2-a^3)z^3+\ldots\implies
\]
\[ 
\mathfrak {M}_f(r)=
+(1-a^2)r+(a-a^2)r? 2+\ldots=a+\frac{(1-a^2)r}{1-ar}
\]
\medskip

\noindent
The last term is $\leq 1$  if and only if
\[
a(1-ar)+(1-a^2)r\leq 1-ar\implies
(1+a-2a^2)r\leq 1-a
\]
With $a=1-s$ and $s>0$ small this gives
\[
s+2s-2s^2)r\leq s\implies r\leq\frac{1}{3}+2s
\]
Since $s$ can be arbitrary small the upper bound
$\frac{1}{3}$ in Theorem 1 is best possible.



\bigskip


\centerline{\bf 5. A theorem by  Fatou and M. Riesz.}
\bigskip

\noindent
{\bf Introduction.} We prove a result 
due to Fatou and M. Riesz. See
the 
article [M. Rie] from  1911.
Let
\[ 
f(z)=\sum\, c_nz^n\tag{1}
\]
be an analytic function in  the open unit disc.
We shall consider the situation
when
$f$ extends analytically along some arc of the unit circle.
For example, the analytic function  $\frac{1}{1-z}$ extends 
analytically outside
the boundary point $z=1$ and
the series
\[ 
\sum\, e^{in\theta}
\]
converges for all $0<\theta<2\pi$.
Let us now consider
some $f\in\mathcal O(D)$  for which
there exists some $0<\theta^*<\pi/2$ such that
$f$ extends to an analytic function in  the union of
$D$ and the sector
\[ 
S=
\{ z=re^{i\theta}\quad\colon\, 1\leq r<R\quad\colon
-\theta^*<\theta< \theta^*\}\tag{ii}
\]
Moreover, we suppose that
$f$ extends to a continuous function
on the closed union of $D\cup S$. See figure XXX.
With these notations one has

\bigskip

\noindent
{\bf 1. Theorem.}
\emph{Assume that $c_n\to 0$. Then the 
partial sums $\{s_n(e^{i\theta})\}$
converge uniformly to $f(e^{i\theta})$
one every compact  interval of $(-\theta^*,\theta^*)$.}
\bigskip

\noindent
\emph{Proof.}
To each $n\geq 1$ we consider the function
\[ 
g_n(z)=\frac{f(z)-(c_0+c_1z+\ldots+c_nz^n)}
{z^{n+1}}\cdot
(z+e^{i\theta^*})(
(z-e^{i\theta^*})\tag{i}
\]
\medskip

\noindent
This is an analytic function in the domain $D\cup S$. Consider a
closed  circular interval $\ell=[-\theta_*\leq\theta\leq\theta_*]$
for some $0<\theta_*<\theta^*$.
It appears as a compact subset of $S\,\cup D$ and
it is clear that
the required uniform convergence of $\{s_n\}$ holds
on $\ell$
if
the $g$-functions converge uniformly to zero on $\ell$,. In fact, this
follows since
the absolute values  

\[
|(z+e^{i\theta^*}|\cdot 
|z-e^{i\theta^*}|\geq (\theta^*-\theta_*)^2\quad\colon\quad z\in\ell
\]
To prove that the maximum norms $|g_n|_\ell\to 0$ it suffices by the 
maximum principle for analytic
functions to show that
$g_n$ converges uniformly to zero on
the boundary of the sector $S$ which by the construction contains
$\ell$.
Here $\partial S$ contains  the outer circular arc
\[
\Gamma=\{ Re^{\i\theta}\quad\colon\, 
\-\theta_*\leq\theta\leq\theta_*\tag{i}
\]
In addition $\partial S$ contains  two rays. Let us regard the two pieces
of
$\partial S$ given by
\[
\Gamma_*=\{ z=re^{i\theta^*}\,\colon\, 0\leq r\leq 1\}\quad\colon
\quad\Gamma^*=
\{ z=re^{i\theta^*}\,\colon\, 1\leq r\leq R\}\tag{ii}
\]


\noindent
There remains to estimate the maximum norms of $g_n$
over each of these pieces of $\partial S$. Of course, in addition to
(ii) we have the contribution when
$z=re^{-i\theta^*}$ but by symmetry the subsequent estimates 
are valid here too. Before we establish the required estimates we introduce
a notation. To each integer $m\geq 1$ we set:
\[
A_m=M+|c_0|+|c_1|\cdot R+\ldots+
|c_m|\cdot R^m\quad\colon\,\epsilon_m=\max_{\nu>m}\, |c_\nu|
\tag{1}
\]

\noindent
By hypothesis we have
\[ 
\lim_{m\to\infty}\,\epsilon_m=0\tag{2}
\]

\medskip

\noindent
{\bf 2. The estimate of $|g_n|_\Gamma$.}
By assumption $f$ extends continuously to the closure of
$S$ so the maximum norm
$|f|_S =M$ is finite. 
If $z\in\Gamma$ we have $|z|=R$ and the triangle inequality
gives for each pair $1\leq M<N$:
\medskip

\[
\bigl|f(z)-(c_0+c_1z+\ldots+c_nz^n)\bigr|\leq
A_m+\epsilon_m(R^{m+1}+\ldots+R^n)\leq 
A_m+\epsilon_m\cdot
\frac{R^{n+1}}{R-1}\tag{i}
\]
\medskip

\noindent
With the constant
\[
K=\max_{z\in\Gamma}\, |z-e^{i\theta^*}|\cdot |z+e^{i\theta^*}|
\] 
we therefore obtain
\[ |g_n|_\Gamma\leq
\frac{K}{R^{n+1}}\cdot \bigl(A_m+\epsilon_m\cdot
\frac{R^{n+1}}{R-1}\bigr)=\frac{K\cdot A_m}{R^{n+1}}+
\frac{K\cdot\epsilon_m}{R_1}
\tag{ii}
\]
If $\delta>0$ we use (2) above and find $m$ so that
$\epsilon_m<\delta$. Once $m$ is fixed
we use that $R>1$ and hence
$\frac{K\cdot A_m}{R^{n+1}}<\delta$ if $n$ is large. Since 
$\delta>0$ is arbitrary
we conlcude that
$|g_n|_\Gamma\to 0$ as required.
\bigskip

\noindent
{\bf 3. Estimate of $|g_n|_{\Gamma^*}$}.
With the same notations as above we  consider some
$z=re^{i\theta^*}$ with $1<r<R$ and obtain:
\[
\bigl|f(z)-(c_0+c_1z+\ldots+c_nz^n)\bigr|\leq
\]
\[
A_m+\epsilon_m(r^{m+1}+\ldots+r^n)\leq 
A_m+\epsilon_m\cdot
\frac{r^{n+1}}{r-1}\implies
\]
\[ 
|g_n(re^{i\theta}|\leq
\bigl(A_m+\epsilon_m\cdot
\frac{r^{n+1}}{r-1}\bigr)
\cdot\frac{1}{r^{n+1}}\cdot
|re^{i\theta^*}-e^{i\theta^*}|\cdot 
|re^{i\theta^*}-e^{-i\theta^*}|\tag{i}
\]
\medskip

\noindent 
Here
$|re^{i\theta^*}-e^{i\theta^*}|=r-1$
and $|re^{i\theta^*}-e^{-i\theta^*}|\leq 2R$ for all $1\leq r\leq R$.
So with $K=(R^2-1)$ we see that (i) gives
\[
|g_n(re^{i\theta}|\leq\frac{K\cdot A_m}
{r^{n+1}}\cdot(r-1)+
R\cdot\epsilon_m\tag{ii}
\]
At this stage we use the  obvious inequality for $r>1$:
\[
\frac{r-1}{r^{n+1}}<\frac{r-1}{r^{n+1}-1}=\frac{1}{
1+r+\ldots+r^n}<\frac{1}{n}\implies
\]
\[ 
|g_n|_{\Gamma_*}\leq \frac{KA_m}{n}+2R\cdot\epsilon_m
\]
Since $\epsilon_m\to 0$ the reader concludes that
$|g_n|_{\Gamma_*}\to 0$ as $n\to\infty$.
\bigskip

\noindent 
{\bf Estimate of $|g_n|_{\Gamma_*}$}.
With $x=re^{i\theta^*}$ and $0<r<1$ the triangle inequality gives
\[
\bigl|f(z)-(c_0+c_1z+\ldots+c_nz^n)\bigr|\leq
|c_{n+1}
|\cdot |z|^{n+1}+
|c_{n+2}|\cdot |z|^{n+2}+\tag{i}
\ldots
\]
Recall that
$\epsilon_n=\max_{\nu>n}\, |c_\nu|$. Hence (i) is majorized by
\[
\epsilon_n \cdot\bigl( |z|^{n+1}+|z|^{n+2}+\ldots\bigr)
=\epsilon_n \cdot 
\frac{|z|^{n+1}}{1-|z|}
\]
Now $z=re^{i\theta^*}$ and we get as before with
$K=\max|re^{i\theta^*}-e^{-i\theta^*}|$:
\[
|g_n(z)|\leq
\frac {\epsilon_n \cdot 
\frac{|z|^{n+1}}{1-|z|}}{|z|^{n-1}}\cdot (1-|z|)\cdot K=
K\cdot \epsilon_n
\]
Again, since $\epsilon_m$ can be chosen arbitrary small
we conclude that
$|g_n|_{\Gamma_*}\to 0$ as $n\to\infty$. 
This finishes the proof of the Theorem 1.







\bigskip



\centerline{\bf 6. On Laplace transforms.}
\bigskip

\noindent
Let $f(t)$ be a bounded function defined on the
real $t$-line.
Consider its Laplace transform
\[
L(z)=\int_0^\infty\, f(t)e^{-zt}\cdot dt
\]
which is
analytic in the open half-plane
$\mathfrak{Re}\, z>0$.
\emph{Assume} that there exists some open
subset $\Omega$ of
${\bf{C}}$ which contains the closed
half-plane
$\mathfrak{Re}\, z\geq 0$ such that $L(z)$ extends to an analytic
function in $\Omega$.
Under this assumption one has
\medskip

\noindent
{\bf 1. Theorem.}
\emph{There exists the limit}
\[
\lim_{T\to\infty}\, \int_0^T\, f(t)\cdot dt
\]
\emph{Moreover, the limit value is equal to $L(0)$ where $L$ under its analytic extension to
$\Omega$ has been evaluated at $z=0$.}
\bigskip

\noindent
\emph{Proof.}
To each $T>0$ we have the entire function
\[ 
L_T(z)=\int_0^T\, f(t)e^{-zt}\cdot dt\tag{i}
\]
Theorem 1 amounts to prove that
\[ 
\lim_{T\to\infty}\, L_T(0)=L(0)\tag{ii}
\]


\noindent
To prove (ii) we consider  certain complex line integrals.
If $R>0$ the assumption on $L$ gives some
$\delta>0$ such that
$\Omega$ contains
the closed set given by the union
of the half disc
$\bar D_R^+=\bar D_R\cap\,\mathfrak{Re}\, z\geq 0$
and the rectangle
\[ 
\square=\{x+iy\quad\colon\, -\delta\leq x\leq 0\quad\colon
-R\leq y\leq R\}
\]
Let $\Gamma$ be the boundary
of
$\bar D_R^+\cup\,\square$ and
introduce the function
\[
g(z)=
\bigl[L(z)-L_T(z)\bigr ]\cdot(1+\frac{z^2}{R^2})\cdot \frac{e^{zT}}{z}\tag{iii}
\]


\noindent
Here $g(z)$ is a meromorphic function in
$\Omega$ with a simple pole at $z=0$
whose residue is $L(0)-L_T(0)$.
Hence residue calculus applied to $g$ and $\square$ gives:
\[
L(0)-L_T(0)=\frac{1}{2\pi i}\cdot
\int_\Gamma\,g(z)\cdot dz\tag{iv}
\]


\noindent
Put $B=\max_t\, |f(t)|$ which for every $T>0$ gives
the inequality:
\[ 
|L(z)-L_T(z)|
=\bigl|\int_T^\infty\, f(t)e^{-zt}dt\,\bigr |\leq 
\]
\[
B\cdot\int_T^\infty\, \bigl| e^{-zT}\bigr |\cdot dt=
B\cdot\int_T^\infty\,  e^{-\mathfrak{Re}\,z\cdot T}\cdot dt=B\cdot 
\frac{e^{-\mathfrak{Re}(z)\dot T}}{
\mathfrak{Re}(z)}\quad\colon\quad \mathfrak{Re}(z)>0\tag{v}
\]
\medskip

\noindent
Now we begin to estimate the line integral in (iv).
Consider first the part of $\Gamma$ given by
the half circle $\partial D_R^+$.
Here we notice that
\[
|1+\frac{R^2e^{2i\theta}}{R^2}|=|1+e^{2i\theta}|=2\cdot\text{cos}\,\theta\tag{vi}
\]
Next, $\frac{dz}{z}= R\cdot d\theta$ holds during the integration on
$\partial D_R^+$ and we also
have
\[
\frac{1}{\mathfrak{Re}(R\cdot  e^{i\theta})}=\frac{1}{R\cdot\text{cos}\,\theta}
\]
Hence  (v) and (vi)  give
\[ 
\bigl|\int_{\partial D^+R}\, g(z)\cdot \frac{dz}{z}\bigr|
\leq 2\cdot B\cdot \int_{-\pi/2}^{\pi/2}\,\frac{\text{cos}\,\theta}{
R\cdot\text{cos}\,\theta}\cdot d\theta
\leq 2\cdot B\cdot \frac{\pi}{R}\tag{*}
\]


\noindent
There remains to estimate the integral over
the part of $\Gamma$ which belongs to
$\partial\square$.
Here we simply perform estimates for the two functions
$L(z)$ and $L_T(z)$ separately.
First, since $L_T(z)$ is entire we can
just as well integrate over the
half-circle $D_R^-$ where
$\mathfrak{Re}(z)<0$.
We notice that
\[
|L_T(z)|\leq B\int_0^T\, e^{-\mathfrak{Re}\, z\cdot t}\cdot dt
\leq B\cdot \frac{e^{-\mathfrak{Re}\, z\cdot T}}{\bigl|\mathfrak{Re}\, z\bigr|}
\quad\colon\quad \mathfrak{Re}\, z<0
\]
Here $e^{-\mathfrak{Re}\, z\cdot T}$ is large when
$z\in D_R^-$ but this factor is cancelled by
the absolute value of $e^{zT}$ which appears in the $g$-function. Hence
we obtain
\[
\bigl|\int_{D_R^-}\, L_T(z)\cdot(1+\frac{z^2}{R^2})
\cdot e^{zT}\cdot \frac{dz}{z}\bigr|
\leq
B\cdot
\int_{\pi/2}^{3\pi/2}\, \frac{\bigl|1+e^{2i\theta}\bigr|}{
R\cdot\bigl|\text{cos}\,\theta\,\bigr |}\cdot d\theta\leq\frac {2\pi\cdot B}{R}\tag{**}
\]


\noindent
Finally,  consider the line integral
along $\Gamma\cap\partial\square$  where 
the analytic function $L(z)$ appears. First we regard the line integral along 
the vertical line where
$\mathfrak{Re}(z)=-\delta$ whose
absolute value becomes:
\medskip

\[
\bigl|\int_R^R\, L(-\delta+iy)\cdot(1+\frac{(-\delta+iy)^2}{R^2})\cdot 
e^{-\delta T}
\cdot e^{iyT}\cdot  \frac{i\cdot dy}{(-\delta+iy}\bigr|\tag{vii}
\]
Notice that we have not imposed any growth condition
Here 
$e^{-\delta T}$ appears and at the same time
$e^{iyT}$ has absolute value one. Hence (vii)
is estimated by
\[
\max_{-R\leq y\leq R}\,
\bigl |\frac{L(-\delta+iy)\cdot(1+\frac{(-\delta+iy)^2}{R^2})}{
(-\delta+iy)}\bigr|\cdot 2R\cdot e^{-\delta T}=M^*(R)\cdot
e^{-\delta T}\tag{***}
\]
where $M^*(R)$ depends on $R$ only.
\medskip

\noindent
For the integrals on the two intervals  where
$z=-s+iR$ and $z=-s-iR$ with $0\leq s\leq \delta$
we also get a constant $M^{**}(R)$ which is independent of
$T$ while the sum of absolute values of the line integrals over
these two lines 
is estimated by
\[
M^{**}(R)\cdot\int_0^\delta\, e^{-sT}\cdot ds=
M^{**}(R)\cdot \frac{1-e^{-\delta T}}{T}\tag{****}
\]


\noindent
Now the requested limit formula (ii) follows from the (*)-inequalities
above.
Namely, for a given
$\epsilon>0$ we first choose $R$ so large
that the sum of (*) and (**) is $\leq\epsilon/2$.
With $R$ kept fixed we can then choose $T$ so large
that (***) and (****) both are $\leq\epsilon/2$ which
finishes the proof of Theorem 1.


\bigskip


\centerline{\bf 7. The
Lagrange series and the Kepler equation}
\bigskip

\noindent
Let $f(w)$ be an 
analytic function of the complex variable $w$
defined in some disc of radius $R$ centered at $w=0$. We assume that
$f(0)=0$ and with another complex variable $z$
we seek an analytic  function $w=w(z)$ such that
\medskip

\noindent
\[ 
w(z)= z\cdot f(w(z))\tag{*}
\]
We will use residue calculus and Rouche's theorem to
find $w(z)$. Let 
$z$ be fixed for a while and consider some
$0<r<R$ such that
\[
\max_{|w|=r}\, |z\cdot f(w)|<r
\]
This means that the analytic function
$g(w)= z\cdot f(w)$ has absolute value $<|w|$ on
the circle $|w|=r$. Rouche's theorem implies that
the analytic function $w-zf(w)$ has a unique simple zero in
the disc
$|w|<r$.
Moreover, by the formula in XX this zero is given by
\[
w(z)=\frac{1}{2\pi i}\cdot\int_{|w|=r}
\frac{1-zf'(w)}{w-zf(w)}\cdot w\cdot dw
\]
We can evaluate the integral using the series expansion
\[
\frac{1}{1-\frac{z^k\cdot f(w)^k}{w}}=
1+\sum_{k=1}^\infty\frac{(zf(w))^k}{w^k}
\]
More precisely, we see that $w(z)$ becomes
\[
\frac{1}{2\pi i}\cdot\int_{|w|=r}\,\sum_{k=1}^\infty\,
\frac{(z(f(w))^k}{w^k}- 
\frac{1}{2\pi i}\cdot\int_{|w|=r}\,\sum_{k=2}^\infty\,
\frac{z^k\cdot f'(w)\cdot f(w))^{k-1}}{w^{k-1}}
\]
\medskip

\noindent
If $k\geq 1$  residue calculus gives
\[
\frac{1}{2\pi i}\cdot\int_{|w|=r}\,
\frac{(z(f(w))^k}{w^k}-\cdot dw=
z^k\cdot 
\frac{f^k)^{(k-1)}(0)}{(k-1)!}
\]
Similarly we find
\[
\frac{1}{2\pi i}\cdot\int_{|w|=r}\,\sum_{k=2}^\infty\,
\frac{z^k\cdot f'(w)\cdot f(w))^{k-1}}{w^{k-1}}\cdot dw=
z^k\cdot 
 \frac{f'\cdot f)^{k-1})^{(k-2)}(0)}{(k-2)!}
\]


\noindent
Next, notice the equality
\[
(f^k)^{(k-1)}(0)= k\cdot
(f'\cdot f^{k-2})^{(k-2)}(0)
\]


\noindent
Since $\frac{1}{(k-1)!}-
\frac{1}{k\cdot (k-2)!}=\frac{1}{k !}$
we conclude that
one has the series formula:
\[ 
w(z)= \sum_{k=1}^\infty\, \frac{(f^k)^{(k-1)}(0)}{k!}\cdot z^k\tag{*}
\]
\medskip


\noindent
{\bf Radius of convergence.}
The analytic function $w(z)$ has the expansion
by the Lagrange series above.
The determination of the radius of convergence depends on
the given function $f(w)$.
A \emph{lower bound} for the radius of convergence
is found by the use of Rouche's theorem above. Assume for simplicity
that $f(w)$ is an entire function. If $r>0$ is given
we find the positive number $\rho(r)$ for which
\[
\rho(r)\cdot \max_{|w|=r}|f(w)|=r
\]
By (*) above and  Rouche's theorem we have seen
that the Lagrange series converges in
the disc $|z|<r$.
Here we have a \emph{free choice} of $r$. But each
time
$r$ is chosen we must take into the account the maximum
of $f(w)$ on $|w|=r$. More precisely, put
\[
M_f(r)=\max_{|w|=r}\, |f(w)|
\]
Then the discussion above gives
\medskip

\noindent
{\bf Theorem.} \emph{The Lagrange series converges in the disc
of the complex $z$-plane whose radius is}
\[ 
\rho^*=\max_r\,\frac{r}{M_f(r)}
\]
\medskip

\noindent
{\bf Example.}
In his far reaching studies of 
the motion of orbits of those planets which astronomers
were able to 
watch before 1600, 
Kepler's work contains a study of   the equation
\[
\zeta=a+z\cdot\text{sin}\,\zeta\tag{*}
\] 
where $a>0$ is a real constant.
We shall determine
the series expansion of $\zeta(z)$.
Notice that if $w=\zeta-a$ then
(*) becomes
\[
w=z\cdot \text{sin}(w+a)
\]
So with the entire function $f(w)=\text{sin}(w+a)$ we encounter the
general case above and conclude that
the series becomes
\[
\zeta(z)=a+z\cdot\text{sin}\, a+\sum_{k=2}^\infty
\frac{z^k}{k!}\cdot\frac{ d^{k-1}(\text{sin}^k\, a)}{da^{k-1}}
\]
\medskip

\noindent
{\bf Exercise.}
Let $r>0$ and show that
the series
$\zeta(z)$ converges when
\[ 
|z|\cdot\frac{e^r+e^{-r}}{2r}<1\implies
|z|<\frac{2r}{e^r+e^{-r}}
\]
and for $z$ in  this disc we get
$|\zeta(z)|<r$.
To obtain a largest possible disc we seek
\[
\max_r\, \frac{2}{e^r+e^{-r}}
\]
The reader is invited to calculate the maximum numerically
and in this way find a \emph{lower bound} for the 
radius of convergence of
the Kepler series.
In contrast to all "heroic computations" by Kepler carried out
in the years 1600-1620 and the subsequent 
refined studies of series expansions by Lagrange around
1760 
described above, today's student can use a computer to determine
the radius of convergence numerically.
This, it is an instructive exercise to determine numerically
the radius of convergence of the Lagrange series for each
real $a$. Here it is of course interesting to analyze hos
the radius of convergence depends on $a$.


\bigskip

\centerline{\bf{8. An example by Bernstein.}}
\medskip

\noindent
Let $n\geq 1$ and consider a polynomial $P(z)=a_0+\ldots+a_nz^n$ 
of some degree  $n$. We have the equality
\[ 
\sum\, |a_n|^2=\frac{1}{2\pi}\cdot \int_0^{2\pi}\,
\bigl|P(e^{i\theta})\bigr |^2\cdot d\theta
\]
So if we consider the maximum norm over the unit disc $D$:
\[ 
||P||_D=\max_\theta\, |P(e^{i\theta})|
\]
then the Cauchy-Schwarz inequality gives
\[
\sum\, |a_k|\leq \sqrt{n+1}\cdot \sqrt{||P||_D}\tag{*}
\]
\medskip

\noindent
It turns out that (*) is sharp, i.e for arbitrary large $n$
we can find a polynomial $P_n(z)$ such that
\[
\frac{\bigl(\sum\, |a_k|\bigr)^2}{ n+1}\simeq  ||P_n||_D\tag{**}
\]


\noindent
The first example of this kind 
comes from a construction by  S. Bernstein from 1914.
He considered a prime number $p$ of the form $4\mu+1$.
For each integer $1\leq k\leq p-1$
there exists the Legendre symbol
$\binom{k}{p}$which is +1 if $k$ is a quadratic remained to $p$
and otherwise $-1$.
Now we get the trigonometric cosine-polynomial
\[ 
B_p(\theta)= \frac{2}{p^{\frac{3}{2}}}\cdot \sum_{k=1}^{p-1}\,
(p-k)\binom{k}{p}\cdot \text{cos}(k\theta)=\sum_{k=1}^{p-1}\, a_k^{[p)}\cdot
\text{cos}(k\theta)
\]
Bernstein proved that
\[ 
\max_\theta |B_p(\theta)\leq 1\quad\text{and}\quad 
\sum_{k=1}^{p-1}\, a_k^{[p)}=
\frac{p-1}{\sqrt{p}}
\]
\medskip

\noindent
With $n=4\mu$ we get the polynomial $Q_n(z)$ where

\[
\mathfrak{Re}(Q_n(e^{i\theta})=B_p(\theta)\quad\text{and}\quad
\mathfrak{Im}(Q_n(0))=0
\]
The maximum norm for $\mathfrak{Im}(P(e^{i\theta})$
is estimated above by the Exercise below.
It follows that
\[
Q_n(z)= \sum_{k=1}^{p-1}\, a_k^{[p)}\cdot z^k\quad\text{and}\quad
|Q_n|_D\leq C\cdot \text{log}\,p 
\] 
where $C$ is the absolute constant from
Exercise XX below.
So for this polynomial the left hand side in (**) becomes
$\frac{(p-1)^2}{p^2}$ which is close to 1 when $p$ is large. At the same time
$\text{log}\, p$ is considerably smaller than the degree $n=p-1$.
So with
\[ 
P_n(z)= \frac{1}{|Q_n|_D}\cdot Q_n(z)
\] 
we get a polynomial whose maximum norm is one
while the left hand side gets close to one as $p$ increases.
\medskip


\medskip

\noindent
{\bf{Exercise.}}
Let 
$u(\theta)=\sum_{k=0}^n
\, a_k\cdot\text{cos}\,\theta+
\sum_{k=1}^n
\, b_k\cdot\text{sin}\,\theta$ 
be a trigonometric polynomial of degree $n$
where $\{a_k\}$ and $\{b_k\}$ are real.
The conjugate trigonometric polynomial is defined by
\[
 v(\theta)=\sum_{k=1}^n \, \bigl[-b_k\cdot\text{cos}\,\theta+
a_k\cdot\text{sin}\,\theta\bigr]
\] 
Show the integral formula 


\[
v(\phi)=\frac{1}{\pi}\cdot \int_0^{2\pi}\,
\frac {\text{sin}\,\frac{n(\phi-\theta)}{2}\cdot
\text{sin}\,\frac{(n+1)(\phi-\theta)}{2}}{\text{sin}\,\frac{(\phi-\theta)}{2}}
\cdot u(\theta)\cdot d\theta\tag{1}
\]
\medskip

\noindent
From (1) the reader should verify that if
$M=\max_\theta\,|u(\theta)|$ then

\[ |v(\phi)|\leq
\frac{M}{\pi}\int_0^{2\pi}\, 
\bigl|\frac {\text{sin}\,\frac{n(\phi-\theta)}{2}}{\text{sin}\,\frac{(\phi-\theta)}{2}}
\bigr|\cdot d\theta
\]
\medskip

\noindent
Finally, show that there is an absolute constant $C$ such that
\[
\frac{1}{\pi}\cdot \int_0^{2\pi}\, \bigl|\frac {\text{sin}\,\frac{n(\phi-\theta)}{2}}{\text{sin}\,\frac{(\phi-\theta)}{2}}
\bigr|\cdot d\theta\leq C\cdot \text{Log}\, n\quad\colon\quad n\geq 2
\]
Hence the maximum norm for the conjugate $v$-function satisfies

\[
\max_\theta\, |v(\theta)|\leq C\cdot \text{Log}\, n\cdot
\max_\theta\, |u(\theta)|\tag{*}
\]


\noindent{
\bf{Remark.}} The  inequality above was first demonstrated by Fekete
in his article (Journal f�r mathematik 146).


\newpage

\centerline{\bf{9. Almost periodic functions and additive number theory.}}

\bigskip

\noindent
{\bf{Introduction.}}
We expose  a result
presented by Beurling at a seminar at Uppsala University in April 1948.
Let $2\leq m\uuu 1<m\uuu 2<\ldots$ be a strictly increasing
sequence
of integers. Denote by $S$  the even set
given by the union  of  $\{m\uuu\nu\}$  and 
$\{\vvv m\uuu\nu\}$.
Assume that the additive group generated by the integers in $S$ is equal to
${\bf{Z}}$ which  means that the sequence $\{m\uuu\nu\}$
has no common prime number $\geq 2$ as factor.
Next, consider some non\vvv negative and even function
$\phi$ defined on $S$. 
By  hypothesis 
every integer $n$ can be represented by a finite sum
of integers from $S$ where repetitions are allowed. Hence we can 
define the function $\mathfrak{p}\uuu\phi$ on ${\bf{Z}}$
by
\[
\mathfrak{p}\uuu\phi(n)=\min \sum \,\phi(m\uuu \nu)\quad \text{such that}\quad
n= \sum\, m\uuu\nu\tag{1}
\] 
where  the minimum is taken over
finite subsets of $S$.
It is obvious that
this function is even and  subadditive:
\[
\mathfrak{p}\uuu\phi(n\uuu 1+n\uuu2)\leq
\mathfrak{p}\uuu\phi(n\uuu 1)+\mathfrak{p}(n\uuu 2)
\]
\medskip


\noindent
In particular $\mathfrak{p}\uuu\phi(n)=0$ for all $n\neq 0$
if and only if $\mathfrak{p}\uuu\phi(1)=0$ and 
this vanishing holds if and only if for every
$\delta>0$ there exists  a finite set
$\{m\uuu\nu\}$ in $S$ such that
\[
\sum\, m\uuu \nu=1\quad\text{and}\quad 
\sum\,\phi(m\uuu\nu)<\delta\tag{*}
\]
We seek conditions on $\phi$ in order that
(*) holds, or equivalently that $\mathfrak{p}\uuu\phi(1)=0$.
To get such a criterion Beurling restricted the attention to a class of
$\phi$\vvv functions satisfying the following extra condition.
An even  subset $W$ of ${\bf{Z}}$
is called relatively  dense if
the additive group generated by $W$ is equal to
${\bf{Z}}$.

\medskip

\noindent
{\bf{9.1 Definition.}}
\emph{Given the even set $S$  above we denote by
$\text{AP}(S)$ the set of even functions
$\phi$ defined on $S$  such that for every $\epsilon >0$
the set}
\[
S\uuu\epsilon(\phi)= \{m\in S\,\colon\, \phi(m)<\epsilon\}
\]
\emph{is relatively dense.}
\medskip

\noindent
{\bf{The zig\vvv zag function $\rho(x)$.}}
Before Theorem 9.2  is announced
we introduce the periodic $\rho$\vvv function on the real $x$\vvv line
where
\[ 
\rho(x)=|x|\quad\colon \vvv 1/2\leq x\leq 1/2
\]
and extended so that
$\rho(x)=\rho(x+1)$ hold for
every  $x$.
\bigskip

\noindent
{\bf{9.2 Theorem.}}
\emph{For each $\phi\in\text{AP}(S)$ the necessary and sufficient
condition in order that (*) holds is that}
\[ 
\max\uuu {m\in S}\, \,\rho(\alpha m)\vvv \eta\cdot \phi(m)\geq 0\tag{**}
\] 
\emph{hold for all pairs $0<\alpha<1$ and $\eta>0$.}
\bigskip

\noindent
Before we enter the proof
we recall some facts about almost periodic functions.
A bounded  complex\vvv valued
function $f$ on ${\bf{Z}}$
is almost periodic
if there to every
$\epsilon>0$ exists a relatively dense set $W$ such that
\[
 \max\uuu {n\in{\bf{Z}}}\, 
|f(n+w)\vvv f(n)|<\epsilon\quad\text{for all}\quad w\in W
\]
From this it follows easily that there exists
the mean\vvv value
defined by
\[ 
\mathcal M(f)= \lim\uuu{b\vvv a\to+\infty}\,
\frac{f(a)+f(a+1)+\ldots f(b)}{b\vvv a+1}
\]
\medskip

\noindent
Next, for each real number
$\alpha$ the exponential function $E\uuu\alpha$
defined by
$E\uuu\alpha(n)= e^{2\pi i\alpha n}$
is almost periodic on ${\bf{Z}}$.
It follows that when $f$ is almost
periodic then there exists the function
\[
\mathcal C\uuu f(\alpha)=
\mathcal M(E\uuu\alpha\cdot f)
\]
A result due to Harald Bohr
asserts that if  $f$ is almost periodic and $f(1)\neq 0$
then the $\mathcal C\uuu f$\vvv function is not identically zero
on $(0,1)$, i.e. there
exists some $0<\alpha<1$ such that
$\mathcal C\uuu f(\alpha)$. 
For the  proof of Theorem 9.2  we shall alos need  the following:
\medskip

\noindent
{\bf{9.3 Proposition}}
\emph{If $\phi$ belongs to $\text{AP}(S)$ it follows that
$\mathfrak{p}\uuu\phi$ is an almost periodic function on
${\bf{Z}}$.}
\medskip

\noindent
{\bf{Exercise.}} Prove this assertion.

\bigskip




\noindent
\emph{Proof of Theorem 9.2}
Suppose first that
$\mathfrak{p}\uuu\phi (1)\neq 0$ which means that (*)
has no solution for small $\delta$. To show that
the inequalitites
(**) in Theorem 9.2 cannot hold for all pairs $\alpha.\eta$
we proceed as follows: Since $\mathfrak{p}\uuu\phi$ by defintion 
is periodic it is in particular almost periodic and by
a  general formula  for $\mathcal M$\vvv functions
attached to almost periodic functions we get for each integer $m\in S$:


\[
|e^{2\pi i\alpha m}\vvv 1|\cdot \mathcal C_{\mathfrak{p}_\phi}(\alpha)|
\leq
\max\uuu n\, |\mathfrak{p}\uuu\phi(n+m)\vvv p(n)\bigr|\leq
\mathfrak{p}\uuu\phi (m)
\]
where the last inequality follows since  that $\mathfrak{p}\uuu\phi $ is subadditive.
Introducing the sine\vvv function we get
\[
2\cdot |\sin (\pi \alpha m)|\cdot \mathcal C_{\mathfrak{p}_\phi}(\alpha)|\leq\mathfrak{p}\uuu\phi(m)
\leq 
\phi(m)\quad\colon\quad m\in S
\]
\medskip

\noindent
Since $\mathfrak{p}\uuu\phi (1)\neq 0$ is assumed
we know from Bohr's theory  that there exists some
$0<\alpha<1$ such that
$\mathcal C_{\mathfrak{p}_\phi}(\alpha)\neq 0$.  
At the same time the zig\vvv zag function
satisfies:

\[ 
\rho(x)\leq \frac{\pi}{2}\cdot |\sin \pi\cdot x\,|
\] 
for every real $x$. Hence we get
\[ 
\rho(\alpha\cdot m)\cdot \mathcal C(\alpha)\leq \frac{4}{\pi}\phi(m)
\]


FINISH ....

\newpage

%\end{document}




