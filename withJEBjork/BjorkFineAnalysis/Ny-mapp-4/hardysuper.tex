%\documentclass{amsart}

%\usepackage[applemac]{inputenc}


%\addtolength{\hoffset}{-12mm}
%\addtolength{\textwidth}{22mm}
%\addtolength{\voffset}{-10mm}
%\addtolength{\textheight}{20mm}

%\def\uuu{_}

%\def\vvv{-}


%\begin{document}

\centerline{\bf\large {III. A. Hardy-Littlewood's maximal function}}
\bigskip

\centerline{\emph{Contents}}

\bigskip

\noindent
\emph{1. The weak type estimate}


\bigskip

\noindent
\emph{2.  An $L^2$-inequality}

\bigskip

\noindent
\emph{3 Harmonic functions  and Fatou sectors}



\bigskip

\noindent
\emph{4. Application to analytic functions}



\bigskip

\noindent
\emph { 5. Conformal maps and the Hardy space
$H^1(T)$}



\bigskip

\noindent
{\bf{Introduction.}}
The results below are foremost due to Hardy, Littlewood and Fatou.
Before we describe results about harmonic and analytic functions
we expose some general facts about maximal functions from measure theory.

\medskip


\centerline{\bf{1. The weak type estimate}}
\bigskip



\noindent
Let $f(x)$ be a non-negative function and integrable function
on the real $x$-line
with support in a finite interval $[0,A]$.
The  forward maximal function  is defined for ever $x\geq 0$ by
\[
 f^*(x)=\max_{h>0}\, \frac{1}{h}\int_x^{x+h}\, f(t)\cdot dt
\]
It is clear that $f^*$ is non-negative and supported by $[0,A]$.
To each $\lambda>0$ we get the
set $\{f^*>\lambda\}$. 
\medskip

\noindent
{\bf{1.1 Theorem}}
\emph{For each $\lambda>0$ one has the inequality}
\[ 
{\bf{m}}(\{f^*>\lambda\})\leq \frac{1}{\lambda}\cdot
\int_{\{f^*>\lambda\}}\, f(x)\cdot dx
\]


\noindent
\emph{Proof.}
Introduce the primitive function
\[ 
F(x)=\int_0^x\, f(t)\cdot dt
\]
With $\lambda>0$ we have the continuous function $F(x)-\lambda$
and define the forward Riesz set by:

\[
\mathcal E_\lambda=\{0\leq  x<A\colon\exists \, y>x\quad
\text{and}\quad F(y)-\lambda y> F(x)-\lambda x\}
\]

\noindent
{\bf{1.2 Exercise.}}
Show  the equality
\[
\mathcal E_\lambda=\{ f^*>\lambda\}
\]

\noindent
Now $\mathcal E_\lambda$ is an open  set and hence 
a disjoint union of intervals $\{(a_k,b_k)\}$.
With these notations one has
\medskip

\noindent
{\bf{1.3 Exercise.}}
Show the following for each interval $(a_k,b_k)$:
\[
F(b_k)-\lambda\cdot b_k=\max_{a_k\leq x\leq b_k}\, F(x)-\lambda
\]
In particular one has
\[
\lambda(b_k-a_k)\leq F(b_k)-F(a_k)
\]
This holds for each $k$ and after a summation over the 
forward Riesz intervals the requested inequality in
Theorem 1.1 follows.
\bigskip

\noindent
Using Theorem 1 can prove the following $L^2$\vvv inequality.

\medskip

\noindent
{\bf{1.4 Theorem.}}
\emph{One has}

\[
||f^*||_2\leq  |f||_2
\]

\noindent
\emph{Proof.} By the general formulas for distribution functions 
from XX we have:
\[
\int_0^A\, f^*(x)^2\cdot dx=
\int_0^\infty\,
\lambda\cdot {\bf{m}}(\{f^*>\lambda\})\cdot d\lambda
\]
By Theorem 1.1 the last integral is majorised by

\[
\int_0^\infty\,[\int_
{{\bf{m}}(\{f^*>\lambda\}}\, f(x)\cdot dx\,\bigr]\cdot d\lambda)=
\iint_{\{f^*(x)>\lambda\}}\,f(x)\cdot dxd\lambda=
\]
\[
\int_0^A\, \bigl[\int_0^{f^*(x)}\, d\lambda\bigr]\cdot f(x)\cdot dx
=\int_0^A\, f^*(x)\cdot f(x)\cdot dx
\]
By the Cauchy-Schwarts in equality the last integral is majorised
by the product of $L^2$-norms
\[
||f^*||_2\cdot |f||_2
\]
Hence
\[
||f^*||_2^2=
\int_0^A\, f^*(x)^2\cdot dx\leq ||f^*||_2\cdot |f||_2
\] 
and Theorem 1.4 follows after
division with $||f^*||_2$.

\medskip



\noindent
{\bf{1.5 Remark.}}
In a similar way we get an $L^2$-inequality using the 
backward maximal  function
\[
 f_*(x)=\max_{h>0}\, \frac{1}{h}\int_{x-h}^x\, f(t)\cdot dt
\]
and also the
full maximal function

\[ f^{**}(x)= \max_{a,b}\, 
\frac{1}{a+b}\int_{x-a}^{x+b}\, |f(t)|\cdot dt\tag{**}
\] 
with the maximum taken over pairs $a,b>0$.
Then we get
the $L^2$-inequality
\[
||f^{**}||_2\leq  2\cdot |f||_2\tag{1.6}
\]






\bigskip

\noindent
\centerline {\bf{2. A study of harmonic functions.}}
\medskip

\noindent
Let $f(t)$ be complex-valued function on the real $t$-line
such that
\[
\int_{-\infty}^\infty\, \frac{|f(t)|}{1+t^2}\cdot dt<\infty
\]
We also assume that
\[ 
\max\,\frac{1}{b+a}\cdot \int_{-a}^b\, |f(t)|\cdot dt<\infty
\] 
where the maximum is taken over all pairs $a,b>0$.
Define the function $V(z)= V(x+iy)$ in the upper half-plane $y>0$
by
\[
V(z)= \frac{1}{\pi}\int_{-\infty}^\infty\,
\frac{y}{(x-t)^2+y^2}\cdot f(t)\cdot dt
\]

\medskip

\noindent
{\bf{2.1 Exercise.}}
Prove the inequality
\[ 
|V(x+iy)|\leq \bigl(\frac{|x|}{y}+2\bigr)\cdot f^{**}(0)\tag{1}
\]
where 
\[ 
f^{**}(0)=\max\uuu{a,b>0}\,\frac{1}{b-a}|\cdot \int_{-a}^b\, |f(t)\cdot dt
\]


\medskip

\noindent
Next, define Fatou's maximal function on the real $x$-line by
\[
V^*(x)=\max_{y\leq |s|}
|V(x+iy)|\tag{3}
\]
We also introduce the
function $f^{**}(x)$ defined for each real $x$ by:
\[
f^{**}(x)=\max\,\frac{1}{b-a}|\cdot \int_{x-a}^{x+b}\, |f(t)\cdot dt
\]
\medskip

\noindent
{\bf{2.2 Exercise.}}
Show the inequality
\[ 
V^*(x)\leq 3\cdot f^*(x)
\] 
\medskip

\noindent
Next, apply the inequality (1.6) which together with Exercise 2.2 give 

\[
 \int_{-\infty}^\infty V^*(x)^2\cdot dx\leq 18\cdot 
\int_{-\infty}^\infty f^2(x)\cdot dx\tag{i}
\]
Since  $V(x)= f(x)$ holds on the real line we conclude the following:

\medskip

\noindent
{\bf{2.3 Theorem.}} \emph{One has the inequality}
\[
||V^*||_2\leq 3\sqrt{2}\cdot  
\sqrt{\int_{-\infty}^\infty V(x)^2\cdot dx}\tag{*}
\]
\bigskip

\noindent
\centerline
{\bf{3. Application to analytic functions.}}
\medskip

\noindent
Let $f(z)$ be analytic in $\mathfrak{Im}(z)>0$
and assume that
there is a constant $C$ such that

\[
\int_{-\infty}^\infty\,
\frac{|f(x+iy)|}{1+x^2}\cdot dx\leq C\quad\text{for all}\quad y>0
\] 
It means that $f$ belongs to the Hardy space $H^1$ in the upper half-plane $U_+$.
We can divide out the zeros via a Blaschke product and write
\[ 
f=B\uuu f\cdot g
\] 
where $g$ again belongs to $H^1$ and has no zeros in $U_+$.
Then $\sqrt{g}$ is defined which 
gives a complex-valued harmonic function
\[ 
V(z)= \sqrt{g(z)}
\]
\medskip

\noindent
{\bf{3.1 Exercise.}}
Apply Theorem 2.3 to the $V$\vvv function and use that
$|f(z)|\leq |g(z)|\leq |V^2(z)|$ to show that
\[
\int_{-\infty}^\infty\, |f^*(x)|\cdot dx\leq
3\sqrt{2}\cdot \int_{-\infty}^\infty\, |f(x)|\cdot dx\tag{1}
\] 
where $f^*(x)$ is Fatou's maximal function
for $f$ defined for each real $x$ by

\[
f^*(x)=\max_{y\leq |s|}\,|f(x+iy)|/tag{2}
\]
\medskip

\noindent
{\bf{3.2 Exercise.}}
Use the conformal map from $U_+$ to the unit disc $D$
defined by
\[
w=\frac{z-i}{z+i}
\]
Explain how the previous result is translated when we start from an
analytic function $f$ in $D$ for which the boundary value function $f(e^{i\theta})$
is in $L^1(T)$.

\newpage

\centerline{\bf{4. Hardy spaces and conformal maps}}


\medskip

\noindent
Let $g(z)=\sum\, a_nz^n$ be analytic in $D$ and assume that 
there exists a constant $C$ such that
\[ 
\int_0^{2\pi}\, |g(re^{i\theta})|\cdot d\theta\leq C
\] 
for every $r<1$.
Thus, by the Brothers Riesz Theorem 
$g$ belongs to the Hardy space $H^1(T)$.
In $D$ there exists a single-valued brach of
$\log(1-z)$ whose imaginary part stays in $(-\pi/2,\pi/2)$ and
with $z=re^{i\theta}$ we  have
\[ 
\mathfrak{Im}\, \log(1-z)=
-\frac{1}{2i}\cdot \sum_{n=1}^\infty\, \frac{r^n}{n}(e^{in\theta}-e^{-in\theta})
\]


\noindent
{\bf{4.1 Exercise.}}
Deduce from the above that
\[
\int_0^{2\pi}\, \mathfrak{Im}\,\log(1-re^{i\theta})\cdot
g(re^{i\theta})\cdot d\theta=
-\pi i\cdot \sum_{n=1}
^\infty\, \frac{a_n}{n}\cdot r^{2n}\tag{*}
\]
\medskip

\noindent
{\bf{The case when $\{b_n\}$ are real and $\geq 0$.}}
If this holds then (*) and the triangle inequality yield:
\[
\pi\cdot  \sum_{n=1}
^\infty\, \frac{a_n}{n}\cdot r^{2n}\leq \frac{\pi}{2}\cdot
\int_0^{2\pi}\, |g(re^{i\theta})|\cdot d\theta
\] 

\noindent
So if we introduce the $H^1(T)$-norm
\[
||g||_1= \int_0^{2\pi}\, |g(e^{i\theta})|\cdot d\theta
\] 
it follows after a passage to the limit when $r\to 1$ that
\[
\sum_{n=1}^\infty\, \frac{b_n}{n}\leq \pi\cdot |g||_1\tag{**}
\]
\medskip

\noindent
{\bf{4.2 A study
of conformal mappings.}}
Let $\phi\colon D\to \Omega$ be a conformal mappng
and assume that the complex derivative
$\phi'(z)$ belongs to the Hardy space $H^1$ as above.
Since $\phi'\neq 0$ in $D$ there exists a single-valued 
analytic square-root:


\[ 
\psi(z)=\sqrt{\phi'(z)}
\]
Then $\psi\in H^2(T)$ 
so if
\[ 
\psi(z)=\sum\, b_nz^n\implies \sum\, |b_n|^2<\infty
\]
\noindent
Let us then consider the $H^2$-function
\[
\Psi(z)= \sum\, |b_n|z^n
\]
We get
\[ 
\Psi^2(z)=\sum\,A_nz^n\quad\text{where}\quad
A_n=\sum_{k=0}^{k=n}\, |b_k|\cdot |b_{n-k}|
\]
and (**) in Exercise 4.1 gives:
\[
\sum_{n=1}^\infty
\,\frac{A_n}{n}\leq \pi\cdot \int_0^{2\pi}\,
|\Psi(e^{i\theta})|^2\cdot d\theta\tag{1}
\]



\noindent
Next, consider the Taylor series
\[ 
\phi'(z)= \sum\, a_nz^n\implies
a_n=\sum_{k=0}^{k=n}\, b_k\cdot b_{n-k}
\] 
The triangle inequality gives $|a_n|\leq A_n$ for each $n$
so (1) entails that
\[ 
\sum_{n=1}^\infty \, \frac{|a_n|}{n}<\infty\tag{2}
\]

\noindent
Finally, consider the Taylor expansion of $\phi(z)$:
\[
\phi(z)=\sum\, c_nz^n
\]
Here
\[ 
nc_n=a_{n-1}\quad\colon\quad n\geq 1
\]
Then it is clear that (2) implies that the series
$\sum\, |c_n|<\infty$. Hence we have proved the following result which is due to Hardy:

\medskip

\noindent
{\bf{4.3 Theorem.}}
\emph{
Let $\phi(z)$
be a conformal map such that
$\phi'$ belongs to $H^1$. Then the Taylor series of $\phi$ 
is absolutely convergent.}
\medskip

\noindent
{\bf{4.4 Exercise}}. Let $\Omega$ be a Jordan domain  whose boundary
curve
$\Gamma=\partial\Omega$ has a finite arc-length.
Let $\phi\colon D\to \Omega$ be the con formal mapping which by results from
(xx) extends to a homeomorphism from the closed disc
$\bar D$ onto $\bar\Omega$.'
Let $\ell(\Gamma)$ be the arc-length of $\Gamma$. Show that the derivative
$\phi'(z)$ belongs to the Hardy space and
\[
\int_0^{2\pi}\, |\phi'(e^{i\theta})|\cdot d\theta
\leq \ell(\Gamma)
\]
From this it follows that the Taylor series of $\phi(z)$ is absolutely convergent.
\medskip

\noindent
\emph{A hint for the exercise.}
To each $n\geq 1$ we set $\epsilon= e^{2\pi i/n}$, i.e. the $n$:th root of the unity.
Now
$\phi$ yields a homeomorphism from $T$ onto $\Gamma$.
The definition of $\ell(\Gamma)$ gives the inequality
below  where we set
$\epsilon^0=1$.

\[
\sum_{k=1}^n\, |\phi(\epsilon^k\cdot e^{i\theta})-\phi(\epsilon^{k-1}
\cdot e^{i\theta})|\leq \ell(\Gamma)\quad\text{for every}
\quad
 0\leq\theta\leq 2\pi\tag{1}
\] 
Keeping $n$ fixed we notice that the function

\[ 
s_n(z)=\sum_{k=1}^n\, |\phi(\epsilon^k\cdot z)-\phi(\epsilon^{k-1}\cdot z)|
\] 
is subharmonic in $D$. So the maximum principle for subharmonic functions
and (1) give

\[
\max_\theta\, s_n(re^{i\theta})\leq \ell(\Gamma)\tag{2}
\] 
for each $r<1$.
Next,
with $r<1$ fixed
the reader may verify the limit formula:

\[
\lim_{n\to\infty}s_n(r)=
\int_0^{2\pi}\, |\phi'(re^{i\theta})|\cdot d\theta\tag{3}
\]
Hence (2-3) give

\[
\int_0^{2\pi}\, |\phi'(re^{i\theta})|\cdot d\theta\leq \ell(\Gamma)
\] 
Now the Brothers Riesz  theorem implies that
$\phi'(z)$ belongs to $H^1(T)$, i.e.
the boundary value function $\phi'(e^{\theta})$ exists and belongs to $L^1(T)$.




%\end{document}
