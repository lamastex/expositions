%\documentclass{amsart}
%\usepackage[applemac]{inputenc}


%\addtolength{\hoffset}{-12mm}
%\addtolength{\textwidth}{22mm}
%\addtolength{\voffset}{-10mm}
%\addtolength{\textheight}{20mm}
%\def\uuu{_}


%\def\vvv{-}

%\begin{document}

\centerline{\bf\large{IV. Nevanlinna-Pick theory}}




\bigskip
\centerline{\emph{Contents}}


\bigskip

\noindent
\emph{0. The Nevanlinna-Pick Interpolation}

\bigskip

\noindent
\emph{1. The Lindel�f-Pick principle
with an application}

\bigskip

\noindent
\emph{2. A result by Julia}
\bigskip

\noindent
\emph{3. Geometric results by L�wner}

\bigskip

\centerline{\bf{Introduction.}}

\bigskip

\noindent
In the unit  disc $D$ there exists the a metric defined by
\[ 
\frac{|dz|}{1-|z|^2}
\]
In a joint article from 1916, Lindel�f and Pick discovered that
if $f(z)\in\mathcal O(D)$ has maximum norm $\leq 1$, then
the map $z\to f(z)$ does not increase the metric (*).
This result turns out to be very useful and
is applied in  � 2
to give a proof of a theorem due  Julia.
In � 3 we prove some results due to L�wner about
geometric properties of analytic mappings.
Section 0
is devoted to an
interpolation theorem
due to  Nevanlinna and Pick.
We give a detailed proof since the result has
a wide range of applications beyond analytic function theory
in various optimization problems.






\bigskip

\bigskip



\centerline{\bf 0. The Nevanlinna-Pick Interpolation Theorem}
\bigskip

\noindent
Let $D$ be the open unit disc.
Given an  $n$-tuple  of distinct points
$z_1,\ldots,z_n$ in $D$ and some $n$-tuple
$w_1,\ldots,w_n$ of complex numbers we put:
\[
\rho(z(\cdot),w(\cdot))= \min_{f\in\mathcal O(D)}\,
\,|f|_D\quad\colon\quad f(z_\nu)=w_\nu\quad\colon\, 1\leq\nu\leq n\tag{*}
\]
Thus we seek to interpolate preassigned values
at the points $\{z\uuu k\}$
with an analytic
function $f(z)$ whose maximum norm is minimal.
The case $n=1$ is trivial for then it is obvious that
the constant function $f(z)=w_1$ minimizes (*) so $\rho(z_1,w_1)=|w_1|$
hold for all $\alpha_1\in D$. If $n\geq 2$
there exists
at least some $f\in\mathcal O(D)$ which gives a minimum.
For let $\{f_\nu\}$
be a sequence of  functions which solve the interpolation
while their  maximum norms tend
to
$\rho(\alpha(\cdot),w(\cdot))$. This is a normal family
and hence we
extract a subsequence which
converges to a limit function $f_*$ whose maximum norm
is equal to $\rho(z(\cdot),w(\cdot))$.
It turns out
that the minimizing $f$ is unique and of a special form.
Before Theorem 1 is announced  we 
introduce the class
$\mathfrak{B}_{n\vvv 1}$ which consists of functions of the form:
\[ 
f(z)=e^{i\theta}\cdot \prod_{\nu=1}^{\nu=n\vvv 1}
\,\frac{z-\alpha_\nu}{1-\bar \alpha_\nu\cdot z}\tag{**}
\]
where  $0\leq\theta\leq 2\pi$ and $(\alpha_1,\ldots,\alpha_{n\vvv 1})$ is some 
$(n\vvv 1)$-tuple of  points in $D$ which are not necessarily distinct.


\bigskip

\noindent {\bf 0.1. Theorem}  \emph{For each pair of $n$-tuples $z(\cdot)$
and $w(\cdot)$
there exists a unique $f_*\in\mathfrak{B}_{n\vvv 1}$ 
and a positive real number $\rho$ such that
the  $\rho\cdot f_*(z)$
minimizes the interpolation (*).}
\bigskip


\noindent
{\bf{Remark.}}
With $\rho=\rho(z(\cdot),w(\cdot))$  the uniquenss
means that if $g\in\mathcal O(D)$ is an arbitrary
interpolating function which is $\neq f_*$ then
$|g|_D>\rho$.



\bigskip

\noindent
The proof of Theorem 0.1 requires several steps. 
First we shall establish a result about Blaschke products.






\medskip

\noindent
{\bf{0.3 Proposition.}}
\emph{Let $f$ be a function in
$\mathfrak{B}\uuu n$. For every 
$k(z)\in \mathcal O(D)$ with  maximum norm $|k|\uuu D= 1$
such that $f\vvv k$ has at least $n$ zeros counted with multiplicity in
$D$, it follows that $f=k$.}


\medskip



\noindent
\emph{Proof.}
We argue by a contradiction. If $k\neq f$
we
denote by $N(f-k:r)$ the number of
zeros of $f-k$ in $|z|<r$ counted with multiplicities.
The hypothesis gives some   $r_*<1$ such that
\[
N(k-f,r_*)\geq n\tag{ii}
\]
Next, to each 0$<r<1$ we set
\[
\mathfrak{m}(r)= \min\uuu\theta\, |f(re^{i\theta})\vvv k(re^{i\theta})|
\]
The hypothesis gives a sequence $\{r\uuu\nu\}$ such hat
$r\uuu\nu\to 1$ and every $\mathfrak{m}(r\uuu\nu)>0$.
Consider some $r\uuu\nu>r\uuu *$
and the analytic function

\[
h_\nu=\epsilon_\nu f+\frac{1}{2}(f-k)\quad\text{where}\quad
\epsilon_\nu=\frac{1}{4}\mathfrak{m}(r_\nu)\tag{iii}
\]
Since $|\frac{1}{2}(f-k)|>\epsilon_\nu \cdot |f|$ holds 
on the circle $|z|=r\uuu\nu$ it follows from
(i) and Rouche's theorem that we have:
\[ 
N(h_\nu : r_\nu)=N(f-k:r_\nu)\geq n\tag{iv}
\]
At the same time we can write
\[
h_\nu=(1+\epsilon_\nu)f-\frac{1}{2}(k+f)\tag{v}
\]
By assumption   $|k|_D=|f|_D=1$  and since $f$ is a finite
Blaschke product
its absolute value tends uniformly to zero as $|z|\to 1$. So
if $r\uuu\nu$ is sufficiently close to 1 we get:
\[ 
(1+\epsilon_\nu)\cdot |f(z)|>\frac{1}{2}|k(z)+f(z)|\quad\colon\quad |z|= r\uuu\nu\tag{vi}
\]
Then
another application of
Rouche's theorem gives:
\[
N(h_\nu : r)=N(f : r)\leq n-1\tag{vii}
\]
where the last inequality follows since
$f\in\mathfrak B_n$.
Now (vii) contradicts (iv) and hence we must have $k=f$
as requested.

\bigskip

\noindent
{\bf{A consequence.}}
Let $z(\cdot)$ and $w(\cdot)$ be some  pair of $n$\vvv tuples
and suppose there exists some $f\uuu *\in\mathfrak{B}\uuu {n\vvv 1}$
and some $\rho>0$
such that $\rho\cdot f\uuu *(z\uuu k)=w\uuu k$
hold for each $k$.
Then the function $f=\rho\cdot f\uuu *$ not only interpolates but it
has also the minimal maximum norm, i.e. we have the equality
$\rho=\rho(z(\cdot),w(\cdot))$.
For suppose we have strict inequality
$\rho(z(\cdot),w(\cdot))<\rho$ which gives an interpolating function $k(z)$
with maximum norm $|k|\uuu D<\rho$.
Since $|\rho\cdot f|=\rho$ holds on $|z|=1$ it follows by Rouche's theorem
that
$f$ and $f\vvv k$ has the same number of zeros i $D$.
Now $f\vvv k$ has at least $n$ zeros while $f$ has at most $n\vvv 1$
many zeros. This  contradiction and proves that
$f$ is minimizing and Proposition 0.2 entails also that
$f$ is uniquely determined in $\mathfrak{B}\uuu{n\vvv 1}$.
\medskip

\noindent
The results   above show that
there remains to establish an \emph{existence result}. Namely, there remains to prove
that for every given 
$n$\vvv tuple 
$z(\cdot)$ in $D$ and  an  arbitrary
$n$\vvv tuple 
$w(\cdot)$ of complex numbers,  there exists a pair  $f\uuu *\in\mathfrak{B}\uuu n$
and  $\rho>0$
such that $\rho\cdot f\uuu *$ solves the interpolation. Moreover, if the pair has been
found
then we have the equality
$\rho= \rho(z(\cdot),w(\cdot))$.
\medskip

\noindent
{\bf{0.4 Induction over $n$}}.
We shall prove the existence above by an induction over
$n$. First we establish a result where M�bius transformations intervene.
To each  $n$\vvv tuple $(z\uuu 1,\ldots,z\uuu n)$
we denote   for every   $\rho>0$  the family  $I\uuu \rho(z(\cdot))$
of all $n$\vvv tuples $w(\cdot)$
such that $\rho(z(\cdot),w(\cdot))=\rho$.
\newpage


\noindent
{\bf{0.5 Exercise.}}
For each point $a\in D$ we have the M�bius transformation
\[ 
M_a(z)=\frac{z-a}{1-\bar a z}
\]
If $f(z)\in \mathcal O(D)$
we get the new analytic function $f\circ M\uuu a$
which  has the same maximum norm as $f$. Use this to show that
\[ 
I\uuu \rho(z(\cdot ))=
I\uuu\rho(M\uuu a(z(\cdot)))
\]
hold for every $n$\vvv tuple $z(\cdot)$.
Thus, we can change the $z$\vvv points via a M�bius transformation
without affecting the
$I\uuu\rho$\vvv sets.
\medskip

\noindent
{\bf{0.6 The induction step.}}
If $g\in\mathfrak {B}\uuu{n\vvv 1}$
and $|a|<1$  then the composed function $g\circ M\uuu a$
again belongs to
$\mathfrak {B}\uuu{n\vvv 1}$.
So by  Exercise 0.5 it suffices to establish the existence of
an interpolation $f\in\mathfrak{B}\uuu{n\vvv 1}$
when $z\uuu 1=0$.
So assume this and consider first the case $w\uuu 1=0$.
Then we seek $f$ of the form
\[ 
f(z)= z\cdot g(z)\tag{1}
\]
where $g$ satisfies $g(z\uuu k)=\frac{w\uuu k}{z\uuu k}$
when $2\leq k\leq n\vvv 1$. By an an induction over
$n$
we can find $g=\rho\cdot g\uuu *$ for some
$g\uuu *\in \mathfrak{B}\uuu{n\vvv 2}$ and 
(1) entails that
\[
f=\rho\cdot z\cdot g\uuu *
\] 
where $z\cdot g\uuu *$ belongs to
$\mathfrak{B}\uuu{n\vvv 1}$ which gives  the requested interpolating function. 
\medskip

\noindent
{\bf{The case $w\uuu 1\neq 0$ }}.
To begin with we find
$f\in\mathcal O(D)$ with the maximum norm
$\rho=\rho(z(\cdot),w(\cdot))$
such that $f(0)=w\uuu 1$ and $f(z\uuu k)=w\uuu k$
when $2\leq k\leq n$. Put
\[ 
\mu\uuu k=\frac{w\uuu k\vvv w\uuu 1}{1\vvv
\rho^{\vvv 2}\cdot \bar w\uuu 1\cdot w\uuu k}\tag{1}
\]
We have also the analytic function
\[ 
g(z)= \frac{f(z)\vvv w\uuu 1}{1\vvv
\rho^{\vvv 2}\cdot \bar w\uuu 1\cdot f(z)}
\]
Since $|f|\uuu D=\rho$ we also have $|g|\uuu D=\rho$ where
$g(z\uuu k)=\mu\uuu k$ for each $k\geq 2$ while $g(0)=0$.
In particular 
\[
(0,\mu\uuu 2,\ldots,\mu\uuu n)\in
I\uuu\gamma(0,z\uuu 2,\ldots,z\uuu n)\quad\text{where}\quad \gamma\leq\rho
\]
By the induction over $n$ and the previous special case we find
$f\uuu *\in\mathfrak{B}\uuu{n\vvv 2}$ such that the function
\[
g\uuu *=\gamma \cdot z\cdot f\uuu *
\]
takes the interpolating values
$(0,\mu\uuu 2,\ldots,\mu\uuu n)$.
Next,  there exists the analytic function
\[ 
\phi(z)=\frac{g\uuu *(z)+w\uuu 1}{1+\rho^{\vvv 2}\cdot \bar w\uuu 1\cdot g\uuu *(z)}
\]
If $\gamma<\rho$
we see that
the maximum norm
$|\phi|\uuu D<\rho$.
At the same time
(1) entails that $\phi(0)=w\uuu 1$ and
$\phi(z\uuu k)= w\uuu k$ for $k\geq 2$.
Since $\rho$ was the interpolation norm
for the pair $(0,z\uuu 2,\ldots,z\uuu n)$ and the $n$\vvv tuple
$w(\cdot)$ we have  contradiction. Hence  
$\gamma=\rho$ holds and from the above it follows that

\[
\rho\cdot \frac{zf\uuu *(z)+w\uuu 1}{1\vvv \rho^{\vvv 2}
\bar w\uuu 1 z f\uuu *(z)}
\]
takes the requested interpolation values
$(w\uuu 1,\ldots,w\uuu n)$ which
gives the induction step over $n$.
\bigskip
















\medskip

\noindent
{\bf{Remark.}}
Consider the case $n=2$ in Theorem 0.1 where we assume that
$w_1\neq w_2$.
Theorem 1 shows  that
there exists a unique triple
\[
(a,\theta,\rho)\quad
\text{where}\quad  a\in D\quad\colon 0\leq\theta\leq 2\pi\quad\colon\rho>0
\]
such that  function
\[
 f(z)=\rho\cdot e^{i\theta}\cdot \frac{z-a}{1-\bar a\cdot z}
\]
solves the interpolation problem.
\medskip

\noindent
{\bf{0.7 Exercise.}}
With $n=2$ and $z\uuu 1=0$ while $z\uuu  2\neq 0$
is kept fixed
we solve the interpolation for a given pair $(w\uuu 1,w\uuu 2)$,
The minimizing interpolation function is of the form
\[
f(z)=\zeta\cdot \frac{z+\frac{w\uuu 1}{\zeta}}{
1+\frac{\bar w\uuu 1\cdot  z}{\bar\zeta}}
\] 
for some $\zeta\neq 0$.
So here $\zeta$ satisfies the equation
\[
z\uuu 2\cdot\zeta+w\uuu 1=
w\uuu 2+w\uuu 2 \bar w\uuu 1\cdot z\uuu 2\cdot \frac{1}{\bar\zeta}
\]
Writing $w\uuu 2= w\uuu 1+\gamma$
this amounts to solve the equation
\[
z\uuu 2\cdot |\zeta|^2=\gamma\cdot \bar\zeta+|w\uuu 1|^2\cdot z\uuu 2+
\gamma\cdot \bar w\uuu 1\cdot z\uuu 2\tag{*}
\]
We assume that $w\uuu 2\neq w\uuu 1$ which means that
$\gamma\neq 0$ and that the minimizing function
$f$ is not reduced to a constant. Hence   its maximum norm
$|\zeta|$ must be $|w\uuu 1|$,
Theorem 1 implies that under these conditions
(*) has a unique solution $\zeta$ with absolute value 
$>|w\uuu 1|$.
It is interesting to
analyze how $|\zeta|$ depends on the triple $z\uuu 2,w\uuu 1w\uuu 2$.
Dividing (*) with $z\uuu 2$ and regarding
$\lambda=\gamma/z\uuu 2$ as a parameter which varies
we are led to  the equation
\[
|\zeta|^2\vvv \lambda\cdot \bar\zeta=  
|w\uuu 1|^2+\gamma\cdot \bar w\uuu 1
\]


\noindent
The reader is invited to analyze the behaviour of $|\zeta|$
with a special attention  to the case when
$z\uuu 2$ is close to the origin while
$\gamma$ stays fixed, So here 
the interpolating function $f$ takes quite distinct  values at 
the origin while   $z\uuu 2$. So  one expects that its maximum norm
increases. Here is

\medskip

\noindent {\bf{0.8 A specific example.}}
Let $\gamma=1$ and $z\uuu 2=\epsilon$ for some small positive 
$\epsilon$ while $w\uuu 1=a$ is real and positive.
So the equation becomes
\[ 
|\zeta|^2\vvv \frac{\bar\zeta}{\epsilon}
= a^2+a
\]
The solution $\zeta$ is therefore real
and we are led to the algebraic equation

\[
s^2\vvv \frac{s}{\epsilon}=a^2+a
\]
Notice that we require that $|\zeta|>|w\uuu 1|=a$
so we seek the unique root $s$ for which $s>a$ and it is given by
\[ 
s=\frac{1}{2\epsilon}+
\sqrt{a+a^2+
4^{\vvv 1}\epsilon^{\vvv 2}}
\]
With $a$ kept fixed we obtain 
$|\zeta|\simeq \frac{1}{\epsilon}$ as $\epsilon\to 0$
which illustrates that 
the maximum norm of
the
interpolating function increases when $\epsilon\to 0$.

\bigskip


\noindent{\bf{0.9 Interpolation constants.}}
Let $E=(z\uuu 1,\ldots,z\uuu n)$
be given. Each $f\in\mathfrak{B}\uuu{n\vvv1}$
has $n\vvv 1$ many roots counted
with multiplicities in
$D$. In particular $f$ cannot vanish identically on
$E$, i.e the maximum norm
\[ 
|f|\uuu E= \max\, |f(z\uuu k)|>0
\] 
This leads us to define the number
\[
\tau(E)=\min\uuu {f\in\mathfrak{B}\uuu {n\vvv 1}} \, |f|\uuu E
\]
We have also the interpolation number:
\[ 
\mathfrak{int}(E)= \max\uuu {w(\cdot)}\, \rho(z(\cdot),w(\cdot))
\]
with the maximum taken over all $w$\vvv sequences with
$|w\uuu k|\leq 1$ for every $k$.
With these notations one has the following result which is due to Beurling:


\newpage

\noindent
{\bf{0.9 Theorem.}} \emph{For every finite set $E$ one has the equality}
\[
\tau(E)=\frac{1}{\mathfrak{int}(E)}\tag{*}
\]


\noindent
\emph{Moreover,  a function $f\in\mathcal B\uuu {n\vvv 1}$
which gives $|f|\uuu E= \tau(E)$ is unique up to a constant
and for such an extremal $f$ one has $|f(\alpha\uuu k)|=\tau(E)$
for every $1\leq k\leq n$.}


\bigskip

\noindent
\emph{Proof.}
With $n$ kept fixed
the family of $\mathcal B\uuu{n\vvv 1}$
enjoys normal properties in the sense of Montel so it
follows  that there exists at least some
extremal $f\in\mathcal B\uuu{n\vvv 1}$ such that
$|f|\uuu E= \tau(E)$.
Now we prove that
$|f(\alpha\uuu k)|= \tau(E)$ for each $k$.
For suppose strict inequality holds at some
$\alpha$\vvv point which we can take to be
$\alpha\uuu 1$.
Consider the Blaschke product
\[ 
B(z)=\prod\uuu {k=2}^{k=n}
\,\frac{z\vvv \alpha\uuu k}{1\vvv \bar\alpha\uuu k\cdot z}
\]
Rouche's theorem gives some $\delta>0$ such that
if $|\zeta|<\delta$ then the analytic function
$f(z)+\zeta\cdot B(z)$ has $n\vvv 1$ zeros in $D$
and we can therefore write
\[
f(z)+\zeta\cdot B(z)=\rho(\zeta)\cdot \psi\uuu\zeta(z)\tag{1}
\]
where the $\zeta$\vvv indexed $\psi$\vvv functions 
belong to $\mathcal B\uuu {n\vvv 1}$
and $\rho(\zeta)$ are complex numbers.
Notice that
\[
f(\alpha\uuu k)= \rho(\zeta)\cdot \psi\uuu\zeta(\alpha\uuu k)\tag{2}
\] 
hold when $2\leq k\leq n$.
Moreover, since
$|f(\alpha\uuu 1)|<\tau(E)$
it is clear by continuity that
if $\delta$ is sufficiently small then
$|\psi\uuu\zeta(\alpha\uuu 1)|<\tau(E)$
when $|\zeta|<\delta$.
Since $f$ is extremal we conclude from
(2) that there exists $\delta>0$ such that
\[
 |\zeta|<\delta\implies |\rho(\zeta)|\geq 1\tag{3}
\]
This gives a contradiction
since the absolute value of the $\rho$\vvv function
cannot have a relative minimum at $\zeta=0$
by the local complex expansion
of this $\rho$\vvv function  in Chapter III:XX.
\medskip


\noindent
\emph{Uniqueness.}
Let $f$ and $g$ be two extremal functions
so that $|f|\uuu E= |g|\uuu E= \tau(E)$ and suppose they are not identical.
For each $\zeta$ where $|\zeta|<\delta$ for a sufficiently small $\delta$
we can write
\[
1\vvv \zeta)\cdot f+\zeta\cdot g= \rho(\zeta)\cdot \psi\uuu\zeta(z)
\]
with $\psi\uuu\zeta\in \mathcal B\uuu{n\vvv 1}$.
The triangle inequality gives
\[
|1\vvv \zeta)\cdot f(\alpha\uuu k)+\zeta\cdot g(\alpha\uuu k)|\leq
\tau(E)
\]
for every $k$ and since $|\psi\uuu \zeta|\geq \tau(E)$ we get as above
that $|\rho(\zeta(\geq 1$
whenever $\zeta$ is sufficiently close to zero.
This contradicts again the complex expansion of this
$\rho$\vvv function from Chapter III.
\medskip

\noindent
{\bf{The equality $\mathfrak{int}(E)= \frac{1}{\tau(E)}$.}}
To begin with, let $f$ be the unique  extremal   above which gives
an $n$\vvv tuple of points on the unit circle so that
\[ 
f(\alpha\uuu k)= \tau(E)\cdot e^{i\theta\uuu k}
\]
The Nevanlinna\vvv Pick theorem shows that
$\frac{f((z)}{\tau(E)}$
has smallest maximum norm over $D$ when 
the $n$\vvv tuple $\{w\uuu k= e^{i\theta\uuu k}\}$.
This implies that
\[
\mathfrak{int}(E)\geq \frac{1}{\tau(E)}
\]
To prove the opposite inequality
we consider some $n$\vvv tuple
$\{w\uuu \bullet\}$ for which the interpolating function
$g(z)$ has the maximum norm $|g|\uuu D=\mathfrak{int}(E)$.
Theorem 0.1 gives
\[
 g=\mathfrak{int}(E)\cdot f\quad\text{where}\quad f\in\mathcal B\uuu{n\vvv 1}
\]
This entails that
\[ 
\tau(E)\leq |f|\uuu E\leq
\frac{1}{\mathfrak{int}(E)}
\]
and the requested equality (*) in Theorem 0.9 follows.
 
 







\newpage











\

\centerline{\bf 1. The Lindel�f-Pick principle.}

\bigskip


\noindent {\bf Introduction.}
The non-euclidian metric on $D$
is defined by
\[ 
\frac{|dz|}{1-|z|^2}\quad\colon\, |z|<1\tag{0.1}
\]
When $D$ is equipped with this metric
one gets  a model of
hyperbolic geometry  in the sense of Bolyai and 
Lobatschevsky which led to an  intense geometric study
around 1890, foremost  by F. Klein and
H. Poincar�. We shall not enter a detailed discussion about
the geometry
since
our main concern  is to apply the  metric (0.1) to derive
inequalities for 
analytic functions.
In a work from 1916, Lindel�f and Pick
discovered that
very analytic function $\phi(z)$ in the unit disc with maximum norm
one at most decreases the
metric (0.1).
This result is called the Lindel�f-Pick principle
and is proved in Theorem XX below.
In section XX it is used to
prove a result by
Caratheodory and Julia concerned with the boundary behaviour
of analytic functions.

\bigskip












\noindent
{\bf{1. Schwarz' inequality}}
The non-euclidian distance between
two point  $z_1$ and $z_2$  in $D$
will be  denoted by
\[
 \mathfrak{h}(z_1,z_2)\tag{1.1}
\]


\noindent
To grasp this distance function we first notice the equality:
\[ 
\mathfrak{h}(0,z)=\frac{1}{2}\cdot
 \text{Log}\,\frac{1+|z|}{1-|z|}\tag{*}
\]
Indeed, (*) follows  since it is obvious from
(0.1) that the geodesic curve from the origin to a point  
 $z\in D$ is  the ray from 0 to $z$. So   with $|z|=r$
one computes
 \[ 
 \int_0^r\,\frac{ds}{1-s^2}
 \]
which after integration gives
(*).
Next, with $a\in D$ we consider a M�bius transformation:
\[ 
w=\frac{z-a}{1-\bar a\cdot z}\implies
\frac{dw}{dz}= \frac{1-|a|^2}{(1-\bar a \cdot z)^2}
\]
At the same time we notice that
\[ 
1-|w|^2= \frac{|1-\bar a z|^2-|z-a|^2}{|1-\bar a \cdot z|^2}=
(1-|a|^2)\cdot\frac{1-|z|^2}{|1-\bar a \cdot z|^2}
\]
From this the reader may deduce that
the M�bius transform preserves the $\mathfrak h$-metric.
\medskip

\noindent
{\bf 1.1 Example.} 
Take $z_1=1/2$ and $z_2=e^{i\theta}/2$
with some $0<\theta<\pi$.
Now
\[ z\mapsto \frac{z-1/2}{1-z/4}
\]
sends $z_1$ to the origin. It follows that
\[
\mathfrak{h}(1/2,e^{i\theta}/2)=
\frac{1}{2}\cdot
 \text{Log}\,\frac{1+r}{1-r}\quad\colon\, 
r=\frac{2\cdot |e^{i\theta}-1|}{|2-e^{i\theta}|}
\]


\noindent
The following
consequence of Schwarz inequality was discovered by
G. Pick in 1915. 


\medskip


\noindent 
{\bf 1.2 Theorem.}
\emph{Let $\phi\colon\,D\to\Omega$
be a conformal map from the unit disc onto a simply
connected domain contained in $|w|<1$. Then
the non-euclidian metric decreases.}
\medskip



\noindent \emph {Proof.}
Let $z_0\in D$ and set
$w_0=\phi(z_0)$. The quotient
\[ 
G(z)= \frac{\phi(z)-w_0}{1-\bar w_0\phi(z)} \, \colon\,
\frac{z-z_0}{1-\bar z_0z} 
\]
Since

\[ \lim\, \frac{|z-z_0|}{|1-\bar z_0z|}=1\quad\text{as}\quad |z|\to 1
\] 
we see that
$|G(z)|\leq 1$ holds for all $z\in D$. With $z=z_0$ we have
\[ 
G(z_0)= \phi'(z_0)\cdot \frac{1-|z_0|^2}{1-|\phi(z_0)^2|}
\]
Since $z_0\in D$ was arbitrary we get the
differential inequality
\[ 
\frac{|d\phi(z)|}{|1-\phi(z)|^2}\leq \frac{|dz|}{1-|z|^2}
\]
and this is precisely the assertion in Pick's theorem.

\medskip

\noindent
{\bf{The Lindel�f-Pick principle.}}
Above $\phi$ was a  conformal mapping.
Since the $\mathfrak{h}$-metric  is
defined locally the inequality in Pick's theorem extends to analytic functions
in $D$ of absolute value $<1$ and leads to the following general
result:
\bigskip

\noindent
{\bf 1.3 Theorem}
\emph{Let $\phi(z)\in\mathcal O(D)$ have maximum norm
$\leq 1$. Then $\phi$ decreases the $\mathfrak{h}$-metric.}
\medskip

\noindent
{\bf{Remark.}}
Thus, if we set $w=\phi(z)$ and
$z_1,z_2$ is a pair in the unit disc $D_z$ one has
\[ 
\mathfrak{h}(\phi(z_1),\phi(z_2))\leq
\mathfrak{h}(z_1),z_2)\tag{*}
\]
\medskip

\noindent 
{\bf 1.4 The $\mathfrak{h}$-metric in half-spaces.}
Passing to the right half-plane $U_+$ where
$\mathfrak{Re}(w)>0$,
the non-euclidian metric  is obtained via the conformal map
\[
z\mapsto w=\frac{1+z}{1-z}
\]
From this it follows that
\[ 
\frac{|dz|}{1-|z|^2}\mapsto
2\cdot\frac{|w+1|^4\cdot |dw|}{|w+1|^2-|w-1|^2}
\]
So with  $w=\xi+i\eta$ the non-euclidian metric in the right half-plane becomes
\[
\frac{|w+1|^4\cdot |dw|}{2\xi}\tag{*}
\]




\medskip

\noindent 
Next, 
the Lindel�f-Pick principle applies after
a conformal mapping from $D$ onto any other simply connected
domain $\Omega$ where one then
regards analytic functions $g\in\mathcal O(\Omega)$ such that
$g(\Omega)\subset\Omega$.
\medskip

{\noindent
{\bf 1.5 Example.}
Let 
$\Phi(z)=u(x,y)+iv(x,y)\in\mathcal O(U^+)$
be such that its real part $u$ is positive in $U_+$.
The Lindel�f-Pick principle applies to
$\Phi$ and using (*) in (1.4) one has the following result:

\medskip

\noindent
{\bf 1.6 Proposition.} \emph{To every $k>0$ there exists another
constant $k^*$ such that
the following inequality holds for
every pair of points $z_0=x_0+iy_0$ and $z_1=x_1+iy_1$ in $U_+$:}
\[
|\Phi(x_1+iy_1)|\leq |v(x_0+iy_0)|+k^*\cdot \frac{x_1\cdot u(x_0,y_0)}{x_0}
\quad\colon\, |y_1|<k\cdot x_1
\]


\noindent 
{\bf 1.7 Exercise.}
Try to prove this result. If necessary, consult
the text-book   [Nevanlinna: page 59-61]
for a proof where it is also shown
that for each $k>0$
one can take
\[
k^*=3+2(k+1)^2\quad\colon\,\,
\text{provided that}\,\, x_1>x_0\quad\text{and}\quad
x_1>|y_0|\tag{*}
\]

\newpage








\noindent
\centerline {\bf{2. A result by  Julia.}} 


\medskip

\noindent
Let $\phi\in\mathcal O(D)$
be such that $|\phi(z)|<1$ when $z\in D$ and 
consider the boundary point $z=1$.
\medskip

\noindent {\bf 2.1 Theorem.} \emph{For every $e^{i\theta}$
there exists the limit}
\[ 
c(\theta)=\lim_{z\to 1}\, \frac{|e^{i\theta}-\phi(z)|}
{|1-z|}\quad\colon 0\leq c(\theta)\leq +\infty\tag{1}
\]


\noindent
\emph{where the limit $z\to 1$ is taken in any Fatou sector at 1. Moreover,
if $\theta$ is such that the limit
$0<c(\theta)<\infty$ then 
there exist the Fatou limits}:
\[
\phi'(z)\to c(\theta)\cdot e^{i\theta}\quad\colon\,
\text{arg}\,\frac{e^{i\theta}-\phi(z)}{1-z}\to \theta\tag{2}
\]
\emph{and the following inequality holds}
\[ 
\frac{1-|\phi(z)|^2}{|e^{i\theta}-\phi(z)|^2}\geq
\frac{1}{c}\cdot \frac{ 1+|z|}{1-|z|}\quad\colon\, z\in D\tag{3}
\]
\medskip

\noindent
{\bf Remark.}
Of course, only the  case when
$c(\theta)<\infty$  is of  interest. Notice that this 
finiteness only can occur for at most one
$\theta$-value.
The
theorem  above was the starting point for an  extensive study of
boundary values of analytic functions in Julia's work
[Ju] and has  later led to a far-reaching study about 
Julia sets in  complex dynamics. See [Carleson-Garnett] for this more recent and
advanced theory in function theory.
The reader may also consult Chapter IV in [Caratheodory]
for an account of Julia's original theorem
where some geometric interpretations appear.

\bigskip



\centerline{\emph{Proof of Theorem 2.1}}
\medskip

\noindent
Applying the two  conformal mappings
\[
z\mapsto\frac{1+z}{1-z}\quad\colon\,w\mapsto
\frac{e^{i\theta}+w}{e^{i\theta}-w}
\]
we can work in the right
half plane where $z=1$  has been mapped into
the point at infinity and $\phi$ has become an analytic function
\[ 
\Phi(x+iy)=u(x+iy)+iv(x+iy)\quad\colon u(x,y)>0
\,\,\text{for all}\,\, (x,y)\in U_+
\]


\medskip


\noindent
The crucial step in the proof  is
to show the result below:
\medskip

\noindent 
\emph{Let $\Phi=u+iv$ be an arbitrary
analytic map from $U_+$ to $U_+$ and assume that}
\[ 
\min_{x+iy\in U_+}\, \frac{u(x+iy)}{x}=0\tag{*}
\]
\emph{Then it follows that}
\[ \lim_{x\to+\infty}\, 
\frac{u(x+iy)}{x}=0\quad\colon\, \text{holds uniformly
inside any  Fatou sector}
\,\,|y|<kx\quad\colon\, k>0
\]
\medskip

\noindent
To prove
this we take some
$k>0$ and for each  $\epsilon>0$
the hypothesis (*) gives a point
$z_0=x_0+iy_0$ in $U_+$ such that
\[ 
\frac{u(x_0,y_0)}{x_0}<\epsilon\tag{1}
\]
Next, if $z=x+iy$  stays in the Fatou sector
$|y|<k|x|$ and $x_1$ is large then   Proposition 1.6 gives:
\[
|\Phi(x+iy)|\leq
|v(x_0+iy_0)|+k^*\cdot \frac{x\cdot u(x_0+iy_0)}{x_0}
<|v(x_0+iy_0)|+\epsilon\cdot k^*\cdot x
\]
\medskip

\noindent
In particular we have
\[
\frac{u(x+iy)}{x}<\frac{|v(x_0+iy_0)|}{x}+\epsilon\cdot k^*
\]
Since $\epsilon>0$ can be chosen arbitrary small the conclusion after (*) follows.


\medskip

\noindent
\emph{Proof continued.}
Next, suppose that
\[ 
c=\min_{x+iy\in U_+}\, \frac{u(x+iy)}{x}>0\tag {1}
\]
is positive. The result above applies
to $\Phi(z)-cz$
and hence $\frac{\Phi(z)}{z}\to c$
holds uniformly  as $|z|\to\infty$ inside any Fatou sector
$|y|<k|x|$. Moreover, this
gives:

\[ 
\liminf_{x\to\infty}\,
\frac{u(x,y)}{x}=c\tag{2}
\]
\medskip

\noindent
Let us no consider the complex derivative of $\Phi$
assuming that (1) above holds for some
$c>0$.
\bigskip




\noindent
\emph{Sublemma}
\emph{One has}
\[
 \lim_{z\to\infty}\, \Phi'(z)=c
\]
\emph{where this limit holds uniformly while
$z$ stays in any given
Fatou sector.}
\medskip

\noindent
\emph{Proof.} Replacing $\Phi$ by $\Psi(z)=\Phi(z)-cz$
it suffices to show that
\medskip
\[ 
\lim_z\, \Psi'(z)=c\quad\colon\text{uniformly when the limit is in a Fatou sector}\tag{i}
\]


\noindent To show (i) we proceed as follows. Consider some
$0<p<1$ and choose also some $q$ so that
$p<q<1$.
For every $r>0$
we consider the disc 
\[
\Delta_r=\{|z-r|<q \cdot r\}
\]
Since $q<1$ this disc stays in
a fixed Fatou sector for all large $r$ and 
Cauchy's inequality gives
\medskip
\[ 
|\Psi'(z)|
\leq \frac{qr}{2\pi}\,\int_0^{2\pi}\,
\frac{|\Psi(r+qre^{i\theta})|}{
|r+qre^{i\theta}-z|^2}\cdot d\theta
\quad\colon\quad
z\in\Delta_r\tag{ii}
\]
\medskip

\noindent
Next, if $\epsilon>0$  Propostion 1.6 gives
some large $r^*$ such that 
\[
|\frac{\Psi(\zeta)}{\zeta}|<\epsilon\quad\colon 
|\zeta-r|= qr\quad\colon\, r\geq r^*\tag{iii}
\]
Hence, if $|z-r|\leq pr$, the Cauchy inequality from (ii)
and a  computation which is left to the reader
gives:
\[
|\Psi'(z)|\leq
\epsilon\cdot\frac{q(1+q)}{(q-p)^2}\tag{iii}
\]
\medskip

\noindent
This proves that
$\Psi'(z)\to 0$ holds uniformly when  $z$ stays in the sector
\[
|\text{arg}\, z|< \text{arc.sin}(p)
\]
Above $p<1$ is arbitrary
which therefore gives
the Caratheodory-Julia theorem after   we have  returned to
the unit disc  via a conformal map between  $D$ and 
$U_+$. 






\newpage

\centerline{\bf\large 3. Some geometric results}


\bigskip



\noindent
{\bf 3.1 A study of convex domains.}}
Let $\Omega$ be a bounded convex domain and  $p\in\Omega$ 
an interior point. The convexity implies that
if we start at some boundary  point $q_0\in\partial\Omega$ where
$q_0-p$ is real and positive, then we obtain a function
\[ 
\phi\mapsto q(\phi)\quad\colon\,
\text{arg}\,[q(\phi)-p\,]=\phi\quad\colon q(\phi)\in \partial\Omega\tag{*}
\]
where $q(2\pi)=q_0$ holds after one turn. The $q$-function is continuous and
1-1, i.e. a homeomorphism between the unit circle and
$\partial\Omega$. 
Let $g(\phi)$ be a
non-negative continuous function on $T$, i.e   here
$g(2\pi)=g(0)$. We get $g^*\in C^0(\partial\Omega)$ satisfying
\[ 
g^*(q(\phi))=g(\phi)
\]
Starting from $g^*$ we solve the Dirichlet problem and find the
harmonic function $G^*$ in $\Omega$ which extends $g^*$.
With these notations we have
\medskip


\noindent
{\bf Theorem 3.2} \emph{One has the inequality}
\[ 
G^*(p)\leq\frac{1}{\pi}\,\int_0^{2\pi}\, g(\phi)d\phi
\]


\noindent{\bf Remark.}
The inequality is of special interest when $p$ approaches the boundary.
Before Theorem 3.2 is proved we consider
a general situation.
Let $W$ be any bounded Jordan domain and $p\in W$
an interior point. Let $a,b$ be two points  on $\partial W$.
Denote by $\gamma$ the Jordan subarc of $\partial W$ 
which joins $a$ and $b$.
Let $L$ be the line passing through these two points.
Suppose that the two infinite half lines from $a$ and $b$ 
are outside $W$, i.e. $W\cap L$ is contained in
the line segment $(a,b)$.
Now $L$ cuts $W$ into two halfs. Let $W^*$ be one of these.
Given a point $p\in W^*$ we shall find an upper bound
for the harmonic measure
$\mathfrak{m}_W(p;\gamma)$.
After a rotation and a translation we may
assume that $a=m$ and $b=-m$ for some $m>0$, i.e. 
$[a,b]$ is an interval on the real axis and that $W^*$
is contained in the upper half plane $U^+=\mathfrak{Im}(z)>0$.
Now $W\subset U$ and Carleman's principle from XX
gives:

\[
\mathfrak{m}_W(p;\gamma)\leq
\mathfrak{m}_{W^*}(p:[a.b])\leq
\mathfrak{m}_U^+(p:[a.b])\tag {1}
\]


\noindent
By the result in XXX the last term is equal to $\frac{1}{\pi}\cdot\alpha$
where $\alpha$ is the angle formed by $a-p$ and $b-p$.
\bigskip



\noindent 
\emph{Proof of Theorem 3.2.}
Consider a small arc $\gamma\subset\partial\Omega$ 
which by the parametrisation (*) above is defined by some $\phi$-interval 
$\phi_*\leq\phi\leq\phi^*$.
Let $\mathfrak{m}_\Omega(p:\gamma)$ be the harmonic measure at
$p$ with respect to this boundary arc.
We can apply the  inequality (1)  and conclude that
\[
\mathfrak{m}_\Omega(p:\gamma)\leq \phi^*-\phi_*
\]
Now the Theorem  3.2 follows after an  integration over $0\leq\phi\leq 2\pi$
where we  use that $G^*(p)$ is evaluated by
the integral of $g^*$ over $\partial\Omega$
with respect to the positive measure on
$\partial\Omega$ defined by the harmonic measure at $p$.


\bigskip


\centerline {\bf 3.3. On the range of analytic functions}
\bigskip

\noindent
Consider a domain $\Omega\in \mathcal D(C^1)$.
Let $\phi\in\mathcal O(\Omega)$ and assume it  extends to
$C^0(\bar\Omega)$.
The $\phi$-function is not supposed to be 1-1.
We get the domain
\[
 W=\phi(\Omega)
\]
Now
the following may occur: There exists a subset
$\Gamma$ of $\partial\Omega$ given as a finite union of
arcs
$\{\gamma_\alpha\}$
such that the image set $\phi(\Gamma)$
gives the boundary $\partial A$ of a domain $A\subset W$, i.e. here 
$A$ is a relatively compact subset of the connected open set
$W$.
Put
\[
\Omega_*=
\{z\in\Omega\quad\colon\,\phi(z)\in W\setminus A\}
\]


\noindent
Here $A\subset\partial (W\setminus A)$
and  we construct a harmonic measures as follows: If
$z\in\Omega_*$ we have $\phi(z)\in W\setminus A$ and get
the function
\[ 
z\mapsto\mathfrak{m}_{W\setminus A}\,(\phi(z);\partial A)
\quad\colon\, z\in\Omega_*
\]
Since $w\mapsto\mathfrak{m}_{W\setminus A}(w;\partial A)$
is a harmonic function in $W\setminus A$ it follows that the function above is harmonic in
$\Omega_*$.
Let us analyze its boundary values on $\partial \Omega_*$.
If $z\in\Omega_*$ approaches $\Gamma$, then
$\phi(z)\to A$ and hence
\[ 
\lim_{z\to\Gamma}\,
\mathfrak{m}_{W\setminus A}\,(\phi(z);\partial A)=1
\]


\noindent
Let us now regard the harmonic measure function
\[ z\mapsto \mathfrak{m}_{\Omega_*}(z:\Gamma)
\]
By definition it has boundary value 1 along $\Gamma$ and otherwise it is zero. 
Hence the maximum principle for harmonic functions gives:

\bigskip
\noindent {\bf 3.4 Theorem.} \emph{In the situation above one has
the
inequality:}
\[
\mathfrak{m}_{\Omega_*}(z:\Gamma)\leq
\lim_{z\to\Gamma}\,
\mathfrak{m}_{W\setminus A}\,(\phi(z);\partial A)\quad\colon\quad 
z\in \Omega_*
\]


\noindent 
{\bf Application.}
Using Theorem 3.4 we  prove a result due to  L�wner.
Let $w(z)\in\mathcal O(D)$ where $w(0)=0$ and
$|w(z)|<1$. Suppose there exists an arc $\gamma$ on the unit circle
such that $w(z)$ extends continuously up to $\gamma$ and that
\[ 
|\gamma(e^{i\theta})|= 1\quad\colon\quad e^{i\theta}\in\gamma
\]
Consider the image $w(\gamma)$ which 
is an another arc on the unit cicle.
With these notations  Theorem 3.4 gives
\medskip

\noindent 
{\bf 3.5 L�wner's inequality.}
\emph{The length of $w(\gamma)$ is $\geq$ the length of $\gamma$
and equality can only hold if $w(z)$ from the start is
$e^{i\alpha}z$ for some $\alpha$.}
\bigskip

\noindent
{\bf 3.6 Remark.} Actually L�wner proved a more precise 
result. Before it is announced  we insert a preliminary remark. 
Given $w(z)$ and an arc $\gamma\subset T$ where $|w(z)|=1$
one should expect that $|w(z)|$ must tend to 1 rather quick
as $z\in D$ approaches $\gamma$.
To put this in a precise form, L�wner proceeds as follows:
Up to a rotation we may take
\medskip
\[ \gamma=\{ e^{i\theta}\quad\colon
-a<\theta<a\}\quad\colon\, 0<a<\pi/2
\]
Now we consider the family of 
circles $K_\lambda$ passing the two end-points
$e^{ia}$ and $e^{-ia}$ where $\lambda>0$
expresses
the angle of intersection beteen
$K_\lambda$ and the unit circle $T$.
\medskip

\noindent
The reader should  draw a picture to see the situation where
the constraint  that the $\lambda$-numbers are chosen
so that  obtain a simple connected domain $\Omega_\lambda\subset D$
bordered by $\gamma$ and a portion of $K_\lambda$.
Next,   regard the image set $w(\Omega_\lambda)$.
On its boundary we find the arc $w(\gamma)$ which by the
hypothesis that $|w|=1$ on $\gamma$, is a sub-arc
of $T$.
At the same time we 
can start with the arc
$w(\gamma)$ and take the circle $K^*_\lambda$
which passes the end-points of $w(\gamma)$. This
gives a domain $\Omega^*_\lambda$ bordered by
$w(\gamma)$ and a subarc of the circle
$K^*_\lambda$.
With these notations the precise  result by L�wner goes as follows:
\medskip

\noindent {\bf 3.7 Theorem.}
\emph{For each $\lambda$ as above one has the inclusion}
\[ 
w(\Omega_\lambda)\subset \Omega^*_\lambda
\]
\medskip

\noindent
{\bf {3.8 Exercise.}}
Deduce  Theorem 3.7 from
Theorem 3.5. The strategy is that if 
$w(z)$ is \emph{outside} the set $\Omega_\lambda$
while $z\in \Omega^*_\lambda$, then
the inequality for harmonic measures is violated.
We leave it to the reader to discover this contradiction which gives
L�wner's theorem.
See also his article [L�:1]:\emph{Untersuchungen �ber
schlichte
konforme Abildungen} for details and further results.



\newpage








%\end{document}



