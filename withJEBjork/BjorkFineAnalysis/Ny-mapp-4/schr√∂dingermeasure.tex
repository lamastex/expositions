\documentclass{amsart}



\usepackage[applemac]{inputenc}
\addtolength{\hoffset}{-12mm}
\addtolength{\textwidth}{22mm}
\addtolength{\voffset}{-10mm}
\addtolength{\textheight}{20mm}

\def\uuu{_}

\def\vvv{-}

\begin{document}


\centerline{\bf\large{An automorphism on product measures}}

\bigskip


\noindent
{\bf{Introduction.}}
The main result is Theorem xx below which was proved by Beurling in
[Beurling]. Before we introduce    measure theoretic
notions  we
insert comments from [Beurling] about 
the  significance of Theorem XX.

\medskip


\noindent
{\bf{Schr�dinger equations.}}
The article
\emph{Th�orie relativiste de l'electron et l'interpr�tation
de la m�canique quantique} was published 1932 where
Schr�dinger raised a new and unorthodox question concerning Brownian motions
leading to new mathematical problems of considerable interest.
More precisely, consider a Brownian motion which takes place in a bounded
region $\Omega$ of some euclidian space
${\bf{R}}^d$ for some $d\geq 2$.
At time $t=0$
the densities of particles 
under observation is given by some
non\vvv negative function $f\uuu 0(x)$ 
defined on $\Omega$.
The density at a later time $t>0$
is classically  equal to
a function
$x\mapsto u(x,t)$ where
$u(x,t)$ solves the heat equation
\[
\frac{\partial u}{\partial t}= \Delta(u)
\] 
with boundary conditions 
\[
u(x,0)=f\uuu 0(x)\quad\text{and}\quad
\frac{\partial u}{\partial {\bf{n}}}(x,t)= 0
\quad\text{on}\quad \partial\Omega
\]
Schr�dinger took into the account
the reality of quantum physics which
means that in an  actual experiment
the observed density of particles at a time
$t\uuu 1>0$ does not coincide with $u(x,t\uuu 1)$.
He posed the problem to find the most probable development during the time
interval $[0,t\uuu 1)$ which leads to the state at time
$t\uuu 1$.
His major conclusion was that
the the requested density
function   which
substitutes the heat\vvv solution $u(x,t)$
should belong to a non\vvv linear class of functions formed by
products
\[ 
w(x,t)= u\uuu 0(x,t)\cdot u\uuu 1(x,t)
\]
where $u\uuu 0$ is a solution to (*) above
defined for $t>0$
while $u\uuu 1(x,t)$ is a solution to an adjoint equation
\[
\frac{\partial u\uuu 1}{\partial t}= \vvv\Delta(u)
\quad\colon\quad
\frac{\partial u\uuu 1}{\partial {\bf{n}}}(x,t)= 0
\quad\text{on}\quad \partial\Omega
\] 
defined when $t<t\uuu 1$.
This leads to a new type of Cauchy problems
where one asks if there exists a unique $w$\vvv function as above
satisfying
\[ 
w(x,0)= f\uuu 0(x)\quad\colon\quad w(x,t\uuu 1)=f\uuu 1(x)
\]
where $f\uuu 0,f\uuu 1$ are non\vvv negative functions
such that
\[
\int\uuu\Omega\, f\uuu 0\cdot dx=
\int\uuu\Omega\, f\uuu 1\cdot dx
\]
\medskip

\noindent
The solvability of this non\vvv linear boundary value problem was left open
by Schr�dinger and the search for  solutions has remained as an active field
in mathematical physics.
When $\Omega$ is a bounded set
and has a smooth boundary
one can use the Poisson\vvv Greens function for the
classical equation (*)  and 
rewrite Schr�dinger's equation  to 
a system of non\vvv linear integral equations.
The interested reader should consult the talk by I.N. Bernstein a the IMU\vvv congress at Z�rich 1932 for a first account about mathematical solutions to
Schr�dinger equations.
Examples occur already  on the product
of two copies of the real line where Schr�dinger's equations lead to
certain non\vvv linear equation for measures which goes as follows:
Consider the Gaussian density function


\[
\frac{1}{\sqrt{2\pi}}\cdot e^{\vvv t^2/2}
\]

\medskip

\noindent
Next, consider the family $\mathcal S^*$
of all non\vvv negative product measures
$\gamma\uuu 1\times\gamma\uuu 2 $ for which
\[
\iint g(x\uuu 1\vvv x\uuu 2)\cdot d\gamma\uuu 1(x\uuu 1)
\cdot d\gamma\uuu 2(x\uuu 2)=1
\]
The product measure gives another product measure 
\[
\mathcal T\uuu g(\gamma\uuu 1\times\gamma\uuu 2)=
\mu\uuu 1\times\mu\uuu 2
\]
where
\[
\mu\uuu 1(E\uuu 1)\cdot \mu\uuu 2(E\uuu 2)=
\iint \uuu {E\uuu 1\times E\uuu 2}\,
g(x\uuu 1\vvv x\uuu 2)\cdot d\gamma\uuu 1(x\uuu 1)
\cdot d\gamma\uuu 2(x\uuu 2)
\] 
hold for all pairs of bounded Borel sets.
Notice that $\mu\uuu 1\times \mu\uuu 2$
becomes a probability measure since (*) above holds.
With these notations one has

\medskip

\noindent
{\bf{Theorem.}}
For every product measure
$\mu\uuu1\times\mu\uuu2$ which in addition is a probability measure there exists
a unique $\gamma\uuu 1\times\gamma\uuu 2$ in $S\uuu g$
such that 
\[
\mathcal T\uuu g(\gamma\uuu 1\times\gamma\uuu 2)=\mu\uuu1\times\mu\uuu2
\]
\medskip


\noindent
In � x below we prove
the result above which actually appears as a special case of
Theorem XX
where the  $g$\vvv function is replaced by
an arbitrary  non\vvv negative and bounded function $k(x\uuu 1,x\uuu 2)$
such that
\[
\iint\uuu{{\bf{R}}^2}\,\log \, k\cdot dx\uuu 1dx\uuu 2>\vvv \infty
\]

  

\centerline{\bf\large{An automorphism on product measures}}

\bigskip


\noindent
Let $n\geq 2$ and consider an $n$\vvv tuple of
sample spaces $\{X\uuu\nu=(\Omega\uuu\nu,\mathcal B\uuu\nu)\}$.
We get the product space
\[ 
Y=\prod X\uuu\nu
\]
whose sample space is the set\vvv theoretic product
$\prod\,\Omega\uuu\nu$ and   Boolean $\sigma$\vvv algebra 
$\mathcal B$ generated by
$\{\mathcal  B\uuu \nu\}$.
\medskip


\noindent
{\bf{0.1 Product measures.}}
Let $\{\gamma\uuu\nu\}$ be an $n$\vvv tuple of signed measures on
$X\uuu 1,\ldots,X\uuu n$ where  each $\gamma\uuu\nu$
has a finite total variation. We get a unique measure $\gamma^*$
on $Y$ such that
\[ 
\gamma^*(
E\uuu 1\times\ldots\times E\uuu n)=
\prod\,\gamma\uuu\nu(E\uuu\nu)
\]
hold for every $n$\vvv tuple of $\{\mathcal B\uuu\nu\}$\vvv measurable sets.
We refer to $\gamma^*$ as the product measure. It is
uniquely determined  because $\mathcal B$ is generated
by
product sets $E\uuu 1\times\ldots\times E\uuu n)$
with each $E\uuu\nu\in\mathcal B\uuu\nu$.
When no confusion is possible we put
\[
 \gamma^*=\prod\,\gamma\uuu\nu
\]

\medskip

\noindent
{\bf{0.2 Remark.}}
The set of product measures is a proper non\vvv linear subset of all
measures on $Y$.
This is already seen when $n=2$ with two discrete sample spaces, i.e.
$X\uuu 1$ and $X\uuu 2$  consists of $N$
points for some integer $N$. A
Every $N\times n$\vvv matrix with 
non\vvv negative elements $\{a\uuu{jk}\}$
give a probability measure $\mu$ on
$X\uuu 1\times X\uuu 2$
when the double sum $\sum\sum\, a\uuu{jk}=1$
The condition that $\mu$ is a product measure is tha there exist
$N$\vvv tuples $\{\alpha\uuu j$ and $\{\beta\uuu k\}$
such that $\sum\,\alpha\uuu \nu=\sum\,\beta\uuu k=1$
and 
$a\uuu{jk}=\alpha\uuu j\cdot \beta\uuu k$.
\medskip


\noindent
{\bf{The operator $T\uuu k$.}}
Consider a positive $\mathcal B$\vvv measurable function $k$
such that  $k$ and $k^{\vvv 1}$
both are bounded functions.
Let $\mu$ be a non\vvv negative
product measure on $Y$ such that
\[ 
\int\uuu Y\, k\cdot d\mu=1
\]
Let $1\leq\nu\leq n$ and $g$ is some
$\mathcal B\uuu\nu$\vvv measurable function. Then we have the integral
\[
\int\uuu Y\, g^*\cdot k\cdot d\mu\tag{ii}
\]
where $g^*$ is the function on the product space defined by
\[
g^*(x\uuu 1,\ldots,x\uuu n)g(x\uuu n)
\]
This gives a measure denoted b $(k\mu)\uuu \nu$ on $X\uuu\nu$
such that (i) is equal to
$\int\, g\cdot (k\mu)\uuu\nu$
for all $g$ as above.
This gives the product measure

\[ 
T\uuu k(\mu)=\prod\, (k\mu)\uuu\nu
\]
\medskip

\noindent
It is clear that (i) entails that $T\uuu k(\mu)$ is a probability measure on
$Y$.
denote by $\mathcal S^*\uuu k$ the family of non\vvv negative product measures
satisfying (i) above, and similarly $\mathcal S^*\uuu 1$ is the set of
product measures which at the same time are probability measures.



\medskip

\noindent
{\bf{Theorem.}}
\emph{$T$ yields a homeomorphism between $S^*\uuu k$ and $S^*\uuu 1$}.

\medskip

\noindent
{\bf{Remark.}}
Above we refer to  the norm topology on the space of measure, i.e.
if $\gamma\uuu 1$ and $\gamma\uuu 2$ are two  measures on
$Y$ then
the norm $||\gamma\uuu 1\vvv \gamma\uuu 2||$ is the total variation of
the signed measure $\gamma\uuu 1\vvv\gamma\uuu 2$.
Recall from XX that the space of meauures on $Y$
is complete under this norm.
In particular, let $\{\mu\uuu \nu\}$
be a Cauchy sequence with respect to the norm where
each $\mu\uuu\nu\in \mathcal S\uuu 1$. Then there exists a strong limit
$\mu^*$ where $\mu^*$ again belongs to $\mathcal S^*\uuu 1$
and
\[
||\mu\uuu\nu\vvv \mu^*||\to 0
\]
\medskip

\noindent
{\bf{0.4 A variational problem.}}
The proof of Theorem 1 relies upon a variational
problem which we begin to describe  before
Theorem 1 is proved  in xx below.
Denote by $\mathcal A$
the linear space 
of functions on $Y$
whose elements are of the form
\[
a=g^*\uuu 1+\ldots+g^*\uuu n
\] 
where each $g^*\uuu\nu$ comes from a function
$g\uuu\nu$ on $X\uuu\nu$ as in (0.3
The exponential function $e^a$ becomes
\[
e^a=\prod\, e^{g^*\uuu\nu}
\]
If 
$\gamma^*$ is a product measure
with factors $\{\gamma\uuu\nu\}$, it follows that
$e^a\cdot \gamma^*$
is a  product measures with factors
$\{e^{g^*\uuu\nu}\cdot \gamma\uuu\nu\}$.
Next, for every pair  $\gamma\in \mathcal S^*\uuu 1$ and
$a\in \mathcal A$  we set
\[ 
W(a,\gamma)=\int\uuu Y\, (e^ak\vvv a)\cdot d\gamma 
\]

\medskip

\noindent
Keeping $\gamma$ fixed we set

\[
W\uuu *(\gamma)=
 \min\uuu{a\in\mathcal A}\,W(a,\gamma)
\]

\medskip

\noindent
The main step towards the proof of Theorem xx is the following:

\medskip

\noindent
{\bf{Proposition.}}\emph{Let $\{a\uuu \nu\}$ be a sequence in
$\mathcal A$ such that}
\[
\lim W(\gamma,a\uuu\nu)= W\uuu*(\gamma)
\]
Then the sequence  $\{e^{a\uuu \nu}\cdot \gamma\}$
converges to a unique probability measure
$\mu$ such that $T\uuu k(\gamma)=\mu$.


\bigskip

\noindent
The proof of Proposition xx is preceeded by the following two results.
\medskip


\noindent
{\bf{0. x. Lemma.}} 
\emph{Let $\epsilon>0$ and $a\in\mathcal A$ be such that
$W(a)\leq m\uuu *(\gamma)+\epsilon$. Then it follows that}

\[
\int\, e^a\cdot k\cdot \gamma\leq 
\frac{1+\epsilon}{1\vvv e^{\vvv 1}}
\]

\noindent
\emph{Proof.}
For every real number $s$   the function $a\vvv s$
again belongs to $\mathcal A$ and by the hypothesis 
$W(a\vvv s)\geq W(a)\vvv \epsilon$. This entails that

\[
\int\, e^a k\cdot d\gamma\leq
\int\uuu Y\, e^{a\vvv s}\cdot kd\gamma+s\int\, k\cdot d\gamma+\epsilon
\implies
\]
\[
\int (1\vvv e^{\vvv s})\cdot e^a\cdot kd\gamma\leq s+\epsilon
\]
Lemma 1 follows if we take $s=1$.


\medskip

\noindent
{\bf{0.X Lemma.}}
\emph{Let $\gamma\uuu 1$ and $\gamma\uuu 2$
be a pair of probability measures on $Y$. Let $\epsilon>0$
and suppose that}
\[ 
\bigl|\int\uuu Y\, G\uuu \nu\cdot d\gamma\uuu 1\vvv
\int\uuu Y\, G\uuu \nu\cdot d\gamma\uuu 2\bigr |\leq\epsilon
\] 
\emph{hold for every $1\leq \nu\leq n$ and every function $g\uuu\nu$ on $X\uuu\nu$ with maximum norm
$\leq 1$. Then the norm}
\[ 
||\gamma\uuu 1\vvv \gamma\uuu 2||\leq n\cdot \epsilon
\]
\medskip

\noindent
The proof is left to the reader
where the hint is to 
make repeated  use of  Fubini's theorem.
\medskip



\bigskip


\noindent
\emph{Proof of Proposition XX}
Let $\epsilon>0$ and consider a pair $a,b$ in $\mathcal A$ such that
$W(a)$ and $W(b)$ both are $\leq m\uuu *(\gamma)+ \epsilon$
where we also suppose that
$\epsilon\leq 1$.
Now $\frac{1}{2}(a+b)$
belongs to $\mathcal A$ and we get

\[
2\cdot W(\frac{1}{2}(a+b))\geq 2\cdot m\uuu *(\gamma)
\geq W(a)+W(b)\vvv 2\epsilon
\]
Next, notice that
\[
W(a)+W(b)\vvv
2\cdot W(\frac{1}{2}(a+b))=
\int\uuu Y\, [e^a+e^b\vvv 2\cdot e^{\frac{1}{2}(a+b))}]\cdot kd\gamma
\]
Now we use the algebraic identity
\[
e^a+e^b\vvv 2\cdot e^{\frac{1}{2}(a+b))}]=(e^{a/2}\vvv e^{b/2})^2
\]
It follows from (x\vvv x) that

\[
\int\uuu Y\, (e^{a/2}\vvv e^{b/2})^2\cdot k\cdot d\gamma \leq 2\epsilon\tag{iv}
\]
Next, we notice the identity

\[
|e^a\vvv e^b)|=
(e^{a/2}+e^{b/2})\cdot |e^{a/2}\vvv e^{b/2}|
\]
Using the Cauchy\vvv Schwarz inequality we get

\[
\bigl[\int\uuu Y\, |e^a\vvv e^b|\cdot k\cdot d\gamma]^2\leq
2\epsilon\cdot \int\uuu Y\, (e^{a/2}+e^{b/2})\cdot k\cdot d\gamma
\]
By the remark in XX the last factor is bounded by
a fixed constant and hence we have proved that

\[
\int\uuu Y\, |e^a\vvv e^b|\cdot k\cdot d\gamma
\leq C\cdot \sqrt{\epsilon}
\]
where $C$ is a fixed constant.
Replacing $C$ by $C/k\uuu *$ where $k\uuu *$ is the minimum of
$k$
we get
\[
\int\uuu Y\, |e^a\vvv e^b|\cdot d\gamma
\leq C\cdot \sqrt{\epsilon}
\] Since the left hand side majorizes the total variation of the signed measures
$e^a\cdot\gamma\vvv e^b/cdot\gamma$ we get Cauchy sequences with
respect to the strong norm and
conclude that there exists a unique limit measure $\mu$
where
$M(a\uuu\nu)\to m\uuu*(\gamma)$ implies that

\[
||e^{a\uuu\nu}\cdot\gamma\vvv \mu||\to 0
\]

\noindent
\emph{The equality $T(\mu)=\gamma$}.
To show this 
we study $a$\vvv functions in the minimizing sequence.
If $\rho\in \mathcal A$ is arbitrary we have
\[ 
W(a\uuu\nu+\rho)\geq W(a\uuu\nu)\vvv \epsilon\uuu\nu
\] 
where $\epsilon\uuu\nu\to 0$.
This gives
\[
\int\uuu Y\, [ke^{a\uuu\nu}(1\vvv\rho)+\rho]\cdot d\gamma\leq \epsilon\uuu\nu
\]
Assuming that the maximum norm
$|\rho|\uuu Y\leq 1$ we can write
\[
e^\rho=1+\rho+\rho\uuu 1
\]
where $0\leq \rho\uuu 1\leq \rho^2$.
Then we see that (xx) gives
\[
\int\uuu Y\,[\rho\vvv ke^{a\uuu\nu}\cdot \rho]\cdot d\gamma\leq
\epsilon\uuu\nu +\int\,\rho\uuu 1\cdot \gamma\leq
\epsilon+||\rho||\uuu Y^2
\]
where the last inequality follows since
$\gamma$ is a probability measure.
The same inequality holds with $\rho$ replaced by $\vvv \rho$
which entails that
\[
\bigl|\int\uuu Y\,(ke^{a\uuu\nu}\vvv 1)\cdot \rho\cdot d\gamma\bigr|
\leq\epsilon\uuu\nu+||\rho||\uuu Y^2
\]
At this stage we apply Lemma xx to the measure
$(ke^{a\uuu\nu}\vvv 1)\cdot d\gamma$ while we use
$\rho$\vvv functions in $\mathcal A$ of norm $\leq\sqrt{\epsilon\uuu\nu}$.
This gives the following inequality for the total variation:
\[
||ke^{a\uuu\nu}\vvv 1)\cdot\gamma||\leq n
\cdot \frac{1}{\sqrt{\epsilon}}\cdot
(\epsilon+\epsilon)= 2n\cdot \sqrt{\epsilon\uuu\nu}
\]




\bigskip


{\bf{Remark.}} For every positive number  $q$ and every
real number $\alpha$ one has the inequality
\[
e^q\cdot \alpha\vvv \alpha\geq 1+\log\,q
\]
Conclude  that
\[ 
W(a)\geq 1+\log k\uuu *
\] 
where $k\uuu *$ is the minium of the positive $k$\vvv function.





\newpage











\newpage







\bigskip






Let $n\geq 2$ and consider an $n$\vvv tuple of
sample spaces $\{X\uuu\nu=(\Omega\uuu\nu,\mathcal B\uuu\nu)\}$.
We get the product space
\[ 
Y=\prod X\uuu\nu
\]
whose sample space is the set\vvv theoretic product
$\prod\,\Omega\uuu\nu$ and its  Boolean $\sigma$\vvv algebra is
generated by
$\{\mathcal  B\uuu \nu\}$.
On $Y$ we consider a positive $\mathcal B$\vvv measurable function $k$
such that  $k$ and $k^{\vvv 1}$
both are bounded functions.
Denote by $\mathcal S\uuu k$ the family of
$\sigma$\vvv additive measures on $Y$ which are non\vvv negative 
and normalized so that
\[
\int\uuu Y\, k\cdot d\mu=1\tag{*}
\]
\medskip

\noindent
{\bf{Some notations.}}
If $g\uuu \nu$ is a bounded $\mathcal B\uuu\nu$ measurable function
on $X\uuu \nu$ we obtain the function $G\uuu\nu$ on $Y$
defined by
\[ 
G\uuu\nu(x\uuu 1,\ldots,x\uuu n)= g\uuu\nu(x\uuu 1)
\]
Notice that if $r$ is a real number then the $\mathcal B$\vvv 
measurable set
\[
\{G\uuu\nu<r \}=
\{g\uuu\nu <r\}\times X\uuu \times\ldots\times X\uuu n
\]
Next, if $\mu\in \mathcal S\uuu 1$
we obtain for each $1\leq\nu\leq n$
a measure on $X\uuu\nu$ defined
by the additive function on
bounded $\mathcal B\uuu\nu$\vvv measurable functions by
\[ 
g\uuu\nu\mapsto  \mu(k\cdot G\uuu\nu)= \int\uuu Y\, k\cdot G\uuu\nu\cdot d\mu
\]
The resulting measure on $X\uuu\nu$ is denoted by
$(k\mu)\uuu \nu$.
\medskip

\noindent
{\bf{0.1 Product measures.}}
Let $\{\gamma\uuu\nu\}$ be an $n$\vvv tuple of signed measures on
$X\uuu 1,\ldots,X\uuu n$. We assume that each $\gamma\uuu\nu$
has a finite total variation. Then we get a unique measure $\gamma^*$
on $Y$ such that
\[ 
\gamma^*(
E\uuu 1\times\ldots\times E\uuu n)=
\prod\,\gamma\uuu\nu(E\uuu\nu)
\]
hold for every $n$\vvv tuple of $\{\mathcal B\uuu\nu\}$\vvv measurable sets.
We refer to $\gamma^*$ as the product measure. It is
uniquely determined  because $\mathcal B$ is generated
by
product sets $E\uuu 1\times\ldots\times E\uuu n)$
with each $E\uuu\nu\in\mathcal B\uuu\nu$.
When no confusion is possible we put
\[
 \gamma^*=\prod\,\gamma\uuu\nu
\]

\medskip

\noindent
{\bf{0.2 Remark.}}
The set of product measures is a proper non\vvv linear subset of all
measures on $Y$.
This is already seen when $n=1$ and we have two discrete sample spaces, i.e.
with a finite set of points. say that $X\uuu 1$ and $X\uuu 2$ both consists of $N$
points for some integer $n$. A
Every $N\times n$\vvv matrix with 
non\vvv negative elements $\{a\uuu{jk}\}$
give a probability measure $\mu$ on
$X\uuu 1\times X\uuu 2$
when the double sum $\sum\sum\, a\uuu{jk}=1$
if $\mu$ is a product measure we can find
$n$\vvv tuples $\{\alpha\uuu j$ and $\{\beta\uuu k\}$
where each tuple has some equal to one and
$a\uuu{jk}=\alpha\uuu j\cdot \beta\uuu k$.
\medskip


\noindent
{\bf{0.3 The operator $T$}}.
Let $\mu$ be a measure in $\mathcal S\uuu k$.
To each $1\leq \nu\leq n$
we obtain the measure $(k\mu)\uuu\nu$ on $X\uuu\nu$
and get  the product measure 
\[
T(\mu)=\prod\, (k\mu)\uuu\nu
\]
If $1$ is the identity function on $Y$ we notice that
\[
\int\uuu Y\, 1\cdot dT(\mu)=
\prod\,\int\uuu{X\uuu\nu} 1\cdot d(k\mu\uuu\nu)=
\prod \int\uuu Y\, 1\cdot k\cdot d\mu=1
\]
Hence the product measure $T(\mu)$ 
is a probability measure on $Y$.
Denote the set of probability measures which in addition are product measures on
$Y$ by $\mathcal S\uuu 1^*$.
Similarly, denote by $\mathcal S^*\uuu k$ the family of measures in
$\mathcal S\uuu k$ which in addition are product measures.
We can restrict the $T$\vvv operator to $\mathcal S\uuu k^*$
and then the following holds.


\medskip

\noindent
{\bf{Theorem.}}
\emph{$T$ yields a homeomorphism between $S^*\uuu k$ and $S^*\uuu 1$}.

\medskip

\noindent
{\bf{Remark.}}
Above we use the norm topology on the space of measure, i.e.
if $\gamma\uuu 1$ and $\gamma\uuu 2$ are two in general signed measures on
$Y$ then
the norm $||\gamma\uuu 1\vvv \gamma\uuu 2||$ is the total variation of
the signed measure $\gamma\uuu 1\vvv\gamma\uuu 2$.
Recall from XX that the space of meauures on $Y$
is complete under this norm.
In particular, let $\{\mu\uuu \nu\}$
be a Cauchy sequence with respect to the norm where
each $\mu\uuu\nu\in \mathcal S\uuu 1$. Then there exists a strong limit
$\mu^*$ where $\mu^*$ again is a probability measure and
\[
||\mu\uuu\nu\vvv \mu^*||\to 0
\]
\medskip

\noindent
{\bf{A variational problem.}}
The proof of Theorem 1 relies upon a variational
problem which we begin to describe  before we prove
Theorem 1 in xx below.
Denote by $\mathcal A$
the linear space 
of functions on $Y$
whose elements are of the form
\[
a=G\uuu 1+\ldots+G\uuu n
\] 
where each $G\uuu\nu$ comes from a function
$g\uuu\nu$
given by (0.xx) above.
Every such $a$ is a bounded function and hence there exists
the exponential function $e^a$ on $Y$. Notice that this function is of the form
\[
e^a=\prod\, e^{G\uuu\nu}
\]
If we consider a product measure
$\gamma^*$ with factors $\{\gamma\uuu\nu\}$
we see that the measure $e^a\cdot \gamma^*$
is a new product measures with factors
$\{e^{G\uuu\nu}\cdot \gamma\uuu\nu\}$.
Now we ill define a variational problem where
product measures of this kind appear.
Let $\gamma\in \mathcal S^*\uuu 1$.
To each function $a\in \mathcal A$  we set
\[ 
W(a)=\int\uuu Y\, (e^ak\vvv a)\cdot d\gamma 
\]

\medskip

\noindent
{\bf{Remark.}} For every positive number  $q$ and every
real number $\alpha$ one has the inequality
\[
e^q\cdot \alpha\vvv \alpha\geq 1+\log\,q
\]
Conclude  that
\[ 
W(a)\geq 1+\log k\uuu *
\] 
where $k\uuu *$ is the minium of the positive $k$\vvv function.
Now we can introduce the number

\[
m\uuu *(\gamma)=
 \min\uuu{a\in\mathcal A}
\, \int\uuu Y\, (e^a \cdot k\vvv a )\cdot d\gamma\tag{*}
\]

\noindent
We are going to find a solution to this variational problem.
First we establish a certain upper bound which will
be used later on.
\medskip

\noindent
{\bf{Lemma.}} 
\emph{Let $\epsilon>0$ and $a\in\mathcal A$ be such that
$W(a)\leq m\uuu *(\gamma)$. Then it follows that}

\[
xxxx\leq xxx
\]

\noindent
\emph{Proof.}
In $\mathcal A$ we have the function $a\vvv s$
and by the hypothesis 
$W(a\vvv s)\geq W(a)\vvv \epsilon$ which gives

\[
\int\uuu Y\, e^{a\vvv s}\cdot kd\gamma\vvv a+s
\geq W(a)\vvv \epsilon\implies
\]
\[
(1\vvv e^{\vvv s})\cdot e^a\cdot kd\gamma\leq s+\epsilon
\]
Lemma 1 follows if we take $s=1$.

\medskip

\noindent
We shall  need another  preliminary result of independent interest.

\medskip

\noindent
{\bf{Lemma.}}
\emph{Let $\gamma\uuu 1$ and $\gamma\uuu 2$
be a pair of probability measures on $Y$. Let $\epsilon>0$
and suppose that}
\[ 
\bigl|\int\uuu Y\, G\uuu \nu\cdot d\gamma\uuu 1\vvv
\int\uuu Y\, G\uuu \nu\cdot d\gamma\uuu 2\bigr |\leq\epsilon
\] 
\emph{hold for every $1\leq \nu\leq n$ and every function $g\uuu\nu$ on $X\uuu\nu$ with maximum norm
$\leq 1$. Then the norm}
\[ 
||\gamma\uuu 1\vvv \gamma\uuu 2||\leq n\cdot \epsilon
\]
\medskip

\noindent
{\bf{Exercise.}} Prove this result where the hint is to 
make succesive use of  Fubini's theorem.
\medskip

\noindent
Now we announce the solution to the variational problem.



\medskip

\noindent
{\bf{Proposition.}}
\emph{Let  $\{a\uuu\nu\}$ be a  sequence 
in $\mathcal A$ such that}
\[
m\uuu *(\gamma)=
 \lim\uuu{\nu\to\infty}
\, \int\uuu Y\, (e^{a\uuu\nu}k\vvv a\uuu\nu )\cdot d\gamma
\]
\emph{Then the sequence 
$\{\mu\uuu\nu= e^{a\uuu\nu}\cdot\gamma\}$
converges strongly to a limit measure $\mu\in \mathcal S\uuu k$. 
Moreover, this limit measure is unique and $T(\mu)=\gamma$.}
\medskip

\noindent
\emph{Proof.}
Let $\epsilon>0$ and consider a pair $a,b$ in $\mathcal A$ such that
$W(a)$ and $W(b)$ both are $\leq m\uuu *(\gamma)+ \epsilon$
where we also suppose that
$\epsilon\leq 1$.
Now $\frac{1}{2}(a+b)$
belongs to $\mathcal A$ and we get

\[
2\cdot W(\frac{1}{2}(a+b))\geq 2\cdot m\uuu *(\gamma)
\geq W(a)+W(b)\vvv 2\epsilon
\]
Next, notice that
\[
W(a)+W(b)\vvv
2\cdot W(\frac{1}{2}(a+b))=
\int\uuu Y\, [e^a+e^b\vvv 2\cdot e^{\frac{1}{2}(a+b))}]\cdot kd\gamma
\]
Now we use the algebraic identity
\[
e^a+e^b\vvv 2\cdot e^{\frac{1}{2}(a+b))}]=(e^{a/2}\vvv e^{b/2})^2
\]
It follows from (x\vvv x) that

\[
\int\uuu Y\, (e^{a/2}\vvv e^{b/2})^2\cdot k\cdot d\gamma \leq 2\epsilon\tag{iv}
\]
Next, we notice the identity

\[
|e^a\vvv e^b)|=
(e^{a/2}+e^{b/2})\cdot |e^{a/2}\vvv e^{b/2}|
\]
Using the Cauchy\vvv Schwarz inequality we get

\[
\bigl[\int\uuu Y\, |e^a\vvv e^b|\cdot k\cdot d\gamma]^2\leq
2\epsilon\cdot \int\uuu Y\, (e^{a/2}+e^{b/2})\cdot k\cdot d\gamma
\]
By the remark in XX the last factor is bounded by
a fixed constant and hence we have proved that

\[
\int\uuu Y\, |e^a\vvv e^b|\cdot k\cdot d\gamma
\leq C\cdot \sqrt{\epsilon}
\]
where $C$ is a fixed constant.
Replacing $C$ by $C/k\uuu *$ where $k\uuu *$ is the minimum of
$k$
we get
\[
\int\uuu Y\, |e^a\vvv e^b|\cdot d\gamma
\leq C\cdot \sqrt{\epsilon}
\] Since the left hand side majorizes the total variation of the signed measures
$e^a\cdot\gamma\vvv e^b/cdot\gamma$ we get Cauchy sequences with
respect to the strong norm and
conclude that there exists a unique limit measure $\mu$
where
$M(a\uuu\nu)\to m\uuu*(\gamma)$ implies that

\[
||e^{a\uuu\nu}\cdot\gamma\vvv \mu||\to 0
\]

\noindent
\emph{The equality $T(\mu)=\gamma$}.
To show this 
we study $a$\vvv functions in the minimizing sequence.
If $\rho\in \mathcal A$ is arbitrary we have
\[ 
W(a\uuu\nu+\rho)\geq W(a\uuu\nu)\vvv \epsilon\uuu\nu
\] 
where $\epsilon\uuu\nu\to 0$.
This gives
\[
\int\uuu Y\, [ke^{a\uuu\nu}(1\vvv\rho)+\rho]\cdot d\gamma\leq \epsilon\uuu\nu
\]
Assuming that the maximum norm
$|\rho|\uuu Y\leq 1$ we can write
\[
e^\rho=1+\rho+\rho\uuu 1
\]
where $0\leq \rho\uuu 1\leq \rho^2$.
Then we see that (xx) gives
\[
\int\uuu Y\,[\rho\vvv ke^{a\uuu\nu}\cdot \rho]\cdot d\gamma\leq
\epsilon\uuu\nu +\int\,\rho\uuu 1\cdot \gamma\leq
\epsilon+||\rho||\uuu Y^2
\]
where the last inequality follows since
$\gamma$ is a probability measure.
The same inequality holds with $\rho$ replaced by $\vvv \rho$
which entails that
\[
\bigl|\int\uuu Y\,(ke^{a\uuu\nu}\vvv 1)\cdot \rho\cdot d\gamma\bigr|
\leq\epsilon\uuu\nu+||\rho||\uuu Y^2
\]
At this stage we apply Lemma xx to the measure
$(ke^{a\uuu\nu}\vvv 1)\cdot d\gamma$ while we use
$\rho$\vvv functions in $\mathcal A$ of norm $\leq\sqrt{\epsilon\uuu\nu}$.
This gives the following inequality for the total variation:
\[
||ke^{a\uuu\nu}\vvv 1)\cdot\gamma||\leq n
\cdot \frac{1}{\sqrt{\epsilon}}\cdot
(\epsilon+\epsilon)= 2n\cdot \sqrt{\epsilon\uuu\nu}
\]









\newpage



\noindent
{\bf{Introduction.}}
Abstract measure theory is often convenient to
achieve general results.
Here we expose material from Beurling's  article 
\emph{An automorphism of product measures}
where Theorem 1  is the main result.
In this theorem appears a continuous function $k$ defined on a product
$Y=X\uuu 1,\ldots,X\uuu n$ where each $X\uuu\nu$ is a locally
compact metric space.
Under the assumption that there are positive real numbers
$0<a<b$ such that the range of $k$ is confined to $[a,b]$
it will be proved that a certain operator $\mathcal K$ yields a homoeomorphism
from the space of
non\vvv negative Riesz measures $\mu$ on $Y$
normalized by the condition
\[
\int\, k\cdot d\mu=1
\]
to the space of probability measures on $Y$.
A much more involved case appears in the singular case, i.e. when
$k(x)$ for example can attain arbitrary small positive values.
In section 2 we discuss the singular case
for a product of two locally compact metric spaces.

\medskip

\noindent
{\bf{Schr�dinger equations.}}
A motivation for the abstract results in Section 1
come from the article
\emph{Th�orie relativiste de l'electron et l'interpr�tation
de la m�canique quantique} published 1932.
In the introduction to [Beurling]
the author ponits out that
Schr�dinger's raised a new and unorthodox question concerning Brownian motions
leading to new mathematical problems of considerable interest.
More precisely, consider a Brownian motion which takes place in a bounded
region $\Omega$ of some euclidian space
${\bf{R}}^d$ for some $d\geq 2$.
At time $t=0$
the densities of particles 
under observation is given by some
non\vvv negative function $f\uuu 0(x)$ 
defined on $\Omega$.
The density at a later time $t>0$
is then equal to
a function
$x\mapsto u(x,t)$ where
$u(x,t)$ solves the heat equation

\[
\frac{\partial u}{\partial t}= \Delta(u)
\] 
with boundary conditions $u(x,0)=f\uuu 0(x)$ and
\[
u(x,0)=f\uuu 0(x)\quad\text{and}\quad
\frac{\partial u}{\partial {\bf{n}}}(x,t)= 0
\quad\text{on}\quad \partial\Omega
\]
Schr�dinger took into the account
the reality of quantum physics which
means that in an  actual experiment
the observed density of particles at a time
$t\uuu 1>0$ does not coincide with $u(x,t\uuu 1)$.
He posed the problem to find the most probable development during the time
interval $[0,t\uuu 1)$ which leads to the state at time
$t\uuu 1$.
His major conclusion was that
the the requested density
function   which
substitutes the heat\vvv solution $u(x,t)$
should belong to a non\vvv linear class of functions formed by
products
\[ 
w(x,t)= u\uuu 0(x,t)\cdot u\uuu 1(x,t)
\]
where $u\uuu 0$ is a solution to (*) above
defined for $t>0$
while $u\uuu 1(x,t)$ is a solution to an adjoint equation
\[
\frac{\partial u\uuu 1}{\partial t}= \vvv\delta(u)
\quad\colon\quad
\frac{\partial u\uuu 1}{\partial {\bf{n}}}(x,t)= 0
\quad\text{on}\quad \partial\Omega
\] 
defined when $t<t\uuu 1$.
This leads to a new type of Cauchy problems
where one asks if there exists a unique $w$\vvv function as above
satisfying
\[ 
w(x,0)= f\uuu 0(x)\quad\colon\quad w(x,t\uuu 1)=f\uuu 1(x)
\]
where $f\uuu 0,f\uuu 1$ are non\vvv negative functions
such that
\[
\int\uuu\Omega\, f\uuu 0\cdot dx=
\int\uuu\Omega\, f\uuu 1\cdot dx
\]
\medskip

\noindent
The solvability of this non\vvv linear boundary value problem was left open
by Schr�dinger and the search for  solutions have been
studied by many mathematicians.
When $\Omega$ is a bounded set
and has a smooth boundary
one can use the Poisson\vvv Greens function for the
classical equation (*)  and in this way
rewrite Schr�dinger's equation  to 
a system of non\vvv linear integral equations.
We refer to page 190 in Beurling's article for details
how one arrives at such integral equations
and why this motivates the result in Theorem 1 below.

\bigskip


\centerline {\bf{1. Product measures.}}

\medskip

\noindent
Let  $X$ be a locally compact metric space. Denote by $C^b(X)$
the linear space of bounded real\\\ valued functions on $X$
which is a Banach a space equipped with the maximum norm.
The linear space of real\vvv valued Riesz measures on
$X$ with finite total variation is denoted by
$\mathfrak{M}(X)$ and the subclass of  
non\vvv negative measures of  total mass one is
denoted by $P^+(X)$.
Next,  consider an $n$\vvv tuple $X\uuu 1,\ldots,X\uuu n$
of locally compact spaces
and
let $Y=X\uuu 1\times\ldots\times X\uuu n$.
be the product space.
If $1\leq \nu\leq n$ and $\phi\in C^b(X\uuu\nu)$
we get the function $\Phi\uuu\nu$ on $Y$ defined by
\[
\Phi\uuu\nu(x\uuu 1,\ldots,x\uuu n)=
\phi\uuu\nu(x\uuu \nu)\tag{1}
\]

\noindent
Then, if  $\mu\in\mathfrak{M}(Y)$
we get the measure factors $\{\mu\uuu \nu\}$
where
\[
\mu(\Phi\uuu \nu)=
\mu\uuu\nu(\phi)\tag{2}
\] 
hold for each $\phi\in C^b(X\uuu\nu)$.
Conversely, let $\{\mu\uuu\nu\}$ 
be an $n$\vvv tuple of measures 
on $X\uuu 1,\ldots, X\uuu n$.
Then we get their product measure $\mu\uuu *$
where
\[
\mu\uuu *(E\uuu 1\times\ldots\times E\uuu n)=
\prod\, \mu\uuu\nu(E\uuu\nu)
\]
hold when   $\{E\uuu\nu\}$ are   Borel sets in
$X\uuu 1,\ldots, X\uuu n$.
\medskip

\noindent
{\bf{Remark.}}
Consider the special case when
each $\mu\uuu\nu$ is non\vvv negative
Then the product  measure $\mu\uuu *$ is
non\vvv negative. Let  $\{\gamma\uuu\nu\}$ be another
$n$\vvv tuple  of non\vvv negative measures
whose product measure $\gamma\uuu *=\mu\uuu *$.
For each fixed $1\leq\nu\leq n$
we take $\phi\in C^b(X\uuu\nu)$
and get

\[
\mu\uuu *(\Phi\uuu \nu)=
\prod\uuu{j\neq \nu}\, \mu\uuu (X\uuu j)\cdot \mu\uuu\nu(\phi)
\]
A similar formula holds for $\gamma\uuu *$ and   we conclude that
an equality $\mu\uuu *=\gamma\uuu *$
gives for each $\nu$ 
a constant $c\uuu \nu$ such that
\[ 
\gamma\uuu \nu=c\uuu\nu\cdot \mu\uuu \nu
\]
We obtain  a unique
$n$\vvv tuple of components representing $\mu\uuu *$
when we choose $\{\mu\uuu\nu\} $ so that each has
total mass given by
the $n$:th root of $\mu\uuu *(Y)$.

\bigskip

\noindent 
{\bf{The operator $\mathcal K$}}.
Consider some $k(x)\in C^b(Y)$
where $a\leq k(x)\leq b$ hold for some  pair $0<a<b$.
To each $\mu\in \mathfrak{M}(Y)$
we get  the measure $\mathcal K\uuu\mu$ on $Y$ which satisfies
\[ 
\mathcal K\uuu\mu(\prod\phi\uuu\nu(x\uuu \nu))=
\prod\, \mu(k(x)\cdot \Phi\uuu \nu(x))
\]
for every $n$\vvv tuple $\{\phi\uuu\nu\in C^b(X\uuu\nu)\}$.
Consider in particular the case when $\mu\in P^+(Y)$ and
\[
\int\uuu Y\, k\cdot d\mu=1\tag{*}
\]
Then  
$\mathcal K\uuu \mu$ has total mass one
and if $\gamma\uuu 1,\ldots,\gamma\uuu n$ are its normalised factors
we have
\[
\gamma\uuu \nu(\phi)= \mu(\Phi\uuu \nu\cdot k)
\]
when $\phi\in C^b(X\uuu\nu)$.
\medskip

\noindent
Denote by $P^+\uuu k(Y)$ the set of non\vvv negative measures
$\mu$ on $Y$ for which
(*) above holds. With these notations one has:
\bigskip

\noindent
{\bf{1. Theorem.}}
\emph{For each   function $k$ as above 
the operator
$\mathcal K$ yields a homeomorphism from
from $P^+\uuu k(Y)$ onto $P^+(Y)$
where each of these sets are equipped with the strong topology.}
\bigskip

\noindent
For the  proof of Theorem 1 we refer to  [Beurling].
At the end of the article a more involved case is studied.

\medskip

\noindent
{\bf{  A singular case.}}
Here we restrict the attention to the case $n=2$
and let $k(x\uuu 1,x\uuu 2)$ be a bounded and strictly positive 
continuous function on
$Y= X\uuu 1\times X\uuu 2$. Let $\mu\in P^+(Y)$
be such that
\[
\int\uuu Y\, \log k\cdot d\mu>\vvv \infty\tag{1}
\]
Under this integrability condition one has

\medskip

\noindent
{\bf{2. Theorem.}}
\emph{There exists a unique non\vvv negative measure
$\gamma$ on $Y$ such that
$\mathcal K(\gamma)=\mu$.}
\medskip

\noindent
{\bf{Remark.}}
In contrast to Theorem 1 the measure $\gamma$ need not have finite mass
but the proof shows that
$k$ belongs to $L^1(\gamma)$.
Concerning the integrability condition in Theorem 2
it can be relaxed a bit, i.e. it suffices to assume that

\[ 
\min\uuu{s>0}\,
\int \, (ke^s\vvv s)\cdot d\mu>\vvv \infty\tag{2}
\]

\noindent
As pointed out by Beurling the result in Theorem 2
can be applied to the case $X\uuu 1=X\uuu 2={\bf{R}}$
both are copies of the real line and
\[
k(x\uuu1,x\uuu 2)=g(x\uuu 1\vvv x\uuu 2)
\] where $g$ is the density of a Gaussian distribution which after  a
normalisation  of
the variance is taken to be
\[
\frac{1}{\sqrt{2\pi}}\cdot e^{\vvv t^2/2}
\]
So the integrability condition for $\mu$ in Theorem 2 becomes

\[
\iint\, (x\uuu 1\vvv x\uuu 2)^2\cdot d\mu(x\uuu 1,x\uuu 2)<\infty
\] 
\medskip

\noindent
The proof of Theorem 2 is given on page 218\vvv 220
in [loc.cit] and relies upon
the method and various estimates from the proof of
Theorem 1.
For higher dimensional cases,  i,e, with $n\geq 3$
Beurling gives the
following comments 
\medskip

\noindent
\emph{Theorem 1 relies heavily on the condition that $k\geq a$ for some
$a>0$.
If this lower bound condition is dropped
the individual equation $\mathcal K(\gamma)=\mu$ may still
be meaningful, but serious complications will
arise concerning the global uniqueness if $n\geq 3$
and the proof of Theorem 2 for the case n$n\geq 3$ cannot be duplicated.}






















\end{document}