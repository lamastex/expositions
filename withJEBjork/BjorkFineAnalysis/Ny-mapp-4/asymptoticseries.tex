%\documentclass{amsart}

%\usepackage[applemac]{inputenc}


%\addtolength{\hoffset}{-12mm}
%\addtolength{\textwidth}{22mm}
%\addtolength{\voffset}{-10mm}
%\addtolength{\textheight}{20mm}

%\def\uuu{_}

%\def\vvv{-}

%\begin{document}

\centerline{\bf\large {V. Uniqueness theorems for analytic functions.}}

\bigskip






















\noindent
\emph{0. Introduction.}
\bigskip

\noindent
\emph{A. A sharp version of the Phragm�n-Lindel�f theorem}

\bigskip

\noindent
\emph{B. Asymptotic series.}

\bigskip

\noindent
\emph {C. A uniqueness theorem for aymptotic series}

\bigskip







\bigskip




\noindent
\centerline {\bf Introduction.}
\medskip




\noindent
A  sharp
version of the Phragm�n-Lindel�f theorem is proved in 
Theorem A.2. It is preceeded by a differential inequality where
Carleman's result in
Theorem A.1 was inspired 
by earlier constructions due to Lindel�f.
Asymptotic series are studied in section B
where
earlier work by Borel  led Carleman to the  general 
construction in Theorem B.1. 
The   question of 
uniqueness is expressed via
Theorem B.6 and is settled via
solutions to a  variational problem  in
Section C.



\bigskip

\centerline{\bf{A. The Phragm�n-Lindel�f theorem.}}
\medskip

\noindent
Let $f(z)$ be an entire function.
To each $0\leq\phi\leq 2\pi$ we set
\[
\rho_f(\phi)=
\max_r\,|f(re^{i\phi})|\tag{*}
\]

\noindent
The text-book \emph{Le calcus des residues} by Ernst Lindel�f
contains examples of entire functions
$f$ where
$\rho_f(\phi)$ is finite for all $\phi$
with the exception 
$\phi=0$, i.e. only along  the positive real axis
the $\rho$-number fails to be  bounded.
An example is the entire function
\[ 
f(z)=
\frac{1}{z^2}\cdot \sum_{\nu=2}^\infty
\frac{z^\nu}{\bigl (\log\,\nu \bigr )^\nu}
\]
Here one verifies that that there exists a constant $k$ such that:
\[
|f(re^{i\phi})|\leq \text{exp}(e^{\frac{k}{|\phi|\cdot |2\pi-\phi|}}\,)\tag{**}
\]


\noindent
It turns out that the example above is essentially sharp. Namely, 
assume that
the $\rho$-number in (*) is finite for almost every
$\phi$. Then the $\rho_f$-function
cannot be too small, unless $f$ is reduced to
a constant. 
Before  Theorem A.1 is announced
we introduce
the non-negative function
\[ 
\omega(\phi)=\log^+\bigl[\,\log^+\rho_f(\phi)\,\bigr]\tag{***}
\]



\noindent
Since we have taken a two-fold logarithm 
$\omega(\phi)$ is considerably smaller  compared
to the $\rho$-function.
\medskip

\noindent
{\bf {A.1.Theorem.}}
\emph{For every non-constant entire function $f(z)$ one has}
\[
 \int_0^{2\pi}\,\omega(\phi)\cdot d\phi=+\infty
\]
\medskip

\noindent
\emph{Proof.} Assume that $f$ is not a constant.
Consider the maximum modulus function
\[
M(r)=\max_{|z|=r}\,|f(z)|
\]
By the ordinary Liouville theorem the $M$-function
increases to infinity. So we may assume that
$M(r)\geq 1$ when $r\geq r_*$ for some $r_*$.
Put
\[ v(r)=\log\, M(r)\quad\colon\quad
U(z)=\log\,|f(z)|\tag{i}
\]
Given $r\geq r_*$ we consider the domain
\[
\Omega_r=\{ U>\frac{v(r)}{2}\}\,\cap\, \{|z|<r\}\tag{ii}
\]
Next,  let $\zeta_r$ be some point on the circle
$|z|=r$ where
$|f(\zeta_r)|=M(r)$ where $\zeta_r$ always can be chosen
so that there exist
arbitrary small $\delta$
where $|f(\zeta_{r-\delta})|=M(r\vvv \delta)$
and $\lim_{\delta\to  0}\, \zeta_{r-\delta}=\zeta\uuu r$.



\noindent
Next,
in $\Omega$ we get the  connected component
$\Omega_*$
whose boundary contains
$\zeta_r$. Put
\[ \gamma=\partial\Omega_*\cap\,\{ |z|=r\}\tag{iii}
\]
Notice  that
\[ 
U(z)\leq\frac{v(r)}{2}\quad\colon\quad
z\in\partial \Omega_*\cap \{|z|<r\}\tag{iv}
\]
So if $W$ is the harmonic function in the disc $D_r$
with boundary values 1 on $\gamma$ and 0 on
$\{|z|= 1\}\setminus\gamma$ we have:
\[ 
U(z)-\frac{v(r)}{2}-
\frac{v(r)}{2}\cdot W(z)\leq 0\quad\colon\quad z\in\partial\Omega_*\tag{v}
\]


\noindent 
The maximum principle entails that (v) also holds in
$\Omega_*$. 
Hence there exist arbitrary small $\delta>0$ such that
\[
v(r-\delta)-\frac{v(r)}{2}-\frac{v(r)}{2}\cdot W(\zeta_{r-\delta})\leq 0\tag{vi}
\]


\noindent
Let $2 r\cdot\ell$ be the total 
length of the intervals which belong
to $\gamma$.
By the general inequality from XX we have
\[
W(\zeta_{r-\delta})\leq 
\frac{1}{2\pi}\int_{-\ell}^\ell\, \frac{r^2-(r-\delta)^2}
{r^2-2r(r-\delta)\text{cos}\,\theta+(r-\delta)^2}\tag{vii}
d\theta
\]
Let $h(r-\delta)$ denote the right hand side in (vii) which by (vi)
gives us  arbitrary small $\delta>0$ such that
\[
v(r-\delta)-\frac{v(r)}{2}-\frac{v(r)}{2}\cdot h(r-\delta)\leq 0\tag{viii}
\]
Rewriting this inequality we obtain
\[
\frac{v(r)-v(r-\delta)}{\delta}\geq
\frac{v(r)}{2}\cdot \frac{1-h(r-\delta)}{\delta}\tag{*}
\]
Next, from the definition of the $h$-function one has the limit formula

\[
\lim_{\delta\to 0}\frac{1-h(r-\delta)}{\delta}=
\frac{1}{2\pi}\cdot \frac{\text{cos}\,\ell}{\text{sin}\,\ell}\tag{ix}
\] 
Passing to the limit as $\delta\to 0$ in (vii) we get the differential inequality:
\[ 
v'(r)\geq 
\frac{v(r)}{2\pi r}\cdot 
\frac{\text{cos}\,\ell}{\text{sin}\,\ell}\tag{**}
\]


\noindent
Next, put
\[ \log\, r=s\quad\text{and}\quad 
 \log\,\frac{v(r)}{2}=g(s)
\]


\noindent
By derivation rules we see that (**) gives
\[
\frac{dg}{ds}\geq\frac{1}{2\pi}\cdot  
\frac{\text{cos}\,\ell}{\text{sin}\,\ell}\tag{***}
\]


\noindent
Next, identifying $\gamma$ with a subset of the periodic
interval $0\leq\phi\leq 2\pi$ it is clear that
the definition of the
$\omega$-function gives the inclusion

\[ 
\gamma\subset\{\omega(\phi)\geq g(s)\}\tag{x}
\]
So if $\lambda(s)$ is the Lebesgue measure of the
set 
$\{\omega(\phi)\geq g(s)\}$ then $\ell\leq\lambda(s)$
and  (***) gives
\[
\frac{dg}{ds}\geq\frac{1}{2\pi}\cdot  
\frac{\text{cos}\,\lambda(s)}{\text{sin}\,\lambda(s)}\tag{****}
\]


\noindent
Next, the inequality $\text{sin}(t)\geq \frac{2}{\pi}\cdot t$
gives a positive constant $k$ 
which is independent of $s$ such that
the following hold for sufficiently large $s$, i.e.
to ensure that
the corresponding $r$-value satisfies
$M(r)\geq 1$:
\[
\frac{dg}{ds}\geq \frac{k}{\lambda(s)}\tag{xi}
\]
Hence, starting from some sufficiently large
$s_0$ one has
\[ 
\int_{s_0}^s\,
\lambda(s)\cdot dg(s)\geq k(s-s_0)\tag{xii}
\]
\medskip

\noindent This inequality implies in particular  that
one has a divergent integral:
\[
\int_{s_0}^\infty\, 
\lambda(s)\cdot dg(s)=+\infty\tag{xiii}
\]
Finally,  the general equality for distribution functions 
from XXX gives:
\[
\int_0^{2\pi}\,\omega(\phi)\cdot d\phi=
\int_0^\infty\, 
\lambda(s)\cdot dg(s)\tag{xiiii}
\]
The last integral is $+\infty$ by (xiii) and  the requested
divergence  for the intergal of the $\omega$-function follows.

\bigskip

\noindent
{\bf Remark.}
At the end of the article [XXX]  Carleman  points
out  that the proof above  gives a sharp version of the Phragm�n-
Lindel�f theorem. More precisely one has  the following:
Let $f(z)$ be analytic in a sector
\[ 
S_\alpha=\{ z= re^{i\phi}\quad\colon\quad -\alpha<\phi<\alpha\}
\]
Define $\omega(\phi)$ as above when
when $-\alpha<\phi<\phi$. With these notations one has:
\medskip

\noindent
{\bf {A.2. Theorem.}}
\emph{Let  $f$ be bounded on the 
half-lines $\text{arg}(z)=\alpha$ and
$\text{arg}(z)=-\alpha$ and assume also that}
\[
\int_{-\alpha}^\alpha\,\omega(\phi)\cdot d\phi<\infty
\]
\emph{Then $f(z)$ is bounded in the whole sector.}

\medskip

\noindent
{\bf {A.3. Exercise.}}
Deduce Theorem A.2 from the preceeding results.







\bigskip








\centerline{\bf B. Asymptotic series.}

\bigskip

\noindent
{\bf Introduction.}
The  notion of asymptotic series was expressed 
as follows by Poincar�:
\medskip

\noindent
\emph{Let $f(z)$ be  complex-valued function
defined in some
subset $E$ of ${\bf{C}}$ and $z_0$ is a boundary point.
We say that $f$ has an asymptotic
series expansion at $z_0$ if there exists a sequence of complex numbers
$c_0,c_1,\ldots$ such that $\lim_{z\to z_0}\, f(z)=c_0$ and for each
$n\geq 0$ one has:}
\[
\lim_{z\to z_0}\,
(z-z_0)^{-n-1}\,\bigl[ f(z)-(c_0+c_1+\ldots+c_n z^n)\bigr ]=c_{n+1}\tag{*}
\]


\noindent
\emph{where the limit is   taken as $z$ stay in 
$E$.}
\medskip

\noindent
It is obvious that if
$f$ has an asymptotic expansion at $z_0$ then
the sequence $\{c_n\}$ is unique.
Constructions of functions which admit asymptotic expansions
appear in   Emile Borel's  thesis
\emph{Sur quelques points de la th�orie des fonctions} from 1895
and 
he  proved for example that for every 
sequence of real numbers
$\{c_n\}$ there exists a $C^\infty$-function $f(x)$ on the real line
whose Taylor expansion at $x=0$ is given
by
the sequence, i.e.
\[ 
\frac{f^{(n)}(0)}{n\,!}= c_n\quad\colon\quad n=0,1,\ldots
\]
Following
[Car: xx, page 29-31]
we prove a complex version of
Borel's result where
$D_+$ denotes  the open half-disc 
$\{\mathfrak{Re}(z)> 0\cap\{ |z|<1\}$.

\medskip

\noindent
{\bf B.1. Theorem.} \emph{To each sequence
$\{c_n\}$ of complex numbers there exists
a bounded analytic function $F(z)$�in $D_+$ which has an asymptotic
series expansion at $z=0$ given by $\{c_n\}$.}
\bigskip

\noindent
\emph{Proof.}
It suffices  to prove this when
$c_0=0$.
Let $a_1,a_2,\ldots$ be a sequence of positive real numbers
such that
$\sum\, a_\nu<\infty$.
Given $\{c_n\}$ we construct a sequence of 
functions $P_1(z),P_2(z),\ldots$ which are analytic in
the half plane
$\mathfrak{Re}(z)>0$ as follows: First
\[
P_1(z)=c_1z(1-\frac{z}{z+\epsilon_1})
\quad\colon\,
\epsilon_1=\frac{\alpha_1}{|c_1|}\implies
\tag{i}
\]
\[
|P_1(z)|=|c_1|\cdot \epsilon_1\cdot \frac{|z|}{|z+\epsilon_1}\leq
\alpha_1\quad\colon\quad\mathfrak{Re}(z)\geq 0\tag{ii}
\]
Now $P_1(z)$ has a series expansion at $z=0$:
\[ 
P_1(z)= \sum_{\nu=1}^\infty\, c_\nu^{(1)}\cdot z^\nu\tag{ii}
\]
Notice that the series converges in the disc
$|z|<\epsilon_1$. Set
\[ 
P_2(z)= \bigl[ c_2-c_2^{(1)}\bigr ]\cdot z^2\cdot \bigl(
1-\frac{z}{z+\epsilon_2}\bigr)\quad\colon\, |c_2-c_2^{(1)}|\cdot \epsilon_2\leq
a_2\tag{iii}
\]
With such a careful choice of a small positive $\epsilon_2$ we see that
\[
|P_2(z)|\leq a_2\cdot |z|\quad\colon\quad \mathfrak{Re}(z)\geq 0\tag{iii}
\]
Again we obtain a convergent series
at $z=0$:
\[
P_2(z)=P_1(z)= \sum_{\nu=2}^\infty\, c_\nu^{(2)}\cdot z^\nu\tag{iv}
\]
\medskip

\noindent
{\bf The inductive construction.}
Let $n\geq 3$ and suppose that
$P_1,\ldots,P_{n-1}$ have been constructed where we for each
$1\leq k\leq n-1$ have a series expansion
\[
P_k(z)= 
\sum_{\nu=k}^\infty\, c_\nu^{(k)}\cdot z^\nu\tag{v}
\]
Then we define
\[
P_n(z)=\bigl[ c_n-(c_n^{(1)}+\ldots+c_n^{(n-1)}\bigr]\cdot 
z^n\cdot \bigl(1-\frac{z}{z+\epsilon_n}\bigr)\quad\colon\quad
\bigl |\,c_n-(c_n^{(1)}+\ldots+c_n^{(n-1)}\,\bigr |\,\cdot \,\epsilon_n\leq\alpha_n
\]
\medskip

\noindent
So we obtain a new series at $z=0$:
\[ 
P_n(z)=\sum_{\nu=n}^\infty\, c_\nu^{(n)}\cdot z^\nu\tag{vi}
\]
Staying in the half-disc $D_+$, the inductive construction gives
\[ 
\max_{z\in D_+}\, |P_n(z)|\leq\alpha_n\quad\colon\quad n=1,2,\ldots
\]
Hence there exists  a bounded analytic function in $D_+$ defined by
\[ 
F(z)=P_1(z)+P_2(z)+\ldots
\]
At this stage we leave as an exercise to the reader to verify that
\[
\lim_{z\to 0}\,z^{-n-1}\cdot 
\bigl[\,  F(z)-(c_1z+\ldots+c_n z^n)\bigr ]\, = c_{n+1}
\]


\newpage


\centerline {\bf B.2. Uniqueness of asymptotic expansions.}
\medskip

\noindent
There exist functions whose asymptotic series is identically zero.
Here is an  example:
\[ 
f(z)= e^{-\frac{1}{z^2}}
\]
If $z= re^{i\theta}$ with
$-\pi/8\leq\theta\leq \pi/8$ we see that
\[
|f(re^{i\theta})|= \text{exp}\,
(-\frac{\text{cos}\,2\theta}{r^2})
\leq\text{exp}
(-\frac{1}{\sqrt{2}\cdot r^2})
\]
It follows that the asymptotic series at $z=0$ is
identically zero.
Via a conformal map from the half-disc $D_+$ 
to the unit circle we are led to the
following problem: Let $f(z)$ be analytic in the open unit disc
$D$. Suppose that
\[ 
\lim_{z\to 1}\, \frac{f(z)}{(1-z)^n}=0
\quad\colon\quad n=1,2,\ldots\tag{*}
\]
We seek growth conditions on $f$ in order that
(*) implies that $f$ is identically zero.
An answer to this uniqueness problem
was proved by Carleman in [Car].
Namely. consider a sequence of real positive numbers
$A_1,A_2,\ldots$.
To each $n\geq 1$ we put
\[ 
I_n=\text{exp}\bigl(\, \frac{1}{\pi}\int_1^\infty\, \log\,
\bigr[\,\sum_{\nu=1}^{\nu=n}\,
\frac{r^{2\nu}}{A_\nu^2}\,\bigr]\, \cdot dr\,\bigr)\tag{**}
\]
\medskip

\noindent
{\bf B.3. Definition.} \emph{Denote by
$\mathfrak{B}$
the set of all sequences
$\{A_n\}$ such that 
$\{I_n\}$ is bounded, i.e. there exists some $K$ such that}
\[
 I_n\leq K\quad\colon\quad n=1,2\ldots
\]


\noindent
In [Car: page  7-52 ] the following 
existence result is proved:
\medskip

\noindent
{\bf B.4. Theorem.}
\emph{To each sequence $\{A_n\}\in\mathfrak{B}$
there exists an analytic function
$f(z)$ in $D$ which is not identically zero and satisfies:}
\[
\frac{|f(z)|}{|1-z|^n} \leq A_n \quad\colon\quad n=1,2,\ldots\tag{1}
\]
\emph{while (*) holds.}
\medskip

\noindent
{A converse result.}
In [loc.cit ] appears  the converse to the
result which ensures uniqueness of the asymoptotic expansion at $z=1$.

\medskip

\noindent
{\bf B.5. Theorem.}
\emph{Let $\{A_n\}$ be a sequence of positive numbers
such that
there exists an analytic
function $f(z)$ in $D$ which is not  reduced to a constant and
satisfies (*) and (1) in Theorem B.4. Then 
$\{ A_n\}\in\mathfrak{B}$.}
\bigskip

\noindent
{\bf{Remark.}}
The results above show that if
$\{A_n\}$ is a sequence for which
$\{I_n\}$ is unbounded then
the asymptotic expansion at $z=1$ is unique for
every analytic function $f(z)$ satisfying (1) in Theorem B.4.
The proofs  of the two results above rely upon
a varational
problem which
is presented below 
while the
deduction after of the two cited results above are left to the reader who  may find
details in  [Carleman].
\bigskip






\centerline{\bf{C. A variational problem.}}





\bigskip

\noindent
Let 
$n\geq 1$ and $a_0,a_1,\ldots,a_n$ some $n$-tuple of 
non-negative real numbers where  $a_0>0$ is assumed.
Let $\mathcal O(*)$ denote the family of analytic functions
$f(z)$ in the unit disc satisfying
$f(0)=1$. Put
\[
I(f)=
\frac{1}{2\pi}\cdot \sum_{\nu=0}^{\nu=n}\, 
a_\nu^2\cdot \int_0^{2\pi}\, \frac{|f(e^{i\theta})|^2}{ |e^{i\theta}-1|^{2\nu}}
\cdot d\theta\quad\colon\quad I_*=\min_{f\in\mathcal O(*)}\,I(f)
\]
\medskip

\noindent
{\bf Remark.}
Above we have a variational problem. It turns out that
there exists a unique function $f_*(z)$
which yields a minimum. To find $f_*$ we shall use
the rational function:
\[
\Omega(z)=\sum_{\nu=0}^{\nu=n}\, a_\nu^2\bigl[ (1-z)(1-\frac{1}{z})\bigr]^{n-\nu}
\]
Notice that
\[
 \Omega(e^{i\theta})=a_0^2+
\sum_{\nu=1}^{\nu=n}\, a_\nu^2\cdot|e^{i\theta}-1|^{2n-2\nu}\tag{i}
\]
In particular $\Omega$ is real and positive on the unit circle
and by symmetry  it has
$n$ zeros $\rho_1,\ldots,\rho_n$
in the unit disc and $\frac{1}{\rho_1},\ldots,\frac{1}{\rho_n}$
are the zeros in the exterior disc which gives the factorization
\[ 
\Omega(z)=z^{-n}\cdot (-1)^n\cdot a_0^2\cdot
p_n(z)\cdot\prod\,
(z-\frac{1}{\rho_\nu})\quad\colon\,
p_n(z)=(z-\rho_1)\cdots (z-\rho_n) \tag{*}
\]


\noindent
Next, for every $f\in\mathcal O(D)$ with $f(0)=1$ we put
\[
\phi(z)=\frac{f(z)}{(1-z)^n}\tag{ii}
\]
Then (i) gives the equality:
\[ 
I(f)=\frac{1}{2\pi}\cdot \int_0^{2\pi}\,
\Omega(e^{i\theta})\cdot |\phi(e^{i\theta})|^2\cdot d\theta\tag{iii}
\]


\noindent
We will use the last expression to prove
\medskip

\noindent
{\bf C.1 Theorem.}\emph{
The variational problem has a unique solution whose
minimum $I_*(n)$ is achieved by the function}
\[ 
f_*(z)=\frac{(1-z)^n}{\prod\, 1-\rho_\nu\cdot z)}\tag{i}
\]
Moreover, 
\[ 
I_*(n)= I(f_*)=
\frac{1}{2\pi}\cdot \int_0^{2\pi}\,
\text{Log}\, \bigl[ \sum_{\nu=0}^{\nu=n}\, a_\nu^2\cdot\frac{1}{
(2\cdot\text{sin}\,\frac{\theta}{2})^{2\nu}}\bigr ]\cdot d\theta\tag{ii}
\]
\medskip

\noindent
\emph{Proof} 
By (iii) the variational problem
is equivalent to seek the minimum of
\[ 
\min_\phi\, I(\phi)=\frac{1}{2\pi}\cdot\int_0^{2\pi}\,
\Omega(e^{i\theta})\cdot |\phi(e^{i\theta})|^2\cdot d\theta\quad\colon
\phi(0)=1\tag{1}
\]
With $f_*$ as in (i) from  Theorem C.1 one has
\[
\phi_*(z)=  \frac{1}{\prod\, 1-\rho_\nu\cdot z)}\tag{2}
\]


\noindent
Now $f_*$ is a unique minimizing function in Theorem C.1
if we have proved the strict 
inequality
\[
I(\phi_*+h)<I(\phi_*)\tag{3}
\] 
for every analytic function $h(z)$ in $D$ such that  $h(0)=0$.
To show (3) we  
notice that (1) can be replaced by a complex line integral over
$|z|=1$ which gives
\[
I(\phi_*+h)=
\frac{1}{2\pi i}\cdot\int_{|z|=1}\,
\Omega(z)\cdot |\phi_*(z)+h(z)|^2\cdot \frac{dz}{z}=
\]
\[
I(\phi_*)+I(h)+
\frac{1}{2\pi i}
\cdot\int_{|z|=1}\,
\Omega(z)\cdot [\bar \phi_*(z)\cdot h(z)+ \phi_*(z)\cdot \bar h(z)]
\cdot \frac{dz}{z}\tag{4}
\]
\medskip

\noindent
Since $I(h)>0$ whenever $h\neq 0$ the requested strict inequality follows if
we show that the last integral is zero.
To prove this we notice that the construction of $\phi_*$ gives the equation

\[
\Omega(e^{i\theta})\cdot \bar \phi(e^{i\theta})=
(-1)^n \cdot a_0^2\cdot \frac{1}{\prod\, (e^{i\theta}-\bar \rho_\nu)}\tag{5}
\]



\noindent
Now
\[ 
k(z)= \frac{1}{\prod\, (z-\bar \rho_\nu)}
\] 
\medskip

\noindent
is analytic in $D$ and since
$h(0)=0$ it follows that

\[
\int_{|z|=1}\, k(z) h(z)\cdot \frac{dz}{z}= 0
\]
This proves that the first term in the integral from (5) vanishes and the second
is its complex conjugate since
$\Omega$ was real on $T$.
Hence $f_*$ yields the unique minimizing function of the variational problem.
The equality (ii) for  $I(f_*)$
follows by a computation which is left to the reader.



\newpage









%\end{document}