%\documentclass{amsart}


%\usepackage[applemac]{inputenc}

%\addtolength{\hoffset}{-12mm}
%\addtolength{\textwidth}{22mm}
%\addtolength{\voffset}{-10mm}
%\addtolength{\textheight}{20mm}

%\def\uuu{_}


%\def\vvv{-}

%\begin{document}



\centerline{\bf\large{I. The disc algebra $A(D)$}}

\bigskip

\centerline{\emph{Contents}}
\bigskip

\noindent
\emph{0. Introduction.}

\bigskip

\noindent
\emph{1. Theorem of Brothers Riesz.}


\bigskip

\noindent
\emph{2. Ideals in the disc algebra}
\bigskip

\noindent
\emph{3. A maximality theorem for uniform algebras.}
\bigskip

\centerline{\bf{Introduction.}}

\bigskip

\noindent
Denote by $A(D)$ the subalgebra of continuous functions on the closed
unit disc $\bar D$ which are analytic in the open disc.
One refers to $A(D)$ as the \emph{disc-algebra}.
If $f\in A(D)$
we have the Poisson representation

\[ 
f(z)=\frac{1}{2\pi}\int_0^{2\pi}\,
\frac{1-|z|^2}{|e^{i\theta}-z|^2}\cdot f(e^{i\theta})\cdot d\theta
\quad \colon z\in D
\]
Since the polynomials in $z$ is a dense subalgebra of $A(D)$
it follows
that a Riesz measure $\mu$ on $T$
is $\perp$ to $A(D)$ if and only if
\[
\int_0^{2\pi}\,e^{ i n\theta}\cdot d\mu(\theta)=0\quad\colon\quad \, n=0,1,2,\ldots\tag{0.1}
\]
In Section 1 we will show that (*) implies that $\mu$ is
absolutely continuous and deduce some facts about boundary
values of analytic functions in the open disc.
Section 2 is devoted to properties of the disc algebra.
Theorem 3.1 in the last section
shows that the disc algebra is maximal in a quite strong sense.
The proof relies upon results  from several complex variables and  has
been inserted to give the reader a perspective upon the
relevance of analytic functions in several variables even for problems which
from the start are formulated in ${\bf{C}}$.





\bigskip



\centerline{\bf 1. Theorem of the Brothers Riesz}
\bigskip


\noindent
At the 4:th Scandinavian Congres held in Stockholm 1916,
Friedrich and Marcel Riesz proved the following:


\medskip

\noindent
{\bf 1.1 Theorem}
\emph{Let $E\subset T$ be a closed null set.  Then there exists
$\phi\in A(D)$ such that
$\phi(e^{i\theta})=1$ when $e^{i\theta}\in E$ while $|\phi(z)|<1$ for 
every $z\in\bar D\setminus E$.}


\bigskip


\noindent
Before the construction of  such peak functions
we
draw  a consequence.


\medskip

\noindent {\bf 1.2. Theorem}
\emph{Let $\mu$ be a   Riesz-measure on $T$  
such that}
\[ 
\int_0^{2\pi}\,e^{ i n\theta}\cdot d\mu(\theta)=\quad\colon\, n=1,2,\ldots
\]
\emph{Then $\mu$ is absolutely continuous.}
\medskip





\noindent
\emph{Proof.} 
Assume the contrary. Then there exists
a closed null set $E$ in $T$ 
such that
\[ 
\int_E\, d\mu(\theta)\neq 0\tag{i}
\]
Theorem 1.1 gives   $\phi\in A(D)$ which is a peak function for 
$E$.
For each positive  integer $m$ we have
$\phi^m\in A(D)$. The hypothesis in Theorem 1.2 and
(0.1)  give:
\[
\int_0^{2\pi}\,
\phi^m(e^{i\theta})\cdot d\mu(\theta)=0\quad\colon m=1,2,\ldots\tag{ii}
\]
Now we get a contraction since $\phi$ was a peak function for $E$.
Namely, this implies that
\[
\lim_{m\to\infty}\, \phi^m(e^{i\theta})\to \chi_E
\] 
where $\chi_E$ is the characteristic function of $E$ and 
the dominated convergence theorem applied to $L^1(\mu)$
would give $\int_E\,d\mu=0$. But this was not the case by (i) above
and  this
contradiction gives Theorem 1.2.
\bigskip

\centerline
{\emph{ Proof of Theorem 1.1}}

\bigskip

\noindent
Let $E\subset T$ be a closed null set and $\{(\alpha_\nu,\beta_\nu)\}$
is the family of open intervals in $T\setminus E$.
Since $b_\nu-a_\nu\to 0$ as $\nu$ increases, we can choose 
a sufficiently spare sequence of
positive numbers
$\{p_\nu\}$ such that
\[ \sum\, p_\nu(\beta_\nu-\alpha_\nu)<\infty\quad\text{and}\quad
\lim_{\nu\to\infty} p_\nu=+\infty
\]


\noindent
To each $\nu$ we define a function $g_\nu(\theta)$ on the open interval
$(\alpha_\nu,\beta_\nu)$ by
\[ 
g_\nu(\theta)=
\frac {p_\nu(\beta_\nu-\alpha_\nu)}
{\sqrt{\ell_\nu^2-(\theta-\gamma_\nu)^2}}
\colon\quad\colon
\ell_\nu=\frac{\beta_\nu-\alpha_\nu}{2}\quad\colon
\gamma_\nu=\frac{\beta_\nu+\alpha_\nu}{2}\tag{1}
\]


\noindent
Next, for each $\nu$ a variable substitution gives:
\[
\int_{\alpha_\nu}^{\beta_\nu}
\frac{d\theta}{\sqrt{\ell_\nu^2-(\theta-\gamma_\nu)^2}}=
\int_0^1
\frac{ds}{\sqrt{\frac{1}{4}-(s-\frac{1}{2})^2}}=C\tag{2}
\]
where  $C$�is a positive constant which the reader may compute.
Next, (2) and the convergence of
$\sum\,p_\nu(\beta_\nu-\alpha_\nu)$ imply
the function 
\[
F(\theta)=\sum\, g_\nu(\theta)\tag{3}
\]
has a finite $L^1$-norm.
Here $F$ is defined outside the null set $E$ and since
each single $g_\nu$-function  restrics to
a real analytic function on $(\alpha_\nu,\beta_\nu)$ the same holds for
$F$. 
Next, we notice that
\[
\theta\mapsto
\frac{(\beta_\nu-\alpha_\nu)}{\sqrt{\ell_\nu^2-(\theta-\gamma_\nu)^2}}
\geq 2\quad\text{for all}\quad \alpha_\nu<\theta<\beta_\nu\tag{4}
\]
In addition to this the reader can verify that

\[
\frac{(\beta_\nu-\alpha_\nu)}{\sqrt{\ell_\nu^2-(\alpha+s-\gamma_\nu)^2}}
\geq \frac{\beta_\nu-\alpha_\nu}{\sqrt{s\cdot (\beta_\nu-\alpha_\nu-s)}}
\quad \colon\quad 0<s<\beta_\nu- \alpha_\nu\tag{5}
\]

\medskip

\noindent 
From (4-5) we can show that $F(\theta)$ 
gets large when we approach $E$.
Namely, let $N$ be an arbitrary positive integer.
Then we find $\nu_*$ such that
\[
\nu>\nu_*\implies p_\nu>N\tag{i}
\]
Next, let $\delta>0$ and 
consider the open set 
$E_\delta$ of points with distance $<\delta$ to $E$.
If $\theta\in E_\delta$ we have $\alpha_\nu<\theta<\beta_\nu$ for some
$\nu$.
If $\nu>\nu*$ then (i) and (4) give
\[ 
F(\theta)> 2N\tag{ii}
\]
Next, set
\[
\gamma=\min_{1\leq\nu\leq \nu_*}\, \rho_\nu\cdot
\sqrt{\beta_\nu-\alpha_\nu}
\tag{iii}
\]
Let us now consider some  $1\leq\nu\leq\nu_*$
and a point $\theta\in E_\delta$.
which  belongs to $(\alpha_\nu,\beta_\nu)$.
Since $E\cap(\alpha_\nu,\beta_\nu=\emptyset$ we see that

\[
\theta-\alpha_\nu<\delta\quad\text{or}\quad
\beta_\nu-\theta<\delta\tag{iv}
\] 
must hold.
In both cases  (4) gives:

\[ g_\nu(\theta)\geq \frac{\rho_\nu\cdot\sqrt{(\beta-\nu-\alpha-\nu}}{\sqrt{\delta}}
\geq \frac{\gamma}{\sqrt{\delta}}\tag{v}
\]
With $\gamma$ fixed we find a small $\delta$ such that
the right hand side is $>N$ and together with (ii) it follows that
\[
\theta\in E_\delta\setminus E\implies
F(\theta)>N\tag{vi}
\]

\noindent
\medskip
{\bf The construction of $\phi$}.
The Poisson kernel gives the harmonic function: 
\[ 
U(re^{i\theta})=\frac{1}{2\pi}\int_0^{2\pi}\,
\frac{1-r^2}{ 1+r^2+\text{cos}(\theta-t)}\cdot F(t) dt
\quad\colon\quad re^{i\theta}\in D
\]
Since $F\geq 0$ we have $U$ it is $\geq 0$ in $D$ and by (vi)
$U(z)$ increases to $+\infty$ as
$z$ approaces $E$. More precisely, the following companion to (vi)
holds:
\medskip

\noindent \emph{Sublemma }
\emph{To every positive integer $N$ there exists $\delta>0$ such that}
\[ 
U(z)>N\quad\colon\quad z\in D\cap E^*_\delta
\]
\emph{where  $E^*_\delta=\{z\in D\,\colon\,\,
\text{dist}(z,E)<\delta\}$.}
\medskip

\noindent
Now we construct the harmonic conjugate:

\[ 
V(re^{i\theta})=\frac{1}{\pi}\int_0^{2\pi}\,
\frac{r\cdot \text{sin}(\theta-t)}{ 1+r^2+\text{cos}(\theta-t)}\cdot F(t) dt
\quad\colon\quad re^{i\theta}\in D
\]
\medskip

\noindent
We have no control for the
limit behaviour of $V(re^{i\theta})$ as $r\to 1$
and $e^{i\theta}\in E$.
But on
the open intervals $(\alpha_\nu,\beta_\nu)$ where
$F$  restricts to a real analytic function there exists 
a limit function $V^*$:

\[
\lim_{r\to 1}\,V(re^{i\theta})=V^*(e^{i\theta)})
\quad\colon\quad \alpha_\nu<\theta<\beta_\nu
\]
Thus, $V^*$ is a function defined on $T\setminus E$.
Similarly, $U(re^{i\theta})$ has a limit function $U^*(e^{i\theta})$
defined on $T\setminus E$.
Now we set
\bigskip
\[ 
\phi(z)= \frac{U(z)+iV(z)}{U(z)+1 +iV(z)}\quad\colon\quad z\in D\tag{*}
\]
This is an analytic function in $D$. Outside $E$ we get the boundary value function
\[
\lim_{r\to 1}\,\phi(re^{i\theta})=
\frac{U^*(e^{i\theta})+iV^*(e^{i\theta})}{U^*(e^{i\theta})+1 +iV^*(e^{i\theta})}
\]
\medskip

\noindent 
\emph{The limit on $E$}.
Concerning the limit as $z\to E$ we have:

\[ 
[1-\phi(z)|=
\frac{1}{|1+U(z)+iV(z)|}\leq \frac{1}{1+U(z)}
\]
By the Sublemma the last term tends to zero as $z\to E$.
We conclude that $\phi\in A(D)$ and here $\phi=1$ on $E$ while 
$|\phi(z)|<1$
for al $z\in \bar D\setminus E$ which gives the requested peak function.

\bigskip

\noindent {\bf Remark.}
The proof of Theorem 1.1 above
was  constructive.
There  exist  proofs using
functional analysis and the Hilbert space $L^2(d\mu)$ attached to a
Riesz measure on $T$.
See the text-book [Koosis: p. 40-47] for such alternative proofs.
\bigskip

\centerline {\bf 1.3 An application of Theorem 1.1}


\medskip

\noindent
Let $f(z)$ be analytic in the open unit disc and assume there exists
a constant $M$ such that
\[
\int_0^{2\pi}\, |f(re^{i\theta})|\cdot d\theta\leq M\quad\colon\quad
0<r<1
\]
Consider the family of measures on the unit circle
defined by 
\[
\{\mu_r=f(re^{i\theta})\cdot d\theta\,\,\colon\,\, r<1\}
\]
The uniform upper bound for their
total variation implies by compactness in the weak topology
that
there exists a sequence
$\{r_\nu\}$ with $r_\nu\to 1$
and a Riesz measure $\mu$
such that
$\mu_{r_\nu}\to\mu$ holds \emph{weakly}.
In particular we have

\[ 
\int_0^{2\pi}\, e^{in\theta}\cdot d\mu(\theta)=
\lim_{r_\nu\to 1}\int_0^{2\pi}\,e^{in\theta} f(r_\nu e^{i\theta})\cdot d\theta
\]


\noindent 
for every integer $n$.
Since $f$ is analytic the right hand side integrals  vanish whenever $n\geq 1$
and 
hence
$\mu$ is absolutely continuous by
Theorem 1.2.
So we have  $\mu=f^*(\theta)d\theta$ for an $L^1$-function $f^*$. 
Now we construct
the analytic function
\[ 
F(z)=
\frac{1}{2\pi}\int_0^{2\pi}\,
\frac{f^*(\theta)\cdot e^{i\theta}d\theta} 
{e^{i\theta}-z}
\]


\noindent
When $z\in D$ is fixed the \emph{weak} convergence applies to the
$\theta$-continuous function
$\theta\mapsto
\frac{e^{i\theta}} 
{e^{i\theta}-z}$ and hence
\[ 
F(z)=\lim_{\nu\to\infty}\,
\frac{1}{2\pi}\int_0^{2\pi}\,
\frac{ f(r_\nu e^{i\theta})e^{i\theta}d\theta}
{e^{i\theta}-z}
\]
At the same time, as soon as $|z|<r_\nu$
one has  Cauchy's formula:

\[
f(z)=
\frac{1}{2\pi}\int_0^{2\pi}\,
\frac{ f(r_\nu e^{i\theta})\cdot r_\nu e^{i\theta}\cdot d\theta}
{r_\nu\cdot e^{i\theta}-z}
\]
Since this hold for every large $\nu$  we can pass to
the limit and conclude that
$F(z)=f(z)$ olds in $D$. Hence $f(z)$ is represented by the Cauchy kernel of
the $L^1(T)$-function $f^*(\theta)$.
At this stage we apply \emph{Fatou's theorem}  to conclude that

\[
\lim_{r\to 1}\,
f(re^{i\theta})=f^*(\theta)\quad\text{holds  almost everywhere}
\]
Moreover, one has convergence in the $L^1$-norms:
\[
\lim_{r\to 1}\,
\int_0^{2\pi}\,
|f(re^{i\theta}-f^*(\theta)|=0
\]
\medskip

\noindent
Thus, thanks to Theorem 1.2 
the $L^1(T)$- sequence 
defined by the functions $\theta\mapsto f(re^{i\theta})$
converges  almost everywhere   to
a unique limit  function $f^*(\theta)\in L^1(T)$.
\medskip

\noindent
{\bf{1.4 Exercise.}}
Show that for every Lebesgue point $\theta_0$ of 
$f^*(\theta)$ there exists a radial  limit: 


\[ 
\lim_{r\to 1}\, f(re^{i\theta_0})= f^*(\theta_0)
\]


\bigskip

\noindent
{\bf{1.5 Exercise.}}
In general, let $K$ be a compact subset of $D$
and $\mu$ a Riesz measure supported by $K$ which is $\perp$ to analytic polynomials,  i.e.
\[ 
\int\, z^n\cdot d\mu(z)=0
\] 
hold for all $n\geq 0$.
Use the existence of peaking functions in $A(D)$ to conclude that if
$E\subset T$ is a null\vvv set for
linear Lebesgue measure $d\theta$, then
$E$ is a null\vvv set for $\mu$. In particular, if
$K$ contains a relatively open set given by an arc $\alpha$
on the unit circle, then the restriction of $\mu$ to $\alpha$
is absolutely continuous






\centerline{\bf{2. Principal ideals in the disc algebra.}}

\bigskip


\noindent
Let�$A(D)$ be the disc algebra.
The point $z=1$  gives
a maximal ideal in $A(D)$:
\[ 
\mathfrak{m}=\{f\in A(D)\quad\colon f(1)=0\}
\]
Let $f\in A(D)$  be such that $f(z)\neq 0$ for all $z$
in the closed disc except at the point $z=1$. 
The question arises if the principal ideal generated by $f$
is dense in 
$\mathfrak{m}$. This is not always true. A counterexample is given by
the function
\[
f(z)= e^{\frac{z+1}{z\vvv 1}}
\]



\noindent
Following  the appendix in [Carleman: Note 3] we
give  a sufficient condition on $f$ in order that its principal ideal is dense in
$\mathfrak{m}$.
Namely, since $f(z(\neq 0$ except when $z=1$
there exists the analytic function

\[
f^*(z)=\text{exp}\bigl\{\, 
\frac{1}{2\pi}\int_0^{2\pi}\,
\frac{e^{i\theta}+z}{e^{i\theta}-z}\cdot \text{log}\, 
\bigl |\frac{1}{f(e^{i\theta})}\bigr |\cdot d\theta\bigr\}
\]
\medskip

\noindent
We say that $f$ has no logarithmic reside a $z=1$ if $f=f*$ and now the following holds:


\medskip

 \noindent
 {\bf{2.2 Theorem.}} \emph{If $f$ has no logarithmic 
 residue 
 then $A(D)f$ is dense in $\mathfrak{m}$.}
\medskip



\noindent
\emph {Proof}.
With  $\delta>0$ we choose a continuous
function $\rho_\delta(\theta)$ on $T$
which is equal to
$\text{log}\, |\frac{1}{f(e^{i\theta})|}$ outside the interval $(-\delta,\delta)$
while
\[ 
0<\rho_\delta(\theta)<\text{log}\, |\frac{1}{f(e^{i\theta})}\bigr|
\quad\colon
-\delta<\theta<\delta\tag{i}
\]


\noindent 
Next, let  $\phi\in\mathfrak{m}$ and set
\[
\omega_\delta(z)= \phi(z)\cdot 
\text{exp}\,\bigl\{\vvv\frac{1}{2\pi}\int_0^{2\pi}
\frac{e^{i\theta}+z}{e^{i\theta}-z}\cdot 
\rho_\delta(\theta)\cdot d\theta\bigr\}\tag{ii}
\]
It follows that
\[
\bigl|\omega_\delta(z)\cdot f(z)-\phi(z)\bigr|=|f(z)|\cdot |\phi(z)|\cdot
\bigl|\, 1- \text{exp}\,\bigl\{
\frac{1}{2\pi}\int_0^{2\pi}\,
\frac{e^{i\theta}+z}{e^{i\theta}-z}\cdot 
\bigl[\,\text{log}\, 
\frac{1}{|f(e^{i\theta})|}-
\rho_\delta(\theta)\,\bigr] \cdot d\theta\,\bigr\}\, |\tag{iii}
\]
\medskip

\noindent
{\bf{Exercise.}}
Show that the limit of the right hand side is zero when
$\delta\to 0$ and conclude that
$\phi$ belongs to the closure of the principal
ideal generated by $f$.

\bigskip

\centerline{\bf{2.6 Some facts about $A(D)$}}
\medskip

\noindent
The disc algebra $A(D)$ is a 
uniform algebra,  
where the spectral radius norm is equal to the 
maximum over the closed disc.
By the maximum principle for analytic functions in
$D$ one has $|f|_D=|f|_T$.
One therefore calls $T$ the \emph{Shilov boundary} of $A(D)$.
A notable point is that $A(D)$ is a Dirichlet algebra which means
that the linear space of real parts of functions
restricted to $T$ is a dense subspace of all real-valued and
continuous functions on $T$.
From XX we recall that if $\rho(\theta)$ is real-valued and continuous on
$T$
then $\rho=\mathfrak{Re}(f)$ on $T$ for some
$f\in A(D)$ if and only if
the function
\[ 
z\mapsto 
\int_0^{2\pi i}\, \frac{\mathfrak{Im}(ze^{-i\theta})}
{|e^{i\theta}-z|^2}\cdot \rho(\theta)d\theta
\] 
extends to a continuous function on the closed disc.
For example, every $C^1$-function on $T$ belongs to
$\mathfrak{Re}(A(D))$.
\medskip

\noindent{\bf{2.7 Wermer's maximality theorem.}}
A result due to J. Wermer asserts that
$A(D)$ is a maximal uniform algebra. It means that
if 
$f\in C^0(T)$ is such that
the 
closed subalgebra of $C^0(T)$ generated by $f$ and $z$ is
not equal to 
$C^0(T)$, then $f$ must belong to $A(D)$.
Another way to phrase the result is that whenever
$f\in C^0(T)$ is such that
\[
\int_0^{2\pi}\, e^{ik\theta}\cdot f(e^{i\theta})\cdot d\theta\neq
0
\] 
holds for at least one positive integer $k$, then
$[z,f]_T= C^0(T)$.
\medskip

\noindent
\emph{Outline of the proof.}
Let $f\in C^0(T)$ and consider  the uniform algebra
$B=[z,f]_T$ on the unit circle. Now there exists the 
maximal ideal space 
$\mathfrak{M}_B$ whose points correspond to
multiplicative functionals on $B$. If
$p\in \mathfrak{M}_B$ and $p^*$ is the corresponding
multiplicative functional it is clear that there exists
a unique point $z(p)\in D$
such that 
$p^*(f)=f(z(p))$ for every $f$ in the subalgebra $A(D)$ of $B$.
If
$z(p)\in T$ holds for every $p$ then the $B$-element $z$
is invertible. But this means that $B$ contains both
$e^{i\theta}$ and $e^{-i\theta}$ and by Weierstrass theorem they
already generate a dense subalgebra of $C^0(T)$.
So if $B\neq C^0(T)$ there must exist at least one point
$p\in \mathfrak{M}_B$ such that $z(p)$
stays in the open unit disc.
In fact, \emph{every point} $z_0\in D$ is of the form $z(p)$ 
for some $p$ for otherwise
$\frac{1}{z-z_0}$
belongs to $B$ and one verifies easily that
the two functions on $T$ given by $e^{i\theta}$ and
$\frac{1}{e^{i\theta}-z_0}$ also generate a dense subalgebra of
$C^0(T)$. There remains to consider   the case
when $p\mapsto z(p)$ sends $\mathfrak{M}_B$
onto the closed disc.
\medskip

\noindent
At this stage one employs a general result from
uniform algebras. Namely, since every multiplicative functional has norm one
it follows that that for every $p\in\mathfrak{M}_B$
there exists a probability measure
$\mu_p$ on the unit circle such that
\[ 
p^*(g)=
\int_T\, g(e^{i\theta})\cdot d\mu_p(\theta)\quad\text{hold for all}\quad
g\in B\tag{*}
\]
Now we use that $A(D)$ is a Dirichlet algebra. Namely,  (*) holds
in particular  for $A(D)$-functions and since $\mu_p$
is a real measure we conclude that it must be equal to
the Poisson kernel of the point $z(p)$.
This proves to begin with that the map $p\to z(p)$ is
\emph{bijective}.
So for every $g\in B$ we get a continuous function on the closed unit disc
defined by
\[
g^*((z(p)=p^*(g)
\] 
But  (*) above means that $g^*$
is the harmonic extension to $D$ of the boundary
function $g$ on $T$.
Finally, since $B$ is algebra one easily verifies that when 
every $B$-function is harmonic in $D$, then $B$ consists
of complex
analytic functions only. This means
precisely that
$B=A(D)$.
At this stage we  conclude
that when
$B=[z,f]_T$ and $B\neq C^0(T)$ is assumed, then $f\in A(D)$
holds 
which is 
the assertion in Wermer's maximality theorem.

\bigskip


\centerline{\bf 3. Relatively maximal algebras}
\bigskip

\noindent
{\bf{Introduction.}}
An extension of Wermer's maximality theorem
was proved in [Bj�rk] and goes as follows.
Let $K$
be a closed subset of $\bar D$
whose  planar Lebesgue measure  is zero.
We also assume that $K$ contains $T$ and that
$\bar D\setminus K$ is connected. Finally we assume
that there exists some
some open interval on $T$ which does no belong to
the closure of $K\setminus T$.
In this situation the following holds:

\bigskip


\noindent
{\bf {3.1. Theorem.}}
\emph{Let $f\in C^0(K)$
be such that the uniform algebra $[z,f]_K\neq C^0(K)$.
Then  $f$ extends from $K$ to an analytic function in
$D\setminus K$.}
\bigskip

\noindent {\bf Remark.}
The case when
$K$ is the union of $T$ and a finite set of Jordan arcs
where each arc has one end-point on $T$ and the other
in the open disc $D$ is of special interest. If these Jordan arcs
are not too fat, then $f$ extends analytically across each arc
which means that the restriction of $f$ to $T$ must belong to the
disc-algebra.
This case was a motivation for
Theorem 3.1 since it is connected to
the problem of finding conditions on a Jordan arc
$J$ in order that it is locally a removable singularity for
continuous functions $g$ which are analytic in
open neighborhoods of $J$.
The interested reader may consult [Bj�rk:x] for a further discussion about
this problem where  comments are given by
Harold Shapiro about the connection to
between
Theorem 3.1 and 
results by Privalov concerning 
analytic extensions across a Jordan arc.
\bigskip

\noindent{\emph{ Proof of Theorem 3.1}}.
The proof 
will employ the 
\emph{Local maximum Principle} by Rossi which
is a powerful tool to study  uniform  algebras whose
Shilov boundary is a proper subset of the maximal ideal space.
Let us then start the proof. Set
\[ 
B=[z,f]_K
\]



\noindent
Since $B\neq C^0(K)$ is assumed there exists a non-zero Riesz measure
$\mu$ on $K$ which annihilates $B$. 
Notice that $\mu$ can  be complex-valued.
Let $\pi$ be the projection from
$\mathfrak{M}_B$ into $D$ which means that when
$z$ is regarded as an element in $B$ then its Gelfand transform
$\widehat z$ satisfies
\[ 
\widehat z(p)=\pi(p)\quad\colon\, p\in \mathfrak{M}_B
\]
As usual $K$ is identified with a compact subset
of
$\mathfrak{M}\uuu B$. If $e^{i\theta}\in T$
we use that it
is a peak point for
$A(D)$ and hence also for $B$. This entails
that
the fiber
$\pi^{\vvv 1}(e^{i\theta})$ is reduced to the natural point
$e^{i\theta}\in K$.
Next, since we assume that $K$ has planar measure zero we know from
XX that the uniform algebra on $K$ generated
by rational functions with
poles outside $K$ is equal to $C^0(K)$.
Since $z\in B$ and $B\neq C^0(K)$ it follows that
$\pi^{\vvv 1}(D\setminus K)\neq \emptyset$.
We are going to prove that the fiber above every point in
$D\setminus K$ is reduced to a single point and for this purpose
we 
define the following two analytic functions in the
open set $D\setminus K$:

\[
W(z)=\int_K\, \frac{f(\zeta)\cdot d\mu(\zeta}{\zeta-z}\quad\text{and}\,\,\,
R(z)=\int_K\, \frac{d\mu(\zeta)}{\zeta-z}\tag{*}
\]

\noindent
The main step in the proof is to show that if $z\in D\setminus K$
and $\xi\in \pi^{\vvv 1}(z)$ then
the Gelfand transform 
$\widehat f$ satisfies:
\[
\widehat f(\xi)\cdot R(z)= W(z)\quad\colon\,\forall\,\, \xi\in \pi^{-1}(z)
\tag{**}
\]

\medskip

\noindent
Here $R(z)$ it cannot be identically zero
in $D\setminus K$ for then the Riesz measure $\mu$
would be identically zero.
If $R(z)\neq 0$ for some $z\in D\setminus K$
then (**) entails that
the fiber
$\pi^{-1}(z)$ is reduced to a single point. This hold for
all points outside the eventual discrete zero\vvv set of
$R$ and when a fiber
$\pi^{-1}(z)$ is reduced to a single point
the meromorphic function $\frac{W}{R}$
has a value taken by the continuous Gelfand transform of $f$ at this
unique fiber\vvv point.
This implies that
$\frac{W}{R}$ is bounded outside the zeros of $R$ and therefore analytic in the whole set $D\setminus K$. From this it follows easily
that
(**) implies that
al fibers are reduced to single points and the analytic function
$\frac{W}{R}$ in $D\setminus K$ is identified with the restriction of $\widehat f$ to this open set in
the maximal ideal space of $B$.
So there remains to give:


\bigskip


\noindent{\emph{Proof of (**).}}
Since
$\mu$  annihilates  the functions 
$z^N$ and  $z^N\cdot f(z)$ for every $N\geq 0$
we have
\medskip
\[
\int_K\, \frac{\bar z\cdot d\mu(\zeta)}{1-\bar z\cdot\zeta}=
\int_K\, \frac{\bar z\cdot f(\zeta)\cdot d\mu(\zeta)}{1-\bar z\cdot\zeta}=
0\quad\text{for every  }\,\,\, 
 z\in D
\]
Adding these
 zero-functions in (*)
 it follows that
 \medskip
\[
W(z)=
\int_K\, \frac{(1-|z|^2|\cdot f(\zeta)\cdot d\mu(\zeta)}{
(\zeta-z)(1-\bar z\zeta)}\quad\text{and}\,\,\,
R(z)=\int_K\, \frac{(1-|z|^2\cdot  d\mu(\zeta)}{
(\zeta-z)(1-\bar z\zeta)}\tag {1}
\]
\medskip

\noindent
The assumption that the closure of $K\setminus T$
does not contain $T$ gives
some open arc
$\alpha=(\theta_0,\theta_1)$
on $T$ which is disjoint from the closure of $K\setminus T$.
The local version of the Brother's Riesz  theorem from Exercise 1.5 implies that
the restriction of $\mu$ to $\alpha$ is absolutely continuous.
Hence, by  Fatou's theorem
there exist the two limits
\[ 
\lim_{r\to 1}
W(re^{i\phi})=W(e^{i\phi})\quad\colon
\lim_{r\to 1}
R(re^{i\phi})=R(e^{i\phi})
\tag{2}
\]
almost every on
$\theta_0<\phi<\theta_1$.
Let us fix $\theta_0<\phi_0<\phi_1<\theta_1$
where the radial limits in (2) exist for $\phi_0$ and $\phi_1$.
Next, consider a point  $z_0\in D\setminus K$ and choose a 
closed Jordan curve
$\Gamma$ which is the union of the $T$-interval $[\phi_0,\phi_1]$
and a Jordan arc $\gamma$ which is disjoint to the closure of
$K\setminus T$ while $z_0$ belongs to the Jordan domain
$\Omega$ bordered by $\Gamma$. We can always   choose
a nice arc $\Gamma$ which is of class $C^1$ and hits
$T$ at $e^{i\phi_0}$ and $e^{i\phi_1}$ at right angles.
Since $\Gamma$ has a positive distance from
$K\setminus T$ there exists $r_*<1$ such that if $r_*<r<1$
then the functions
\[ 
W_r(z)=W(rz)\quad\colon\, R_r(z)=R(rz)\tag{3}
\] 
are analytic
in a neighborhood of the closure of $\Omega$.
Now we consider  the set $\pi^{-1}(\Omega)=\Omega^*$ in
$\mathcal M_B$ whose boundary in $\mathcal M_B$
is contained in 
$\pi^{-1}(\Gamma)=\Gamma^*$.
If $Q(z)$ is an arbitrary  polynomial the\emph{ Local Maximum Principle} 
gives
\[
|Q(z_0)|\cdot[ \hat g(\xi)\cdot R_r(z_0)-W_r(z_0)|\leq
\bigl |Q\cdot( \widehat f\cdot R-W_r)\bigr |_{\Gamma^*}\tag{4}
\]



\noindent
Recall that $\pi^{-1}(T)$ is a copy of $T$ 
Identifying  the subinterval
$[\phi_0,\phi_1]$
with a closed subset of $\mathcal M_B$ we can write
\[
\Gamma^*=\gamma^*\cup\,[\phi_0,\phi_1]
\quad\colon\,\gamma^*=
\pi^{-1}(\Gamma\setminus(\phi_0,\phi_1))\tag{5}
\]


\noindent
Now (4) and the continuity of the Gelfand transform
$\widehat f$
give a constant $M$ which is independent of
$r$ such  that
the maximum norms
\[
\bigl | \widehat f\cdot R-W_r)\bigr |_{\Gamma^*}\leq M\quad\colon
r_*<r<1\tag {6}
\]
\medskip
Since $\widehat f(e^{i\theta})=f(e^{i\theta})$
holds on $T$ it follows from (2) that the maximum norms:
\[
\delta(r)=\bigl |\hat g\cdot R_r-W_r|_{[\phi_0,\phi_1]}=0\tag{7}
\]


\noindent
tend to zero as $r\to 1$.
Next, let $\epsilon>0$. Runge's theorem gives a
polynomial $Q(z)$ such that
\[
Q(z_0)=1\quad\colon\, |Q|_\gamma<\frac{\epsilon}{M}\tag{8}
\]
When $\xi\in \pi^{\vvv 1}(z\uuu 0)$
it follows from (6) that

\[
|\widehat f(\xi) R(z\uuu 0)\vvv W(z\uuu 0)|\leq
\text{Max}\,(\epsilon,\bigl ||Q \bigr ||_{[\phi_0,\phi_1]}\cdot\delta(r))\tag{9}
\]
Passing to the limit as $r\to 1$ we use that $\delta(r)\to 0$
together with the
obvious limit formulas
$R_r(z_0)\to R(z_0)$ and $W_r(z_0)\to W(z_0)$, and conclude that
that
\[
\bigl |\widehat f(\xi)\cdot R(z_0)-W(z_0)\bigr |\leq\epsilon\tag{10}
\]
Since we can choose $\epsilon$ arbitrary small
we get
\[
\widehat f(\xi)\cdot R(z_0)=W(z_0)\quad\colon\,\xi\in \pi^{-1}(z_0)\tag{11}
\]
Since $z\uuu 0\in D\setminus k$ was arbitrary we have proved (**)
and as explained after (**) it follows that
\[
\pi^{-1}(D\setminus K)\simeq D\setminus K\tag{12}
\]
\bigskip

\noindent
{\bf 3.2 The extension to $K$.} 
At this stage we can easily finish the proof of Theorem 3.1.
We have already found the analytic function
$\widehat f(z)$ in $D\setminus K$ and 
it is clear that it extends to $f$ on the free circular arc
$(\theta_0,\theta_1)$ of $T$.
To see that $\widehat f$ extends to $K$ and gives a continuous function on
the whole closed unit disc we  solve the Dirichlet problem
for the continuous functions $\mathfrak{Re}\,f$ and
$\mathfrak{Im}\,f$ on $K$ and conclude that
$\widehat f$ extends and moreover its boundary value function on
$K$ is equal to the restriction of $f$ to $K$.
The proof of Theorem 3.1 is therefore finished if we have shown the equality:
\[
\mathcal M_B\simeq D
\]
To see that this  holds  we put $U=\pi^{-1}( \mathcal M_B\setminus D)$
and notice that its boundary in
$\mathcal M_B$ is contained in
the closure of $K\setminus T$. Call, this compact set $K_*$.
Since we have the free arc 
$(\phi_0,\phi_1)$
and $D\setminus K$
is connected it follows that
${\bf{C}}\setminus K_*$ is connected, i.e. only the unbounded
component exists. So by Mergelyan's Theorem
polynomials in $ z$ generate a dense subalgebra of
$C^0(K_*)$.
But then the Local  Maximum Principle implies that
$U$ must be empty and the proof of Theorem 3.1 is finished.


\newpage










%\end{document}



