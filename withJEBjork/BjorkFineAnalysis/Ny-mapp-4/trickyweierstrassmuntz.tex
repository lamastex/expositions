%\documentclass{amsart}

%\usepackage[applemac]{inputenc}

%\addtolength{\hoffset}{-12mm}
%\addtolength{\textwidth}{22mm}
%\addtolength{\voffset}{-10mm}
%\addtolength{\textheight}{20mm}


%\def\uuu{_}


%\def\vvv{-}

%\begin{document}

\centerline{\bf\large 10. Approximation theorems in complex domains}
\bigskip

\noindent 
\emph{0. Introduction.}

\medskip


\noindent
\emph{A. Weierstrass approximation theorem}\bigskip


\noindent
\emph{B. Polynomial approximation with bounds}

\bigskip


\noindent
\emph{C. Approximation by fractional powers}


\bigskip


\noindent
\emph{D. Theorem of M�ntz}




\bigskip


\noindent
\centerline {\bf 0. Introduction.}
\medskip


\noindent
This chapter is devoted to results  concerned with
approximation by analytic functions
due to Carleman, Lindel�f and M�ntz.







\bigskip
\centerline{\bf A. Weierstrass approximation theorem.}

\medskip

\noindent
A wellknown result due to Weierstrass
asserts that if $f(x)$ is a complex valued continuous
function on a bounded interval $[a,b]$ then it can
be uniformly approximated by polynomials.
It turns out that
uniform approximations exist
on the whole real line where
the approximating functions are entire.
\medskip

\noindent
{\bf A.1 Theorem.} \emph{Let $f(x)$ be a continuous function on
the real $x$-line. To every $\epsilon>0$ there exists
an entire function
$G(z)$ such that}
\[
\max_{x\in{\bf{R}}}\, \bigl|G(x)-f(x)\bigr|\leq\epsilon
\]
\medskip

\noindent
An elementary proof using Cauchys integral formula
only was given  in [Carleman].
Here we extend Theorem A.1 to a  more general
situation.
Let $K$ be a closed null\vvv set in ${\bf{C}}$ which in general is unbounded.
If $0<R<R^*$ we put
\[
K[R,R^*]= K\cap \{ R\leq |z|\leq R^*\}
\]
If $R>0$ we denote by $\bar D\uuu R$ the disc of radius $R$
and  $K\uuu R= K\cap \bar D\uuu R$.
With these notations one has:

\medskip

\noindent
{\bf{A.2 Theorem.}}
\emph{Assume there exists a strictly increasing sequence $\{R\uuu\nu\}$
where $R\uuu\nu\to +\infty$ such that the sets}
\[ 
\Omega\uuu\nu={\bf{C}}\setminus \bar D\uuu {R\uuu \nu}\,\cup K[R\uuu\nu,
R\uuu{\nu+1}]
\] 
\emph{are connected for each $\nu\geq 1$ together with
the set ${\bf{C}}\setminus K\uuu{R\uuu 1}$.
Then every continuous function on $K$ can be uniformly approximated by entire functions.}
\bigskip


\noindent
To prove this result we first establish the following.

\medskip

\noindent {\bf{A. 3 Lemma.}}
\emph{Let $\nu\geq 1$ and $\psi$ is
a continuous function
on 
$S= \bar D\uuu{R\uuu \nu}\cup\, K[R\uuu \nu,R\uuu{\nu+1}]$
where $\psi$ is analytic in the open disc $D\uuu{R\uuu\nu}$.
Then $\psi$ can be uniformly approximated on $S$ by polynomials in $z$.}

\medskip

\noindent 
\emph{Proof}.
If
we have found a sequence of polynomials
$\{p\uuu k\}$ which approximate $\psi$ uniformly on
$S\uuu *=\{|z|=R\uuu\nu\}\cup K[R\uuu \nu,R\uuu{\nu+1}]$
then this sequence approximates $\psi$ on $S$. In fact, this follows since
$\psi$ is analytic in the disc $D\uuu{R\uuu \nu}$
so by the maximum principle for analytic functions in a disc
we have
\[
||p\uuu k\vvv\psi||\uuu S=
||p\uuu k\vvv\psi||\uuu {S\uuu *}
\] 
for each $k$.
Next, if uniform approximation on $S\uuu *$ fails
there exists a Riesz\vvv measure $\mu$ supported by
$S\uuu *$ which is $\perp$ to all analytic polynomials while
\[
\int\, \psi\cdot d\mu\neq 0\tag{1}
\]
To see that this cannot occur we consider the Cauchy transform
\[
\mathcal C(z)=\int\, \frac{d\mu(\zeta)}{z\vvv\zeta}
\]
Since $\int\, \zeta^n\cdot d\mu(\zeta)=0$ for every $n\geq 0$
we see that $\mathcal C(z)=0$ in the exterior disc
$|z|> R\uuu{\nu+1}$.
The connectivity hypothesis implies that
$\mathcal C(z)=0$ in the whole open complement of $S$.
Now $K$ was a null set which means that the
$L^1\uuu{\text{loc}}$\vvv function 
$\mathcal C(z)$ is zero in the exterior disc
$|z|>R\uuu\nu$ and hence its distribution derivative
$\bar\partial(\mathcal C\uuu\nu)$ also vanishes in this exterior disc.
At the same time we have the equality
\[ \bar\partial(C\uuu\nu)= \mu
\]
We conclude that the support of $\mu$ is confined to the circle
$\{|z|=R\uuu\nu\}$.
But then (1) cannot hold since
the restriction of $\psi$ to
this circle by assumption extends to be analytic in the disc
$D\uuu {R\uuu\nu}$ and therefore can be uniformly approximated by
polynomials on the circle.

\medskip

\noindent
\emph{Proof of Theorem A.2.}
Let $\epsilon>0$ and
$\{\alpha\uuu \nu\}$ is a sequence of positive numbers such that
$\sum\,\alpha\uuu\nu<\epsilon$.
Consider some $f\in C^0(K)$. Starting with the set
$K\uuu{R\uuu 1}$
we use the assumption that its complement is connected
and using   Cauchy transforms as in  Lemma A.3
one shows that the restriction of $f$ to this compact set
can be uniformly approximated by polynomials. So we find
$P\uuu 1(z)$ such that
\[
||P\uuu 1\vvv f||\uuu {K\uuu{R\uuu 1}}<\alpha\uuu 1\tag{i}
\]
From (i) one easily construct a continuous function
$\psi$ on $\bar D\uuu{R\uuu 1}\cup K[R\uuu 1,R\uuu 2]$
such that
$\psi=P\uuu1 $ holds in the disc
$\bar D\uuu{R\uuu 1}$ and
the maximum norm
\[
||\psi\vvv f||\uuu { K[R\uuu 1,R\uuu 2]}\leq \alpha\uuu 1
\]
Lemma A.3 gives a polynomial $P\uuu 2$ such that
\[
||P\uuu 2\vvv P\uuu 1||\uuu {D\uuu{R\uuu 1}}<\alpha\uuu 2
\quad\text{and}\quad 
||P\uuu 2\vvv f||\uuu {K[R\uuu 1,R\uuu 2]}\leq
\alpha\uuu 1+\alpha\uuu 2
\]
Repeat the construction where Lemma A.3 is used as $\nu$ increases.
This gives
a sequence of polynomials $\{P\uuu\nu\}$
such that
\[
||P\uuu \nu \vvv P\uuu {\nu\vvv 1}||\uuu {D\uuu{R\uuu \nu }}<\alpha\uuu \nu
\quad\text{and}\quad 
||P\uuu\nu\vvv f||\uuu {K[R\uuu {\nu\vvv 1},R\uuu \nu]}<
\alpha\uuu 1+\ldots+\alpha\uuu\nu
\]
hold for all $\nu$.
From this it is easily seen that we obtain an entire function
\[ 
P^*(z)= P\uuu 1(z)+\sum\uuu{\nu=1}^\infty
P\uuu{\nu+1}(z)\vvv P\uuu\nu(z)
\]
Finally the reader can check that the inequalities above imply that
the maximum norm
\[
 ||P^*\vvv f||\uuu K\leq \alpha\uuu 1+ \sum\uuu{\nu=1}^\infty\,\alpha\uuu\nu
\]
Since the last sum is $\leq 2\epsilon$ and $\epsilon>0$ was arbitrary
we have proved Theorem A.3.

 
\bigskip




\noindent
{\bf Exercise.}
Use similar methods as above to show that if
$f(z)$ is analytic in the upper half plane
$U^+=\mathfrak{Im}(z)>0$
and has continuous boundary values on the real line, then
$f$ can be uniformly approximated by an entire function, i.e. to
every $\epsilon>0$ there exists an entire function
$F(z)$ such that
\[
\max_{z\in U^+}\, |F(z)-f(z)|\leq\epsilon
\]


\bigskip



\centerline{\bf B. Polynomial approximation with bounds}

\medskip

\noindent
{\bf Introduction.} We  begin with a  result
due to Lindel�f.
Let $U$ be a Jordan domain and set $\Gamma=\partial U$.
For each  $f(z)\in\mathcal O(U)$,  
Runge's theorem
gives a 
sequence of polynomials $\{Q_\nu(z)\}$ which approximates $f$ uniformly over
each compact subset of $U$.
If we impose some bound on $f$ one may ask
if 
an approximation exists  where the
$Q$-polynomials satisfy a similar bound as $f$.
Lindel�f proved that  bounds exist
for many different  norms on
the given function $f$ 
in the
article 
\emph{Sur un principe g�neral de l'analyse et ses applications
� la theorie de la repr�sentation conforme}
from 1915.
Let us  announce  two results from [Lindel�f]
of this nature.
Let  $p>0$ and consider the $H^p$-space
of analytic functions in $U$ for which
\[ 
||g||_p=\iint_U\,|g(z)|^p\cdot dxdy<\infty
\tag{1}
\]


\noindent
{\bf B.1 Theorem } \emph{Let $p>0$ and suppose that
$f(z)$ has a finite $H^p$-norm. Then there exists a sequence of polynomials
$\{Q_n(z)\}$ which converge uniformly to
$f$ in compact subsets of $U$ while}
\[ 
||Q_n||_p\leq ||f||_p\quad\colon\quad n=1,2,\ldots
\] 
\medskip

\noindent
A similar approximation holds when
the $H^p$-norm is replaced by the maximum norm.
Thus, if $f$ is a bounded analytic function
in $U$ there exists a sequence of polynomials
$\{Q_n\}$ which converge to $f$ in every relatively
compact subset of $U$ and at the same time
the maximum norms satisfy:
\[
||Q_n||_U\leq ||f||_U\quad\colon\quad n=1,2,\ldots
\]


\noindent
To prove Theorem B.1 one constructs for each
$n\geq 1$ a Jordan curve $\Gamma_n$
which surrounds $\bar U$, i.e. its interior Jordan domain
$U_n$ contains
$\bar U$ and for every point  $p\in\Gamma_n$ the
distance of $p$ to $\Gamma$ is $<1/n$.
It is trivial to see that such a family of Jordan curves
exist where the domains
$U_1,U_2,\ldots$ decrease.
Next, fix a point $z_0\in U$. There exists the unique  conformal map
$\psi_n$ from $U_n$ onto $U$ such that
\[ 
\psi_n(z_0)=z_0\quad\colon\quad
\psi'_n(z_0)\,\,\text{is real and positive}
\]
With these notations Lindel�f used the following lemma whose proof
is  left as
an exercise:
\medskip

\noindent
{\bf B.2. Lemma}
\emph{For each compact subset
$K$ of $U$ the
maximum norms
$|\psi_n(z)-z|_K$ tend to zero as $n\to\infty$. Moreover,
the complex derivatives $\psi'_n(z_0)\to 1$.}
\medskip

\noindent
\emph{Proof of Theorem B.1.}
To each $n$ we set
\[
F_n(z)=f(\psi_n(z))\cdot\psi'_n(z)^{\frac{2}{p}}
\]


\noindent
By Lemma B.2 the sequence $\{F_n\}$ converges uniformly to
$f$ on compact subsets of $U$. Moreover, each
$F_n\in\mathcal O(U_n)$ and it is clear that

\[
\iint_U\,|f(z)|^p\cdot dxdy=
\iint_{U_n}\,|F_n(z)|^p\cdot dxdy
\]
hold for every $n$.
To get the required  polynomials $\{Q_n\}$
in Theorem B.1 for $H^p$-spaces it suffices to apply Runge's theorem for
each single $F_n$. This  detail of the proof is left to the reader.
For maximum norms we use the functions
\[
F_n(z)=f(\psi_n(z))
\]
and after apply Runge's theorem in the domains $\{U_n\}$.

\bigskip

\noindent{\bf B.3 Remark.}
More delicate
approximations by polynomials where
other norms such as the 
modulus  of continuity,  were 
established later by Lindel�f and  De Vall�-Poussin.
We shall not pursue this any further.
The reader can
consult
articles by  De Vall�-Poussin which contain
many interesting results
concerned with approximation theorems.



\bigskip


\centerline{\bf C. Approximation by fractional powers }

\bigskip

\noindent 
Here is the set-up in
the article \emph{�ber die approximation analytischer funktionen}
by Carleman from 1922.  
Let $0<\lambda_1<\lambda_2<\ldots$ 
be a sequence of positive real numbers and
$\Omega$ is a simply connected
domain contained in
the right half-space
$\mathfrak{Re}(z)>0$.
Notice that  
the functions 
$q_\nu(z)= z^{\lambda_\nu}$
are analytic in the half-plane,  i.e. with
$z=re^{i\theta}$ and $-\pi/2<\theta<\pi/2$ we have:
\[ 
q_\nu(z)=r^{\lambda_\nu}\cdot e^{i\lambda_\nu\cdot\theta}
\]


\noindent
{\bf C.1 Definition.}
\emph{We say that the sequence $\Lambda=\{\lambda_\nu\}$
is dense for approximation if there for each
$f\in\mathcal O(\Omega)$ exists a sequence of
functions of the form}
\[ 
Q_N(z)=\sum_{\nu=1}^N
\, c_\nu(N)\cdot q_\nu(z)\quad\colon\quad N=1,2,\ldots
\]
\emph{which converges uniformly to $f$ on compact subsets of $\Omega$.}

\bigskip

\noindent
{\bf C.2 Theorem.}
\emph{A sequence $\Lambda$ is dense if }
\[
\limsup_{R\to\infty}\,
\frac{\sum_R\,\frac{1}{\lambda_\nu}}{\text{Log}\,R}>0\tag{*}
\]
\emph{where $\sum_R$ means that we take the sum over
all $\lambda_\nu<R$.}
\medskip

\noindent
{\bf Remark.}
Above condition (*)  is the same for every
simply connected  
domain $\Omega$. Theorem C.2 gives
a \emph{sufficient} condition for an   approximation.
To get  necessary condition one must specify the domain
$\Omega$ and we shall not try to discuss this
more involved problem.
The proof of Theorem C.2 requires several steps, the crucial
is the 
uniqueness theorem in C.4 while the proof of Theorem C.2 is
postponed until C.5.
\medskip


\centerline {\bf C.3 A uniqueness theorem.}
\medskip

\noindent
Consider a closed Jordan curve
$\Gamma$ of class $C^1$ which is contained in
$\mathfrak{Re}\, z>0$.
When $z=re^{i\theta}$ stays in the right
half-plane  we get an entire function of the complex 
variable $\lambda$ defined by:
\[ 
\lambda\mapsto z^\lambda= r^\lambda \cdot e^{i\theta\cdot\lambda}
\] 
We conclude that
a real-valued 
and continuous function $g$ on $\Gamma$ gives
an entire function of $\lambda$ defined by:
\[
 G(\lambda)=\int_\Gamma\, g(z)\cdot z^\lambda\cdot \bigl|dz\bigr|_\Gamma
 \]
where $|dz|_\Gamma$  is the arc-length on $\Gamma$.
With these notations one has
\bigskip

\noindent
{\bf C.4 Theorem.}
\emph{Assume that $\Lambda$ satisfies the condition in
Theorem C.2. Then, if $G(\lambda_\nu)=0$ for every
$\nu$ it follows that 
the $g$-function is  identically zero.}

 


\medskip

\noindent
\emph{Proof.} If we have shown that
the $G$-function is identically zero then the reader may verify
that $g=0$.  There remains to show that if
$G(\lambda_\nu)=0$ for every $\nu$ then
$G=0$.
To attain this one first   shows that
there exist constants $A,K$ and $0<a<\frac{\pi}{2}$
such that:
\[ 
|G(\lambda)|\leq K\cdot e^{|\lambda|}\quad\text{and}\quad
|G(is)|\leq K\cdot e^{|s|\cdot a}\quad\colon\,\lambda\in{\bf{C}}\,\,
\colon s\in{\bf{R}}\tag{i}
\]


\noindent
The easy verification of (i) is left to the reader. 
Next,  the first inequality in (i) means that
$G$ is an  entire function of exponential type one.
By assumption  $G(\lambda_\nu)=0$ for every  $\nu$.
Now we can use
Carleman's  formula for analytic functions in a half-space
from XXX to conclude that $G=0$. Namely, set
\[ 
U(r,\phi)=\log\,\bigl |G(re^{i\phi})\bigr |\tag{ii}
\]


\noindent
Let $\{ r_\nu e^{i\phi_\nu}\}$
be the zeros of $G$ in $\mathfrak{Re}(z)>0$ which 
by the hypothesis contains the set $\Lambda$.
By Carleman's formula the following hold for each
$R>1$:
\[
\sum_{1<r_\nu<R}\,
\bigl[\frac{1}{r_\nu}-\frac{r_\nu}{R^2}\bigr]\cdot\text{cos}\,\theta_\nu=
\frac{1}{\pi R}\cdot\int_{-\pi/2}^{\pi/2}\,
U(R,\phi)\cdot\text{cos}\,\phi\cdot d\phi+
\]
\[
\frac{1}{2\pi R}\cdot\int_1^R\, \bigl(\frac{1}{r^2}-\frac{1}{R^2}\bigr)
\cdot\bigr[\, U(r,\pi/2)+U(r,-\pi/2)\,\bigr]\cdot dr+
c_*(R)
\]
where $c_*(R)\leq K$ holds for some constant which is independent of
$R$.
Finally, the set  
$\Lambda$ satisfies (*)  in
Theorem C.2  and the  sum over zeros in Carleman's  formula above
majorizes   the  sum extended over
the real $\lambda$-numbers
from $\Lambda$ satisfying $1<\lambda_\nu<R$.
At this stage we leave it to the reader to
verify that the second inequality in (i) above implies 
that $G$ must be identically zero.



\bigskip

\centerline{\emph{C.5 Proof of Theorem C.2}}

\bigskip
\noindent
Denote by $\mathcal O^*(\Lambda)$ the linear space of analytic functions
in the right half-plane given by finite ${\bf{C}}$-linear combinations of the fractional
powers $\{z^{\lambda_\nu}\}$.
To obtain uniform approximations over relatively compact
subsets when 
$\Omega$ is a simply connected  domain in $\mathfrak{Re}(z)>0$,
it suffices to regard a 
closed Jordan arc
$\Gamma$ 
which borders a Jordan domain $U$ where
$U$ is a relatively compact subset of
$\Omega$. In particular $\Gamma$ has a positive distance to
the imaginary axis and there remains  to show
that when (*) holds in Theorem C.2, then
an arbitrary analytic function $f(z)$ defined in some open
neighborhood of $\bar U$ can be uniformly approximated
by $\mathcal O^*(\Lambda)$-functions over a relatively compact subset
$U_*$ of $U$.
To achieve this 
we shall use a trick which
reduces the proof of uniform approximation
to a problem concerned with $L^2$-approximation on
$\Gamma$.
To begin with
we have
\medskip

\noindent
{\bf C.6 Lemma.}
\emph{The uniqueness in Theorem C.4 implies that
if $V$ is a real-valued function on
$\Gamma$ then there exists a sequence 
$\{Q_n\}$ from the family $\mathcal O(\Lambda)$
 such that}
\[
\lim_{n\to\infty}\, \int_\Gamma\,\bigl|Q_n-V|^2\cdot |dz|=0
\]
\medskip

\noindent 
The proof of this result is left as an exercise.

\medskip

\noindent
{\bf C.7 A tricky construction.}
Let $f(z)$ be analytic in a neighborhood of the closed
Jordan domain $\bar U$ bordered by
$\Gamma$. Define a new analytic function
\[ 
F(z)=\int_{z_*}^z\,\frac{f(\zeta)}{\zeta}\cdot d\zeta\tag{1}
\]
where $z_*$ is some point in $\bar U$ whose specific  choice
does not affect the
subsequent discussion.
We can write $F=V+iW$ where $V=\mathfrak{Re}(F)$.
Lemma C.6 gives a sequence $\{Q_n\}$ which approximates
$V$ in the $L^2$-norm on $\Gamma$.
Using this $L^2$-approximation we get
\medskip

\noindent
{\bf Lemma C.8}
\emph{Let $U_0$ be relatively compact in $U$.
Then there exists a sequence of real numbers
$\{\gamma_n\}$ such that}
\[
\lim_{n\to\infty}\, \bigl|| Q_n(z)-i\cdot\gamma_n-F(z)\bigl|_{U_0}=0
\]
\medskip

\noindent
Again we leave out the proof as an exercise.
Next, taking complex derivatives  Lemma C.8  implies that
if $U_*$ is even smaller, i.e. taken to be a relatively compact in
$U_0$, then we a get uniform approximation of derivatives:
\[
Q'_n(z)\to F'(z)=\frac{f(z)}{z}
\]
This means that
\[
 z\cdot Q'_n\to f(z)
\]
holds uniformly in
$U_*$. Next, notice that  
\[
z\cdot \frac{d}{dz}(z^{\lambda_\nu})
=\lambda_\nu\cdot z^{\lambda_\nu}
\]
hold for each $\nu$. Hence
$\{z\cdot Q'_n(z)\}$  again belong to  the $\mathcal O(\Lambda)$-family.
So we achieve the required uniform approximation of
the given $f$ function on $U_*$ which
completes the proof of
Theorem C.2.




\bigskip



\centerline{\bf D. Theorem of M�ntz}
\bigskip



\noindent
{\bf Introduction.} Theorem D.1   below
is due to M�ntz. See his article
\emph{�ber den Approximationssatz von
Weierstrass} from 1914. The simplified version of the original proof below
is given in [Car].
Here is the set up: Let
$0<\lambda_1<\lambda_2<\ldots$. To
each $\nu$ we get the function $x^{\lambda_\nu}$
defined on the real unit interval $0\leq x\leq 1$.
We say that the sequence $\Lambda=\{\lambda_\nu\}$
is $L^2$-dense if the family
$\{x^{\lambda_\nu}\}$
generate a dense linear subspace of the Hilbert space of
square integrable functions on $[0,1]$.
\medskip

\noindent
{\bf D.1 Theorem.} \emph{The necessary and sufficient condition for
$\Lambda$ to be $L^2$-dense is that
$\sum\,\frac{1}{\lambda_\nu}$ is convergent.}
\bigskip

\noindent
\emph{D.2 Proof of necessity.} 
If
$\Lambda$ is not $L^2$-dense there
exists some $h(x)\in L^2[0,1]$
which is not identically zero while
\[ 
\int_0^1\, h(x)\cdot x^{\lambda_\nu}\cdot dx=0
\quad\colon\quad\nu=1,2,\ldots\tag{1}
\]
Now 
consider the function
\[ 
\Phi(\lambda)= 
\int_0^1\, h(x)\cdot x^{\vvv i\lambda}\cdot dx\tag{2}
\]
It is clear that
$\Phi$ is analytic in
the right half plane
$\mathfrak{Im}\,\lambda>0$.
If $\lambda=s+it$ with $t>0$ we have
\[ 
|x^\lambda|= x^t\leq 1
\]
for all $0\leq x\leq 1$. 
From this and the Cauchy\vvv Schwarz inequality we
see that
\[
|\Phi(\lambda)|\leq ||h||_2\quad\colon \lambda\in U_+\tag{3}
\]


\noindent 
Hence $\Phi$ is a bounded analytic function in
the upper half-plane. At the same time
(1) means that the zero set of
$\Phi$ contains the sequence 
$\{\lambda\uuu\nu\cdot i\}$.
By the  integral formula formula we have seen in XX that this entails that
\[
\sum\,\frac{1}{\lambda_\nu}<\infty\tag{*}
\]
which proves the necessity.

\medskip

\noindent
\emph{Proof of sufficiency.}
There remains to show that ifwe have the convergence in 
(*) above then
there exists a non-zero
$h$-function i $L^2[0,1]$ such that
(1) above holds. 
To find $h$ we first construct an analytic function
$\Phi$ by
\[
\Phi(z)= \frac{\prod_{\nu=1}^\infty\,
\bigl(1-\frac{z}{\lambda_\nu}\bigr)}
{ \prod_{\nu=1}^\infty\,\bigl(1+\frac{z}{\lambda_\nu}\bigr)}\cdot
\frac{1}{(1+z)^2}\quad\colon\quad \mathfrak{Re}\, z>0\tag{i}
\]
Notice that $\Phi(z)$ is defined in the right half-plane
since 
the series (*) is convergent. 
When $\mathfrak{Re}(z)\geq 0$ we notice that
each quotient
\[
\frac{1-\frac{z}{\lambda_\nu}}
{1+\frac{z}{\lambda_\nu}}
\]
has absolute value $\leq 1$.
It follows that
\[ 
|\Phi(x+iy)|\leq\frac{1}{1+x+iy|^2}=\frac{1}{(1+x)^2+y^2}\tag{ii}
\]
In particular the function $y\mapsto \Phi(iy)$
belongs to  $L^2$ on the real $y$\vvv line.
Now we set
\[
f(t)=
\frac{1}{2\pi}
\int_{-\infty}^\infty\, e^{ity}\cdot
\Phi(iy)\cdot dy\tag{ii}
\]
using the inequality (ii)
If $t<0$
we can move the line integral of $e^{tz}\cdot \Phi(z)$ from
the imaginary axis to a line $\mathfrak{Re}(z)= a$
for every $a>0$ and it is clear that

\[
\lim\uuu{a\to +\infty}\, 
\int_{-\infty}^\infty\, e^{\vvv at+ity}\cdot
\Phi(a+iy)\cdot dy=0
\]
We conclude that $f(t)=0$ when $t<0$.
Next, since $y\mapsto \Phi(iy)$ is an $L^2$\vvv function it follows by
Parseval's equality that

\[
\int\uuu 0^\infty\, |f(t)|^2\cdot dt<\infty
\]
Moreover, for a fixed $\lambda\uuu \nu$ we have
\[
\int_0^\infty\, f(t) e^{-\lambda_\nu t}\cdot dt=
\frac{1}{2\pi}\cdot \int\uuu 0^\infty\,[
\int_{-\infty}^\infty\, e^{ity}\cdot
\Phi(iy)\cdot dy\,]\cdot e^{\vvv \lambda\nu t}\cdot dt=
\]
\[
\int_{-\infty}^\infty\, \frac{1}{iy\vvv \lambda\uuu\nu}\cdot 
\Phi(iy)\cdot dy
\]
where the last equality follows when the repeated integral is reversed.
By construction $\Phi(z)$ has a zero at $\lambda\uuu\nu$
and therefore (xx) above remains true with
$\Phi$ replaced by
$\frac{\Phi(z)}{z\vvv \lambda\uuu\nu}$ which entails that

\[
\int\uuu 0^\infty\, f(t)\cdot e^{\vvv \lambda\uuu\nu t}\cdot dt=0
\]

\noindent
At this stage we obtain the requested
$h$\vvv function. Namely, since $t\mapsto e^{\vvv t}$ identifies
$(0,+\infty)$ with $(0,1)$ we get a function $h(x)$ on $(0,1)$
such that
\[ 
h(e^{\vvv t})= e^{t/2}\cdot f(t)
\]
The reader may verify
that
\[
\int\uuu 0^1\, |h(x)|^2\cdot dx=\int\uuu 0^\infty\, |f(t)|^2\cdot dt
\]
and hence $h$ belongs to $L^2(0,1)$.
Moreover, one verifies that the vanishing in (xx) above entails that
\[
\int\uuu 0^1\,  h(x)\cdot x^{\lambda\uuu\nu}\cdot dx=0
\]
Since this holds for every $\nu$ we have proved the
sufficiency which therefore finishes the proof of Theorem XX.

















\bigskip

\centerline {\bf{D.2 Another density result}}
\bigskip

\noindent
Density results using exponential functions are used in many
applications. For example, 
equi-distant sequences  appear in the
the \emph{Sampling Theorem} by Shanning
used in telecommunication engineering.
When this
result is expressed in analytic function theory
via characteristic functions it 
can be formulated as follows:
\medskip

\noindent {\bf {D.3  Theorem.}}
\emph{Let $T>0$ and $g(t)$ is an $L^2$-function on the interval $[0,T]$
which is not identically zero and suppose that
$a>0$ is such that}
\[
\int_0^T\, e^{inat}\cdot g(t)\cdot dt=0
\quad\colon\quad n\in{\bf{Z}}
\]
\emph{Then we must have}
\[ 
a\geq\frac{2\pi}{T}
\]
\medskip

\noindent
{\bf Remark} This result  is due to Fritz Carlson in his article
[xx] from 1914.
let us recall   that Carlson and Carleman later became 
collegues at Stockholm University for more than
two decades.
Carlson's result was later improved by Titchmarsh
and goes as follows:

\bigskip

\noindent
{\bf{D.4 Theorem.}}
\emph{Let $0<m\uuu 1<m\uuu 2<\ldots$
be an increasing sequence of positive real numbers such that}
\[
\limsup\uuu{n\to\infty}\, \frac{n}{m\uuu n}>1\tag{*}
\]
\emph{Then if $f(x)\in L^2(0,2\pi )$ and}
\[
\int\uuu 0^{2\pi}\, e^{\frac{+}{-} im\uuu n x}\cdot f(x)\cdot dx=0\tag{**}
\]
\emph{hold for each $n$, it follows that $f=0$.}
\medskip

\noindent
We give a proof below taken from the text\vvv book
[Paley\vvv Wiener: page
84\vvv 85].
To begin with we notice that
(**) implies that we also have vanishing integrals using
$f(\vvv x)$.
Replacing $f$ by $f(x)+f(\vvv x)$ or $f)x)\vvv f)\vvv x)$
it suffices to prove the result when $f$ is even or odd.
Let us show that there cannot exist an even
$L^2$\vvv function $f$ such that the integrals (**) vanish while
\[
\int\uuu {-\pi}^\pi\, f(x)\cdot dx=1\tag{i}
\]
With  $f$ as above we set
\[
\phi(z)= \int\uuu{\vvv \pi}^\pi\, e^{izt}\cdot f(t)\cdot dt\tag{ii}
\]
The entire function $\phi$ is even and (i) means that
$\phi(0)=1$.
Moreover, $\phi$  is an entire function of exponential type
and Cauchy\vvv Schwarz inequality gives
\[
|\phi(x+iy)|\leq ||f||\uuu 2\cdot e^{\pi|y|}\tag{iii}
\]
for all $z=x+iy$.
Next, Parseval's equality shows that
the restriction of $\phi$ to the real $x$\vvv line belongs to
$L^2$ which by the Remark in (xx) implies
that $\phi$ belong to the Carleman class $\mathcal N$.
So Theorem � xx [Section $\mathcal E$] gives the existence of a limit
\[
\lim\uuu{R\to \infty}\,
\frac{N\uuu\phi(R)}{R}= A\tag{iv}
\]
The inequality (iiii) and the result in XXX entails that
\[
A\leq 2\tag{v}
\]

\noindent
Next, since 
the zeros of $\phi$ contains the even sequence $\{m\uuu\nu\}\cup
\{-m\uuu\nu\}$
we have the inequality

\[
N\uuu\phi(m\uuu n)\geq 2n\tag{vi}
\]
At the same time the limit formula (iv) gives:
\[ 
A=\lim\uuu{n\to \infty}\,
\frac{N\uuu\phi(m\uuu n)}{m\uuu n}\tag{vii}
\]
Finally, (vi) and the hypothesis (*) in  Theorem D.4 give

\[
\limsup\uuu{n\to \infty}
\frac{N\uuu\phi(m\uuu n)}{m\uuu n}\geq
2\cdot \limsup\uuu{n\to \infty}\frac{n}{m\uuu n}>2\tag{viii}
\]
This contradicts (v) and we conclude that
a non\vvv zero function  $\phi$ cannot exist which by the
Laplace-Fourier formula (i) entails that
a non-zero $f$ cannot
exist.



\newpage








%\end{document}