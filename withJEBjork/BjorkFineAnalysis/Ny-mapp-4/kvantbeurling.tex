%\documentclass{amsart}



%\usepackage[applemac]{inputenc}
%\addtolength{\hoffset}{-12mm}
%\addtolength{\textwidth}{22mm}
%\addtolength{\voffset}{-10mm}
%\addtolength{\textheight}{20mm}

%\def\uuu{_}

%\def\vvv{-}

%\begin{document}


\centerline{\bf\large{An automorphism on product measures}}

\bigskip


\noindent
{\bf{Introduction.}}
The  results is   expose material from
the article 
[Beurling]. Before  the   measure theoretic study starts 
we
insert comments from [Beurling] about 
the  significance of the main theorem in 0.�� below.

\medskip


\noindent
{\bf{Schr�dinger equations.}}
The article
\emph{Th�orie relativiste de l'electron et l'interpr�tation
de la m�canique quantique} was published 1932. Here
Schr�dinger raised a new and unorthodox question concerning Brownian motions
leading to new mathematical problems of considerable interest.
More precisely, consider a Brownian motion which takes place in a bounded
region $\Omega$ of some euclidian space
${\bf{R}}^d$ for some $d\geq 2$.
At time $t=0$
the densities of particles 
under observation is given by some
non\vvv negative function $f\uuu 0(x)$ 
defined on $\Omega$.
Classically the density at a later time $t>0$
is equal to
a function
$x\mapsto u(x,t)$ where
$u(x,t)$ solves the heat equation
\[
\frac{\partial u}{\partial t}= \Delta(u)
\] 
with boundary conditions 
\[
u(x,0)=f\uuu 0(x)\quad\text{and}\quad
\frac{\partial u}{\partial {\bf{n}}}(x,t)= 0
\quad\text{when }\quad x\in \partial\Omega\quad\text{and}\quad t>0\tag{1}
\]
Schr�dinger took into the account
the reality of quantum physics which
means that in an  actual experiment
the observed density of particles at a time
$t\uuu 1>0$ does not coincide with $u(x,t\uuu 1)$.
He posed the problem to find the most probable development during the time
interval $[0,t\uuu 1)$ which leads to the state at time
$t\uuu 1$.
He concluded  that
the the requested density
function   which
substitutes the heat\vvv solution $u(x,t)$
should belong to a non\vvv linear class of functions formed by
products
\[ 
w(x,t)= u\uuu 0(x,t)\cdot u\uuu 1(x,t)\tag{*}
\]
where $u\uuu 0$ is a solution to (1) 
while $u\uuu 1(x,t)$ is a solution to an adjoint equation
\[
\frac{\partial u\uuu 1}{\partial t}= \vvv\Delta(u)
\quad\colon\quad
\frac{\partial u\uuu 1}{\partial {\bf{n}}}(x,t)= 0
\quad\text{on}\quad \partial\Omega\tag{2}
\] 
defined when $t<t\uuu 1$.
This leads to a new type of Cauchy problems
where one asks if there exists a  $w$\vvv function given by (*)
satisfying
\[ 
w(x,0)= f\uuu 0(x)\quad\colon\quad w(x,t\uuu 1)=f\uuu 1(x)
\]
where $f\uuu 0,f\uuu 1$ are non\vvv negative functions
such that
\[
\int\uuu\Omega\, f\uuu 0\cdot dx=
\int\uuu\Omega\, f\uuu 1\cdot dx
\]
\medskip

\noindent
The solvability of this non\vvv linear boundary value problem was left open
by Schr�dinger and the search for  solutions has remained as an active field
in mathematical physics.
When $\Omega$ is a bounded set
and has a smooth boundary
one can use the Poisson\vvv Greens function for the
classical equation (*)  and 
rewrite Schr�dinger's equation  to 
a system of non\vvv linear integral equations.
The interested reader should consult the talk by I.N. Bernstein a the IMU\vvv congress at Z�rich 1932 for a first account about mathematical solutions to
Schr�dinger equations.
Examples occur already  on the product
of two copies of the real line where Schr�dinger's equations lead to
certain non\vvv linear equation for measures which goes as follows:
Consider the Gaussian density function


\[
g(x)=\frac{1}{\sqrt{2\pi}}\cdot e^{\vvv x^2/2}
\]

\medskip

\noindent
Next, consider the family $\mathcal S\uuu g^*$
of all non\vvv negative product measures
$\gamma\uuu 1\times\gamma\uuu 2 $ for which
\[
\iint g(x\uuu 1\vvv x\uuu 2)\cdot d\gamma\uuu 1(x\uuu 1)
\cdot d\gamma\uuu 2(x\uuu 2)=1\tag{i}
\]
The product measure gives another product measure 
\[
\mathcal T\uuu g(\gamma\uuu 1\times\gamma\uuu 2)=
\mu\uuu 1\times\mu\uuu 2
\]
where
\[
\mu\uuu 1(E\uuu 1)\cdot \mu\uuu 2(E\uuu 2)=
\iint \uuu {E\uuu 1\times E\uuu 2}\,
g(x\uuu 1\vvv x\uuu 2)\cdot d\gamma\uuu 1(x\uuu 1)
\cdot d\gamma\uuu 2(x\uuu 2)
\] 
hold for all pairs of bounded Borel sets.
Notice that $\mu\uuu 1\times \mu\uuu 2$
becomes a probability measure since (i) above holds.
With these notations one has

\medskip

\noindent
{\bf{0.1 Theorem.}}
\emph{For every product measure
$\mu\uuu1\times\mu\uuu2$ which in addition is a probability measure there exists
a unique $\gamma\uuu 1\times\gamma\uuu 2$ in $S^*\uuu g$
such that}
\[
\mathcal T\uuu g(\gamma\uuu 1\times\gamma\uuu 2)=\mu\uuu1\times\mu\uuu2
\]
\medskip


\noindent
In [Beurling]
a more general result is established
where the  $g$\vvv function can be replaced by
an arbitrary  non\vvv negative and bounded function $k(x\uuu 1,x\uuu 2)$
such that
\[
\iint\uuu{{\bf{R}}^2}\,\log \, k\cdot dx\uuu 1dx\uuu 2>\vvv \infty
\]
\bigskip
  

\centerline{\bf{1. The $\mathcal T$\vvv operator and
product measures}}

\bigskip


\noindent
Let $n\geq 2$ and consider an $n$\vvv tuple of
sample spaces $\{X\uuu\nu=(\Omega\uuu\nu,\mathcal B\uuu\nu)\}$.
We get the product space
\[ 
Y=\prod X\uuu\nu
\]
whose sample space is the set\vvv theoretic product
$\prod\,\Omega\uuu\nu$ and   Boolean $\sigma$\vvv algebra 
$\mathcal B$ generated by
$\{\mathcal  B\uuu \nu\}$.
\medskip


\noindent
{\bf{0.1 Product measures.}}
Let $\{\gamma\uuu\nu\}$ be an $n$\vvv tuple of signed measures on
$X\uuu 1,\ldots,X\uuu n$ where  each $\gamma\uuu\nu$
has a finite total variation. There exists a unique measure $\gamma^*$
on $Y$ such that
\[ 
\gamma^*(
E\uuu 1\times\ldots\times E\uuu n)=
\prod\,\gamma\uuu\nu(E\uuu\nu)
\]
hold for every $n$\vvv tuple of $\{\mathcal B\uuu\nu\}$\vvv measurable sets.
We refer to $\gamma^*$ as the product measure. It is
uniquely determined  because $\mathcal B$ is generated
by
product sets $E\uuu 1\times\ldots\times E\uuu n)$
with each $E\uuu\nu\in\mathcal B\uuu\nu$.
When no confusion is possible we put
\[
 \gamma^*=\prod\,\gamma\uuu\nu
\]

\noindent
The family of all such product measures is denoted by $\text{prod}(\mathcal M\uuu B)$.

\medskip

\noindent
{\bf{0.2 Remark.}}
The set of product measures is a proper non\vvv linear subset of the space
$\mathcal M\uuu B$ of all signed measures on $Y$.
This is already seen when $n=2$ with two discrete sample spaces, i.e.
$X\uuu 1$ and $X\uuu 2$  consists of $N$
points for some integer $N$. A
Every $N\times n$\vvv matrix with 
non\vvv negative elements $\{a\uuu{jk}\}$
give a probability measure $\mu$ on
$X\uuu 1\times X\uuu 2$
when the double sum $\sum\sum\, a\uuu{jk}=1$
The condition that $\mu$ is a product measure is tha there exist
$N$\vvv tuples $\{\alpha\uuu j$ and $\{\beta\uuu k\}$
such that $\sum\,\alpha\uuu \nu=\sum\,\beta\uuu k=1$
and 
$a\uuu{jk}=\alpha\uuu j\cdot \beta\uuu k$.
\bigskip

\noindent
{\bf{0.3 The space $\mathcal A$}}.
We have the linear space 
of functions on $Y$
whose elements are of the form
\[
a=g\uuu 1^*+\ldots+g\uuu n^*\tag{i}
\] 
where  $\{g\uuu\nu\} $  are functions on the separate
product factors
$\{X\uuu\nu\}$. It is clear that a pait of product measures
$\gamma$ and $\mu$ on $Y$ are equal if and only if
\[
\int\uuu Y\, a\cdot d\gamma=
\int\uuu Y\, a\cdot d\mu
\] 
hold for every $a\in\mathcal A$.
\medskip


\noindent
{\bf{0.4 The measure $e^a\cdot \gamma^*$}}
Let $a=\sum\, g^*\uuu\nu$ be as above.
Then we get the exponential function
\[
e^a=\prod \, e^{g^*\uuu\nu}
\]
If $\gamma^*=\prod\,\gamma\uuu\nu$ is some product measure
we get a new product measure defined by

\[
e^a\cdot \gamma\uuu *= \prod\, e^{g\uuu\nu}\cdot \gamma\uuu\nu
\]


\noindent
{\bf{0.5 The $\mathcal T$\vvv operators.}}
To every bounded
$\mathcal B$\vvv measurable function $k$ we shall construct a map $\mathcal T\uuu k$
from the space of product measures into itself.
First, let 
$1\leq\nu\leq n$ be given and  $g\uuu\nu $ is some
$\mathcal B\uuu\nu$\vvv measurable function. 
Then there exists the function $g\uuu\nu^*$
on the product space $Y$ defined by
\[
g\uuu\nu ^*(x\uuu 1,\ldots,x\uuu n)=g\uuu\nu (x\uuu \nu)
\]
Let us now consider a product measure
$\gamma$.  If $1\leq\nu\leq n$ we find a unique measure on $X\uuu \nu$
denoted by $(k\cdot\gamma)\uuu\nu$ such that
\[
\int\uuu Y\, g^*\uuu\nu\cdot k\cdot d\gamma=
\int\uuu{X\uuu\nu}\, g\uuu\nu\cdot d(k\cdot\gamma)\uuu\nu
\]
hold for every bounded
$\mathcal B\uuu\nu$\vvv measurable function $g\uuu\nu$ on $X\uuu\nu$.
Now we get
the product measure
\[ 
\mathcal T\uuu k(\gamma)=\prod\, (k\gamma)\uuu\nu\tag{*}
\]

\medskip

\noindent
{\bf{Remark.}}
In the the case when  
\[
k(x\uuu 1,\ldots,x\uuu n)=g^*\uuu 1\cdots g^*\uuu n
\]
we see that 
\[ 
\mathcal T\uuu k(\gamma)= \prod\, g\uuu\nu\cdot \gamma\nu
\]
\medskip

\noindent
{\bf{Exercise.}}
Consider the case $n=2$ where $X\uuu 1$ and $X\uuu 2$ both consist of two points, say
$a\uuu 1,a\uuu 2$ and $b\uuu 1,b\uuu 2$ respectively.
A measure $\gamma\in S^*\uuu 1$
is given by $\gamma\uuu 1\times\gamma\uuu 2$ and we can identify
this product measure by a $2\times 2$\vvv matrix
\[
XXX
\]
where $\alpha\uuu i\cdot \beta\uuu \nu$ is the mass of $\gamma$ at
the point $(a\uuu i,b\uuu \nu)$. Next, let $k$ be a positive function on
the product space which means that we assign four positive numbers
\[
k\uuu{i,\nu}=
k(a\uuu i,b\uuu\nu)
\]
Find the measure $\mathcal T\uuu k(\gamma)$ and
express it as above by a $2\times 2$\vvv matrix.



\bigskip





\noindent
Now we are prepared to announce the main result in
this section.
Consider a positive $\mathcal B$\vvv measurable function $k$
such that  $k$ and $k^{\vvv 1}$
both are bounded functions.
Denote by $\mathcal S^*\uuu k$ the family of non\vvv negative product measures
$\gamma$ on $Y$ such that
\[
\int\uuu Y\, k\cdot d\gamma=1
\]
We have also the set $\mathcal S^*\uuu 1$ 
of product measures $\mu$ which are non\vvv negative and have
total mass one, i.e.
\[
\int\uuu Y\, d\mu=1
\]


\noindent
It is easily seen   that $\mathcal T\uuu k$ yields
an injective
map from  $S^*\uuu k$ into $S^*\uuu 1$. It turns out that the map also
is surjective, i.e. the following hold:




\medskip

\noindent
{\bf{Main Theorem.}}
\emph{$\mathcal T\uuu k$ yields a homeomorphism between $S^*\uuu k$ and $S^*\uuu 1$}.

\bigskip

\noindent
{\bf{0.6 Remark.}}
Above we refer to  the norm topology on the space of measure, i.e.
if $\gamma\uuu 1$ and $\gamma\uuu 2$ are two  measures on
$Y$ then
the norm $||\gamma\uuu 1\vvv \gamma\uuu 2||$ is the total variation of
the signed measure $\gamma\uuu 1\vvv\gamma\uuu 2$.
The reader may verify that
$S^*\uuu k$ and $S^*\uuu 1$
both appear as closed subsets in the
normed space of all signed measures on $Y$.
Recall also from XX that the space of measures on $Y$
is complete under this norm.
In particular, let $\{\mu\uuu \nu\}$
be a Cauchy sequence with respect to the norm where
each $\mu\uuu\nu\in \mathcal S^*\uuu 1$. Then there exists a strong limit
$\mu^*$ where $\mu^*$ again belongs to $\mathcal S^*\uuu 1$
and
\[
||\mu\uuu\nu\vvv \mu^*||\to 0
\]
This completeness property will
be used in the subsequent proof.
We shall also need some inequalities
which are announced below.

\medskip

\noindent
{\bf{0.7 Some useful inequalities.}}
Let $\gamma\uuu 1$ and $\gamma\uuu 2$
be a pair of product measures 
such that
\[ 
\bigl|\int\uuu Y\, g^*\uuu \nu\cdot d\gamma\uuu 1\vvv
\int\uuu Y\, g^*\uuu \nu\cdot d\gamma\uuu 2\bigr |\leq\epsilon
\quad\colon\quad 1\leq\nu\leq n
\] 
hold for 
some $\epsilon>0$
and every function $g\uuu\nu$ on $X\uuu\nu$ with maximum norm
$\leq 1$. Then the norm
\[ 
||\gamma\uuu 1\vvv \gamma\uuu 2||\leq n\cdot \epsilon\tag{i}
\]


\noindent
The proof of (i) is  left to the reader
where the hint is to 
make repeated  use of  Fubini's theorem.
More generally, let
$k$ be a bounded measurable function on $Y$
and $\gamma,\mu$ is a pair of product measures.
Denote by $\mathcal A\uuu *$ the set of
$\mathcal A$\vvv functions $a$ with maximum norm
$\leq 1$. Then there exists a constant $C$ which only depends on $k$ and
$n$ such that




\[
||\mathcal T\uuu k(\mu)\vvv \gamma||\leq
\max\uuu {a\in A\uuu *}\, \bigl|\int\uuu Y\, a(kd\mu\vvv d\gamma)\,\bigr|\tag{*}
\]

\noindent
Again we leave the proof as an exercise.
\medskip










\noindent
{\bf{0.8 A variational problem.}}
Since we already have observed that
$\mathcal T\uuu k$ is injective there remains to
prove surjectivity.
For this we shall study a
a variational
problem which we begin to describe  before the proof is finished in 0.�� X below.
We are given the function $k$ on $Y$ where both $k$ and $k^{\vvv 1}$
are bounded and the space
$\mathcal A$ was defined in 0.3.
For every pair  $\gamma\in \mathcal S^*\uuu 1$ and
$a\in \mathcal A$  we set
\[ 
W(a,\gamma)=\int\uuu Y\, (e^ak\vvv a)\cdot d\gamma 
\quad\text{and}\quad W\uuu *(\gamma)=
 \min\uuu{a\in\mathcal A}\,W(a,\gamma)
\]




\medskip

\noindent
{\bf{0.9 Proposition.}} \emph{Let $\{a\uuu \nu\}$ be a sequence in
$\mathcal A$ such that}
\[
\lim W(a\uuu\nu,\gamma)= W\uuu*(\gamma)
\]
\emph{Then the sequence  $\{e^{a\uuu \nu}\cdot \gamma\}$
converges to a   measure
$\mu\in S^*\uuu 1$ such that $\mathcal T\uuu k(\gamma)=\mu$.}




\medskip

\noindent
Before we enter the proof we insert a preliminary result which will
be used later on.
\medskip

\noindent
{\bf{0.10. Lemma.}} 
\emph{Let $\epsilon>0$ and $a\in\mathcal A$ be such that
$W(a)\leq W\uuu *(\gamma)+\epsilon$. Then it follows that}

\[
\int\, e^a\cdot k\cdot \gamma\leq 
\frac{1+\epsilon}{1\vvv e^{\vvv 1}}
\]

\noindent
\emph{Proof.}
For every real number $s$   the function $a\vvv s$
again belongs to $\mathcal A$ and by the hypothesis 
$W(a\vvv s)\geq W(a)\vvv \epsilon$. This entails that

\[
\int\, e^a k\cdot d\gamma\leq
\int\uuu Y\, e^{a\vvv s}\cdot kd\gamma+s\int\, k\cdot d\gamma+\epsilon
\implies
\]
\[
\int (1\vvv e^{\vvv s})\cdot e^a\cdot kd\gamma\leq s+\epsilon
\]
Lemma 0.10 follows if we take $s=1$.






\bigskip


\noindent
\emph{Proof of Proposition 0.9} Keeping $\gamma$ fixed we set
$W(a)= W(a,\gamma)$.
Let $0<\epsilon<1$ and consider a pair $a,b$ in $\mathcal A$ such that
$W(a)$ and $W(b)$ both are $\leq W\uuu *(\gamma)+ \epsilon$.
Since  $\frac{1}{2}(a+b)$
belongs to $\mathcal A$  we get
\[
2\cdot W(\frac{1}{2}(a+b))\geq 2\cdot W\uuu *(\gamma)
\geq W(a)+W(b)\vvv 2\epsilon\tag{i}
\]
Notice that
\[
W(a)+W(b)\vvv
2\cdot W(\frac{1}{2}(a+b))=
\int\uuu Y\, [e^a+e^b\vvv 2\cdot e^{\frac{1}{2}(a+b))}]\cdot kd\gamma\tag{ii}
\]
Next,  we have  the algebraic identity
\[
e^a+e^b\vvv 2\cdot e^{\frac{1}{2}(a+b))}]=(e^{a/2}\vvv e^{b/2})^2
\]
It follows from (i\vvv ii) that
\[
\int\uuu Y\, (e^{a/2}\vvv e^{b/2})^2\cdot k\cdot d\gamma \leq 2\epsilon\tag{iii}
\]
Next, the identity
$|e^a\vvv e^b)|=
(e^{a/2}+e^{b/2})\cdot |e^{a/2}\vvv e^{b/2}|$
and the Cauchy\vvv Schwarz inequality give:

\[
\bigl[\int\uuu Y\, |e^a\vvv e^b|\cdot k\cdot d\gamma\,\bigr ]^2\leq
2\epsilon\cdot \int\uuu Y\, (e^{a/2}+e^{b/2})\cdot k\cdot d\gamma\tag{iv}
\]
By Lemma 0.6  the last factor is bounded by
a fixed constant and hence (iv) gives a constant $C$ such that
\[
\int\uuu Y\, |e^a\vvv e^b|\cdot k\cdot d\gamma
\leq C\cdot \sqrt{\epsilon}\tag{v}
\]
Next, let $k\uuu *$ be the minimum value taken by
$k$ on $Y$ which by assumption is positive since
$k^{\vvv 1}$ is bounded.
Replacing $C$ by $C/k\uuu *$ where 
we get
\[
\int\uuu Y\, |e^a\vvv e^b|\cdot d\gamma
\leq C\cdot \sqrt{\epsilon}\tag{vi}
\] 


\noindent
Now (v) applies to pairs in the sequence $\{a\uuu\nu\}$
and shows that
$\{e^a\cdot d\gamma\}$ is a Cauchy sequence with respect 
to the norm of measures on $Y$. So from Remark 0.6
there exists a non\vvv negative measure $\mu$ such that
\[
\lim\uuu{\nu\to \infty}\, ||e^{a\uuu\nu}\cdot\gamma\vvv \mu||=0\tag{vii}
\]

\medskip


\noindent
\emph {The equality $\mathcal T\uuu k(\mu)=\gamma$}.
Consider the 
$a$\vvv functions in the minimizing sequence.
If $\rho\in \mathcal A$ is arbitrary we have
\[ 
W(a\uuu\nu+\rho)\geq W(a\uuu\nu)\vvv \epsilon\uuu\nu
\] 
where $\epsilon\uuu\nu\to 0$.
This gives
\[
\int\uuu Y\, [ke^{a\uuu\nu}(1\vvv e^\rho)+\rho]\cdot d\gamma\leq \epsilon\uuu\nu\tag{1}
\]
When the maximum norm
$|\rho|\uuu Y\leq 1$ we can write
\[
e^\rho=1+\rho+\rho\uuu 1\quad\text{where}\quad
 0\leq \rho\uuu 1\leq \rho^2\tag{2}
\]
Then we see that (1) gives
\[
\int\uuu Y\,(\rho\vvv ke^{a\uuu\nu}\cdot \rho)\cdot d\gamma\leq
\epsilon\uuu\nu +\int\,\rho\uuu 1\cdot \gamma\leq
\epsilon+||\rho||\uuu Y^2\tag{3}
\]
where the last inequality follows since
$\gamma$ is a probability measure and the inequality in (2) above.
The same inequality holds with $\rho$ replaced by $\vvv \rho$
which entails that
\[
\bigl|\int\uuu Y\,(ke^{a\uuu\nu}\vvv 1)\cdot \rho\cdot d\gamma\bigr|
\leq\epsilon\uuu\nu+||\rho||\uuu Y^2
\]

\noindent
Notice that Lemma 0.10 entails that
the sequence of functions
$\{ke^{a\uuu\nu}\}$ are uniformly bounded.
Now we apply the inequality (*) from 0.7
while we use
$\rho$\vvv functions in $\mathcal A$ of norm $\leq\sqrt{\epsilon\uuu\nu}$.
It follows
that there exists a constant $C$ which is independent of $\nu$
such that
the  following inequality for the total variation:
\[
||\mathcal T\uuu k(e^{a\uuu\nu}\cdot\gamma)\vvv \gamma||\leq C\cdot n
\cdot \frac{1}{\sqrt{\epsilon}}\cdot
(\epsilon\uuu\nu+\epsilon\uuu\nu)= 2\cdot Cn\cdot \sqrt{\epsilon\uuu\nu}
\]
Passing to the limit it follows from (vii) that we have
the equality
\[ 
\mathcal T\uuu k(\mu)=\gamma
\]

\noindent
Since $\gamma\in S^*\uuu1$ was arbitrary
we have proved that
the $\mathcal T\uuu k$ yields a surjective map from
$S^*\uuu k$ to $S^*\uuu 1$
which finishes the proof of the
Main Theorem.







\bigskip


\centerline {\bf{  0.11 The  singular case.}}
\bigskip

\noindent
We restrict to the case $n=2$ where
$k(x\uuu 1,x\uuu 2)$ is a bounded and strictly positive 
continuous function on
$Y= X\uuu 1\times X\uuu 2$. Let $\gamma\in S^*\uuu 1$ satisfy:
\[
\int\uuu Y\, \log k\cdot d\gamma>\vvv \infty\tag{1}
\]
Under this integrability condition the following hold:

\medskip

\noindent
{\bf{2. Theorem.}}
\emph{There exists a unique non\vvv negative 
product measure
$\mu$ on $Y$ such that
$\mathcal T\uuu k(\mu)=\gamma$.}
\medskip

\noindent
{\bf{Remark.}}
In general the measure $\mu$ need not have finite mass
but the proof shows that
$k$ belongs to $L^1(\mu)$, i.e.
\[
\int\uuu Y\, k\cdot d\mu<\infty
\]


\noindent
As pointed out by Beurling  Theorem 0.12
can be applied to the case $X\uuu 1=X\uuu 2={\bf{R}}$
both are copies of the real line and
\[
k(x\uuu1,x\uuu 2)=g(x\uuu 1\vvv x\uuu 2)
\] where $g$ is the density of a Gaussian distribution which after  a
normalisation  of
the variance is taken to be
\[
\frac{1}{\sqrt{2\pi}}\cdot e^{\vvv t^2/2}
\]
So the integrability condition for $\mu$  becomes

\[
\iint\, (x\uuu 1\vvv x\uuu 2)^2\cdot d\mu(x\uuu 1,x\uuu 2)<\infty
\] 
\medskip

\noindent
A proof of Theorem 0.12  is given on page 218\vvv 220
in [loc.cit] and relies upon similar but technically more involved
methods as in the Main Theorem.
Concerning higher dimensional cases,  i.e.
singular versions of the Main Theorem when
$n\geq 3$, 
Beurling gives the
following comments at the end of [ibid]
where the citation below has changed numbering of
the theorems as compared to [ibid]:
\medskip

\noindent
\emph{The proof of the Main Theorem 
relies heavily on the condition that $k\geq a$ for some
$a>0$.
If this lower bound condition is dropped
the individual equation $\mathcal K(\gamma)=\mu$ may still
be meaningful, but serious complications will
arise concerning the global uniqueness if $n\geq 3$
and the proof of Theorem 0.12  for the case $n\geq 3$ cannot be duplicated.}






















%\end{document}

