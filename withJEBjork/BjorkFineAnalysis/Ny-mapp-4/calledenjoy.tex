

%\documentclass{amsart}
%\usepackage[applemac]{inputenc}


%\addtolength{\hoffset}{-10mm}
%\addtolength{\textwidth}{20mm}
%\addtolength{\voffset}{-10mm}
%\addtolength{\textheight}{20mm}

%\def\uuu{_}


%\def\vvv{-}


%\begin{document}



\centerline{\bf\large{XI. The Denjoy conjecture}}

\bigskip

\noindent {\bf Introduction.}
Let $\rho$ be a positive integer and 
$f(z)$  is an entire function such that there exists some
$0<\epsilon<1/2 $  and a constant $A_\epsilon$ such that
\[
|f(z)|\leq A_\epsilon\cdot e^{|z|^{\rho+\epsilon}}\tag{0.1}
\]


\noindent
hold for every $z$. Then we say that 
$f$ has integral order $\leq\rho$.
Next, the entire function $f$ has an asymptotic value $a$
if there exists a Jordan curve
$\Gamma$ parametrized
by
$t\mapsto\gamma(t)$ for $t\geq 0$ such that
$|\gamma(t)|\to \infty$ as $t\to+\infty$ and
\[
\lim_{t\to+\infty}\, f(\gamma(t))=a\tag{0.2}
\]


\noindent
In 1920 Denjoy raised the conjecture that
(0.1) implies that the entire function $f$ has at most
$2\rho$ many different asymptotic values. Examples show that
this upper bound is sharp.
The Denjoy conjecture was proved in 1930 by  Ahlfors in [Ahl].
A few years later T. Carleman found an alternative  proof based
upon a certain differential inequality.
Theorem A.3 below 
has  applications beyond the proof of
the Denjoy conjecture
for estimates of
harmonic measures. See [Ga-Marsh].

\bigskip

\centerline{\bf A. The  differential inequality.}
\bigskip


\noindent
Let $\Omega$ be a connected open set in ${\bf{C}}$ whose
intersection $S\uuu x$ between
a  vertical line  $\{\mathfrak{Re}\, z=x\}$
is a bounded set
on the real $y$-line for every $x$.
When $S_x\neq\emptyset $ it is the disjoint union
of open
intervals $\{(a_\nu,b_\nu)\}$ and we set
\[ 
\ell(x)=\max\uuu \nu \,(b_\nu-a_\nu)\tag{*}
\]

\medskip


\noindent
Next, let
$u(x,y)$ be a positive harmonic   function
in $\Omega$ which 
extends to a continuous function on
the closure $\bar\Omega$  with the boundary values identical to zero.
Define the function $\phi$ by:
\[ 
\phi(x)=\int_{S_x}\,u^2(x,y)\cdot dy\tag{1}
\]
The Federer-Stokes theorem gives
the following  formula for the derivatives of $\phi$:
\[ 
\phi'(x)=2\int_{S_x}\,u_x\cdot u(x,y) dy\tag{2}
\]
\[
\phi''(x)=2\int_{S_x}\,u_{xx}\cdot u(x,y) dy+
2\int_{S_x}\,u^2_x\cdot dy\tag{3}
\]


\noindent
Since $\Delta(u)=0$ when $u>0$ we have
\[
2\int_{S_x}\,u_{xx}\cdot u(x,y) dy=-
2\int_{S_x}\,u_{yy}\cdot u(x,y) dy=
2\int u_y^2 dy\tag{4}
\]
The Cauchy-Schwarz inequality applied in (2) gives
\[
\phi'(x)^2\leq 4\cdot \int_{S_x}\,u^2_x\cdot\int_{S_x} u^2(x,y) dy
=4 \cdot \phi(x)\cdot \int_{S_x}u_x^2dy\tag{5}
\]
Hence (4) and (5) give:
\[ 
\phi''(x)\geq 2\int\uuu{S\uuu x} \,u^2_y(x,y)\cdot dy+\frac{1}{2}\cdot \frac{\phi'^2(x)}{\phi(x)}\tag{6}
\]
Next, since $u(x,y)=0$ at the end-points of all intervals of $S_x$,
\emph{Wirtinger's 
inequality}
and the definition of $\ell(x)$ give:
\[
\int\uuu{S\uuu x} u^2_y(x,y)\cdot dy\geq \frac{\pi^2}{\ell(x)^2}\cdot\phi(x)\tag{7}
\]


\noindent
Inserting (7) in (6) we have  proved
\medskip

\noindent
{\bf A.1 Proposition} \emph{The $\phi$-function satisfies the differential inequality}
\[
\phi''(x)\geq \frac{2\pi^2}{\ell(x)^2}\cdot\phi(x)+\frac{\phi'^2(x)}{2\phi(x)}
\]


\noindent
\emph{Proof continued.}
The maximum
principle for harmonic functions implies that the $\phi(x)>0$
when $x>0$ and hence there exists
a $\psi$-function where
$\phi(x)= e^{\psi(x)}$. It follows that

\[ 
\phi'=\psi' e^\psi\quad\text{and}\quad\,\phi''=
\psi''e^\psi+\psi'^2e^\psi
\]

\noindent
Now  Proposition A.1 gives
\[
 \psi''+\frac{\psi'^2}{2}\geq \frac{2\pi^2}{\ell(x)^2}\tag{*}
\]
\medskip


\noindent
{\bf {A.2 An integral inequality.}}
From (*) we obtain
\[
\frac{2\pi}{\ell(x)}\leq\sqrt{\psi'(x)^2+2\psi''(x)}\,\leq
\psi'(x)+\frac{\psi''(x)}{\psi'(x)}
\]
\medskip

\noindent
Taking the integral we get
\[ 
2\pi\cdot\int_0^x\,\frac{dt}{\ell(t)}\leq \psi(x)+\log\,\psi'(x)+O(1)
\leq \psi(x)+,\psi'(x)+O(1)\tag{**}
\]
where $O(1)$ is a remainder 
term which is bounded  independent of $x$. Taking the integral once more we 
obtain:
\medskip

\noindent
{\bf {A.3 Theorem.}}
\emph{The following inequality holds:}


\[
2\pi\cdot\int_0^x\, \frac{x-s}{\ell(s)}\cdot ds\leq
\int_0^x\,\psi(s)\cdot ds+\psi(x)+O(x)
\]
\emph{where the remainder term $O(x)$ is bounded by $Cx$ for 
a fixed constant.}

\bigskip


\centerline {\bf B. Solution to the Denjoy conjecture}
\bigskip

\noindent
{\bf B.1 Theorem.}
\emph{Let $f(z)$ be entire of some integral order
$\rho\geq 1$. Then $f$ has at most $2\rho$ many different asymptotic values.}
\bigskip

\noindent
\emph{Proof.}
Suppose $f$ has $n$ different asymptotic values
$a_1,\ldots,a_n$.
To each  $a_\nu$ there exists a Jordan arc $\Gamma_\nu$ as described in the introduction.
Since the $a$-values are different the $n$-tuple of
$\Gamma$-arcs are separated from each other when
$|z|$ is large.
So we can find some $R$ such that
the arcs are disjoint in the exterior disc
$|z|>R$. We may also consider the tail of each arc, i.e. starting from
the last point on $\Gamma_\nu$ which intersects the circle
$|z|=R$. So now we have an $n$-tuple of disjoint Jordan
curves in $|z|\geq R$ where each curve intersects $|z|=R$ at some point $p_\nu$ and after the curves moves to the point at infinity. See figure.
Next, we take one of these curves, say $\Gamma_1$. Let
$D_R^*$ be the exterior disc $|\zeta |>R$.
In the domain
$\Omega={\bf{C}}\setminus \Gamma_1\cup\,D_R^*$ we can choose a single-valued
branch of $\log \zeta$ and with $z=\log\,\zeta $
the image of $\Omega$ is a simply connected domain 
$\Omega^*$
where
$S_x$ for each $x$ has length strictly less than
$2\pi$
The images of the $\Gamma$-curves
separate $\Omega^*$
into $n$ many disjoint connected
domains
denoted by $D_1,\ldots,D_n$ where
each $D_\nu$ is bordered by a pair of images of $\Gamma$-curves
and a portion of the vertical line $x=\log \,R$.

\medskip

\noindent
Let $\zeta=\xi+i\eta$ be the complex coordinate
in $\Omega^*$.
Here we get the analytic function
$F(\zeta)$ where
\[ F(\text{log}(z))= f(z)
\]
We notice that $F$ may have more growth than $f$. Indeed, we get
\[
|F(\xi+i\eta)|\leq \text{exp}\bigl(e^{(\rho+\epsilon)\xi}\bigr)\tag{1}
\]

\medskip

\noindent
With $u=\text{Log}^+\,|F|$ it follows that
\[ 
u(\xi,\eta)\leq e^{(\rho+\epsilon)\xi}\tag{2}
\]


\noindent
Hence the $\phi$-function constructed
during the proof of Theorem A.3  satisfies
\[
\phi(\xi)\leq e^{2(\rho+\epsilon)\xi}
\]
It follows that the $\psi$-function satisfies


\[ 
\psi(\xi)=2\cdot (\rho+\epsilon)\xi+O(1)\tag{3}
\]


\noindent Now we  apply Theorem A.3  in each region $D_\nu$
where we have a function
$\ell_\nu(\xi)$ constructed by (0) in section A. 
This gives the inequality
\[
2\pi\cdot \int_R^\xi\, \frac{\xi-s}{\ell_\nu(s)}\cdot ds\leq
\int_R^\xi\, (\rho+\epsilon)s\cdot ds+(\rho+\epsilon)\xi+O(1)\quad\colon\quad
1\leq \nu\leq n\tag{4}
\]
\medskip

\noindent
Next, recall the elementary inequality which asserts that if $a\uuu 1,\ldots,a\uuu n$
is an arbitrary $n$\vvv tuple of positive numbers then
\[
\sum\,a\uuu \nu\cdot \sum\, \frac{1}{a_\nu}\geq n^2
\tag{5}
\]
For each $s$ we apply this to the $n$\vvv tuple
$\{\ell\uuu \nu(s)\}$ where we also have
\[ 
\sum\,\ell\uuu \nu(s)\leq 2\pi
\] 
\noindent
So  a summation in (4) over $1\leq \nu\leq n$ gives
\[
n\cdot  \int_R^\xi\, (\xi-s)\cdot ds\leq
\int_R^\xi\, (\rho+\epsilon)s\cdot ds+(\rho+\epsilon)\xi+O(1)\tag{6}
\]
Another integration gives:
\[ 
n\cdot\frac{\xi^2}{2}\leq (\rho+\epsilon)\cdot \xi^2+O(\xi)\tag{7}
\]
\medskip

\noindent
This inequality can only hold for large $\xi$ if
$n\leq 2(\rho+\epsilon)$ and since  $\epsilon<1/2$ is assumed
it follows that 
$n\leq 2\rho$ which finishes the proof of the Denjoy conjecture.



\newpage



%\end{document}




