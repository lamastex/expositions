\documentclass{amsart}
\usepackage[applemac]{inputenc}

\addtolength{\hoffset}{-12mm}
\addtolength{\textwidth}{22mm}
\addtolength{\voffset}{-10mm}
\addtolength{\textheight}{20mm}
\def\uuu{_}

\def\vvv{-}

 
\begin{document}



\centerline{\bf \large{Fourier series}}
\bigskip


\noindent

\centerline{\emph{Contents}}
\bigskip


\noindent
\emph{A: The Fejer kernel}

\bigskip


\noindent
\emph{B: Legendre polynomials}



\bigskip


\noindent
\emph{C. The space $\mathcal T_n$}


\bigskip


\noindent
\emph{D. Tchebysheff  polynomials}


\bigskip


\noindent
\emph{E. Almost periodic functions.}



\bigskip

\noindent
{\bf{Introduction.}}
In section A we recall basic constructions of Fourier series.
The major
results give solutions to extremal problems
such as the theorems in
B.8, C.9  and D.4. The final section is more extensive and
gives an exposition of Harald Bohr's theory about almost periodic functions on 
the real line.


\bigskip

\centerline
{\bf{A: Dini's and Fejer's kernels}}
\bigskip

\noindent
We consider complex-valued 
and continuous functions $f(\theta)$ defined in the interval
$[0,2\pi]$ which satisfiy
$f(0)=f(2\pi)$. To each such function $f$ and every integer $n$ we set
\[ 
\widehat f(n)=\frac{1}{2\pi}\cdot \int_0^{2\pi}\
e^{-in\phi}f(\phi)\cdot d\phi
\]
We refer to $\{\widehat f(n)\}$ as the Fourier coefficients of $f$.
Next, if $N\geq 1$ we set
\[
S_N(\theta)=\sum_{n=-N}^{n=N}\, \hat f(n)\cdot e^{in\theta}
\]
We refer to $S_N$ as Fourier's partial sum function of degree $N$.
\medskip

\noindent
{\bf A.1.The Dini kernel.}
If $N\geq 1$ we set
\[ 
D_N(\theta)=
\frac{1}{2\pi}\sum_{n=-N}^{n=N}\, e^{in\theta}
\]


\noindent
{\bf A.2 Proposition.} \emph{One has the formula}
\[
D_N(\theta)=\frac{\text{sin}((N+\frac{1}{2})\theta)}{\text{sin}\,\frac{\theta}{2}}
\]
\medskip
\noindent
\emph{Proof.}
We have
\[
\sum_{n=-N}^{n=N}\, e^{in\theta}
=e^{-iN\theta}\cdot
\frac{1}{2\pi}\sum_{n=0}^{n=2N}\, e^{in\theta}
=e^{-iN\theta}\cdot
\frac{
e^{i(2N+1)\theta}-1}{e^{i\theta}-1}
\]
Proposition A.2 follows 
after multiplication with
$e^{i\theta/2}$
and  the two equalities
\[
e^{i(N+1/2)\theta}-e^{i(N+1/2)\theta}=2i\cdot\text{sin}((N+1/2)\theta)
\]
\[
e^{i \theta/2}-e^{-i\theta/2}=2i\cdot\text{sin}(\theta/2)
\]


\noindent
{\bf A.3 Exercise.}
Show that
\[ S_N(\theta)=\int_0^{2\pi}\, D_N(\theta-\phi)\cdot f(\phi)\cdot d\phi=
\int_0^{2\pi}\, D_N(\phi)\cdot f(\theta+\phi)\cdot d\phi
\]
\medskip

\noindent
{\bf A.4 The Fejer kernel.}
Given $f$ as above and $N\geq1$ we set
\[
F_N(\theta)=\frac{S_0(\theta)+\ldots+S_N(\theta)}{N+1}\implies
\]
\[ F_N(\theta)=\int_0^{2\pi}\,\mathcal F_N(\phi)\cdot f(\theta+\phi)\cdot d\phi
\quad\text{where}\quad
\mathcal F_N(\phi)= D_0(\phi)\ldots+D_N(\phi)
\]
\medskip

\noindent
{\bf A.5 Proposition} \emph{One has the formula} 
\[
 F_N(\theta)=\frac{1}{N+1}\cdot\frac{1-\text{cos}((N+1)\theta)}{2\cdot \text{sin}^2(\frac{\theta}{2})}
 \]

 
 \noindent
 \emph{Proof.}
To each $\nu\geq 0$ we have
\[
\text{sin}((\nu+1/2)\theta)=
\mathfrak{Im} \bigl [e^{i(\nu+1/2)\theta)}\bigr]
\]
It follows that $F_N(\theta)$ is equal to the imaginary part of
\[
\frac{e^{i\theta/2}}{\text{sin}(\theta/2)}\cdot
\sum_{\nu=0}^{\nu=N}\, e^{i\nu\theta}=
\frac{e^{i\theta/2}}{\text{sin}(\theta/2)}\cdot
\frac{e^{i(N+1)\theta}-1}{e^{i\theta-1}}
=
\frac{e^{i(N+1)\theta}-1}{\text{sin}(\theta/2)}\cdot
\frac{1}{e^{i\theta/2}-e^{-i\theta/2}}=
\]
\[
\frac{e^{i(N+1)\theta}-1}{2i\cdot \text{sin}^2(\theta/2)}
=i\cdot \frac{1-e^{i(N+1)\theta}}{2\cdot \text{sin}^2(\theta/2)}
\]
Finally,  the imaginary part of the last term is equal to
\[
\frac{1-\text{cos}((N+1)\theta)}{2\cdot \text{sin}^2(\theta/2)}
\]
which proves Proposition A.5.

 \bigskip
 
 \noindent
{\bf A.6 A limit  formula.}
When $f$ is given we set
\[ 
\mathcal F_N(\theta)= \int_0^{2\pi}\, F_N(\phi)\cdot f(\theta+\phi)\cdot d\phi
\]
If $a>0$ and $a\leq\theta\leq 2\pi-a$
the sine-function  $\text{sin}^2(\theta/2)$ is bounded below, i.e.
\[
\text{sin}^2(\theta/2)\geq \text{sin}^2(a/2)
\]
So if  $M$ is the maximum norm of
$|f(\theta)|$ over $[0,2\pi]$
it follows that
\[
 \int_a^{2\pi-a}\, F_N(\phi)\cdot f(\theta+\phi)\cdot d\phi
\leq
\] 
\[
\frac{M}{(N+1)\cdot \text{sin}^2(a/2)}
 \int_a^{2\pi-a}\, (1-\text{cos}(N\phi))\cdot d\phi
\leq\frac{2M}{(N+1)\cdot \text{sin}^2(a/2)}
\]
\medskip

\noindent
{\bf A.7 Exercise.}
Given some $\theta_0$ and $a>0$ we set
\[
\omega_f(a)=\max_{|\theta-\theta_0|\leq a}\, |f(\theta)-f(\theta_0)|
\]
Show  that
\[
|\mathcal F_N(\theta_0)-f(\theta_0)|
\leq\frac{2M}{(N+1)\cdot \text{sin}^2(a/2)}+\omega_f(a)
\quad\text{for all}\quad 0<a<\pi
\]
Finally, use the \emph{uniform continuity}
of the function $f$ over the interval $[0,2\pi]$ to  conclude that
the sequence $\{\mathcal F_N\}$ converges uniformly to $f$ over
the interval $[0,2\pi]$.
\medskip







\bigskip

\centerline{\bf{B. Legendre polynomials.}}
\bigskip


\noindent
If $n\geq 1$ we denote by $\mathcal P_n$ the linear space of real-valued
polynomials of degree $\leq n$. A bilinear form
is defined by
\[
\langle q,p\rangle=
\int_{-1}^1\, q(x)p(x)\cdot dx
\]
Since 
$1,x,\ldots,x^{n-1}$ generate a subspace of  co-dimension  one
in $\mathcal P\uuu n$ we get:

\medskip

\noindent
{\bf {B.1 Proposition.}}
\emph{There exists a unique 
$Q_n(x)= x^n+q_{n-1}x^{n-1}+\ldots+q_0$
such that}
\[
\int_{-1}^1\, x^\nu\cdot Q_n(x)\cdot dx=0\leq  \nu\leq n-1
\]
\medskip

\noindent
To find $Q_n(x)$.
we consider the polynomial $(1-x^2)^n$ which vanishes up to order
$n$ at the  end-points 1 and -1. 
Its the derivative of order $n$ gives a polynomial of  degree $n$
and partial integrations show that
\[
\int_{-1}^1\, x^\nu\cdot \partial^n((x^2-1)^n))\cdot dx=0\leq  \nu\leq n-1
\]
The leading coefficient of $x^n$ 
in $\partial^n((x^2-1)^n))$
becomes
\[
c_n=2n(2n-1)\cdots (n+1)
\]
Hence we have
\[ Q_n(x)=\frac{1}{c_n}\cdot \partial ^n((x^2-1)^n)
\]
\medskip

\noindent
{\bf B.2 Definition.}
\emph{The Legendre polynomial of degree
$n$ is  given by}
\[
P_n(x)=k_n\cdot\partial ^n((x^2-1)^n)
\]
\emph{where the constant $k\uuu n$ is determined so that
$P\uuu n(1)=1$.}

\medskip

\noindent
Since $P_n$ is equal to $Q_n$ up to a constant we still have
\[
\int_{-1}^1\, x^\nu\cdot P_n(x)\cdot dx=0\leq  \nu\leq n-1
\]
From this we conclude that 
\[
\int_{-1}^1\, x^\nu\cdot P_n(x)\cdot P_mx)dx=0\quad  n\neq m
\]
Thus, $\{P_n\}$ is an orthogonal family
with respect to the inner product defined by the integral over
$[-1,1]$.
\medskip

\noindent
{\bf B.3 A generating function.}
Let $w$ be a new variable and set
\[
\phi(x,w)=1-2xw+w^2
\]
Notice that $\phi\neq 0$ when
$-1\leq x\leq 1$ and $|w/<1$.
Keeping $-1\leq x\leq 1$ fixed
we have the function
\[ 
w\mapsto
\frac{1}{\sqrt{1-2xw+w^2}}
\]
Recall that when  $|\zeta|<1$ one has the Newton series
\[
\frac{1}{\sqrt{1-\zeta}}=\sum\, g_n\cdot \zeta^n\quad
\text{where}\quad g_n=\frac{3\cdot5\cdots(2n-1)}{2^n}
\]
It follows that
\[
\frac{1}{\sqrt{1-2xw+w^2}}=\sum\, g_n(2xw-w^2)^2
\]
With $x$ kept fixed the series is expanded into $w$-powers, i.e.  set
\[
\frac{1}{\sqrt{1-2xw+w^2}}=\sum\, \rho_n(x)\cdot w^n
\]
It is easily seen that as $x$ varies then
$\rho_n(x)$ is a polynomial of degree
$n$. Moreover, we notice that the coefficient of
$x^n$ in $\rho_n(x)$ is equal to
\[
g_n\cdot 2^n
\]
Next, if $x=1$ we have
\[
\frac{1}{\sqrt{1-2w+w^2}}= \frac{1}{1-w}= \sum\, w^n
\]
From this we conclude that
\[ 
\rho_n(1)=1\quad\text{for all}\quad n\geq 0
\]
\medskip

\noindent
{\bf B.4 Theorem.} \emph{One has the equality $\rho_n(x)=P_n(x)$ for each $n$, i.e.} 
\[
\frac{1}{\sqrt{1-2xw+w^2}}=\sum\, P_n(x)\cdot w^n
\]
\emph{holds when $-1\leq x\leq 1$ and $|w|<1$.}


\bigskip

\noindent
{\bf{B.5 Exercise.}} Prove this result.
\bigskip

\noindent {{\bf{B.6 The series for $P_n(\text{cos}\,\theta)$}}.
With $x$ replaed by $\text{cos}\,\theta$ we notice that
\[
1-2\text{cos}\,\theta\cdot w+w^2=
(1-e^{i\theta}w)(1-e^{-i\theta}w)
\]
It follows that
\[
\frac{1}{\sqrt{1-2\text{cos}(\theta)w+w^2}}=
\frac{1}{\sqrt{1-1-e^{i\theta}w)}}\cdot 
\frac{1}{\sqrt{1-e^{-i\theta}w)}}
\]
The last product becomes
\[ \sum\sum\, g_m e^{im\theta}w^m\cdot g_\nu e^{-i\nu\theta}w^\nu
\]
Collecting $w$ powers the double sum becomes
\[
\sum\, \gamma_n(\theta)\cdot w^n
\quad\gamma_n(\theta)=\sum_{m+\nu=n}\, g_mg_\nu e^{i(m-\nu)\theta}
\]

\noindent
By Theorem B.4 the last sum represents $P_n(\text{cos}(\theta))$.
One has for example
\[ 
P_3(\text{cos}(\theta)=
2g_3\cdot\text{cos}(3\theta)+
2g_2g_1\cdot \text{cos}(\theta)
\]
where we used that $g_0=1$.
\medskip

\noindent
{\bf{B.7 An inequality  for $|P(x)|$.}}
Since the $g$-numbers are  $\geq 0$
we obtain
\[
|P_n(\text{cos}(\theta)|\leq
g_ng_0+
g_{n-1}g_1+\ldots+
g_1g_{n-1}+
g_0g_n=P_n(1)\quad\colon\,\,0\leq\theta\leq 2\pi
\]

\noindent
Hence we have proved
\medskip

\noindent
{\bf B.8 Theorem.}\emph{ For each $n$ one has}
\[ 
|P_n(x)|\leq 1\quad\colon\,\, -1\leq x\leq 1
\]


\noindent
Next, we study the values when $x>1$. Here one has
\medskip

\noindent
{\bf B.9 Theorem.} \emph{For each $x>1$ one has}
\[
1<P_1(x)<P_2(x)<\ldots
\]


\noindent
\emph{Proof.}
Let us put
\[
\psi(x.w)= 1+ \sum_{n=1}^\infty\, [P_n(x)-P_{n-1}(x)]\cdot w^n
\]
By Theorem B.4 this is equal to
\[
\frac{1-w}{\sqrt{1-2xw+w^2}}\tag{*}
\]
With $x>1$ we set $x=1+\xi$ and notice that
\[
1-2xw+w^2=(1-w)^2-2\xi w
\]
Hence (*) becomes
\[
\frac{1}{\sqrt{1-\frac{2\xi w}{1-w^2}}}=\sum\, g_n\cdot \frac
{(2\xi w)^n} {(1-w^2)^n}=\sum\, g_n\cdot (2\xi)^n\cdot
\frac{w^n}{(1-w^2)^n}\tag{**}
\]
Next, for each $n\geq 1$ we notice that the power series
of 
$\frac{w^n}{(1-w^2)^n}$ has positive coefficients. Since
$g_n(2\xi)^n>0$ also hold we conclude that
(**) is of the form
\[
1+ \sum_{n=1}^\infty\, q_n(\xi)\cdot w^n\quad\text{where}\quad q_n(\xi)>0
\]
Finally, Theorem B.9 follows since
\[ 
P_n(1+\xi)-P_{n-1}(1+\xi)=q_n(\xi)
\]



\centerline {\bf {B.10 An $L^2$-inequality.}}
\medskip

\noindent
Let $n\geq 1$ and denote by $\mathcal P_n[1]$ the space of
real-valued polynomials $Q(x)$ of degree $\leq n$ for which
$\int_{-1}^1\, Q(x)^2\cdot dx=1$ and set
\[ 
\rho(n)=\max_{Q\in\mathcal P-n[1]}\,
|Q|_\infty
\] 
where $|Q|_\infty$ is the maximum norm over
$[-1,1]$.
To find  $\rho(n)$ we use the orthonormal basis $\{P_k^*\}$
and write
\[
Q(x)=t_0\cdot P_0^*(x)+\ldots+t_n\cdot P_n^*(x)
\]
Since $Q\in\mathcal P_n[1]$ we have
$t_0^2+\ldots+t_n^2=1$. Recall also that
\[
P_\nu^*(x)=\sqrt{\frac{2\nu+1}{2}}\cdot P_\nu(x)
\]
Given $-1\leq x_0\leq 1$ the Cauchy-Schwarz inequality gives
\[
Q(x_0)^2\leq \sum_{\nu=0}^{\nu=n}\,
\frac{2\nu+1}{2}\cdot |P_\nu(x_0)|\leq 
\sum_{\nu=0}^{\nu=n}\,\frac{2\nu+1}{2}
\]
where the last inequality follows since the maximum norm of
each $P_\nu$ is $\leq 1$.
Finally, we notice that
\[
\sum_{\nu=0}^{\nu=n}\,\frac{2\nu+1}{2}
=\frac{(1-n)^2}{2}
\]
We conclude that
\[
|Q(x_0)|\leq \frac{n+1}{\sqrt{2}}
\]
\medskip

\noindent
{\bf B.11 The case of equality.}
To have equality above we take  $x_0=1$ and 
\[
 t_\nu=\alpha\cdot P^*_\nu(1)\quad\colon\quad \nu\geq 0
\]
\bigskip

























\centerline{\bf {C. The space $\mathcal T_n$}}
\bigskip

\noindent
Let $n\geq 1$ be a positive integer.
A real-valued trigonometric polynomial of degree
$\leq n$ is given by
\[
g(\theta)=a_0+
a_1\text{cos}\,\theta+\ldots
+
a_n\text{cos}\,n\theta+
b_1\text{sin}\,\theta+\ldots
b_n\text{sin}\,n\theta
\]
Here $a_0,\ldots,a_n,b_1,\ldots,b_n$ are real numbers.
The space of such functions is denoted by
$\mathcal T_n$ which  is a vector space over
${\bf{R}}$ of dimension $2n+1$.
We can write
\[
\text{cos}\,kx=\frac{1}{2}[e^{ikx}+e^{-ikx}]\quad\text{and}\quad
\text{sin}\,kx=\frac{1}{2i}[e^{ikx}-e^{-ikx}]\quad\colon\, k\geq 1
\]
It follows that there exist complex numbers
$c_0,\ldots,c_{2n}$ such that
\[
g(\theta)=e^{-in\theta}\cdot[c_0+c_1e^{i\theta}+\ldots+c_{2n}e^{i2n\theta}]
\]
\medskip

\noindent
{\bf Exercise.}
Show that
\[
c_\nu+c_{2n-\nu}=2a_\nu\quad \text{and}\quad c_{\nu}-c_{2n-\nu}=2 b_\nu\implies
\]
\[
c_{2n-\nu}=\bar c_\nu\quad 0\leq\nu\leq n
\]


\noindent
{\bf C.1 The polynomial $G(z)$.}
Given $g(\theta)$ as above we set
\[ 
G(z)= c_0+c_1z+\ldots+c_{2n}z^{2n}
\]
Then we see that
\[ 
e^{-in\theta}\cdot G(e^{i\theta})= g(\theta)
\]


\noindent
{\bf C.2 Exercise.}  Set
\[ 
\bar G(z)= \bar c_0+c-1z+\ldots+\bar c_{2n}z^{2n}
\]
and show that
\[ z^{2n}G(1/z)= \bar G(z)\tag{*}
\]
Use this to show that if
$0\neq z_0$ is a zero of $G(z)$ then
$\frac{1}{\bar z_0}$ is also a zero of $G(z)$.
\medskip

\noindent
{\bf{C.3 The case when $g\geq 0$}}. Assume that the $g$-function is non-negative.
Let
\[ 
0\leq \theta_1<\ldots<\theta_\mu<2\pi
\]
be the zeros on the half-open interval $[0,2\pi)$.
Since $g\geq 0$ every such zero has a multiplicity given by
an \emph{even} integer.
Consider also the polynomial $G(z)$. From Exercise C.2    it follows that
$\{e^{i\theta_\nu}\}$ are complex zeros of $G(z)$
with multiplicities given by even integers.
Next, if $\zeta$ is a zero where
$\zeta\neq 0$ and $|\zeta|\neq 1$, then (*) in C.2
implies that
$\frac{1}{\bar\zeta}$ also is  a zero and hence
$G(z)$ has a factorisation
\[
G(z)= c_{2n}
\cdot \prod_{\nu=1}^{\nu=\mu}\,
(z-e^{i\theta_\nu})^{2k_\nu}\cdot \prod_{j=1}^{j=m}\,
(z-\zeta_j)(z-\frac{1}{\bar\zeta_j})\cdot z^{2r}
\quad\text{where}\quad 2\mu+2m+2r=2n
\]
Here $0<|\zeta_j|<1$ hold for each $j$ and it may occur that
multiple zeros appear,  i.e. the
$\zeta$-roots need not be distinct and  the integer $r$ may be zero
or positive.
\medskip

\noindent
{\bf {C.4 The $h$-polynomial}}. Let
$\delta=\sqrt{|\zeta_1|\cdots|\zeta_m|}$ and put
\[
h(z)=c_{2n}\dot \delta\cdot \prod_{\nu=1}^{\nu=\mu}\,
(z-e^{i\theta_\nu})^{k_\nu}\cdot \prod_{j=1}^{j=m}\,
(z-\zeta_j)\cdot z^{r}
\]


\noindent
{\bf C.5 Proposition.} \emph{One has the equality}
\[
|h(e^{i\theta})|^2=g(\theta)
\]


\noindent
\emph{Proof.}
With $z=e^{i\theta}$ and $0<|\zeta|<1$ one has
\[
(e^{i\theta}-\zeta)
(e^{i\theta}-\frac{1}{\bar \zeta})=
(e^{i\theta}-\zeta)\cdot (\bar\zeta -e^{-i\theta})\cdot
e^{i\theta}\cdot\frac{1}{\bar \zeta}
\]
Passing to absolute values it follows that
\[
\bigl|(e^{i\theta}-\zeta)
(e^{i\theta}-\frac{1}{\bar \zeta})\bigr|=
\bigl|e^{i\theta}-\zeta\bigr|^2\cdot 
\frac{1}{|\zeta|}
\]
Apply this to every root $\zeta_\nu$ and take the 
product which
gives 
Proposition C.5.
\bigskip

\noindent
{\bf C.6 Application.}
Let $g\geq 0$ be as above 
and assume that the constant coefficient $a_0=1$.
This means that
\[
1=\frac{1}{2\pi}\cdot 
\int_0^{2\pi}\, g(\theta)\cdot d\theta
\]
With
$h(z)=d_0+d_1z+\ldots+d_nz^n$
we get
\[
1 =\frac{1}{2\pi}\cdot 
\int_0^{2\pi}\,h(e^{i\theta})|^2\cdot d\theta
=|d_0|^2+\ldots+|d_n|^2
\]
Notice that
\[ |d_n|^2=|c_{2n}|\cdot\delta\quad\text{and}\quad
|d_0|^2= |c_{2n}\cdot \delta|\cdot\prod\,|\zeta_j|^2=|c_{2n}|\cdot \frac{1}{\delta}\tag{i}
\]
From this we see that
\[
|c_{2n}|\cdot(\delta+\frac{1}{\delta})= |d_0|^2+d_n|^2\leq 1\tag{iii}
\]
Here $0<\delta<1$ and therefore
$\delta+\frac{1}{\delta}\geq 2$ which together with
(iii) gives
\[
|c_{2n}|\leq \frac{1}{2}
\]
At the same time we recall  that
\[
c_{2n}=\frac{a_n+ib_n}{2}
\]
Hence we have proved the inequality:
\[
|a_n+ib_n|\leq 1\tag{*}
\]
\medskip

\noindent
\emph{Summing up} we have proved the following:
\medskip

\noindent
{\bf {C.7 Theorem.}}
\emph{Let $g(\theta)$
be non-negative in $\mathcal T_n$ with constant term
$a_0=1$. Then}
\[
|a_n+ib_n|\leq 1
\]


\noindent
{\bf {C.8 An application}}.
Let $n\geq 1$ and consider the space of all monic polynomials 
\[
P(x)=x^n+c_{n-1}x^{n-1}+\ldots+c_0
\]
where $\{c_\nu\}$ are real-
To each such polynomial we can consider the maximum norm
over the interval $[-1,1]$.
Then one has
\medskip

\noindent
{\bf {C.9 Theorem.}}
\emph{For each $P\in\mathcal P_n^*$ one has the inequality}
\[
\max_{-1\leq x\leq 1}\, |P(x)|\geq 2^{-n+1}
 \]
\medskip

\noindent
\emph{Proof}. 
Consider some $P\in\mathcal P_n^*$ and
define the trigonometric polynomial
\[ 
g(\theta)= (\text{cos}\,\theta)^n
+c_{n-1}(\text{cos}\,\theta)^{n-1}+\ldots+c_0
\]
So here
$P(\text{cos}\,\theta)= g(\theta)$ and 
Theorem C.9 follows if we have proved that
\[
2^{-n+1}\geq 
\max_{0\leq \theta\leq 2\pi}\, |g(\theta))|\tag{1}
\]
To prove this we set
$M=\max_{0\leq \theta\leq 2\pi}\, |g(\theta))|$.
Next, we can write
\[ 
g(\theta)= a_0+a_1\text{cos}\,\theta\ldots+
a_n\text{cos}\,n\theta
\]
Moreover, since
\[
(\text{cos}\,\theta)^n=2^{-n}\cdot[e^{i\theta}+e^{-\theta}]^n
\]
we get
\[ 
a_n=2^n
\]
Now we shall apply Theorem C.8. For this purpose we construct
non-negative trigonometric polynomials. First we define
\[
g^*(\theta)= \frac{M-g(\theta)}{M-a_0}
\]
Then $g^*\geq 0$ and its constant term is 1.
We have also
\[ 
g^*(\theta)= 1-\frac{1}{M-a_0}\cdot \sum_{\nu=1}^{\nu=n}a_\nu\text{cos}\,\nu\theta
\]
Hence Theorem C.7 gives
\[
\frac{1}{|M-a_0|}\cdot |a_n|\leq 1\implies
|M-a_0|\geq 2^{-n+1}\tag{1}
\]
Next, we have also the function
\[
g_*(\theta)= \frac{M+g(\theta)}{M+a_0}
\]
In the same way as above we obtain:
\[ 
|M+a_0|\geq 2^{-n+1}\tag{2}
\]
Finally, (1) and (2) give
\[
M\geq 2^{-n+1}
\]
which proves Theorem C.9
\bigskip

\centerline{\bf{D. Tchebysheff polynomials.}}
\medskip

\noindent
The inequality in Theorem C.9 is sharp. To see this
we shall construct a special polynomial $T_n(x)$ of degree $n$.
Namely, with $n\geq 1$ we can write
\[
\text{cos}\,n\theta=
2^{n-1}\cdot 
(\text{cos}\,\theta)^n+
c_{n-1}\cdot 
(\text{cos}\,\theta)^{n-1}+
\ldots+c_0
\]
Set
\[ T_n(x) =2^{n-1}x^n+
c_{n-1}\cdot x^{n-1}+\ldots+c_0
\]
Hence
\[ 
T_n(\text{cos}\,\theta)= \text{cos}\, n\theta
\]
We conclude that the polynomial
\[ 
p_n(x)=2^{-n+1}\cdot T_n(x)
\]
belongs to $\mathcal P_n^*$ and its maximum norm
is $2^{-n+1}$. This proves that the inequality in Theorem 10 is sharp.
\medskip

\noindent
{\bf D.1 Zeros of $T_n$}.
Set
\[ 
\theta_\nu=\frac{\nu\pi}{n}+\frac{\pi}{2n}
\]
It is clear that
$\theta_1,\dots,\theta_n$ are zeros of
$\text{cos}\, n\theta$.
Hence the  zeros of $T_n(x)$ are:
\[
x_\nu= \text{cos}\, \theta_\nu
\]
Notice that
\[
-1<x_n<\ldots<x_1<1
\]
Since $T_n(x)$ is a polynomial of degree
$n$ it follows that $\{x_\nu\}$ give all zeros and we have
\[
T_n(x)=2^{n-1}\cdot \prod\,(x-x_\nu)
\]


\noindent
{\bf{D.2 Exercise}}.
Show that
\[ 
T'_n(x_\nu)\cdot\sqrt{1-x_\nu^2}=n
\] 
hold for every zero of $T_n(x)$.

\medskip

\noindent
{\bf{D.3 An interpolation formula.}}
Since $x_1,\ldots,x_n$ are distinct it follows 
that if $p(x)\in\mathcal P_{n-1}$ is a polynomial of degree
$\leq n-1$ then
\[
 p(x)=
\sum_{\nu01}^{\nu=n}\, p(x_\nu)\cdot
\frac{1}{T'(x_\nu)}\cdot \frac{T(x)}{x-x_\nu}
\]
By the exercise above we get
\[
 p(x)=
\frac{1}{n}\cdot \sum_{\nu=1}^{\nu=n}\, (-1)^{\nu-1}p(x_\nu)\cdot
\sqrt{1-x_\nu^2}\cdot \frac{T(x)}{x-x_\nu}
\]
\medskip
\noindent
We shall use the interpolation formula above to prove
\medskip

\noindent
{\bf {D.4 Theorem}}
\emph{Let $p(x)\in\mathcal P_{n-1}$ satisfy}
\[
\max_{-1\leq x\leq 1}\,
\sqrt{1-x^2}\cdot |p(x)|\leq 1\tag{1}
\]
\emph{Then it follows that}
\[
\max_{-1\leq x\leq 1}\,
|p(x)|\leq n\tag{2}
\]
\medskip

\noindent
\emph{Proof.}
First, consider the case when 
\[
-\text{cos}\frac{\pi}{2n}\leq
x\leq \text{cos}\,\frac{\pi}{2n}\tag{*}
\]
Then we have
\[
\sqrt{1-x^2}\geq 
\sqrt{1-\text{cos}^2\frac{\pi}{2n}}=
\text{sin}\,\frac{\pi}{2n}
\]
Next,  recall the inequality
$\text{sin}\, x\geq \frac{2}{\pi}\cdot x$.
It follows that
\[
\sqrt{1-x^2}\geq \frac{1}{n}
\]
So when (1) holds in the theorem we have
\[ 
|p(x)|=\frac{1}{\sqrt{1-x^2}}\cdot
\sqrt{1-x^2}\cdot |p(x)|\leq
\frac{1}{\sqrt{1-x^2}}\leq n
\]
Hence the required inequality in Theorem D.4  holds when
$x$ satisfies (*) above.
Next, suppose that
\[
x_1\leq x\leq 1\tag{**}
\]
On this interval $T_n(x)\geq 0$ and from the interpolation formula
xx and the triangle  inequality we have
\[
|p(x)\leq\frac{1}{n}
\sum_{\nu=1}^{\nu=n}\,
\sqrt{1-x_\nu^2}\cdot |p(x_\nu)|\cdot
\frac{T(x)}{x-x_\nu}\leq\frac{1}{n}
\sum_{\nu=1}^{nu=n}\,
\frac{T(x)}{x-x_\nu}
\]
Next, the sum
\[
\frac{T(x)}{x-x_\nu}=T'_n(x)=n\cdot U_{n-1}(x)
\]
So when (**) holds we have
\[
|p(x)|\leq |U_{n-1}(x)|\tag{***}
\]
By xx the maximum normmof $U_{n-1}$ over $[-1,1]$ is $n$ and hence
(***) gives
\[
 |p(x)|\leq n
\]
In the same way one proves htat
\[
-1\leq x\leq x_n\implies 
 |p(x)|\leq n
\] 
Together with the upper bound in the case (xx) we get Theorem D.4.

\bigskip

\centerline{\bf{D.5 Berstein's inequality.}}
\medskip

\noindent 
Let $g(\theta)\in\mathcal T_n$.
The derivative $g'(\theta)$ is another trigonometric polynomial and we have
\medskip

\noindent
{\bf Theorem.} \emph{For each $g\in\mathcal T_n$ one has}
\[
\max_{0\leq \theta\leq 2\pi}\,
|g'(\theta)|\leq n\cdot 
\max_{0\leq \theta\leq 2\pi}\,|g(\theta)|
\]
\medskip
\noindent
Before we prove this result
we establish an inequality for certain trigonometric polynomials.
Namely, consider a real-valued sine-polynomial
\[
S(\theta)= 
c_1\text{sin}(\theta)+ \ldots+
c_n\text{sin}(n\theta)
\]
Now $\theta\mapsto \frac{ S(\theta)}{\text{sin}\,\theta}$
is an even function of $\theta$ and therefore
one has
\[
\frac{ S(\theta)}{\text{sin}\,\theta}= 
a_0+a_1\text{cos}\,\theta+
\ldots+a_{n-1}(\text{cos}\,\theta)^{-n-1}
\]
Consider the polynomial
\[
p(x)= a_0+a_1x+\ldots+a_{n-1}x^{n-1}
\]
Then e see that:
\[
|p(\text{cos}\,\theta)|=
\frac{|S(\theta)|}{\sqrt{1-\text{cos}^2\,\theta}}
\]
Using this we apply Theorem D.4 to the polynomial $p(x)$ and conclude
\medskip

\noindent
{\bf{D.6 Theorem.}}
\emph{Let $S(\theta)= 
c_1\text{sin}(\theta)+ 
c_n\text{sin}(n\theta)$ be a sine-polynomial as above.
Then}
\[
\max_{0\leq\theta\leq 2\pi}\,
\frac{|S(\theta)|}{\text{sin}\,\theta}\leq n\cdot
\max_{0\leq\theta\leq 2\pi}\,
|S(\theta)|
\]
\bigskip

\noindent
{\bf D.7 Proof of Bernstein's theorem.}
Fix an arbitrary $0\leq\theta-0<2\pi$.
Set
\[
S(\theta)=g(\theta_0+\theta)-g(\theta_0-\theta)
\]
We notice that $S(\theta$is an odd polynomial of
$\theta$ and
$S(0)=0$, It follows that
$S(\theta)$ is a sine-polynomial as above of degree
$\leq n$. Notice also that
\[
\max_{0\leq\theta\leq 2\pi}\,|S(\theta)|\leq 2\cdot
\max_{0\leq\theta\leq 2\pi}\,|g(\theta)|
\max_{0\leq\theta\leq 2\pi}\,|g(\theta)|
\]
Theorem D.6 applied to $S(\theta)$ gives
\[
\bigl|\frac{g(\theta_0+\theta)-g(\theta_0-\theta)}{\text{sin}\,\theta}
\bigr|\leq 2n\cdot 
\max_{0\leq\theta\leq 2\pi}\,|g(\theta)|\tag{i}
\]
Next, in the left hand side we can take the limit as $\theta|\to 0$ and notice that
\[
2\cdot  g'(\theta_0)=
\lim_{\theta\to 0}\, 
\frac{g(\theta_0+\theta)-g(\theta_0-\theta)}{\text{sin}\,\theta}
\]
Hence (i) gives
\[
|g'(\theta_0)|\leq n\cdot 
\max_{0\leq\theta\leq 2\pi}\,|g(\theta)|
\]
Finally, since $\theta_0$ was arbitrary we get Bernstein's theorem.











\newpage

\centerline{\bf{Almost periodic functions.}}
\bigskip

\noindent
The theory about almost periodic functions on the real line was created and developed by
Harald  Bohr.
The original proofs were rather cumbersome and 
alternative methods and simplifications of proofs
which also led to more precise approximations
were found by Weyl and Bochner.
One should also mention contributions by
Besivcovich  whose articles
[Bes:1+2]  completed Bohr's
initial studies  of almost periodicity  for bounded complex analytic functions
in strip domains. The interested reader 
can consult Bohr's plenary talk at the IMU\vvv congress
1950 for a survey about the development of the theory about
almost periodic functions where contributions due to 
xxxx in [Bel ??] are pointed out since his work
brings the theory about  almost periodic functions
closer to problems related to analytic number theory.
In addition to this we refer to work by
J. Favard about almost periodic harmonic functions
and for applications to the study of
differential and difference equations
an overall reference is the collected work by Bochner and 
articles by
Neugebauer.
\medskip

\noindent
We shall
restrict the study to almost periodic
 functions on the real line but
remark that the theory
extends to a a general set\vvv up where
one starts from an arbitrary discrete abelian group $G$
which yields a compact dual group
$\widehat G$ whose elements are maps $\chi$ from $G$ into
the unit  circle satisfying
\[ 
\chi(g\uuu 1+g\uuu 2)= \chi(g\uuu 1)\cdot \chi(g\uuu 2)\quad
\text{for all pairs}\quad  g\uuu 1,g\uuu 2\in G\tag{*}
\]
One refers to such maps as characters 
and when $G$ is equipped with the discrete topology then 
(*) is the sole assumption, i.e. no continuity property is involved.
Keeping $g\in G$ fixed we get  the   exponential
function on
$\widehat G$ defined by
\[
E\uuu g(\chi)= \chi(g)\quad \colon\quad \chi\in \widehat G
\]
Finite linear combinations of such $E$\vvv functions form a linear space and
using the maximum norm for functions on $\widehat G$
the  closure  yields a Banach space
of functions on $\widehat G$  denoted by
$\mathcal F(\widehat G)$.
One equips $\widehat G$ with the weakest topology such that
every function
in $\mathcal F(\widehat G)$ is continuous. 
Tychonof's Theorem from general topology shows that 
$\widehat G$ becomes a compact topological space 
equipped with the structure of an abelian group since
the product of two $\chi$\vvv functions on $G$ again is a character function.
We shall not enter a detailed discussion about this general construction which leads to
harmonic analysis on  locally compact abelian groups.
The interested reader should consult the
excellent text\vvv book by Rudin in [Rudin] 
devoted to   Fourier analysis on locally compact
abelian groups.
\medskip

\noindent
Now we turn to the special study on the real line
which means that $G$ as a group is the real $x$\vvv line line equipped with
the discrete topology. If $\xi$ is a real number we get the character defined by
\[ 
\chi\uuu \xi(x)= e^{ix\xi}
\]
In this way the dual group $\widehat {\bf{R}}\uuu{\text{dis}}$
contains a copy of the real $\xi$\vvv line.
But the whole compact dual group contains more characters
whose existence rely upon the axiom of choice.
So apart from the evident characters
$\{\chi\uuu \xi\}$ which identify the real $\xi$\vvv line with a subset
of $\widehat {\bf{R}}\uuu{\text{dis}}$, there exist 
more characters which no longer are
constructible in an elementary fashion.
One refers to $\widehat {\bf{R}}\uuu{\text{dis}}$
as the Bohr group over the real line. The induced topology on the embedded
$\xi$\vvv line is much weaker  than the
ordinary topology. For example, a fundamental system
of relatively open neighborhoods of $x=0$ consists of sets of the form
\[
U\uuu\epsilon(\xi\uuu1,\ldots,\xi\uuu m)=
\bigcap\uuu{\nu=1}^{\nu=m} \,\{e^{i\xi\uuu \nu\cdot x}\vvv 1|<\epsilon\}
\]
where $\epsilon>0$ and $\xi\uuu 1,\ldots,\xi\uuu m$ is some 
$m$\vvv tuple of real numbers.
Notice that every such set is unbounded.

\medskip

\noindent
In the sequel we shall not discuss the Bohr group  since
the constructions and results will be  expressed via ordinary calculus,
and
the subsequent proofs    have the merit that 
no appeal  to the Axiom of Choice is needed.



\medskip

\noindent
Now we present the major results in Bohr's theory.
Denote by
$C\uuu *({\bf{R}})$ the set of bounded and uniformly continuous functions
$f(x)$ on the real $x$\vvv line. Recall that
uniform continuity means that the non\vvv increasing function
defined by
\[
\omega\uuu f(\delta)= \max\uuu {0<s\leq \delta}\,[
\max\uuu x\,|f(x+\delta\vvv f(x)|\,]
\] 
tends to zero as $\delta\to 0$.
Following Bohr we  give:

\medskip





\noindent
{\bf{0.1 Definition.}} \emph{A function $f(x)$ is almost periodic if 
it belongs to $C\uuu *({\bf{R}})$
and to each $\epsilon>0$ there exists some $\ell >0$
such that
every interval $(a,a+\ell)$ contains a point $\tau\uuu a$ where
the maximum norm}
\[
\max\uuu x\, |f(x+\tau\uuu a)\vvv f(x)|<\epsilon
\]
\emph{The class of these functions is denoted by $\mathcal{AP}$.}
\bigskip


\noindent
{\bf{Remark. }} Apart from the condition that $\lim \omega\uuu f(\delta)=0$
as $\delta\to 0$, almost periodicity means that
translates of $f$ satisfy an addition condition.
To each $\epsilon>0$ we put
\[ 
E\uuu f(\epsilon)=
\{\tau\,\,\colon\,\,
\max\uuu x\, |f(x+\tau)\vvv f(x)|\leq \epsilon\,\}
\]
A point $\tau$ in this set is called  a translation number of size $\leq \epsilon$ with
respect to $f$.
The continuity of $f$ shows that
every  $E\uuu f(\epsilon)$ is a closed set and 
almost periodicity means that
for every $\epsilon>0$ there exists  $\ell\uuu f(\epsilon)>0$
such that
every open interval of length
$\ell\uuu f(\epsilon)$ contains at least one point from
$E\uuu f(\epsilon)$. One  says that
$E\uuu f(\epsilon)$ is a relatively dense subset of
${\bf{R}}$.
Notice that the function $\epsilon\mapsto \ell\uuu f(\epsilon)$ is
non\vvv decreasing.


\medskip

\noindent
{\bf{Exercise.}}
Show that the sum and the product of two almost periodic functions
is almost periodic.
Show also that if $\{f\uuu n\}$ is a sequence in $\mathcal{AP}$
which converges uniformly to a limit function $f$
then $f$ is almost periodic.
Hence $\mathcal {AP}$ appears a a closed subalgebra of
$C\uuu *({\bf{R}})$.
\medskip


\noindent
{\bf{Exponential polynomials.}}
 If $\tau$ is a real number it is clear that
the function $e^{i\tau x}$ is almost periodic, and 
we consider the  algebra of functions of the form

\[
p(x)= \sum\, c\uuu k\cdot e^{i\tau\uuu k\cdot x}
\]
where $\{c\uuu k\}$ is a finite set of complex numbers
and $\{\tau\uuu k\}$ a finite set of real numbers.
Denote this algebra by   $\mathcal {TP}$.
A first major result in Bohr's theory is
the following density result:

\medskip

\noindent
{\bf {0.2 Theorem.}}
\emph{$\mathcal {AP}$ is a closed subspace of
$C\uuu *({\bf{R}})$ where
$\mathcal {TP}$ appears as a dense
subspace.}


\bigskip

\noindent
The non\vvv trivial part  is the density
which is  proved in XXX.
Now we  introduce 
the spectrum of an almost periodic function
whose construction relies upon the following:


\medskip

\noindent
{\bf{0.3 Proposition.}} \emph{Let $f\in \mathcal{AP}$. Then there exists a limit}
\[
M\uuu f(\lambda)=\lim\uuu {T\to\infty}\, \frac{1}{T}\cdot \int\uuu 0^T\,
e^{\vvv i\lambda x}\cdot f(x)\cdot dx\quad\text{for every real number}\quad  \lambda.
\]
\medskip

\noindent
\emph{Proof.}
Since $e^{\vvv i\lambda x}\cdot f(x)$ are almost periodic for every
$\lambda$ it suffices to prove the result when $\lambda=0$.
Let  $\epsilon>0$ and pick some $\tau>0$ in the set
$E\uuu f(\epsilon)$. If $T$ is a large number we find
the positive integer $N$ such that
$N\tau\leq T<(N+1)\tau$.
Now
\[ 
\int\uuu 0^T\,  f\cdot dx=
\sum\uuu{k=0}^{k=N\vvv 1}
\int\uuu {k\tau}^{(k+1)\tau}\, f\cdot dx+
\int\uuu {N\tau}^T\, f\cdot dx\tag{i}
\]
For each $k$ we have
\[
\int\uuu {k\tau}^{(k+1)\tau}\, f\cdot dx=
\int\uuu 0^\tau\, f(x+k\tau)\cdot dx
\] 
At the same time
\[
\bigl|\int\uuu 0^\tau\, f(x+k\tau)\cdot dx
\vvv\int\uuu 0^\tau\, f(x)\cdot dx\,\bigr|\leq \epsilon\cdot \tau
\]
The triangle inequality
therefore gives
\[
\bigl|\, 
\frac{1}{T}\int\uuu 0^T\,  f\cdot dx
\vvv \frac{N}{T}\cdot \int \uuu 0^\tau\, f(x)\cdot dx\,
\bigr|
\leq \epsilon\cdot \frac{N\cdot \tau}{T}+
\frac{1}{T}\cdot \bigl|\int\uuu {N\tau}^T\, f\cdot dx\,\bigr|
\]


\medskip

\noindent
{\bf{Bessel's inequality.}}
It turns out that $M\uuu f(\lambda)\neq 0$.
holds in set which is at most
denumerable.
To prove  this we consider the product $f\cdot \bar f=|f|^2$ which again is
almost periodic  and this gives the existence of the  number:
\[
||f||^2=\lim\uuu {T\to\infty}\, \frac{1}{T}\cdot \int\uuu 0^T\,
 |f(x)|^2\cdot dx\tag{*}
\]
We refer to $||f||^2$ as the squared mean of $f$.

\medskip

\noindent
{\bf{0.4 Proposition.}} \emph{For every 
finite set $\{\lambda\uuu ,\ldots,\lambda\uuu m\}$
one has the inequality}
\[
\sum\uuu {\nu=1}^{\nu=m}\,
 |M\uuu f(\lambda\uuu  k) |^2\leq ||f||^2
\]
\medskip

\noindent
\emph{Proof.}} Put $a\uuu k=M\uuu f(\lambda\uuu  k)$. Now
\[
|f(x)|^2\vvv \sum\, a\uuu k\cdot e^{i\lambda\uuu k x}|^2=
|f(x)|^2+\sum\,|a\uuu k|^2\vvv
\sum\, \bar a\uuu k\cdot f\dot e^{\vvv i\lambda\uuu k x}\vvv
\sum\, a\uuu k\cdot \bar f\dot e^{i\lambda\uuu k x}
\]
\medskip


\noindent
Passing to mean values over $[0,T]$ while $T\to \infty$
the reader may verify that
\[
||f||^2=\sum\,|a\uuu k|^2+
\lim\uuu {T\to\infty}\, \frac{1}{T}\cdot \int\uuu 0^T\,
|f(x)|^2\vvv \sum\, a\uuu k\cdot e^{i\lambda\uuu k x}|^2\cdot dt
\]
which  gives Bessel's inequality.

\medskip

\noindent
{\bf{0.5 Bohr's spectral set
$\sigma(f)$.}} Bessel's inequality shows that
$M\uuu f(\lambda)\neq 0$  for at most
a denumerable set. The set of all such $\lambda$ is 
called Bohr's spectrum of $f$ and is
denoted by
$\sigma(f)$.
Every denumerable set
$\{\lambda\uuu k\}$ can appear as a Bohr spectrum.
For consider some $\ell^1$\vvv sequence of non\vvv zero complex numbers
$\{c\uuu k\}$, i.e. $\sum\,|c\uuu k|<\infty$.
We get the almost periodic function
\[ 
\phi(x)= \sum\, c\uuu k\cdot e^{i\lambda\uuu k x}
\]
\medskip

\noindent
{\bf{Exercise.}}
Show that $\sigma(\phi)=\{\lambda\uuu k\}$.
The hint is that
\[ \
\lim\uuu{T\to \infty}\, \frac{1}{T}\int\uuu 0^T\, e^{i\alpha x}\cdot dx=0
\quad\text{hold for every}\quad\alpha\neq 0
\] 




\medskip

\noindent
{\bf{0.7 Parseval's equality.}}
\emph{For every $f\in \mathcal {AP}$ one has}
\[
||f||^2= \sum\uuu {\lambda\uuu k\in \sigma(f)}\, |M\uuu f(\lambda\uuu k)|^2
\]
\medskip

\noindent
The proof will be given in XX and relies upon the following result 
which is proved in XX below.

\medskip

\noindent{\bf{0.8 Theorem}}
\emph{The Bohr spectrum is non\vvv  empty for every $f\in\mathcal {AP}$
which is not identically zero.}


\bigskip

\noindent{\bf{0.9 Bochner\vvv Fejer kernels.}}
Parseval's equality
leads not only to a proof of Theorem 0.2 but
gives  a   procedure to approximate every
$f$ in $\mathcal{AP}$.
Namely, let $\{\lambda\uuu k\}=\sigma(f)$
and if $\epsilon>0$ we choose $m$ so large that
\[
\sum\uuu{k=1}^{k=m} \, |M\uuu f(\lambda\uuu k)|^2>||f||^2\vvv \epsilon
\]
The finite set $\lambda\uuu 1,\ldots,\lambda\uuu m$
generates a free abelian group where we choose some basis $\beta\uuu 1,\ldots
\beta\uuu p$.
So each $\lambda\uuu k$ is an
integer combination of the $\beta$\vvv numbers.
For each $p$\vvv tuple of integers $w\uuu 1,\ldots,w\uuu p$
we assign the exponential function
\[
E\uuu{w\uuu1,\ldots,w\uuu p}(x)=e^{i(\beta\uuu 1\cdot w\uuu k1+\ldots+
\beta\uuu k\cdot w\uuu k)\cdot x}
\]

\noindent
Next, for each positive integer $N$ we set

\[ 
\mathcal B\uuu N(x)=
\sum\, \prod\,\bigl(1\vvv \frac{|w\uuu k|}{N})\cdot M\uuu f(
w\uuu 1\beta\uuu 1+
\ldots+w\uuu p\beta\uuu p)\cdot 
E\uuu{w\uuu1,\ldots,w\uuu p}(x)
\]
where $\Sigma$ extends over all $w$\vvv tuples for which
\[ 
\vvv N\leq w\uuu k\leq N\quad\colon\quad 1\leq k\leq p
\]

\noindent
We refer to $\mathcal B\uuu N$ as a Bochner\vvv Fejer sum
associated
to $f$.
Each $\mathcal B\uuu N$ belongs to
$\mathcal {TP}$ and 
its finite spectrum is confined to numbers
of the form
$w\uuu 1\beta\uuu 1+
\ldots+w\uuu p\beta\uuu p$ where  $M\uuu f\neq 0$.
In particular $\sigma(\mathcal B\uuu N)$ is a  subset of
$\sigma(f)$.
In XX we  prove that these  Bochner\vvv Fejer kernels
can be used to approximate $f$ uniformly
by functions in $\mathcal{TP}$.
A special case occurs when
the abelian group generated
by the spectrum of $f$  is finitely generated. Then
we construct $\{\mathcal B\uuu N\}$ as above starting from
a finite set $w\uuu 1,\ldots,w\uuu k$ such that 
every point in $\sigma(f)$ is an integer combination of
this $m$\vvv tuple. In XXX we prove that
\[ 
\lim\uuu{N\to \infty}\, \max\uuu x\,|\mathcal B\uuu N(x)\vvv f(x)|=0\tag{*}
\]

\noindent
Notice   that this result is similar to
the uniform approximation by Fejer sums
of continuous and periodic functions on $[0,2\pi]$.
\medskip

\noindent
{\bf{0.10 Diophantic considerations.}}
Let $\{f\uuu k\}$ be a sequence of periodic functions, where
the periodic of $f\uuu k$ is some number $\tau\uuu k$.
Suppose that this sequence converges uniformly to an almost 
periodic function $f$ with a spectrum $\sigma(f)$.

\medskip

\noindent{\bf{0.11 Theorem.}} \emph{For each non\vvv zero
$\lambda\in\sigma(f)$ 
it follows
that $\tau\uuu k$ is a rational multiple of $\lambda$ for all sufficiently
large
$k$.
As a consequence every pair of points
$\lambda\uuu 1,\lambda\uuu 2$ in $\sigma(f)$ must be
$Q$\vvv linearly dependent.}
\bigskip

\noindent
{\bf{0.12 Remark.}}
Theorem 0.11 shows that only a restricted class
in $\mathcal {AP}$\vvv  can
be approximated uniformly
by periodic functions.
Theorem o.10 gives  the  necessary condition  that
the the  numbers in $\sigma(f)$ are rationally dependent.
Conversly, if
there exists some $\lambda\uuu *\neq 0$ such that
every $\lambda\in \sigma(f)$ is a rational multiple of
$\lambda\uuu *$
then
we will show in XX that $f$ can be uniformly approximated by periodic functions.

\bigskip

\noindent
{\bf{0.13 Arithmetical properties of translation numbers.}}
In the remark after definition 0.1 we introduced
the sets $E\uuu f(\epsilon)$.
We can consider the subset of integers, i.e. put
\[
{\bf{Z}}\uuu f(\epsilon)=E\uuu f(\epsilon)\,\cap\, {\bf{Z}}
\]
It turns out hat this set is non\empty and even relatively dense, i.e.
there exists some $\ell>0$ such that
every interval of length $\ell$ contains at least one integer from
${\bf{Z}}\uuu f(\epsilon)$.
Keeping $\epsilon>0$ fixed we
impose an "almost periodic condition" which we begin to describe.
Consider an interval $(a,a+\ell)$ which gives us a finite set of integers
\[
{\bf{Z}}\uuu f(\epsilon)\,\cap (a,a+\ell)\tag{1}
\]
Consider also some
$0<\rho<\epsilon$ which  gives the set
${\bf{Z}}\uuu f(\rho)$.
We say that an
integer $n$ which belongs to (1)
survives up to order  $\rho$ if one has the inclusion
\[
\{n\}+{\bf{Z}}\uuu f(\rho)\subset {\bf{Z}}\uuu f(\epsilon)\tag{2}
\]
\bigskip

\noindent
{\bf{0.14 Definition.}}
\emph{The set ${\bf{Z}}\uuu f(\epsilon)$ is said to be almost periodic
if there for each $\eta>0$ exists a pair $(\rho,\ell\uuu *)$
for which the following hold: Whenever $\ell>\ell\uuu *$ and
$a$ is a real number, it follows that the number of points in
${\bf{Z}}\uuu f(\epsilon)\,\cap (a,a+\ell)$
which survive up to order
$\rho$ is $\geq$ then 
$1\vvv\eta)$ times the number of integers in 
${\bf{Z}}\uuu f(\epsilon)\,\cap (a,a+\ell)$.}
\medskip


\noindent
{\bf{0.15 Theorem.}}
\emph{The set of all $0<\epsilon<1$ for which
${\bf{Z}}\uuu f(\epsilon)$ is almost periodic
has Lebesgue measure one, i.e. almost periodicity 
holds for all $\epsilon$ except for
a possible empty null\vvv set.}
\medskip

\noindent 
The proof is given in XXX below.



\bigskip






\centerline{\bf{Proof of Theorem 0.8}}
\bigskip

\noindent
Let $f$ be almost periodic and assume that $\sigma(f)=\emptyset$. To prove that
$f$ is identically zero the crucial step is the following:

\medskip

\noindent
{\bf{1.1 Lemma.}}
\emph{Assume that $\sigma(f)=\emptyset$.
Then, for each $\epsilon>0$ 
there exists $T\uuu \epsilon$ such that}
\[
T\geq T\uuu\epsilon\implies
\frac{1}{T}\cdot \bigl|\, \int\uuu 0^T\,e^{i\lambda x} f(x)\cdot dx\bigr|\leq \epsilon
\quad\text{for all}\quad \lambda
\]
\medskip

\noindent
We prove this technical result in XX below and first show why
Lemma 1.1 gives Theorem 0.8.
The idea is to employ certain periodic functions.
Consider some $T>0$ and let
$F\uuu T(x)= f(x)$ on $0\leq x\leq T$ and after $F$ is extended to a
$T$\vvv periodic function.
The  function $F\uuu T(x)$ on the bounded interval 
$(0,T)$ has an ordinary Fourier series
\[
F\uuu T(x)= \sum\, A\uuu k(T)\cdot e^{2\pi i k x/T}
\]
where  we have
\[ 
A\uuu k(T)= \frac{1}{T}\int\uuu 0^T\, e^{2\pi i k x/T}\cdot f(x)\cdot dx
\]
for every integer $k$.
Parseval's equality for ordinary Fourier series gives:
\[
\sum\, |A\uuu k(T)|^2=\frac{1}{T}\int\uuu 0^T\, |f(x)|^2\cdot dx\leq |f|\uuu\infty^2\tag{1}
\]
Next, introduce the following pair  of $T$\vvv periodic functions:


\[ 
G\uuu T(x)= \sum\, |A\uuu k(T)|^2\cdot e^{2\pi i k x/T}
\quad\text{and}\quad 
H\uuu T(x)= \sum\, |A\uuu k(T)|^4\cdot e^{2\pi i k x/T}\tag{2}
\]

\medskip

\noindent
By the general formula for periodic functions
in XX we have
\[ 
G\uuu T(x)= \int\uuu 0^T\,H\uuu T(x+t)\cdot \bar H\uuu T(t)\cdot dt
\]

\noindent
At this stage we apply Lemma 1.1 which gives:

\[ 
\lim\uuu{T\to\infty}\, \max\uuu k\, A\uuu T(k)=0\tag{3}
\]
Together with the inequality (1) for the sum of squares of $|A\uuu k(T)|$
it follows that
\[ 
\lim\uuu{T\to \infty}\,\sum\,  |A\uuu T(k)|^4\to 0\tag{4}
\]
This implies that
the $T$\vvv periodic
function $H\uuu T(x)$ tends uniformly  to zero 
on $[0,T]$ and 
(2) entails that
\[
\lim\uuu{T\to \infty}\, G\uuu T(0)=0
\]

\medskip

\noindent
Next,
the equality $F=f$ on $[0,T]$
and the $T$\vvv periodicity of $F$ gives  for 
each $0<x<T$:


\[
G\uuu T(x)= \frac{1}{T\uuu n}\int\uuu 0^{T\vvv x}\,f(x+t)\cdot \bar f(t)\cdot dt
+\int\uuu{T\vvv x}^T\, f(x+t\vvv T)\cdot dt\tag{5}
\]
Since $f$ is almost periodic we can find
an increasing  sequence $\{T\uuu n\}$
where $T\uuu n\to+\infty$ where
\[
\max\uuu x\, |f(x\vvv T\uuu n)\vvv f(x)|\leq \frac{1}{n}\tag{6}
\]
hold for every $n$.
With $0<x<T\uuu n$ we get from (5):

\[ 
G\uuu{T\uuu n}(x)=\frac{1}{T\uuu n}\int\uuu 0^T\, 
f(x+t)\cdot \bar f(t)\cdot dt+
+\frac{1}{T\uuu n}\int\uuu {T\uuu n\vvv x}^{T\uuu n}
\,[f(x+t\vvv T\uuu n)\vvv f(x+t)]
\cdot \bar f(t)\cdot dt
\]
By (6) the absolute value of the last term is majorized by
$\frac{|f|\uuu\infty}{n}$ which gives:
\[
|G\uuu{T\uuu n}(x)\vvv
\frac{1}{T\uuu n}\int\uuu 0^T\,f(x+t)\cdot \bar f(t)\cdot dt
\bigl|\leq \frac{|f|\uuu\infty}{n}
\]
\medskip

\noindent
Apply this  with $x=0$ and a passage to the limit
gives:
\[
\lim\uuu{T\uuu n\to\infty}
\frac{1}{T\uuu n}\int\uuu 0^{T\uuu n}\,f(t)\cdot \bar f(t)\cdot dt=0
\]
\medskip

\noindent
By the observation from (XX) this limit formula implies that
$f$ is identically zero which finishes the proof that Lemma 1.1 gives
of Theorem 0.8

\bigskip

\centerline{\emph{B. Proof of Lemma 1.1.}}
\bigskip

EASY.. via uniform continuity estimates...

\bigskip
\centerline{\bf{C. Proof of Parseval's formula.}}

\bigskip

\noindent
Let $\{\lambda\uuu k\}$ be the Bohr spectrum of $f$.
By Bessel's inequality we know that the series
formed by $\{|M\uuu f(\lambda\uuu k)|^2\}$
converges which gives us the almost periodic function
\[
\phi(x)=\sum\, |M\uuu f(\lambda\uuu k)|^2\cdot e^{\i\lambda\uuu k\cdot x}
\]
At the same time the result in (xx) gives us the almost periodic function
\[
g(x)=\lim\uuu{T\to \infty}\,
\frac{1}{T}\cdot\int\uuu 0^T\, f(x+t)\cdot \bar f(t)\cdot dt
\]
Moreover, $\sigma(g)=\sigma(f)$ and

\[ 
M\uuu g(\lambda\uuu k)=|M\uuu f(\lambda\uuu k)|^2
\] 
hold for each $k$.
The same hold for the Bohr spectrum of $\phi$
and it follows that the Bohr spectrum of the difference $\phi\vvv g$
is empty and hence $\phi=g$ by Theorem 0.8.
In particular we get
\[
\sum\, |M\uuu f(\lambda\uuu k)|^2=\phi(0)=g(0)=
||f||^2
\] 

\noindent
which proves Parseval's equality for $f$.

\bigskip

\centerline
{\bf{D. The Bochner\vvv Fejer approximation.}}
\bigskip

\noindent
Let $f\in\mathcal{A}$ be given and
consider the $\mathcal B$\vvv functions defined as in (xx).
They all arise via Fejer kernels and using
the formula from the periodic case one has

\medskip

\noindent
{\bf{Proposition.}}
\emph{For every $\mathcal B$\vvv function constructed via (*) wiuth
an arbitrary trary $N$ and an initial finite family from$\sigma(f)$ one has
the inequality below for every real number $\tau$:}
\[
\max\uuu x\, |\mathcal B\uuu N(x+\tau)\vvv \mathcal B\uuu N(x)|\leq
\max\uuu x\, |f(x+\tau)\vvv f(x)|
\]
\bigskip

\noindent{\bf{Exercise.}} Prove this result using 
the explicit formulas for Fejer kernels.
\bigskip


\noindent
Let us now see how one can approximate the given function $f$.
let us consider the case when the abelian group
generated by $\sigma(f)$ is finitely generated.
So now we have some $p$\vvv tuple $\beta\uuu 1,\ldots,\beta\uuu p$
and each $w\in \sigma(f)$ is an integer combination
of the $\beta$\vvv tuple.
If

\[ 
w= m\uuu 1\beta\uuu 1+\ldots+m\uuu p\beta\uuu p
\]
then we have the equality
\[
M\uuu {\mathcal B\uuu N}(w)=
\prod\,(1\vvv \frac{|m\uuu \nu|}{N})\cdot M\uuu f(w)
\]
From this and the $L^2$\vvv convergence of $\{ M\uuu f(w)\}$
taken over $\sigma(f)$, it follows that
\[
\lim\uuu{N\to \infty}\, || f\vvv \mathcal B\uuu N||^2=0
\]
There remains to see that the $L^2$\vvv convergence entails 
uniform convergence in the maximum norm, i.e. that

\[ 
\max\uuu x\, |f(x)\vvv \mathcal B\uuu N(x)|=0
\]

\bigskip

Proof is EASY via (xx). Call it the Arzela\vvv Bohr Lemma via equi\vvv continuity
and uniform
$\ell(\epsilon)$ choice in the almost periodicity condition.


































 











\end{document}

