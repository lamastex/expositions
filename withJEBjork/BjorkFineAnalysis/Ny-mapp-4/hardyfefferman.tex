                                         %\documentclass{amsart}
%\usepackage[applemac]{inputenc}

%\addtolength{\hoffset}{-12mm}
%\addtolength{\textwidth}{22mm}
%\addtolength{\voffset}{-10mm}
%\addtolength{\textheight}{20mm}

%\def\uuu{_}


%\def\vvv{-}

%\begin{document}


\centerline {\bf\large {III.B The Hardy space $H^1$}}



\bigskip


\noindent
\emph{0. Introduction.}\bigskip


\noindent
\emph{1. Zygmund's inequality}
\bigskip


\noindent
\emph{2. A weak type estimate.}
\bigskip


\noindent
\emph{3. Kolmogorov's inequality.}
\bigskip


\noindent
\emph{4. The dual space of $H^1(T)$}

\bigskip


\noindent
\emph{5. The class BMO}
\bigskip




\noindent
\emph{6. The dual of $\mathfrak{Re}\, H^1\uuu 0(T)$}
\bigskip




\noindent
\emph{7. Theorem of Gundy and Silver}
\bigskip

\noindent
\emph{8. The Hardy space on ${\bf{R}}$.}

\bigskip

\noindent
\emph{9. BMO and radial norms on measures in $D$}.


\bigskip






\centerline {\bf {0. Introduction.}}
\medskip


\noindent
At several occasions we have met the situation where
$F(z)\in\mathcal O(D)$
has bounded $L^1$-norms over circles of radius $r<1$.
The Brothers Riesz theorem in  Section I
shows that
if there
is a constant $M$ such that
\[
\int_0^{2\pi}\, |F(re^{i\theta})|d\theta\leq M\quad\colon\quad
0<r<1
\]
then there exists an $L^1$-function $F(e^{i\theta})$
on the unit circle and
\[
\lim_{r\to 1}\, \int_0^{2\pi}\, 
|F(re^{i\theta})-F(e^{i\theta}|\cdot d\theta=0\tag{*}
\]

\noindent
The class of analytic functions $F$ with boundary function in
$L^1(T)$ is denoted by $H^1(T)$ and called the \emph{Hardy space}.
It is tempting to start with a real valued $L^1$-function
$u(\theta)$ on the unit circle and apply the
Herglotz integral formula  which  produces both
the harmonic extension of $u$ and  its conjugate
harmonic function by the equation:
\[
g\uuu\mu(z)=
\frac{1}{2\pi}
\int_0^{2\pi}\,\frac{e^{i\theta}+z}
{e^{i\theta}-z}\cdot u(\theta)d\theta\tag{**}
\]
\medskip

\noindent
It turns out that $g\uuu\mu$ in general does not belong to $H^1(T)$,
i.e. the condition that
$u\in L^1(T)$ does not imply that $g\uuu\mu\in H^1(T)$.
Theorem  0.1 below is due to Zygmund and
gives a necessary and sufficient condition for
the inclusion
$F\in L^1(T)$ when
$u$ is non-negative.
\medskip

\noindent
{\bf 0.1 Theorem.} \emph{Let $u(\theta)$ be a non-negative $L^1$-function on $T$.
Then  $g\uuu\mu(z)\in H^1(T)$ if and only if}

\[
\int_0^{2\pi}
\,u(\theta)\cdot\log^+\,|u(\theta)|\cdot d\theta<\infty
\]
\medskip

\noindent
{\bf Remark.}
That the condition is necessary is proved in � 1.
The proof of sufficiency relies upon  study of
linear operators satisfying 
weak type estimates where  a result due to  Kolmogorov is
the essential point. To profit upon 
Kolmogorov's result in section 3
we   need a weak-type estimate
for the harmonic
conjugation functor which is proved in � 2.
\medskip


\noindent
{\bf{0.2 The dual space of $H^1(T)$.}}
On the unit circle  the Banach space
$C^0(T)$ of continuous complex-valued functions
contains the closed subspace $A_*(D)$ which consists of
those continuous function $f(e^{i\theta})$
on $T$ which extend to analytic functions in the open
disc $|z|<1$ and vanish at $z=0$.
In Theorem 4.3 we prove that
$H^1(T)$ is the dual of the quotient space
\[
B=\frac{C^0(T)}{A_*(D)}
\]
The proof uses the Brothers Riesz theorem.
We shall also consider the subspace $H^1\uuu 0(T)$
of those functions in the Hardy space for which $f(0)=0$.
Here we find that
\[ 
H^1\uuu 0(T)\simeq \bigl[\frac{C^0(T)}{A(D}\bigr]^*\tag{1}
\]
Next,
we seek the dual space
of $H\uuu 0^1(T)$. Using the Brothers Riesz theorem
one finds that
\[ 
H\uuu 0^1(T)^*\simeq \frac{L^\infty(T)}{H^\infty(T)}\tag{2}
\]
where $H^\infty(T)$ is the space of 
boundary values of bounded analytic functions in $D$.

\medskip


\noindent
{\bf{0.3 The dual of $\mathfrak{Re}\, H^1\uuu 0(T)$.}}
The real part determine functions in
$H^1\uuu 0(T)$ which
means that we can identify $H^1\uuu 0(T)$ with a real subspace of
$L^1\uuu{{\bf{R}}}(T)$ whose elements consist of those real\vvv valued
and integrable functions $u(\theta)$ for which the
the Riesz transform also is integrable. Or equivalently,
if we take the harmonic extension
$H\uuu u$ then 
the harmonic conjugate has a boundary function in $L^1\uuu{{\bf{R}}}(T)$
which we denote by $u^*$.
The norm of such a $u$\vvv function 
is defined as

\[ 
||u||= ||u||\uuu 1+||u^*||\uuu 1\tag{*}
\]



\noindent
The norm in (3) is not equivalent to the $L^1$\vvv norm
so we cannot conclude that the dual space is reduced to
real\vvv valued functions in $L^\infty(T)$.
To exhibit elements in the dual space
we first consider some
real\vvv valued function
$F(\theta)$ on $T$.
Let $H\uuu F$ be its harmonic extension to $D$.
For each $0<r<1$
we define the linear functional on 
$\mathfrak{Re}\, H^1\uuu 0(T)$ by:
\[
u\mapsto \int\uuu 0^{2\pi}\, H\uuu F(re^{i\theta})\cdot u(\theta)\cdot d\theta\tag {**}
\]
If  the limit  (*) exists for every $u$
when $r\to 1$
and
the absolute value of this limit is $\leq C\cdot ||u||$ for a constant
$C$, then we have produced a continuous linear functional on
$\mathfrak{Re}\, H^1\uuu 0(T)$.
This leads to a description of the dual space which goes as follows.
The definition of the norm in (*) and the Hahn\vvv Banach theorem
yields for each $\Lambda$ in the dual space a pair
$(\phi,\psi)$ in $L^\infty(T)$ such that
when $f=u+iu^*$ is in $H^1\uuu 0(T)$ then
\[
\Lambda(u+iu^*)=
\int\uuu 0^{2\pi}\, u(\theta)\cdot \phi(\theta)\cdot d\theta+
\int\uuu 0^{2\pi}\, u^*(\theta)\cdot \psi(\theta)\cdot d\theta
\]
Let $\psi^*$ be the harmonic conjugate of $\psi$ which gives
the analytic function $H\uuu\psi+iH\uuu \psi^*$ in $D$.
Since $f=u+iu^*$ vanishes at $z=0$
we get
\[
\int\uuu 0^{2\pi}\, (u+iu^*)(\psi+i\psi^*)
\cdot d\theta=0
\]
Regarding the imaginary part it follows that
\[
\int\uuu 0^{2\pi}
u^*\cdot \psi\cdot d\theta=\vvv \int\uuu 0^{2\pi}\, u\cdot \psi^*\cdot d\theta
\]
We conclude that $\Lambda$ is expressed by
\[
\Lambda(u)= \lim\uuu{r\to 1}\int\uuu 0^{2\pi}\, H\uuu F(re^{i\theta})\cdot
u(\theta)\cdot d\theta
\]
where
\[ 
F(\theta)= \phi(\theta)\vvv \psi^*(\theta)\tag{***}
\]
\medskip

\noindent
Above $\psi^*$ is the harmonic conjugate of a bounded
$\psi$\vvv function where an arbitrary 
$\psi\in L^\infty(T)$ can be chosen.
Next,
recall from XXX that if $\psi\in L^\infty(T)$ then its conjugate
$\psi^*$ belongs to $\text{BMO}(T)$. Hence (***) identifies
the of
$\mathfrak{Re}\, H^1\uuu 0(T)$ with  a subspace of
$\text{BMO}(T)$.
It turns out that one has equality.
More precisely, Theorem 0.4 below which is 
due to C. Fefferman and E. Stein 
asserts that $F$ yields such a continuous linear form if and only if
$F$ has a bounded mean oscillation in the sense of F. John and L. Nirenberg.


\medskip

\noindent
{\bf{0.4 Theorem.}}
\emph{A real\vvv valued $L^1$\vvv function $F$
on $T$ yields a continuous linear functional on
$H^1\uuu 0(T)$ as above if and only if $F\in \text{BMO}(T)$.
Moreover, there exists an absolute constant $C$ such that}
\[
\bigl|\int\uuu 0^{2\pi}\, H\uuu F(re^{i\theta})\cdot u(\theta)\cdot d\theta\,\bigr|\leq
C\cdot ||F||\uuu{\text{BMO}}\cdot ||u||\uuu 1
\]
\emph{for all $r<1$ and $u\in H^1\uuu 0(T)$.}
\bigskip

\noindent
We refer to Section 6 for  details of the  proof which involves several steps
where the essential
step  is to exhibit
certain Carleson measures.
The space of real\vvv valued functions of bounded mean oscillation is
denoted by $\text{BMO}(T)$ and  studied in
Section 5 where
Theorem 5.5 is an important result which clarifies many properties
of functions in $\text{BMO}(T)$.

\medskip

\noindent
{\bf{0.5 The Hardy space on ${\bf{R}}$.}}
It consists of analytic functions $F(z)$ in the upper half\vvv plane
for which there exists a constant $C$ such that
\[ 
\int\uuu{\vvv\infty}^\infty\, |F(x+i\epsilon)\cdot dx\leq C\tag{*}
\] 
hold for every $\epsilon>0$.
This space is denoted by
$H^1({\bf{R}})$.
Let us remark that it differs from $H^1({\bf{T}})$ even if we
employ  the conformal map
\[
w=\frac{z\vvv i}{z+i}\tag{i}
\] 
onto the unit disc.
More precisely, with $F(z)$ given in the upper half\vvv plane
we set
\[ 
f(w)= 
F(\frac{i+iw}{1\vvv w})\tag{ii}
\]
Then the reader can verify that
\[
\lim\uuu{r\to 1}\int\uuu 0^{2\pi}\, |f(re^{i\theta})|\cdot d\theta=
\int\uuu{\vvv\infty}^\infty\, 
\frac{|F(x)|}{1+x^2}\cdot dx\tag{iii}
\]
where $F(x)$ is the almost everywhere defined limit of $F$
on the real $x$\vvv line which by (*)
identifies $F(x)$ with an element in
$H^1({\bf{R}})$.
Since
$\frac{1}{1+x^2}$ is bounded 
it follows that the right hand side is finite in (iii) and hence
$f$ belongs to $H^1({\bf{T}})$.
However, the map $F\to f$ is not bijective
because the convergence in (iii) need not imply that
(*) is finite.
In other words, the Hardy space on the real line is
more constrained and via $F\mapsto f$
it appears s a proper subspace
of $H^1({\bf{T}})$  and the corresponding norms are not equivalent.
Sections 7 and  9  study  $H^1({\bf{R}})$ and at the end of
� 9 we  introduce Carleson norms on
non\vvv negative Riesz measures in $\mathfrak{Im}(z)>0$
which will be used for interpolation of bounded analytic functions
in Chapter XXX.

























\newpage









\centerline {\bf 1. Zygmund's  inequality}


\bigskip


\noindent
Let
$u(\theta)$ be a non-negative real-valued function on
$T$ such that
\[
\frac{1}{2\pi}
\int_0^{2\pi}u(\theta)d\theta=1\tag{*}
\]
Put
\[
F(z)=
\frac{1}{2\pi}
\int_0^{2\pi}\,\frac{e^{i\theta}+z}
{e^{i\theta}-z}\cdot u(\theta)d\theta
\]
We write
\[ 
F=u+iv
\] 
where $u$ is the harmonic extension of
$u(\theta)$ from $T$ to $D$ and 
$v$ is the harmonic conjugate  which by the
Herglotz
formula is normalised so that $v(0)=0$
The
sufficency part in Zygmund's theorem
follows from the general inequality below:

 \medskip
 
 \noindent
{\bf 1.1 Theorem.} \emph{When   $u(\theta)$ is non-negative and (*) holds
we have}
\[
\int_0^{2\pi}
\,u(\theta)\cdot\log^+\,|u(\theta)|\, d\theta\leq
\frac{\pi}{2}\cdot
\int_0^{2\pi}
|v(\theta)|
\cdot d\theta+
\int_0^{2\pi}\,\log^+|F(e^{i\theta}]|\,d\theta\tag{*}
\]




\bigskip

\noindent \emph{Proof.}
Since $\mathfrak{Re}(F)>0$ holds in $D$
we can write
\[ 
\log \,F(z)=\log\,|F(z)|+i\gamma(z)\quad\colon\quad
-\pi/2<\gamma(z)<\pi/2\tag{i}
\]
Set $G(z)=F(z)\cdot\log\,F(z)$.
Since $F(0)=1$ we have
$G(0)=0$    and  the mean value formula for harmonic functions gives
\[
\int_0^{2\pi}
u(e^{i\theta})\cdot\log\,|F(e^{i\theta})|\, d\theta=
\int_0^{2\pi}
\gamma(e^{i\theta)})\cdot v(e^{i\theta})\, d\theta\tag{iii}
\]


\noindent
By (i) the absolute value of the right hand side is majorised by
\[
\frac{\pi}{2}\cdot
\int_0^{2\pi}
|v(\theta)|
\cdot d\theta\tag{iii}
\]

\noindent
Now we use the decomposition
\[
\log\,|F(e^{i\theta})|=
\log^+\,|F(e^{i\theta})|+\log^+\,\frac{1}{|F(e^{i\theta})|}
\]
Then (ii) and (iii) give
the inequality
\[
\int_0^{2\pi}
u(e^{i\theta})\cdot\log^+\,|F(e^{i\theta})|\, d\theta\leq
\]
\[\frac{\pi}{2}\cdot
\int_0^{2\pi}
|v(\theta)|
\,d\theta+
\int_0^{2\pi}
u(e^{i\theta})\cdot\log^+\frac{1}{\,|F(e^{i\theta})|}\,d\theta
\tag{iv}
\]

\noindent
Since
$\log^+\frac{1}{\,|F(e^{i\theta})|}\neq 0$
entails that $|F|\leq 1$ and hence also $u\leq 1$,
it follows that the last integral above is majorised by
\[
\int_0^{2\pi}
\log^+\frac{1}{\,|F(e^{i\theta})|}\, d\theta\tag{v}
\]
Next, $\log\,|F(z)|$ is a harmonic function whose value at $z=0$ is zero.
So the mean-value formula for harmonic functions in $D$ gives
the equality:
\[
\int_0^{2\pi}
\log^+\frac{1}{\,|F(e^{i\theta})|}\, d\theta=
\int_0^{2\pi}
\log^+\,|F(e^{i\theta})|\,d\theta\tag{vi}
\]
\medskip


\noindent
Using this and (iv) we get the requested inequality in
Theorem 1.1 since
we also have the trivial estimate
\[
\log^+\,u(e^{i\theta})\leq
\log^+\,|F(e^{i\theta})|\tag{vii}
\]

















\newpage

\centerline
 {\bf 2. The weak type estimate.}
\bigskip

\noindent
Let $u(\theta)$ be non-negative and denote by $v(\theta)$ its harmonic conjugate
function which is obtained via Herglotz formula.If $E$ is a subset of $T$ we denote its
linear Lebesgue measure by
$\mathfrak{m}(E)$.
With these notations the following weak-type estimate hods:

\medskip

\noindent
{\bf 2.1 Theorem.} \emph{For each non-negative $u$-function on $T$
with mean-value one the following holds:}
\[
\mathfrak{m}(\{|v|)>\lambda\}\leq
\frac{ 4\pi}{1+\lambda}\quad\colon\quad \lambda>0
\]


\noindent
\emph{Proof.}
For a given $\lambda>0$ we set
\[ 
\phi(z)=1+\frac{F(z)-\lambda}{F(z)+\lambda}\tag{1}
\]
where $F(z)$  the analytic function constructed as in section 1.
Here $F(0)=u(0)=1$ which gives:
\[
\phi(0)=\frac{2}{\lambda+1}\tag{2}
\]
Next, since $\mathfrak{Re}\, F=u\geq 0$ it
follows that
\[
\bigl |\frac{F(z)-\lambda}{F(z)+\lambda}\bigr |\leq 1\tag{3}
\]

\medskip

\noindent
Hence (1)  gives $\mathfrak{Re}(\phi)\geq 0$ and
mean value formula for the harmonic function
$\mathfrak{Re}(\phi)$ gives:
\[
\frac{4\pi}{1+\lambda}=
\int_0^{2\pi}\,
\mathfrak{Re}\,\phi(e^{i\theta})\cdot d\theta\geq
\int_{\mathfrak{Re}\,\phi\geq 1}
\mathfrak{Re}\,\phi(e^{i\theta})\cdot d\theta\geq
\mathfrak{m}(\{\mathfrak{Re}\,\phi\geq 1\})\tag{4}
\]
Rewriting the last inequality we get:
\[
\mathfrak{m}(\{ \mathfrak{Re}\,\phi\geq 1\})\leq
\frac{4\pi}{1+\lambda}\tag{5}
\]


\noindent
Next,  the construction of
$\phi$ yields the following equality of sets:
\[
\{\mathfrak{Re}\,\phi(e^{i\theta})\geq 1\}=\{\mathfrak{Re}
\,\frac{F(e^{i\theta})-\lambda}
{F(e^{i\theta})+\lambda}\geq 0\}\tag{6}
\]


\noindent
Finally,  with $F(e^{i\theta})= u(\theta)+iv(\theta)$
one has
\[
\mathfrak{Re}\,[\,\frac{F(e^{i\theta})-\lambda}
{F(e^{i\theta})+\lambda}\,]=\frac{u^2+v^2-\lambda^2}{(u+\lambda)^2+v^2}
\]
The right hand side is $\geq 0$ when
$|v|\geq \lambda$ which  gives the
set-theoretic inclusion:
\[
\{|v|>\lambda\}\subset
\{\mathfrak{Re}\,\phi\geq 1\}).\tag{7}
\]
Then  (5) above gives Theorem 2.1.





\bigskip


\centerline{\bf 3. Kolmogorov's inequality}

\bigskip


\noindent {\bf 3.1 Notations.}
Consider a measure space equipped with a probability measure
$\mu$.
Let $f$ be a complex-valued and $\mu$-measurable 
function. For each  $\lambda>0$ we get
the $\mu$-measurable set $\{|f|>\lambda\}$ and then
\[ 
\lambda\mapsto \mu(\{|f|>\lambda\})
\] 
is a decreasing function. Construct
the differential function defined for every $\lambda>0$ by:
\[ 
d\rho_f(\lambda)=\lim_{\delta\to 0}\,
\frac{\mu(\{|f|>\lambda-\delta\})-\mu(\{|f|>\lambda\})}{\delta}\tag{*}
\]
For an arbitrary continuous function
$Q(\lambda)$ defined when $\lambda\geq 0$ 
the  formula in  XX gives the equality: 
\[
\int_0^\infty 
Q(|f|)d\mu=
\int_0^\infty\, Q(\lambda)\cdot d\rho_f(\lambda)\tag{**}
\]
Recall also from XX the formula
\[ 
\int_0^\infty \,
\mu[\{|f|>\lambda\}]\cdot d\lambda=\int\,|f|\cdot d\mu\tag{***}
\]

\medskip

\noindent
{\bf 3.2 Operators of Weak type (1,1).}
Let $\gamma$ be a probability measure on another sample space
and  $T$ is  some   linear map from
$\mu$-measurable functions into
$\gamma$-measurable functions.
\medskip

\noindent
{\bf 3.3 Definition. }\emph{We say that
$T$ satisfies a weak-type estimate of type (1,1)
if
there is a constant $K$ such that  the inequality below holds
for every $\lambda>0$:}
\[ 
\gamma(\{|Tf|>\lambda\}\,)\leq\frac{K}{\lambda}\cdot\int\,|f|\cdot d\mu
\quad\text{when}\quad f\,\,\,\text{is} \,\mu-\text{measurable}
\]

\medskip
\noindent
We can  also regard $L^2$-spaces. The operator
$T$ is $L^2$-continuous if there exists
a constant $K_2$ such that one has the inequality
\[
\int\,|T(f)|^2\cdot d\gamma\leq K_2^2\cdot
\int\,|f|^2\cdot d\mu
\]
Taking square roots it means that the $L^2$-norm
is $K_2$. 
\medskip

\noindent 
{\bf 3.4 Theorem.}
\emph{Let $T$ be a linear operator whose $L^2$-norm is 1 and with
finite weak-type norm $K$.
Then the following holds for each $\mu$-measurable function $f$:}
\[ 
\int\,
|T(f)|\cdot d\gamma\leq 1+4\cdot\int\,|f|\cdot d\mu+
2K\cdot \int\, |f|\cdot\log^+\,|f|\cdot d\mu
\]

\medskip

\noindent
\emph{Proof.} 
When $\lambda>0$ we decompose $f$
as follows:
\[
f=f_*+f^*\quad\colon
f_*=\chi_{\{|f|\leq\lambda\}}\cdot f\quad\colon\quad
f^*=\chi_{\{|f|>\lambda\}}\cdot f\tag{i}
\]

\noindent For the lower $f_*$-function we use that $T$ has
$L^2$-norm $\leq 1$ and get
\[
\gamma[\{|Tf_*|>\lambda/2\}\,]
\leq\frac{4}{\lambda^2}\int_0^\lambda\, s^2\cdot d\rho_f(s)\tag{ii}
\]

\noindent For $Tf^*$ we apply the weak-type estimate which gives 
\[
\gamma[\{|Tf^*|>\lambda/2\}\,]\leq
\frac{2K}{\lambda}\cdot\int_\lambda^\infty s\cdot d\rho_f(s)\tag{iii}
\]

\noindent 
where we used that
$\int_\lambda^\infty s\cdot d\rho_f(s)$ is the $L^1$-norm of $f^*$.
The set-theoretic inclusion 
\[
\{|Tf|>\lambda\}\subset
\{|Tf_*|>\lambda/2\}\cup\,\{|Tf^*|>\lambda/2\}\implies
\]
\[
\gamma[\{|Tf|>\lambda\}\,]\leq
\frac{4}{\lambda^2}
\int_0^\lambda\, s^2\cdot d\rho_f(s)+
\frac{2K}{\lambda}\cdot\int_\lambda^\infty sd\rho_f(s)\tag{iv}
\]
Next, since $\gamma$ has total mass one the  inequality:
\[ 
\int_0^\infty\,|Tf|\cdot d\gamma\leq 1+
\int_{\{|Tf|>1 \}}\,|Tf|\cdot d\gamma\tag{v}
\]


\noindent
Now (***) in (3.1) is applied to $Tf$ and the measure $\gamma$ which gives
\[
\int_{\{|Tf|>1 \}}\,|Tf|\cdot d\gamma=
\int_1^\infty\, \gamma[\{|Tf|>\lambda\}\cdot d\lambda
\]

\noindent 
By (iv) the
last integral in (v) is majorised by:
\[
4\cdot \int_1^\infty
\,\bigl[\frac{1}{\lambda^2}
\int_0^\lambda\, s^2\cdot d\rho_f(s)\bigr]\cdot d\lambda
+2K\int_1^\infty\, \frac{1}{\lambda}\cdot [\int_\lambda^\infty sd\rho_f(s)]\cdot d\lambda\tag{vi}
\]



\noindent
Next,  from (XX)  one has the equality:
\[
\int_0^\infty\,\bigl[\frac{1}{\lambda^2}
\int_0^\lambda\, s^2\cdot d\rho_f(s)\bigr]\cdot d\lambda=\int\,|f| \cdot d\mu\tag{vii}
\]
The left hand side is only smaller if the 
$\lambda$-integration starts at 1. It follows that the
first term in (vi) above is majorised by
$4\cdot \int\,|f| \cdot d\mu$ and together with (v) we
conclude that

\[
\int_0^\infty\,|Tf|d\gamma\leq 1+4\int\,|f|d\mu+
2K
\int_1^\infty\,[\frac{1}
{\lambda}\cdot\int_\lambda^\infty sd\rho_f(s)]\cdot d\lambda\tag{viii}
\]
Finally, 
\[
\int_1^\infty\,[\frac{1}
{\lambda}\cdot\int_\lambda^\infty sd\rho_f(s)]\cdot d\lambda=
\iint_{1\leq\lambda\leq s}\,
\frac{1}{\lambda}\cdot
s\rho_f(s) ds=\int_1^\infty s\cdot \log\, s\cdot d\rho_f(s)
\]
\medskip

\noindent
The last integral is equal to $\int \,f\cdot\log^+\,|f|\cdot d\mu$
by the general formula XX. 
Inserting this in (viii) we get  Theorem 3.4.

\bigskip

\noindent{\bf 3.5. Final part of  Theorem 0.1}
There remains to show that if $u$ is non-negative and if 
$u\cdot \log^+u$ is integrable so is $v$.
To prove this we use $d\mu=d\gamma=\frac{d\theta}{2\pi}$
on the unit circle.
Theorem 2.1  which shows that the harmonic conjugation operator
$T\colon u\mapsto v$ is of weak-type (1,1) and it is
continuous on $L^2(T)$  by
Parseval's formula. Hence  Kolomogorv's Theorem gives
$v\in L^1(T)$ which proves the necessity in Theorem 0.1.


\medskip

\noindent
{\bf Remark.} Notice that
Theorem 3.4 applies when we start from any
real-valued function  $u(\theta)$.
So have the following supplement to Theorem 0.1.

\bigskip

\noindent
{\bf 3.6 Theorem.} \emph{There exists an absolute constant $A$ such that}
\[
\int_0^{2\pi}\, |v(\theta)|\, d\theta\leq
A\cdot\bigr[\int_0^{2\pi}\, |u(\theta)|\, d\theta+
\int_0^{2\pi}\, |u(\theta)|\cdot \log^+|u(\theta)|\,  d\theta\,\bigr]
\]




\bigskip

\centerline{\bf 4. The Dual space of $H^1(T)$}
\medskip

\noindent
On the unit circle $T$
we have the Banach space
$L^1(T)$ 
where $H^1(T)$ is a closed subspace.
Next, let $C^0(T)$ be the Banach space of continuous
functions on $T$ equipped with the maximum norm.
It contains the closed subspace $A(D)$
whose functions can be extended as
analytic  functions in the open disc $D$.
We have also the subspace $A_*(D)$
which consists of  the functions in $A(D)$ whose analytic
extensions are zero at the origin.
As explained in XXX a continuous function $f$ on $T$
belongs to $A_*(D)$ if and only if
\medskip
\[
\int_0^{2\pi}\, e^{in\theta}\cdot f(e^{i\theta})\cdot d\theta=0
\quad\colon\quad n=0,1,\ldots
\tag{*}
\]


\noindent
From (*) it follows that
\[
\int_0^{2\pi}\,g(e^{i\theta})\cdot f(e^{i\theta})\cdot d\theta=0
\quad\colon\quad f\in A_*(D)\quad\text{and}\quad \, g\in H^1(T)\tag{**}
\]
Let us now regard the
Banach space
\[ 
B=\frac{C^0(T)}{A_*(D}
\]


\noindent
Riesz representation formula identifies the
dual space of $C^0(T)$ with Riesz measures on $T$.
Since $B$ is a quotient space its dual space becomes
\[ 
B^*=\{\,\mu\in M(T)\quad\colon\, \mu\perp A_*(D\,\}\tag{i}
\]


\noindent
Now a Riesz measure $\mu$ is $\perp A_*(D$
if and only if
\[
\int_0^{2\pi}\, e^{in\theta}\cdot d\mu(\theta)=0
\quad\colon\quad n=1,2\ldots\tag{ii}
\]


\noindent The Brothers Riesz theorem means
that  (ii) holds if and only if
$\mu$ is absolutely continuous, i.e. $\mu$
is given by some $L^1$-function $f$ which 
satisfies:
\[
\int_0^{2\pi}\, e^{in\theta}\cdot f(e^{i\theta})\cdot d\theta=0
\quad\colon\quad n=1,2\ldots\tag{iii}
\]
This is precisely the condition  that
$f\in H^1(T)$. Hence the whole discussion gives:
\bigskip

\noindent 
{\bf 4.1 Theorem.}
\emph{The Hardy space $H^1(T)$ is the dual of $B$.}
\bigskip

\noindent
{\bf 4.2 The dual of $H^1(T)$.}
Recall that $L^\infty(T)$ is the dual space of
$L^1(T)$.
So by a general formulafrom Appendix: Functional analysis
we get:

\[
H^1(T)^*=
\frac{L^\infty(T)}{H^1(T)^\perp}
\]


\noindent
Next, an $L^\infty$-function $f$ is $\perp H^1(T)$ if and only if
\[
\int_0^{2\pi}\, e^{in\theta}\cdot f(e^{i\theta})\cdot d\theta=0
\quad\colon\quad n=0,1,2\ldots
\]


\noindent
But this means precisely that $f$ is the boundary value of an analytic function
in $D$ which vanishes at the origin.
Let us identify  $H^\infty(D)$ with
a subalgebra of  $L^\infty(T)$
which is denoted by $H^\infty(T)$. Then we also get the subspace
$H_0^\infty(T)$ of those functions which are zero at
the origin.
Hence we have proved
\medskip

\noindent 
{\bf Theorem  4.3} \emph{The dual space of $H^1(T)$ is equal to the quotient
space}
\[ 
\frac{L^\infty(T)}{H^\infty_0(T)}
\]
\medskip



\centerline{\bf 5. BMO}


\bigskip


\noindent
{\bf Introduction.}
Functions of bounded mean oscillation were
introduced by F. John and L. Nirenberg in
[J-N].
This class of Lebesgue measurable functions
can be defined in $\bf R^n$ for every $n\geq 1$. Here we are content to
study the case $n=1$ and  restrict the attention to periodic functions
which is adapted to the class BMO on the unit circle. 
So let $F(x)$ be a locally integrable function on the real $x$-line
which is $2\pi$-periodic, i.e. 
$F(x+2\pi)=F(x)$.
If $J=(a,b)$ is an interval we  get the mean value
\[ 
F_J=\frac{1}{b-a}\cdot\int_a^b\, F(x)dx
\]
To every interval $J$ we put
\[
|F|_J^*=\int_J\,|F(x)-F_J|\cdot dx
\]
\medskip

\noindent 
{\bf 5.1 Definition.}
\emph{The function $F$ has a bounded mean oscillation if there exists a constant
$C$ such that
$|F|_J\leq C$ for all intervals $J$. When this holds the smallest constant
is denoted by $|F|_{\text{BMO}}$.}
\medskip

\bigskip

\noindent
{\bf 5.2 The case $n\geq 2$.}
Even though these notes are devoted to complex analysis in
dimension one, we cannot refrain from
mentioning a result which illustrates how the class BMO enters
in Fourier analysis. Let $F(x)$ be an $L^1$-function with
compact support in $\bf R^n$. Assume 
that there exists a constant $C$ such that
the  Fourier transform
$\widehat F(\xi)$
satisfies the decay condition
\[ 
|\widehat F(\xi)\leq C\cdot(1+|\xi|)^{-n}\quad\colon\quad \xi\in\bf R^n\tag{*}
\]
This is not quite enough for $\widehat F$
to be integrable. So we cannot expect
that (*) implies that $F(x)$ is a bounded function.
However, its belongs to BM0 and more precisely one has:

\medskip

\noindent
{\bf{5.3 Theorem.}}
\emph{To each $M>0$ there exists a constant $C_M$ such that if $F(x)$
has support in the ball $\{|x|\leq M\}$
then}
\[ ||F||_{\text{BMO}}\leq C_M\cdot \max_\xi\,
\bigl[1+|\xi|)^n\cdot |\widehat F(\xi)|\bigr]
\]
For the proof we refer to [Bj�rk] and   [Sj�lin] contains the 
improved result
that $F$ belongs to BMO under less restrictive conditions expressed by
certain $L^2$-integrals of $\widehat F$ over   dyadic
grids. 
\bigskip

\noindent{\bf{5.4 The John-Nirenberg inequality.}}
We turn to the main topic in this section and prove
an 
inequality due to  F. John and L. Nirenberg  which is
presented for the 1\vvv dimensional periodic case.
See  [J-N] for
higher dimensional results.
\medskip









\noindent
{\bf 5.5 Theorem}
\emph{Let $F(x)$ be a $2\pi$-periodic function on
the real $x$-line which belongs to  BMO on $T$.  For every interval $J$ on
${\bf{R}}$ 
and every positive integer $n$ one has}
\[ 
\mathfrak{m}[\{ x\in J\,\colon\, \bigl|F(x)-F_J\bigr|\geq
4n\cdot |F|_{\text{BMO}}\}]\leq 2^{-n}\cdot |J|
\]
\medskip


\noindent
The proof requires several steps.
To begin with we make some trivial observations.
The BMO-norm of $F$ is unaffected when we add a constant to
$F$ and also 
under a translation, i.e. when we regard $F_a(x)=F(x+a)$ for some
real number $a$. Moreover, the BMO-norm is 
unchanged under dilations, i.e. when 
$t>0$ and $F_t(x)=F(tx)$.
Before we enter the proof we  need a preliminary result.
\medskip

\noindent {\bf 5.6 Lemma.}
\emph{Let $F$ belong to BMO. Let $I\subset J$ be two intervals with the same 
mid-point. Then}
\[
|F_J-F_I|\leq  2\cdot[\text{Log}_2\,\frac{|J|}{|I|}+1]\cdot |F|_\text{BMO}
\]




\noindent{\bf{Exercise.}} Prove this result.

\bigskip



\noindent
\emph{Proof of Theorem 5.5}.
Replacing $F$ by $cF$ for some positive constant 
we may assume that its BMO-norm is 1/2.
and that $F_J=0$. Moreover, by the invariance under dilations
and translations  we may
assume that $J$ is the unit interval.
Thus, there remains to consider the set
\[ 
E_n=\{ x\in [0,1]\,\, \colon\,\, F(x)> 2n\}\tag{i}
\]
and  show that 
\[
\mathfrak{m}(E_n)\leq 2^{-n}\tag {ii}
\]

\noindent
Let us begin with the case $n=1$. 
For every $x\in E_1$ which is a Lebesgue point for $F$ we find the
unique largest dyadic interval $J(x)$ such that
\[
x\in J(x)\subset [0,1]\quad\colon\quad
\frac{1}{\mathfrak{m}(J(x))}\int_{J(x)}\, F(t)dt >1\tag{iii}
\]
Up to measure zero, i.e. ignoring the null set where
$F$ fails to have Lebesgue points, we have the inclusion 
\[ 
E_1\subset\,\cup_{x\in E_1}\, J(x)\tag{iv}
\]


\noindent 
Next, suppose we have a \emph{strict} inclusion $J(x)\subset J(y)$
for a pair of dyadic intervals in this family
which means that
$\mathfrak{m}(J(y))>\mathfrak{m}(J(x))$.
But this is impossible for then $x\in J(y)$ which contradicts 
the maximal choice of $J(x))$ as the dyadic interval of largest possible length containing $x$.
Hence the family $\{J(x_\nu)\}$ consists of dyadic intervals which either are equal or disjoint.
We can therefore pick a disjoint family where the corresponding $x$-points are
denoted by $x^*_\nu$
and obtain the set-theoretic inclusion
\[ 
E_1
\subset\,\cup\,\, (J(x^*_\nu)\tag{v}
\]
Next, put $\mathcal E=\cup\, J(x^*_\nu)$.
Since the mean value of $F$ over each $J(x^*_\nu)$ is 
$\geq 1$ we obtain
\[
\mathfrak{m}( \mathcal E)\leq
\sum\,\int_{J(x^*_\nu)}\,F(x)dx=
\int_{\mathcal E}\,F(x)dx\leq
\int_{\mathcal E}\,|F(x)|dx\leq\int_0^1\, |F(x)|dx
\leq |F|_{\text{BMO}}
\]
where the last inequality follows from the definition of the BMO-norm
and the condition that the mean-value of $F$ over the unit interval was zero.
Since the BMO-norm of $F$ was 1/2 the inclusion (v) gives:
\[
\mathfrak{m}( E_1)\leq\mathfrak{m}(\mathcal E)\leq 1/2\tag{*}
\] 


\noindent
This proves the case $n=1$ and  we proceed by an
induction over $n$.
Let us first regard one of the dyadic intervals $J(x^*_\nu)$
from the family covering $E_1$.
If $2^{-N}$ is the length of $J(x_\nu^*$
the
dyadic exhaustion of $[0,1]$ gives
a dyadic interval $J'$ of length $2^{-N+1}$
which contain $J(x_\nu^*)$.
The maximal choice of $J(x_\nu^*)$ gives:
\[
\frac{1}{\mathfrak{m}(J')}\int_{J'} F(t)dt\leq 1\tag{vi}
\]
Apply Proposition XX to the pair $J(x_\nu^*)$ and $J'$.
Since $|F|_{\text{BMO}}=1/2$ is assumed and $\text{Log}_2(2)=0$
we obtain
\[
F_{J(x_\nu^*)}=\frac{1}{\mathfrak{m}(J(x_\nu^*))}\int_{J(x_\nu^*)}\, F(t)dt \leq 2
\tag{vii}
\]
Let $n\geq 2$ and for every $\nu$ we set:
\[
E_n(\nu)=E_n\cap J(x_\nu^*)\tag{viii}
\]
Since $F(x)>2n$ holds on $E_n$ we  get
\[
F(x)-F_{J(x_\nu^*)}>2(n-1)\quad\colon\quad  x\in E_n(\nu)\tag{ix}
\]
 
 
 
\medskip

\noindent Hence we have the inclusion
\medskip
\[ 
E_n(\nu)\subset W_n(\nu)= \{
x\in J(x_\nu^*)\quad\colon\,
F(x)-F_{J(x_\nu^*)}>2(n-1)\}\tag{x}
\]



\noindent
By a change of scale we can use
the interval $J(x_\nu^*)$ instead of the unit interval 
and by an induction assume that the inequality in
Theorem 5.5 holds for $n-1$. It follows that
the set in right hand side  in (x) is estimated by:
\medskip
\[ 
\mathfrak{m}(W_n(\nu))\leq 2^{-n+1}\cdot
\mathfrak{m}(J(x_\nu^*))\tag{xi}
\]


\noindent
The set-theoretic inclusion (x)
therefore gives
\[
\mathfrak{m}(E_n(\nu))\leq 2^{-n+1}\cdot
\mathfrak{m}(J(x_\nu^*))\tag{xii}
\]

\noindent
Finally, since $E_n\subset E_1$ and
we already have the inclusion (iv)
we obtain
\[ 
\mathfrak{m}(E_n)=
\sum\,
\mathfrak{m}(E_n(\nu))\leq 2^{-n+1}\cdot\sum\, 
\mathfrak{m}(J(x_\nu^*))=2^{-n+1}\mathfrak{m}(\mathcal E)
\leq2^{-n+1}\cdot\frac{1}{2}= 2^{-n}
\]
\medskip
This proves the induction step and Theorem 5.5 follows.
\bigskip

\centerline{\bf{5.7 An $L^2$\vvv inequality}}
\medskip


\noindent
Let $F\in\text{BMO}(T)$ be given.
Given some interval $J\subset T$
and $\lambda>0$ we set
\[
\mathfrak{m}\uuu J(\lambda)=
\{\theta\in J\,\colon\, |F(\theta)\vvv F\uuu J|>\lambda\}
\]
Consider the integral

\[ 
I=\frac{1}{|J|}\cdot\int\uuu 0^\infty\,\lambda\cdot \mathfrak{m}\uuu J(\lambda)\cdot d\lambda\tag{*}
\]
Set $A=4\cdot ||F||\uuu{\text{BMO}}$. 
Theorem 5.5. gives

\[
I=\frac{1}{|J|}\cdot \sum\uuu{n=0}^\infty\,
\int\uuu {nA}^{(n+1)A}\,\lambda\cdot \mathfrak{m}\uuu J(\lambda)\cdot d\lambda
\leq
\frac{1}{|J|}\cdot \sum\uuu{n=0}^\infty\,(n+1)A\cdot |J|\cdot \cdot 2^{\vvv n}=
C||F||\uuu{\text{BMO}}
\]
where $C= 4\cdot \sum\uuu{n=0}^\infty\,(n+1)\, 2^{\vvv n}$
is an absolute constant.
Next, by the general result in XX (*) is equal to
\[
\frac{1}{|J|}\cdot \int\uuu J\, |F(x)\vvv F\uuu J|^2\cdot dx\tag{**}
\]
So by the above  (**) is majorized by an absolute constant times
the BMO\vvv norm of $F$.




\bigskip




\centerline{\bf{5.8 BMO and the Garsia norm.}}
\bigskip

\noindent
Using the $L^2$\vvv inequality in (5.7)
an elegant description of $\text{BMO}(T)$
was discovered by Garsia which we shall use in
Section 6. First we give:
\bigskip


\noindent
{\bf{5.9 Definition.}} \emph{To each real\vvv valued $u\in L^1(T)$
we define a function in $D$ by}

\[
\mathcal G\uuu u(z)=
\frac{1}{8\pi^2}
\cdot \iint\,\frac{(1\vvv |z|^2)^2}{|e^{i\theta}\vvv z|^2\cdot |e^{i\phi}\vvv z|^2}
\cdot
[u(\theta)\vvv u(\phi)]^2\cdot d\theta d\phi
\]


\noindent
\emph{If this function is bounded  we set}
\[ 
\mathcal G(u)=\max\uuu{z\in D}\, \sqrt{\mathcal G\uuu u(z)}\tag{*}
\]
\emph{and say that $u$ has a finite Garsia norm.}

\medskip

\noindent
{\bf{Remark.}} Notice that constant functions have
zero\vvv norm. So just as for BMO the 
$\mathcal G$\vvv norm is defined on the quotient of functions modulu
constants.



\medskip

\noindent
{\bf{5.10 Exercise.}}
Expanding
the square $[u(\theta)\vvv u(\phi)]^2$
the reader can verify that
\[ 
\mathcal G\uuu u= H\uuu{u^2}\vvv H^2\uuu u\tag{*}
\] 
where $H\uuu {u^2}$ is the harmonic extension of $u^2$.
and $ H^2\uuu u$ the square of the harmonic extension $H\uuu u$.


\medskip

\noindent
{\bf{5.11 Theorem.}} 
\emph{An $L^1$\vvv function $u$ has finite Garsia norm if and only if
it belongs to BMO. Moreover, 
there exists a constant $C\geq 1$ such that}

\[ 
\frac{1}{C}\cdot 
||u||\uuu{\text{BMO}}\leq
\mathcal G(u)\leq 
C\cdot ||u||\uuu{\text{BMO}}
\]
\medskip

\noindent{\bf{5.12 Exercise.}}
The reader is invited to prove this result using
the previous facts about BMO and also
straightforward properties of the Poisson kernel.
if necessary, consult [Koosis  p. xxx\vvv xxx]
for details.
\bigskip


\noindent
{\bf{5.13  The Garsia norm and Carleson measures.}}
Let $f$ be a real\vvv valued continuous function on $T$.
We get the two harmonic functions $H\uuu f$ and $H\uuu{f^2}$ and recall from
(5.10) that
\[ 
\mathcal G\uuu f=H\uuu{f^2}\vvv(H\uuu f)^2
\]
In XX we  introduced the family of Carleson
sectors in $D$ and now we prove
an important inequality.


\medskip

\noindent
{\bf{5.14 Theorem.}}
\emph{For every Carleson sector $S\uuu h$ with $0<h<1/2$
and each $f\in C^0(T)$
one has the inequality}

\[
\frac{1}{h}\cdot \iint\uuu{S\uuu h}\, 
|z|\cdot \log\,\frac{1}{|z|}\cdot
\bigl |\nabla(H\uuu f)\bigr |^2\cdot dxdy\leq
96\cdot \mathcal G(f)^2
\]
\medskip


\noindent
\emph{Proof.}
We use the conformal map where $z=\frac{\zeta\vvv i}{\zeta+i}$.
If $\phi(z)$ is a function in $D$ we get the function
$\phi^*(\zeta)$  in the upper half\vvv plane where 
\[
\phi(\frac{\zeta\vvv i}{\zeta+i})=\phi^*(\zeta)
\]
One easily verifies that 
\[ 
(|z|\cdot \log\,\frac{1}{|z|})^*(\xi+i\eta)\leq
8\cdot \eta\tag{i}
\]
Set $w(\zeta)= H\uuu f^*(\zeta)$.
Then (i) implies that
the double integral which appears in the Theorem  5.14 is majorised by 
\[ 
J^*=8\cdot \iint\uuu{S\uuu h^*}\, \eta\cdot|\nabla(w)|^2 \cdot d\xi d\eta\tag{ii}
\]
where $S\uuu h^*$ is the image of $S\uuu h$ under the conformal map
and 
$|\nabla(w)|^2=w\uuu\xi^2+w\uuu\eta^2$.
Next, from (*) in Exercise 5.10 we have

\[
w^2=H\uuu{f^2}^*\vvv\mathcal G\uuu f^*
\]
Since $H^*\uuu{f^2}$ is harmonic  we obtain
\[
2\cdot |\nabla(w)|^2=\Delta(w^2)=\vvv\Delta(\mathcal G\uuu f^*)\tag{iii}
\]
where the first easy equality follows since $w$ is harmonic.
As explained by figure XX the set $S^*\uuu h$
is placed above an interval on the real $\xi$\vvv line and 
and since the subsequent estimates are invariant under the center of this interval
we therefore may take it as $\xi=0$.
Let us  introduce the half\vvv disc
\[
 D\uuu{2h}=\{|\zeta|<2h\}\cap \{\eta>0\}
 \]
Then a figure shows that
\[
S^*\uuu h\subset D\uuu{2h}\tag{iv}
\]
Next, consider the 
function $1\vvv\frac{|\zeta|}{2h}$ and notice that it is
$\geq 1/4$ in $D\uuu{2h}$.
Recall from the above that
\[
\Delta(\mathcal G\uuu f^*)=\vvv 2\cdot |\nabla(w)|^2\leq 0
\]
From the inclusion (iv) and taking the minus signs into the account
it follows from (ii) that
\[ 
J^*\leq \vvv 16\cdot \iint\uuu{D\uuu{2h}}
 \eta(1\vvv\frac{|\zeta|}{2h})\cdot\Delta (\mathcal G^*\uuu f) \cdot d\xi d\eta\tag{v}
\]
Apply Green's formula
to the pair $\mathcal G^*\uuu f$ and $\rho=\eta(1\vvv\frac{|\zeta|}{2h})$.
Here $\rho=0$ on the boundary of $D\uuu{2h}$
and it is easily checked that the outer normal 
$\partial\uuu n(\rho)\leq 0$. At the same time
$\mathcal G^*\uuu f\geq 0$ and from this it follows that
(v) gives:
\[ 
J^*\leq \vvv 16\cdot \iint\uuu{D\uuu{2h}}
 \Delta(\eta(1\vvv\frac{|\zeta|}{2h}))\cdot\mathcal G^*\uuu f
 \cdot d\xi d\eta\tag{iv}
\]
Next, using polar coordinates
$(r,\phi)$ an easy computation gives
\[
 \Delta(\eta(1\vvv\frac{|\zeta|}{2h}))=\vvv\frac{3}{2h}\cdot \sin\,\phi
 \]
It follows that
\[ 
J^*\leq  \frac{24}{h}\cdot 
\iint\uuu{D\uuu{2h}}
\mathcal G^*\uuu f\cdot \sin\,\phi\cdot rdr d\phi
\]
Finally, by definition  $\mathcal G(f)^2 $ is the maximum norm of 
$\mathcal G\uuu f$ in $D$ which is $\geq$ the maximum norm of
$\mathcal G^*\uuu f$ in $D\uuu{2h}$. So
the last integral is majorised by
\[
\frac{24 \mathcal G(f)^2}{h}\cdot 
\iint\uuu{D\uuu{2h}}
\sin\,\phi\cdot rdr d\phi=
96\cdot  \mathcal G(f)^2\cdot h
\]
After a division with $h$ we get Theorem 5.14.
\medskip

\noindent
{\bf{5.15 Remark.}}
Since $|z|\geq 1/2$ holds in sectors $S\uuu h$ with $0<h<1/2$
we can remove the factor $|z|$ and hence  Theorem 5.14
shows that
if $\mathcal G\uuu f(z)$ is  bounded  in $D$
then we obtain a Carleson measure in $D$ defined by
\[
\mu\uuu f=
\log\,\frac{1}{|z|}\cdot
\bigl |\nabla(H\uuu f)\bigr |^2\tag{*}
\]
Moreover, its Carleson norm is estimated via Theorem 5.11 and
Theorem 5.24 by an absolute constant times
$|F|\uuu{\text{BMO}}$.


\bigskip

\centerline{\bf{6. The dual of $\mathfrak{Re}(H_0^1(T)$}}

\bigskip

\noindent
By the observations before Theorem 0.4 there remains to prove that if
$F\in\text{BMO}(T)$ then (*) holds in Theorem 0.4 for some
constant $C$. To obtain this we need some
preliminary results.
\medskip

\noindent
{\bf{6.1 Some integral formulas.}}
To simplify notations we set
\[ 
\int\uuu0^{2\pi}\, g(e^{i\theta})\cdot d\theta= \int\uuu T\, g\cdot d\theta
\]
for integrals over the unit circle.
Now follow some results which are left as exercises and  proved by
Green's formula.
\medskip

\noindent
{\bf{A. Exercise.}}
For every $C^2$\vvv function
$W$ in the closed unit disc with $W(0)=0$
we have
\[
\int\uuu T\, W\cdot d\theta=\iint\uuu D\, \log\,\frac{1}{|z|}\cdot \Delta(W)\cdot dxdy\tag{1}
\]
Next, if
 \[
W= |z|\cdot W\uuu 1\tag{2}
\]


\noindent
{\bf{B. Exercise.}}
Let $u,v$ is a pair of $C^2$\vvv functions which both are harmonic
in the open disc. Show that 
\[
\Delta(uv)=2\cdot \langle \nabla(u),\nabla(v)\rangle\tag{i}
\]
and use (A) to prove the equality
\[ 
\int\uuu T\, uv\cdot d\theta=\iint\uuu D\, \log\,\frac{1}{|z|}\cdot
\langle \nabla(u),\nabla(v)\rangle\cdot dxdy\tag{ii}
\]
\medskip

\noindent
{\bf{C. Exercise.}}
Let $f=u+iv$ be analytic in $D$. Verify that
\[
\Delta(|f|)=\frac{1}{|f|}\cdot |\nabla(u)|^2\tag{i}
\]
holds outside the zeros of $f$. Show also that
if 
\[ 
f=z\cdot g\tag{ii}
\] 
where $g$ is zero\vvv free in $D$ then
\[
\int\uuu T\, |f|\cdot d\theta=
\iint\uuu D\, \log\,\frac{1}{|z|}\cdot\frac{|\nabla(u)|^2}{|f|}\cdot dxdy\tag{iii}
\]

\noindent
Finally, let  $f$ be as in (ii) and $F$ a  real\vvv valued $C^2$\vvv function in
$D$. Show that
\[
\frac{1}{2}\int\uuu T\, u\cdot F\cdot d\theta=
\iint\uuu D\, \log\,\frac{1}{|z|}\cdot
\langle \nabla(u),\nabla H\uuu F\rangle\cdot dxdy\tag{iii}
\]






\bigskip

\centerline{\emph{6.2 Proof of Theorem 0.4}}
\bigskip

\noindent
Let $f\in H^1\uuu 0(T)$. Then one finds a Blaschke product
$B$ such that
\[
f(z)=z\cdot B(z)\cdot g(z)
\]
where $g$ is zero\vvv free in $D$.
It follows that
\[
2f=z(B+1)\cdot g+z(B\vvv 1)\cdot g=f\uuu 1+f\uuu 2
\]
where 
$||f\uuu\nu||\uuu 1\leq 2\cdot ||f||\uuu 1$ hold for each $\nu$.
Using this trick we conclude that
it suffices to estasblish 
Theorem 0.4 for $H^1(T)$\vvv functions
of the form
$f(z)= z\cdot g(z)$ with a
zero\vvv free function $g$ in $D$.
We write $f=u+iv$
and for each real\vvv valued $C^2$\vvv function $F(\theta)$ on $T$
we have by (iii) from Exercise C:

\[
\frac{1}{2}\cdot 
\int\uuu 0^{2\pi}\, F(\theta)\cdot u(\theta)\cdot d\theta=
\iint\uuu D\, \log\,\frac{1}{|z|}\cdot \langle \nabla(u),\nabla(H\uuu F)\rangle
\cdot dxdy\tag{1}
\]
Insert the factor $1=\sqrt{|f|}\cdot \frac{1}{\sqrt{|f|}}$
and apply the Cauchy\vvv Schwarz inequality which
estimates the absolute value of (i)
by

\[ 
J=\sqrt{
\iint\uuu D\, \log\,\frac{1}{|z|}\cdot \frac{|\nabla(u)|^2}{|f(z)|}
\cdot dxdy}\cdot
\sqrt{\iint\uuu D\, \log\,\frac{1}{|z|}\cdot |\nabla(H\uuu F)|^2\cdot |f(z)|}
\cdot dxdy\tag{2}
\]
The equality (iii) in Exercise C shows the first factor is equal to
$\sqrt{||f||\uuu 1}$.
In the second factor appears the density function 
$\log\,\frac{1}{|z|}\cdot |\nabla(H\uuu F)|^2$ in $D$.
\medskip

\noindent
Finally,  by
the  Remark in (5.15) 
the Carleson norm of the density
$\log\,\frac{1}{|z|}\cdot |\nabla(H\uuu F)|^2$
is bounded by an absolute constant
$C$ times the BMO\vvv norm of $F$.
Together with the result in XXX in Section XXX we get an absolute constant
$C$ such that the last factor in (2) above is bounded by
\[
C\cdot |F|\uuu{\text{BMO}}\cdot\sqrt{||f||\uuu 1}\tag{3}
\]
which finishes
the proof of
Theorem 0.4.












\bigskip

\centerline {\bf 7. A theorem by Gundy Silver}


\bigskip




\noindent
{\bf Introduction.}
Let $U(x)$ be in $L^1({\bf{R}})$ and construct its
harmonic extension  to the upper half plane:
\[
U(x+iy)=\frac{1}{\pi}\cdot \int\,\frac{y}{(x-t)^2+y^2}\cdot
U(t)\cdot dt
\]
\noindent
The  harmonic conjugate of $U(x+iy)$   is given by:
\[ 
V(x+iy)=\frac{y}{\pi}\int\,\frac
{U(t)\cdot dt}{(x-t)^2+y 2}\tag{0.1}
\]
\medskip


\noindent
Next, to each real  $x_0$  the Fatou sector in the upper half-plane is defined by
\[
\{x+iy)\quad\text{such that}\quad \,|x-x_0|\leq y\}\tag{0.2}
\]
and the maximal function $U^*$ over Fatou sectors 
is defined on the real $x$\vvv axis by
\[ 
U^*(x_0)=\max\, |U(x+iy)|\colon\quad\colon |x-x_0|\leq y\tag{0.3}
\]

\medskip

\noindent
In XXX we proved that if $V\in L^1(R)$
then $U^*(x)\in L^1({\bf{R}})$
and there exists
an absolute constant $C_0$ such that
\[ 
||U^*||_1\leq \int_{-\infty}^\infty\, \bigl(\,|U(x)|+|V(x)| \,\bigr)dx\tag{*}
\]


\noindent
A reverse inequality is due to
Burkholder, Gundy and Silverstein.

\medskip

\noindent {\bf Theorem 7.1.} \emph{One has the inequality}
\[ 
\int\,|V(x)|dx\leq  4\int\,U^*(x)dx
\]
\medskip


\noindent
{\bf Remark.}
Hence $U^*$ belongs to $L^1$ if and only if
the boundary value function $V(x)$ belongs to $L^1$.
The  original proof in [BGS] used 
probabilistic methods. Here we  give a proof
based upon
methods from [Feff-Stein].
Since we 
shall establish an \emph{a priori estimate}, it suffices to assume that
$U(x)$ from the start is a nice function. In particular
we may assume that both $U(x+iy)$ and $V(x+iy)$
have rapid decay when $y\to+\infty$ in the upper half-plane.
This assumption is used below to ensure that a certain complex line
integral is zero.

\bigskip



\centerline{\emph{Proof of Theorem 7.1}}
\medskip


\noindent
Given $\lambda>0$ we put
\[ 
J_\lambda=\{ x\colon U^*(x)>\lambda\}
\]
\medskip


\noindent
The closed complement
${\bf{R}}\setminus J_\lambda$ is denoted by $E$.
Let $\{(a_\nu,b_\nu)\}$ be the disjoint intervals of $J_\lambda$.
Construct the piecewise linear
$\Gamma$-curve
which stays on the real $x$-line on 
$E$ while it follows the two sides
of the triangle $T_\nu$
standing on $(a_\nu,b_\nu)$ for each $\nu$.
So  the  corner point of $T_\nu$ in the upper half-plane is:
\[
p_\nu=\frac{1}{2}(a_\nu+b_\nu)(1+i)
\]
Set $\partial T= \Gamma\setminus E$ and notice 
that the construction of Fatou sectors gives
\[ 
U^*(x)\leq \lambda\quad\colon\quad x\in T\tag {1}
\]

\noindent
In $\mathfrak{Im}(z)>0$
we have the
analytic function
$G(z)=(U+iV)^2$.
By hypothesis  $U$
$y\mapsto G(x+iy)$ decreases quite rapidly which gives
a vanishing complex line integral:
\[ 
\int_\Gamma\,G(z)dz=0
\] 



\noindent
Now $\Gamma$ is the union  of $E$
and the union of  of the broken lines which give the two sides
of the $T_\nu$-triangles.
Let $\partial T$ denote the union of these
broken lines. Since the complex differential $dz=dx+idy$
the real part of the complex line integral is zero which gives

\[
\int_E\, (U^2-V^2)\cdot dx+
\int_{\partial T}(U^2-V^2)\cdot dx-2\cdot 
\int_{\partial T}U\cdot Vdy\tag{2}
\]
On the sides of  the $T$-triangles
the slope is plus or minus $\pi/4$  and hence $|dy|=|dx|$
where $|dx|= dx$ is positive.
Hence the he inequality
$2ab\leq a^2+b^2$ for any pair of  non-negative numbers gives:
\[ 
2\cdot \bigl|\int_{\partial T}\,
 UV dy\,\bigr |\leq  \int_{\partial T}\, U^2\cdot dx+\int_{\partial T}\,V^2\cdot dx\tag{3}
\]


\noindent
Since (2) is zero we see that (3) and the triangle inequality
give:
\[
\int_E\, V^2\cdot dx\leq 
\int_E\, U^2\cdot dx+
2\cdot \int_{\partial  T}\, U^2\cdot dx
\tag{4}
\]

\medskip



\noindent
Next, put
\[
V^+_\lambda=\{x\,\colon |V(x)|>\lambda\}
\]
Then (4) gives:
\[
\mathfrak{m}(V^+_\lambda\cap E)\leq
\frac{1}{\lambda^2}\cdot
\int_E  V^2\cdot dx\leq
\frac{1}{\lambda^2}\cdot\int_E\, U^2\cdot dx+
\frac{2}{\lambda^2}\cdot\int_{\partial  T}\, U\cdot dx\tag{5}
\]


\medskip

\noindent
Next, 
Since the
integral  $\int_{T_\nu}dx=(b_\nu-a_\nu)$ for each $\nu$
and (1) holds we have
\[
\frac{2}{\lambda^2}\cdot
\int_{\partial T} U^2\cdot dx\leq 2\cdot \sum (b_\nu-a_\nu)=
2\cdot \mathfrak{m}(J_\lambda)\tag{6}
\]
Using the set-theoretic inclusion $V^+_\lambda\subset 
(V^+_\lambda
\cap E_\lambda)\,\cup
J_\lambda$ it follows after adding $\mathfrak{m}(J\uuu\lambda)$
on both sides in (5):
\[
\mathfrak{m}(V^+_\lambda)\leq 3\cdot\mathfrak{m}(J_\lambda)+
\frac{1}{\lambda^2}\cdot\int_E \,U^2\cdot dx\tag{6}
\]
Finally, $U\leq U^*$ holds on $E$
and since $E$ is the complement of $J\uuu\lambda$
we have $E=\{x\,\colon U^*(x)\leq\lambda\}$.
Now we apply general integral formulas which after integration over 
$\lambda\geq 0$
gives

\[
\int\,|V(x)\cdot dx=3\cdot \int\,U^*(x)\cdot dx+
\int\uuu 0^\infty\, \frac{1}{\lambda^2}\bigl [\int\uuu {(U^*\leq\lambda}\,(U^*)^2\cdot dx\,\bigr ]
\cdot d\lambda
\]
By the integral formula from XX the last term is equal to
$\int\,U^*(x)\cdot dx$ and Theorem 7.1 follows.




\newpage


\centerline{\bf{8. The Hardy space on ${\bf{R}}$}}


\medskip

\noindent
Consider an analytic function $F(z)$ in the upper half\vvv plane
whose boundary value function $F(x)$ on the real line is integrable.
This class of analytic functions in $\mathfrak{Im}\,z>0$
is denoted by $H^1({\bf{R}})$.
To each such $F$ we introduce the
non\vvv tangential maximal function
\[ 
F^*(x)=\max\uuu {z\in\mathcal F(x)}
\, |F(z)|\tag{*}
\]
where $\mathcal F(x)$ is the Fatou sector of points
$z=\xi+i\eta$ for which $|\xi\vvv x|\leq \eta$.
With these notations one has

\medskip

\noindent{\bf{8.1 Theorem.}}
\emph{There exists an absolute constant $C$ such that}
\[
\int\uuu{\vvv\infty}^\infty |F^*(x)|\cdot dx\leq
C\cdot \int\uuu{\vvv\infty}^\infty |F(x)|\cdot dx
\]
\medskip


\noindent
To prove this we shall first study harmonic functions
and reduce the proof of Theorem 8.1 to a certain $L^2$\vvv inequality.
To begin with, let  $u(x)$ is a real\vvv valued function on
the $x$\vvv axis such that the integral
\[ 
\int\uuu{\vvv\infty}^\infty\, \frac{|u(x)|}{1+x^2}\cdot dx<\infty
\]
The harmonic extension to the upper half\vvv plane becomes:

\[
U(x+iy)=
\frac{1}{\pi}\cdot \int\uuu{\vvv\infty}^\infty\, \frac{y}{(x\vvv t)^2+y^2}
\cdot u(t)\cdot dt
\]
The non\vvv tangential maximal function is defined by:
\[
 U^*(x)=\max\uuu {z\in\mathcal F(x)}
\, |U(z)|\tag{*}
\]
When $u(x)$ belongs to $L^2({\bf{R}})$
it turns out that one there is an  $L^2$\vvv inequality.

\bigskip

\noindent{\bf{8.2 Theorem.}}
\emph{There exists an absolute constant $C$ such that}

\[
\int\uuu{\vvv\infty}^\infty\,(U^*(x))^2\cdot dx\leq
\int\uuu{\vvv\infty}^\infty\,u^2(x)\cdot dx
\]
\emph{for every $L^2$\vvv function on the $x$\vvv axis.}
\medskip

\noindent
In 8.X below   we show how Theorem 8.2 gives Theorem 8.1.
The proof of Theorem 8.2 relies upon a point\vvv wise estimate
of $U$ via the Hardy\vvv Littlewood maximal function of
$u$.
Let us first consider a function $u(x)$ supported by
$x\geq 0$ such that the function
\[
t\mapsto \frac{1}{t}\, \int\uuu 0^t\, |u(x)|\cdot dx
\]
is bounded on $(0,+\infty)$.
Let $u^M(0)$ denote this supremum over $t$.
Then one has

\medskip

\noindent
{\bf{8.3 Proposition.}}
\emph{For each $z=x+iy$ in the upper half\vvv plane one has}

\[
|U(x+iy)|\leq (1+\frac{|x|}{2y})\cdot u^M(0)
\]
\medskip

\noindent
\emph{Proof.}
Since the absolute values  $|U(x+iy)|$
increase when $u$ is replaced by $|u|$ we may assume that
$u\geq 0$ from the start. Put
\[ 
\Phi(t)=
\int\uuu 0^t\, u(x)\cdot dx
\]
which yields a primitive of $u$ and
a partial integration gives
\[
U(x+iy)= \lim\uuu {A\to \infty}
\frac{1}{\pi}
\cdot\bigr| \frac{y}{(x\vvv t)^2+y^2}
\cdot  \Phi(t)\bigl|  \uuu 0^A+
\lim\uuu {A\to \infty}
\frac{2}{\pi}\cdot\int\uuu 0^A\, \frac{y(t\vvv x)}{((x\vvv t)^2+y^2)^2}
\cdot \Phi(t)\cdot dt
\]
With $(x,y)$ kept fixed the finiteness of $u^M(0)$ entails that
$t^{\vvv 2}\cdot \Phi (t)$ tends  to zero 
with $A$ and there remains

\[
U(x+iy)=
\frac{2}{\pi}\cdot \int\uuu 0^\infty\, \frac{y(t\vvv x)}{((x\vvv t)^2+y^2)^2}
\cdot \Phi(t)\cdot dt
\]
Now $\Phi(t)\leq u^M(0)\cdot t$
gives the inequality

\[
U(x+iy)=\frac{2 u^M(0)}{\pi}\cdot \int\uuu 0^\infty\, 
\frac{y(t\vvv x)\cdot t}{((x\vvv t)^2+y^2)^2}\cdot dt
\]
To estimate the integrand we notice that it is equal to
\[
\frac{y}{((x\vvv t)^2+y^2)}+
\frac{y(t\vvv x)x}{((x\vvv t)^2+y^2)^2}
\]
The Cauchy\vvv Schwarz inequality gives
\[
|\frac{2y(t\vvv x)x}{((x\vvv t)^2+y^2)^2}|\leq
\frac{|x|}{(x\vvv t)^2+y^2}
\]
It follows that

\[ 
|U(x+iy)|\leq \frac{2u^M(0)}{\pi}\cdot 
\int\uuu 0^\infty\, \frac{y}{(x\vvv t)^2+y^2)}+\frac{u^M(0)\cdot |x|}{\pi}
\cdot 
\int\uuu 0^\infty\, \frac{1}{(x\vvv t)^2+y^2}\cdot dt
\]
The last sum of integrals is obviously
majorised by
$u^M(0)(1+\frac{|x|}{2y})$ and Proposition XX is proved.
\medskip


\noindent
{\bf{8.4 General cae.}}
If no constraint is imposed on the support of $u$
it is written as $ u\uuu 1+u\uuu 2$ where $u\uuu 1$ is supported by
$x\leq 0$ and $u\uuu 2$ b $x\geq 0$.
Here we consider the  maximal function
\[
u^M(0)=
\max\uuu t\, \frac{1}{2t}\int\uuu{\vvv t}^t\, |u(x)|\cdot dx
\]
Exactly as above the reader may verify that 
\[ 
|U(x,y)|\leq u^M(0)(2+\frac{|x|}{y})\tag{i}
\]


\noindent
In the Fatou sector at $x=0$ we have $x\leq |y|$
and hence (i) gives
\[
U^*(0)\leq \leq 3\cdot u^M(0)
\]
After a translation with respect to $x$ a similar inequality holds.
More precisely, put
\[ 
u^M(x)= \max\uuu t\, \frac{1}{2t}\int\uuu{\vvv t}^t\, |u(x+s)|\cdot ds
\]
for every $x$, Then we have
\[
U^*(x)\leq 3\dot u^M(x)
\]
\medskip

\noindent
Now we apply the Hardy\vvv Littlewood inequality from XX
for the $L^2$\vvv case and obtain the conclusive result:

\medskip

\noindent
{\bf{8.5 Theorem.}} \emph{There exists an absolute constant $C$ such that}

\[ 
\int\uuu{\vvv\infty>}^\infty\, U^*(x)^2\cdot dx\leq
C\cdot \int\uuu{\vvv\infty>}^\infty\, u^2(x)\cdot dx
\]
\emph{for every $L^2$\vvv function $u$ on the real line.}
\bigskip

\noindent
{\bf{8.6 Proof of  Theorem 8.1}}
We use a factorisation via Blaschke products which enable us to
write

\[
 F(z)= B(z)\cdot g^2(z)
 \]
 where $g(z)$ is a zero\vvv free analytic function in
 the upper half\vvv plane.
 Since $|B(z)|\leq 1$ holds in $\mathfrak{Im}(z)>0$
 we have trivially
 \[
 F^*(x)\leq g^*(x)^2
 \]
On the real axis we have $|F(x)|= |g(x)|^2$ almost everywhere so the 
$L^1$\vvv norm of $F$ is equal to the $L^2$\vvv norm of $g$.
Next, with $g=U+iV$
we have a pair of harmonic functions and since
$|g|^2= U^2+V^2$ we can apply Theorem 8.5 to each of 
these harmonic functions
and at this stage we leave it to the reader to confirm
the assertion in Theorem 8.1
\bigskip

\centerline{\bf{8.7 Carleson measures }}
\bigskip

\noindent
Let $F(z)$ be in the Hardy space $H^1({\bf{R}})$.
If $\lambda>0$ we put
\[
J\uuu\lambda= \{F^*(x)>\lambda\}
\]
We assume that the set is non\vvv empty and hence this open set
is a union of disjoint intervals $\{(a\uuu k,b\uuu k)\}$.
To each interval we construct the triangle $T\uuu k$
with corners at the points
$a\uuu k,b\uuu k$ and $p\uuu k=
\frac{1}{2}(a\uuu k+b\uuu k)+ \frac{i}{2}(b\uuu k\vvv a\uuu k)$.
Put
\[ 
\Omega=\cup\, T\uuu k
\]


\noindent{\bf{Exercise.}}
Use the construction of Fatou sectors 
and the definition of $F^*$ to show that
\[ 
\{|F(z)|>\lambda\} \subset \Omega\tag{1}
\]

\noindent
Let us now consider a non\vvv negative 
Riesz measure $\mu$ in the upper half\vvv plane.
For the moment we assume that $\mu$ has compact support
and that  $F(z)$ extends to a continuos function on
the closed upper half\vvv plane
This is to ensure that various integrals exists but 
does not affect the final a priori
inequality in Theorem X below.
General formulas for distribution functions give:
\[
\int\,|F(z)|\cdot d\mu(z)=
\int\uuu 0^\infty\, \lambda\cdot \mu( \{|F(z)|>\lambda\})\cdot d\lambda\tag{2}
\]


\noindent
To profit upon (1) we impose a certain norm  on $\mu$.
To each $x$ and every $h$ we construct the triangle $T\uuu x(a)$
standing on the interval $(x\vvv a/2,x+a/2]$.

\medskip

\noindent
{\bf{8.8 Definition.}}
\emph{The Carleson norm of $\mu$ is defined as
smallest constant $C$ such that}
\[
\mu(T\uuu x(a)\leq C\cdot a
\]
\emph{hold for all pairs $x\in{\bf{R}}$ and $a>0$ and is
denoted by $\mathfrak{car}(\mu)$}.
\medskip

\noindent{\bf{8.9 Application.}}
Given $\mu$ with its Carleson norm
the inclusion (1)  gives

\[
\mu( \{|F(z)|>\lambda\})\leq \sum\, \mu(T\uuu k)\leq
\mathfrak{car}(\mu)\cdot \sum\, (b\uuu k\vvv a\uuu k)\tag{i}
\]

\noindent
The last sum is the Lebesgue measure of $\{F^*>\lambda\}$
and hence the right hand side in (i) is estimated above by
\[
\mathfrak{car}(\mu)\cdot \int\uuu 0^\infty\, \lambda\cdot 
\mathfrak{m}(\mu( \{F^*>\lambda\})
\cdot
d\lambda= 
\mathfrak{car}(\mu)\cdot 
\int\uuu{\vvv \infty}^\infty\, F^*(x)\cdot dx\tag{ii}
\]
Together with Theorem 8.1 we arrive at the conclusive result:

\bigskip

\noindent
{\bf{8.10 Theorem.}}
\emph{There exists an absolute constant $C$ such that}
\[ 
\int\, |F(z)|\cdot d\mu(z)\leq C\cdot
\mathfrak{car}(\mu)\cdot 
\int\uuu{\vvv \infty}^\infty\, |F(x)|\cdot dx
\]
\emph{hold for each $F\in H^1({\bf{R}})$ and every 
non\vvv negative Riesz measure $\mu$ in the upper half\vvv plane.}










\newpage

\centerline{\bf{9. BMO and radial norms of measures}}

\bigskip


\noindent
Theorem 0.4 together with the preceeding description of the dual space of
$\mathfrak{Re}\, H^1\uuu 0(T)$
implies that every BMO\vvv function $F$ can be written as a sum
\[
F=\phi+v^*\tag{i}
\]
where $\phi$ is bounded and $v^*$ is the harmonic conjugate of 
a bounded function.
However, this decomposition is not unique.
A \emph{constructive} procedure to find a pair $u,v$ in
for a given BMO\vvv function $F$
was given by P. Jones in [Jones].
See also the article [Carleson] from 1976.



 

\bigskip

\noindent
{\bf{9.1 Radial  norms on measures.}}
Let $D$ be the unit disc. An $L^1$-function $u(z)$ in
$D$ is radially bounded if there exists a constant $C$ such that
\[
\frac{1}{\pi}\cdot \iint_{S_h}\, |u(z)|\cdot dxdy\leq C\cdot h\tag{*}
\] 
for each sector
\[
S_h=\{z\colon \theta-h/2<\arg z\theta+h/2\}\quad\colon\quad h>0
\]

\medskip

\noindent
The smallest   $C$ for which (*) holds is denoted by $|u|^*$.
Notice that $|u|^*$ in general is strictly larger than
the $L^1$-norm over $D$ which
occurs when we take $h=\pi$ above.
If $u$ satisfies (*) we define a function $P_u$ on the unit circle by

\[
P_u(\theta)=
\frac{1}{\pi}\cdot \iint_D\, \frac{1-|z|^2}{|e^{i\theta}-z|^2}\cdot u(z)\cdot dxdxy
\]

\noindent
With these notations Fefferman proved:
\medskip


\noindent
{\bf{9.2 Theorem}} \emph{There exists an absolute constant $C$ such that}
\[
|P_u|_{\text{BMO}}\leq C\dot |u|^*
\]

\noindent
Thus, $u\mapsto P_u$ sends radially bounded
$L^1(D)$-functions to
$\text{BMO}(T)$.
The proof of Theorem 8.1  relies upon Theorem 0.4 and 
the following observation:

\medskip

\noindent
{\bf{9.3 Exercise.}}
Show that when $u$ is radially bounded and $H(z)$ is a harmonic function in
$D$ with continuous boundary values on $T$ then
\[
 \iint_D\, H(z)\cdot u(z)\cdot dxdy=
\int_0^{2\pi}\, H(e^{i\theta})\cdot P_u(\theta)\cdot d\theta
\]

\bigskip

\noindent
The following  result is also due to
Fefferman:

\medskip

\noindent
{\bf{9.4 Theorem.}}
\emph{Let $F(\theta)\in \text{BMO}(T)$. Then there exists a radially bounded
$L^1(D)$-function $u$ and some
$s(\theta)\in H^\infty(T)$ such that}
\[
 F(\theta)= s(\theta)+ P_u(\theta)
\]
For detailed proofs of the results above we refer to Chapter XX in [Koosis].









 
  








































































%\end{document}
