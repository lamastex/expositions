

\documentclass{amsart}

\usepackage[applemac]{inputenc}


\addtolength{\hoffset}{-12mm}
\addtolength{\textwidth}{22mm}
\addtolength{\voffset}{-10mm}
\addtolength{\textheight}{20mm}

\def\uuu{_}

\def\vvv{-}


 \begin{document}

\newpage





\centerline{\bf{H�rmander's  $L^2$-estimate
in dimension one }}

\bigskip


\noindent
Following original work by Lars H�rmander
we establish a result about the $\bar\partial$-operator  
in planar domains. Thus we restrict the attention to ${\bf{C}}$
where $z=x+iy$ is the complex coordinate. 

\medskip

\noindent
{\bf{Remark.}}
The full strength of  $L^2$-estimates
appears in dimension $n\geq 2$ where one works with
\emph{plurisubharmonic functions}  and 
impose  conditions on 
strictly pesudo-convex subsets of  ${\bf{C}}^n$ where one 
seeks solutions
of inhomogeneous $\bar\partial$-equations for 
differential forms of every bi-degree $(p,q)$ where $0\leq p,q\leq n$.
See H�rmander's text-boook in several complex variables for details.
\medskip


\noindent
The Cauchy-Riemann operator
sends a differentiable function $f$ into
\[
\bar\partial(f)= \frac{1}{2}(\partial_x(f)+i\cdot \partial_y(f))
\]
\medskip


\noindent
{\bf{The Hilbert space $\mathcal H_\phi(\Omega)$.}}
Let $\Omega$ be an open  set in  ${\bf{C}}$.
A real-valued continuous and non-negative function 
$\phi$ on $\Omega$ gives
the Hilbert space 
$\mathcal H_\phi$
whose elements are complex-valued Lebesgue measurable functions $f$ in $\Omega$ such that
\[ 
\int_\Omega\, |f|^2\cdot e^{-\phi}\, dxdy<\infty\tag{1}
\]
The square root  yields norm   denoted by
$||f||_{\phi}$. Let $\psi$ be another
continuous and non-negative function 
which gives the Hilbert space 
$\mathcal H_\psi $ where the norm of an element $g$ is denoted by $||g||_{\psi}$.
We  consider the
$\bar\partial$-operator
which sends
a function $f\in \mathcal H_\phi$  to
$\bar\partial(f)=df/d \bar z$
and study the 
equation
\[
\bar\partial(f)= g\quad\colon g\in H_\psi\tag{2}
\]
One  seeks  conditions for  the pair $(\phi,\psi)$ in order
that there exists a constant  $C$ such that
(2) has a solution $f$ for every $g$ where
\[
 ||f||_\phi\leq C\cdot ||g||_\psi\tag{3}
\]
Notice that  (2) does not have a unique solution since
$f$ can be replaced by $f+a(z)$ where $a$ is a holomorphic function
which belongs to
$\mathcal H_\phi$.
For example, non-uniquneness fails when
$\Omega$ is a bounded open set
and the function $e^{-\phi}$ is bounded in
$\Omega$. For then $f$ can be replaced by $f+p$ for
an arbitrary
polynomial $p(z)$.
We shall find a sufficient condition
in order that (2-3) above hold.
\medskip

\noindent
{\bf{H�rmander's condition.}}
The pair  $\phi,\psi$ satisfies the H�rmander condition if 
$\psi$ is a $C^2$-function and $\phi$ is at least a $C^1$-function,
and there  exists a
positive constant $c_0$ such that the following pointwise inequality holds in
$\Omega$:
\[ 
\Delta(\psi)\vvv 2\cdot |\nabla(\psi)|^2+
\psi\uuu x\phi\uuu x+\psi\uuu y\phi\uuu y\geq
2\cdot c_0^2\cdot e^{\psi(z)-\phi(z)}
\tag{*}
\]
where we have put $|\nabla(\psi)|^2=\psi\uuu x^2+\psi\uuu y^2$.
\medskip


\noindent
{\bf{Main Theorem.}}
\emph{If the pair $(\phi,\psi)$ satisfies (*) 
the equation
$\bar\partial(f)=g$ has a solution for every $g\in\mathcal H_\psi$ where}
\[
||f||_\phi\leq \frac{1}{c_0}\cdot ||q||_\psi
\]


\noindent
Before  the proof starts we recall
some facts sbout linear operators between
Hilbert spaces.
In general, let $\mathcal H_0$ and $\mathcal H_1$ be a pair of complex Hilbert spaces
and $T\colon\mathcal H_0\to \mathcal H_1$ is a densely defined linear operator.
Following Torsten Carlemsn's famous monograph about unbounded
operators on Hilbert spaces 
published by  Uppsala university in 1923, 
we recall the construction of an adjoint.
Namely, a  vector $y\in\mathcal H_1$ belongs to the domain 
of definition for the adjoint
operator $T^*$ if and only if there exists a constant $C$ such that
\[
\bigl|\langle Tx,y\rangle_1\bigr|\leq C\cdot |x|_0\quad\colon x\in\mathcal D(T)\tag{i}
\]
where $|x|_0$ is the norm of the vector $x$ taken in
$\mathcal H_0$, 
and in the left hand side
we  considered the hermitiain inner product on
$\mathcal H_1$.
Since $\mathcal D(T)$ is dense and Hilbert spaces are self-dual, each
$y$ for which (i) holds yields a unique vector
$T^*(y)\in\mathcal H_0$ such that
\[
\langle Tx,y\rangle_1=\langle x,T^*y\rangle_0\tag{ii}
\]
In general $\mathcal D(T^*)$ is not a dense subspace of
$\mathcal H_1$. But let us add this as an hypothesis on $T$.
Moreover,  assume that the two densely defined operators
$T$ and $T^*$ both are closed, i.e. their graphs are
closed in the product of the two Hilbert spaces.
\medskip

\noindent
{\bf{Exercise.}}
Suppose that both
$T$ and $T^*$ are closed with dense domains of definition.
Assume in addition that there exists a positive constant $c$ such that
\[
|T^*y|_0\geq |y|_1\quad\colon\quad y\in \mathcal D(T^*)
\]
Show that this implies that the range
$T^*(\mathcal D(T^*))$ is
a closed subspace of $\mathcal H_0$ which is equal
to the orthogonal complement of
the nullspace $\text{Ker}(T)$
Moreover, show that for
each $y\in\mathcal H_1$
we can find
$x\in\mathcal D(T)$ such that
\[
Tx=y\quad\&\quad  |x|_0\leq c^{-1}\cdot |y|_1
\] 
\bigskip






\centerline{\bf{Proof of the Main Theorem}}
\medskip


\noindent
Since $C_0^\infty(\Omega)$ is a dense subspace of $\mathcal H_\phi $
the linear operator $T\colon f\mapsto \bar\partial(f)$ 
from $\mathcal H_\phi$ to $\mathcal H_\psi$  is densely defined
and we leave as an exercise to the reader to check that
$T$ is closed. In fact, this relies upon a general fact about closedness
of operators defined by differential operators.
The reader may also check that Stokes Teorem entails that
test-functions in $\Omega$ belong to
$\mathcal D(T^*)$ and since
$C_0^\infty(\Omega)$ is dense in
the Hilbert space $\mathcal H_\psi$ the adjoint is also densely defined.
Let us then consider some
$g\in\mathcal D(T^*)$. For each $f\in C_0^\infty(\Omega)$
Stokes theorem gives
\[ 
\langle T(f),g\rangle=\int \bar\partial(f)\cdot \bar g\cdot e^{-\psi}\,dxdy=
-\int\, f\cdot 
\bigl[\bar\partial(\bar g)-\bar g\cdot \bar\partial(\psi)\bigr]\cdot
 e^{-\psi}\,dxdy\tag{i}
\]


\noindent
Since $\psi$ is real\vvv valued, 
$\bar\partial(\bar w)-\bar w\cdot \bar\partial(\psi)$
is equal to the complex conjugate of
$\partial(w)-w\cdot\partial(\psi)$. We conclude that (i) gives
\[
T^*(g)=-\bigl[\partial(g)-g\cdot\partial(\psi)\bigr]\cdot
 e^{\phi-\psi}\tag{ii}
\]
In particular $T^*$ is defined via a differential operator and has therefore
a closed graph.
Taking the squared $L^2$-norm in $\mathcal H_\phi$
we obtain
\[ 
||T^*(g)||^2_\phi=
\int\, |\partial(g)-g\cdot\partial(\psi)|^2\cdot
e^{\phi-2\psi}=
\]
\[
\int\, \bigl(|\partial(g)|^2+|g|^2\cdot|\partial(\psi)|^2\bigr)\cdot
e^{\phi-2\psi}-2\cdot\mathfrak{Re}\bigl (
\int\, \partial(g)\cdot\bar g\cdot\bar \partial(\psi)\cdot
e^{\phi-2\psi}\bigr)\tag{iii}
\]
By partial integration
the last integral in (iii) is equal to
\[
2\cdot\mathfrak{Re}\bigl(\,  \int\,g\cdot[\partial(\bar w)\cdot\bar \partial(\psi)+
\bar g\cdot \partial\bar\partial(\psi)-
2\bar w\cdot \bar \partial(\psi)\cdot \partial (\psi)+
\bar g\cdot \bar\partial(\psi)\cdot \partial(\phi)]\cdot e^{\phi-2\psi}\,\bigr)\tag{iv}
\]


\noindent
Next, the Cauchy-Schwarz inequality gives
\[
\bigl|2\cdot 
\int \, g\cdot  \partial(\bar g)\cdot \bar \partial(\psi)\cdot e^{\phi-2\psi}\,\bigr|\leq
\int\, \bigl(|\partial(g)|^2+|g|^2\cdot|\partial(\psi)|^2\bigr)\cdot
e^{\phi-2\psi}\tag{v}
\] 
Together (iii-v) give
\[
||T^*(g)||^2_\phi\geq 
2\cdot \mathfrak{Re}\,  \int\, |g|^2\cdot \bigl[\,\partial\bar\partial(\psi)\vvv
2\cdot \bar \partial(\psi)\cdot \partial (\psi)+
\bar\partial(\psi)\cdot \partial(\phi)\,\bigr ]\cdot e^{\phi-2\psi}=
\]
\[
2\cdot \mathfrak{Re}\,  \int\, |g|^2\cdot
\frac{1}{4}\bigl[\Delta(\psi)-2\cdot |\nabla(\psi)|^2+
\psi_x\phi\uuu x+\psi_y\phi_y\,\bigr]\, \cdot e^{\phi-2\psi}
\tag{vi}
\]
where the last equality follows since
$\phi$ and $\psi$ are real-valued.
Funally, since (4.1) is assumed
it follows that
\[
||T^*(g)||^2_\phi\geq c_0^2\cdot \int\, |g|^2\cdot
e^{\psi-\phi}\cdot e^{\phi-2\psi}=
c\uuu 0^2\cdot ||g||^2\uuu\psi\tag{vi}
\]
Now we apply the Exercise  above
and
the proof of the Main Theorem is finished.

 \medskip

\noindent
{\bf{Remark.}}
Let  $\Omega$ be  an open subset of a disc $\{|z|<r\}$ for somne $r<1$
which is centered at the origin.
Consider the function
\[
\phi(z)= \log\,( 1-|z|^2)= \log\, (1-x^2-y^2)
\]
Now we can take $\psi=\phi$ and H�rmander's condition (*) is valid.
To see this we notice that
\[
\Delta(\psi)=\frac{4}{(1-x^2-y^2)^2}\tag{i}
\]
\[
\psi_x^2+\psi_y^2= \frac{4x^2+4y^2}{(1-x^2-y^2)^2}\tag{ii}
\]
Since $\phi=\psi$ we see that the right hand side in (*)
becomes
\[
\frac{4}{(1-x^2-y^2)^2}-\frac{4x^2+4y^2}{(1-x^2-y^2)^2}\tag{iii}
\]
Inside the disc of radius $r<1$  we notice that (iii) is
\[
4\cdot \frac{1-r^2}{(1-r^2)^2}
\]
which can be taken as $c_0$ in the Main Theorem.
\medskip

\end{document}
