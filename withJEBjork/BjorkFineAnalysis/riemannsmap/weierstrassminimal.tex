\documentclass{amsart}
\usepackage[applemac]{inputenc}
\addtolength{\hoffset}{-12mm}
\addtolength{\textwidth}{22mm}
\addtolength{\voffset}{-10mm}
\addtolength{\textheight}{20mm}


\def\uuu{_}

\def\vvv{-}

\begin{document}


\centerline{\bf{On minimal surfaces.}}

\bigskip


\noindent
{\bf{Introduction.}}
To avoid possible confusion we  remark that the subsequent material deals with
an  isoperimetric problem with a fixed boundary curve. 
Considerably  more involved situations
 appear when the boundary value problem for minimal surfaces
do not   have entirely fixed Jordan curves, but are free to move on
prescribed manifolds.
Here one encounters the problems by Plateau and Douglas.
For an account about the general case we refer to Courant's article
\emph{The existence of minimal surfaces of given toplogical structure
under prescribed boundary conditions}. (Acta. Math. Vol 72 1940]) where the reader
also finds an extensive references to relevant literature.
\medskip

\noindent
From now on we discuss the problem when the boundary curve is fixed in
${\bf{R}}^3$.
Let $(x,y,z)$ be the coordinates
in ${\bf{R}}^3$.
Consider a rectifiable   closed Jordan curve
$C$ and denote by $\mathcal S(C)$ the family of surfaces 
which are bordered by $C$.
A surface $M$ in this family is minimal if
it has smallest possible area.
To find such a minimal surface
corresponds to a problem in the calculus of variation and  was 
studied by Weierstrass in a series of articles
starting from \emph{Untersuchungen �ber die Fl�chen
deren mittlere Kr�mmung �berall gleich null ist}  from 1866.
A revised version  appears in volume I of 
his collected work.
To begin with a local study shows that 
minimal surface $M$ in $\mathcal S(C)$ has vanishing 
imean curvature. Moreover, Weierstrass proved that
$M$ has no singular points and is simply connected.
These properties for a minimal surface
are natural and rather easy to prove.
After a more careful analysis, Weierstrass
proved that there exists a homeomorphic
parametrization
of $M$ above an open disc in the complex $u$-plane which can be
achieved via complex analytic functions, or as expressed by
Weierstrass in the introduction to [Wei]:
\emph{Ich habe mich mit der Theorie die Fl�chen, deren mittlere
Kr�mmung �berall gleich null ist, besonders auf dem grunde eingehender besch�ftigt, weil
sie, wie ich zeigen werde, auf das Innigste mit der Theorie der analytischen funktionen
einer komplexen Argumentz zusammenhh�ngt.}
Or shortly phrased: 
\emph{The theory about minimal surfaces
is closely linked to the theory of  analytic functions in one complex variable.}
\medskip


\noindent
{\bf{The isoperimetric inequality.}}
Using Weierstrass'  description of minimal surfaces 
the following was proved
by
Carleman in the article
\emph{Zur Theorie der Minimalfl�chen} in 1920:

\bigskip

\noindent
{\bf{Theorem 1.}}
\emph{For every rectifiable simple closed curve
$C$ the area $A$ of the minimal surface
in $\mathcal S(C)$ satisfies the inequality}
\[ 
A\leq \frac {\ell(C)^2}{4\pi}
\]
where $\ell(C)$ is the arc-length of $C$.
\medskip

\noindent
{\bf{Remark.}}
For historic reasons one may wonder why this result was not already
discovered by Weierstrass. The reason might be that 
analytic function theory was not   developed enough in 1890.
Carleman's proof relies upon 
the Jensen-Blascke factorisation
of analytic functions which was not know prior to 1900.
Another obstacle was the discovery by Hermann Schwarz that
the minimal surface in the family $\mathcal S(C)$
is not
determined by vanishing mean curvarute alone. See
Volume II, page 264 and  151-167 in the collected work
of Hermann Schwarz for this "ugly phenomenon" which was one reason
why Weierstrass  paid much attention to existence problems in
the calculus of variation.
As remarked by Carleman at the end of his article, the
alternative proof by Blaschke which was given after
[Carleman[ had been published, is restricted to
a special class of minimal surfaces.
The case when $C$ is piecewise linear and  consists of
$n$ many line segements $L_1,\ldots ,L_n$ means
in the words of Weierstrass  that one regards the following problem:
\emph{Es soll ein einfach zusamenh�ngenden Minimalfl�chenst�ck
$M$ analytish bestimmt werden, dess vorgeschiebinen
begrenzung $C$
aus $n$ geradlinigen strecken  bestecht, welche eine einfache,
geschlossene, nicht verknotete Linie bilden.}
In  [Weierstrass] appears a  far reaching study of this problem. 
The main result shows that
$M$ is determined via
a speciai pair of analytic functions $G(u)$ and $H(u)$ defined in
the lower half-plane $\mathfrak{Im}u<0$
for which the three functions defined by

\newpage

\[
\phi_1(u)=\text{det}\,
\begin{pmatrix} G(u)&H'(u)\\
G'(u)&H'(u)\end{pmatrix}
\]
\[
\phi_2(u)= \text{det}\,
\begin{pmatrix} G(u)&H'(u)\\
G''(u)&H''(u)\end{pmatrix}
\]
\[
\phi_3(u)= \text{det}\,
\begin{pmatrix} G'(u)&H'(u)\\
G''(u)&H''(u)\end{pmatrix}
\]
become rational functions of $u$.
Moreover, [ibid]  exhibits 
second order differential equations of the Fuchsian type 
satisfied by the rational functions which appear above and the position of their
poles are described in terms of the 
geometric configuration
of $C$. It would lead us too far to
enter
the  material in [Weierstrass] so its rich contents is left
to the interested reader for further  studies.
\medskip


\noindent
{\bf{The planar case.}}
If $C$ is a simple closed curve in the complex $z$-plane
the isoperimetric inequality follows easily via analytic function theory.
Namely, let $M$ be the Jordan domain bordered by $C$.
By Riemann's theorem there exists 
a conformal mapping $\phi\colon\, D\to M$ and we have
\[ 
\ell(C)= \int_0^{2\pi}\ |\phi'(e^{i\theta})|\, d\theta
\quad \colon\quad
A=\text{area}(M)=
\iint_D\, |\phi'(z)|~2\, dxdy
\]
We leave as an exercise to verify the isoperimetric
inequality in
Theorem 1 and that equality holds if and only if the complex derivative satisfies
\[ 
\phi'(z)=\frac{a}{(1-qz)^2}
\] 
for a pair of constants $a,b$. This means that $\phi$ is
is a M�bius transform and hence $C$ must be a circle, i.e.
equality in Theorem 1 for a planar curve holds
if and only if $C$ borders a disc,








\bigskip

\centerline{\emph{B. Proof of Theorem 1.}}
\medskip




\noindent
The crucial step in the proof relies upon the following:
\medskip

\noindent
{\bf{B.1 Proposition.}}
\emph{Let $M$ be a minimal surface in $\mathcal S(C)$.
Then there exists an analytic function $F(u)$ in the open 
unit disc such that
points $(x,y,z)\in M$ are given by the equations:}
\[ 
x=\mathfrak{Re} \int\, (1-u^2) F(u)\,du
\quad\colon\quad
y=\mathfrak{Re}\int \, i(1+u^2)F(u)\,du\quad\colon\quad
z=\mathfrak{Re}\int \, 2F(u)\, du
\]
\medskip

\noindent
The proof is  quite involved and occupies
more than five pages in [Weierstrass].
Let us at least indicate some
steps in Weierstrass' constructions which lead to
Proposition B.1. To begin with
the one finds a planar domain
$\Sigma$ with real coordinates
$(p,q)$ and a diffeomorphism
between
$M$ and $\Sigma$ which is conformal, i.e. 
$M$ is defined by the equations
\[ 
x=x(p,q)\quad\colon
y=y(p,q)\quad\colon z=z(p,q)
\]
where the 
vectors ${\bf{v}}=
(\frac{\partial x}{\partial p}, 
(\frac{\partial y}{\partial p}, (\frac{\partial x}{\partial p},)$
and
${\bf{w}}=(\frac{\partial x}{\partial q}, (\frac{\partial y}{\partial q}, (\frac{\partial z}{\partial q})$
are pairwise orthogonal unit vectors.
Since
the mean curvature of $M$ vanishes the three functions in (1) are harmonic, i.e.
\[
\Delta(x)= \frac{\partial^2 x}{\partial p^2}+\frac{\partial ^2x}{\partial q^2}=0\tag{i}
\]
and similarly for $y$ and $z$.
In fact, this  follow from classical calculus of variation
and basic differential geometry on surfaces which
is exposed in many text-books such as [Darbroux].
The harmonic functions above are real parts of analytic functions
which yields a triple $f,g,h$ in $\mathcal O(\Sigma)$
such that
\[ 
x= \mathfrak {Re}\, f(u)
\]
The orthogonality
of
the vectors ${\bf{v}}$ and ${\bf{w}}$ above entails via the
Cauchy Riemann equations that
\[
(f'(u))^2+(g'(u))^2+(h'(u))^2=0
\]


\medskip

\noindent
Starting from this, Weierstrass
used  sterographic projections and Riemann's conformal mapping theorem
to construct
an analytic function
$F(u)$ which gives the equations in Proposition B.1.
Admitting this one  we can prove the following:
 

\medskip

\noindent
{\bf{B.2 Propostion.}} \emph{One has the formulas}
\[
\text{area}(M)=
\iint_D\, (1+|u|^2)^2\cdot |F(u)|^2\, d\xi d\eta\quad\colon\quad 
\ell(C)=2\cdot \int_0^{2\pi}\,  |F(e^{i\theta})|\, d\theta
\]
\medskip

\noindent
\emph{Proof.}
With $u=\alpha+i\beta$ this amounts to show that
\[ 
dx^2+dy^2+dz^2=
(1+|u|^2)|F(u)|^2\cdot (d\alpha^2+d\beta^2)\tag{i}
\]

\noindent
To prove  (i) we consider some point $u\in D$.
Set $F(u)= |F(u)|\cdot e^{i\theta}$
and $u= se^{i\alpha}$. With
$du=d\alpha$ real we have
\[
dx=\mathfrak{Re}(1-u^2)F(u))\cdot d\alpha=
|F(u)|\cdot  \bigl
(\cos\theta-|u|^2\cos\theta\cdot \cos 2\alpha
-|u|^2\sin\theta\cdot \sin 2\alpha\bigr)\cdot d\alpha
\]
By trigonometric forulas it follows that

\[ 
(dx)^2=|F(u)|\cdot \bigl[\cos^2\theta+|u|^4\cos^2(2\alpha-\theta)-
2|u|^2\cos\theta\cdot \cos(2\alpha-\theta)\,\bigr]\cdot (d\alpha)^2\tag{i}
\]
\medskip


\noindent
In the same way we find that
\[
(dy)^2 =|F(u)|^2 \cdot  \bigl[\sin^2 \theta+|u|^4\sin^2(2\alpha-\theta)+
2|u|^2\sin\theta\cdot \sin(2\alpha-\theta)\bigr]\cdot d\alpha\tag{ii}
\]

\[ 
(dz)^2=4|F(u)|^2\cdot |u|^2\bigl(\cos^2(\theta-\alpha)]\cdot
(d\alpha)^2\tag{iii}
\]
\medskip

\noindent
Adding (i-ii) we get
\[
(dx)^2+(dy)^2= |F(u)|^2\cdot \bigl[1+|u|^4-
2\cdot|u|^2\cos(2\theta-2\alpha)\bigr]\cdot (d\alpha)^2
\]
Finally, the trigonometric formula

\[
4 \cos^2\phi=2-2\cos\,2\phi
\]
shows that
\[
(dx)^2+(dy)^2+(dz)^2 = |F(u)|^2\cdot (1+|u|^2)^2\cdot (d\alpha)^2
\]
\medskip

\noindent
The same hold when $u=id\beta$ and this infinitesmal
equations give 
Proposition B.2.





\bigskip



\centerline {\bf{Proof of Theorem 0.1}}

\bigskip


\noindent
With $F$ as above we put
\[
f_1(u) =F(u)u^2\quad\text{and}\quad  f_2(u)=F(u)\implies
\]
\[
A=\text{area}(M)=
\iint_D\,\bigl[|f_1(u)|^2+|f_2(u)|^2\,\bigr]\, d\xi d\eta+
2\cdot \iint_D\,|f_1(u)|\cdot |f_2(u)|\, d\xi d\eta\tag{i}
\]
where the implication follows from Proposition B.2.
Since $|f_1|=|f_2|= |F|$ holds on the unit circle we also get
\[
\ell(C)^2=
\bigl[ \int_0^{2\pi}\, |f_1(e^{i\theta})|\, d\theta\bigr ]^2+
\bigl[ \int_0^{2\pi}\, |f_2(e^{i\theta})|\, d\theta\bigr ]^2+
2\cdot  \int_0^{2\pi}\, |f_1(e^{i\theta})|\, d\theta\cdot
\int_0^{2\pi}\, |f_2(e^{i\theta})|\, d\theta\tag{ii}
\]
\medskip


\noindent
At this stage we shall need:

\medskip

\noindent
{\bf {B.3 Lemma.}} \emph{For each pair of analytic functions $g,h$ in the unit disc
one has}
\[
\iint_D\,\bigl[g(u)|\cdot |h(u)|\, d\xi d\eta\leq
\frac{1}{4\pi}\cdot 
\int_0^{2\pi}\, |g(e^{i\theta})|\, d\theta\cdot 
\int_0^{2\pi}\, |h(e^{i\theta})|\, d\theta
\]
\medskip

\noindent
Let us first notice that  Lemma B.3 applied to the pairs $g=h=f_1$,
$g=h=f_2$ and the pair $g=f_1$ and $h=f_2$
together with (i-ii) give Theorem 1.
So there remains  to prove Lemma B.3.
We can write
\[ 
g=B_1\cdot g^*\quad\colon\quad h= B_2\cdot h^*
\]


\noindent
where $B_1,B_2$ are Blaschke products and the analytic functions
$g^*$ and $h^*$ are zero free in the unit disc.
Since
$|B_1|= |B_2|=1$ hold on the unit circle it suffices to prove
Lemma B.3 for the pair $g^*,h^*$,  i.e. we may assume that both $g$ and $h$
are zero-free.
Then they posses square roots so we
can find analytic functions $G,H$ in the unit disc where
\[
g=G^2\quad\colon\quad h=H^2
\]
Consider the Taylor series
\[
G(z)= \sum\, a_ku^k\quad\colon\quad H(z)= \sum\, b_ku^k
\]
Now $GH= \sum c_ku^k$
where
\[ 
c_k= a_0b_k+\ldots+a_kb_0\tag{i}
\]


\noindent
Using polar coordintes to perform double integrals it follows that

\[
\iint_D\,|G^2(u)|\cdot |H^2(u)|\, d\xi d\eta=
\pi\cdot \sum_{k=0}^\infty\, \frac{|c_k|^2}{k+1}
\]
At the same time one has
\[
\int_0^{2\pi}\, |G^2(e^{i\theta})|\, d\theta= 2\pi\cdot
\sum _{k=0}^\infty\, 
|a_k|^2
\]
with a similar formula for the integral of $H^2$.
Hence Lemma B.2 follows if we have proved the inequality
\[
\sum_{k=0}^\infty\, \frac{|c_k|^2}{k+1}\leq 
\sum _{k=0}^\infty\, 
|a_k|^2\cdot 
\sum _{k=0}^\infty\, 
|b_k|^2\tag{ii}
\]
To get (ii) we use (i) which for every $k$ gives:
\[
|c_k|^2\leq (|a_0||b_k|+\ldots+|a_k||b_0|)^2\leq
(k+1)\cdot (|a_0|^2|b_k|^2+\ldots+|a_k|^2|b_0|^2)
\]
Finally, a  summation over $k$ entails (ii) and Lemma B.3 is proved.





















\end{document}
