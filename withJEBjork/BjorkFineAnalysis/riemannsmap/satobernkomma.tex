\documentclass[12pt]{amsart}


\usepackage[applemac]{inputenc}


\addtolength{\hoffset}{-12mm}
\addtolength{\textwidth}{22mm}
\addtolength{\voffset}{-10mm}
\addtolength{\textheight}{20mm}

\def\uuu{_}

\def\vvv{-}

\begin{document}




\centerline{\bf{Comments to Bernstein-Sato polynomials, central  characters
and stratification.}}
\bigskip

\noindent
The authors deal with the algebraic case but let me
formulate
the set-up in the analytic context.
Let $X$ be some $n$-dimensional complex manifold  and
$f\in\mathcal O(X)$. Set $T=f^{-1}(0)$ where
$f$ in general can have multiple factors, i.e. the principal ideal 
$(f)$ can be strictly larger than the defining ideal $\mathcal I_T$.
Shrinking $X$ to a suitable neighbrhood of $T$ we assume
that
the 1-form
$df\neq 0$ outside $T$.
Next, let $\mathcal D_X$ be the sheaf of holomorphic differential operators on
$X$.
Let $s$ be a new parameter which yields the left $\mathcal D_X$-module
\[
\mathfrak{N}=\mathcal O_X[f^{-1},s]\cdot f^s
\]
where sections in
$\mathcal O_X[f^{-1},s]$ are $s$-polynomials with coefficients in
the sheaf
$\mathcal O_X[*T]$ of meromorphic functions with poles
contained in $T$.
The left $\mathcal D_X$-module structure on $\mathfrak{N}$ is determined via
the
rule
\[
\delta(f^s)= \delta(f)^{-1}s\cdot f^s\quad\colon \delta\in \mathfrak{g}_X
\]
where $\mathfrak{g}_X$ is the sheaf of holomorphic vector fields regarded as first
order holomorphic differential operators.
Following Kashiwara's work in his article on $b$-functiones
(Inventiones 1975) 
one introduces the $\mathcal D_X$-submodule
\[
\mathcal N=
\mathcal D_X[s]\cdot f^s
\]
Multiplication with $s$ yields a left
$\mathcal D_X$-linerar map on $\mathcal N$.
A crucial  result from [Ka:1] is that
$\mathcal N$ is a coherent $\mathcal D_X$-module which is subholonomic, i.e.
its singular spectrum $\text{SS}(\mathcal N)$
has dimension $d_X+1$.
Let us remark that this follows easily since
$f$ belongs to the integral closure of
the $\mathcal O_X$-ideal  generated
by its first order derivatives.
Next, $\mathcal N$ contains the $\mathcal D_X$-submodule
\[
\mathcal N_*=\mathcal D_X[s]\cdot f^{s+1}
\]
Since $s$ is a parameter the
$\mathcal D_X$-modules $\mathcal N$ and $\mathcal N_*$ are isomorphic.
From this it follows that the quotient
module
\[
\mathcal M=\frac{\mathcal N}{\mathcal N_*}
\] 
is holonomic.
Multiplication  with
$s$ is a $\mathcal D_X$-linear map on $\mathcal M$ and
holonomicity entails that
it has locally minimal polynomials.
Working locally around a point $p\in X$ where $X$ is taken as
a polydisc centered at $p$
we find a unique minimal polynomial
$b(s)$ such that
$b(s)\mathcal M=0$ holds in $X$.
In [Ka:1] desingularisation was used to prove that
the roots of the $b$-function are strictly negative rational numbers
where eventual multiple zeros can occur.
If
\[
b(s)=\prod\, (s+q_\nu)
\] 
where eventual multipe zeros are repeated we can consider
subproducts
\[
\beta(s)= \prod\, (s+q_j)
\]
where $\{q_j\}$ is a proper subfamily of the $b$-roots.
Now there exists the analytic set
\[
V_\beta=\text{Supp}(\beta(s)\mathcal M))\tag{*}
\]
From this collection of analytic subsets of $X$ one gets a stratification
where the minimal stalkwise defined $b$-function for
$f^s$ is constant over each stratum.
This is the topic treated in the abstract by the authors.
Of course, it has been considered
by many authors and
certainly deserves attention since
roots of the $b$-functions are closely related
to
delicate topological properties of
the hypersurface
$T$ and also to asymptotic expansion of current-valued functions such as
\[
s\mapsto \int_{f=s}
\] 
studied in great detail by Barlet.
Whether it is really possible to exhibit
explicit stratifications for a given
$f$ is not so clear to me, i.e. even with the aid of computer
algebra it appears
to be a hard problem
which only can be sttled in special cases. The authors give however a nice example 
for a polynomial in 3 variabes.
A special case which was studied at an early stage 
and was suggested by Frederic Pham  in 1978
arises when we start with
$f$ and assume that
there exists a holomorphic vector field
$\rho$ such that
\[
\rho(f)=f
\]
If we know $f$ and $\rho$ the
stratification where the minmal $b$-function is constant can 
be
expressed as follows:
In $\mathcal D_X$ where
$X$ is the $n$-dimensional polydisc with coordinstes
$x_1,\ldots,x_n$ we have the $n$-tuple of
derivation operators
$\{\partial_i\}$.
Let $\mathcal L$ be the left ideal generated
by the $n$-tuple of first order differential operators
\[
f\cdot \partial_i-\partial_i(f)\rho\quad\colon 1\leq i\leq n\tag{**}
\]
Then (*) corresond to the inclusions
\[
\beta(\rho)\in \mathcal L+\mathcal D_Xf
\]
where the last term is the left principal ideal generated by $f$.
\medskip


\noindent
{\bf{Remark.}}
Above we recalled Kasiwara's studies of $b$-functions for a single holomorphic function.
The interested reader should also consult his plenary talk
about $b$-functions from the IMU-cogress at Helsinki in 1978
for futher interesting comments about $b$-functions.
More generally one associates $b$-functions to pairs
$(\mathcal M,f)$ where
$\mathcal M$ is a regular holonomic
$\mathcal D_X$-module and $f$ is a non-zero divsor on
$\mathcal M$, i.e. $\mathcal M$ does not contain
non-vanishing section supported by the hypersurface
$T$.
My book \emph{Analytic $\mathcal D$-modules} expose
the major results in this situation where
similar strafications as for a single
holomorphic function are  achieved.
\medskip


\noindent
{\bf{Concluding remarks}}
As already said the proposed topics by the authors certainly  deserve
attention.
But their  abstract does not make it clear to what extent
"computational progess" has been achived
to illustrate and eventually also consolidate the general theory which
was established  a long time ago, foremost  by
Kashiwara., and in the algebraic case one should  also give
tribute to Bernsteins's work.
In  this connection I also think 
it woud be interesting to study the specific cases
by Sato where $b$-functions appear in a quite explicit way
in his work on prehomogenous spaces from 1960.
For readers more familiar with analysis
one should mention that
the issue in the abstract is related to  natural and important
problems concerning meromorphic extensions of distribution-valued functions.
With $f$ as above there exists for every positive integer $m$
the $\mathfrak{Db}(X)$-valued function
\[
\mu_{m\lambda}(\phi)= \int_X\, f^{-m}\cdot |f|^{2\lambda}\cdot \phi\tag{*}
\] 
where $\phi$ are test-forms on $X$ of maximal bi-degree
$(n,n)$.
The local existence of $b$-functions imply that
(*) extends to a meromorphi function in the whole complex
$\lambda$-plane whose poles are confined to a finite union of arithmetic progressions
of the form
$\{-q-\nu\quad\colon \nu\geq 0\}$ where $q$ are positive rational numbers
abnd this inclusion of poles hold for
every $m\geq 0$ above.
Suppose for example thaat $0<q<1$ is a rational number
and poles coccur at $-q-\nu$ fora non.empty set 
of non-negative integers $\nu$.
The resulting polar distributions
which appear in Laurent expansions
of $\mu_{m\lambda}$ at such poles
are of interest in many applications to  PDE-theory.
Recall here a theorem due to Barlet which asserts that poles of
$\mu_{m,\lambda}$  give an \emph{effective contribution} to
roots of the $b$-function. So if one for example
is content to analyse stratifications where the $b$-function is constant
up to multiple zeros and shifts by integers, the support of
such distributions would serve as a stratification.
This remark is given to illustrate the great complexity - but also the interest - of the topics
proposed by  the abstract.
\medskip

\noindent
As an \emph{overall evaluation}  I suggest that the
abstract should be ranked as "weakly accepted" though I would add
that it may be regarded to be rather close to the higher evaluation "accept".

\bigskip


\centerline{Jan-Erik Bj�rk}

\bigskip

Dep. of mathematics Stocholm University.
e-mail:  jeb@math.su.se






















\end{document}