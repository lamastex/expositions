
\documentclass{amsart}
\usepackage[applemac]{inputenc}


\addtolength{\hoffset}{-12mm}
\addtolength{\textwidth}{22mm}
\addtolength{\voffset}{-10mm}
\addtolength{\textheight}{20mm}

\def\uuu{_}


\def\vvv{-}

\begin{document}





\centerline{\bf{� A. Eigenvalues and eigenfunctions for the Laplace operator
in ${\bf{R}}^2$}}.
\bigskip


\noindent
Theorem 1 below  was  presented by Carleman at the
Scandinavian Congress in mathematics held in Stockholm 1934:
In
${\bf{R}}^2$ we 
consider a bounded Dirichlet regular domain
$\Omega$, i.e. every $f\in C^0(\partial\Omega)$
has a harmonic extension to $\Omega$.
A wellknown  
fact which goes back to original work by Dirichlet
gives the following: There
exists the Greens' function
\[
G(p,q)= \log\,\frac{1}{|p-q|}-H(p,q)\tag{*}
\]
where $H(p,q)= H(q,p)$ is continuous in
the product set
$\overline{\Omega}\times\overline{\Omega}$.
Moreover, $H(p,q)= H(q,p)$ is symmetric and when
$q\in\Omega$ is fixed, then
$p\mapsto H(p,q)$ is harmonic in
$\Omega$ and 
\[ 
H(p,q)= \log\,\frac{1}{|p-q|}\quad\colon\, p\in \partial\Omega
\]
This means that $p\mapsto  G(p,q)$ vanishes on the boundary.
Next, (*) shows  that $p\mapsto G(p,q)$ is superharmonic and
and the minimum principe for superharmonic functions
plus symmetry entail that 
\[
G(p,q)>0
\]
hold in $\Omega\times\Omega$. 
Next, it is obvious that
\[
 \iint_{\Omega\times\Omega}\,|G(p,q)|^2\, dpdq<\infty
 \]
Hence the linear operator on the Hilbert space $L^2(\Omega)$ defined by
the symmetric kernel $G(p,q)$
is a Hilbert-Schmidt operator 
and therefore a compact operator.
Since the kernel is symmetric and positive the eigenvalues
are positive, General Hilbert space theory
applied to the symmetric $G$-kernel gives a 
a sequence of pairwise orthogonal functions
$\{\phi_n\}$
whose $L^2$-norms are one and
\[
\int_\Omega\, G(p,q)\phi_n(q)\, dq=2\pi\cdot \mu_n\cdot \phi_n(p)\tag{1}
\]
where
$\{\mu_n\}$
is a non-increasing sequence of positive eigenvalues which tend to zero.
Next, we apply the Laplace operator on both sides.
Recall that
\[
\Delta(\log\, \frac{1}{|z|})=-2\pi\cdot \delta_0
\]
where $\delta_0$ is the Dirac measure.
It follows that 
the Laplacian of the left hand side in (1)
becomes
$-2\pi\cdot \phi(p)$ and hence (1) gives the equation
\[
\Delta(\phi_n)(p)+\frac{1}{\mu_n}\cdot \phi_n(p)=0
\]
We prefer to use $\lambda_n=\frac{1}{\mu_n}$. Then
$\{\lambda_n\}$ is a non-decreasing sequence of real numbers
which tends to $+\infty$.
Since the kernel $G(p,q)$ is positive it follows, again   by general 
Hilbert space theory, 
that $\{\phi_n\}$ is an orthonormal basis in
$L^2(\Omega)$, i.e. each $L^2$-function $f$ has an expansion
\[
f=\sum\, a_n\cdot \phi_n\quad\colon\,\,
a_n= \int_\Omega\, f_n(p)\cdot\overline{\phi_n}(p)\, dp\tag{2}
\]
\medskip

\noindent
{\bf{Exercise.}}
Verify from the above   that each $\phi$-function is  a continuous function
in $\Omega$ whose boundary  values on
$\partial\Omega$ are zero.
Show also that
\[
G(p,q)=
\sum_{n=1}^\infty\, \frac{\phi_n(p)\cdot \phi_n(q)}{\lambda_n}\tag{0.1}
\]
where the right hand side is a convergent series when $p\neq q$.
\bigskip


\noindent
Notice that $G(p,p)=+\infty$
so the series
above with $p=q$ is divergent.
However, there exists a limit when 
we employ  larger denominators.
\medskip


\noindent
{\bf{Main Theorem.}} \emph{For 
$p\in\Omega$�one has the limit formula}
\[ 
\lim_{N\to\infty}\, \lambda_N^{-1}\cdot \sum_{n=1}^{n=N}\, \phi^2_n(p)= \frac{1}{4\pi}\tag{*}
\]
\medskip

\noindent
{\bf{Remark.}} Carleman was inspired by an earlier result due to
H. Weyl which asserts that
the set of eigenvalues satisfy the asymptotic formula
\[
\lim_{N\to \infty}\, \frac{\lambda_N}{N}=\frac{\text{Area}(\Omega)}{4\pi}
\]
Notice that Weyl's asymptotic formula together with (*) gives
\[
\lim_{N\to \infty}\,N^{-1}\cdot \sum_{n=1}^{n=N}\, \phi^2_n(p)=xxxx
\]
The notable point is that this asymptotic limit is the same for
\emph{every} point $p\in\Omega$.
\medskip


\noindent
To get the main Theorem we proceed as follows.
First, since  $\mathcal G$ is a Hilbert-Schmidt operator a  result due to Schur
gives
\[
\sum\, \lambda_n^{-2}<\infty \tag{i}
\]
Let us also remark that since
each $\phi_n$ is harmonic we have
the mean-value equality
\[
\phi_n(p)= \frac{1}{\pi r^2}\cdot \int_{D_p(r)}\, \phi(q)\,dq
\] 
where $D_p(r)$ is the disc of radius $r$ centered at $p$ and
$r$ is chosen so small that
the disc stays in $\Omega$.
Since the $L^2$-norms of the $\phi$-functions are equal to one,
the Cauchy-Schwarz inequality  gives a constant $C$ such that
\[
|\phi_n(p)|\leq C\quad\colon \, n=1,2,\ldots\tag{ii}
\]
Now (i-ii) entail that 
that  the  Dirichlet series
\[
\Phi_p(s)=\sum_{n=1}^\infty \frac{\phi_n^2(p)}{\lambda_n^s}
\]
is an analytic function of the complex variable $s$
in the half-plane
$\mathfrak{Re}\,  s>2$.
With these notations we shall prove:
\medskip

\noindent
{\bf{Theorem 2.}}
\emph{For each $p\in\Omega$ there exists an entire function
$\Psi_p(s)$ such that}
\[
\Phi_p(s)=\Psi_p(s)+\frac{1}{4\pi(s-1)}
\]
\medskip

\noindent
{\bf{Remark.}}
In � xx we explain how
Theorem 2 gives
Theorem 1 from   Ikehara's limit formula.
So the main task is to establish Theorem 2.
The proof of Theorem 2 employs analytic function theory and is inspired by Riemann's
work on the $\zeta$-function.
One can establish more general results, where $\Delta$ is replaced by a higher order
elliptic  operator with constant coefficients in ${\bf{R}}^n$ where $n\geq 3$ can hold.
and the reader is invited to
continue this analysis which does not seem to be covered in the literature.
In the cited article  such extensions are pointed out by Carleman

\medskip

\noindent
\emph{Remarquons que la m�thode dont nous
nous sommes servis est aussi appliquable � une equation
elliptique quleconque � un nombre quleconque de dimensions.}
\medskip

\noindent
In � xx we shall present Carleman's asymptotic formula for
eigenvalues of a second order ellitpic operator
in ${\bf{R}}^3$
which in general has variable coefficiebts and need not be self-adjoint.
Of course, to get a resyukt such as
Theorem 1 with an asymptotic limit formula which
is independent of the point $p$
in the domain where
the eigenfunctions appear, usually requires that the elliptic
PDE-operator has constant coefficients.
It goes without saying that many specific problems desevere to be analyzed in more detail.
A broader perpective concerning asymptotic
reperesentations arises when one for example regards
spectral functions associated to
self-adjoint operators defined via elliptic PDE:s.
See � xx below whgere we give some comments about
Carlea'ns discussion of
the Schr�dinger equation
\[
\Delta(u)- c(x,y,z)\cdot u=i\frac{\partial u}{\partial t}
\]
Here $\Delta is the Laplace operator and
$c(x,y,z) is a real-valued function which
is locally square integrable and
there exist  constants $R$  and  $M$ such that
\[
c(x,y,z)\leq M\quad\colon x^2+y^2+z2\geq R^2
\] 
When (*) holds it was proved by Carleman  in 1931
that the densely defined operator
$\Delta-c$ is self-adjoint on $L^2({\bf{R}}^3)$
whose  spectrum is confined to
$[\ell,+\infty)$ for some real number
$\ell$.
If $\Theta$ is the associated spectral function
we get a solution to (xx) wiuth
initial condition
$u(p,0)= f(p)$ by
\[
u(p,t)= \int_\ell^\infty\, e^{it\lambda}\cdot \bigl[\,{\int_{\bf{R}}^3}
\Theta(p,q;\lambda)\cdot f(q)\, dq\,\bigr ]\,d\lambda
\]












\newpage



\centerline{\bf{Proof of Theorem 2}}

\bigskip

\noindent
For each $\lambda$ outside the discrete set $\{\lambda_n\}$ we put
\[
G(p,q;\lambda)=
G(p,q)+
2\pi\lambda\cdot \sum_{n=1}^\infty\,
\frac{\phi_n(p)\phi_n(q)}{\lambda_n(\lambda-\lambda_n)}\tag{1}
\]
Notice that (i-ii) above entail that the last sum is
converges and gives a meromorohic function of
the complex variable $\lambda$ whose poles
are at most simple and confied to the set $\{\lambda_n\}$
Moreover, we get 
the integral operator
$\mathcal G_\lambda$ defined on $L^2(\Omega)$ by
 \[ 
 \mathcal G_\lambda(f)(p)
 =\frac{1}{2\pi}\cdot \iint_\Omega\, G(p,q;\lambda )\cdot f(q)\, dq\tag{2}
\]

\medskip

\noindent
{\bf{A. Exercise.}} Use that the eigenfunctions $\{\phi_n\}$ is an orthonormal basis in
$L^2(\Omega)$ to show that
\[
(\Delta+\lambda)\cdot \mathcal G_\lambda=-E
\]


\noindent{\bf{B. The function $F(p,\lambda)$}}.
Set
\[ 
F(p,q,\lambda)= G(p,q;\lambda)- G(p,q)
\]
Keeping $p$ fixed we see that (1) gives
\[
\lim_{q\to p}\, F(p,q,\lambda)=
2\pi\lambda\cdot \sum_{n=1}^\infty\,
\frac{\phi_n(p)^2}{\lambda_n(\lambda-\lambda_n)}\tag{B.1}
\]
Set
\[
F(p,\lambda)=
\lim_{q\to p}\, F(p,q,\lambda)\tag{B.2}
\]
From (i-ii) above we see that
$F(p,\lambda)$
is a meromorphic function in
the complex $\lambda$-plane with at most simple poles
at $\{\lambda_n\}$.
\medskip

\noindent{\bf{C. Exercise.}}
Let $0<a<\lambda_1$. Use  residue calculus to show
the equality below in the  half-space
$\mathfrak{Re}\, s>2$:
\[ 
\Phi_p(s)=
\frac{1}{4\pi^2 \cdot i}\cdot \int_{a-i\infty}^{a+i\infty}\, 
F(p,\lambda)\cdot \lambda^{-s}\, d\lambda\tag{C.1}
\]
where the line integral  is taken on the vertical  line
$\mathfrak{Re}\,\lambda=a$.

\medskip

\noindent
{\bf{D. Change of contour integrals.}}
At this stage we employ a device which goes to
Riemann and
move the integration into the half-space
$\mathfrak{Re}(\lambda)<a$.
Consider  the curve $\gamma_+$
defined as the union of the
negative real interval $(-\infty,a]$ followed by
the upper
half-circle $\{\lambda= ae^{i\theta}\,\colon 0\leq\theta\leq \pi \}$
and the 
half-line $\{\lambda= a+it\,\colon t\geq 0\}$.
Cauchy's theorem entails that 
\[ 
\int_{\gamma_+}\, F(p,\lambda)\cdot \lambda^{-s}\, d\lambda=0
\]
We leave it to the reader to contruct the
similar
curve
$\gamma_-=\bar \gamma_+$. Using 
the vanishing of these line integrals and taking the branches of the 
multi-valued function
$\lambda^s$ into the account the reader should verify the following:

\medskip


\noindent
{\bf{E. Lemma.}}
\emph{When $\mathfrak{Re}\, s$ is sufficientyl large
one has the equality}
\[ 
\Phi(s)=\frac{a^{s-1}}{4\pi}\cdot \int_{-\pi}^\pi\,
F(ae^{i\theta})\cdot e^{(i(1-s)\theta}\,d\theta
+
\frac{\sin \pi s}{2\pi^2}\cdot \int_a^\infty\, F(p,-x)\cdot x^{-s}\,dx\tag{E.1}
\]
\medskip

\noindent
The first term in the sum of the right hand side of (E.1)
is obviously an entire function of $s$. So Theorem 2 follows if
\[
 s\mapsto  \frac{\sin \pi s}{2\pi^2}\cdot \
 \int_a^\infty\, F(p,-x)\cdot x^{-s}\,dx\tag{E.2}
\]
is meromorphic with
a single pole at $s=1$ whose residue is $\frac{1}{4\pi}$.
To prove  this we shall  express $F(p,-x)$ when $x$ are real and positive in another way.
\medskip

\noindent
{\bf{F.  The $K$-function.}}
In the half-space $\mathfrak{Re}\,z>0$ there exists the analytic function
\[
K(z)= \int_1^\infty\, \frac{e^{-zt}}{\sqrt{t^2-1}}\,dt
\]
\medskip

\noindent
{\bf{Exercise.}}
Show that $K$ extends to a multi-valued analytic function outside
$\{z=0\}$ given by
\[
K(z)=-I_0(z)\cdot \log z+ I_1(z)\tag{F.1}
\] 
where $I_0$ and $I_1$ are entire functions
with series expansions
\[
I_0(z)=\sum_{m=0}^\infty\, \frac{2^{-2m}}{(m!)^2}\cdot
z^{2m}\tag{i}
\]
\[ 
I_1(z)= \sum_{m=0}^\infty\, \rho(m)\cdot
\frac{2^{-2m}} {(m!)^2} \cdot z^{2m}\quad
\colon \rho(m)=1+\frac{1}{2}+\ldots+\frac{1}{m}-\gamma\tag{ii}
\]
where $\gamma$ is the usual Euler constant.

\bigskip


\noindent
Next, with   $p$ kept fixed and $\kappa>0$ 
we solve the Dirichlet problem and find
a  function $q\mapsto H(p,q;\kappa)$ which satisfies  the
equation
\[
 \Delta(H)-\kappa\cdot H=0\tag{F.2}
\] 
in $\Omega$ with boundary values
\[ 
H(p,q;\kappa)=K(\sqrt{\kappa}|p-q|)\quad\colon q\in \partial\Omega
\]


\noindent
{\bf{G. Exercise.}}
Verify the equation
\[ 
G(p,q;-\kappa)=K(\sqrt{\kappa}\cdot |p-q|)- H(p,q;\kappa)\quad\colon \kappa>0\tag{G.1}
\]



\noindent
Together with the  construction of $G(p,q)$ the reader can verify the equation
\[
 F(p,-\kappa)=
 \lim_{q\to p}\,
 [K(\sqrt{\kappa}\cdot |p-q|)+\log\,|p-q|]+
 \lim_{q\to p}\,[H(p,q)- H(p,q,\kappa)]\tag{G.2}
\]
The last term above has the  "nice limit" 
$H(p,p)+H(p,p,\kappa)$ and from  (F.1)  the reader can  verify the limit formula:
\[
 \lim_{q\to p}\,
 \bigl(\, K(\sqrt{\kappa}\cdot |p-q|)+\log\,|p-q|\bigr)=
 -\frac{1}{2}\cdot \log \kappa +\log 2-\gamma\tag{G.3}
\]
where $\gamma$ is  Euler's constant.

\bigskip

\noindent
{\bf{H. Final part of the proof.}}.
Set $A=  +\log 2-\gamma+H(p,p)$. Then (G.1) and (G.2)
give
\[
F(p,-\kappa)= -\frac{1}{2}\cdot \log \kappa +A+H(p,p;\kappa)
\]
With $x=\kappa$ in (E.2 ) we  proceed  as follows.
To  begin with it is clear that
\[
s\mapsto A\cdot 
\frac{\sin \pi s}{2\pi^2}\cdot \int_a^\infty\,  x^{-s}\,dx
\]
is an entire function of $s$.
Next,  consider the function
\[ 
\rho(s)=
 -\frac{1}{2}\cdot 
\frac{\sin \pi s}{2\pi^2}\cdot \int_a^\infty\,  \log x\cdot x^{-s}\,dx
\]
Notice that the complex derivative
\[
\frac{d}{ds}\,  \int_a^\infty\,  x^{-s}\,dx=
- \int_a^\infty\,  \log x\cdot x^{-s}\,dx
\]

\medskip
\noindent
{\bf{H.1 Exercise.}}
Use the  above to show that
\[
\rho(s)-\frac{1}{4\pi(s-1)}
 \]
is an entire function.
\medskip


\noindent
From the above we see that Theorem 2  follows if we have proved
\medskip

\noindent
 {\bf{H.2 Lemma.}}
\emph{The following function  is entire}:
\[
s\mapsto \frac{\sin\,\pi s}{2\pi^2}\cdot
\int_a^\infty\, H(p,p,\kappa)\cdot  \kappa^{-s}\,d\kappa
\]
\medskip

\noindent
\emph{Proof.}
When $\kappa>0$
the equation (F.1) shows that $q\mapsto H(p,q;\kappa)$
is subharmonic  in $\Omega$ and the maximum principle gives
\[
0\leq  H(p,q;\kappa)\leq \max_{q\in\partial\Omega}\,K(\kappa|p-q|)\tag{i}
\]
With  $p\in\Omega$ fixed there is 
a positive number
$\delta$ such that
$|p-q|\geq\delta\,\colon q\in \partial\Omega$ which  
gives
positive constants
$B$ and  $\alpha$  such that
\[
H(p,p;\kappa)\leq e^{-\alpha\kappa}\quad\colon \kappa>0\tag{ii}
\]
The reader may now check that this
exponential decay gives Lemma H.2.

\newpage


\centerline{\bf{Neumann's resolvent operators and
the spectral theorem for self-adjoint operators}}


\bigskip

\noindent
{\bf{Introduction.}}
We shall review basiuc facts from operator theory
where the spectral theorem for self-adjoint operators on
Hilbert spaces is in the focus.
The notable point is that one in general allows unbounded
self-adjoint operators.
\medskip


\centerline{\bf{Neumann resolvents.}}
\medskip

\noindent
Let $X$ be a Banach space and $T\colon X\to X$
is an unbounded and  densely defined linear operstor.
It means that the domain of definition is a dense subspace denoted by
$\mathcal D(T)$, while
thr norms $||Tx||$ can take arbirary large values whiule
$x$ varies over vectors of unit norm in
$\mathcal D(T)$.
We say that $T$ has an inverse in Neumann's sense 
if the range
$T(\mathcal D(T))=X$ and there exists a posotiove constant $c$ such that
\[
||Tx||\geq c\cdot ||x||\quad\colon\, x\in\mathcal D(T)
\]
When this hold it is easily seen that nthere exists a bounded inear operator
$R$ whose range is equalm tom $\mathcal D(T)$ and
\[
R\circ T(x)=x\quad\colon x\in\mathcal D(T)\quad
\&\quad T\circ R(x)=x\quad \colon\,\ x\in X
\]
We put $R= T^{-1}$ and refer to $R$ as Neumann's resolvent
of the densely defined operator $T$.
The reader should notice that
$R$ is not invertible as a bounded operator, i.e.
$\{0\}$ belongs to its compact spectrum
$\sigma(R)$.
\medskip

\noindent
{\bf{Exercise.}}
Let $E$ be the identity operator on $X$.
Keeping $T$ fixed we 
regard the set of non-zero complex numbers
$\lambda$ for which $\lambda\cdot E-T$ has an inverse in Neumann's sense.
The reader may verify that
$(\lambda\cdot E-T)^{-1}$ exists if and only if
the bounded operator $E-\lambda\cdot R$ is invertible
and that
\[
(\lambda\cdot E-T)^{-1}=(E-\lambda\cdot R)\tag{*}
\]
If $\lambda|\cdot ||R||<1$ then the operator
$E-\lambda\cdot R$ is invertible and hence
$(\lambda\cdot E-T)^{1}$ exists when $\lambda$ belongs to the disc of radius
$||R||^{-1}$ centered at the origin.
The closed complement where
$(\lambda\cdot E-T)^{-1}$ does not exist is denoted by
$\sigma(T)$. It is in general an unbunded closed subet of
the complex $\lambda$-plane.
More precisely we have
\[
\sigma(T)=\{\lambda\neq 0\,\colon
\lambda^{-1}\in\sigma(R)\}
\]
\medskip

\noindent
{\bf{Remark.}}
Above we regard the spectrum of $T$
in the finite complex plane, i.e. the special point at infinity may be included to
add that $T$ itself is an unbounded operator.
Conversely, let $R$ be a bounded linear operator whose kernel is the zero space and
the range $R(X)$ is dense in $X$. Then
we see that $R=T^{-1}$ where $T$ is the densely defined operstor for ehich
$T(Rx)=x$ for all $x\in X$.
\medskip

\noindent
{\bf{Exercise.}}
Find a pair $(R,X)$ where
$R$ as above is injective with a dense range
and $\sigma(R)$ is reduced to the origin.
So in that case th unbounded operator
$T$ is such that
$(\lambda\cdot E-T)^{1}$ exists for
all $\lambda\in {\bf{C}}$.
\bigskip

\noindent
{\bf{The case when $X$ is a Hilbert space.}}
Now we assume that $X=\mathcal H$ is a complex Hilbert space.
Let $T$ as above be densely defined and unbounded
where $T^{-1}$ exists.
Wr impose the extra condition that
Neumann's resolvent $R$ is a \emph{normal operator}, i.e.
$R$ commutes with the adjoint $R^*$ in the algebra of bounded linear opertors
on $\mathcal H$.











To begin with, let $\mathcal H$ be a complex Hilbert space
equiped with a hermitian inner priduict, i.e.
to each pair of vectors $x,y$ one assigns a complex number
$\langle x,y\rangle$ which satisfies
\[
\langle x,y\rangle=\overline{\langle y,x\rangle}\quad
\&\quad \langle x,x\rangle>0\,\colon\, x\neq 0
\]
Moreover, $\mathcal H$ is complete under
the norm defined by
$|x|= \sqrt{\langle x,x\rangle}$.
Denoten by $L(\mathcal H)$
the space of bounded linear operators on $\mathcal H$
equipped with the norm
\[
||T||=\max_{|x|=1}\ |T(x)|
\]
Recall that every bounded inear operator
$T$ has a spectrum
$\sigma(T)$ which appears
as a compact subset of
${\bf{C}}$. A complex number $\lambda$ is outside
$\sigma(T)$ if and only if
the operator $\lambda\cdot E_T)$ is invertible. The inverse operator
\[ 
R_T(\lambda)= (\lambda\cdot E-T)^{-1}
\] 
is called a Neumann resolvent of $T$.
A wellknown fact asserts that the $L(\mathcal H)$-valued function
\[
\lambda\mapsto R_T(\lambda)
\] 
is analytic in ${\bf{C}}\setminus \sigma(T)$.
Moreover, Neumann's resolvent operators commute
weith each other, and also with the given operator $T$.
Thus follows from Neuann's equation
\[
R(\mu)-R(\lambda)=\frac{R(\mu)\cdot R(\lambda)}{\lambda-\mu}
\]
In particular the complex derivatives satsify
\[
\frac{d}{d\lambda}(R_T(\lambda)= -R_T(\lambda)^2
\]
for every $\lambda$ outside $\sigma(T)$.
\medskip

\noindent
Jext, recall the spectral radius formula which is valid for elements in arbitrary
commutative Banach algebras which posses an identty element.
For a bounded operator $T$ on $\mathcal H$ this gives
the equation
\[
\lim_{n\to \infty}\, ||T^n||^{\frac{1}{n}}=
\max_{\lambda\in\sigma(T)}\tag{0.1}
\,|\lambda|
\]
The common value in (0.1) is denoted by $\rho(T)$.
Notice that $\rho(T)=0$ can occur, i.e this holds when
$\sigma(T)$ is reduced to the origin.
The multiplicative inequality for operators norms gives
\[
\rho(T)\leq ||T||\tag{0.2}
\]
In general this is a strict inequality.
Next, let
\[
p(T)=a_0\cdot E+a_1T\ldots+a_nT^n
\]
be a polynomial in $T$.
At the same time we have the polynomial $P(\lambda)$. of the complex
variable $\lambda$. The following is left to the reader:

\medskip

\noindent
{\bf{Exercise.}}
Show that
\[
\sigma(P(T))=P(\sigma(T))\tag{0.3}
\]


\bigskip


\centerline{\bf{A. Self-adjoint operators.}}
\medskip

\noindent
A bounded linear operator $A$ is self-adjoint if
\[
\langle Ax,y\rangle=\langle x,Ay\rangle\tag{A.1}
\] 
hold for all pairs $x,y$.
it follows that
\[
|Ax|^2=\langle Ax,Ax\rangle=\langle A^2x,x\rangle\tag{i}
\]
hold for every vector $x$. Form this the reader can check the following equality
for operator norms:
\[
||A||^2=||A^2||\tag{A.2}
\]
\medskip

\noindent
{\bf{A.3 Exercise.}}
Deduce from (A.2) that
\[
||A||= \rho(A)\tag{A3.1}
\]
Show also
that when $A$ is self-adjoint, then
$\sigma(A)$ is a compact set on the real $\lambda$-line.
A hint is that
if $\lambda$ is complex and $x$ a unit vector then
\[
\langle \lambda x-Ax,\lambda x-Ax\rangle=
|\lambda|^2+|Ax|^2-2\mathfrak{Re}(\lambda)\cdot \langle Ax,x\rangle
\geq |\mathfrak{Im}\lambda|^2\tag{A.3.2}
\]
Finally, show that if
$p(A)$ is a polynomial with
real coefficients, then
$p(A)$ is self-adjoint and
\[
||p(A)||= \max_{\lambda\in \sigma(A)}\, |p(\lambda)|\tag{A.3.3}
\]
\bigskip

\centerline{\bf{B. Normal operators.}}
\bigskip

\noindent
A bounded linear operator $R$ is normal if it connutes
with its adjoint $R^*$.
Since $R^{**}=R$ and $(ST)^*= T^*S^*$ hold for
every pair $S,T$ in $L(\mathcal H)$, it follows that
the operator $RR^*$ is self-adjoint.
\medskip

\noindent
{\bf{B.1 Exercise.}}
Use the equality $||T||=||T^*||$ for every bounded operator, together with
(A.3.1) applied to $A=RR^*$ to conclude that
\[
\rho(R)=||R||\tag{B.1.1}
\]
Next, let
\[
p(R)= c_0+c_1R+\ldots+c_mR^m
\]
be a polynomial where $\{c_\nu\}$ are complex numbers.
The adjoint becomes
\[
p(R)^*=  \bar c_0+\bar c_1R^*+\ldots+\bar c_m(R^*)^m
\]
Conclude that when $R$ is normal, it follows that
$p(R)$ also is normal, and that (0.3) gives
\[
||p(R)||=
\max_{\lambda\in \sigma(R)}\,|p(\lambda)|\tag{B.1.2}
\]
\medskip

\noindent
More generally, consider a polynomial in $\lambda$ and its complex conjugarte:
\[
Q=\sum\sum\, c_{\nu,k}\cdot \lambda^\nu\cdot \bar\lambda^k
\]
To $Q$ we associate the operator
\[
\widehat{Q}= \sum\sum\, c_{\nu,k}\cdot R^\nu\cdot (R^*)^k
\]
Te reader shouod check that
this gives a normal operator
and also  verify
the equality
\[
||Q||= \max_{\lambda\in\sigma(R)}\, |Q(\lambda)|\tag{B.1.3}
\]


\bigskip

\noindent
{\bf{B.2 The algebra $\mathcal B^\infty(R)$.}}
Keeping $R$ fixed we have the compact
set
$K=\sigma(R)$ in the complex $\lambda$-plane.
Let $C^0(K)$ be the Banach algebra of continuous and complex-valued
functions on $K$ equipped with the maxium norm.
The Stone-Weierstrass theorem
asserts that
every
$\phi\in C^0(K)$ can be uniformly approximated by
polynomials in $\lambda$ and its complex conjugate $\bar\lambda$.
So given $\epsilon>0$ we can find
\[
Q=\sum\sum\, c_{\nu,k}\cdot \lambda^\nu\cdot \bar\lambda^k
\]
such that the maximum norm $|\phi-Q|_K<\epsilon$.
Let us choose a sequence of such polynomials $\{Q_n\}$
where $|\Phi-Q_n|_K\to 0$
Now (B.1.3) entials that
\[
\lim\, ||Q_n-Q_m||=0
\] 
when $n$ and $m$ both tend to infty.
Since $L(\mathcal H)$ is compete under the operator norm
there exists a bounded linear operator
$T$ such that
\[
\lim\,||Q_n-T||=0
\]
The reader should check that $T$ does not depend upon the
sequence $\{Q_n\}$ which approximates the given
$\Phi$-function.
We set $T=\widehat{\Phi}$ and in this way we
have constructed a bounded linear opertor starting from
an arbitrary continuous  function on $K$.
Moreover, we have the equality
\[
|\Phi|_K= ||\widehat{\Phi}||\tag{B.2.1}
\]
\medskip

\noindent
{\bf{B.3 Exercise.}}
Use (B.2.1) to conclude that
$\Phi\mapsto \widehat{\Phi}$
is a norm-preseribnfg algebra isomorphism from
$C^0(K)$ onto a closed subalgebra of $L(\mathcal H)$
which we denote by $\mathcal B(R)$.
Moreover, every operator  in this algebra is normal
and  commutes with
$R$. In this algebra 
$R$ corresponds to
the $\Phi$-function defined by
$\lambda$, while $R^*=\widehat{\Psi}$
with
$\Psi(\lambda)=\bar\lambda$.
\medskip

\noindent
{\bf{B.4 The algebra $\mathcal B^\infty(R)$.}}
Let $x,y$ be a pair of vectors in
$\mathcal H$..
To each $\Phi\in C^0(K)$ we get the complex number
\[
\langle \widehat{\Phi}(x),y\rangle\tag{B.4.1}
\]
From (B.3) we see that
the absolute value of this inner product is
\[
\leq |x|\cdot |y|\cdot |\Phi|_K\tag{B.4.2}
\]
Tjs emans that (B.4.1) in  defines a linear functional on
the complex vector space
$C^0(K)$ whose norm is majorised by
$|x|\cdot |y|$.
The famous represetnationm theoremn by F. Riesz
gives a compex Borel measure $\mu$ of totalvariation
$\leq |x|\cdot |y|$ such that
\[
\langle \widehat{\Phi}(x),y\rangle=\int_K\Phi(\lambda)\,d\mu(\lambda)
\]
The measure depends on the pair $x,y$ and is denoted
by
$\mu_{\{x,y\}}$.
In this way we have condtructed a map
\[
\mathcal H\times\mathcal H\to M(K)\tag{B.4.3}
\]
where $M(K)=nC^0(K)^*$ is the space of
Borel measure of finite total variation on $K$.
let us now consider a bounded complex-valued
Borel function
$\Psi$ on $K$.
In other words, for
every open set $U$ in the complex $\lambda$-plane the inverse image
$\Psi^m{-1}(U)$ belongs to
the Boolean $\sigma$-algebra of subsets of $K$ generated by
its compact subsets.
Now we apply classic measure theory from
original work by Emile Borel
In  particular there exist  integrals
\[
\int_K\, \Psi(\lambda)\cdot d\mu_{\{x,y\}}(\lambda)\tag{B.4.4}
\]
for every pair $x,y$ in  $\mathcal H$.
\medskip



\noindent
{\bf{The operator $\widehat{\Psi}$.}}
Keeping $\Psi$ and $y$ fixed
we notice that
the integral in (B.4.4) depends linearly upon $x$.
Recall that a Hibert space is self-dual.
This gives a unique vector
$\xi\in \mathcal H$ such that
\[
\int_K\, \Psi(\lambda)\cdot d\mu_{\{x,y\}}(\lambda)=\langle x,\xi \rangle
\quad\colon\,\forall\,\,x\in\mathcal H
\]
Next, $\xi$ depends upon the pair $\Psi$ and $y$
and is dentoed by $\xi(\Psi;y)$.
Keeping $\Psi$ fixed while $y$ varies the
reader should verify that
\[
y\mapsto \xi(\Psi;y)
\]
is linear with resepct to
$y$.
The conclusion is that there exists a  linear operator
$\widehat{\Psi}$ such that
\[
\int_K\, \Psi(\lambda)\cdot d\mu_{\{x,y\}}(\lambda)=\langle x,\widehat{\Psi}(y) \rangle
\]
hold for every pair $x,y$ in $\mathcal H$.
The reader should alsomcheck that
thisoinear operator is bounded and the operator norm
satisfies
\[
||\widehat{\Psi}||\leq |\Psi|_K
\]




\newpage


\centerline{\bf{Self-adjoint operators.}}






















\newpage





\centerline{\bf{� A. Eigenvalues and eigenfunctions for the Laplace operator
in ${\bf{R}}^2$}}.
\bigskip


\noindent
An instructive exampe of a densely defined but unbounded self-adjoint operator
appears in PDE-thery where we consider propagation of sound.
More precisely, let $U$ be a bounded open domain in
${\bf{R}}^3$ whose boundsry
$S=\partial U$ is a union of a finite number of
surfaces, each of class $C^2$.
Let $\Omega= {\bf{R}}^\setminus \overline{U}$
be the unbounded complement.
With $t$ as a time variable we seek solutions to
the wave equation
\[
\partial^2 u/\partial t^2=\\Delta(u)
\]
where $\Delta$ is the Laplace operator in
${\bf{R}}3$.
Above $u=u(x,t)$ is defined for
$t\geq 0$ and $x\in\Omega$
and satisfies the following boundary conditions:
To begin with
$x\mapsto u(x,t)$ belongs to $L^2(\Omega9$ for
every $t\geq 0$.
Moreover, applying the Lapace operator we also request that
$x\mapsto \Delta u(x,t)$ belongs to $L^2(\Omega)$ for
every $t\geq 0$.
Finally
\[
\frac{\partial u}{\partial n}(p,t)=0
\] 
for every $p\in S$ and $t\geq 0$ where we have taken
outer nortmal derivatives along
$S$.
Denote by $\mathfrak{U}$ the family of all functions
$u(x,t)$ satisfying the wave equation and
the boundary condition (*) together with
the impposed $L^2$-conditions.
With these notations one has the following
classic result:

\medskip

\noindent
{\bf{Theorem.}}
\emph{For each pair of functions $f_0,f_1$ in
the family $\mathcal F(\Omega)$ there
exists a unique $u\in\mathfrak{U}$
such that}
\[
u(p,0)= f_0(p)\quad\colon \frac{\partial u}{\partial t}(p,0)=f_1(p)
\]
\medskip

\noindent
In [ibid] Carleman gave a proof of this theorem using his theory
about unbounded
self-adjoint operators.It has the merit that
it yields an expression of the unique solution $u$
and at the same time
clarifies the physically expected fact that
for every $u$-solution one has
\[
\lim_{t\to +infty}\, \nabla u(p,t)=0
\] 
where the convergence holds uniformly while $p$ satays in a compact subset of
$\Omega$.
Here $\nabla(u)=(u_x,u_y,u_z)$ is the gradient vector
and (*) can be expressed by saying that as $t\to +infty$ then every
compact portion of $\Omega$ comes to rest.
The novelty in Carelan's demonstration of (*) was to
analyze the spectral function assocviated toa certain 
symmetric Green's function.
This method has later been adotpoted in PDE-theory dealing with
various boundary value problems.
Following  (ibid; page 1xx-1xx] we expose Carleman's
proof of Theorem x and of (*) which in my opinion offers a very
instructiuve lesson about
constructions of unbounded self-adjoint operators.
First we recall some facts which were established by G. Neumann and H.Poincar�
around 1880.

\medskip

\centerline{\bf{The Green's function $G(p,q)$.}}
\medskip


\noindent
Given $\Omega$ as above with its $C^2$-boundary $S$
there exists a unique function $G(p,q)$
defined in
$\Omega\times\Omega$ which is symmetric
and has Newton's singulairty along the diagonal,  i.e.
\[
G(p,q)= \frac{1}{[p-q|}+ H(p,q)
\] 
where
the $H$-function is symmetric and continuous on
$\overline{\Omega}\times\overline{\Omega}$
and
\[
\lim_{|p|\to \infty} \, |p-q|\cdot H(p,q)=0
\]
hold for every fixed
$q$.
Moreover, for every fixed
$p\in \Omega$ one has
\[
\frac{\partial G(p,q)}{\partial n_q}=0\quad\colon q\in S
\]
let us also recall that during the construction of
the syymeric $G$-ffun ction, Poincar� established the following
estimate: For every $p\in \Omega$ there
exists a constant $C(p)$ such that 
\medskip

\[
 \max_{q\in \overline{\Omega}}\,
|p-q|\cdot |G(p,q)|\leq C(p)
\]
\medskip

\noindent
Keeping $p$ fixed
one has 
\[
G^2(p,q)\simeq |p-q|^{-2}
\]
when $|q|$ tends to infinity.
So the integral
\[
\int_\Omega\, G^2(p,q)\, dq
\] 
is divergent.
However, if $p$ and $p'$ is a pair of points in
$\Omega$ then
Poincar� proved that
\[
\int_\Omega\, (G(p,q)-G(p',q))^2\, dq<\infty
\] 
The finiteness of these $L^2$-integrals will
be sued below to
analyze the unbounded operator defined via
the symmetric $G$-function.
More precisely, we
notice that
there exists a densely defined operator on $L^2(\Omega$ defined by
\[
\mathcal G(f)(p)= \int_\Omega\, G(p,q)\cdot f(q)\, dq
\]

\medskip

\noindent
So far we have collected classic results due to G. Neumann and H. Poincar�.
The new feature in Carleman's work goes as follows:

\medskip

\noindent
{\bf{Theorem.}}
\emph{The densely defined linear operator
$\mathcal G$ is self-adjoint and its  spectrum is confined to
the non-negative real line. Moreover, $\mathcal G$  is complete in the sense that}
\[
f=\int_0^\infty\, \frac{d\Theta}{d\lambda}(f)
\]
\emph{hold for every $f\in\mathcal F(\Omega)$
where $\lambda\to \Theta(\lambda)$ is
the spectral function attached to $\mathcal G$.}
\bigskip


\noindent
{\bf{Remark.}}
See � xx for the meaning of the
operator-valued $\Theta$-function attched to the
self-adjoint operator
$\mathcal G$.
The theorem above  implies that
the unique solution $u$ in Theorem xx is given by
\[
u=\int_0^\infty\, \cos\,\sqrt{\lambda}\,t\cdot \frac{d\Theta}{d\lambda}(f_0)+
\int_0^\infty\, \sin\,\sqrt{\lambda}\,t\cdot \frac{d\Theta}{d\lambda}(f_1)
\]
Finally, (*) follows via Riemann's Lemma for Fourier series if
the spectral function $\Theta$ is absolutely continuous with
respect to
$\lambda$.
\bigskip

\centerline{\bf{About the proof of Theorem xx}}
\bigskip

\noindent
As expected Green'sformuka wull
be applied in several situations.
To begin with we shal need the following:
\medskip

\noindent
{\bf{A. Lemma}}\emph{For each $f\in\mathcal F(\Omega)$ it follows that
that the Dirichlet integral} 
\[
\int_\Omega\, |\nabla(f)|^2\, dq<\infty
\]
\medskip

\noindent
\emph{Proof.}
Consider large $R$ so that ${\bf{R}}^3\setminus \Omega$
is a compact subset of the ball $B(R)$.
Greens formula gives
\[
\int_{\Omega\cap B(R)}\, f\cdot \Delta(f)\, dq+
\int_{\Omega\cap B(R)}\, |\nabla(f)|^2 \, dq=
\int_{S(R)}\, u\cdot \frac{\partial u}{\partial r}\, d\sigma-
\int_{\partial\Omega}\, u\cdot \frac{\partial u}{\partial n}\, d\sigma
\]
where $S=\partial\Omega$ and $S(R)$ is the shere of radius $R$.
Since both $u$ and $\Delta(u)$ belong to
$L^2(\Omega)$ the Cauchy-Schwarz inequaslity entials that
the first term above is bounded as a funvtion of $r$ and
the last area integral over $\partial\Omega$ is also bounded by
(xx).
Now we see that lemma A follows if
\[
\liminf_{R\to \infty}\, \int_{S(R)}\, u\cdot \frac{\partial r}{\partial n}\, d\sigma=0\tag{A.1}
\]
To orve (A.1) we introduce the function
\[
\psi(R)= \int_{\Omega\cap B(R)}\, u^2\,dq
\]
It follows that
\[
\psi'(R)=\int-{S(R)}\, u^2\,d\sigma
\] 
Passing to the second derivative the reader can check that
\[
\psi''(R)=\frac{2}{R}\cdot \psi'(R)+2\cdot\int_{S(R)}\, \, u\cdot \frac{\partial u}{\partial r}\, d\sigma
\]
So (A.1) follows if
\[
\liminf_{R\to \infty}\, \frac{\psi''(R)}{2}-\frac{1}{R}\cdot \psi'(R)=0
\]
To prove this we frist notice that
$\psi(R)$ is a positive and  increasing function. Moreover, 
since
$u\in L^2(\Omega)$ 
the $\psi$-function has  a finite limit $A$ as $R\to\infty$.
Now two csses can occur:

\medskip

\noindent 
{\bf{Case 1}}. Here we assume that 
 $R\mapsto \psi'(R)$
is non-increasing and tends to zero.  
With $R_2>R_1$ we apply Rolle's theorem and obtain
\[
\psi '(R_2)-\psi '(R_1)=(R_2-R_1)\cdot \psi ''(r)\quad\colon R_1<r<R_2
\]
We can apply this for pair $R_2>>R_1>>1$ and from this the reader
can check (xx).

\medskip

\noindent
{\bf{Case 2.}}
Here the non-negative function $\psi'(R)$ takes infinitely
many local  minima.
Say that they occur at points $q_1<q_2<\ldots$ where $q_\nu\to \infty$.
At these points we have $\psi''(q_\nu))=0$.


\medskip

\noindent
{\bf{exercise.}}
deducde from the above that if $f$ and $g$ is a pair in
$\mathcal F(\Omega)$ then
\[
\int_\Omega\, g\cdot \delta(f)\ dq=
\int_\Omega\, f\cdot \delta(q)\ dq
\]

\medskip

\noindent
{\bf{An integral equation.}}
Keeping $p_*\in\Omega$ fixed
we consider a complex number $\lambda$ and seek
complex-valued functions
$u$ in $\mathcal F(\Omega)$ which satisfy
\[
u(p)-u(p_*)= \frac{\lambda}{4\pi}\cdot \int_\Omega\, 
(G(p,q)-G(p_*,q))u(q)\, dq
\]
From the results in � xx it is clear that
every such $u$-function
satisfies the equation
\[ 
\Delta(u)+\lambda\cdot u=0
\]
and the outer normal derivative
$\frac{\partial u}{\partial n}$ is zero on
$\partial\Omega$.
Moreover, using Lemma xx the reader can check that
\[
\lambda\cdot \int_\Omega\, |u|^2\, dq=
 \int_\Omega\, \bigl [\, |u_x|^2+|u_y|^2+(u_z|^2\bigr ]\, dq
\]
This entails that
$\lambda$ must be a real and positive number when
$u$ is not identically zero.
\medskip

\noindent
{\bf{Conclusion.}}
The fact that (xx) above has no non-trivial solution when
$\mathfrak{Im}\,\lambda\neq 0$
entails via Carleman's  general theory in [ibid] that
the densely defined operator $\mathcal G$ is self-adjoint.
Moroever, we leave it to the reader to check
that every $g\in L^2(\Omega)$�for which the function
\[
p\mapsto \int_\Omega\, \bigl (G(p,q)-G(p_*,q)\bigr )\cdot g(q)\, dq=0
\] 
must be a zero function in
$L^2(\Omega)$.
From this it follows that
the self-adjoint operator $\mathcal G$ is complete, i.e. one
has the integral represetnation formua (xx) for
every $f\in\mathcal F(\Omega)$.





















\newpage







\centerline{\bf{� A. Eigenvalues and eigenfunctions for the Laplace operator
in ${\bf{R}}^2$}}.
\bigskip


\noindent
Theorem 1 below  was  presented by Carleman at the
Scandinavian Congress in mathematics held in Stockholm 1934:
In
${\bf{R}}^2$ we 
consider a bounded Dirichlet regular domain
$\Omega$, i.e. every $f\in C^0(\partial\Omega)$
has a harmonic extension to $\Omega$.
A wellknown  
fact which goes back to original work by Dirichlet
gives the following: There
exists the Greens' function
\[
G(p,q)= \log\,\frac{1}{|p-q|}-H(p,q)\tag{*}
\]
where $H(p,q)= H(q,p)$ is continuous in
the product set
$\overline{\Omega}\times\overline{\Omega}$.
Moreover, $H(p,q)= H(q,p)$ is symmetric and when
$q\in\Omega$ is fixed, then
$p\mapsto H(p,q)$ is harmonic in
$\Omega$ and 
\[ 
H(p,q)= \log\,\frac{1}{|p-q|}\quad\colon\, p\in \partial\Omega
\]
This means that $p\mapsto  G(p,q)$ vanishes on the boundary.
next, (*) means that $p\mapsto G(p,q)$ is superharmonic and
and the minimum principe for superharmonic functions
plus symmetry entail that 
\[
G(p,q)>0
\]
hold in $\Omega\times\Omega$. 
Next, it is obvious that
\[
 \iint_{\Omega\times\Omega}\,|G(p,q)|^2\, dpdq<\infty
 \]
Hence the linear operator on the Hilbert space $L^2(\Omega)$ defined by
the symmetric kernel $G(p,q)$
is a Hilbert-Schmidt operator on
the Hilbert space $L^2(\Omega)$ and therefore a compact operator.
Since the kernel symmetric and positive the eigenvalues
are positive, and general Hilbert space theory
applied to the symmetric $G$-kernel gives a 
a sequence of pairwise orthogonal functions
$\{\phi_n\}$
whose $L^2$-norms are one and
\[
\int_\Omega\, G(p,q)\phi_n(q)\, dq=2\pi\cdot \mu_n\cdot \phi_n(p)\tag{1}
\]
where
$\{\mu_n\}$
is a non-increasing sequence of positive eigenvalues which tend to zero.
Next, we apply the Laplace operator on both sides.
Recall that
\[
\Delta(\log\, \frac{1}{|z|})=-2\pi\cdot \delta_0
\]
where $\delta_0$ is the Dirac measure.
It follows that 
the Laplacian of the left hand side in (1)
becomes
$-2\pi\cdot \phi(p)$ and hence (1) gives the equation
\[
\Delta(\phi_n)(p)+\frac{1}{\mu_n}\cdot \phi_n(p)=0
\]
We prefer to use $\lambda_n=\frac{1}{\mu_n}$. Then
$\{\lambda_n\}$ is a non-decreasing sequence of real numbers
which tends to $+\infty$.
Since the kernel $G(p,q)$ is positive it follows  by general 
Hilbert space theory -
that $\{\phi_n\}$ is an orthonormal basis in
$L^2(\Omega)$, i.e. each $L^2$-function $f$ has an expansion
\[
f=\sum\, a_n\cdot \phi_n\quad\colon\,\,
a_n= \int_\Omega\, f_n(p)\cdot\overline{\phi_n}(p)\, dp\tag{2}
\]
\medskip

\noindent
{\bf{0. Exercise.}}
Verify from the above   that each $\phi$-function is  a continous function
in $\Omega$ whose boundary  values on
$\partial\Omega$ are zero.
Show also that
\[
G(p,q)=
\sum_{n=1}^\infty\, \frac{\phi_n(p)\cdot \phi_n(q)}{\lambda_n}\tag{0.1}
\]
where the right hand side is a convergent series when $p\neq q$.
\bigskip


\noindent
Notice that $G(p,p)=+\infty$
so the series
above with $p=q$ is divergent.
However, there exists a limit when 
we employ  larger denominators.
\medskip


\noindent
{\bf{Theorem 1.}} \emph{For every Dirichlet regular domain
$\Omega$ and each $p\in\Omega$�one has the limit formula}
\[ 
\lim_{N\to\infty}\, \lambda_N^{-1}\cdot \sum_{n=1}^{n=N}\, \phi^2_n(p)= \frac{1}{4\pi}\tag{*}
\]
\medskip

\noindent
{\bf{Remark.}} Carleman was iinspired by an earlier result due to
H. Weyl which asserts that
the set of eigenvalues satisfy the asymptotic formula
\[
\lim_{N\to \infty}\, \frac{\lambda_N}{N}=\frac{\text{Area}(\Omega)}{4\pi}
\]
Notice that Weyl's asymptotic formula together with (*) gives
\[
\lim_{N\to \infty}\,N^{-1}\cdot \sum_{n=1}^{n=N}\, \phi^2_n(p)=xxxx
\]
The notable point is that this asymptotic limit is the same for
\emph{every} point $p\in\Omega$.
The proof of Theorem 1 requires several steps.
First, since  $\mathcal G$ is a Hilbert-Schmidt operator a wellknown result due to Schur
gives
\[
\sum\, \lambda_n^{-2}<\infty \tag{i}
\]
Let us also remark that since
each $\phi_n$ is harmonic we have
the mean-value equality
\[
\phi_n(p)= \frac{1}{\pi r^2}\cdot \int_{D_p(r)}\, \phi(q)\,dq
\] 
where $D_p(r)$ is the disc of radius $r$ centered at $p$ and
$r$ is chosen so small that
the disc stays in $\Omega$.
Since the $L^2$-norms of the $\phi$-functions are equal to one,
the Cauchy-Schwarz inequality  gives a constant $C$ such that
\[
|\phi_n(p)|\leq C\quad\colon \, n=1,2,\ldots\tag{ii}
\]
Now (i-ii) entail that 
that  the  Dirichlet series
\[
\Phi_p(s)=\sum_{n=1}^\infty \frac{\phi_n^2(p)}{\lambda_n^s}
\]
is an analytic function of the complex variable $s$
in the half-plane
$\mathfrak{Re}\,  s>2$.
Less obvious  is the following:

\medskip

\noindent
{\bf{Theorem 2.}}
\emph{For each $p\in\Omega$ there exists an entire function
$\Psi_p(s)$ such that}
\[
\Phi_p(s)=\Psi_p(s)+\frac{1}{4\pi(s-1)}
\]
\medskip

\noindent
{\bf{Remark.}}
In � xx we explain how
Theorem 2 gives
Theorem 1 from   Ikehara's limit formula.
So the main task is to establish Theorem 2.
\bigskip


\noindent
{\bf{Remark.}}
The proof of Theorem 2 employs analytic function theory and is inspiredby Riemann's
work on the $\zeta$-function.
The interested reader is invited to
establish more general results, where $\Delta$ is replaced by a higher order
elliptic  operator in ${\bf{R}}^n$ where $n\geq 3$ can hold.
In the cited article  such extensions are poibnted out by Carleman on
p. xx after the proof of Theorem 1.
\medskip

\noindent
\emph{Remarquons que la m�thode dont nous
nous sommes servis est aussi appliquable � une equation
elliptique quleconque � un nombre quleconque de dimensions.}
\medskip

\noindent
In � xx we shall present Carleman's asymptotic formula for
eigenvalues of a second order ellitpic operator
in ${\bf{R}}^3$
which in general has variable coefficiebts and need not be self-adjoint.
Of course, to get a resyukt such as
Theorem 1 with an asymptotic limit formula which
is independent of the point $p$
in the domain where
the eigenfunctions appear, usually requires that the elliptic
PDE-operator has constant coefficients.
It goes without saying that many specific problems desevere to be analyzed in more detail.
A broader perpective concerning asymptotic
reperesentations arises when one for example regards
spectral functions associated to
self-adjoint operators defined via elliptic PDE:s.
See � xx below whgere we give some comments about
Carlea'ns discussion of
the Schr�dinger equation
\[
\Delta(u)- c(x,y,z)\cdot u=i\frac{\partial u}{\partial t}
\]
Here $\Delta is the Laplace operator and
$c(x,y,z) is a real-valued function which
is locally square integrable and
there exist  constants $R$  and  $M$ such that
\[
c(x,y,z)\leq M\quad\colon x^2+y^2+z2\geq R^2
\] 
When (*) holds it was proved by Carleman  in 1931
that the densely defined operator
$\Delta-c$ is self-adjoint on $L^2({\bf{R}}^3)$
whose  spectrum is confined to
$[\ell,+\infty)$ for some real number
$\ell$.
If $\Theta$ is the associated spectral function
we get a solution to (xx) wiuth
initial condition
$u(p,0)= f(p)$ by
\[
u(p,t)= \int_\ell^\infty\, e^{it\lambda}\cdot \bigl[\,{\int_{\bf{R}}^3}
\Theta(p,q;\lambda)\cdot f(q)\, dq\,\bigr ]\,d\lambda
\]












\newpage



\centerline{\bf{Proof of Theorem 2}}

\bigskip

\noindent
For each $\lambda$ outside the discrete set $\{\lambda_n\}$ we put
\[
G(p,q;\lambda)=
G(p,q)+
2\pi\lambda\cdot \sum_{n=1}^\infty\,
\frac{\phi_n(p)\phi_n(q)}{\lambda_n(\lambda-\lambda_n)}\tag{1}
\]
Notice that (i-ii) above entail that the last sum is
converges and gives a meromorohic function of
the complex variable $\lambda$ whose poles
are at most simple and confied to the set $\{\lambda_n\}$
Moreover, we get 
the integral operator
$\mathcal G_\lambda$ defined on $L^2(\Omega)$ by
 \[ 
 \mathcal G_\lambda(f)(p)
 =\frac{1}{2\pi}\cdot \iint_\Omega\, G(p,q;\lambda )\cdot f(q)\, dq\tag{2}
\]

\medskip

\noindent
{\bf{A. Exercise.}} Use that the eigenfunctions $\{\phi_n\}$ is an orthonormal basis in
$L^2(\Omega)$ to show that
\[
(\Delta+\lambda)\cdot \mathcal G_\lambda=-E
\]


\noindent{\bf{B. The function $F(p,\lambda)$}}.
Set
\[ 
F(p,q,\lambda)= G(p,q;\lambda)- G(p,q)
\]
Keeping $p$ fixed we see that (1) gives
\[
\lim_{q\to p}\, F(p,q,\lambda)=
2\pi\lambda\cdot \sum_{n=1}^\infty\,
\frac{\phi_n(p)^2}{\lambda_n(\lambda-\lambda_n)}\tag{B.1}
\]
Set
\[
F(p,\lambda)=
\lim_{q\to p}\, F(p,q,\lambda)\tag{B.2}
\]
From (i-ii) above we see that
$F(p,\lambda)$
is a meromorphic function in
the complex $\lambda$-plane with at most simple poles
at $\{\lambda_n\}$.
\medskip

\noindent{\bf{C. Exercise.}}
Let $0<a<\lambda_1$. Use  residue calculus to show
the equality below in the  half-space
$\mathfrak{Re}\, s>2$:
\[ 
\Phi_p(s)=
\frac{1}{4\pi^2 \cdot i}\cdot \int_{a-i\infty}^{a+i\infty}\, 
F(p,\lambda)\cdot \lambda^{-s}\, d\lambda\tag{C.1}
\]
where the line integral  is taken on the vertical  line
$\mathfrak{Re}\,\lambda=a$.

\medskip

\noindent
{\bf{D. Change of contour integrals.}}
At this stage we employ a device which goes to
Riemann and
move the integration into the half-space
$\mathfrak{Re}(\lambda)<a$.
Consider  the curve $\gamma_+$
defined as the union of the
negative real interval $(-\infty,a]$ followed by
the upper
half-circle $\{\lambda= ae^{i\theta}\,\colon 0\leq\theta\leq \pi \}$
and the 
half-line $\{\lambda= a+it\,\colon t\geq 0\}$.
Cauchy's theorem entails that 
\[ 
\int_{\gamma_+}\, F(p,\lambda)\cdot \lambda^{-s}\, d\lambda=0
\]
We leave it to the reader to contruct the
similar
curve
$\gamma_-=\bar \gamma_+$. Using 
the vanishing of these line integrals and taking the branches of the 
multi-valued function
$\lambda^s$ into the account the reader should verify the following:

\medskip


\noindent
{\bf{E. Lemma.}}
\emph{When $\mathfrak{Re}\, s$ is sufficientyl large
one has the equality}
\[ 
\Phi(s)=\frac{a^{s-1}}{4\pi}\cdot \int_{-\pi}^\pi\,
F(ae^{i\theta})\cdot e^{(i(1-s)\theta}\,d\theta
+
\frac{\sin \pi s}{2\pi^2}\cdot \int_a^\infty\, F(p,-x)\cdot x^{-s}\,dx\tag{E.1}
\]
\medskip

\noindent
The first term in the sum of the right hand side of (E.1)
is obviously an entire function of $s$. So Theorem 2 follows if
\[
 s\mapsto  \frac{\sin \pi s}{2\pi^2}\cdot \
 \int_a^\infty\, F(p,-x)\cdot x^{-s}\,dx\tag{E.2}
\]
is meromorphic with
a single pole at $s=1$ whose residue is $\frac{1}{4\pi}$.
To prove  this we shall  express $F(p,-x)$ when $x$ are real and positive in another way.
\medskip

\noindent
{\bf{F.  The $K$-function.}}
In the half-space $\mathfrak{Re}\,z>0$ there exists the analytic function
\[
K(z)= \int_1^\infty\, \frac{e^{-zt}}{\sqrt{t^2-1}}\,dt
\]
\medskip

\noindent
{\bf{Exercise.}}
Show that $K$ extends to a multi-valued analytic function outside
$\{z=0\}$ given by
\[
K(z)=-I_0(z)\cdot \log z+ I_1(z)\tag{F.1}
\] 
where $I_0$ and $I_1$ are entire functions
with series expansions
\[
I_0(z)=\sum_{m=0}^\infty\, \frac{2^{-2m}}{(m!)^2}\cdot
z^{2m}\tag{i}
\]
\[ 
I_1(z)= \sum_{m=0}^\infty\, \rho(m)\cdot
\frac{2^{-2m}} {(m!)^2} \cdot z^{2m}\quad
\colon \rho(m)=1+\frac{1}{2}+\ldots+\frac{1}{m}-\gamma\tag{ii}
\]
where $\gamma$ is the usual Euler constant.

\bigskip


\noindent
Next, with   $p$ kept fixed and $\kappa>0$ 
we solve the Dirichlet problem and find
a  function $q\mapsto H(p,q;\kappa)$ which satisfies  the
equation
\[
 \Delta(H)-\kappa\cdot H=0\tag{F.2}
\] 
in $\Omega$ with boundary values
\[ 
H(p,q;\kappa)=K(\sqrt{\kappa}|p-q|)\quad\colon q\in \partial\Omega
\]


\noindent
{\bf{G. Exercise.}}
Verify the equation
\[ 
G(p,q;-\kappa)=K(\sqrt{\kappa}\cdot |p-q|)- H(p,q;\kappa)\quad\colon \kappa>0\tag{G.1}
\]



\noindent
Together with the  construction of $G(p,q)$ the reader can verify the equation
\[
 F(p,-\kappa)=
 \lim_{q\to p}\,
 [K(\sqrt{\kappa}\cdot |p-q|)+\log\,|p-q|]+
 \lim_{q\to p}\,[H(p,q)- H(p,q,\kappa)]\tag{G.2}
\]
The last term above has the  "nice limit" 
$H(p,p)+H(p,p,\kappa)$ and from  (F.1)  the reader can  verify the limit formula:
\[
 \lim_{q\to p}\,
 \bigl(\, K(\sqrt{\kappa}\cdot |p-q|)+\log\,|p-q|\bigr)=
 -\frac{1}{2}\cdot \log \kappa +\log 2-\gamma\tag{G.3}
\]
where $\gamma$ is  Euler's constant.

\bigskip

\noindent
{\bf{H. Final part of the proof.}}.
Set $A=  +\log 2-\gamma+H(p,p)$. Then (G.1) and (G.2)
give
\[
F(p,-\kappa)= -\frac{1}{2}\cdot \log \kappa +A+H(p,p;\kappa)
\]
With $x=\kappa$ in (E.2 ) we  proceed  as follows.
To  begin with it is clear that
\[
s\mapsto A\cdot 
\frac{\sin \pi s}{2\pi^2}\cdot \int_a^\infty\,  x^{-s}\,dx
\]
is an entire function of $s$.
Next,  consider the function
\[ 
\rho(s)=
 -\frac{1}{2}\cdot 
\frac{\sin \pi s}{2\pi^2}\cdot \int_a^\infty\,  \log x\cdot x^{-s}\,dx
\]
Notice that the complex derivative
\[
\frac{d}{ds}\,  \int_a^\infty\,  x^{-s}\,dx=
- \int_a^\infty\,  \log x\cdot x^{-s}\,dx
\]

\medskip
\noindent
{\bf{H.1 Exercise.}}
Use the  above to show that
\[
\rho(s)-\frac{1}{4\pi(s-1)}
 \]
is an entire function.
\medskip


\noindent
From the above we see that Theorem 2  follows if we have proved
\medskip

\noindent
 {\bf{H.2 Lemma.}}
\emph{The following function  is entire}:
\[
s\mapsto \frac{\sin\,\pi s}{2\pi^2}\cdot
\int_a^\infty\, H(p,p,\kappa)\cdot  \kappa^{-s}\,d\kappa
\]
\medskip

\noindent
\emph{Proof.}
When $\kappa>0$
the equation (F.1) shows that $q\mapsto H(p,q;\kappa)$
is subharmonic  in $\Omega$ and the maximum principle gives
\[
0\leq  H(p,q;\kappa)\leq \max_{q\in\partial\Omega}\,K(\kappa|p-q|)\tag{i}
\]
With  $p\in\Omega$ fixed there is 
a positive number
$\delta$ such that
$|p-q|\geq\delta\,\colon q\in \partial\Omega$ which  
gives
positive constants
$B$ and  $\alpha$  such that
\[
H(p,p;\kappa)\leq e^{-\alpha\kappa}\quad\colon \kappa>0\tag{ii}
\]
The reader may now check that this
exponential decay gives Lemma H.2.

\newpage


\end{document}

