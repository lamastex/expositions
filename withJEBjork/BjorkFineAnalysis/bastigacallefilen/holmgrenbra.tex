\documentclass{amsart}

\usepackage[applemac]{inputenc}
\usepackage[applemac]{inputenc}
\addtolength{\hoffset}{-12mm}
\addtolength{\textwidth}{22mm}
\addtolength{\voffset}{-10mm}
\addtolength{\textheight}{20mm}






\def\uuu{_}

\def\vvv{-}





\begin{document}

\centerline{\bf{Cauchy transforms.}}

\medskip
\noindent
Let $\mu$ be a measire with compact support $K$ in the
complex $z$-plane  We write  $z0=x+iy$ and identify ${\bf{C}}$ with 
the 2-dimensionsl real $x(,y)$-space-
Wen $z$ is outside $K$ there exists thr integral
\[
\widehat{\mu}(z)= \int\,\frac{d\mu(\zeta)}{z-\zeta}
\]
Notice that (*) yields an analytic funtion in
${\bf{C}}\setminus K$. For example, the complex derivative
\[
\frac{d\widehat{\mu}}{dz}= - \int\,\frac{d\mu(\zeta)}{(z-\zeta)^2}
\]
Less onbious is thst (*)  is defined for all $z$  outside a null-set in
lebegue's sense and in this way
$\widehat{mu}$ is a locally integrable function in
the $x,y)$.space.
In fact, this follows sine (*) is the convolution of
$\mu$ asnd the locally integrable function $z^{-1}$.
More precisely, Lebesgue theoty teaches that if
$\phi$ is an arbitrary $L^1_{loc}$-function in
the $(x,y)$-plane and $\mu$ as above a measure with compact support, then
the convolution
$\phi*\mu$ is again locally integrable.
In particular $\phi*\mu$ is a distriobuytion in the $x,y$-plane and
we csn apply the $\bar\partial$-operator.
A general formula in distribution  theory teaches thst
\[
\bar\partial(\phi*\mu)= \bar\partial(\phi)*\mu
\]
Apked with $\phi=z{-1}$ it follows from
the Caciuhy-Pompieu theorem thst
\[
\bar\partial(\widehat{\mu})= \pi\cdot \mu
\]
\medskip


\noindent
Let us thrn consider a measure $\mu$ supported by
a union $D(R)\cup K$, where
$D(R)= \{|z|\leq R\}$ snd
$K$ a compsct subet of $\{|z|\geq R\}$.
We assume that $K$ is a null-set and that
\[
\Omega= {\bf{C}}\setminus (D(R)\cup K)
\]
is connected.
Choose $R^*>R$ so tyaht
$K$is contsined in the disc $D(R^*)$.
Then $\widehat{mu}$ is analyric in$\{|z|>R^*\}$
and in this exterior disc it is reprsetned by
a series
\[
\widehat{\mu}(z)= \sum_{n=1}^\infty\, c_n\cdot z^{-n}
\quad\colon\quad c_n= \int\, \zeta^{n-1}d \mu(\zeta)
\]
So if we assume that
$\mu$ is ortgohongsl to analytic polynomials, then
$\widehat{\mu}=0$ in the exteriro disc
$\{z|>R^*$.
Sunce $K$ is a null set it measnt aht
the loclly integrble funvtion
$\widehat{\mu}$ is supported by
$D(R)$ snd hence $\bar\partial(\mu)$ is also supppoetd by
$D(R)$, Then the Cauvhy-Pompieue theorr entslis thst
the govrn mesfire $\mu$ is suppored by $D(R)$.
\medskip


\noindent
{\bf{An applicstiuon.}}
Now start:
Fnd a polynomial  $P_1(z)$ such that
the maximum norm
\[
|P_1-f|_{K_1}<\alpha_1\quad P_1(1)= f(1)\,\&\, P_1(-1)= f(-1)
\]
Now we consider the compact set
\[
S_2= D(1)\cup K_2
\]
 where we find the continuous function
$g$ with $g=P_1$ in $D(1)$ while $g(x)=f(x)$ when
$1\leq |x|\leq 2$.
Let $\mu$ be a measiure supported by $S_2$ which is $\perp$ toanalytic poumonials $q(z)$.
Since thunion of the two reslintevrlas $[-2,-1]$ nd $[1,2]$ is a null set, it
follows from
XX that $\mu$ is suppotted by $D(!1)$ whivh entials that
\[
\int \, g\cdot \mu= \int\, P_1\cdot  d\mu=0
\]
where the ladt equality follows since
$P(z)$ is an analytic polymomial.
Now (*) hold for every $\mu$ as above and hence
$g$ csn be uniformly appriximated on $S_2$ by analytic polynomials.
So we find $P_2(z)$ such tht
\[
|P_2-g|_{S_1}<\alpha_1\quad P_2(2)= f(2)\,\&\, P_2(-2)= f(-2)
\]
We can contonue and for every $N\geq 2$ we put
\[
S_N=D(N)\cup K_{N+1}
\]
and find a sequece of polynomials $\{P_N(z)$ such that
\[
|P_{N+1}-P_N|_{D(N)}<\alpha_{N+1}\quad \&\quad 
|P_{N+1}-f|<\alpha_{N+1}
\]
Then the series 
\[
P_1+(P_2-P_1)+ (P_3-P_2)+\ldots
\]
xonverges uniforly on every cpma�ct disc $D(R)$
and the o it is an entire function $p(z)$.
done ....
\medskip


\noindent
An extension. 
We do $P_1$ so that
$|P_1-f|_{K-1}<\alpha-1$.
On $K[1]= K\cap \{z|=1\}$
we have the contomoud funtion $P_1-f$ whose maximum norm is small.
on this null set the disc algebra interpolates.
But how good is sup-norm prdsered. i.e. given
$g$ on a null set of thr unit circle.
To find $G$ in disc algebra with $G=g$ on the null set and not too large.
Thus,  are null sets of interpolation .
So one regards
\[
g\mapsto g_S\in C^0(S)
\] 
from 
$A(D)$ into $C^0(S)$.
Image is dense by the Brother Riesz's theorem.
So we do it for small $\epsilon$
This, first a polynomial
$P-1$ with
\[ 
|P-1-f[_{K-1}<\epsilon
\]
Next, we have $P_1-f$ on $K[1]$ with a small sup-norm.
Find $g$ in disc algebra so that
\[
|g+(P_1-f)|_{K[1]}<\epsilon
\]
and not too large $g$-norm on $D$.
So we can modify ...
Better, modify $f$ to $f_1=f+\rho$
so thst
\[
P_1=f+\rho
\]
holds on $K[1]$.
At he same time
$\rho$ has a small sup-norm 
on $K$.













\newpage






\centerline{\bf \large A uniqueness theorem for PDE-equations.}
\bigskip

\noindent
{\bf Introduction.}
We shall work in  ${\bf{R}}^2$ with  coordinates
$(x,y)$.
An example of a boundary value
problem is to determine a function
$u(x,y)$ 
which is harmonic in some open half-disc
\[
D_+(r)=\{x^2+y^2<r^2\}\,\cap \{x>0\}
\]
and satisfies the boundary conditions:
\[
u(0,y)=\psi(y)\quad\text{and}\quad
u_x(0,y)=\phi(y)
\]
where $\phi$ and $\psi$ are given in advance.
When $\phi$ and  $\psi$
are real-analytic one proves  easily that
(*) has a unique solution.
With less regularity Hadamard gave examples where
this fails to hold.
In fact, Hadamard proved that a necessary and sufficient condition for
the Cauchy problem to be well posed is that the function
\[ 
y\mapsto \phi(y)+\frac{1}{\pi}\int_a^b\,
\text{Log}\bigl[\frac{1}{|s-y|}\bigr]\cdot \psi(s)\cdot ds
\]
is real analytic.
Here we shall focus upon
uniqueness of solutions to  homogeneous
elliptic boundary value problems 
expressed
by
a system of first order partial differential equations.
Let
$n=2m$ be an even positive integer and consider two $n\times n$-matrices
$\mathcal A=\{A_{pq}\}$ and $\mathcal B=\{B_{pq}\}$
whose elements are real-valued functions of $x$ and $y$
where
the $B$-functions are continuous and the
$A$-functions  of class $C^2$. 
The two matrices give
a system of first order
PDE-equations whose solutions are 
vector valued functions $(f_1,\ldots,f_n)$
defined in a half-disc
\[
D_+(\rho)=\{ x^2+y^2<\rho^2\quad\colon\, x>0\}
\]
where these $f$-functions satisfy the system:
\[ 
\frac{\partial f_p}{\partial x}+
\sum_{q=1}^{q=n}\,
A_{pq}(x,y)\cdot
\frac {\partial f_p}{\partial y}+
\sum_{q=1}^{q=n}\,
B_{pq}(x,y)\cdot f_q(x,y)=0\tag{*}
\]


\noindent
together with the boundary conditions:
\[
f_p(0,y)=0\quad\text{for all}\quad 1\leq p\leq n\tag{**}
\]

\medskip

\noindent
We  get   eigenvalues of the
$\mathcal A$-matrix 
when $(x,y)$-varies,  i.e.
the $n$-tuple of  roots
$\lambda_1(x,y),\ldots,\lambda_n(x,y)$ which solve
\[
\text{det}\,\bigl(\lambda\cdot E_n-\mathcal A(x,y)\bigr)=0\tag{1}
\]



\noindent
If the $\lambda$-roots are non-real we say that (*)  is an elliptic system.
Assuming vanishing Cauchy data expressed by (**) above
one expects that a  solution
$f$  is identically zero.
This uniqueness  was proved by
Erik Holmgren in the  article [Holmgren]
from 1901
under the assumption that the $A$-functions and the
$B$-functions are real analytic.
The question remained if the uniqueness still holds under less
regularity on the coefficient functions. This was settled 30 years later 
by  Carleman where
the following is proved in [Carleman]:
\bigskip 

\noindent
{\bf 1. Theorem.} \emph{Assume that the
$\lambda$-roots are all simple and non-real as 
$(x,y)$ varies in the open half-disc. Then every solution $f$ to (**) 
with vanishing Cauchy-data  is identically zero}.
\medskip

\noindent
The proof requires several steps and 
the methods which occur below
have inspired more recent work where   Carleman estimates 
are used to handle  boundary value problems
in PDE-theory.
\newpage


\centerline {\bf {A. Proof of Theorem 1: First part}}
\medskip

\noindent
The system  in (*) is equivalent to a system of
$m$-many equations where one  seeks
complex-valued functions
$g_1,\ldots,g_m$   satisfying:
\[ 
\frac{\partial g_p}{\partial x}+
\sum_{q=1}^{q=m}\,
\lambda_p(x,y)\cdot
\frac {\partial g_p}{\partial y}=
\]
\[
\sum_{q=1}^{q=m}\,
a_{pq}(x,y)\cdot g_q(x,y)+
b_{pq}(x,y)\cdot \bar g_q(x,y)=0\,\,\colon\,1\leq p\leq m\tag{**}
\]
Above $\{a\uuu{pq}\}$ and $\{b\uuu{pq}\}$
are complex\vvv valued.
The  reduction  to this complex family of equations
is left to the reader and from  now
on we study the system (**). Theorem 1 amounts
to prove that
if the $g$-functions satisfy (**) 
in a half-disc
$D_+(\rho)$
and
\[
g_p(0,y)=0\quad\colon\quad 1\leq p\leq m
\]
then there exists some $0<\rho_*\leq \rho$ such that the $g$-functions
are identically zero in 
$D_+(\rho_*)$. To attai this we
introduce domains as follows:
For a pair  $\alpha>0$ and 
$\ell>0$ we put
\[ 
D_\ell(\alpha)=\{
x+y^2-\alpha x^2<\ell^2\}\cap\{ x>0\}\tag{1}
\]


\noindent
Notice that the boundary
\[ 
\partial D_\ell(\alpha)=\{0\}
\times[-\ell,\ell]\,\cup T_\ell\quad\text{where}\quad
x+y^2-\alpha x^2=\ell^2\,\, \text{holds on}\,\, T_\ell
\]



\noindent
Above $\alpha$ and $\ell$ are small so the the $g$-functions satisfy (**)
in $D_\ell(\alpha)$.
For  each  $t>0$
we define  the  $m$-tuple of functions
by
\[
\phi_p(x,y)= g_p(x,y)\cdot e^{-t(x+y^2-\alpha x^2)}\tag{2}
\]
Since the $g$-functions satisfy   (**)
one verifies easily that
the $\phi$-functions
satisfy the system
\[
\frac{\partial \phi_p}{\partial x}+
\frac {\partial}{\partial y}\bigl(\lambda_p\cdot\phi_p)+
t(1-2\alpha x+2y\lambda_p\bigl)\cdot\phi_p=H_p(\phi)\tag{3}
\]
where 
\[
H_p(\phi)=\sum_{q=1}^{q=n}\,
a_{pq}(x,y)\cdot \phi_q(x,y)+
b_{pq}(x,y)\cdot \bar \phi_q(x,y)=0\,\,\colon\,1\leq p\leq m
\]

\noindent
Next, we set
\[ 
\Phi(x,y)=\sum_{p=1}^{p=m}\, |\phi_p(x,y)|\tag{4}
\]


\noindent
The crucial step in the proof
of Theorem 1
is to establish the following inequality.
\medskip

\noindent
{\bf A.1 Proposition.} \emph{Provided that $\alpha$ from the start is sufficiently large
there exists some $0<\ell_*\leq\ell$ and a constant $C$
which is independent of $t$  such that}
\[
\iint_{D_{\ell_*}}\, \Phi(x,y)\cdot dxdy \leq
C\cdot \int_{T_{\ell_*}}\, \sum_{p=1}^{p=n}\,|\phi_p|\cdot
 |dy-\lambda_p\cdot dx|
\]
\medskip

\noindent
\centerline{\emph{How to deduce  Theorem 1.}}
\medskip

\noindent
Let us show why Proposition A.1 gives Teorem 1.
In addition to
$\ell_*$ we fix some
$0<\ell_{**}<\ell_*$.
In  (2) above we have used the function
\[
w(x,y)= e^{-t(x+y^2-\alpha x^2)}\implies
\]
\[ w(x,y)= e^{-t\ell_*^2}\quad\colon\quad
(x,y)\in T_{\ell_*}\quad\colon\quad
w(x,y)\geq e^{-t\ell_{**}^2}\quad\colon\quad (x,y)\in D_{\ell_{**}}\tag{i}
\]

\medskip

\noindent
Next, we have
$|\phi_p|= |g_p|\cdot w$ for each $p$. Replacing the left hand side in Proposition
A.1 by the area integral over the smaller
domain
$D_{\ell_{**}}$
we obtain the inequality;
\[
\iint_{D_{\ell_{**}}}\,
\sum_{p=1}^{p=m}\,| g_p(x,y)|\cdot dxdy\leq
C\cdot  e^{t(\ell_{**}^2-\ell_*^2)}\cdot 
\int_{T_{\ell_*}}\, \sum_{p=1}^{p=n}\,|g_p|\cdot
 |dy-\lambda_p\cdot dx|\tag{ii}
\]
\medskip

\noindent
Here (ii)  holds for every $t>0$. When
$t\to+\infty$ we have 
$ e^{t(\ell_{**}^2-\ell_*^2)}\to 0$
and can therefore conclude that



\[
\iint_{D_{\ell_{**}}}\,
\sum_{p=1}^{p=m}\,| g_p(x,y)|\cdot dxdy=0
\]

\noindent
This means  that the $g$-functions are all zero in
$D_{\ell_{**}}$ and Theorem 1 follows.


\bigskip

\centerline{\bf B. Proof of Proposition A.1}
\medskip

\noindent
The proof  relies upon the
construction of certain $\psi$-functions.
More precisely, when
$t>0$ and  a point $(x_*,y_*)\in D_\ell$ are given we shall construct
an $m$-tuple of $\psi$-functions satisfying
the following:

 \bigskip

\noindent
{\bf Condition 1.} Each
$\psi_p$ is defined in
the punctured domain
$D_\ell\setminus\{(x_*,y_*)\}$
where $\psi_p$ for a given $1\leq p\leq m$
satisfies the equation
\[
\frac{\partial\psi}{\partial x}+
\lambda_p\cdot\frac{\partial\psi}{\partial y}-
t(1-2\alpha x+2y\lambda_p)\psi_p=0\tag{i}
\]





\noindent
{\bf Condition 2}. For each $p$ the
singularity of $\psi_p$ at $(x_*,y_*)$
is such that the line integrals below have a limit:
\[
\lim_{\epsilon\to 0}\,
\int_{[z-z_*|=\epsilon}\,\psi_p\cdot (dx-\lambda_p\cdot dy)=2\pi\tag{ii}
\]


\noindent
{\bf Condition 3.} There exists a constant $K$
which is independent both of
$(x_*,y_*)$ and of $t$ such that
\[
|\psi_p(z)|\leq\frac{K}{|z-z_*|}\tag{iii}
\]
Notice that the $\psi$-functions depend on the parameter
$t$, i.e. they are found for each $t$ but the constant $K$ in (3) is independent of
$t$.
\bigskip

\centerline {\emph{ The deduction of Proposition A.1}}
\bigskip

\noindent
Before the $\psi$-functions are  constructed in Section C
we show how  they  give
Proposition A.1.
Consider a point  $z_*\in D_+(\ell)$.
We get the associated $\psi$-functions from � B
at this particular point.
Remove a small disc
$\gamma_\epsilon$ centered at $z_*$ and
consider some fixed
$1\leq p\leq m$.
Now $\phi_p$ satisfies the differential equation (3) from section A and 
$\psi_p$ satisfies (i) in Condition 1 above.
Stokes theorem gives:
\[
\int_{T_\ell}\phi_p\cdot\psi_p\cdot \bigl(dy-\lambda_p\cdot dx\bigr)
=\iint_{D_\ell\setminus\gamma_\epsilon}\,
H_p(\phi)\cdot \psi_p\cdot dxdy+
\int_{|z-z_*|=\epsilon}
\, \phi_p\cdot\psi_p\cdot \bigl(dy-\lambda_p\cdot dx\bigr)
\]
\medskip


\noindent
Passing to the limit as $\epsilon\to 0$, Condition 2 
\[
\phi_p(x_*,y_*)=\frac{1}{2\pi}
\int_{T_\ell}\phi_p\cdot\psi_p\cdot \bigl(dy-\lambda_p\cdot dx\bigr)
-\frac{1}{2\pi}\cdot \iint_{D_\ell}\,
H_p(\phi)\cdot \psi_p\cdot dxdy\tag{1}
\]


\noindent
Let $L$ be the maximum over $D_\ell$ of the 
coefficient functions  of $\phi$ and $\bar\phi$
which appear in  $H_p(\phi)$
from (3) i � A. We have also the constant $K$ from
Condition 3 for $\psi_p$. The
triangle inequality gives:


\[
\bigl|\phi_p(x_*,y_*)\bigr|\leq
\frac{K}{2\pi}
\int_{T_\ell}\frac{|\phi_p|\cdot |dy-\lambda_p\cdot dx|}{z-z_*|}
+\frac{LK}{\pi}\cdot \sum_{q=1}^{q=m}\,
\iint_{D_\ell}\,
\frac{|\phi_q|}{|z-z_*|}\cdot dxdy\tag{*}
\]
\medskip

\noindent
Next, we use the elementary  inequality

\[ \iint_\Omega\, \frac{dxdy}{\sqrt{x-a)^2+(y-b)^2}}
\leq 2\cdot\sqrt{\pi}\cdot \sqrt{\text{Area}(\Omega)}\tag{**}
\]
where $\Omega$ is an arbitrary  bounded domain
and $(a,b)\in\Omega$.
Apply (**)  with
$\Omega=D_\ell$ and set
$S=\text{area}(D_\ell)$.
Integrating both sides in (*) over $D_\ell$
for every
$p$ and taking the sum we get
\[
\iiint_{D_\ell}\, \Phi\cdot dxdy\leq
\]
\[
K\cdot\sqrt{\frac{S}{\pi}}\cdot \int_{T_\ell}\,
\sum_{p=1}^{p=m}\, |\phi_p|\cdot |dy-\lambda_p\cdot dx|
+2\pi m LK\cdot \sqrt{\frac{S}{\pi}}\iint_{D_\ell}\,
\Phi\cdot dxdy
\]
\medskip

\noindent
This inequality hold for all small $\ell$.  Choose
$\ell$ so small that
\[
2\pi m LK\cdot \sqrt{\frac{S}{\pi}}\leq\frac{1}{2}
\]
Then the inequality above gives
\[
\iiint_{D_\ell}\, \Phi\cdot dxdy\leq
2\cdot 
K\cdot\sqrt{\frac{S}{\pi}}\cdot \int_{T_\ell}\,
\sum_{p=1}^{p=m}\, |\phi_p|\cdot |dy-\lambda_p\cdot dx|\tag{***}
\]


\noindent
Finally, consider some relatively compact
domain
$\Delta$ in $D_\ell$. Then there exists
$0<\ell_*<\ell$ such that
\[
\Delta\subset D_{\ell_*}
\]
Now we notice that

\[ 
|\phi_p(z)|\geq e^{-t\ell_*^2}\cdot |u_p(z)|
\quad\colon\quad z\in \Delta\quad\colon\quad
|\phi_p(z)|\geq e^{-t\ell^2}\cdot |u_p(z)|
\quad\colon\quad z\in T_\ell
\]


\noindent
We conclude that
\[
e^{-t\ell_*^2}
\,\iiint_\Delta\, \sum_{p=1}^{p=m}\, |u_p(z)|\cdot dxdy\leq
e^{-t\ell^2}\cdot 2\cdot 
K\cdot\sqrt{\frac{S}{\pi}}\cdot \int_{T_\ell}\,
\sum_{p=1}^{p=m}\, |u_p|\cdot |dy-\lambda_p\cdot dx|\tag{****}
\]
\medskip

\noindent Here (****) hold for every $t>0$. Passing to the limit
as
$t\to+\infty$ it follows that

\[
\cdot \iiint_\Delta\, \sum_{p=1}^{p=m}\, |u_p(z)|\cdot dxdy\leq
\]
Since $\Delta$ was any relatively compact subset of
$D_\ell$, we conclude  that the $u$-functions are zero in
$D_\ell$ and Theorem 1   follows.







\bigskip

\centerline{\bf {C. Construction of the $\psi$-functions.}}

\bigskip


\noindent
Before we embark upon specific constructions
we investigate the whole family of solutions to
a first order differential operators  of the form
\[
Q=\partial_x+\lambda(x,y)\cdot\partial_y\tag{*}
\] 
where $\lambda(x,y)$ is a complex 
valued $C^2$-function whose
imaginary part is $>0$. 
Set
\[ 
\lambda(x,y)=\mu(x,y)+i\cdot\tau(x,y)\quad\colon\, \tau(x,y)>0
\]


\noindent
Now we look for
solutions 
$h(x,y)$ to the equation $Q(h)=0$.
With
$h(x,y)=\xi(x,y)+i\cdot\eta(x,y)$
where $\xi$ and $\eta$ are real-valued $C^2$-functions this gives
the differential system:
\[
\frac{\partial\xi}{\partial x}+\mu_p\cdot
\frac{\partial\xi}{\partial y}-
\tau_p\cdot \frac{\partial\eta}{\partial y}=0
\]
\[
\frac{\partial\eta}{\partial x}+\mu_p\cdot
\frac{\partial\eta}{\partial y}+
\tau_p\cdot \frac{\partial\xi}{\partial y}=0
\]

\medskip

\noindent
Suppose we have found one solution 
 $h=\xi+i\cdot\eta$ where the 
Jacobian 
$\xi_x\eta_y-\xi_y\eta_x$
is $\neq 0$ at the origin. 
Then $(x,y)\mapsto (\xi,\eta)$ is a local $C^2$-diffeomorphism.
With  $\zeta=\xi+i\eta$ 
we have the usual Cauchy\vvv Riemann operator.
\[ 
\frac{1}{2}(\frac{\partial}{\partial\xi}+
i\cdot \frac{\partial}{\partial\eta})
\]

\noindent
Let $g(\xi+i\eta)$ be  a holomorphic function
in the complex $\zeta$-space with $\zeta=\xi+i\eta$ and put
\[ 
g_*(x,y)=g(\xi(x,y)+i\eta(x,y))
\] 
Then one easily verifies that $Q(g\uuu*)=0$
and conversely, every solution
to this equation is expressed by 
a $g\vvv*$function derived from an
analytic function in the complex $\zeta$space.satisfies  $Q(g_*)$.
\medskip

\noindent
{\bf{Conclusion.}} \emph{If a non-degenerate solution $h=\xi+i\eta$ has been found
then the homogenous solutions to $Q$ is in a 1-1 correspondence to
analytic functions in the $\zeta$-variable.}
\medskip

\noindent
{\bf{Remark.}} The effect
of a coordinate transformation
as above is that
the $Q$-operator 
is transported to the
Cauchy-Riemann operator in
the complex $\zeta$-space where $\zeta=\xi+i\eta$.
Later we   employ such  $(\xi,\eta)$-transformations
to
construct  
solutions to an inhomogeneous equation of the form
\[ 
Q(\psi)=(t-\alpha x+2y\lambda(x,y))\cdot \psi(t,x,y)
\]
where $t$�is a positive parameter and the
$\psi$-functions 
will have certain specified properties.
Notice that it suffices to construct
the
$\psi$-functions separately, 
i.e. we no longer have to bother about
a differential system. With a fixed
$p$ fixed   $\lambda_p(x,y)=\mu_p+\tau_p$ and
from now on we may drop the index $p$ and explain how to obain
$\psi$-functions satisfying
the three conditions from � B.
So we consider
the first order differential operator
\[
Q=\frac{\partial}{\partial x}+(\mu(x,y)+i\tau(x,y))\cdot
\frac{\partial}{\partial y}\tag{1}
\]
where 
$\tau(x,y)>0$.
\bigskip



\noindent
{\bf C.1 A class of $(\xi,\eta)$-functions.}
Let $V(x,y)$ and $W(x,y)$
be two quadratic forms, i.e. both are homogeneous polynomials of
degree two. Given a point $(x_*,y_*)$ 
and with $z=x+iy$
we seek  a coordinate transformation $(x,y)\mapsto(\xi,\eta)$
of the form:

\[ 
\xi(z)= \tau_p(z_*)\cdot (x-x_*)+V(x-x_*,y-y_*)+\gamma_1(z)
\cdot|z-z_*|^2
\]
\medskip
\[ 
\eta(z)= (y-y_*)-\mu_p(z_*)\cdot (x-x_*)+W(x-x_*,y-y_*)+
\gamma_2(z)\cdot|z-z_*|^2
\]
\medskip

\noindent{\bf Lemma.}
\emph{There exists  a pair of quadratic forms
$V$ and $W$ whose coefficients depend on
$(x_*,y_*)$ and a pair of
$\gamma$-functions which both vanish at
$(x_*,y_*)$ up to order one such that the complex\vvv valued function
$\xi+i\eta$ solves the homogeneous equation $Q(\xi+i\eta)=0$.}
\medskip

\noindent
A solution above
gives
a change of variables
so that
$Q$ is expressed in new real coordinates
$(\xi,\eta)$ by the operator
\[
\frac{\partial}{\partial \xi}+i\cdot \frac{\partial}{\partial\eta}\tag{2}
\]
There exist many coordinate transforms
$(x,y)\to(\xi,\eta)$
which change $Q$ into (2).
This \emph{flexible choice} of coordinate transforms
is used to  construct  the required
$\psi$-functions. Notice  that
Condition (2) in � B is of a pointwise character, i.e. 
it suffices to find a $\psi$-function for a given point
$z_*=x_*+iy_*$. With this in mind
the required construction in � B boils down to perform
a suitable coordinate transformation
adapated to $z_*$,
and after use the existence of
a $\psi$-function which to begin with  is expressed in the
$(\xi,\eta)$-variables where  the $Q$-operator is replaced by
the Cauchy-Riemann operator. In this special case
the required $\psi$-function is easy to find, i.e. see 
the remark in � B.0.
So all that remains is to exhibit suitable coordinate transformations
which send $Q$ to the $\bar\partial$-operator. We leave it to the reader to
carry out such coordinate transformations. If
necessary, consult
Carleman's article where
a very detailed construction appears.


\bigskip



\end{document}

\centerline{\bf{The Schr�dinger equation.}}

\bigskip


\noindent
We work in ${\bf{R}}^3$ with the coordinates
$(x,y,z)$. Let $c(x,y,z)$ be a real-valued function in 
$L^2_{\text{loc}}({\bf{R}}^3$.
In order that the subsequent
formulas can be stated in a precise manner we also assume that
$c$is almost everywhere continuous which of course is a rather weak condition and
in any case satisfied in applications.
Next, let $\Delta$ be the Laplace operator and
define the operator $L$ by

\[ 
L(u)=\Delta(u)+c\cdot u\tag{*}
\]
Denote by $E_L({\bf{R}}^3)$
the set of functions $u$ such that both
$u$ and $L(u)$ belong to $L^2({\bf{R}}^3)$. Given a
pair $(f,\lambda)$ where
$f\in L^2({\bf{R}}^3$
and $\lambda$ is a complex number
we seek solutions $u\in E_L({\bf{R}}^3)$ such that

\[ 
L(u)+\lambda\cdot u=f\tag{**}
\]
\medskip

\noindent
{\bf{The case $\mathfrak{Im}(\lambda\neq 0$.}}
By
a classic result about solutions to the Neumann
boundary value problem  in
open balls in ${\bf{R}}^3$ one proves that
(1) has at least one solution $u$ whenever $\lambda$ is not real.
The remains to investigate the uniqueness, i.e, when one has the implication


\[
\mathfrak{Im}(\lambda\neq 0\quad\text{and}\quad 
L(u)+\lambda\cdot u=0\implies u=0\tag{***}
\]
\medskip


\noindent
This uniqueness property depends on the $c$-function.
A sufficient  condition is the following:


\medskip

\
\medskip

\noindent
{\bf{Theorem.}} \emph{Assume that there exists a constant $M$
and some $r_*>0$ such that}
\[ 
c(x,y,z)\leq M
\quad\text{when}\quad x^2+y^2+z^2\geq r_*^2
\]
Then (***) above holds.


\bigskip


\noindent
{\bf{The spectral $\theta$-function.}}
When (***)  holds it was proved in
[Carleman] that
classical solutions to the Neumanns boundary value problem in
open balls yield a $\theta$-function
which enable us to describe  solutions to 
(*) for real $\lambda$-values.
More precisely, there exists two increasing sequence of positive real numbers
$\{\lambda^*(\nu)\}$ and $\lambda_*(\nu)\}$
and two sequence of pairwise orthogonal functions
$\{\phi_\nu(p))\}$ and $\{\psi_\nu(p)\}$ in $L^({\bf{R}})^3$
where all these functions have $L^2$-norm equal to one such that
the following hold. First, set 


\[
\theta(p.q,\lambda)= \sum_{0<\lambda^*(\nu)\leq\lambda}\, \phi_\nu(p)\cdot\phi_\nu(q)
\quad\colon\quad \lambda>0
\]

\[
\theta(p.q,\lambda)= -\sum_{\lambda\leq \lambda_*(\nu)\leq<0}
\, \psi_\nu(p)\cdot\psi_\nu(q)
\quad\colon\quad \lambda<0
\]
such that the following hold:
\bigskip

\[
v(p)=\lim_{R\to \infty}\, \sum_{[\lambda_\nu<R}\, 
\theta(p,q,\lambda)\cdot v(q)\cdot dq\quad\text{for all}
\quad v\in L^2({\bf{R}}^3)\tag{1}
\]

\[
v\in E_L({\bf{R})^3}\quad\text{if and only if}\quad
xxxx\tag{2}
\]

\[
L(v)(p)=
\lim_{R\to \infty}\, \sum_{[\lambda_\nu<R}\, 
\lambda\cdot [
\int_{{\bf{R}}^3}
\theta(p,q,\lambda)\cdot v(q)\cdot dq]\cdot d\lambda\quad\text{for all}\quad
v\in E_L({\bf{R}}^3)\tag{3}
\]
Here the equaliy holds in $L^2$, i.e, in the sense of a Plancherel's limit.
\medskip


\noindent
{\bf{Remark.}}
Here the equaliy holds in $L^2$, i.e, in the sense of a Plancherel's limit.


\medskip

\noindent
{\bf{Construction of the $\phi$-functions. }}
For each finite $r$ we have the ball $B_r$ and consider the space $E_L(B_r)$ of 
functions 
$u$ in $B_r$ such that both $u$
and $L(u)$ also belong to $L^2(B_r)$.
By a classical result in the Fredholm theory that
exist discrete sequences of real numbers
$\{\lambda*(\nu)$ and $\lambda_*(\nu)$ as above
and two families of orthonormal functions
$\{\phi^{(r)}_\nu\}$ and $\{\psi^{(r)}_\nu\}$
satisfying (xx) 

and here a classical result shows that the real eigenvalues 
to the equation $L(u)+\lambda\cdot u=0$
xxx

\[ 
xxx
\]


\noindent
The proofs of the assertions above rely on a systematic use of Green's formula.
To begin with we recall how to
express solutions to an inhomogeneous
the Laplace equation by an integral formula.
\medskip


\noindent
{\bf{A. The equation $\Delta(u)=\phi$}}.
let $D$ be a domain in
${\bf{R}}^3$ and $\phi$
a function in $L^2(D)$.
Then a function $u$ for which both $u$ and $\Delta(u)$ belong to
$L^2(D)$ gives $\Delta(u)=\phi$ if and only if
the following hold for every $p\in D$ and every 
$\rho<\text{dist}(p,\partial D)$:

\[
u(p)=
\frac{1}{2\pi \rho^2}\cdot \int_{B_p(\rho)}\,
\frac{1}{|p-q|}\cdot u(q)\cdot dq+
\frac{1}{4\pi \rho^2}\cdot \int_{B_p(\rho)}\,
A(p,q)\cdot \phi(q)\cdot dq\tag{i}
\]
where we have put

\[ 
A(p,q)= \frac{2}{\rho}-\frac{1}{|p-q|}-\frac{|p-q|}{\rho^2}\tag{ii}
\]
\medskip

\noindent
{\bf{Exercise.}} Prove this result.The hint is to apply Green's formula
while $\phi$ is replaced by $\Delta(u)$ in the last integral.
\medskip

\noindent
{\bf{Remark.}}
Let us also recall
also that when $\Delta(u)$ is in $L^2$, then $u$ is automatically
a continuous function in $D$.
\medskip


\noindent
{\bf{The class $\mathfrak{Neu}(B_r)$}}.
Let $B_r$ be the open ball of radius $r$ centered at the origin.
The class of functions $u$  which are continuous on the closed ball
and whose interior normal derivative
$\frac{\partial u}{\partial{\bf{n}}}$ is continuous on the boundary
$S^2[r]$ is denoted by
$\mathfrak{Neu}(B_r)$. 
\medskip


\noindent
{\bf{The Neumann equation.}}
Let $c(x,y,z)$ be a function in $L^2({\bf{R}}$
and consider also a pair $a,H$ where
$a$ be a continuous function on
$S^2[r]$ and 
$H(p,q)$ a continuous hermitian function on
$S^2[r]\times S^2[r]$, i.e.  $H(q,p)=\bar H(p,q)$ hold for all pairs 
of point $p,q$ on
the sphere $S^2[r]$.
With these notations the following hold:


\bigskip


\noindent
{\bf{Theorem}} For each $f\in L^2(B_r)$ and every
non-real complex number $\lambda$ there exists a unique $u\in 
\mathfrak{Neu}(B_r)$
such that $u$ satisfies the two equations:


\[ 
L(u+\lambda\cdot u=f\quad\text{holds in}\quad B_r
\] 

\[
\partial u/\partial {\bf{n}}(p)=xx
\]
Moreover, one has the $L^2$-estimate

\[
\int_{B_r}\, |u|^2\cdot dxdydz\leq
\bigl|\frac{1}{\mathfrak{Im}(\lambda)}\bigr|\leq
\int_{B_r}\, |f|^2\cdot dxdydz
\]












\newpage




Classic result: $R>0$ we have unit ball $B_R$ and unit sphere $S_R$.
Let $\mathfrak{Neu}(R)$
set uf $u$ -functions where $\Delta(u)$ in $L^2$, continuous on closed ball
and limit of interior derivative as a continuous funtion.
Given $c\in L^2(B_R)$ define

\[
L(u)= \delta(u)+c\cdot u
\]
\medskip


\noindent
{\bf{Theorem}} For each pair $(a,H)$ in (*) there exists a unique
$u\in \mathfrak{Neu}(R)$ such that
\[ 
L(u+\lambda\cdot u=f\quad\text{holds in}\quad B_R
\] 
and $u$ saisfies th boundsry condition

\[
\partial u/\partial {\bf{n}}(p)=xx
\]

from that $L^2$-esteimate as well.

\bigskip

Do if for $R=m$ running over positive integers.
Catch up sequence with $L^2$-convergence bounded uniformly.
$u_m\to u_*$ weak sense and see tha $u_*$ is a solution to (*) on al over space.

Second point about eventual uniquenss. Class I type. Euquialent condition-


\end{document}


