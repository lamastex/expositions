

\documentclass{amsart}
\usepackage[applemac]{inputenc}


\addtolength{\hoffset}{-12mm}
\addtolength{\textwidth}{22mm}
\addtolength{\voffset}{-10mm}
\addtolength{\textheight}{20mm}

\def\uuu{_}


\def\vvv{-}

\begin{document}






\centerline{\bf{Asymptotic formulas for
eigenfunctions of the Laplace operator in Dirichlet domains}}

\bigskip


\noindent
The subsequent  results  were  presented by Carleman at the
Scandinavian Congress in mathematics held in Stockholm 1934:
Consider a bounded Dirichlet domain
$\Omega$ 
${\bf{R}}^2$,
i.e. every $f\in C^0(\partial\Omega)$
has a harmonic extension to $\Omega$.
For every fixed point $p\in\Omega$
one regards the continuous function
\[
q\mapsto \log\,\frac{1}{|p-q|}\quad\colon q\in \partial\Omega
\]
which gives a unique harmonic function
$x\mapsto H(p,x)$ in $\Omega$ such that 
\[
H(p,q)+\frac{1}{|p-q|}=0\quad\colon q\in \partial\Omega
\]
A wellknown  
fact  
established by
G. Neumann and H. Poincar�
shows that
the $H$-function is symmetric, i.e.
\[
H(p,q)= H(q,p)
\]
hold  for each pair of points $p,q$ in $\Omega$.
Greens' function is defined by 
\[
G(p,q)= \log\,\frac{1}{|p-q|}+H(p,q)
\]
Notice that for each fixed $p\in\Omega$ it follows that the function $q\mapsto G(p,q)$ is super-harmonic and zero 
when $q\in \partial\Omega$. The  minimum principle for superharmonic functions
entails that
\[
G(p,q)>0\quad\colon p,q\in \Omega\tag{i}
\]
and the reader shold check that
\[
 \iint_{\Omega\times\Omega}\,|G(p,q)|^2\, dpdq<\infty\tag{ii}
 \]
Hence the linear operator $\mathcal G$ on the Hilbert space $L^2(\Omega)$ defined by
\[
\mathcal G(\phi)(p)=\frac{1}{2\pi}\cdot \int_\Omega\, G(p,q)\phi(q)dq
\]
is of  the Hilbert-Schmidt type  and  therefore compact on
the Hilbert space $L^2(\Omega)$. 
Since the kernel  positive the eigenvalues
are positive, and  wellknown  facts about such nice integral operators
give
a sequence of pairwise orthogonal functions
$\{\phi_n\}$
whose $L^2$-norms are one and
\[
\mathcal G(\phi_n)=\mu_n\cdot \phi_n\tag{1}
\]
where
$\{\mu_n\}$
is a non-increasing sequence of positive eigenvalues which tend to zero.
Moreover, since the kernel $G(p,q)$ is positive it follows - again by general facts -
that $\{\phi_n\}$ is an orthonormal basis in
$L^2(\Omega)$, i.e. each real-valued function $f\in L^2(\Omega)$
 has an expansion
\[
f=\sum\, a_n\cdot \phi_n\quad\colon
a_n= \int_\Omega\, f_n(p)\cdot\phi_n(p)\, dp\tag{2}
\]
\medskip

\noindent
{\bf{Exercise.}}
Verify that each $\phi$-function extends to a continuous function on $\overline{\Omega}$
whose  boundary values are zero.
\medskip

\noindent
Next, let $\Delta$ be the Laplace operator.
Since
$\frac{1}{2\pi}\cdot \log\,|z|$ is a fundamental solution it follows that
\[ 
\Delta\circ \mathcal G(f)=-f\quad\colon f\in L^2(\Omega)\tag{3}
\]
Thus, the composed operator $\Delta\circ \mathcal G=-E$ where $E$ is the identity operator.
Put
\[
\lambda_n= \mu_n^{-1}
\]
Then (1) and (3) give
\[ 
\Delta(\phi_n)=-\lambda_n\cdot \phi_n\quad\colon\, n=1,2,\ldots\tag{4}
\] 
where we now have 
$0<\lambda_1\leq \lambda_2\leq \ldots\}$.
\bigskip

\noindent
With these notations we can
announce
Carleman's theorem.

\bigskip

\noindent
{\bf{A.1. Theorem. }}\emph{For every Dirichlet domain
$\Omega$ and each $p\in\Omega$�one has the limit formula}
\[ 
\lim_{N\to\infty}\, \lambda_N^{-1}\cdot \sum_{n=1}^{n=N}\, \phi_n(p)^2= \frac{1}{4\pi}\tag{*}
\]
\medskip

\noindent
To prove this  we consider some point $p\in\Omega$.
Since every $\phi_n$is harmonic and has $L^2$-norm one, the reader can check that with
a fixed $p$ there exists a constant $C(p)$ such that
\[
\phi_n(p)^2\leq C(p)\quad\colon n=1,2,\ldots
\]
Hence there exists
the Dirichlet series
\[
\Phi_p(s)=\sum_{n=1}^\infty \frac{\phi_n(p)^2}{\lambda_n^s}
\]
which is analytic 
in the  half-space
$\mathfrak{Re}\, s>1$.
We are going to prove the following result:


\medskip

\noindent
{\bf{A. 2 Theorem.}}
\emph{There exists an entire function
$\Psi_p(s)$ such that}
\[
\Phi_p(s)=\Psi_p(s)+\frac{1}{4\pi(s-1)}
\]
\medskip

\noindent
{\bf{Remark}}.
 Theorem A.2  gives Theorem A.1 by a 
result  due to Norbert Wiener in his  article
\emph{Tauberian theorem} [Annals of Math.1932] which 
asserts that if $\{\lambda_n\}$ is a non-decreasing sequence of
positive numbers  tending to infinity and
$\{a_n\}$ a sequence of  non-negative real numbers such that
there exists the limit
\[ 
\lim_{s\to 1}\,(s-1)\cdot \sum\, \frac{a_n}{\lambda_n^s}=A
\] 
then it follows that
\[
\lim_{n\to \infty}\,\lambda_n^{-1}\cdot
\sum_{k=1}^{k=n}\, a_k=A
\]
\medskip

\centerline{\bf{Proof of Theorem A. 2}}

\bigskip

\noindent
Since $\mathcal G$ is a Hilbert-Schmidt operator a wellknown result due to Schur
gives
\[
\sum\, \lambda_n^{-2}<\infty \tag{i}
\]
This convergence entails that various constructions below are defined.
For each $\lambda$ outside the discrete set $\{\lambda_n\}$ we put
\[
G(p,q;\lambda)=
G(p,q)+
2\pi\lambda\cdot \sum_{n=1}^\infty\,
\frac{\phi_n(p)\cdot \phi_n(q)}{\lambda_n(\lambda-\lambda_n)}\tag{ii}
\]
This gives the integral operator
$\mathcal G_\lambda$ defined on $L^2(\Omega)$ by
 \[ 
 \mathcal G_\lambda(f)(p)
 =\frac{1}{2\pi}\cdot \iint_\Omega\, G(p,q;\lambda )\cdot f(q)\, dq\tag{iii}
\]

\medskip

\noindent
{\bf{Exercise.}} Use that the eigenfunctions $\{\phi_n\}$ is an orthonormal basis in
$L^2(\Omega)$ to show that
\[
(\Delta+\lambda)\cdot \mathcal G_\lambda=-E
\]


\noindent{\bf{B. The function $F(p,\lambda)$}}.
Set
\[ 
F(p,q,\lambda)= G(p,q;\lambda)- G(p,q)
\]
Keeping $p$ fixed we see that (ii) gives  a function defined by
\[
F(p,\lambda)= \lim_{q\to p}\, F(p,q,\lambda)=
2\pi\lambda\cdot \sum_{n=1}^\infty\,
\frac{\phi_n(p)^2}{\lambda_n(\lambda-\lambda_n)}\tag{B.1}
\]
From (i) and (B.1) it follows that this yields a meromorphic   is a function in
the complex $\lambda$-plane with at most simple poles
at $\{\lambda_n\}$.
\medskip

\noindent{\bf{C. Exercise.}}
Let $0<a<\lambda_1$. Use  residue calculus to show
the equality below in the  half-space
$\mathfrak{Re}\, s>2$:
\[ 
\Phi_p(s)=
\frac{1}{4\pi^2 \cdot i}\cdot \int_{a-i\infty}^{a+i\infty}\, 
F(p,\lambda)\cdot \lambda^{-s}\, d\lambda\tag{C.1}
\]
where the line integral  is taken on the vertical  line
$\mathfrak{Re}\,\lambda=a$.

\medskip

\noindent
{\bf{D. Change of contour integrals.}}
At this stage we employ a device which goes to
Riemann and
move the integration into the half-space
$\mathfrak{Re}(\lambda)<a$.
Consider  the curve $\gamma_+$
defined as the union of the
negative real interval $(-\infty,a]$ followed by
the upper
half-circle $\{\lambda= ae^{i\theta}\,\colon 0\leq\theta\leq \pi \}$
and the 
half-line $\{\lambda= a+it\,\colon t\geq 0\}$.
Cauchy's theorem entails that 
\[ 
\int_{\gamma_+}\, F(p,\lambda)\cdot \lambda^{-s}\, d\lambda=0
\]
We leave it to the reader to contruct the
similar
curve
$\gamma_-=\bar \gamma_+$. Using 
the vanishing of these line integrals and taking the branches of the 
multi-valued function
$\lambda^s$ into the account the reader should verify the following:

\medskip


\noindent
{\bf{E. Lemma.}}
\emph{One has the equality}
\[ 
\Phi_p(s)=\frac{a^{s-1}}{4\pi}\cdot \int_{-\pi}^\pi\,
F(ae^{i\theta})\cdot e^{(i(1-s)\theta}\,d\theta
+
\frac{\sin \pi s}{2\pi^2}\cdot \int_a^\infty\, F(p,-x)\cdot x^{-s}\,dx\tag{E.1}
\]
\medskip

\noindent
The first term in the right hand side 
is obviously an entire function of $s$. So there remains to
prove that
\[
 s\mapsto  \frac{\sin \pi s}{2\pi^2}\cdot \
 \int_a^\infty\, F(p,-x)\cdot x^{-s}\,dx\tag{E.2}
\]
is meromorphic with
a single pole at $s=1$ whose residue is $\frac{1}{4\pi}$.
To show  this we are going to express $F(p,-x)$ when $x$ are real and positive in another way.
\medskip

\noindent
{\bf{F.  The $K$-function.}}
In the half-space $\mathfrak{Re}\,z>0$ there exists the analytic function
\[
K(z)= \int_1^\infty\, \frac{e^{-zt}}{\sqrt{t^2-1}}\,dt
\]
\medskip

\noindent
{\bf{Exercise.}}
Show that $K$ extends to a multi-valued analytic function outside
$\{z=0\}$ given by
\[
K(z)=-I_0(z)\cdot \log z+ I_1(z)\tag{F.1}
\] 
where $I_0$ and $I_1$ are entire functions
with series expansions
\[
I_0(z)=\sum_{m=0}^\infty\, \frac{2^{-2m}}{(m!)^2}\cdot
z^{2m}\tag{i}
\]
\[ 
I_1(z)= \sum_{m=0}^\infty\, \rho(m)\cdot
\frac{2^{-2m}} {(m!)^2} \cdot z^{2m}\quad
\colon \rho(m)=1+\frac{1}{2}+\ldots+\frac{1}{m}-\gamma\tag{ii}
\]
and $\gamma$ is the usual Euler constant.

\bigskip


\noindent
Next, with   $p$ kept fixed and $\kappa>0$ 
we solve the Dirichlet problem and find
a  function $q\mapsto H(p,q;\kappa)$ which satisfies  the
equation
\[
 \Delta(H)-\kappa\cdot H=0\tag{F.2}
\] 
in $\Omega$ with boundary values
\[ 
H(p,q;\kappa)=K(\sqrt{\kappa}|p-q|)\quad\colon q\in \partial\Omega
\]


\noindent
{\bf{G. Exercise.}}
Verify the equation
\[ 
G(p,q;-\kappa)=K(\sqrt{\kappa}\cdot |p-q|)- H(p,q;\kappa)\quad\colon \kappa>0
\]



\noindent
From  (F.1)  the reader can  verify the limit formula:
\[
 \lim_{q\to p}\,
 [K(\sqrt{\kappa}\cdot |p-q|)+\log\,|p-q|]=
 -\frac{1}{2}\cdot \log \kappa +\log 2-\gamma\tag{G.1}
\]
where $\gamma$ is  Euler's constant.
Next,   the construction of $G(p,q)$ gives
\[
 F(p,-\kappa)=
 \lim_{q\to p}\,
 [K(\sqrt{\kappa}\cdot |p-q|)+\log\,|p-q|]+
 \lim_{q\to p}\,[H_p(q)+ H(p,q,\kappa)]\tag{G.2}
\]
The last term above has the  "nice limit" 
$u_p(p)+H(p,p,\kappa)$ and from  (F.1)  the reader can  verify the limit formula:
\[
 \lim_{q\to p}\,
 [K(\sqrt{\kappa}\cdot |p-q|)+\log\,|p-q|]
 -\frac{1}{2}\cdot \log \kappa +\log 2-\gamma\tag{G.3}
\]
where $\gamma$ is  Euler's constant.

\bigskip

\noindent
{\bf{H. Final part of the proof.}}
Set 
\[
A=  +\log 2-\gamma+H_p(p)
\]
Then (G.1-3))
give
\[
F(p,-\kappa)= -\frac{1}{2}\cdot \log \kappa +A+H(p,p;\kappa)\tag{H.1}
\]
Above $\kappa>0$
and using $x=-\kappa$ in (E.2 ) we can  proceed  as follows.
To  begin with it is clear that
\[
s\mapsto A\cdot 
\frac{\sin \pi s}{2\pi^2}\cdot \int_a^\infty\,  x^{-s}\,dx
\]
is an entire function of $s$.
Next,  consider the function
\[ 
\rho(s)=
 -\frac{1}{2}\cdot 
\frac{\sin \pi s}{2\pi^2}\cdot \int_a^\infty\,  \log x\cdot x^{-s}\,dx
\]
Notice that the complex derivative
\[
\frac{d}{ds}\,  \int_a^\infty\,  x^{-s}\,dx=
- \int_a^\infty\,  \log x\cdot x^{-s}\,dx 
\]
\medskip
\noindent
{\bf{H.1 Exercise.}}
Use the last equality  to show that
\[
\rho(s)-\frac{1}{4\pi(s-1)}
 \]
is an entire function.
\medskip


\noindent
From the above we see that Theorem D.2  follows if we have proved
\medskip

\noindent
 {\bf{H.2 Lemma.}}
\emph{The following function  is entire}:
\[
s\mapsto \frac{\sin\,\pi s}{2\pi^2}\cdot
\int_a^\infty\, H(p,p,\kappa)\cdot  \kappa^{-s}\,d\kappa
\]
\medskip

\noindent
\emph{Proof.}
When $\kappa>0$
the equation (F.1) shows that $q\mapsto H(p,q;\kappa)$
is subharmonic  in $\Omega$ and the maximum principle gives
\[
0\leq  H(p,q;\kappa)\leq \max_{q\in\partial\Omega}\,K(\kappa|p-q|)\tag{i}
\]
Next, since $p\in\Omega$ is fixed there is 
a positive number
$\delta>0 $ such that
\[
|p-q|\geq\delta\,\colon q\in \partial\Omega
\]
and it follows from the above that we can take $q=p$ and obtain 
 \[
H(p,p;\kappa)\leq 
\int_1^\infty\, \frac{e^{-\delta\cdot t}}{\sqrt{t^2-1}}\,dt\tag{i}
\]
If we choose some $0<\alpha<\delta$ the reader can check that
(i) yelds a constant $b$ so that
\[
H(p,p;\kappa)\leq B\cdot e^{-\alpha\cdot \kappa}
\]
and finally it is clear that this exponential decay gives
Lemma H.2.



\newpage


\centerline{\bf{� B. Elliptic operators in
${\bf{R}}^3$}}.

\bigskip


\noindent
Consider an elliptic operator
\[
L=
\sum_{p=1}^{p=3}\sum_{q=1}^{q=3}\, a_{pq}(x)\cdot \frac{\partial ^2}{\partial x_p\partial x_q}+
\sum_{p=1}^{p=3}\, a_p(x)
\frac{\partial }{\partial x_p}+a_0(x)\tag{*}
\]
where the  $a$-functions are real-valued and
defined in a neighborhood of the closure of a bounded
domain
$\Omega$ with a $C^1$-boundary and satisfy 
the symmetry 
\[
a_{pq}=a_{qp}
\]
Moreover, we assume that 
$\{a_{pq}\}$ are
of class $C^2$, $\{a_p\}$  of class $C^1$ and $a_0$ is continuous.
The elliptic property of
 $L$ means that
for
every $x\in\overline{\Omega}$ the eigenvalues of the symmetric
matrix
$A(x)$ with elements $\{a_{pq}(x)\}$
are positive.
Under these conditions, a   result which goes back to work by 
Neumann and Poincar�,
gives
a positive constant
$\kappa_0$ such that
if $\kappa\geq \kappa_0$ then
the inhomogeneous equation
\[
L(u)-\kappa^2\cdot u=f\quad\colon f\in L^2(\Omega)\tag{0,1}
\]
has a unique solution $u$ which is a $C^2$.-function 
in
$\Omega$ and  extends to the closure where it is zero on
$\partial\Omega$.
Moreover, there exists some $\kappa_0$
and for each $\kappa\geq \kappa_0$ a
Green's function
$G(x,y;\kappa)$ such that
\[
(L-\kappa^2)(\frac{1}{4\pi}\cdot \int_\Omega\, 
G(x,y;\kappa)\, f(y)\, dy )= -f(x)\quad\colon f\in L^2(\Omega)\tag{0.2}
\]
This  means  that the bounded linear operator on
$L^2(\Omega$ defined by
\[
f\mapsto 
-\frac{1}{4\pi}\cdot \int_\Omega\, 
G(x,y;\kappa)\, f(y)\, dy\tag{0.3}
\]
is Neumann's resolvent
to the densely defined operator
$L-\kappa^2$ on the 
Hilbert space $L^2(\Omega)$.
The discrete sequence of eigenvalues $\{\lambda_n\}$ are in general 
and arranged with non-decreasing 
the absolute values. 
Notice that the operator $L$ in general is not symmetric
since  first order variabla  coefficients appear in  (*).
However, the eigenvalues are not too far from real numbers.
More precisely, in � xx we show that
that there exist positive constants
$C$ and $c$ such that
\[
|\mathfrak{Im}(\lambda_n)|\leq
C\cdot(\mathfrak{Re}(\lambda_n)+c)\tag{0.4}
\] 
hold for every $n$.
Next, the elliptic
hypothesis means that
the determinant function
\[
D(x)=\det(a_{p,q}(x))\tag{0.5}
\]
is positive in $\Omega$. With these notations one has

\medskip



\noindent
{\bf{B.0  Theorem.}}
\emph{The following limit formula holds:}
\[
\lim_{n\to\infty}\, \frac{\mathfrak{Re}(\lambda_n)}{n^{\frac{2}{3}}}
=\frac{1}{6\pi^2}\cdot \int_\Omega\, 
\frac{1}{\sqrt{D(x)}}\, dx\tag{*}
\]
\medskip

\noindent
{\bf{Remark.}}
This asymptotic  formula above was established by Courant and Weyl  when
$L$ is symmetric and was  extended to
non-symmetric operators by 
Carleman during  lectures at Institute Mittag-Leffler in  1935.
Weyl and Courant used calculus of variation
 while Carleman employed  different methods which
have the merit that the passage to the non-symmetric case
does not cause any  trouble. 
As pointed out by
Carleman the methods in the  proof  
give similar asymptotic formulas 
in   other boundary value problems such as
those considered by Neumann where one imposes boundary value conditions on
outer normals, and so on.
A crucial step during the  proof of the theorem above
relies upon  
\emph{constructions
of  a fundamental solutions} and we start with this while the proof of
Theorem B.0 is postponed until � XX.
Let us remark that  the subsewu8ent material includes a proof of the
existence of Green's function and
the cited results above from original work by Carl Neumann and Henri Poincar�.
\newpage



\newpage




\centerline{\bf{Some results about elliptic PDE-operators.}}

\bigskip

\noindent
{\bf{Introduction.}} We  expose material from two of 
Carleman's articles devoted to asymptotic distributions of eigenvalues to
elliptic boundary value problems.
The first section deals with the Laplace operator in
the plane  and 
in � B 
studies second order elliptic operators in
${\bf{R}}^3$ with variabe coefficients.
The crucial steps in the proofs rely upon analytic functiuon theory and
some ratehr explicit contructions where 
precise estimates for certain Green's functions are needed.
in addition to this one derives asymptotic formulas using
general facts about series, i.e. Tauberian theorems.
Apart from the results in Theorem A.1 and Theorem B.1
the subsequent material 
is worth  studying in detail by readers interested in PDE-theory
since it contains constructions of fundasemtnal soilutions 
which rely upon solutions to integral equations
via convergent Neumann series.
The merit is that this gives sharp estimates for
the fundamental solutions, and after also for Greens' functions
while one reagrds eigenfunctions to an elliptic operator in
bounded domains.
Let us remark that in � B we treat elliptic operators which in general are not symmetric.
The subsequent material  is not entirely self-contained since
we admit the existence of solutions to the Dirichlet problem for
positive elliptic operators while Greens' functions are constructed.
The existence of solution to the Dirchlet problem is exposed in
many text-books so we refrain from further comments about this
wellknown result.
In addition the asymptotic formulas in Theorem A.1 and
Theorem B.0
are derived after the "hard wortk" where various esitamtes are
proved, together with a general \emph{Taueberian  Theorem}
which also is taken for granted, i.e. here we only recall that this
is also exposed in many text-.books and here we shall use one version
of a Tauberian Theorem due to Norbert Wiener.



 
 
 
 



\bigskip




\[
L_f(p)=\int  \log\,\frac{1}{|p-q|}\cdot f(q)\,dq
\implies \Delta L_f= f
\]
valid all $f\in C^0$ with compact support.
Hiven DSitichlet domain $\Omega$.
Start from $f\in C^0)(\bar\Omega)$
and make $L_f$.
after solve dirisghlet for $L_f|\partial\Omega$
and subtract until we find
$H_\in C^0)(\bar\Omega)$
with $H_f=0$ on boundary and
$\Delta(H_f)= f$ in $\Omega$.
Solution is unique.
can then express
\[ 
H(p)= \int G(p,q,)f(q)\, dq
\]
Symmetry direct. Green's function and ON-basis for eigenvalues of all.











\centerline{\bf{� A. Eigenvalues and eigenfunctions for the Laplace operator
in ${\bf{R}}^3$}}.
\bigskip


\noindent
The result below  were  presented by Carleman at the
Scandinavian Congress in mathematics held in Stockholm 1934:
In
${\bf{R}}^2$ we 
consider a bounded Dirichlet regular domain
$\Omega$, i.e. every $f\in C^0(\partial\Omega)$
has a harmonic extension to $\Omega$.
For every fixed point $p\in\Omega$
one regards the continuous function
\[
q\mapsto \log\,\frac{1}{|p-q|}\quad\colon q\in \partial\Omega
\]
Thus gives a unique harmonic function
$x\mapsto H(p,x)$ in $\Omega$ such that 
\[
H(p,q)+\frac{1}{|p-q|}=0\quad\colon q\in \partial\Omega
\]
A wellknown  
fact  
established by
G. Neumann and H. Poincar�
during the years 1879-1895 shows that
the $H$-fuyction is symmetric, i.e.

\[
H(p,q)= H(q,p)
\]
holds, and moreover it extends to a continou function on
the product set
$\overline{\Omega}\times\overline{\Omega}$.
The Greens' function is defiuned by 
\[
G(p,q)= \log\,\frac{1}{|p-q|}+H(p,q)
\]
For each fixed $p\in\Omega$, the function $q\mapsto G(p,q)$ is super-harmonic and zero 
when $q\in \partial\Omega$. Hence the  minimum principe for superharmonic functions
entails that
\[
G(p,q)>0\quad\colon p,q\in \Omega
\]
Next, it is obvious that
\[
 \iint_{\Omega\times\Omega}\,|G(p,q)|^2\, dpdq<\infty
 \]
Hence the linear operator $\mathcal G$ on the Hilbert space $L^2(\Omega)$ defined by
the symmetric kernel $G(p,q)$
is of  Hilbert-Schmidt type  and therefore compact on
the Hilbert space $L^2(\Omega)$. 
Since the kernel  positive the eigenvalues
are positive, and  wellknown  facts about such nice integral operators
give
a sequence of pairwise orthogonal functions
$\{\phi_n\}$
whose $L^2$-norms are one and
\[
\mathcal G(\phi_n)=\mu_n\cdot \phi_n\tag{1}
\]
where
$\{\mu_n\}$
is a non-increasing sequence of positive eigenvalues which tend to zero.
Moreover, since the kernel $G(p,q)$ is positive it follows - again by general facts -
that $\{\phi_n\}$ is an orthonormal basis in
$L^2(\Omega)$, i.e. each real-valued $L^2$-function $f$ has an expansion
\[
f=\sum\, a_n\cdot \phi_n\quad\colon
a_n= \int_\Omega\, f_n(p)\cdot\phi_n(p)\, dp\tag{2}
\]
\medskip

\noindent
{\bf{Exercise.}}
Verify that each $\phi$-function extends to a continuous function on $\overline{\Omega}$
whose  boundary values are zero.
\medskip

\noindent
Next, let $\Delta$ be the Laplace operator.
Since
$\frac{1}{2\pi}\cdot \log\,|z|$ is a fundamental solution, it follows that
\[ 
\Delta\circ \mathcal G_f=-f\quad\colon f\in L^2(\Omega)\tag{3}
\]
Set
\[
\lambda_n= \mu_n^{-1}
\]
Then (1) and (3) give
\[ 
\Delta(\phi_n)=-\lambda_n\cdot \phi_n\quad\colon\, n=1,2,\ldots\tag{4}
\] 
where we now have 
$0<\lambda_1\leq \lambda_2\leq \ldots\}$.
\bigskip

\noindent
After these prelminary remarks  from classical theory
we announce Carleman's theorem.

\bigskip

\noindent
{\bf{A.1. Theorem. }}\emph{For every Dirichlet regular domain
$\Omega$ and each $p\in\Omega$�one has the limit formula}
\[ 
\lim_{N\to\infty}\, \lambda_N^{-1}\cdot \sum_{n=1}^{n=N}\, \phi_n(p)^2= \frac{1}{4\pi}\tag{*}
\]
\medskip

\noindent
To prove this  we consider some point $p\in\Omega$.
Since every $\phi_n$is harmonic and has $L^2$-norm one, the resder can check that eith
a fixed $p$ there exists a constant $C(p)$ such that
\[
\phi_n(p)^2\leq C(p)\quad\colon p=1,2,\ldots
\]
Hence there exists
the Dirichlet series
\[
\Phi_p(s)=\sum_{n=1}^\infty \frac{\phi_n(p)^2}{\lambda_n^s}
\]
which is analytic 
in the  half-space
$\mathfrak{Re}\, s>1$.
Less trivial is the following:

\medskip

\noindent
{\bf{A. 2 Theorem.}}
\emph{There exists an entire function
$\Psi_p(s)$ such that}
\[
\Phi_p(s)=\Psi_p(s)+\frac{1}{4\pi(s-1)}
\]
\medskip

\noindent
Let us first remark that Theorem A.2  gives Theorem A.1 by a general 
result  due to Norbert Wiener in the article
\emph{Tauberian theorem} [Annals of Math.1932].
His theorem 
asserts that if $\{\lambda_n\}$ is a non-decreasing sequence of
positive numbers which tends to infinity and
$\{a_n\}$ are non-negative real numbers such that
there exists the limit
\[ 
\lim_{s\to 1}\,(s-1)\cdot \sum\, \frac{a_n}{\lambda_n^s}=A
\] 
then it follows that
\[
\lim_{n\to \infty}\,\lambda_n^{-1}\cdot
\sum_{k=1}^{k=n}\, a_k=A
\]
\medskip

\centerline{\bf{Proof of Theorem A. 2}}

\bigskip

\noindent
Since $\mathcal G$ is a Hilbert-Schmidt operator a wellknown result due to Schur
gives
\[
\sum\, \lambda_n^{-2}<\infty \tag{i}
\]
This convergence entails that various constructions below are defined.
For each $\lambda$ outside the discrete set $\{\lambda_n\}$ we put
\[
G(p,q;\lambda)=
G(p,q)+
2\pi\lambda\cdot \sum_{n=1}^\infty\,
\frac{\phi_n(p)\cdot \phi_n(q)}{\lambda_n(\lambda-\lambda_n)}\tag{ii}
\]
This gives the integral operator
$\mathcal G_\lambda$ defined on $L^2(\Omega)$ by
 \[ 
 \mathcal G_\lambda(f)(p)
 =\frac{1}{2\pi}\cdot \iint_\Omega\, G(p,q;\lambda )\cdot f(q)\, dq\tag{iii}
\]

\medskip

\noindent
{\bf{A. Exercise.}} Use that the eigenfunctions $\{\phi_n\}$ is an orthonormal basis in
$L^2(\Omega)$ to show that
\[
(\Delta+\lambda)\cdot \mathcal G_\lambda=-E
\]


\noindent{\bf{B. The function $F(p,\lambda)$}}.
Set
\[ 
F(p,q,\lambda)= G(p,q;\lambda)- G(p,q)
\]
Keeping $p$ fixed we see that (ii) gives
\[
\lim_{q\to p}\, F(p,q,\lambda)=
2\pi\lambda\cdot \sum_{n=1}^\infty\,
\frac{\phi_n(p)^2}{\lambda_n(\lambda-\lambda_n)}\tag{B.1}
\]
Set
\[
F(p,\lambda)=
\lim_{q\to p}\, F(p,q,\lambda)\tag{B.2}
\]
From (i) and (B.1) it follows that (B.2)  is a meromorphic function in
the complex $\lambda$-plane with at most simple poles
at $\{\lambda_n\}$.
\medskip

\noindent{\bf{C. Exercise.}}
Let $0<a<\lambda_1$. Use  residue calculus to show
the equality below in the  half-space
$\mathfrak{Re}\, s>2$:
\[ 
\Phi(s)=
\frac{1}{4\pi^2 \cdot i}\cdot \int_{a-i\infty}^{a+i\infty}\, 
F(p,\lambda)\cdot \lambda^{-s}\, d\lambda\tag{C.1}
\]
where the line integral  is taken on the vertical  line
$\mathfrak{Re}\,\lambda=a$.

\medskip

\noindent
{\bf{D. Change of contour integrals.}}
At this stage we employ a device which goes to
Riemann and
move the integration into the half-space
$\mathfrak{Re}(\lambda)<a$.
Consider  the curve $\gamma_+$
defined as the union of the
negative real interval $(-\infty,a]$ followed by
the upper
half-circle $\{\lambda= ae^{i\theta}\,\colon 0\leq\theta\leq \pi \}$
and the 
half-line $\{\lambda= a+it\,\colon t\geq 0\}$.
Cauchy's theorem entails that 
\[ 
\int_{\gamma_+}\, F(p,\lambda)\cdot \lambda^{-s}\, d\lambda=0
\]
We leave it to the reader to contruct the
similar
curve
$\gamma_-=\bar \gamma_+$. Using 
the vanishing of these line integrals and taking the branches of the 
multi-valued function
$\lambda^s$ into the account the reader should verify the following:

\medskip


\noindent
{\bf{E. Lemma.}}
\emph{One has the equality}
\[ 
\Phi(s)=\frac{a^{s-1}}{4\pi}\cdot \int_{-\pi}^\pi\,
F(ae^{i\theta})\cdot e^{(i(1-s)\theta}\,d\theta
+
\frac{\sin \pi s}{2\pi^2}\cdot \int_a^\infty\, F(p,-x)\cdot x^{-s}\,dx\tag{E.1}
\]
\medskip

\noindent
The first term in the sum of the right hand side of (E.1)
is obviously an entire function of $s$. So there remains to
prove that
\[
 s\mapsto  \frac{\sin \pi s}{2\pi^2}\cdot \
 \int_a^\infty\, F(p,-x)\cdot x^{-s}\,dx\tag{E.2}
\]
is meromorphic with
a single pole at $s=1$ whose residue is $\frac{1}{4\pi}$.
To attain this we  express $F(p,-x)$ when $x$ are real and positive in another way.
\medskip

\noindent
{\bf{F.  The $K$-function.}}
In the half-space $\mathfrak{Re}\,z>0$ there exists the analytic function
\[
K(z)= \int_1^\infty\, \frac{e^{-zt}}{\sqrt{t^2-1}}\,dt
\]
\medskip

\noindent
{\bf{Exercise.}}
Show that $K$ extends to a multi-valued analytic function outside
$\{z=0\}$ given by
\[
K(z)=-I_0(z)\cdot \log z+ I_1(z)\tag{F.1}
\] 
where $I_0$ and $I_1$ are entire functions
with series expansions
\[
I_0(z)=\sum_{m=0}^\infty\, \frac{2^{-2m}}{(m!)^2}\cdot
z^{2m}\tag{i}
\]
\[ 
I_1(z)= \sum_{m=0}^\infty\, \rho(m)\cdot
\frac{2^{-2m}} {(m!)^2} \cdot z^{2m}\quad
\colon \rho(m)=1+\frac{1}{2}+\ldots+\frac{1}{m}-\gamma\tag{ii}
\]
where $\gamma$ is the usual Euler constant.

\bigskip


\noindent
With  $p$ kept fixed and $\kappa>0$ 
we solve the Dirichlet problem and find
a  function $q\mapsto H(p,q;\kappa)$ which satisfies  the
equation
\[
 \Delta(H)-\kappa\cdot H=0\tag{F.2}
\] 
in $\Omega$ with boundary values
\[ 
H(p,q;\kappa)=K(\sqrt{\kappa}|p-q|)\quad\colon q\in \partial\Omega
\]


\noindent
{\bf{G. Exercise.}}
Verify the equation
\[ 
G(p,q;-\kappa)=K(\sqrt{\kappa}\cdot |p-q|)- H(q;\kappa)\quad\colon \kappa>0
\]



\noindent
Next,   the construction of $G(p,q)$ gives
\[
 F(p,-\kappa)=
 \lim_{q\to p}\,
 [K(\sqrt{\kappa}\cdot |p-q|)+\log\,|p-q|]+
 \lim_{q\to p}\,[u_p(q)+ H(p,q,\kappa)]\tag{G.1}
\]
The last term above has the  "nice limit" 
$u_p(p)+H(p,p,\kappa)$ and from  (F.1)  the reader can  verify the limit formula:
\[
 \lim_{q\to p}\,
 [K(\sqrt{\kappa}\cdot |p-q|)+\log\,|p-q|]=
 -\frac{1}{2}\cdot \log \kappa +\log 2-\gamma\tag{G.2}
\]
where $\gamma$ is  Euler's constant.

\bigskip

\noindent
{\bf{H. Final part of the proof.}}.
Set $A=  +\log 2-\gamma+u_p(p)$. Then (G.1) and (G.2)
give
\[
F(p,-\kappa)= -\frac{1}{2}\cdot \log \kappa +A+H(p,p;-\kappa)
\]
With $x=\kappa$ in (E.2 ) we  proceed  as follows.
To  begin with it is clear that
\[
s\mapsto A\cdot 
\frac{\sin \pi s}{2\pi^2}\cdot \int_a^\infty\,  x^{-s}\,dx
\]
is an entire function of $s$.
Next,  consider the function
\[ 
\rho(s)=
 -\frac{1}{2}\cdot 
\frac{\sin \pi s}{2\pi^2}\cdot \int_a^\infty\,  \log x\cdot x^{-s}\,dx
\]
Notice that the complex derivative
\[
\frac{d}{ds}\,  \int_a^\infty\,  x^{-s}\,dx=
- \int_a^\infty\,  \log x\cdot x^{-s}\,dx
\]

\medskip
\noindent
{\bf{H.1 Exercise.}}
Use the  above to show that
\[
\rho(s)-\frac{1}{4\pi(s-1)}
 \]
is an entire function.
\medskip


\noindent
From the above we see that Theorem D.2  follows if we have proved
\medskip

\noindent
 {\bf{H.2 Lemma.}}
\emph{The following function  is entire}:
\[
s\mapsto \frac{\sin\,\pi s}{2\pi^2}\cdot
\int_a^\infty\, H(p,p,\kappa)\cdot  \kappa^{-s}\,d\kappa
\]
\medskip

\noindent
\emph{Proof.}
When $\kappa>0$
the equation (F.1) shows that $q\mapsto H(p,q;\kappa)$
is subharmonic  in $\Omega$ and the maximum principle gives
\[
0\leq  H(p,q;\kappa)\leq \max_{q\in\partial\Omega}\,K(\kappa|p-q|)\tag{i}
\]
With  $p\in\Omega$ fixed there is 
a positive number
$\delta$ such that
$|p-q|\geq\delta\,\colon q\in \partial\Omega$ which  
gives
positive constants
$B$ and  $\alpha$  such that
\[
H(p,p;\kappa)\leq e^{-\alpha\kappa}\quad\colon \kappa>0\tag{ii}
\]
The reader may now check that this
exponential decay gives Lemma H.2.



\medskip

\noindent
{\bf{A final remark.}}
As pointed out by Carleman in his cited article
\emph{La m�thode dont nous sommes servis est aussi applicable � une
�quation elliptique � un nombre quelconque de dimensions}.
The reader is for example invited to find the companion to Theorem  A.1
for Dirichlet regular domains in ${\bf{R}}^n$ when $n\geq 3$.



\newpage


\centerline{\bf{� B. Elliptic operators in
${\bf{R}}^3$}}.

\bigskip


\noindent
Consider an elliptic operator
\[
L=
\sum_{p=1}^{p=3}\sum_{q=1}^{q=3}\, a_{pq}(x)\cdot \frac{\partial ^2}{\partial x_p\partial x_q}+
\sum_{p=1}^{p=3}\, a_p(x)
\frac{\partial }{\partial x_p}+a_0(x)\tag{*}
\]
where the  $a$-functions are real-valued and
defined in a neighborhood of the closure of a bounded
domain
$\Omega$ with a $C^1$-boundary and satisfy 
the symmetry 
\[
a_{pq}=a_{qp}
\]
Moreover, we assume that 
$\{a_{pq}\}$ are
of class $C^2$, $\{a_p\}$  of class $C^1$ and $a_0$ is continuous.
The elliptic property of
 $L$ means that
for
every $x\in\overline{\Omega}$ the eigenvalues of the symmetric
matrix
$A(x)$ with elements $\{a_{pq}(x)\}$
are positive.
Under these conditions, a   result which goes back to work by 
Neumann and Poincar�,
gives
a positive constant
$\kappa_0$ such that
if $\kappa\geq \kappa_0$ then
the inhomogeneous equation
\[
L(u)-\kappa^2\cdot u=f\quad\colon f\in L^2(\Omega)\tag{0,1}
\]
has a unique solution $u$ which is a $C^2$.-function 
in
$\Omega$ and  extends to the closure where it is zero on
$\partial\Omega$.
Moreover, there exists some $\kappa_0$
and for each $\kappa\geq \kappa_0$ a
Green's function
$G(x,y;\kappa)$ such that
\[
(L-\kappa^2)(\frac{1}{4\pi}\cdot \int_\Omega\, 
G(x,y;\kappa)\, f(y)\, dy )= -f(x)\quad\colon f\in L^2(\Omega)\tag{0.2}
\]
This  means  that the bounded linear operator on
$L^2(\Omega$ defined by
\[
f\mapsto 
-\frac{1}{4\pi}\cdot \int_\Omega\, 
G(x,y;\kappa)\, f(y)\, dy\tag{0.3}
\]
is Neumann's resolvent
to the densely defined operator
$L-\kappa^2$ on the 
Hilbert space $L^2(\Omega)$.
The discrete sequence of eigenvalues $\{\lambda_n\}$ are in general 
and arranged with non-decreasing 
the absolute values. 
Notice that the operator $L$ in general is not symmetric
since  first order variabla  coefficients appear in  (*).
However, the eigenvalues are not too far from real numbers.
More precisely, in � xx we show that
that there exist positive constants
$C$ and $c$ such that
\[
|\mathfrak{Im}(\lambda_n)|\leq
C\cdot(\mathfrak{Re}(\lambda_n)+c)\tag{0.4}
\] 
hold for every $n$.
Next, the elliptic
hypothesis means that
the determinant function
\[
D(x)=\det(a_{p,q}(x))\tag{0.5}
\]
is positive in $\Omega$. With these notations one has

\medskip



\noindent
{\bf{B.0  Theorem.}}
\emph{The following limit formula holds:}
\[
\lim_{n\to\infty}\, \frac{\mathfrak{Re}(\lambda_n)}{n^{\frac{2}{3}}}
=\frac{1}{6\pi^2}\cdot \int_\Omega\, 
\frac{1}{\sqrt{D(x)}}\, dx\tag{*}
\]
\medskip

\noindent
{\bf{Remark.}}
This asymptotic  formula above was established by Courant and Weyl  when
$L$ is symmetric and was  extended to
non-symmetric operators by 
Carleman during  lectures at Institute Mittag-Leffler in  1935.
Weyl and Courant used calculus of variation
 while Carleman employed  different methods which
have the merit that the passage to the non-symmetric case
does not cause any  trouble. 
As pointed out by
Carleman the methods in the  proof  
give similar asymptotic formulas 
in   other boundary value problems such as
those considered by Neumann where one imposes boundary value conditions on
outer normals, and so on.
A crucial step during the  proof of the theorem above
relies upon  
\emph{constructions
of  a fundamental solutions} and we start with this while the proof of
Theorem B.0 is postponed until � XX.
Let us remark that  the subsewu8ent material includes a proof of the
existence of Green's function and
the cited results above from original work by Carl Neumann and Henri Poincar�.
\newpage




\centerline {\bf{� B.1. Fundamental solutions to second order
Elliptic operators.}}
\bigskip


\noindent
Consider as an  elliptic operator 
\[
L=
\sum_{p=1}^{p=3}\sum_{q=1}^{q=3}\, a_{pq}(x)\cdot \frac{\partial ^2}{\partial x_p\partial x_q}+
\sum_{p=1}^{p=3}\, a_p(x)
\frac{\partial }{\partial x_p}+a_0(x)\tag{B.1.1}
\]
where 
$a$-functions are real-valued and
one has the symmetry $a_{pq}=a_{qp}$.
To ensure existence of a \emph{globally defined} fundamental solutions we
suppose the $a$-functions are defined in the whole space
${\bf{R}}^3$ and the 
 following limit formulas hold
as $|x|\to \infty$ in ${\bf{R}}^3$.
\[
\lim a_\nu(x)=0 \colon 0\leq p\leq 3\quad\colon\,
\lim a_{pq}(x)= \text{Kronecker's delta function}
\]
Thus, $L$ approaches the Laplace operator as $|x\to +\infty $.
The elliptic property 
means that
the eigenvalues of the symmetric
matrix with elements $\{a_{pq}(x)\}$
are positive for every $x$.
Next, recall the  notion of fundamental solutions.
First the adjoint
operator is defined by:
\[ 
L^*(x,\partial _x)=L-2\cdot \bigl(
\sum_{p=1}^{p=3}\, \bigl(\sum_{q=1}^{q=3}\, 
\frac{\partial a_{pq}}{\partial x_q}\bigr)\cdot \frac{\partial}{\partial x\uuu p}
-\sum_{p=1}^{p=3}\, \frac{\partial a_p}{\partial x_p}
+2\cdot \sum\sum\, \frac{\partial^2 a_{pq}}{\partial x_p\partial x_q}\tag{0.1}
\]
Partial integration gives  the equation  below for every pair of
$C^2$-functions $\phi,\psi $ in ${\bf{R}}^3$ with compact support:
\[
\int\, L(\phi)\cdot \psi\, dx=
\int\, \phi\cdot L^*(\psi)\, dx\tag{0.2}
\] 
with volume integrals  taken over
${\bf{R}}^3$.
By definition  a locally integrable function $\Phi(x)$ in ${\bf{R}}^3$
is  a fundamental solution to $L(x,\partial _x)$
if
\[
\psi(0)=\int\, \Phi\cdot L^*(\psi)\, dx\tag{0.3}
\] 
hold for every $C^2$-function  $\psi$ with compact support.
Next, to each 
positive number
$\kappa$ we  get  the PDE-operator $L-\kappa^2$ and a
function $x\mapsto \Phi(x;\kappa)$ is a fundamental solution to $L-\kappa^2$
if
\[
\psi(0)=\int\, \Phi(x:\kappa)\cdot (L^*-\kappa^2)(\psi(x))\, dx\tag{0.4}
\] 
hold for compactly supported $C^2$-functions $\psi$.
Finally, the  origin can  replaced by a variable point $\xi$ in
${\bf{R}}^3$ and then one seeks
a function
$\Phi(x,\xi;\kappa)$ with the property that
\[
\psi(\xi)=\int\, \Phi(x,\xi;\kappa)\cdot (L^*(x,\partial_x)-\kappa^2)(\psi(x))\, dx\tag{0.5}
\] 
hold for all  $\xi\in{\bf{R}}^3$
and every $C^2$-function $\psi$ with compact support.
Keeping $\kappa$ fixed this means that  
$\Phi(x,\xi;\kappa)$ is a function of 6 variables  defined in 
${\bf{R}}^3\times {\bf{R}}^3$.
We are going to construct
$\Phi(x,\xi;\kappa)$
in a canonical way, starting from Isaac Newton's
formulas for  elliptic operators with
constant coefficients.





\bigskip


\centerline {\bf{1. The case with constant coefficients.}}

\medskip



\noindent
Following contructions  from 
Newton's  famous
text-books from 1666. Suppose thst the $a$-funvtions in (B.1.1) are constants.
In particular we have the  
positive  and symmetric $3\times 3$-matrix
$A= \{a_{pq}\}$. Let
$B=\{b_{pq}\}$ be the  inverse matrix and define the quadratic form
 \[ 
 B(x)= \sum_{p,q}\, b_{pq} x_px_q
\]
and noticr that it is positive definite.
next, put
\[
\alpha=\sqrt{\kappa^2+\frac{1}{2}\, \sum_{p,q}\, b_{pq} a_pa_q
-a_0}
\]
where $\kappa$ is chosen so large that
the term under the square-root is $>0$.
With $\kappa$ chosen as above this gives the locally integrable function
\[
H(x;\kappa)= \frac{1}{4\pi\cdot \sqrt{\Delta\cdot B(x)}}
\cdot e^{-\alpha \sqrt{B(x)}-
\frac{1}{2}\sum_{p,q}b_{pq} a_p\cdot x_q}\tag{1.1} 
\]


\medskip




\noindent
{\bf{Exercise.}} Verify via  Stokes  formula
that $H(x;\kappa)$  yields  a fundamental solution
to the PDE-operator 
\[
L(\partial_x)-\kappa^2=
\sum_{p=1}^{p=3}\sum_{q=1}^{q=3}\, a_{pq}\cdot \frac{\partial ^2}{\partial x_p\partial x_q}+
\sum_{p=1}^{p=3}\, a_p
\frac{\partial }{\partial x_p}+a_0-\kappa^2
\]



\bigskip

\centerline {\bf{1.2 The case with variable coefficients.}}
\bigskip



\noindent
Now $L$ is given as in (B.1.1)
with varaible coeffcients.
For each $\xi\in{\bf{R}}^3$ the elements of the inverse matrix
to $\{a_{pq}(\xi)$
are denoted by $\{b_{pq}(\xi)\}$. Thee assumption in (B.1.2) 
entails that there exists some 
$\kappa_0>0$ such that
\[
\kappa_0^2+\frac{1}{2}\, \sum_{p,q}\, b_{pq}(\xi) a_p(\xi)a_q(\xi)
-b(\xi)>0\quad\text{hold for all}\quad  \xi\in{\bf{R}}^3\tag{1.2.1}
\] 
Next, for  every $\kappa\geq \kappa_0$ we set
\[
\alpha_\kappa(\xi)=
\sqrt{\kappa^2+\frac{1}{2}\, \sum_{p,q}\, b_{pq}(\xi) a_p(\xi)a_q(\xi)
-b(\xi)}\tag{i}
\]
Following Newton's construction in (1.1) we
put:
\[
H(x,\xi;\kappa)=\frac{1}{4\pi}\cdot
 \frac{\sqrt{\Delta(\xi)}^{-\frac{1}{2}}}{
\sqrt{ \sum_{p,q}\, b_{pq}(\xi)\cdot x_px_q}}
\cdot e^{-\alpha_\kappa(\xi) \sqrt{B(x)}-
\frac{1}{2}\sum_{p,q}b_{pq}(\xi) a_p(\xi)\cdot x_q} \tag{ii}
\]

\noindent
When $\xi$ is kept fixed this  function of
$x$ is real analytic  outside the origin and we  notice
that $x\to H(x,\xi;\kappa)$ is locally integrable
as a function of $x$ in a neighborhood of the origin.
We are going to find a fundamental solution
satisfying (0.5)
which takes the form
\[
\Phi(x,\xi;\kappa)=
H(x-\xi,\xi;\kappa)+\int_{{\bf{R}}^3}\, 
H(x-y,\xi;\kappa)\cdot\Psi(y,\xi;\kappa)\, dy\tag{1.2.2}
\]
where the $\Psi$-function is the solution to an integral equation
to be given  (1.5) below. But first we need some further constructions.


\medskip

\noindent
{\bf{1.3 The function $F(x,\xi;\kappa)$.}}
For every fixed $\xi$ we consider the  differential operator in the
$x$-space:
\[ 
L_*(x,\partial_x,\xi;\kappa)=
\]
\[\sum_{p=1}^{p=3}\sum_{q=1}^{q=3}\, (a\uuu{pq}(x)-
(a\uuu{pq}(\xi))\cdot 
\frac{\partial^2}{\partial x_p\partial x_q}+
\sum_{p=1}^{p=3}\,
(a_p(x)-a_p(\xi))\frac{\partial}{\partial x_p}+ (b(x)-b(\xi))
\]

\medskip

\noindent
With $\xi$ fixed we apply $L_*$ to the function
$x\mapsto H(x-\xi,\xi;\kappa)$
and put
\[ 
F(x,\xi;\kappa)=\frac{1}{4\pi}\cdot L_*(x,\partial_x,\xi;\kappa)(H(x-\xi,\xi,\kappa)) \tag{1.3.1}
\]

\bigskip

\noindent
{\bf{1.4 Two  estimates.}}
The limit conditions  in (0.0)     give  positive constants
$C,C_1$ and $k$ such that the following hold when $\kappa\geq\kappa_0$:
\[
|H(x-\xi,\xi;\kappa)|\leq C\cdot \frac{e^{-k\kappa|x-\xi|}}{|x-\xi|}
\quad\colon\quad \,
[F(x,\xi;\kappa)|\leq C_1\cdot 
\frac{e^{-k\kappa|x-\xi|}}{|x-\xi|^2}
\tag{1.4.1}
\]


\noindent
The verification of (1.4.1) is left as an exercise.



\newpage



\noindent
\centerline {\bf{1.5 An integral equation.}}
\medskip


\noindent
With $F$ defined in (1.3.1)
we shall  prove
the following:

\medskip





\noindent
{\bf{1.5.0 Theorem.}} \emph{There exists a positive number
$\kappa_0^*$ such that
the integral equation below has a solution for each
$\kappa\geq \kappa_0^*$ }
\[ 
\Psi(x,\xi;\kappa)= \int_{{\bf{R}}^3}\,  F(x,y;\kappa)\cdot \Psi(y,\xi;\kappa)\,dy+F(x,\xi;\kappa)\tag{1.5.1}
\]


\medskip

\noindent
\emph{Proof.}
We construct 
the Neumann series of $F$.
Thus, starting with $F^{(1)}=F$ we set
\[
F^{(\nu)}(x,\xi;\kappa)=\int_{{\bf{R}}^3}\, F(x,y;\kappa)\cdot
F^{(\nu-1)}(y,\xi;\kappa)\, dy\quad\colon\quad \nu\geq 2\tag{1.5.2}
\]
The last inequality in  (1.4.1 ) gives 
\[
|F^{(2)}(x,\xi;\kappa)|
\leq C_1^2\iiint
\frac{e^{-k\kappa|\xi-y|}}{|x-y|^2\cdot |\xi-y|^2}\cdot dy\tag{i}
\]


\noindent
To estimate (i) we  notice that the triple integral after
the substitution $y-\xi\to u$
becomes
\[
C_1^2\iiint
\frac{e^{-k\kappa|u|^2}}{|x-u-\xi|^2\cdot |u|^2}\cdot du\tag{ii}
\]


\noindent
In  (ii)  the volume integral can be integrated in polar
coordinates
and becomes
\[
C_1^2\cdot \int_0^\infty\int_{S^2}\, 
\frac{e^{-k\kappa r^2}}{|x-r\cdot w-\xi|^2}\cdot dwdr\tag{iii}
\]
where $S^2$ is the unit sphere and $dw$ the area measure on
$S^2$ and we see that (iii) becomes
\[
2\pi C_1^2\cdot
\int_0^\infty\int_0^\pi\, 
\frac{e^{-k\kappa r}}{(x-\xi)^2+r^2-
2r\cdot |x-\xi|\cdot \sin\theta}\cdot d\theta dr=
\]
\[
\frac{2\pi C_1^2}{|x-\xi|}\cdot\int_0^\infty\, e^{-k\kappa |x-\xi|t}\cdot
\log\, |\frac{1+t}{1-t}|\cdot \frac{dt}{t}\tag{iv}
\]
where the last equality follows by a straightforward computation.

\medskip


\noindent
{\bf{1.6 Exercise.}}
Show that (iv) gives the estimate
\[
|F^{(2)}(x,\xi;\kappa)|\leq \frac{2\pi\cdot  C_1^2\cdot C_1^*}{\kappa\cdot |x-\xi|^2}
\]
where $C_1^*$ is a fixed positive constant
which is independent of $x$ and $\xi$ and 
show by an induction over $n$
that one has:
\[
|F^{(n)}(x,\xi;\kappa)|\leq\frac{C_1}{|x-\xi|^2}\cdot 
\bigl[\frac{2\pi C_1^2\cdot C_1^*}{\kappa}\bigr]^{n-1}
\quad\text{hold for every}\quad  n\geq 2\tag{1.6.1}
\]
\medskip

\noindent
{\bf{The choice of
$\kappa_0^*$}}.
It is taken so that 
$\kappa_0^*$  so large that
\[
2\pi C_1^2\cdot C_1^*<\kappa_0^*
\]
Then
(1.6.1) entails that
the Neumann series
\[
\sum_{n=1}^\infty F^{(n)}(x,\xi;\kappa)
\]
converges when 
$\kappa\geq \kappa_0^*$ 
and
gives the requested solution $\Psi(x,\xi;\kappa)$ in (1.5.1).


\newpage


\noindent
{\bf{1.7 Conclusion.}}
Above we  have found 
$\Psi$ which satisfies the integral equation in � 1.5.1
Using   Green's formula the reader may verify:
\medskip

\noindent
{\bf{1.8 Proposition.}}
\emph{With $\Psi$ as above it follows that
the function
$\Phi(x,\xi;\kappa)$ defined in (1.2.2) is   a fundamental solution
of $L(x,\partial_x)-\kappa^2$.}

\bigskip



\centerline {\bf {1.9 Some  estimates.}}
\bigskip


\noindent
The  constructions above show that
the  functions
\[
x\mapsto \Phi(x,\xi;\kappa)\quad\text{and}\quad 
x\mapsto H(x-\xi,\xi;\kappa)
\]


\noindent
have the same  singularities at $x=\xi$.
Consider the difference
\[
G(x,\xi;\kappa)=\Phi(x,\xi;\kappa)-
H(x-\xi,\xi;\kappa)\tag{1.9.1}
\]
\medskip


\noindent
{\bf{1.9.2 Exercise.}}
Use the previous constructions to show
that for every $0<\gamma\leq 2$
there is a constant $C_\gamma$
such that
\[
\bigl |G(x,\xi;\kappa)\,\bigr|\leq \frac{C_\gamma}{(\kappa|x-\xi|)^{\gamma}}
\]
hold for every pair $(x,\xi)$ and every $\kappa\geq \kappa_0$.
Together with the  the inequality for the
$H$-function in (1.4.1)
we arrive at the following
estimate for the fundamental solution $\Phi$.

\medskip

\noindent
{\bf{1.9.3 Theorem.}} \emph{With $\kappa_0^*$ chosen from the proof of
Theorem 1.5.0 
there exist positive constants $C$ and $k$ and
for each $0<\gamma\leq 2$ a constant $C_\gamma$ such that}
\[
|\Phi(x,\xi;\kappa)|\leq
C\cdot \frac{e^{-k\kappa|x-\xi|}}{|x-\xi|}+
 \frac{C_\gamma}{(\kappa|x-\xi|)^{\gamma}}
\]
\emph{hold for all pairs $(x,\xi)$ in ${\bf{R}}^3$ and every
$\kappa\geq \kappa_0^*$.}


\medskip

\noindent
{\bf{Remark.}}
Above $C$ and $k$ are independent of
$\kappa$ as soon as $\kappa_0^*$ has been chosen as above.
The size of these constant  depend on
the $C^2$-norms of the functions $\{a_{pq}(x)\}$ and 
as well as the $C^1$-norms of $\{a_1,a_2,a_3\}$ and the maximum norm of $a_0$.
The merit in Theorem 1.9.3 is that one gets a control both
in a finite region as well as
the behaviour of $\Phi$ when $|x-\xi|$ gets large where one has the damping
exponential factor which
is useful during constructions of Green's functions for exterior boundary
value problems.
Notice that the whole construction is canonical.
Let us  finally remark that similar constructions as above can be carried out for
elliptic operators
of even degree $2m$ when  $m\geq 2$. Here
Newton's solution for constant  coefficients is replaced by those
 of
Fritz John which arise via the wave deompostion of the Dirac measure.
It would be interesting to analyze the resulting version of Theorem 1.9.3, i.e. to exhibit
estimates of a similar nature when $m\geq 2$.
Of course, one  can also extend
everything to
elliptic operators of $n$ variables were $n\geq 4$
in which case the
denominator $|x-\xi|^{-1}$ is replaced by
$|x-\xi|^{-n+2}$, starting  from 
Newton's fundamental solution for the case of second order
elliptic operators
with constant
constant coefficients in
${\bf{R}}^n$.

\newpage

\centerline {\bf{� B.2. Green's functions.}}
\bigskip


\noindent
Let $\Omega$ be a bounded domain in
${\bf{R}}^3$.
A Green's function $G(x,y;\kappa)$ attached to this domain
and the elliptic PDE-operator $L(x,\partial_x)$
is a function which for fixed $\kappa$ is a function in 
$\Omega\times\Omega$
with the following properties:
\[ 
G(x,y;\kappa)=0\quad\text{when}\quad x\in\partial \Omega\quad \text{and}\quad
y\in\Omega\tag{*}
\]
\[
\psi(y)=\int_\Omega\, (L^*(x,\partial_x)-\kappa^2)
(\psi(x))\cdot G(x,y;\kappa)\, dx\quad\colon\quad y\in\Omega\tag{**}
\]


\noindent
hold for all $C^2$-functions $\psi$ with compact support in
$\Omega$.
To find $G$ one 
solves a   Dirchlet problem.
With
$\xi\in\Omega$ kept fixed we have a continuous function on
$\partial\Omega$:
$x\mapsto \Phi(x,\xi;\kappa)$.
Solving Dirchlet's problem gives  a unique $C^2$-function
$w(x)$ which satisfies:
\[ 
L(x,\partial_x)(w)+\kappa^2\cdot w=0 \quad\text{holds  in}\quad \Omega
\quad\text{and}\quad w(x)= \Phi(x,\xi;\kappa)\quad\colon x\in\partial\Omega=0
\]

\medskip

\noindent
From the above
we get the requested Greens'-function, i.e. 
the reader can check the result below.
\medskip

\noindent
{\bf{2.1 Proposition.}} \emph{The 
function }
$
G(x,\xi;\kappa)=
\Phi(x,\xi;\kappa)-w(x)
\quad \text{satisfies}\quad (*-**)$

\medskip
 
\noindent
Next,
using the estimates for the $\Phi$-function in Theorem 1.9.3
we shall establish 
estimates for the $G$-function above where
we start with  a sufficiently large
$\kappa_0$ so  that 
$\Phi^*(x,\xi;\kappa_0)$ is a positive function of
$(x,\xi)$. 
\medskip


\noindent
{\bf{2.2 Theorem.}}
\emph{One has}
\[ 
G(x,\xi;\kappa_0)=
\frac{1}{\sqrt{\Delta(x)}\cdot\sqrt{\Phi(x,\xi;\kappa_0)}}
+R(x,\xi)
\]

\noindent
\emph{where the remainder function satisfies the following for all pairs
$(x,\xi)$ in $\Omega$:}
\[ 
|R(x,\xi)|\leq C\cdot |x-\xi|^{-\frac{1}{4}}
\]
\emph{with a    constant $C$ which   depends on the  domain 
$\Omega$ and
the PDE-operator $P$.}
\medskip

\noindent
{\bf{2.3 Exericse.}}
Above the negative power  of 
$|x-\xi|$ is  a fourth-root which means that  the remainder term $R$ 
is more regular  compared
to the first term which behaves like $|x-\xi|^{-1}$ on the diagonal $x=\xi$.
The proof is left to the reader. If necessary, consult
[Carleman: page 125-127]. 

\medskip

\noindent
{\bf{2.4 The integral operator
$\mathcal J$}}. 
Theorem 2.2 enable us to study 
the integral operator
which sends a function
$u$ in $\Omega$ to 
\[
\mathcal J_u(x)=\int_\Omega\, G(x,\xi;\kappa_0)\cdot u(\xi)\, d\xi
\]


\noindent
The construction of the Green's function gives:
\[
(L(x,\partial_x)-\kappa_0^2)(\mathcal J_u)(x)=u(x)\quad\colon\quad x\in\Omega\tag{2.4.1}
\] 
Thus,  if $E$ denotes the identity we have
the operator equality
\[
L(x,\partial_x)\circ \mathcal J_u=\kappa_0^2\cdot  \mathcal J+E\tag{2.4.2}
\]

\noindent
Consider a  real number
$\gamma$ and some $u$-function which satisfies:
\[
u(x)+ \gamma\cdot \mathcal J_u(x)=0\quad\colon\quad x\in\Omega\tag{2.4.3}
\] 
The vanishing from (*) in � B.2 for the $G$-function 
implies that $J_u(x)=0$ on $\partial\Omega$. Hence  every
$u$-function which satisfies
in (2.4.3) for some  constant $\gamma$
vanishes   on $\partial\Omega$.
Next, when $L$ is applied to
(2.4.3) 
the operator equation  (2.4.2) gives

\[
0=P(u)+\gamma \cdot \kappa_0^2\cdot \mathcal J_u+\gamma\cdot u\implies
P(u)+(\gamma-\kappa_0^2)u=0
\]
\medskip

\noindent
{\bf{2.4.4 Conclusion.}}
The boundary value problem (*) from 0.B
is equivalent to find eigenfunctions of 
$\mathcal J$ via (2.4.3) above.

\medskip

\noindent
\centerline {\bf{3. Almost reality of eigenvalues.}}
\medskip

\noindent
Consider the set of eigenvalues $\lambda$ to (*) in (0.B).
Then we have:

\medskip

\noindent
{\bf{3.1 Proposition.}}
\emph{There exist positive constants $C_*$ and $c_*$ such that
every eigenvalue  $\lambda$ to  (*) in (0.B) satisfies}

\[
|\mathfrak{Im}\,\lambda|^2\leq C_*(\mathfrak{Re}\,\lambda)+c_*)
\]
\medskip

\noindent
\emph{Proof.}
Let $u$ be an eigenfunction where
$P(u)+\lambda\cdot u=0$.
Stokes theorem and the vanishing of $u|\partial\Omega$
give:
\[
0=\int_\Omega\, \bar u\cdot (P+\lambda)(u)\,dx
=-\int_\Omega \, \sum_{p,q}\, a_{pq}(x)\cdot \frac{\partial u}{\partial x_p}
\frac {\partial \bar u}{\partial x_q}\, dx+
\int_\Omega\, \bar u\cdot( \sum \, a_p(x)
\frac{\partial u}{\partial x_p}\,)\, dx+
\] 
\[
\int_\Omega\, |u(x)|^2\cdot b(x)\, dx+
\lambda\cdot \int\, |u(x)|^2\, dx
\]
\medskip


\noindent
Write $\lambda=\xi+i\eta$.
Separating real and imaginary parts we find the two equations:
\[
\xi\int\, |u|^2\, dx=
\int\, \sum_{p,q} a_{p,q}(x)\,\frac{\partial u}{\partial x_p}\cdot 
\frac{\partial \bar u}{\partial x_q}\, dx+
\int\, \bigl(\frac{1}{2}\cdot \sum\, \frac{\partial a_p}{\partial x_p}- b\,\bigr )
\cdot |u|^2\, dx\tag{i}
\]
\[
\eta\int\, |u|^2\, dx=\frac{1}{2i}\int \sum\, a_p\bigl(
u\frac{\partial \bar u}{\partial x_p}-
\bar u \frac{\partial u}{\partial x_p}\,\bigl )\, dx\tag{ii}
\]


\noindent
Set
\[ A= \int\, |u|^2\,dx\quad\colon\quad
B= \int\, |\nabla(u)|^2\,dx
\]












Since $P$ is elliptic there exists a positive constant $k$ such that
\[
\sum_{p,q} a_{p,q}(x)\,\frac{\partial u}{\partial x_p}>
k\cdot |\nabla(u)|^2
\]
From this we see that (i-ii) gives positive constants $c_1,c_2,c_3$ such that
\[
A\xi>c_1B-c_2B\quad\colon\quad A|\eta|<c_3\cdot \sqrt{AB}\tag{iii}
\]
\medskip

\noindent
Here (iii) implies that $\xi>-c_2$ and the reader can also confirm that
\[ 
B<\frac{A}{c-1}(\xi+c-2)\quad\colon\quad
A|\eta|< A\cdot c_2\cdot \sqrt{\frac{\xi+c_2}{c_1}}\quad\colon\quad
|\eta|< c_3\cdot \sqrt{\frac{\xi+c_2}{c_1}}\tag{iv}
\]


\noindent Finally it is obvious that (iv) above gives the requested
inequality in Proposition 3.1.

\bigskip

\centerline{\bf{4. Proof of the asymptotic  formula.}}
\bigskip


\noindent
Using the results above
where we have found  a good control of the integral operator
$\mathcal J$ and the identification of eigenvalues to $\mathcal j$ and those from (*) in (0.B), one can proceed and apply Tauberian theorems to derive
the asymptotic formula  in Theorem B.0 using 
similar methods as  in � A where we treated
the Laplace operator. 
The details 
are left as an exercise to the reader. if necessary, consult
[Carleman: page xx-xx].










\newpage






































\end{document}



