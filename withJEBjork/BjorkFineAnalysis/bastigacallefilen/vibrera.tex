\documentclass{amsart}

\usepackage[applemac]{inputenc}

\addtolength{\hoffset}{-12mm}
\addtolength{\textwidth}{22mm}
\addtolength{\voffset}{-10mm}
\addtolength{\textheight}{20mm}

\def\uuu{_}

\def\vvv{-}


\begin{document}






\newpage


\noindent
Analytic  function theory and Fourier analysis offer  useful tools
in
many applications to theoretical probability theory.
We shall illuminate this
by some specific examples and start with the moment problem.
In  general, let $g(x)$
be a non-negative continous function on $[0,+\infty]$
whose integral
\[
\int_0^\infty\, g(x)\, dx=1
\]
Thus, $g$ is the frequency function of a probability distribution
with mass concentrated to $\{x\geq 0\}$.
Let us assume that the distribution has finite moments, i.e.
\[
\frak{m}_k^2= \int_0^\infty\, x^k\cdot g(x)\, dx<\infty\quad\colon\, k=1,2,\ldots
\]
where 
$\frak{m}_k$ is a sequence of positive numbers.
The pioneering work by Stieltjes about continued fractions from
1894 gave examples of distinct frequency functions
$f$ and $g$ as above with equal moments.
Moreover, it was proved by Stieltjes , and independently  by Emile Borel,
that the family of all possible moment sequences 
$\{\frak{m}_k\}$
"is as ample as it possibly can be".
More precisely, a classic result which goes back to Heine in 1850
shows that a sequence $\{\frak{m}_k\}$
give moments of a frequency  function if and only if
the quadratic forms
\[
\sum_{p=0}^{p=n}\, \frak{m}^2_{p+q}\, x_px_q
\]
are positive definite for every $n\geq 1$.
To obtain
such sequences one can Fourier's expansions of
real-valued infinitely differentiable functions.
Let us take a test-function $f(x)$ on the real $x$-line with support contained in
the closed unit intervak $[0,1]$.
Notice  that this implies that $f$ is flat at the end-points $x=0$ and $x=\pi$, i.e.
at these points the derivaties vanish in every order.
We define the  function of the real variable $s$ by:
\[
\Phi(s)= \int_0^1\, \cos(st)\cdot f(t)\,dt
\]
Parseval's equality applied to $f$ gives
\[
\int_0^\infty\, f(x)^2\, dx=\frac{2}{\pi}\cdot \int_0^\infty\, \Phi(s)^2\, ds
\]
next, define a non-decreasing function $G(x)$ on $x\geq 0$ by
\[
G(x)= \frac{2}{\pi}\cdot \int_0^{\sqrt{x}}\, \Phi(s)^2\, ds
\]
With $f$ choen so that the left hand side in (x) us equal to one,it follows
that
$G(x)$ is a distribnution function of a stochastic variable with mass concenttrated to
$\{x\geq 0\}$.
The frequerncy functiuon of $G$ becomes
\[
g(x)= \frac{1}{\pi\cdot \sqrt{x}}\cdot \, \Phi(\sqrt{x})^2
\]
If $p\geq 1$ we obtain
\[
\int_0^\infty\, x^p\cdot g(x)\, dx=
\int_0^\infty\, s^{2p}\cdot 2s\cdot g(s^2)\, ds=
\frac{2}{\pi}\cdot 
\int_0^\infty\, s^{2p}\cdot\Phi^2(s)\, ds
\]
These moment integrals
can be recaptured from the given test-function $f$.
Namely, since $f$ is flat at $x=0$ and $x=1$, a partial integration gives
the following equality for every $p\geq 1$:
\[
(-1)^p\cdot 
\int_0^1\, \cos(st)\cdot f^{(p)}(t)\,dt
=\int_0^1\, \frac{d^p}{dt^p}(\cos (st))\cdot f(t)\, dt
\]
Applying Paresval's equality to
$f^{(p)}$ and
cosinie. respecetly sdine-ttenaforms the reasder can deduce that
(xx) gives the equlaity
\[
\frak{m}_p^2=\int_0^1\, f^{(p)}(t)^2\, dt
\]
In particular all the moments of the distribution function $G$ are finite.




















\newpage





\centerline{\bf{Lecture IV: Eigenvalues for the Laplace operator.}} 


\bigskip


\noindent
Let $\Omega$ be a connected bounded domain in the complex
$z$-plane of class $\mathcal D(C^1)$.
In the Hilbert space $L^2(\Omega)$ 
we seek functions $u$ satisfying
\[ 
\Delta(u)+\lambda\cdot u=0\tag{0.1}
\] 
for some constnat $\lambda$, which in
addition
extend to continuous functions which are
identically zero on the boundary.
Stokes theoren gives
for every $u$
satisfying (0.1):
\[
\iint_\Omega\, [\,(\frac{\partial u}{\partial x})^2+
(\frac{\partial u}{\partial y})^2]\, dxdy
=\lambda\cdot \iint_\Omega\, u^2\, dxdy\tag{0.2}
\]
Hence non-zero  solutions in (0.1)  can only exist when
$\lambda$ are real and positive.
The result below 
is a special case from the Fredholm-Hilbert theory:

\medskip

\noindent 
{\bf{0.3 Theorem.}}
\emph{There exists a non-decreasing sequence 
$0<\lambda_1\leq \lambda_2\leq \ldots$
of positive real numbers and to each $\lambda_n$ one has a 
real-valued function $\phi_n$
where $\phi_n=0$ on $\partial\Omega$ and
\[ 
\Delta(\phi_n)+\lambda_n\cdot \phi_n=0
\]
holds in $\Omega$. Here $\{\phi_n\}$
is an orthonormal set in the Hibert space
$L^2(\Omega)$ and  Green's function
satisfies the equation}
\[ 
G(p,q)=
\frac{1}{2\pi}\, \sum_{n=1}^\infty\, \frac{\phi_n(p)\phi_n(q)}{\lambda_n}
\]
\emph{where the right hand side converges when $p\neq q$.}

\medskip

\noindent
{\bf{Remark.}} As usual eigenvalues are repated when
the corresponding finite dimensional eigenspace has dimension $>1$.


\bigskip

\centerline {\bf{� 1. Asymptotic formulas.}}

\bigskip


\noindent
Theorem 0.1 and 
Ikehara's theorem lead to
asymptotic formulas.
The first result  is due to Weyl:


\medskip

\noindent
{\bf{1.1 Theorem.}}
\emph{One has the limit formula}
\[ 
\lim_{n\to \infty}\, \frac{\lambda_n}{n}=
\frac{4\pi}{\text{area}(\Omega)}
\]


\noindent
Concerning the eigenfunctions the following 
remarkabel resut was presented by
Carleman
at the Scandinavian Congress in  Copenhagen  1934.

\medskip


\noindent
{\bf{1.2 Theorem.}}
\emph{For every  $p\in\Omega$ one has}
\[
\lim_{n\to \infty}\, 
\frac{1}{\lambda_n}\cdot \sum_{k=1}^{k=n}\,
\phi_k(p)^2=\frac{1}{4\pi}
\]


\noindent
Passing to partial derivatives of the $\phi$-functions
similar asymptotic
formulas hold. The result for
first order partial derivatives is:
\[
\lim_{n\to \infty}\, 
\frac{1}{\lambda^2_n}\cdot \sum_{k=1}^{k=n}\,
\bigl(\frac{\partial\phi_k }{\partial x}(p)\bigr)^2=\frac{1}{16\pi}\tag{1.2.1 }
\]
and similarly for the partial $y$-derivative.

\bigskip


\noindent
The remarkable faxt is of course thst thr teo limit formulas hold for 
every point in $\Omega$
and the limit does not depend on the domain.
The detailed proof is presented in � xx.


\newpage



\centerline{\bf{4. Vibrating planes.}}
\bigskip

\noindent
Let $D$ be a membrance with contant density of mass $m$
and  tension $k>0$. The boundary is fixed by a plane curve
$C$ placed in the horizontal $(x,y)$-plane and
the function $u=u(x,y,t)$ is the deviation in the vertical direction
while the membrance is in motion. Here $t$ is a time variable 
and by Hooke's law the $y$-function satisfies the wave equation

\[
\frac{d^2u}{dt^2}= \frac{k}{m}\cdot \Delta u\tag{*}
\]
where the boundary condition is that $u(p,t)=0$ for each
$p\in C$.
The time dependent kinetic energy becomes
\[
T(t)= \frac{m}{2}\iint_\Omega\, (\frac{du}{dt})^2dxdy
\]
The potential energy becomes
\[
V(t)= \frac{k}{2}\iint_\Omega\,
\bigl[ (\frac{\partial u}{\partial x})^2+
\frac{\partial u}{\partial x})^2\,\bigr]\, dxdy
\]


\noindent
With $\kappa_\nu=\sqrt{\lambda_\nu}$
the general solution to (*) becomes: 
\[ 
u(p,t)=
2\cdot \sum_{\nu=1}^\infty \, c_\nu \cos(\kappa_\nu t)\,\phi_\nu(p)\tag{**}
\]
where $\{c_\nu\}$ is a sequence of real numbers.
Define the mean kinetic energy at individual points
$p\in D$ by
\[ 
L(p)=
\frac{m}{2}\cdot \lim_{\tau\to\infty}\, \frac{1}{\tau}\cdot \int_0^\tau\,
(\frac{du}{dt})^2(p)\cdot d\tau
\]

\medskip

\noindent
{\bf{Exercise.}}
Show that (**) entails that
\[ 
L(p)=
k\cdot \sum\, |c_\nu|^2\lambda_\nu \phi_\nu(p)^2\tag{***}
\]
\medskip

\noindent
The $c$-numbers decay in a physically realistic solution so
that the series above converges. 

\bigskip

\noindent
{\bf{High frequencies.}}
For each positive number
$w$ the contribution from high frequencies is defined by:
\[ 
L_w(p)=k\cdot \sum_{\lambda_\nu>w}
\, |c_\nu|^2\lambda_\nu \cdot \phi_\nu(p)^2
\]


\noindent
Similarly, the   mean potential energy  from high frequencies
is defined by

\[
V_w(p)=k\cdot \sum_{\lambda_\nu>w}\, |c_\nu|^2\cdot \bigl[\frac{\partial \phi_\nu}{\partial x}(p)^2
+\frac{\partial \phi_\nu}{\partial y}(p)^2\bigr]
\]

\medskip



\noindent
Let us analyze
the limit behaviour of the two functions above when
$w\to+\infty$.
Let  $a(\lambda)$ be a $C^2$-function defined for$\lambda>0$ such that
$a(\lambda_\nu)=|c_\nu|^2$ for each $\nu$ and set
\[
P(\lambda)= \sum_{\lambda_\nu\leq \lambda}\,\lambda_\nu\cdot \phi_\nu(p)^2
\]
It follows that
\[
L_w(p)
= k\cdot \int_w^\infty\, a(\lambda)\cdot dP(\lambda)
\]
\medskip


\noindent
{\bf{Exercise.}}
Show first that
Theorem 0.1 entails that
\[
P(\lambda)\simeq\frac{1}{8\pi}\cdot \lambda^2
\]
Suppose now that $a(\lambda)$ is decreasing where
\[ 
\lambda^2\cdot a(\lambda)\leq K\cdot 
\int_\lambda^\infty \lambda\cdot a(\lambda)\, d\lambda
\]
hold for some constant $K$.
Show that this gives the asymptotic formula:
\[ 
L_w(p)\simeq \frac{k}{4\pi}\, \int_w^\infty\, a(\lambda)\, d\lambda
\]
The point is that
the right hand side is independent of $p$.
So when (xx) holds it follows that

\[
\lim_{w\to \infty}\, \frac{L_w(p)}{L_w(q)}=1
\] 
hold for all  pairs $p,q$ in $\Omega$.
\medskip

\noindent
{\bf{Exercise}} Use Theorem 0.1 to deduce a similar asymptotic formula for
the
$V$-function and conclude that
\[
\lim_{w\to \infty}\, \frac{L_w(p)}{V_w(q)}=1
\] 
hold for each point $p$ in $\Omega$.







\newpage

\noindent
When the regularity of $\mathcal \partial\Omega$
is relaxed, for example if $\partial\Omega$ 
is a union of planar parts where pairs
intersect at
lines and "ugly corner points" appear when more than two
planar parts meet,
then the kernel function
$K_h$ is  
unbounded and may even fail to the square integrable, i.e.
it can occur that
\[
\iint_{\partial\Omega \times\partial\Omega}\,
|K(p,q)|^2\, d\sigma (p)d\sigma(q)=+\infty
\] 
In this situation the analysis becomes  more involved
and leads to the spectral theory of
unbounded linear operators.
One can also go further and allow
$u$-solutions to  the equation
$u=\lambda\cdot \mathcal K_h(u)$
which   are 
measurable functions.
In other words, 
the domain 
of definition for the integral operator
$\mathcal K_h$ is extended.
Then it turns out that the spectrum
of $\mathcal K_h$ may 
contain non-discrete parts outside the real line.
We  treat this case for planar domains in � XX where 
a specific case occurs if $\Omega$ is a bounded open subset of ${\bf{R}}^2$ 
bordered by a finite family of 
disjoint piecewise linear Jordan 
curves, i.e. by polygons. When $h$ is a positive function on
$\partial\Omega$ the planar kernel is given by
\[
K_h(p,q)=\frac{1}{\pi}\cdot 
\frac {\langle p-q,{\bf{n}}_*(q)\rangle}{|p-q|^2}
\]
Let  $\{\alpha_\nu\}$ 
be  the family  of  interior angles at the corner points from the union of
the polygons above. So here
$0<\alpha_\nu<\pi$ for each $\nu$ and put: 
\[ 
R= \min_\nu\, \frac{\pi}{\pi-\alpha_\nu}
\]
In his thesis \emph{�ber das Neumann-Poincar�sche Problem
f�r ein gebiet mit Ecken}
from 1916, Carleman proved that
$\mathcal K_h(\lambda)$
extends to a meromorphic function in the open disc
$|\lambda|<R$ where a finite set of real and simple
poles can occur. But in contrast to 
the smooth case
the continuation beyond this disc is in general quite complicated.
More precisely, when the domain of $\mathcal K_h$ is extended to
measurable functions $u$ with finite 
logarithmic energy:
\[
\iint_ {\partial\Omega\times\partial\Omega}
\, \bigl|\log\,\frac{1}{|p-q|}\bigr |\cdot |u(p)|\cdot |u(q)|\, d\sigma(p)d\sigma(q)
<\infty
\]
there appears in general a non-real spectrum outside 
the disc of radius $R$ which need not consist of discrete points.
We remark that Carleman's 
study of the Neumann-Poincar� operators for 
non-smooth domains 
led to the theory about
unbounded self-adjoint operators on Hilbert spaces.





Carleman's book \emph{Sur les �quations singuliers
� noyeau r�el et symmetrique} from 1923
proves the spectral theorem for unbounded
operators and constitutes one if his major contributions in mathematics.










\newpage 

\medskip

\noindent



We shall prove the results above using material 
from
Carleman's lecture at the Scandinavian Congress in  Copenhagen  1934.
The major step is to
establish properties of the function
$\Phi(p,s)$ introduced in � 1.4 below.
First we recall the construction of
Green resolvents.
For every complex number
$\lambda$ outside the discrete set
$\{\lambda_n\}$
we find the function
which for each fixed
$q\in\Lambda$  satisfies
\[ 
\Delta G(p,q;\lambda)
+\lambda\cdot G(p,q;\lambda)=0\quad\colon\quad p\in \Omega\setminus 
\{q\}
\]
and at $p=q$ it has the same singularity as $G(p,q)$, i.e.
$\log\,\frac{1}{|p-q|}$.
Finally, $G(p,q;\lambda)=0$ when
$p\in\partial\Omega$.
Theorem 0.1 gives the equation:


\medskip

\noindent
{\bf{1.4 Theorem}} \emph{One has the equality}
\[
G(p,q;\lambda)-G(p,q)=
\frac{1}{2\pi\lambda  }\, 
\sum_{n=1}^\infty\, \frac{\phi_n(p)\phi_n(q)}{\lambda_n(\lambda-\lambda_n)}
\]








\noindent
{\bf{1.5 The function $\Phi(p,s)$}}.
For each $p\in\Omega$ we define  a function of the 
complex variable $s$:
\[
\Phi(p,s)=\sum_{n=1}^\infty \,
\frac{\phi^2_n(p)}{\lambda_n^s}
\]

\medskip

\noindent
{\bf{1.6 Theorem}}
\emph{The function $\Phi(p,s)$ extends to a  meromorphic
in the whole complex
$s$-plane with a simple pole at $s=1$ whose residue is
$\frac{1}{4\pi}$ and   zeros at 
$0,-1,-2,\dots$.}


\bigskip

\centerline{\bf{2. Proof of Theorem 1.6.}}


\medskip

\noindent
We shall need two
equations which are proved via residue calculus
and left to the reader.



\medskip

\noindent
{\bf{2.1 Lemma}}
\emph{For every pair $0<a<b$ of real numbers 
and each comlex number
$s$ with
$\mathfrak{Re}\, s>1$ one has
the equations:}

\[
b^{-s}=
 \frac{1}{2\pi i}\,\int_{a-i\infty}^{a+i\infty}\,
\frac{\lambda}{b(\lambda-b)\lambda^s}\cdot d\lambda=
\]

\[
\frac{a^{s-1}}{2\pi}\int_{-\pi}^\pi\, 
\frac{e^{i(1-s)\theta}}{b(b-a^{i\theta})}\, d\theta+
+\frac{\sin \pi s}{\pi}\cdot \int_a^\infty \frac{1}{b(b+\lambda)\lambda^s} \, d\lambda
\]
\medskip




\noindent
{\bf{2.2 The function $F(p,\lambda)$}}.
Since $G(p,q;\lambda)$ and $G(p,q)$ have the same singularity
$\log\,\frac{1}{|p-q|}$ along the diagonal it follows that 
\[ 
G(p,p;\lambda)-G(p,p)=\frac{1}{2\pi\lambda  }\, 
\sum_{n=1}^\infty\, \frac{\phi_n(p)^2}{\lambda_n(\lambda-\lambda_n)}
\tag{i}
\]
when $p\in\Omega$ where
the right hand side is   a meromorphic function of $\lambda$ with
poles confined to the
set $\{\lambda_n\}$. Set
\[ 
F(p,\lambda)= G(p,p;\lambda)-G(p,p)
\]

\medskip



\noindent
Keeping $p\in\Omega$  fixed 
we apply Lemma 2.1  with $b=\lambda_n$ for 
every $n\geq 1$
and with a real $a$ such that
$0<a<\lambda_1$.
A  summation over $n$ gives the equation  below for each
$\mathfrak{Re}\, s>1$: 
\medskip

\noindent
{\bf{Lemma 2.3}}
\emph{One has the equality}
\[
\Phi(p,s)= \frac{a^{s-1}}{4\pi^2}\int_{-\pi}^\pi\, 
e^{i(1-s)\theta}\cdot F(p,ae^{i\theta})\, d\theta
+\frac{\sin \pi s}{2\pi^2 }\cdot \int_a^\infty \,\frac{F(p,-\lambda)}{\lambda^s}\, 
d\lambda\tag{*}
\]

\medskip


\noindent
The 
first term in (*)  is an entire function of $s$ since
$0<a<\lambda_1$ is a fixed real number and $F(p,\lambda)$ is analytic in
the open disc of radius $|\lambda_1|$ centered at the origin.
So $\Phi(p,s)$ extends to a  meromorphic function
with a simple pole at
$s=1$ if the same is true for the function
\[
F_*(p,s)=
\frac{\sin \pi s}{2\pi^2 }\cdot \int_a^\infty \,\frac{F(p,-\lambda)}{\lambda^s}\, 
d\lambda\tag{2.4}
\]
where we in addition should verify that
the residue at 
$s=1$ is  $\frac{1}{4\pi}$.
To attain this we shall find
another
integral formula for the $F_*$-function.






\medskip


\noindent
{\bf{ 2.5 The functions $H(p,q;\kappa). $}}
Define  the analytic function $K(z)$  in the
half-plane $\mathfrak{Re}\, z>0$ by
\[ 
K(z)= \int_1^\infty\, \frac{e^{-zt}}{\sqrt{t^2-1}}\, dt
\]
If $p$ and $q$ is a pair of points on
$\Omega$ their euclidian distance is denoted by
$|p-q|$.
To each $\kappa>0 $ we get the function defined in
$\Omega\times\Omega$ by:
\[ 
K_\kappa (p,q)= K(\kappa|p-q|)
\] 


\noindent
{\bf{Exercise.}}
Verify the limit formula below where $\gamma$ is the Euler constant:
\[
\lim_{q\to p}\, K_\kappa(p,q)=\log\,\frac{1}{|p-q|}
-\log\kappa +\log 2-\log\gamma\tag{iii}
\]
Hence we encounter the same singularity as for $G(p,q)$
which means that the function
\[
g_\kappa(p,q)=K_\kappa(p,q)-G(p,q)\tag{iv}
\]
is defined in the whole product $\Omega\times\Omega$.
Next,
for each $\lambda>0$ we set
$\kappa=\sqrt{\lambda}$ and 
define the function
$H(p,q;\kappa)$
outside the diagonal in $\Omega\times\Omega$ by:
\[ 
H(p,q;\kappa)= 
K_\kappa(p,q)-G(p,q;\lambda)\tag{2.5.1}
\]


\noindent
By (iii) the functions
$K_\kappa(p,q)$ and
$G(p,q;\lambda)$
have the same logarithmic singularity
along the diagonal which entails that
$H(p,q;\kappa)$ is defined in
the whole prodct $\Omega\times\Omega$.
When $p\in\partial\Omega$ is kept fixed
the vanishing of the Greeen's function
gives
\[
H(p,q;\kappa)= K_\kappa(p,q;\kappa)\quad\colon q\in\Omega
\]
The reader may also verifiy that
when $p\in\partial \Omega$ is kepts fixed
then the function
$q\mapsto H(p,q;\kappa)$ satisfies
the equation below in $\Omega$:
\[ 
\Delta H(p,q;\kappa)-\kappa^2\cdot H(p,q;\kappa)=0
\]


\noindent
{\bf{Exercise.}}
Use the results above
to show that there exist  constants $A$ and $\alpha>0$ 
which are independent of $\kappa$ and of $p\in\Omega$ such that
\[ 
0\leq H(p,p;\kappa)\leq K(\kappa\cdot \text{dist}(p,\partial\Omega)\quad\text{and}\quad
H(p,p;\kappa)\leq A\cdot e^{-\alpha\kappa}
\] 
\medskip




\noindent
{\bf{2.6 Passage to a limit.}}
In (2.5.1) we can pass to the limit as $q\to p$
inside $\Omega$ and the construction of $F$ in (2.2) gives:
\[
H(p,p;\kappa) =
-\log\kappa +\log 2-\log\gamma+g_\kappa(p,p)-F(p,-\lambda)\tag{2.6.1}
\]
where we recall that
$\kappa^2=\lambda$. It follows that
\[ 
F_*(p,s)=\frac{\sin\pi s}{2\pi^2}\cdot \bigl[
-\int_a^\infty\, \frac{\log\,\sqrt{\lambda}}{\lambda^s}\, d\lambda+
\int_a^\infty
\frac{1}{\lambda^s}\,[
 g_\kappa(p,p)+\log 2-\log \gamma- H(p,p;\kappa)]\,d\lambda\tag{2.6.2}
\] 
\medskip


\noindent
A computation gives:

\[
-\int_a^\infty\, \frac{\log\,\sqrt{\lambda}}{\lambda^s}\, d\lambda=
-\frac{1}{2}\cdot 
\frac{a^{s-1}\log a}{s-1}+
\frac{1}{2}\cdot \frac{a^{s-1}}{(s-1)^2}\tag{2.6.3}
\]
\medskip

\noindent
In (2.6.3 ) we have a double pole at $s=1$
which after multiplication with
the sine-function gives
a simple pole and the reader can verify that
the residue is $\frac{1}{4\pi}$.
Finally, using
the estimate in Exercise 2.5.2 the reader may verify that
the second term in (2.6.2) yields an entire function of $s$
Hence $s\mapsto F_*(p,s)$
is meromorphic with a simple pole at $s=1$ with
residue
$\frac{1}{4\pi}$ and  
Theorem 1.6 is proved.



\newpage


\centerline{\emph{3. Proofs of the asymptotic formulas.}}
\bigskip


\noindent
Theorem 1.6 and Ikehara's theorem from � xX
entail that
\[
\lim_{n\to\infty}\, \frac{1}{\lambda_n}\cdot
\sum_{k=1}^{k=n}\,\phi_k(p)^2
= \frac{1}{4\pi}\tag{3.1}
\]
\medskip

\noindent
This proves Theorem 1.2 is proved and to
get Theorem 1.1 we perform an integration over
$\Omega$  so that
\[
\sum_{n=1}^\infty\,\frac{1}{\lambda_n^s}
=\iint_\Omega\, \Phi(p,s)\, dxdy
\] 
where we simply have used that each $\phi$-function 
has a squared integral equal to one over
$\Omega$.
When $\mathfrak{Re}\, s>1$
the equations from the proof of Theorem 1.6  show that after
an integration over $\Omega$ one has
\[
\sum_{n=1}^\infty\,\frac{1}{\lambda_n^s}=
\frac{\text{Area}(\Omega)}{4\pi} \cdot \frac{1}{s-1}+ J(s)
\]
where $J(s)$ is analytic in
$\mathfrak{Re}>1$. 
By (*) in Lemma 2.3 the $J$-function is a sum of an entire function
and the
function

\[
s\mapsto \iint_\Omega\, \bigl[\frac{\sin\pi s}{2\pi^2}\cdot
\int_a^\infty
\frac{1}{\lambda^s}\,[
 g_\kappa(p,p)+\log 2-\log \gamma- H(p,p;\kappa)]\,d\lambda\, \bigr]\, dxdy\tag{2.6.2}
\]
\medskip


\noindent
From this it is clear that the requested continuity of $J$ up to $s=1$
follows if the integrals
\[
\iint_\Omega\bigl[\int_a^\infty
\frac{1}{\lambda}\,
H(p,p;\kappa)]\,d\lambda\,\bigr ]\, dxdy<\infty
\tag{i}
\]
and similarly with
$H(p,p;\kappa)$ replaced by $g_\kappa(p,p)$.
To prove (i) we use the estimate in (2.5.2) which
shows that (i) is majorised by
\[ 
A\cdot\iint_\Omega\bigl [
\int_a^\infty\,\frac{1}{\lambda}e^{-\alpha\cdot \sqrt{\lambda}
\cdot\text{dist}(p,\partial\Omega)}\, d\lambda\bigr]\, dxdy
\]
Put  $\ell(p)=\text{dist}(p,\partial\Omega)$ and consider for a fixed
$p\in\Omega$:
\[
\int_a^\infty\,\frac{1}{\lambda}e^{-\alpha\cdot \sqrt{\lambda}
\cdot\ell(p)}\, d\lambda
\]
Above $a$ and $\alpha$ are fixed positive constants
and the reader may verify that this gives another pair of constants
$b,c$ which are independent of $p$ such that (i) is majorized by
\[
c\cdot [\log^+\,\frac{b}{\ell (p)}]^2
\]
This function of $p$ is integrable over $\Omega$ and
in the same way the reader can check the convergence when
$H(p,p,\kappa)$ is replaced by $g_\kappa(p,p)$.


\bigskip

\noindent
\emph{3.2. The limit formula (1.3.)}
The proof of (1.3) uses similar methods as above but 
some extra technicalities appear  because
estimates
for partial derivatives of the Green's function are needed.
The reader may consult
[Carleman: page 38-40] for the details which gives (1.3)
or try to carry out the proof.
In [ibid]
a general limit formula for
higher order mixed partial derivatives of the $\phi$-functions
is proved. The result is:

\bigskip

\noindent
{\bf{3.3. Theorem.}}
\emph{For every pair  of non-negative integers $j,m$ 
and each $p\in\Omega$ one has the limit formula}

\[ \lim_{n\to\infty}\,
\frac{1}{\lambda_n^{j+m+1}}\cdot
\sum_{k=1}^{k=n}\, \bigl(\frac{\partial^{j+m}\phi_k}{\partial x^j\partial y^j}
\bigr)^2(p)=
\frac{1}{\pi\cdot 2^{2m+2j+2}}\cdot
\frac{(2m)!\cdot (2j)!}{m!\cdot j!\cdot (m+j+1)!}
\]


\bigskip


\centerline{\bf{4. Vibrating planes.}}
\bigskip

\noindent
Let $D$ be a membrance with contant density of mass $m$
and  tension $k>0$. The boundary is fixed by a plane curve
$C$ placed in the horizontal $(x,y)$-plane and
the function $u=u(x,y,t)$ is the deviation in the vertical direction
while the membrance is in motion. Here $t$ is a time variable 
and by Hooke's law the $y$-function satisfies the wave equation

\[
\frac{d^2u}{dt^2}= \frac{k}{m}\cdot \Delta u\tag{*}
\]
where the boundary condition is that $u(p,t)=0$ for each
$p\in C$.
The time dependent kinetic energy becomes
\[
T(t)= \frac{m}{2}\iint_\Omega\, (\frac{du}{dt})^2dxdy
\]
The potential energy becomes
\[
V(t)= \frac{k}{2}\iint_\Omega\,
\bigl[ (\frac{\partial u}{\partial x})^2+
\frac{\partial u}{\partial x})^2\,\bigr]\, dxdy
\]


\noindent
With $\kappa_\nu=\sqrt{\lambda_\nu}$
the general solution to (*) becomes: 
\[ 
u(p,t)=
2\cdot \sum_{\nu=1}^\infty \, c_\nu \cos(\kappa_\nu t)\,\phi_\nu(p)\tag{**}
\]
where $\{c_\nu\}$ is a sequence of real numbers.
Define the mean kinetic energy at individual points
$p\in D$ by
\[ 
L(p)=
\frac{m}{2}\cdot \lim_{\tau\to\infty}\, \frac{1}{\tau}\cdot \int_0^\tau\,
(\frac{du}{dt})^2(p)\cdot d\tau
\]

\medskip

\noindent
{\bf{Exercise.}}
Show that (**) entails that
\[ 
L(p)=
k\cdot \sum\, |c_\nu|^2\lambda_\nu \phi_\nu(p)^2\tag{***}
\]
\medskip

\noindent
The $c$-numbers decay in a physically realistic solution so
that the series above converges. 

\bigskip

\noindent
{\bf{High frequencies.}}
For each positive number
$w$ the contribution from high frequencies is defined by:
\[ 
L_w(p)=k\cdot \sum_{\lambda_\nu>w}
\, |c_\nu|^2\lambda_\nu \cdot \phi_\nu(p)^2
\]


\noindent
Similarly, the   mean potential energy  from high frequencies
is defined by

\[
V_w(p)=k\cdot \sum_{\lambda_\nu>w}\, |c_\nu|^2\cdot \bigl[\frac{\partial \phi_\nu}{\partial x}(p)^2
+\frac{\partial \phi_\nu}{\partial y}(p)^2\bigr]
\]

\medskip



\noindent
Let us analyze
the limit behaviour of the two functions above when
$w\to+\infty$.
Let  $a(\lambda)$ be a $C^2$-function defined for$\lambda>0$ such that
$a(\lambda_\nu)=|c_\nu|^2$ for each $\nu$ and set
\[
P(\lambda)= \sum_{\lambda_\nu\leq \lambda}\,\lambda_\nu\cdot \phi_\nu(p)^2
\]
It follows that
\[
L_w(p)
= k\cdot \int_w^\infty\, a(\lambda)\cdot dP(\lambda)
\]
\medskip


\noindent
{\bf{Exercise.}}
Show first that
Theorem 0.1 entails that
\[
P(\lambda)\simeq\frac{1}{8\pi}\cdot \lambda^2
\]
Suppose now that $a(\lambda)$ is decreasing where
\[ 
\lambda^2\cdot a(\lambda)\leq K\cdot 
\int_\lambda^\infty \lambda\cdot a(\lambda)\, d\lambda
\]
hold for some constant $K$.
Show that this gives the asymptotic formula:
\[ 
L_w(p)\simeq \frac{k}{4\pi}\, \int_w^\infty\, a(\lambda)\, d\lambda
\]
The point is that
the right hand side is independent of $p$.
So when (xx) holds it follows that

\[
\lim_{w\to \infty}\, \frac{L_w(p)}{L_w(q)}=1
\] 
hold for all  pairs $p,q$ in $\Omega$.
\medskip

\noindent
{\bf{Exercise}} Use Theorem 0.1 to deduce a similar asymptotic formula for
the
$V$-function and conclude that
\[
\lim_{w\to \infty}\, \frac{L_w(p)}{V_w(q)}=1
\] 
hold for each point $p$ in $\Omega$.







\newpage

\noindent
When the regularity of $\mathcal \partial\Omega$
is relaxed, for example if $\partial\Omega$ 
is a union of planar parts where pairs
intersect at
lines and "ugly corner points" appear when more than two
planar parts meet,
then the kernel function
$K_h$ is  
unbounded and may even fail to the square integrable, i.e.
it can occur that
\[
\iint_{\partial\Omega \times\partial\Omega}\,
|K(p,q)|^2\, d\sigma (p)d\sigma(q)=+\infty
\] 
In this situation the analysis becomes  more involved
and leads to the spectral theory of
unbounded linear operators.
One can also go further and allow
$u$-solutions to  the equation
$u=\lambda\cdot \mathcal K_h(u)$
which   are 
measurable functions.
In other words, 
the domain 
of definition for the integral operator
$\mathcal K_h$ is extended.
Then it turns out that the spectrum
of $\mathcal K_h$ may 
contain non-discrete parts outside the real line.
We  treat this case for planar domains in � XX where 
a specific case occurs if $\Omega$ is a bounded open subset of ${\bf{R}}^2$ 
bordered by a finite family of 
disjoint piecewise linear Jordan 
curves, i.e. by polygons. When $h$ is a positive function on
$\partial\Omega$ the planar kernel is given by
\[
K_h(p,q)=\frac{1}{\pi}\cdot 
\frac {\langle p-q,{\bf{n}}_*(q)\rangle}{|p-q|^2}
\]
Let  $\{\alpha_\nu\}$ 
be  the family  of  interior angles at the corner points from the union of
the polygons above. So here
$0<\alpha_\nu<\pi$ for each $\nu$ and put: 
\[ 
R= \min_\nu\, \frac{\pi}{\pi-\alpha_\nu}
\]
In his thesis \emph{�ber das Neumann-Poincar�sche Problem
f�r ein gebiet mit Ecken}
from 1916, Carleman proved that
$\mathcal K_h(\lambda)$
extends to a meromorphic function in the open disc
$|\lambda|<R$ where a finite set of real and simple
poles can occur. But in contrast to 
the smooth case
the continuation beyond this disc is in general quite complicated.
More precisely, when the domain of $\mathcal K_h$ is extended to
measurable functions $u$ with finite 
logarithmic energy:
\[
\iint_ {\partial\Omega\times\partial\Omega}
\, \bigl|\log\,\frac{1}{|p-q|}\bigr |\cdot |u(p)|\cdot |u(q)|\, d\sigma(p)d\sigma(q)
<\infty
\]
there appears in general a non-real spectrum outside 
the disc of radius $R$ which need not consist of discrete points.
We remark that Carleman's 
study of the Neumann-Poincar� operators for 
non-smooth domains 
led to the theory about
unbounded self-adjoint operators on Hilbert spaces.
Carleman's book \emph{Sur les �quations singuliers
� noyeau r�el et symmetrique} from 1923
proves the spectral theorem for unbounded
operators and constitutes one if his major contributions in mathematics.




\end{document}















\newpage