
\documentclass{amsart}
\usepackage[applemac]{inputenc}
\addtolength{\voffset}{-10mm}
\addtolength{\textheight}{20mm}
\def\uuu{_}
\def\vvv{-}
\begin{document}



\centerline{\bf\large{14. Sets of harmonic measure zero }}

\bigskip

\noindent
{\bf Introduction.} 
The study of harmonic measures
and other areas in potential theory goes back to
a problem raised by G. Robin in the article [Rob]
from 1886 which had  
physical background in electric engineering.
The problem is: \emph{Let $E$ be a compact set in
${\bf{C}}$. Find a probability
measure $\mu$ on $E$ such that the function}
\[ 
U_\mu(z)=
\int_E\,\log\,\frac{1}{[z-\zeta|}\,\cdot d\mu(\zeta)\quad
\colon\, z\in{\bf{C}}\setminus E\tag{*}
\]
\emph{takes  constant boundary values on $E$.}
For every 
probability measure $\mu$ on $E$, i.e. a non\vvv negative
Riesz measure of unit  mass  supported by $E$, 
the integral (*) is defined for points in $E$
where the value can be finite or infinite.
To be precise, if $z\uuu *\in E$ is fixed and $n$ is a positive integer
we put $E\uuu n= E\setminus \{|z\vvv z\uuu *|\geq 1/n\}$. Since
$\log\,\frac{1}{|z\uuu *-\zeta|}\geq 0$ when $|\zeta\vvv z\uuu *|\leq 1$
it follows 
\[
n\mapsto \int_{E\uuu n}\,\log\,\frac{1}{|z\uuu *-\zeta|}\,\cdot d\mu(\zeta)
\]
is increasing and by  definition the integral (*) taken
on $E$ with $z=z\uuu *$ is equal to
the limit if (1) which therefore is  finite or $+\infty$.
\medskip

\noindent
{\bf{Remark.}}
The  limits above which  compute
$U\uuu\mu(z)$ at points in $E$ imply that
the function
\[ 
z\mapsto U\uuu\mu(z)
\] 
is superharmonic function  
which is harmonic in the open complement of 
the support of $\mu$.
Recall also from � XX that
the Laplacian  of $U_\mu$ taken in
the distribution sense is
equal to the negative measure
$-\pi\cdot \mu$.
One refers to $U\uuu\mu(z)$ as the logarithmic potential of
$\mu$. 
After measure theory had been developed
and the notion of
subharmonic functions
had been consolidated a
first major discovery is due to Lindeberg
who proved that a
compact set $E$  whose 
Hausdorff measure with respect to the function
$h(r)= \log\,\frac{1}{r}$  ois zero is a set of  removable singularities for bounded
harmonic functions, i.e. a bounded harmonic function
$H$ defined  in $U\setminus E$ for some open neighborhood
$U$  extends
to a harmonic function in  $U$.
In � A we expose more precise results due to Myhrberg
about the
connection between sets which are removable siungularities for
bounded harmonic functions and the class of
harmonic null sets, i.e. compact sets with a vanishing
logarithmic capacity.
In � C we expose a result  due to
H. Cartan (the son of Elie Cartan)
which gives a \emph{necessary condition} in order that
a compact set $E$
has harmonic measure zero.
For several decades it was an open question of
Cartan's metric conditions
entail the converse. More precisely,
let $E$ be a  compact set which is thin in
Cartan's sense, i.e.  its 
Hausdorff measure is zero for every $h$-function in
Cartan's class. Then one may ask if
$E$
 has logartihmic capacity zero.
In 1952 Carleson constructed a totally disconnected
comppact
set $E$ on the real line which is thin in Cartan's sense
and has positive capacity. This example  shows that
the search for metric conditions in order that
a compact set has harmonic measure zero is
more or less
hopeless, unless one adds
some extra kind of properties. A notable point is that
if $E$ is an "ordinary Cantor set" on the real line then
it is a harmonic null set if and only if its logarithmic length
is zero, i.e. its Hausdorff measure with respect to
$\log\,\frac{1}{r}$ is zero.
We prove this result in � XX and remark that
Carleson's construction in � XX is far more subtle and general
compared to
the standard construction of Cantor sets.
\medskip

\noindent
\emph{About literature.}
Many text-books treat potential theory with special emphasis on
the present study of the logarithmic potential in
${\bf{C}}$.
The text-book [Nevanlinna] covers the results in this chapter
except for
Carleson's constructions in � F.
Passing to potential theory in
dimension $\geq 3$, Frostman's book
\emph{xxx}  is a veritable master-piece which contains 
concise and yet very
complete proofs leading to  existence results  of equilibrium measures in
a quite general context.
The reader may also consult the text-book
[Garnett-Marshall] where �� XX contains material from
lectures by Carleson at UCLA during the years 19xx-xx devoted to complex potential theory.
In this chapter we
foremost follow material from Nevanlinna's 
text-book [nev].


















\bigskip


\noindent
\centerline {\bf{� 0. Energy integrals.}}
\medskip


\noindent
We can  integrate $U\uuu\mu$ with respect to
$\mu$ and put
\[ 
J(\mu)=\int\, U\uuu\mu(z)\cdot d\mu(z)
\]
$J(\mu)$ is called the energy integral of $\mu$. Notice that
the energy is expressed  by a double integral:
\[
J(\mu)=\iint \,\log\,\frac{1}{|z-\zeta|}\,\cdot d\mu(\zeta)\cdot d\mu(z)
\]
\medskip

\noindent
In general the energy can be $+\infty$, i.e. it is not always
true that   $U_\mu$ belongs to $L^1(\mu)$.
If there exists
at least one probability measure $\mu$ on $E$ with finite energy
one is led to the variational problem
\[
 \min_\mu\, J(\mu)
\] 
with the minimum taken over all probability measures on
$E$.
Let $\mathcal J_*(E)$ denote the minimum and set
\[ 
\text{Cap}(E)= e^{-\mathcal J_*(\mu)}\tag{*}
\]
One refers to (*) as the logarithmic capacity of
$E$.
In � E we  prove that
$\text{Cap}(E)$ is equal to the transfinite diameter
of $E$ which is obtained via a limit of
well-posed variational problems.
One is also led to ask if there
exists a probability measure $\mu_*$ on $E$ such that
$J(\mu_*)= \mathcal J_*(E)$.
This is indeed true and
can be seen as follows: Let
$\{\mu_n\}$ be a sequence
of probability measires on $E$ such that
$\{J(\mu_n)\}$ converge to $\mathcal J_*(E)$. Passing to a subsequence
we can assume that $\{\mu_n\}$ converge weakly to a limit measure
$\mu_*$.
For each positive integer $N$ we wet
\[ 
J_N(\mu_*)=
\iint_{|z-\zeta|\geq N^{-1}}\, 
\,\log\,\frac{1}{|z\uuu *-\zeta|}\,\cdot d\mu_*(\zeta)\cdot d\mu_*(z)
\]
Then it is clear that
$\{J_N(\mu_*)\}$ is increasing and the limit is $J(\mu_*)$.
Next,
for each fixed $N$
we have
\[
J_N(\mu_*)=\lim_{n\to \infty}\,J_N(\mu_n)\leq
\lim_{n\to \infty}\,J(\mu_n)=\mathcal J_*(E)
\]
Since this hold for every $N$ we conclude that
$J(\mu_*)=\mathcal J_*(E)$.
\medskip


\noindent
{\bf{0.1 Equilibrium measures.}}
Let  $E$ be  a compact set with positive capacity. in � xx we prove that 
there exists a unique
probability measure $\mu_*$ on $E$ such that
$J(\mu_*)= \mathcal J_*(E)$. Moreover, the potential function
$U_\mu(z)$ regarded as a
function in $L^1(\mu)$
is almost everywhere equal to
the constant
$\mathcal J_*(E)$.
This result settles Robin's problem. In the case when
$E$ is a closed rectifiable Jordan curve
$\Gamma$ we remark that 
the equilibrium measure is found
via Riemann's conformal mapping theorem as explained in
� XX.









\medskip


\noindent
{\bf{0.2 Harmonic null sets.}}
Let
$E$ be 
compact and
totally disconnected which means that 
if 
$\Omega={\bf{C}}\setminus E$ is 
the open complement. So here
the connected component of every boundary point
to $\Omega$
is reduced to a singleton set and therefore it is not sure that
the Dirichlet problem can be solved.
But we can consider
harmonic functions defined in
open complementary sets to $E$.
If $z\uuu *\in E$ there exist
arbitrary small Jordan
domains $U$ which where $z\uuu *$ is an interior point while the
closed Jordan curve $\partial U$ has empty intersection with $E$.
This leads to the following:

\medskip

\noindent{\bf{0.3 Definition.}}
\emph{A compact and totally disconnected set $E$
is a removable singularity for bounded harmonic functions if every bounded
harmonic function $H$ in $U\setminus E$ for a pair $(z\uuu *,U)$ as above
extends to a harmonic function in the whole Jordan domain $U$.}
\medskip

\noindent
The following result is due to Myhrberg.

\medskip

\noindent
{\bf{0.4 Theorem.}} \emph{A compact and totally disconnected set $E$
is a removable singularity for bounded harmonic functions if and only if
there exists a probability measure $\mu$ on $E$ such that
$U\uuu \mu(z)=+\infty$ for every $z\in E$.}

\medskip



\noindent
When the two equivalent conditions hold in Theorem 0.4  we say that
$E$ is a  harmonic null\vvv set and
denote by $\mathcal N_{\text{harm}}$
the family  of totally 
disconnected sets   harmonic null\vvv sets.
\medskip

\noindent
{\bf{Remark.}}
In the literature one also refers to sets of  capacity zero, i.e. a compact and totally disconnected 
subset $E$ if ${\bf{C}}$ has capacity zero if and only if its is a harmonic
null set.
One may ask for metric conditions in order that
a given compact and totally disconnected set $E$ belongs to
$\mathcal N_{\text{harm}}$. 
To analyze this we 
consider Hausdorff measures.
In general, 
$h(r)$ be  a continuous and non-decreasing function
defined for $r>0$ and $h(0)=0$.  If $F$ is a compact set
we  consider open coverings of $F$ by discs and
define its outer $h$-measure by
\[ 
h^*(F)=\min\, \sum h(r_\nu)
\] 
where the minimum is taken over
coverings of $F$ by open discs
$\{D_\nu\}$ of radius  $\{r_\nu\}$. 
The family of
compact sets whose outer $h$-measure is zero
is denoted by
$\mathcal N(h)$.
The case $h(r)=r^2$  means precisely that
$F$ has planar Lebesgue measure zero.
If $h(r)$ tends more slowly to zero as $r\to 0$
we get a more restrictive class. If
\[
h(r)=\frac{1}{\log\,\frac{1}{r}}
\] 
one says that a compact set in $\mathcal N(h)$
has
logarithmic capacity
zero .
The first major result
about harmonic null\vvv sets
was proved by 
Lindeberg in 1918:

\medskip

\noindent
{\bf{0.5 Theorem.}} \emph{Let $E$ be a compact set whose logarithmic measure zero.
Then $E$ has harmonic measure zero.}


\medskip







\noindent
Theorem 0.5 was improved in
the article \emph{Note on the 
transfinite diameter} [London.Math. Soc. Vol. 12: 1937)
by Erd�s and Gillis
improved Theorem 0.5 where it was shown that 
if $E$ has a finite logarithmic measure
then it has harmonic measure zero. We prove this result in � XX.
\medskip



\centerline{\bf{Metric conditions.}}
\medskip


\noindent
A   necessary metric condition 
for a set $E$ to be a  harmonic null\vvv set
was established  by
Henri Cartan in [Cartan]. First we give:
\medskip

\noindent
{\bf 0.6 Definition}
\emph{Let $\mathfrak{H}_*$ denote
the class of 
non-decreasing and continuous function
$h(r)$ satisfying}
\[
 \int_0^1\,\frac{h(r)}{r}\cdot dr<\infty
\]


\bigskip

\noindent
{\bf 0.7 Theorem.} \emph{For every
$E\in\mathcal N_{\text{harm}}$
it follows that}
\[
E\in\mathcal N(h)=0\quad\colon\,\forall\,\, h\in \mathfrak{H}_*
\]





\noindent
{\bf Remark.} 
Cartan's result   is
close to Lindeberg's sufficiency result.
Namely,  if $\eta>0$ we set
\[ 
h(r)=\frac{1}{
\bigl[\text{Log}\,\frac{1}{r}]^{1+\eta}}
\]
It is clear that
$h\in\mathfrak{H}_*$
and hence
$h^*(E)=0$ for every   $E\in\mathcal N_{\text{harm}}$.
With $\eta$ small this comes close to say
that the logarithmic capacity of $E$ is zero.
However, Cartan's Theorem 
does not give sufficient  conditions in order that a compact
and totally disconnected  set
$E$ has harmonic measure zero.
\medskip


\noindent
{\bf{0.8 Carleson's example.}}
In the article \emph{On the connection between Hausdorff measures
and capacity} [Arkiv f�r matemtaik: vol. 3 (1956)], 
Carleson constructed a compact totally disconnected subset
$E$ of the  unit interval $[0,1]$ which 
is outside the class
$\mathcal N_{\text{harm}}$ while the
Hausdorff measure $h(E)=0$  is zero for every
$h\in\mathfrak{H}_*$.
Carleson's construction shows that the
search for both necessary and sufficient metric condtions in order
that a given totally disconnected compact set belongs to
$\mathcal N_{\text{harm}}$ appears to be more or less hopeless.
See
also Carleson's text-book
\text{Exceptional sets} for a discussion.
Let us also 
remark that  Carelson's set $E$ above is not  a standard Cantor set.
In fact, if $\mathcal C$ is a Cantor   set in
$[0,1]$ as described in � D below 
then the vanishing of its Hausdorff measure for every
$h\in\mathcal H_*$ implies that $\mathcal C$ is a harmonic null set.
So via cartan's theorem there exists 
a necessary and sufficient metric condition in order that
Cantor sets on the real line
 belongs to $\mathcal N_{\text{harm}}$.

\bigskip




\centerline{\bf{A. Proof of Myhrberg's theorem}}


\medskip

\noindent
{\bf{A.1 Nested coverings }}
We shall employ  
a construction which was originally introduced 
by
De Vall� Poussin. 
Let $E$  be a totally disconnected
and compact set and consider  some  $z\uuu *\in E$.
Choose a small Jordan domain $U$ which contains $z\uuu *$
while $\partial U\cap E=\emptyset$.
For each positive integer $N$ one has  the dyadic grid
$\mathcal D_N$
of closed  squares whose sides are $2^{-N}$.
We consider only those $N$ such
that 
\[ 
2^{-N}<\text{dist}(E,\partial U)
\]

\medskip
\noindent
We get the finite family
$\mathcal D_N(E\cap U)$  of squares
in $\mathcal D_N$
which have a non-empty intersection with
$E\cap U$ and let
Let $V_N$ be the union of these squares.
Next, denote by
$\Omega^*_N$ the connected component of 
$D\setminus \bar V_N$ whose closure contains the Jordan curve
$\partial U$. It follows that
$\Omega^*\uuu N$
is a doubly connected domain whose boundary is the disjoint union of
$\partial U$ and a closed Jordan curve $\Gamma\uuu N$
formed by line segments from squares in the finite family
$\mathcal D_N(E\cap U)$.
Notice  also that
$\{\Omega^*\uuu N\}$ is  an increasing sequence of
open sets where $\Gamma\uuu N$ appears as a compact subset
of $\Omega^*\uuu{N+1}$ for each $N$ and  finally:
\[
 \cup\,\, \Omega_N^*=D\setminus E
\]
\medskip



\noindent
Next, fix some point $z_0\in D\setminus E$ and  from now on $N$ are so large that
$z\uuu 0\in \Omega^*\uuu N$ hold.
The Dirichlet problem has a solution in
each domain $\Omega^*\uuu N$.
This gives a unique pair of non\vvv negative measures
$\mu\uuu N,\rho\uuu N$
where $\mu \uuu N$ is supported by $\Gamma\uuu N$ and $\rho \uuu N$
by $\partial U$ such that
\[
h(z_0)=
\int_{\Gamma \uuu N}
\, h(\zeta)\cdot d\mu_N(\zeta)+
\int_{\partial U}\,h(\zeta)\cdot d\rho_N(\zeta)\tag{*}
\] 
hold for every $h$-function which
is harmonic
in $\Omega^*_N$ with continuous boundary values.
In particular we let $h$ be the harmonic  function which is zero on
$\partial U$ and one on $\Gamma_N$.
This gives the harmonic measure
\[ 
\mathfrak{m}_{\Gamma _N}(z\uuu 0)=
\int_{\Gamma \uuu N}
\, d\mu_N(\zeta)|\tag{**}
\]
Since $\Gamma\uuu N\subset \Omega^*\uuu {N+1}$
we have $\mathfrak{m}\uuu {N+1}\leq
\mathfrak{m}\uuu N$ in $\Omega^*\uuu N$ so 
the masses of the non-negative $\mu$-measures descrease, i.e.
\[
||\mu\uuu {N+1}||\leq ||\mu\uuu N||
\]
Hence there exists the limit
\[
\alpha=\lim\uuu{N\to\infty}\, ||\mu\uuu N||\tag{***}
\]










\medskip


\noindent
{\bf{A.2 The case�$\alpha=0$.}}
When this holds the mass of $\rho\uuu N$ tends to one and
since (*) in particular hold for $h$\vvv functions which are harmonic in
the whole set $U$ with continuous boundary values on $\partial U$
the reader may verify:

\medskip

\noindent
{\bf{A.3 Proposition.}}
\emph{If $\alpha=0$
the sequence  $\{\rho\uuu N\}$ converges weakly to
the representing measure $m(z\uuu 0)$ for which}
\[ 
H(z\uuu 0)= \int\uuu{\partial U}\, H(\zeta)\cdot dm(z\uuu 0,\zeta)
\]
\emph{when $H$ is harmonic in $U$ and continuous on $\bar U$.}
\medskip

\noindent
{\bf{Exercise.}}
Apply Harnakck's inequality inthe domains
$\Omega^*_N$ to show that
when $\alpha=0$ holds for one chosenpoint
$z_0\in U\setminus e$ then
we get a similar vanishing for every other  point
$z_1\in U\setminus E$.
Thus, the condition that
$\alpha)=0$ is intrinsic.
\medskip

\noindent
{\bf{A.4 Extending bounded harmonic functions.}}
Suppose that $\alpha=0$ and let $H$ be a bounded
harmonic function
defined in $U\setminus E$ which in addition extends to
a continuous function on
$\partial U$.
For each $z_0\in U\setminus E$
we start with large $N$ so that $z_0\in \Omega^*_N$
and represent $H(z_0)$ by (1).
Since $H$ is bounded the hypothesis $\alpha=0$ 
entails that

\[
\lim_{N\to \infty}\, 
\int_{\Gamma \uuu N}
\, H(\zeta)\cdot d\mu_N(\zeta)=0
\]
It follows that
\[
H(z_0)= \lim_{N\to \infty}\, \int_{\partial U}
\, H(\zeta)\cdot d\rho_N(\zeta)
\]
DSince this hold for every
$z_0\in U\setminus E$
it follows from Propostion A.3 that
$H$ is equal to the everywhere defined harminic function in
$u$ which extends  the continuous boundary value function
$H|\partial U$.
This shows that 
if $\alpha=0$ then $E$ is a removable singularity for buounded
harmonic functions.
\medskip


\noindent
{\bf{A.6 An infinite potential function.}}
The condition that
$\alpha=0$ is intrinsic and by a conformal mapping
we may assume that
$U$ is the open unit disc
which is convenient because now we can write out more
explicit formulas.
\medskip

\noindent
{\bf{A.5 Proposition.}}
\emph{When $\alpha=0$   there exists a pair of positive numbers
$0<a<A$ such that}
\[
a\leq 
\int_{\Gamma \uuu N}
\, \log\,\frac{1}{|\zeta\vvv w|}
\cdot d\mu_N(\zeta)\leq A
\]
\emph{hold for all $w\in E$ and every $N$.}
\medskip

\noindent
\emph{Proof.}
Fix some $z_0\in D\setminus E$ and consider a
point  $w\in E$. Since 
$|e^{i\theta}\vvv w|= |1\vvv \bar w\cdot e^{i\theta}|$ hold for
all $e^{i\theta}$ on the unit circle we get 
\[
\int_T
\, \log\,\frac{1}{|\zeta\vvv w|}\cdot d\rho\uuu N(\zeta)=
\int_T
\, \log\,\frac{1}{|1\vvv \bar w\zeta|}\cdot
d\rho\uuu N(\zeta)\quad N=1,2\ldots \tag{i}
\]
Next, in $D$ we have
the harmonic function $H(z)=
\log\,\frac{1}{|1\vvv \bar w z|}$ in $D$ and Proposition A.3 entails that
\[
\log\,\frac{1}{|1\vvv \bar w z\uuu 0|}=
\lim\uuu{N\to\infty}\, 
\int_T
\, \log\,\frac{1}{|\zeta\vvv w|}\cdot d\rho\uuu N(\zeta)\tag{ii}
\]
At the same time
(*)  in � A.1 applied with $h(z)= \log\,\frac{1}{|z\vvv w|}$
gives
\[
 \log\,\frac{1}{|z\uuu 0\vvv w|}=
\int_{\Gamma\uuu N}
\, \log\,\frac{1}{|\zeta\vvv w|}\cdot d\mu\uuu N(\zeta)+
\int_T
\, \log\,\frac{1}{|\zeta\vvv w|}\cdot d\rho\uuu N(\zeta)\tag{iii}
\]
 By (ii) the last integral converges to
 $\log\,\frac{1}{1\vvv \bar wz\uuu 0|}$ and hence (iii) gives the limit formula

\[
\lim\uuu{N\to \infty}\, \int_{\Gamma\uuu N}
\, \log\,\frac{1}{|\zeta\vvv w|}\cdot d\mu\uuu N(\zeta)
= \log\frac{1\vvv \bar wz\uuu 0|}{|z\uuu 0\vvv w|}
\]
Since $|z\uuu 0|<1$ and there is some $r<1$ such that
$|w|\leq r$ for every $w\in E$
the last term is between $a$ and $A$ for a pair of positive numbers
which proves Proposition A.5.

\medskip

\noindent
{\bf{A.7 The limit measure $\mu\uuu *$.}}
Assume as above that
$\alpha=0$ and set $\alpha_N= ||\mu_N||$.
On $E$ we  get the probability measures
\[ 
\nu\uuu N=\frac{1}{\alpha_N}\cdot \mu\uuu N
\]
We can extract a subsequence which converges weakly to a probability measure
$\mu$ which by (A.1.0)  is supported by $E$. The left hand side in
Proposition A.5 gives the inequality
\[
\min_{w\in E} \int_{\Gamma \uuu N}
\, \log\,\frac{1}{|\zeta\vvv w|}\cdot d\nu_N(\zeta)\geq
\frac{a}{\alpha_N}
\]
Since $|alpha_N|to 0$
while $a>0$ is a fixed positive constant, it
follows that
the potential function of the weak limit $\mu_*$ is everywhere 
$+\infty$ on $E$, i.e. 
\[
\int_E
\, \log\,\frac{1}{|\zeta\vvv w|}\cdot d\mu_*(\zeta)=+\infty
\] 
hold for
every $w\in E$.






\bigskip



\centerline {\emph{A.9 Proof of Theorem 0.4}}
\medskip


\noindent
Suppose first that $E$ is a removable singularity for bounded
harmonic functions.
Working locally as above around  
some $z\uuu *\in E$ and a Jordan domain $U$,
we obtain for each $N$ the harmonic measure function
$\mathfrak m\uuu N$ in $\Omega^*\uuu N$.
They take values in $(0,1)$ and passing to a subsequence we obtain
a bounded harmonic limit function $\mathfrak m\uuu *$
defined in $U\setminus E$.
By construction $\mathfrak m\uuu *=0$ on $\partial U$
so if it extends to a harmonic function in $U$ it must be identically zero which 
entails that
\[
\lim\uuu{N\to\infty}\, 
\mathfrak m\uuu N(z\uuu 0)=0
\]
This means precisely that $\alpha=0$ and
from the above we construct a potential function which is everywhere $+\infty$ on $E$.
\medskip

\noindent
\emph{The converse.}
Suppose there exists a probability measure
$\mu$ on $E$ whose potential
$U\uuu \mu(w)=+\infty$ for all $w\in E$.
Let $z_*\in E$ and choose
some small Jordan domain $U$
around $z_*$ as in � A.1 where we have constructed the
nested sequence of $\Gamma$-curves. 
Put
\[ 
C_N=\min_{z\in\Gamma_N}
U_\mu(z)
\]
Since
the distances from $\Gamma_N$ to   $E$ tend to zero 
it follows that $C_N\to +\infty$.
Next,  $U\uuu\mu$ restricts to a continuous function on
$\partial U$ and we find its harmonic extension $H(z)$ to the  Jordan domain $U$.
If $C_*$ is the maximum of
$U_\mu$ on $\partial U$we have $H\leq C_*$ in the whole Jordan
domain. With large $N$ we have
$C_N>C_*$ and at the same time
\[
U\uuu\mu(z\uuu 0)\vvv H(z\uuu 0)=
\int_{\Gamma_N}\,(U_\mu-H)\ddot d\mu_N
\geq (C_N-C_*)\cdot ||\mu_N||
\]
Since $C_N\to +\infty$ we conclude that $||\mu_N||\to 0$
which means tha
$\alpha=0$ and then we have proved that
bounded harmonic functions outside $E$ can be extended which  finishes 
the proof of Myhrberg's theorem.

\bigskip

\centerline{\bf{B. Equilibrium distributions
and Robin's constant.}}
\bigskip


\noindent
Let $E$ be a compact set in ${\bf{C}}$.
To each probability measure $\mu$ supported by $E$ we get the potential function

\[
U\uuu \mu(z)= \int\,\log\,\frac{1}{|z\vvv \zeta|}\cdot d\mu(\zeta)
\]
We are going to construct a special $\mu$ for which
$U\uuu \mu$ either is identically $+\infty$ or else 
takes a constant value almost everywhere 
on $E$ with respect to $\mu$.
First we carry out the construction in the special case when
$E$ is a finite union of pairwise disjoint 
and closed Jordan domains
$U\uuu 1,\ldots,U\uuu m$ for some $m\geq 1$.
We also assume that each Jordan curve
$\partial U\uuu k$ is of class $C^1$.
When this holds we get the connected exterior domain

\[
 \Omega^*={\bf{C}}\cup \{\infty\}\setminus
 \cup \,\, \bar U\uuu k
\]
Here we can solve Dirichlet's problem. in particular we obtain the
unique probability measure $\mu$ on $\partial \omega^*$ such that
\[
H(\infty)=\int\, H\cdot d\mu
\]
for every harmonic function $H$ in $\Omega^*$ with continuous boundary
values.
If $z\uuu 1$ and $z\uuu 2$ are two points in
$/cup \,U\uuu k$ which may or may not belong to the same Jordan
domain then we notice that the function
\[
H(z)=\log |z\vvv z\uuu 1|\vvv\log |z\vvv z\uuu 2|
\] 
is harmonic i $\Omega^*$.
Moreover, as $|z|\to \infty$ we notice that
\[
H(z)=\log |1\vvv \frac{z\uuu 1}{z}|\vvv 
\log |1\vvv \frac{z\uuu 2}{z}|
\]
and in the limit we have $H(\infty)=0$.
Since $\log r=\vvv \log\,\frac{1}{r}$ for each $r>0$
 it follows that
\[ 
U\uuu \mu(z\uuu 1)= U\uuu \mu(z\uuu 2)
\]
Hence the function $z\mapsto U\uuu \mu(z)$ is constant in
the interior of $E$
Since the boundary curves $\{\partial U\uuu k\}$ are $C^1$ it follows that
$U\uuu\mu$ extends to a continuous function
with constant value on the whole set $E$.
Of course, $U\uuu\mu$ is also continuous outside $E$ where it is
harmonic.
In fact, we conclude that $U\uuu\mu$ is a globally defined and continuous super\vvv harmonic function in
${\bf{C}}$.
The measure $\mu$ is called the equilibrium distribution
of $E$.
\medskip

\noindent
{\bf{Remark.}} If $E$ is contained in the unit disc
it is clear that the constant value of $U\uuu\mu$ is positive.
On the other hand, let $R>1$ and $E$ is the disc $|z\leq R$.
here $\mu$ is the measure $\frac{1}{\pi}\cdot d\theta$ on
the circle of radius $R$ and we find that the constant value is 
$\vvv
\log \,R$.
 \medskip
 
 \noindent
{\bf{Notation.}}
If $a$ is the constant value of $U\uuu\mu$ we set

\[
 \text{cap}(E)= e^{\vvv a}
\]
and refer to this as the capacity of $E$.
For example, if $E$ is the disc $|z|\leq r$ where $r$ is small we see that
the capacity becomes $r$.
\bigskip

\noindent
{\bf{The general case.}}
Now $E$ is an arbitrary compact set.
To construct a special probability
measure $\mu\uuu E$
we use a similar construction as in section A.
Thus, for $N\geq 1$ we get the family of cubes in $\mathcal D\uuu N$
which have a non\vvv empty intersection with $E$
and then we construct the outer boundary curves of thus set
which borders a connected exterior domain $\Omega^*\uuu N$
whose boundary now
will be a union of closed and piecewise linear Jordan  curves
where two of these may interest at corner points.
We solve the Dirichlet problem and exactly as above we find 
the equilibrium measure $\mu\uuu N$ supported by
$\partial\Omega^*\uuu N$.

 









\bigskip









\centerline {\bf C. Cartan's theorem}
\bigskip

\noindent
We shall 
actually establish
an inequality in
Theorem C.1  below which
has independent interest since
it applies to compact sets $E$ which are not necessarily
harmonic null sets. 
Consider a pair $(h,\mu)$ where
$\mu$ is a probability  measure with compact support
in a compact  set  $E$ of ${\bf{C}}$
with planar Lebesgue measure zero
while 
$h\in \mathfrak{H}_*$.
To each point $a\in E$
and every $r>0$ we have the open disc $D_r(a)$ centered at
$a$ and can regard its $\mu$-mass. This   gives anon\vvv decreasing
function
\[ 
r\mapsto \mu(D_r(a))\quad\colon\, r>0
\]
Put
\[ 
\mathcal U^*=\{ a\in E\quad\colon\,\exists\,\,r>0
\quad\colon\,
\mu(D_r(a))>h(r)\}\tag{1}
\]


\noindent
We assume that the pair $(h,\mu)$ is such that
this set is non\vvv empty.
Since $\mu$ is a Riesz measure one has the limit formula
\[
\lim_{\rho\to r}\,\mu(D_\rho(a)=\mu(D_r(a)
\]
for each $r>0$ where the limit is taken as $\rho$ increases to $r$.
From this it is obvious that $\mathcal U^*$ is a relatively open
subset of $E$ and 
in the closed complement we have
\[ 
a\in E\setminus\mathcal U^*\,\implies\,
\mu(D_r(a))\leq h(r)\quad\colon\,\,\forall\,\, r>0\tag{2}
\]

\noindent
Now the size of $\mathcal U^*$ is
controlled as  follows:
\bigskip

\noindent {\bf C.1 Cartan's Covering Lemma.}
\emph{There  exists a sequence $\{a_\nu\}$
in $E$ and a sequence of positive numbers
$\{r_\nu\}$ such that the following hold:}
\[
\mathcal U^*\subset\,\cup\,\bar D_{r_\nu}(a_\nu)
\quad\text{and}\quad
\sum\, h(r_\nu)\leq 6
\]
\emph{Moreover, for each $z\in{\bf{C}}$
at most five discs from the family
$\{D_{r_\nu}(a_\nu)\}$ contains $z$.}
\bigskip


\noindent \emph{Proof} We may assume that
$\mathcal U^*\neq\emptyset$. Set
\[
\lambda_1^*(r)=
\max_{a\in E}\, \mu(D_r(a))\tag{1}
\]
Since the functions $r\mapsto \mu(D_r(a))$ are lower semi-continuous for
each $a$, it follows that the maximum function
$\lambda_1^*(r)$ also is lower semi-continuous.
Hence the set 
$\{r\colon\, \lambda_1^*(r)>h(r)\}$ is open
and we find
its least upper bound
$r_1^*$. Thus,
\[
\lambda_1^*(r_1^*)=h(r_1^*)\quad\colon\,
\lambda_1^*(r)<h(r)\,\,\,\text{for all}\,\, \, r>r_1^*\tag{2}
\]
Pick $a_1\in E$ so that
\medskip
\[
\lambda_1^*(r_1^*)<\mu(D_{r^*_1}(a_1))+1/2\tag{3}
\]
Next, set $E_1=E\setminus D_{r^*_1}(a_1)$ and define
\[
\lambda_2^*(r)=
\max_{a\in E_1}\, \mu(D_r(a))
\]
If $\lambda_2^*(r)\leq h(r)$ for every $r$ we stop the process.
Otherwise we find the unique largest $r_2^*$ such that
\[
\lambda_2^*(r^*_2)=h(r_2^*)
\]
Notice that $r_2^*\leq r_1^*$
holds since
$h$ is non-decreasing while it is obvious that
$\lambda_2^*\leq\lambda_1^*$.
This time we pick $a_2\in E$ so that
\[
\lambda_2^*(r_2^*)<\mu(D_{r^*_2}(a_2))+2^{-2}
\]
Put $E_2=E_1\setminus D_{r^*_2}$ and continue as above, i.e.
inductively
we get  $E_n$ and set
\[ 
\lambda_{n+1}(r)=\max_{a\in E_n}\,\mu(D_r(a))
\] 
The process continues if we have found $r^*_{n+1}$
so that
$\lambda_{n+1}(r^*_{n+1})=h(r^*_{n+1})$, then we pick 
$a_{n+1}\in E_n$ where
\[
\lambda_{n+1}(r^*_{n+1})\leq \mu(D_{r^*_{n+1}}(a_{n+1}))+ 2^{-n-1}\tag{4}
\]


\noindent
In this way we get the sequence $r_1^*\geq r_2^*\geq\ldots$
and a family of discs $\{ D_{r_\nu^*}(a_\nu)\}$.
To simplify notations we set
\[
D^*_\nu=D_{r_\nu^*}(a_\nu)
\]


\noindent \emph{Sublemma}\emph{ Every point $a\in E$ belongs to at most five
many $D^*$-discs.}
\medskip

\noindent 
\emph{Proof.}
If some $a$ belongs to six discs then elementary geometry gives
a pair $a_k,a_\nu$
such that the angle between the lines $[a,a_k]$ and $[a.a_\nu]$ is $<\pi/3$.
Suppose that for example that $|a-a_k|\geq [a-a_\nu|$.
Euclidian geometry gives
\[ 
|a_k-a_\nu|<|a-a_k|
\]
But this is impossible. For say that $k<\nu$.
Now the disc $D^*_k$ was removed and $a_\nu$ is picked from
the subset $E_\nu$ of $E_k$ while $E_k\cap\Delta_k=\emptyset$.
\medskip

\noindent
\emph{Proof continued.}
The Sublemma implies that
\[ 
\sum\,\mu(D^*_\nu)\leq 5\cdot \mu(E)= 5\tag{5}
\]
The convergence of (5) and (4) imply that
$\lim_{\nu\to\infty} r^*_\nu= 0$. From this it follows that
\[
\mathcal U^*\subset\,\cup\,\bar D_{r_\nu}(a_\nu)\tag{6}
\]

\noindent
Finally we have
\[ 
\sum h(r^*_\nu)= \sum\,\lambda^*_\nu(r_\nu^*)\leq
\sum\,[ \mu(D^*_\nu)+ 2^{-\nu}]\leq
5\cdot \mu(E)+\sum\, 2^{-\nu}=6\tag{7}
\]


\noindent
This completes the proof of Cartan' s Covering Lemma.





\bigskip


\noindent
{\bf{The family $\mathcal G\uuu h$}}.
Let $g(r)$ be a positive   function defined on $(0,+\infty)$
which satisfies:
\[ 
\lim_{r\to 0}\,g(r)=+\infty
\]
In this family we get those $g$\vvv functions for which
\[ 
\int_0^1\, g(r)\cdot dh(r)<\infty\tag{*}
\]
This family is denoted by $\mathcal G\uuu h$.
With this notation we have:
\bigskip

\noindent
{\bf C.2 Lemma}
\emph{For each $g\in\mathcal G\uuu h$
and every point $a\in E\setminus \mathcal U^*$
one has}
\[ 
\int_E\,
g(|z-a|)d\mu(z)\leq\int_0^\rho\, g(r)dh(r)\quad\text{where}\,\,\,h(\rho)=1
\]

\medskip

\noindent
\emph{Proof.} Since $a$ is outside $\mathcal U^*$
we have
\[
\mu(D\uuu r(a))\leq h(r)
\]
for every $r>0$.
Moreover, we recall that $\mu$ has total mass one
and now the reader can verify the inequality in Lemma C.2 b
using a partial integration.




\bigskip

\noindent
{\bf C.3 A special choice of $g$.}
Let us take
\[ 
g(r)=\text{Log}\,\frac{1}{r}\,\quad\colon\, 0<r<1\quad\colon\quad
g(r)=0\quad\colon\,r\geq 1
\]
\medskip

\noindent
This $g$-function belongs to $\mathcal G\uuu h$
by the  
condition on  $h$-functions in Cartan's theorem.
Next, for every  $\lambda>1$ we get the function
$h_\lambda=\lambda\cdot h$ in $\mathfrak{H}_*$ and set:
\[ 
E\setminus\mathcal U^*(\lambda)=\{
a\in E\quad\colon\,\mu(D_r(a))\leq\lambda\cdot h(r)\,\,\colon\, 
\forall\,\, r>0\}
\]


\noindent
Proposition XX(measure general)  applied with $h_\lambda$
gives:
\[ 
\int_E\,
g(|z-a|)d\mu(z)\leq\lambda\cdot \int_0^{\rho/\lambda}\, g(r)dh(r)
\quad\colon\, a\in E\setminus\mathcal U^*(\lambda)\tag{1}
\]

\medskip

\noindent
A partial integration shows that the right hand side in (1)  becomes
\[
g(\rho)+\lambda\cdot \int_0^{\rho}\, \frac{h(r)dr}{r}
\]


\noindent
Hence we have the inequality
\[
\int\,g(|z-a|)\cdot d\mu(z)\leq g(\rho)+\lambda\cdot \int_0^{\rho}\, \frac{h(r)dr}{r}
\quad\colon\quad a\in E\setminus\mathcal U^*(\lambda)\tag{2}
\]


\noindent
In addition to this,  the Covering Lemma gives
an  inclusion
\[
\mathcal U^*(\lambda)\subset \,\cup\,\, D_{r_\nu}(a_\nu)\quad\text{where}\, \sum
\,\,\, h_\lambda(r_\nu)<6\tag{3}
\]
Since $h_\lambda=\lambda\cdot h$
this means that the outer $h$-measure
\[
h^*(\mathcal U^*(\lambda))\leq \frac{6}{\lambda}\tag{4}
\]


\noindent
Hence we have proved the following
where we  recall that
$g(r)=\text{Log}\,\frac{1}{r}$:

\medskip

\noindent
{\bf C.4 Theorem.}
\emph{For
every  triple $(E,\mu,h)$, where $\mu$ is a probability measure supported by
$E$ and $h\in \mathfrak{H}_*$, and any $\lambda>1$
there exists a relatively open
subset $\mathcal U^*(\lambda)\subset E$ such that
the following two inequalities hold:}
\[
\int_E\,\log\,\frac{1}{|z-a|}\cdot d\mu(z)\leq \log\,\frac{1}{\rho}
+
\lambda\cdot \int_0^{\rho}\, \frac{h(r)dr}{r}
\quad\colon\quad a\in E\setminus\mathcal U_*(\lambda)\tag{i}
\]
\[
h^*(U_*(\lambda))<\frac{6}{\lambda}\tag{ii}
\]


\noindent
{\bf {C.11 Proof of Theorem 0.5}}.
Let $E\in\mathcal N_{\text{harm}}$ which by
Theorem 0.2  gives
a probability measure
$\mu$ supported by $E$ such that that the left hand side in (i) 
is $+\infty$ for every $a\in E$.
It follows that the set
$E\setminus \mathcal U^*(\lambda)$ is empty for
every $\lambda>1$. With a fixed $\lambda$ we apply
Cartan's covering Lemma and since
$E=\mathcal U^*(\lambda)$ it follows that
\[
h^*(E)\leq \frac{6}{\lambda}
\]
Here  $\lambda>1 $ is
arbitrary  which gives 
$h^*(E)=0$ as required and 
of Cartan's theorem follows.






\newpage




\bigskip

\centerline{\bf D. Cantor sets.}

\bigskip
\noindent
We construct  
a  family of 
closed subsets of  $[0,1]$ as follows.
Let $1<p_1<p_2<\ldots$ be some  strictly increasing sequence
of real numbers such that the products
$\{p_1\cdots p_n\}$ tend to $+\infty$ as $n$ increases.
Then we can construct  a
decreasing sequence
of closed sets $E_1,E_2,\ldots$
where each $E_n$ is the union of $2^n$-many closed intervals
with equal
length
\[ 
\ell_n= 2^{-n}\cdot\frac{1}{p_1\cdots p_n}
\]


\noindent
{\bf D.1 The construction.}
First $E_1$ is any closed interval $[a_1,b_1]$
with 
\[ 
b_1-a_1=\frac{1}{2p_1}
\]
Inside this closed interval we 
pick two pairwise disjoint closed interval both of length
$\ell_2$ and  let $E_2$ be  their union.
In the next step we pick a pair of closed intervals both of length
$\ell_3$ from each of the two intervals in $E_2$. Their union gives the set $E_3$
and we continue in the same way for every $n$ and arrive at
a closed set
\[ 
\mathcal E=
\,\cap\, E_n
\]

\noindent
We refer to $\mathcal E$ as a Cantor set.
The construction  is  flexible since
we do not impose any condition on specific positions while we at
stage $n$ pick pairs of intervals of length
$\ell_{n+1}$ from each of the $2^n$ many intervals of $E_n$.
Thus, for a given $p$-sequence we obtain a whole family of
Cantor sets denoted by $\text{Cantor}(p_\bullet)$.
The next result gives a condition for such  Cantor sets
to have
harmonic measure zero.
\bigskip

\noindent
{\bf D.2 Theorem.} \emph{The following are equivalent
for an arbitrary  sequence $p_\bullet$ as above}:
\[ 
\text{Cantor}(p_\bullet)\subset
\mathcal N_{\text{harm}}
\,\,\text{holds if and only if}\,\,\,
\sum_{\nu=1}^\infty\,
\frac{\text{Log}\,p_\nu}{2^\nu}=+\infty
\]
\bigskip

\noindent
The proof 
uses  the explicit formulas for 
Robin constants  of  intervals on the real line. For the detailed proof
we refer to page xx-xx in [Nevanlinna].


\bigskip


\centerline {\bf{E. Transfinite diameters and  the logarithmic capacity.}}
\bigskip

\noindent
Let $E$ be a compact set where we
do not assume that $E$ is totally disconnected.
To each $n$\vvv tuple of distinct points
$z\uuu 1,\ldots,z\uuu n$ we put:
\[ 
L\uuu n(z\uuu\bullet)=\frac{1}{n(n\vvv 1)}\cdot
\sum\uuu {k\neq j}\,
\log\,\frac{1}{|z\uuu j\vvv z\uuu k|}
\quad\colon\quad  
\mathcal L\uuu n(E)=\min \, L\uuu n(z\uuu\bullet)
\]
where the minimum in the right hand side
is taken over all $n$\vvv tuples in $E$.
Since $\log\,\frac{1}{r}$ is large when $r\simeq 0$ this means intuitively that
we use  separated $n$\vvv tuples  to
minimize the $L\uuu n$\vvv function.
For example, if
$n=2$ 
the minium is achieved for a pair of points in $E$ whose
distance is maximal, i.e. $\mathcal L\uuu 2$ is the diameter of $E$.

\medskip

\noindent
{\bf{E.1 Proposition.}}
\emph{The sequence $\{\mathcal L\uuu n\}$ is non\vvv decreasing.}

\medskip

\noindent
\emph{Proof.}
Let $z\uuu1,\ldots,z\uuu{n+1}$ minimize  the $L\uuu{n+1}$\vvv
function which gives
\[
\mathcal L\uuu{n+1}(E)=\frac{1}{n(n+1)}\cdot
\sum^{(1)}\uuu {k\neq j}\, \log\,\frac{1}{|z\uuu j\vvv z\uuu k|}
+\frac{2}{n(n+1)}\cdot\sum\uuu{k=2}^{k=n+1}\log\,\frac{1}{z\uuu 1\vvv z\uuu k|}
\]
where $(1)$ above the sum above means that we only consider pairs
$k,j$ which both are $\geq 2$.
Since $z\uuu 2,\ldots,z\uuu {n+1}$ is an $n$\vvv tuple we get the inequality
\[ 
\mathcal L\uuu{n+1}(E)\geq 
\frac{1}{n(n+1)}\cdot  n(n\vvv1)\cdot \mathcal L\uuu n(E)
+\frac{2}{n(n+1)}\cdot\sum\uuu{k=2}^{k=n+1}\log\,\frac{1}{z\uuu 1\vvv z\uuu k|}
\]
The same inequaliiy holds when when  some
$z\uuu j\,\colon\, 2\leq j\leq n+1$ is deleted.
Taking the sum of the resulting inequalities we obtain
\[
(n+1)\mathcal L\uuu{n+1}(E)\geq 
\frac{1}{n}\cdot n(n\vvv1)\cdot \mathcal L\uuu n(E)+
\frac{2}{n(n+1)}\cdot\sum\uuu{k\neq j}\log\,\frac{1}{|z\uuu j\vvv z\uuu k|}
\]
The last term is $2\cdot \mathcal L\uuu{n+1}$ which gives:
\[
(n\vvv 1)\cdot \mathcal L\uuu{n+1}(E)\geq 
\frac{1}{n}\cdot n(n\vvv1)\cdot \mathcal L\uuu n(E)=(n\vvv 1)\mathcal L\uuu n(E)
\]
A division by $n\vvv 1$ gives the requested inequality.
\medskip

\noindent
{\bf{E.2 Definition.}}
\emph{Put}
\[
\mathfrak{D}(E)=
\lim\uuu{n\to\infty}\, e^{\vvv \mathcal L\uuu n(E)}
\]
\emph{This non\vvv negative number is called the transfinite diameter of $E$.}
\medskip

\noindent
{\bf{Remark.}} The definition means that
$\mathfrak D(E)=0$ if and only if
$\mathcal L\uuu n(E)$ tends to $+\infty$ as $n$ increases.
Intuitively this means that we are not able to choose  large tuples in $E$ separated
enough to keep  the sum of the $\log$\vvv terms bounded.
Another number is associated to $E$ is defined by:
\[
\mathcal J\uuu *(E)= \min\uuu\mu\, J(\mu)
\] 
where the minimum is taken over all probability measures in
$E$ and $J(\mu)$ are the energy integrals from � 0.2.

\medskip

\noindent
{\bf{E.3 Definition.}}
\emph{The  
logarithmic capacity of $E$ is defined by:}
\[
\text{Cap}(E)=e^{\vvv J\uuu *(E)}
\]
\medskip


\noindent
{\bf{E.4 Theorem.}}\emph{ For each compact set $E$ one has the equality}
\[
\text{Cap}(E)=\mathfrak D(E)
\]
\medskip

\noindent
\emph{Proof.}
Let $n\geq 2$ and $z^*\uuu 1,\ldots,z^*\uuu n$
is some $n$\vvv tuple where $L\uuu n(z\uuu\bullet)= \mathcal L\uuu n(E)$.
Now we have the probability measure
\[ 
\mu= \frac{1}{n}\cdot \sum\uuu{k=1}^{k=n}\,\delta\uuu{z^*\uuu k}
\]
It is clear that
the energy
\[ 
J(\mu)= \frac{n(n\vvv 1)}{n^2}\cdot L\uuu n(z\uuu\bullet)
\]
Hence we have the inequality
\[ 
\mathcal J\uuu *(E)\leq  \frac{n(n\vvv 1)}{n^2}\cdot\mathcal L\uuu n(E)
\]
Since $\frac{n(n\vvv 1)}{n^2}$ tends to one as $n\to\infty$
a passage to the limit gives:
\[
\mathcal J\uuu *(E)\leq \lim\uuu{n\to\infty}\, \mathcal L\uuu n(E)
\]
Taking exponentials and recalling the negative signs in E.2 and E.3
we conclude that
\[
\mathfrak{D}(E)\leq\text{Cap}(E)\tag{i}
\]

\medskip

\noindent
{\bf{Exercise.}} Prove the opposite inequality. The hint
is that
probability measures on $E$
can be approximated
by discrete measures. 



\newpage


\centerline{\bf{F. Thin  sets with positive capacity on the real line.}}
\bigskip

\noindent
{\bf{Introduction.}}
We expose results 
from the article [Carleson].
In contrast to 
 "ordinary Cantor sets" the
inductive process while open intervals are removed
is far more flexible in Carleson's construction below.
This leads to 
an   extensive class of totally disconnected compact subsets 
of the interval $[0,1]$ which are null sets for all $h$ in the  family
from Definition 0.4. Foloowing [ibid]
we show in � X how to
construct a compact totally disconnected set
$E$ with a positive capacity while $h(E)=0$ for all $h$ in Cartan's family.
Let us remark that the construction of $E$
gave a negative answer to previously open question
 whether
 a converse to Cartan's theorem for
 sets on a real line was  
 valid or not.
 
 \bigskip


\noindent
{\bf{F.1. A set-theoretic construction}}
Let $J=[a,b]$ be a subinterval of $[0,1]$
and $(m,q)$ is a pair
of positive integers
such that
\[
\sum_{\nu=0}^{\nu=m}\, e^{-m-\nu}+
m\cdot e^{-q}= b-a\tag{i}
\]
Then $[a,b]$ is covered by
$2m+1$ intervals
taken in increasing order with lengths
\[ 
e^{-m}, e^{-q}, e^{-m-1}\,\ldots, e^{-q},e^{-2m}\tag{ii}
\]
We refer to this as a decomposition of type $(m,q)$ of the  interval $[a,b]$.
When this has been done
we remove
the $q$-intervals and
are left with $m+1$ many intervals $J_0,J_1,\ldots,J_m$ where
$J_\nu$ has length
$e^{-m-\nu}$.
Consider a  pair
$(m^*_0,q^*_0)$ adapated to the length
of $J_0$, i.e. 
\[
\sum_{\nu=0}^{\nu=m^*_0}\, e^{-m^*_0-\nu}+
m_0\cdot e^{-q^*_0}= e^{-m}
\]
Exactly as above we
get a partion of $J_0$ and remove
the $m_0^*$ many
open $q$-intervals  from $J_0$ which gives
a closed set
$J_0^*$ formed by $m^*_0+1$ many closed intervals
of length $\{ e^{-m^*_0-\nu}\}$
Next,  consider some  pair $(m^*_1,q^*_1)$   adapted to
$J_1$ and proceed as above. Removing $m^*_1$ many
$q$-intervals from $J_1$ there remains a set  $J_1^*$.
In the next stage we chooise a pair $(m^*_2,q^*_2)$ apated to $J_2$
and remove $q$-intervals. At  the final stage we
have a pair $m^*_m,q^*_m$
where $q$-intervals are removed from $J_m$.
In this way a chosen $(m+1)$-tuple $\{(m_k^*,q_k^*)\}$
leads to the removal of $q$-intervals on the family
$J_0,J_1,\ldots,J_m$
and we get a smaller closed set
which  consists of 
$M=m_0+ m_1+\ldots+m^*_m$
many disjoint and closed intervals denoted by
$J_1^*,\ldots, J^*_M$.
We  repeat the construction starting from this $M$-tuple 
and begin with a pair $m_0^{**},q_0^{**})$ adapated to $J_0^*$ and so on.
After one has removed $q$-intervals from
$J_0^*,J_1^*,\ldots,J_M^*$ there
remains a family $J^{**}$
which consists of $M^*_0+M^*_1+\dots+M^*_M$ many disjoint intervals.
This  process can be repeated an infinite number of times and yields
a decreasing sequence of closed
sets $\{E_n\}$ where $E_n$ is the
union of $J$-intervals after $n$ many constructions.
At last we obtain
the intersection $E_*=\cap\, E_n$.
\medskip

\noindent
{\bf{F.2 How to make $E_*$ thin.}}
At every single step above where one has some
$J$-interval in the $n$-th partition of some
length
$\ell$ we  choose an adapted pair $(m,q)$ to this intervall
where $m>2\dot \ell$.
Under this condition one verifies that if
we for an arbitrary  large $n$ pick a finite set of
intervals which appear in $E_n$ with
lengths
$e^{-s_1},\ldots,e^{-s_p}$ then
the integers $s_1,\dots,s_p$ are all distinct !
As pointed out by
in [Carleson: page 405] 
this is \emph{a  crucial of  the  construction.}.
Let us  now consider some Cartan-function  $h$.
To estimate the $h$-measure of $E_*$
we  choose $n$ large and cover
$E_*$ by a finite set of intervals with lenghts
$\{e^{-s_\nu}\}$ where $s_1<\ldots<s_p$ are different integers
and $s_1$ can be made arbitrary large.
This entails that

\[
\sum_{j=1}^{j=p}\, h(e^{-{s_j}})\leq
\sum_{\nu=s_1}^\infty\, h(e^{-\nu})<
\int_0^{e^{-s_1}}\, \frac{h(r)}{r}\,dr
\]
So if $h$ belongs to the  family from  Definition 0.4 
we conclude that
$\mathfrak{h}(E_*)=0$.
\bigskip

\noindent
{\bf{The construction of a $\mu$-measure.}}
By the above one has a flexible way to obtain sets
$E_*$ with vanishing $h$-measures for 
$h\in \mathfrak{H}$.
There remains to make  one such construction along  with
an inductively defined sequence of positive measures
$\{\mu_n\}$  supported by $\{E_n\}$ and then
pass to a limit measure $\mu_*$ supported by $E_*$ whose
energy integral
\[ 
\iint\,\log\bigl|\frac{1}{x-y}\bigr |\, d\mu(x)d\mu(y)<\infty
\]


\bigskip

TO FINISH











\newpage 




\bigskip

\centerline {\bf{E. Further result  and someexamples.}}



\medskip





\centerline {\emph{A. The Robin constant}}.

\medskip


\noindent
First we consider a finite union
of pairwise disjoint Jordan domains $U_1,\ldots U_p$
and get the compact boundary
$\Gamma=\,\cup\,\partial U_\nu$.
The closed Jordan curves $\{\partial U_\nu\}$ are in general disjoint but
we may also allow that a pair intersect at some finite set of points.
We also impose the condition that
the open complement 
\[
\Omega={\bf{C}}\setminus\,\cup \bar U_\nu
\]
is connected.
We  add the point at infinity to
$\Omega$ where
$w=1/z$ is a new coordinate.
A harmonic function $H(z)$ defined in $\Omega$
becomes harmonic at infinity if $w\mapsto H(1/w)$
extends to a harmonic function in a whole disc centered at $w=0$.
\medskip

\noindent
{\bf{A.1 A basic construction}}
In the exterior domain  $\Omega^*=\Omega\cup\{\infty\}$ 
the Dirichlet problem is well posed. 
Hence there exists  a unique probability measure
$\mu$ on $\partial\Omega^*$  such that
\[ 
H(\infty)= \int\, H(\zeta)\cdot d\mu(\zeta)\tag{*}
\]
holds for every harmonic function $H$ in
$\Omega^*$. Notice that
$\partial\Omega^*$ is equal to the union of the closed
Jordan curves $\{\partial U\uuu\nu\}$.
Let us then consider two points
$a$ and $b$ in
in $\cup\,U_\nu$.
It may occur that they belong to a common Jordan
domain or in two different $U$-domains.
In any case we get a harmonic function in $\Omega^*$
defined by:
\[
 H(z)=\text{Log}\, |z-a|-\text{Log}\, |z-b|
\] 
Here
$H$  is harmonic at the point at infinity. For if
$w=1/z$  we have 
\[
H(w)= \text{Log}\, |1-aw|-\text{Log}\, |1-bw|
\] 
which is harmonic
in a disc $|w|<\delta$ and  zero if $w=0$, i.e. 
$H(\infty)=0$.
Since the pair $a,b$ above was arbitrary we conclude that the function
\[ 
a\mapsto \int  \text{Log}\,\frac{1}{ |\zeta-a|}\cdot d\mu(\zeta)\tag{**}
\] 
is constant as $a$ varies in
$\cup\,U_\nu$. Here $\mu$ is supported by the union of the Jordan curves
$\{\partial U_\nu\}$
Since every point on  a single Jordan curve
can be approximated from the inside by
$a$-points in $U_\nu$, it follows by continuity that
the potential function (**) is constant
on $\partial\Omega^*$. So when $E$ is the union of the closed
Jordan curves $\{\partial U_\nu\}$
we conclude that $\mu$ is
a probability measure which settles Robin's problem, i.e. $U\uuu\mu(z)$
is constant on the compact set $E=\cup\,\,\partial U\uuu\nu$.


\medskip


\noindent
{\bf{A.2 An inequality.}}
Consider a pair of
open sets $\Omega$ and $\Omega_1$ which both consist of a finite union of Jordan domains as above and
$\Omega$ is a relatively compact
subset of  $\Omega_1$, i.e. the closure of each Jordan domain in
$\Omega$ is a compact subset of
$\Omega_1$.
Pick some point $a\in\Omega$. In the exterior domain
$\Omega^*$ we find the harmonic function $H\uuu a(z)$
whose boundary values on $\partial\Omega^*$ is equal to the continuous function $z\mapsto \log\,|z\vvv a|$.
Set
\[ 
g\uuu a(z)= \log |z\vvv a|\vvv H\uuu a(z)\tag{1}
\]
This gives a super\vvv harmonic function in
$\Omega^*$
with a singularity at $\infty$ while $g\uuu a=0$ on
the boundary.
It follows from the minimum principle for super\vvv harmonic
functions that 
\[
g\uuu a(z)>0\tag{2}
\]
for all $z\in\Omega^*$.
Next, replacing $\Omega$ by $\Omega\uuu 1$
we find the harmonic function $H\uuu a^1(z)$
which equals $\log|z\vvv a|$ on $\partial \Omega^*\uuu 1$
and  the super\vvv harmonic function
\[
g^1 \uuu a(z)= \log |z\vvv a|\vvv H^1\uuu a(z)\tag{3}
\]
Now $\partial\Omega_1$ is contained in $\Omega^*$ so
$g_a$ restricts to a non-negative function on
$\partial\Omega_1$. 
At $\infty$ the two $g$\vvv functions have the same
logarithmic singularity and hence  $G\uuu a=g_a-g^1_a$ is harmonic in $\Omega_1^*$.
Now $\partial\Omega^*_1$ is contained in $\Omega^*$ and (2) entails that
$G\uuu a>0$ on $\partial\Omega\uuu 1^*$.
We conclude that
$G_a(\infty)\geq 0$.
At the same time we notice that
\[
G_a(\infty)=H\uuu a^1(\infty)\vvv H\uuu a(\infty)
\]
It follows that
\[
H\uuu a ^1(\infty)\geq H\uuu a(\infty)\tag{4}
\]


\noindent
The monotonic property expressed by (4) above
will be used to study the case when
$E$ is a
totally  disconnected and compact
subset of
${\bf{C}}$.
Let  $\Omega^*$ be the exterior domain of $E$, i.e. the union of
$\infty$ and  
${\bf{C}}\setminus E$. 
Since $E$ is totally disconnected the 
boundary points $a\in\partial\Omega^*$ fail to satisfy the
Perron condition for  Dirichlet's problem.
However, we shall   define Robin's constant for
$E$ by a limit process
and    find  a probability measure $\mu$ on $E$ 
such that the potential

\[
z\mapsto 
\int_E\,\text{Log}\,|1-\frac{\zeta}{z}|\,\cdot \mu(\zeta)\tag{*}
\]
is constant as $z$ varies in $E$. But in contrast to the case 
case when $E$ is a union of Jordan arcs,  it may occur that the "constant value" is 
$+\infty$.  
To attain this
we  use a construction which goes back to De 
Vall�\vvv Poussin.
\medskip








\noindent
{\bf{A.2 Nested coverings }}
Let $E$ be a totally disconnected
and compact set.
Without essential loss of generality we
may  assume that $E$ is a compact subset
of the open unit disc $D$.
We
construct a sequence of open sets $\{V_N\}$ as follows.
For each positive integer $N$ one has  the dyadic grid
$\mathcal D_N$
of open squares whose sides are $2^{-N}$.
We get the finite family
$\mathcal D_N(E)$ of  dyadic squares in $\mathcal D_N$
which have a non-empty intersection with
$E$. The union of this finite 
family of open squares gives an open neighborhood $V_N$ of $E$. 
If necessary one starts with a large $N$ so that
$2^{-N}$ is strictly larger than the distance of $E$ to the unit
circle $T$ which entails that the closure of $V_N$ is a compact subset
of $D$.
Let $\Omega^*_N$ be the exterior connected component of
$D\setminus \bar V_N$ whose closure contains the unit circle.
So here
$\Omega^*_N$  contains an annulus $r<|z|<1$ 
for some $0<r<1$
and $\partial \Omega_N$
is the union of $T$ and a closed subset $\Gamma_N$
of $\partial V_N$
which is a finite union of line segments with a finite set of corner
points.
During this construction it is clear that the open sets $\{\Omega_N^*\}$
increase and since $D\setminus E$ is connected we have
\[
 \cup\, \Omega_N^*=D\setminus E
\]


\noindent
Next, fix some point $z_0\in D\setminus E$
which is chosen relatively close to $T$.
For all sufficiently large $N$ we have $z_0\in\Omega^*_N$
and get the unique   probability measure $\mathfrak{m}_N$ which is supported
by $\partial\Omega_N$ and satisfies:
\[ 
g(z_0)=\int_T\, g(e^{i\theta})\cdot d\mathfrak{m}_N(\theta)+
\int_{\Gamma_N}\, g(\zeta)\cdot d\mathfrak{m}_N(\zeta)
\] 
for every $g$-function which
is harmonic
in $\Omega^*_N$ with continuous boundary values.
Since $\{\Omega^*_N\}$ is a nested sequence
it follows that the masses
\[ 
\rho_N=\int_{\Gamma_N}\, d\mathfrak{m}_N(\zeta)
\] 
is a decreasing sequence of positive real numbers.
Hence there exists the limit
\[ 
\rho_*=\lim_{N\to\infty}\,\rho_N\tag{*}
\]
Above two cases may occur. Either $\rho_*>0$ or it is zero.
The condition that $\rho_*=0$ is intrinsic since Harnack's inequality
shows that 
the choice of $z_0$ in $D\setminus E$
is irrelevant. 

\medskip


\noindent
{\bf{A.3 The case�$\rho_*=0$.}}
If $a\in E$ is kept fixed then
$g(z)=\text{Log}\frac{1-\bar az|}{|z-a|}$ is a harmonic function
in the domain $\Omega^*_N$ for every integer $N$ which is
identically zero on the unit circle. Hence
\[ 
g(z_0)=\int_{\Gamma_N}\, g(\zeta)\cdot d\mathfrak{m}_N(\zeta)\tag{1}
\]
Consider the probability measure
$\mu_N=\frac{1}{\rho_N}\cdot \mathfrak{m}_N$ which gives:
\[
\frac{1}{\rho_N}\cdot g(z_0)=
\int_{\Gamma_N}\, \log\, \frac{1}{|\zeta-a|}\cdot d\mu_N(\zeta)
+\int_{\Gamma_N}\, \log\,|1-\bar a\zeta | \cdot d\mu_N(\zeta)\tag{2}
\]
\medskip

\noindent
Next, from $\{\mu_N\}$
we  can extract a subsequence which
converges weakly to a Riesz measure $\mu_*$. It is clear that 
$\mu\uuu *$ is supported by $E$ and
the weak convergence gives:
\[
\lim_{N\to\infty}\, 
\int_{\Gamma_N}\, \log\,|1-\bar a\zeta | \cdot d\mu_N(\zeta)=
\int\, \log\,|1-\bar a\zeta | \cdot d\mu_*(\zeta)\tag{3}
\]
Indeed, this holds since
the integrand
$\text{Log}\,|1-\bar a\zeta z|$ is continuous on $E$.
Next, when $\rho_N\to 0$ is assumed
it follows from (2) that
\[ 
\lim_{N\to\infty}\, 
\int_{\Gamma_N}\, \log\, \frac{1}{|\zeta-a|}\cdot d\mu_N(\zeta)=+\infty\tag{4}
\]
\medskip

\noindent
{\bf{A.4 Exercise.}} 
Show 
that (4) together with  the weak convergence of  $\{\mu_N\}$ to $\mu_*$ imply that
\[
\int_E\log\, \frac{1}{|\zeta-a|}\cdot d\mu_*(\zeta)=+\infty
\]

\noindent
This holds for an arbitrary $a$ in $E$ and hence we have proved:

\medskip


\noindent
{\bf{A.5 Theorem.}} \emph{Let $E$ be a 
totally disconnected and compact set where $\rho_*=0$.
Then there exists 
at least one  probability measure
$\mu$ supported by $E$ such that}
\[ 
\int_E\,\log\, \frac{1}{|a-\zeta|}\cdot d\mu(\zeta)=+\infty
\quad\text{for all}\quad a\in E
\]






\bigskip

\centerline {\emph{B. Removable singularities}}.
\medskip

\noindent
Let $E$ be a totally disconnected and compact subset of some open domain
$\Omega$. 
We say that $E$ is a removable singularity if
every 
bounded harmonic function in $\Omega\setminus E$
extends to be harmonic in the whole domain $\Omega$.
Now we shall use Theorem A.5 to prove Myrberg's result.



\noindent
\emph{Proof.} XXXX
TO BE GIVEN 

\newpage






\end{document}




















