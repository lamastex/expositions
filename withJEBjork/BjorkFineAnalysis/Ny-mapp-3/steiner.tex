\documentclass{amsart}

\usepackage[applemac]{inputenc}
\def\uuu{_}

\def\vvv{-}

\begin{document}


\bigskip

\centerline{\bf\large{19. An isoperimetric inequality.}}
\bigskip

\noindent
A planar domain whose boundary
curve has  prescribed length
has a maximal   area when it it is a disc.  
It turns out that discs solve a more extensive class
of extremal problems.
Consider a function
$f(r)$ defined for
$r>0$ which is continuous and increasing with
$f(0)=0$. 
If $p$ and $q$ are two points in ${\bf{R}}^2$ their euclidian distance is denoted by
$|p-q|$.
When $U$ is a bounded open domain we set
 \[
J(U)=\iint_{U\times U}\, f(|p-q|)\cdot dA_p\cdot dA_q
 \]
where $dA_p$ and $dA_q$ denote the  area measures.
Given a positive number $\mathcal A$ one seeks
to maximize the $J$-functional  in the
family of domains with  prescribed area  $\mathcal A$. 
The $J$-number is  unchanged under a
translation  or a rotation
of a  domain and the family of discs is
stable under these operations. So
the following result makes sense:
\medskip

\noindent
{\bf{1. Theorem.}}
\emph{The $J$-functional 
takes its maximum on discs $D$
of radius $r$ with $\pi r^2=\mathcal A$.
Moreover, for every domain $U$ with area $\mathcal A$
which is not a disc one has a strict inequality}
\[ 
J(U)<J(D)
\] 
\medskip

\noindent
When $f(r)$ is a strictly convex function 
Theorem 1  was established by Blaschke. 
For a general $f$-function which need not be convex the
theorem   was proved by Carleman in [Car]
using 
the symmetrisation process   by
W. Gros from  his article (Monatshefte math.physik 1917).
At the end of [Carleman] it is pointed out that
Theorem 1 leads to another property of discs.

\medskip

\noindent
{\bf{2. Theorem.}}
\emph{Let $\Omega$ be a domain in the family $\mathcal D(C^1)$
and denote by $ds$ the arc\vvv length measure on its boundary.
Then, if the
function}

\[
 p\mapsto \int\uuu{\partial\Omega}\, f(|p\vvv q|)\cdot ds(q)
 \] 
\emph{is constant as $p$ varies in $\partial\Omega$ it follows that
$\Omega$ is a disc.}
 


\medskip

\noindent
{\bf{Remark.}} 
It is very likely that
Theorem 1  can be extended to 
any dimension $n\geq 3$ using
the fact due to    Gros that
successive symmetrisations of domains
taken in different directions converge to the unit
ball in ${\bf{R}}^n$. The reader is invited to supply details using
similar methods as in the subsequent proof
for the case $n=2$.
The crucial step toward the proof of Theorem 1 relies upon
Proposition 2.1 below.
\medskip










\bigskip

\centerline {\bf{2. A fundamental  inequality}}
\medskip

\noindent
Let $M>0$ be a given constant.
On the    vertical lines $\{x=0\}$ and $\{x=M\}$
we consider 
two subsets $G_*$ and $G^*$ which both consist of a finite union of closed intervals.
Let $\{[a_\nu,b_\nu]\}$ be the $G_*$-intervals taken in the $y$\vvv coordinates
and  
$\{[c_j,d_j]\}$ are the $G^*$-intervals.
Here $a_\nu<b_\nu<a_{\nu+1}$ holds, and similarly
the $G^*$-intervals are ordered with increasing $y$-coordinates.
The number of intervals of the two sets are arbitrary and 
need not be the same.
Given  $f(r) $ as in the Theorem 1 we set

\[ 
I(G_*,G^*)=
\sum_\nu\sum_j\, \int_{a_\nu}^{b_\nu}\int_{c_j}^{d_j}\,f(|y-y'|)\cdot dydy'
\]
Consider the variational problem where we seek to minimize these $I$-integrals
for pairs all pairs $(G_*,G^*)$ as above under the
constraints:
  
\[
\sum\,(b_\nu-a_\nu)=\ell_*\quad\text{and}\quad  \sum\,(d_j-c_j)=\ell^*
\]
\medskip

\noindent
That is, the sum of the lengths of the intervals are prescribed on $G_*$ and $G^*$.

\medskip

\noindent
{\bf{2.1 Proposition.}} \emph
{For every  pair $(\ell_*,\ell^*)$ the $I$-integral is minimized when
both $G_*$ and $G^*$ 
consist of a single interval and the mid-points of
the two intervals have equal $y$-coordinate.}
\medskip

\noindent
{\bf{Proof.}} 
First we prove the
result when both $G_*=(a,b)$ and $G^*=(c,d)$ both are intervals.
We must prove that the $I$-intergal is a minimum when
\[
\frac{a+b}{2}=\frac{c+d}{2}\tag{i}
\]
Suppose that inequality holds. Since the $I$-integral is symmetric with respect to the pair 
of intervals, we may assume that
\[
\frac{c+d}{2}=s+\frac{a+b}{2}\quad\text{where}\quad  s>0
\]
Now $I(G_*,G^*)$ is unchanged 
when we translate the two intervals, i.e. if we for some
number
$\xi$ take
$(a+\xi,b+\xi)$ and $(c+\xi,d+\xi)$.
By such a translation we can assume that $a=-b$
so  the mid-point of
$G_*$ becomes $y=0$ and we have:


\[
I=\int_{-b}^b\int _c^d\, f(\sqrt{M^2+(y-y')^2})\cdot dydy'
\]
Using  the variable substitutions
$u=y'-y$ and $v=y'+y$ we see that
\[
-b+c\leq u\leq d+b
\]
and obtain
\[
I=2b\int_{-b+c}^{d+b}\, f(\sqrt{M^2+v^2})\cdot dv
\]
With
\[ 
s=d-\frac{d+c}{2}= \frac{d-c}{2}
\]
we can write
\[ 
I=2b\cdot \int_{w-s}^{w+s} f(\sqrt{M^2+u^2})\cdot dv\quad\colon
w=b+\frac{d+c}{2}
\]
The last integral is a function of $s$, i.e. for every $s\geq 0$ we set

\[
\Phi(s)=2b\cdot \int_{w-s}^{w+s} f(\sqrt{M^2+u^2})\cdot dv\quad\colon
w=b+\frac{d+c}{2}
\]
The derivative of $s$ becomes

\[ 
\Phi'(s)= f(\sqrt{M^2+(w+s)^2})- 
 f(\sqrt{M^2+(w-s)^2})
 \] 
Since $f(r)$ was increasing the derivative is 
$>0$ when $s>0$.  Hence the minimum is achieved when $s=0$
which means that $G_*$ and $G^*$ have a common mid-point and
Proposition 2.1 is proved for the case of an interval pair.
\medskip


\noindent
\emph{The general case.}
If $G_*=\{(a_\nu,b_\nu)\}$ and $G^*=\{(c_k,d_k)\}$
we make an induction over the total number of intervals which appear in the 
two families.
Let 
\[
\xi^*=\frac{c^*+d^*}{2}
\]
be the largest mid-point from the $G^*$-family
which means that $k$ is maximal,
In the $G_*$-family
we also get the largest mid-point
is
\[
\eta^*= \frac{a^*+b^*}{2}
\]
If $\xi^*>\eta^*$ the previous 
case shows that the double sum representing $I$ decreases as long as
when the interval $(c^*,d^*)$ is lowered.
In this process two cases can occur:
Namely the lowered $(c^*,d^*)$-interval hits $(c_{k\vvv 1},d_{k\vvv 1})$ before
the mid-point equality appears. To be precise, this occurs if

\[
c^*-d_{k\vvv 1}<\xi^*-\eta^*
\]
In this case we replace $G^*$ by a union of intervals where
the number of intervals therefore has decreased by one.
If (ii) does not hold we lower $(c^*,d^*)$ until
$\xi^*=\eta^*$.
After this we lower the two top-intervals at the same time
until
one of them hits
the second largest $G$-interval and in this way the total number of intervals is
decreased while the double sum for $I$ is not enlarged.
This gives the requested induction step and the proof of
Proposition 2.1 is finished.






 
 
 










\bigskip

\centerline {\bf{3. Proof of Theorem 1 in the convex case}}
\medskip

\noindent
Consider a  domain $U$ 
defined by
\[ 
g_1(x)\leq y\leq g_2(x)\quad\colon\quad a\leq x\leq b\tag{1}
\]
where $g_1(a)=g_2(a)$ and $g_1(b)=g_2(b)$.
To $U$ we associate the symmetric domain $U^*$
defined by 
\[
- \frac{1}{2}\bigl[ g_2(x)-g_1(x)\bigr]\leq y\leq
 \frac{1}{2}\bigl[ g_2(x)-g_1(x)\bigr]\quad\colon\quad a\leq x\leq b\tag{2}
\]
\medskip


\noindent
Notice that $U$ and $U^*$ have the same area.
Set

\[ J=\iint_{U\times U}\,
f\bigl(\sqrt{(x-x')^2+(y-y')^2}\bigr)\cdot dxdx'dydy'
\]

\[ J^*=\iint_{U^*\times U^*}\,
f\bigl(\sqrt{(x-x')^2+(y-y')^2}\bigr)\cdot dxdx'dydy'
\]
\medskip

\noindent
{\bf{Lemma 3.1.}} \emph{One has the inequality}
$J\leq J^*$.
\medskip

\noindent 
\emph{Proof.} Set 
$h(x)=\frac{1}{2}[g_2(x)-g_1(x)]$
and 
introduce the  function
\medskip

\[ 
H^*(x,x')=\int_{y=-h(x)}^{h(x)}
\int_{y'=-h(x')}^{h(x')}
\,\rho\bigl(\sqrt{(x-x')^2+(y-y')^2}\bigr)\cdot dydy'
\]
We have also the function
\[
H(x,x')=\int_{y=g_1(x)}^{g_2(x)}
\int_{y'=g_1(x')}^{g_2(x')}
f\,\bigl(\sqrt{(x-x')^2+(y-y')^2}\bigr)\cdot dydy'
\]
It is clear that

\[ J=\int_a^b\int_a^b\, H(x,x')dxdx'\quad\text{and}\quad
J^*=\int_a^b\int_a^b\, H^*(x,x')dxdx'
\]

\noindent
Lemma 3.1 follows if we have proved the inequality 
\[
H(x,x')\leq H^*(x,x')\tag{*}
\]
for all pairs $x,x'$ in $[a,b]$.
But this follows via Fubini's theorem
when Proposition 2.1 applied in the special case where
$G_*$ and $G^*$ both consist of a single interval.

\bigskip

\noindent
{\bf{3.2 Variation with convex sets.}}
Let $\mathcal A$ be the prescribed area in Theorem 1 and consider
a convex domain $U$ whose area is $\mathcal A$.
By elementary geometry we see that after a translation and a rotation the convex
domain
$U$
can be represented as in(1) from � 3 above.
Then we construct 
$U_*$ which is a new convex domain and
by Proposition 2.1 gives  $J(U_*)\leq J(U)$.
In the next step
we perform a symmetrisation of $U_*$ 
along some other line which cuts $U_*$ to get 
a new domain
$U_{**}$ where we now have
$J(U_{**})\leq J(U_*)\leq J(U)$.
Finally we use the  geometric result
due to Steiner for  convex domains which
asserts   that when
symmetrisations as above are repeated infinitely often while 
the angle to the angle of the directions to the
$x$-axis  change with some
irrational multiple of $2\pi$, then
the resulting sequence of convex domains converge to a disc.
This  proves that the $J$-functional on a disc is $\leq J(U)$ for every convex domain.
 
\bigskip


\centerline {\bf{4. The non-convex case}}

\medskip

\noindent 
Here we  use 
the symmetrisation process by
Gros.
Let $U$ be a domain. Its symmetrisation in the $x$-direction
is defined  as follows: To every $x$ we get the open
set
\[
\ell_U(x)=\{y\,\colon\,(x,y)\in U\}\tag{1}
\]
Let
$\{(a_\nu,b_\nu)\}$ be the  disjoint intervals of $\ell\uuu U(x)$
and put 
\[
 d(x)=\frac{1}{2}\cdot
 \sum\, (b_\nu-a_\nu)
\]
 
\noindent
We get  the domain $U^*$ which is symmetric with respect to the $x$-axis where
\[ 
\ell_{U^*}(x)=(-d(x),d(x))
\]
Notice that $U$ and $U^*$ have equal  area.
Proposition 2.1 applies and gives
the inequality
\[
J(U)\leq J(U^*)\tag{2}
\]
Now Theorem 1 follows when we start from a non-convex domain $U$.
Namely, by the  result proved in  [Gros], it holds
that after inifinitely many symmetrizations as above using
different directions, the sequence of
$U$-sets converge to a disc.



\newpage






\end{document}