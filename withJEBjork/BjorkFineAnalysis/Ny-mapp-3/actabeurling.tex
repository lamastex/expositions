\documentclass{amsart}


\usepackage[applemac]{inputenc}

\addtolength{\hoffset}{-12mm}
\addtolength{\textwidth}{22mm}
\addtolength{\voffset}{-10mm}
\addtolength{\textheight}{20mm}

\def\uuu{_}


\def\vvv{-}

\begin{document}


\centerline{\bf{XI. Radial limit of functions with  finite Dirichlet integral}}

\bigskip



\noindent
We expose results from the article
\emph{Ensembles exceptionnels}
by Beurling in [Beur] devoted to
the study of functions
$f(\theta)$ on the unit circle $T$
whose harmonic extensions $H\uuu f$
to  $D$ have a finite Dirichlet integral.
A real\vvv valued functions $f(\theta)$  on the unit circle $T$
has a Fourier series:
\[
 f(\theta)=\frac{a\uuu 0}{2}+
\sum\uuu {n=1}^\infty\,a\uuu n\cdot \cos\,n\theta+
\sum\uuu {n=1}^\infty\,b\uuu n\cdot \sin\,n\theta
\]
We say that $f$ belongs to the class
$\mathcal D$ if
\[
\sum\uuu {n=1}^\infty\,n(a^2\uuu n+b^2\uuu n)<\infty\tag{*}
\]
The sum in (*) is denoted by $D(f)$
and is called the Dirichlet norm.
Denote by $\mathcal E\uuu f$ the set of all $\theta$ where the 
partial sums of the Fourier series of $f$ does not converge.

\medskip

\noindent
{\bf{0.1 Theorem.}}
\emph{For each $f\in \mathcal D$ 
the outer capacity of $\mathcal E\uuu f$ is zero.}
\medskip

\noindent
{\bf{Remark.}}
Recall from XXX that if $E\subset T$
then its outer capacity is defined by
\[ 
\text{Cap}^*(E)=\inf\uuu {E\subset U}\, \text{Cap}(U)
\]
with the infimum taken over open neighborhoods of $E$.
\medskip

\noindent
The proof of Theorem 0.1 has two essential ingredients which are announced in
Theorem 0.2 and 0.3 below.
First, given some $f\in\mathcal D$ with constant term $a\uuu 0=0$
we obtain the harmonic
function
in the open disc defined by
 \[
 f(r,\theta)=
\sum\uuu {n=1}^\infty\,r^n(a\uuu n\cdot \cos\,n\theta+
b\uuu n\cdot \sin\,n\theta)
\]
The partial derivative with respect to $r$ becomes:
\[
 f'\uuu r(r,\theta)=\sum\uuu {n=1}^\infty\,n\
 \cdot r^{n\vvv 1}(a\uuu n\cdot \cos\,n\theta+
b\uuu n\cdot \sin\,n\theta)\tag{1}
\]
Define the function $F$ in $D$ by
\[
F(r,\theta)= \int\uuu 0^r\, |f'\uuu s(s,\theta)|\cdot ds\tag{2}
\]
Thus, for each $\theta$ we integrate
the absolute value of  (1) along a ray from the origin.
For every fixed $\theta$
$r\mapsto F(r,\theta)$ is non\vvv decreasing 
and hence
there exists a limit
\[
\lim\uuu{r\to 1}\, F(r,\theta)=F^*(\theta)\tag{3}
\]
The limit value can be finite or $+\infty$.
It is clear that if (3) is finite then there exists the radial limit
\[
\lim\uuu{r\to 1}\, f(r,\theta)=f^*(\theta)\tag{4}
\]

\medskip

\noindent
Next, recall from the result in [Series]
that when the radial limit (4)
exists, then
Fourier's partial sums converge to  $f^*(\theta)$
which entails that
the following inclusion holds
for every $\rho>0$:
\[
\mathcal E\uuu f\subset \{ F^*(\theta)>\rho\}\tag{5}
\]


\noindent
We conclude that  
Theorem 0.1 follows if 
\[
\lim_{\rho\to+\infty}\,\text{Cap}\{F^*>\rho\}=0\tag{6}
\]
Here (6) follows from the following:

\medskip

\noindent
{\bf{0.2 Theorem.}}
\emph{Let $f\in\mathcal D$ where $a\uuu 0=0$
and $D(f)=1$. Then}
\[ 
\text{Cap}(\{F^*>\rho\})\leq e^{\vvv \rho^2}
\quad\text{hold for every}\,\,\,  \rho>0
\]

\medskip


\noindent
The essential step to get
Theorem 0.2 relies upon  the following
inequality:

\bigskip

\noindent{\bf{0.3 Theorem.}}
\emph{For each $f\in\mathcal D$ with $a\uuu 0=0$
one has   $F^*\in\mathcal D$ and}
\[ 
D(F^*)\leq D(f)
\]


\noindent
Theorem 0.3 is
proved in � 1 and
after we deduce
deduce 
Theorem 0.2 in � 2.
Before we proceed to � 1 we 
need some preliminary  results.
\medskip

\centerline{\bf{0.4 On logarithmic potentials.}}
\medskip


\noindent
Let $\mu$ be a probability measure on $T$ and put
\[
U\uuu \mu(z)=\int\ \log\,\frac{1}
{|z\vvv\zeta|}\cdot d\mu(\zeta)
\]
This is a harmonic function in $\{|z|<1\}$
and passing to its radial limits as $r\to 1$
the energy integral is defined by:
\[
J(\mu)=\lim\uuu{r\to 1}\,\int\, U\uuu \mu(r,\theta)\cdot d\mu(\theta)=
\int\, U\uuu \mu(\theta)\cdot d\mu(\theta)\tag{*}
\]
One says that $\mu$ has finite energy when (*) is finite.
To check when this holds
we use
polar coordinates
in $D$ and the   series expansion: 
\[ 
U\uuu\mu(r,\theta)=\sum\, \frac{r^n}{n}(
h\uuu n\cos\,n\theta+
k\uuu n\sin\,n\theta)
\]
where $\{h\uuu n\}$ and $\{k\uuu n\}$ are real 
numbers which will be determined in (2) below.
Then  $J(\mu)$ is 
the limit of the following expression as $r\to 1$:
\[
 \int \,U\uuu\mu(r,\phi)\cdot d\mu(\phi)
=\iint\, \log\, \frac{1}{|1\vvv re^{i(\phi\vvv\theta)}|}
d\mu(\phi)\cdot d\mu(\theta)\tag{1}
\]
To compute the right hand side we expand the
complex log\vvv function:
\[
\log\,\frac{1}{1\vvv re^{i(\phi\vvv\theta)}}
=\sum\uuu {n=1}^\infty \, \frac{r^n}{n}\cdot
e^{in(\phi\vvv\theta)}
\]
Taking real parts it follows that
(1) is equal to
\[
\sum\uuu {n=1}^\infty \, \frac{r^n}{n}\cdot
\cos\, n(\phi\vvv\theta)\cdot d\mu(\phi)\cdot d\mu(\theta)
\]
Now we use the trigonometric formula
\[
\cos\, n(\phi\vvv\theta)=\cos n\phi\cdot \cos n\theta+
\sin n\phi\cdot \sin n\theta
\]
Put 
\[ 
h\uuu n=\int\, \cos\,n\theta\cdot d\mu(\theta)\quad\text{and}\quad
k\uuu n=\int\, \sin\,n\theta\cdot d\mu(\theta)\tag{2}
\]
From the above it follows that
\[ 
J(\mu)= \sum \frac{1}{n}(h^2\uuu n+k\uuu n^2)\tag{3}
\]
\medskip

\noindent
Next, let   $g(\theta)\in\mathcal D$
with Fourier coefficients
$\{a\uuu n\}$ and $\{b\uuu n\}$ where $a\uuu 0=0$.
Then we have

\[
\int\, g\cdot d\mu=
\sum\, a\uuu n\cdot h\uuu n+b\uuu n\cdot k\uuu n
\]
and Cauchy\vvv Schwarz inequality gives:
\[
[\int\, g\cdot d\mu]^2\leq S(g)\cdot J(\mu)\tag{4}
\]

\medskip

\noindent
From the above we obtain the following:
\medskip


\noindent
{\bf{0.5 Theorem.}}
\emph{For each probability measure $\mu$ with finite energy and every
function $g(\theta)\in\mathcal D$  which is lower semi\vvv continuous
one has the 
inequality}

\[
\bigl[ \int\, g(\theta)\cdot d\mu(\theta)\bigr]^2\leq
S(g)\cdot J(\mu)
\]
\bigskip


\noindent
{\bf{Remark.}} Above the lower semi\vvv continuity is imposed in order to
ensure that the Borel integral of $g$ with respect to
$\mu$ is defined.









\bigskip





\noindent
\centerline{\bf{1. Proof of Theorem 0.3}}

\medskip

\noindent
To begin with one has

\medskip

\noindent
{\bf{1.1 Lemma.}} \emph{The function $F$ is subharmonic in $D$.}
\medskip

\noindent
For each fixed $0<\alpha<1$
we define the function $\phi\uuu\alpha$ in $D$ by
\[
\phi\uuu \alpha(x,y)= \frac{\partial}{\partial\alpha}\, f(\alpha x,\alpha y)
=x\cdot f'\uuu x(\alpha x,\alpha y)+y\cdot f'\uuu y(\alpha x,\alpha y)
\]
Notice
that the function $f\uuu\alpha(x,y)= f(\alpha x,\alpha y)$
is harmonic and (1) means that 
\[ 
\phi\uuu \alpha=(x\partial\uuu x+y\partial \uuu y)(f\uuu \alpha)
\] 
where
$\mathfrak{e}= x\partial\uuu x+y\partial \uuu y$ is the 
Euler field. As explained in XX this first order operator
satisfies the identity
\[ 
\Delta\circ \mathfrak{e}= \Delta+\mathfrak{e}\cdot \Delta
\]
in the ring of differential operators and hence
$\phi\uuu\alpha$ is harmonic.
Next, the absolute value of a harmonic function is subharmonic so
$\{|\phi\uuu\alpha|\}$  yield subharmonic functions and
a change of variables gives:
\[ 
F=\int\uuu 0^1\, |\phi\uuu\alpha|\cdot d\alpha
\]
This shows that $F$ is a Riemann integral of subharmonic functions
which in compact subsets of $D$ is uniformly approximated by 
finite sums 
\[
\frac{1}{N}\sum\uuu {k=1}^{k=N}  |\phi\uuu{k/N}|
\]
Lemma 1.1 follows since
a convex sum of subharmonic
functions again is subharmonic.
\bigskip

\noindent
{\bf{An inequality.}}
The function $F(r,\theta)$
is continuous and its derivative with respect to $r$
exists and equals $|f'\uuu r(r,\theta)|$.
But the partial derivative $\partial F/\partial\theta$ may have jump 
discontinuities 
along rays where the derivative  $f'\uuu r$ has a zero. However,
this cannot occur too  often so when $0<r<1$ is fixed there exists the integral
\[
I(r)= \int\uuu 0^{2\pi}\, (\frac{\partial F}{\partial\theta}(r,\theta))^2 \cdot d\theta
\]
\medskip

\noindent
We have proved that $F$ is subharmonic and by its construction
the partial derivative
$\partial F/\partial r$ is non\vvv negative.
The  result in Chapter V:B:xxx gives
 
\medskip
 
 \noindent
{\bf{1.2 Lemma.}} \emph{The inequality below holds for each $0<r<1$:}
\[ 
I(r)\leq 
r^2\cdot \int\uuu 0^{2\pi}\, 
\bigl(\frac{\partial F}{\partial r}(r,\theta)\bigr)^2 \cdot d\theta\tag{*}
\]







\medskip

\noindent
{\bf{1.3 Dirichlet integrals.}}
Let $f\in \mathcal D$ with $a\uuu 0=0$ and
construct the Dirichlet integral 
\[
\text{Dir}(f)=\frac{1}{\pi}\cdot \iiint\uuu D\,  
[(f'\uuu x)^2+(f'\uuu y)^2]\cdot dxdy
\]
Then one has the equality:
\[
\text{Dir}(f)=D(f)\tag{*}
\]
To see this we identify $f(r,\theta)$ with the real part of the analytic function

\[
G(z)=\sum\, (a\uuu n\vvv i\cdot b\uuu n)\cdot z^n
\]
The Cauchy\vvv Riemann equations give
\[
\text{Dir}(f)=\frac{1}{\pi}\cdot \iiint\uuu D\,  
|G'(z)|^2\cdot dxdy
\]
Now the reader can verify that the double integral above is equal to $D(f)$.
Notice that (*) identifies  $\mathcal D$ with the space
of real\vvv valued functions on $T$ whose harmonic extensions to�$D$ have a finite Dirichlet
integral.
\medskip

\noindent
{\bf{1.4 Exercise.}}
Show that the Dirichlet integral 
of a function $g$ of class $C^2$ in $D$
also is given by the double integral
\[
\frac{1}{\pi}\cdot 
\int\uuu 0^1\int\uuu 0^{2\pi}\, 
\bigl[\, r^2\cdot \bigl(\frac{\partial g}{\partial r}\bigr)^2+
\frac{1}{r^2}\cdot\bigl(\frac{\partial g}{\partial \theta}\bigr)^2\,\bigr]
\cdot r\cdot d\theta dr\tag{i}
\]
Show also that if $g$ is harmonic then
\[
 \text{Dir}(g)= \frac{2}{\pi}\cdot 
\int\uuu 0^1\int\uuu 0^{2\pi}\, 
\bigl(\frac{\partial g}{\partial r}\bigr)^2
\cdot r\cdot d\theta dr\tag{ii}
\]

\bigskip


\centerline{\emph{1.5 Final part of the proof of Theorem 0.3}}
\bigskip

\noindent 
Apply (i) in 1.4 with $g=F$ where
the inequality in Lemma 1.2 and
an integration 
with respect
to $r$ give
\[ 
\text{Dir}(F)\leq \frac{2}{\pi}\cdot 
\int\uuu 0^1\int\uuu 0^{2\pi}\, 
\bigl(\frac{\partial F}{\partial r}\bigr)^2\cdot r\cdot d\theta dr\tag{1}
\]

\noindent
Next,  the construction of $F$ gives the equality
\[
(\frac{\partial F}{\partial r}\bigr)^2=
(\frac{\partial f}{\partial r}\bigr)^2
\] 
in the whole disc $D$.
Then (1) and the equality (ii) applied to the harmonic function
$f$ give:
\[
\text{Dir}(F)\leq \text{Dir}(f)=D(f)\tag{2}
\]
where the last equality used (*) in 1.3.
Next, construct the harmonic extension of 
the boundary function $F^*(\theta)$ which we denote by $H\uuu F$.
Here we have the equations
\[
D(F^*)= D(H\uuu F)\tag{3}
\]
Next, recall  that the Dirchlet integral is minimized when we take
a harmonic extension which entails that
\[
\text{Dir}(H\uuu F)\leq
\text{Dir}(F)\tag{4}
\]
Hence (2\vvv 4) give the requested inequality
\[
D(F^*)\leq D(f)
\]






\bigskip


\centerline{\bf{2. Proof of Theorem 0.2}}

\bigskip



\noindent
Let $\rho>0$ and
apply  Theorem 0.5  to the function $g=F^*$ and the equilibrium 
distribution $\mu$ assigned to the
set $E=\{F^*>\rho\}$. This   gives
\[ 
\rho^2\leq \bigl[\int F^*\cdot d\mu\,\bigr]^2\leq S(F^*)\cdot J(\mu)\tag{1}
\]
Now $D(F^*)\leq D(f)=1$ holds by Theorem 0.3
and hence we have:
\[
\rho^2\leq J(\mu)\tag{2}
\]
Next, recall from � XX that $J(\mu)$ is 
the the constant value $\gamma(E)$
of the potential function 
$U\uuu\mu$ restricted to $E$.
Hence (1) gives
\[
e^{\vvv\gamma(E)}\leq e^{\vvv \rho^2}\tag{3}
\]
By definition the left hand side is the capacity of $E$
which proves Theorem 0.2.



\bigskip

\centerline{\bf{3. An application}}
\bigskip


\noindent
Let $\Omega$ be a simply connected domain which contains
the origin in the complex $\zeta$\vvv plane and
$\partial\Omega$ contains a relatively open set given
by
an interval $\ell$ situated on the line
$\mathfrak{Re}\,\zeta=\rho$ for some $\rho>0$.
Consider the harmonic measure
$\mathfrak{m}\uuu 0^\Omega(\ell)$.
In other words, the value at the origin of the harmonic function
in $\Omega$ which is 1 on $\ell$ and zero on
$\partial\Omega\setminus\ell$.
We shall find an upper bound for (*) 
from the introduction in the family of simply connected domains
which contain the origin and $\ell$ and  have area $\pi$.
To attain this we
consider the conformal map $\phi$ from the unit disc onto
$\Omega$ with $\phi(0)=0$.
The invariance of harmonic measures gives:
\[
\mathfrak{m}\uuu 0^\Omega(\ell)=
\mathfrak{m}\uuu 0^D(\alpha)
\] 
where $\alpha$ is the interval on $T$ such that
$\phi(\alpha)=\ell$.
For an interval on the unit circle one has  the equality
\[
\text{Cap}(\alpha)= \sin\,\alpha/4
\]
At the same time
$\mathfrak{m}\uuu 0^D(\alpha)=\frac{\alpha}{2\pi}$ which entails that
\[
\mathfrak{m}\uuu 0^\Omega(\ell)=
\frac{2}{\pi}\,\arcsin\, \text{Cap}(\alpha)\tag{1}
\]


\noindent
There remains to estimate  last term above.
Put $u=\mathfrak{Re}\,\phi$.
The inclusion $\ell\subset\mathfrak{Re}\zeta=\rho$
means that $u=\rho$ on $\ell$.
So when $\phi$ is considered in the class $\mathcal S$
we have the inclusion
\[
\alpha\subset\{|\phi|>\rho\vvv \epsilon\}
\]
for each $\epsilon>0$.
Next, since the area of $\phi(D)=\pi$
we have $S(u)=1$
and Theorem 0.2 gives 
\[
\text{Cap}(\alpha)\leq e^{\vvv\rho^2}
\]
Hence we have proved the general inequality
\[
\mathfrak{m}\uuu 0^\Omega(\ell)\leq \frac{2}{\pi}\cdot \arcsin\,
e^{\vvv\rho^2}\tag{**}
\]
\medskip

\noindent
{\bf{Remark.}}
There exists
a special simply connected domain $\Omega$ for which equality holds
in (**).
See [Frostman: p. 39] : Potential theory.




















\newpage





















\end{document}