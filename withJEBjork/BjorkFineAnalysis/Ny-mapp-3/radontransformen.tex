%\documentclass{amsart}
%\usepackage[applemac]{inputenc}

%\addtolength{\hoffset}{-12mm}
%\addtolength{\textwidth}{22mm}
%\addtolength{\voffset}{-10mm}
%\addtolength{\textheight}{20mm}


%\def\uuu{_}


%\def\vvv{-}

 
%\begin{document}

\centerline{\bf\large{Homegeneous distributions and the Mellin transform}}

\bigskip

\centerline{\emph{Contents}}

\bigskip

\noindent
\emph {A. Polar distributions}
\bigskip

\noindent
\emph {B. Homogeneous distributions}
\bigskip

\noindent
\emph {C. The family $|P(x,y)|^\lambda$}
\bigskip

\noindent
\emph{The Radon transform}



\bigskip

\noindent
\emph{The Mellin transform}

\bigskip

\noindent
{\bf{Introduction.}}
We establish some results where analytic function theory
is used in connection with distributions and
asymptotic expansions.
Special attention in the first sections is given to
homogeneous distributions in
${\bf{R}}^2$
while the last section exposes a famous result due to Mellin in [Mellin]
which has a wide range of applications.
A separate section is devoted to the Radon transform which is
established directly via Fourier analysis, i.e. here one  does not need
analytic function theory.





\bigskip


\centerline{\bf  {A. Polar distributions}}

\bigskip

\noindent
In the $(x,y)$-plane we can use polar coordinates where
$x=r\cdot\text{cos}\,\theta$ and
$y=r\cdot\text{sin}\,\theta$.
If $\phi(x,y)$ belongs to the Schwartz space $\mathcal S$
of rapidly decreasing $C^\infty$-functions  we  restrict $\phi$ to the
circle of radius $r$ which after a dilation is identified
with  unit circle $T$ and
obtain the $\theta$-periodic function
\[ 
\theta\mapsto \phi_r(\theta)= \phi(r\cdot\text{cos}\,\theta,r\cdot\text{sin}\,\theta)\tag{0.1}
\]
\medskip

\noindent
Let $\nu$ be a distribution on $T$ which via Fourier series expansions is
defined as in XX.
For each $r>0$ we
can evaluate $\nu$ on the
$C^\infty(T)$-function $\phi_r$ which yields a map:
\[ 
r\mapsto \nu(\phi_r)\quad\colon\, r>0\tag{0.2}
\]

\medskip

\noindent
{\bf{A.0 Exercise.}}
Show that (2) gives a $C^\infty$-function defined on
$\{r>0\}$. More precisely, verify that the derivative becomes:


\[
\frac{d}{dr}(\nu(\phi_r))=
\nu(\text{cos}\,\theta\cdot\partial_x(\phi)(r\cdot\text{cos}\,\theta,r\cdot\text{sin}\,\theta)+
\text{sin}\,\theta\cdot\partial_y(\phi)(r\cdot\text{cos}\,\theta,r\cdot\text{sin}\,\theta))
\]

\noindent
More generally, show that for each $m\geq 2$ the derivative of order $m$ becomes:
\[
\frac{d^m}{dr^m}(\nu(\phi_r))=\sum_{j=0}^{j=m}\,
\binom{m}{j}\,
\nu\bigl(\text{cos}^j\,\theta\cdot \text{sin}^{m-j}\,\theta\cdot
\partial^j_x\partial_y^{m-j}(\phi)_r\bigr)\tag{*}
\]
\medskip


\noindent
Show also that the function in (0.2) decreases as $r\to+\infty$. More precisely, for
each positive integer $N$ one has
\[
\lim_{r\to\infty} r^N\cdot\nu(\phi_r)=0\tag{**}
 \]
 
\medskip



\noindent
{\bf{A.1 The function $V_\lambda$.}}
Let $\phi\in\mathcal S$.
Using (**) it follows that
if $\lambda$ is a complex number with
$\mathfrak{Re}(\lambda)>-2$, then  there exists  the absolutely convergent integral
\[ 
V_\lambda(\phi)=
\int_0^\infty\,
r^{\lambda+1}\cdot \nu(\phi_r)\cdot dr\tag{1}
\]


\medskip

\noindent
{\bf{A.2 Exercise }}
Show that $V_\lambda$
is an analytic function of
$\lambda$ in the open half-plane
$\mathfrak{Re}(\lambda)>-2$.
Next, use a partial integration with respect to $r$ and show that:
\[
(\lambda+2)V_\lambda(\phi)=-\int_0^\infty
r^{\lambda+2}\cdot \frac{d}{dr}[\nu(\phi_r)]\cdot dr
\]
Continue this procedure and show that for every
$N\geq 1$ one has:
\[
(\lambda+2)\cdots(\lambda+2+N)\cdot V_\lambda(\phi)=
(-1)^{N+1}\int_0^\infty
r^{\lambda+2+N}\cdot \frac{d^N}{dr^N}[\nu(\phi_r)]\cdot dr\tag{*}
\]




\noindent
From (*) in the exercise above
we can conclude that following:


\medskip

\noindent
{\bf{A.3 Proposition.}} \emph{The function $V_\lambda(\phi)$ 
extends to a meromorphic
function in the whole complex
$\lambda$-plane with at most simple poles at
the integers $-2,-3,\ldots$.}



\bigskip


\noindent{\bf{A.4 Polar distributions.}}
As $\phi$ varies in $\mathcal S$ 
we obtain a
distribution-valued function
$V_\lambda$.
If $\delta>0$ and 
$\phi(x,y)\in\mathcal S$ is identically zero in the
disc $\{x^2+y^2<\delta^2\}$, then we only integrate
(1) in A.1 when $r\geq\delta$ and we notice that the function
\[
 \lambda\mapsto \int_\delta^\infty\,
r^{\lambda+1}\cdot \nu(\phi_r)\cdot dr
\]
is an entire function of $\lambda$ whose complex derivative is given by
\[
 \lambda\mapsto \int_\delta^\infty\,
\log r\cdot r^{\lambda+1}\cdot \nu(\phi_r)\cdot dr
\]

\medskip

\noindent
Regarding the distribution-valued function $V_\lambda$
this means that 
eventual poles  consist of Dirac distributions at the origin.
Let us first study if   a pole can occur  at $-2$.
With $\lambda=-2+\zeta$ we have
\[
\zeta\cdot V_{-2+\zeta}(\phi)=-
\int_0^\infty\, r^\zeta\cdot 
\frac{d}{dr}[\nu(\phi_r)]\cdot dr\tag{i}
\]
Since $r^\zeta\to 1$ holds for each $r>0$ as
$\zeta\to 0$, the right hand side has the limit
\[
\int_0^\infty\, 
\frac{d}{dr}[\nu(\phi_r)]\cdot dr=\nu(\phi_0)=
\nu(1_T)\cdot \phi(0)\tag{ii}
\]
where $1_T$ is the identity function on $T$ on which $\nu$ is evaluated.
Hence $V_\lambda$ has a pole at $\lambda=-2$ if and only if
$\nu(1_T)\neq 0$ and in this case the polar distribution is
$\nu(1_T)$ times the Dirac distribution $\delta_0$.

\medskip

\noindent
{\bf{A.5 Exercise}}
Use the functional equation formula (*)
in A.2 to show the following:

\bigskip


\noindent
{\bf{A.6 Proposition }}
\emph{For each $N\geq 1$ the polar distribution at
$\lambda=-N-2$ is zero if and only if }

\[ 
\nu(\text{cos}^j\,\theta\cdot \text{sin}^k\,\theta)=0
\quad\text{when }\,\,\,j+k=N
\]

\medskip

\noindent
Thus, no pole occurs if and only if $\nu$ vanishes on the
$N+1$-dimensional subspace of $C^\infty(T)$
spanned by
$\{\text{cos}^j\,\theta\cdot \text{sin}^{N-j}\,\theta)\quad\colon
0\leq j\leq N\}$.
Next, if a pole occurs we have a Laurent series:
\[
V_{-N-2+z}=
\frac{\gamma_N}{\zeta}+ V_{-N-2}+\sum_{n=1}^\infty
\rho_j\cdot \zeta^j
\]
where $\gamma_N$ is the polar distribution.
\medskip

\noindent
{\bf{A.7 Exercise,}}
Show that if a pole occurs then
$\gamma_N$ is the Dirac distribution given by:
\[ 
\gamma_N(\phi)=\frac{1}{N\,!}\cdot
\sum_{j=0}^{N}\,
\nu((\text{cos}^j\,\theta\cdot \text{sin}^{N-j}\,\theta)\cdot
\partial_x^j\partial_y^{N-j}(\phi)(0)\tag{*}
\]
\newpage







\centerline
{\bf{B. Homogeneous distributions.}}
\medskip


\noindent
A distribution $\mu$ defined outside the origin in
${\bf{R}}^2$ is homogeneous of degree $\lambda$ if
\[ 
\mathcal E(\mu)=\lambda\cdot \mu\tag{*}
\] 
where $\mathcal E=x\partial_x+y\partial_y$ is the radial vector field.
Denote by $\mathcal S^*(\lambda)$ the family of all
$\lambda$-homogeneous distribution in
${\bf{R}}^2\setminus\{0\}$.
\bigskip

\noindent
{\bf{B.1 Proposition.}}
\emph{$\mathcal S^*(\lambda)$ is in a 1-1 correspondence with
$\mathfrak{Db}(T)$ when  we for every distribution $\nu$ on $T$
consider the restriction of $V_\lambda$ to
${\bf{R}}^2\setminus\{0\}$.}
\medskip

\noindent
{\bf{B.2 Exercise.}}
Prove this result. The hint is to verify that
one has the equality

\[ 
\mathcal E(V_\lambda)=\lambda\cdot V_\lambda
\] 
when one  starts from an arbitrary distribution $\nu$ on $T$.
\bigskip


\noindent
{\bf{B.3 The space $\mathcal S^*[\lambda]$}}.
This the space of tempered distributions on
${\bf{R}}^2$ which are everywhere homogeneous. So a tempered
distribution $\mu$ belongs to
$ \mathcal S^*[\lambda]$ 
if and only if
\[ 
\mathcal E(\mu)=\lambda\cdot\mu
\]
where the equality holds in $\mathcal S^*$.
\medskip


\noindent
{\bf{B.4 Example of distributions in
$\mathcal S^*[\lambda]$}}.
If $\nu$ is a distribution on $T$
we construct the meromorphic function $V_\lambda$
here we have the equality

\[
 \mathcal E(V_\lambda)=\lambda\cdot V_\lambda
 \quad\colon\,\mathfrak{Re}(\lambda)>-2\tag{i}
 \]
 
\noindent
Let 
$\lambda_*$ be a complex number such that
$V_\lambda$ has no pole at 
$\lambda_*$. By analyticity it follows from (i) that the constant term
$V_{\lambda_*}$ satisfies
\[
\mathcal E(V_{\lambda_*})=\lambda_*\cdot V_{\lambda_*}\tag{ii}
\]
Hence $V_{\lambda_*}$ belongs to $\mathcal S^*[\lambda_*]$.
By Proposition A.3 no poles occur 
when
$\lambda_*$ is outside the set
$\{-2,-3,\ldots\}$. This gives the following:


\medskip

\noindent
{\bf{B.5 Proposition.}}
\emph{For each $\lambda_*$ outside
the set $\{-2,-3,\ldots\}$ there exists a map}
\[
\nu\mapsto V_{\lambda_*}
\] 
\emph{from $\mathfrak{Db}(T)$ into $\mathcal S^*[\lambda_*]$.}
\bigskip





\noindent
{\bf{B.6 The action  $\mathcal E$
on
Dirac distributions.}}
The complex vector space of all Dirac distributions
is a direct sum of the subspaces
\[
\text{Dirac}[m]=
\oplus\,{\bf{C}}\cdot \partial_x^k\partial_y^j(\delta_0)
\quad\colon\, j+k=m\tag{1}
\]


\noindent
Next, in the ring
$\mathcal D$ of differential operators we
have the identity
\[ 
\mathcal E=\partial_x\cdot x+\partial_y\cdot y-2
\]
Since $x\cdot \delta_0=y\cdot \delta_0=0$ it follows that
\[ 
\mathcal E(\delta_0)=-2\cdot \delta_0
\]


\medskip

\noindent
In general the reader may verify by an induction over $m$ that

\[ 
\mathcal E(\gamma)= -(m+2)\cdot\gamma
\quad\text{hold for all}\quad \gamma\in\text{Dirac}[m]\tag{2}
\]
\medskip



\noindent
{\bf{B.7 Exercise.}}
Apply Proposition B.1 above and
the results about the action by
$\mathcal E$ on Dirac distributions to show
that the map in Proposition B.5  is \emph{bijective}, i.e. it gives
linear isomorphism between
$\mathfrak{Db}(T)$ and $\mathcal S^*[\lambda]$.

\newpage


\centerline {\bf{B.8 The description of $\mathcal S^*[-2-N]$}}.
\medskip

\noindent
Let $N$ be a non-negative integer. 
Denote by $\mathfrak{Db}(T)[N+2]$
the set of distributions
$\nu$  on
$T$ such that $V_\lambda$ has no pole at
$-2-N$.
Proposition A.6 shows that the quotient space

\[
\frac{\mathfrak{Db}(T)}{\mathfrak{Db}(T)[N+2]}
\]


\noindent
is a complex vector space whose dimension is
$N+1$.
Now the following conclusive result holds:

\bigskip

\noindent
{\bf{B.9 Theorem.}}
\emph{The map $\nu\to V_{-2-N}$
from
$\mathfrak{Db}(T)[N+2]$ 
into $S^*[-2-N]$ is bijective}

\bigskip

\noindent
\emph{Proof.}
That the map $\nu|\to V_{-2-N}$ is 
defined and injective on
$\mathfrak{Db}(T)[N+2]$ is  clear.
There remains to show that the map is surjective. So let
$\mu\in S^*[-2-N]$ be given.
By Proposition B.1 we find
$\nu\in \mathfrak{Db}(T)$ such that $\nu=\mu$ outside the origin. it means that
\[
\mu-\nu=\eta\tag{1}
\]
for some Dirac distribution $\eta$.
There remains  to see that this implies that
$\nu$ belongs to $\mathfrak{Db}(T)[N+2]$.
To show this we first study $V_\lambda$ where $\lambda=-N-2+\zeta$
and consider the Laurent series after Proposition A.6.
By Exercise B.2 we have

\[
\mathcal E(\frac{\gamma_N}{\zeta}+V_{-N-2}+\sum\rho_j\zeta^j)=
(-N-2+\zeta)V_{-N-2+\zeta}=
\]


\[ 
(-N-2)\cdot\frac{\gamma_N}{\zeta}+
\gamma_N+(-N-2)V_{-N-2}+(-N-2+\zeta)\cdot \sum\rho_j\zeta^j
\] 
Identifying the constant term we get

\[
\mathcal E(V_{-N-2})=\gamma_N-(N+2)V_{_N-2}\tag{2}
\]
At the same time $\mu=V_{-N-2}+\eta$  and since
$\mu\in S^*[-N-2]$ is assumed, it follows that

\[
\mathcal E(V_{-N-2})+(N+2)V_{-N-2}+
\mathcal E(\eta)+
(N+2)\eta=0 \tag{3}
\]
Together (2-3) give

\[
\mathcal E(\eta)+
(N+2)\eta=
-\gamma_N\tag{4}
\]
At this stage we use the results from B.6. 
From the direct sum decomposition (1) in B.6
we can expand $\gamma$ and write

\[
\eta=\sum\,\eta_m\quad\colon\,\eta_m\in
\text{Dirac}[m]
\]
Then (2) in B.6 gives

\[
\mathcal E(\eta)+
(N+2)\eta=\sum\, (N+2-m-2)\cdot\eta_m\tag{5}
\]
\medskip

\noindent
At the same time
(*) in Exercise A.7 gives
$\gamma_N\in\text{Dirac}[N]$.
and by (4) and (5) we have

\[
-\gamma_N=\sum\, (N+2-m-2)\cdot\eta_m\tag{6}
\]
The direct sum decomposition (1) in B.6 entails that $\eta_m=0$ for each
$m\neq N$ and with $m=N$ we do not get any contribution in
the right hand side which gives $\gamma_N=0$ and hence
$\nu$ belongs to $\mathfrak{Db}(T)[N+2]$ as required.
\bigskip

\noindent
{\bf{B.10 Example.}}
Let $\nu=1_T$ and consider the distribution $V_{-1}$.
It is given by


\[ 
V_{-1}(\phi)= 
\int_0^\infty\,\bigl [\int_0^{2\pi}\, \phi(r,\theta)d\theta \,\bigr]\cdot dr=
\iint\, \frac{\phi(x,y)}{\sqrt{x^2+y^2}}\cdot dxdy
\]
Hence the $L^1$-density $\frac{1}{\sqrt{x^2+y^2}}$ has no homogeneous
extension. On the other hand, with $\nu=\cos\theta$
we see that the $L^1$-density
$\frac{x}{x^2+y^2}$ is $-1$-homogeneous.
With $z=x+iy$ we have the $L^1$-density
\[
\frac{1}{z}= \frac{x-iy}{x^2+y^2}
\] 
and conclude that it is homogeneous of degree $-1$.
\bigskip


\centerline {\bf{B.11 Fourier transforms.}}
\medskip

\noindent
The Fourier transform maps tempered distributions in the $(x,y)$-space 
to tempered distribution the $(\xi,\eta)$-space.
By the laws from XX we see tha the radial field $\mathcal E=x\partial_x+y\partial_y$
is sent into the first order differential operator 

\[
(i\partial_\xi)\cdot i\xi+(i\partial_\eta)\cdot i\eta=
-\partial_\xi\cdot \xi-\partial_\eta\cdot \eta=-\xi\partial_\xi-1-
-\eta\partial_\eta-1=-\mathcal E^*-2
\]

\noindent
where $\mathcal E^*$ is the Euler field in the 
$(\xi,\eta)$-space.
It follows that for every complex number one has
the implication

\[
\mu\in S^*[\lambda]\implies\widehat\mu\in S^*[-\lambda-2]\tag{*}
\]
Moreover, by Fourier's inversion formula we
get the opposite implication. Hence the Fourier transform
yields a bijective map in (*). 


\noindent
{\bf{B.12 Example.}}
The Dirac measure $\delta_0$
belongs to $S^*[-2]$. Its Fourier transform
is therefore in $S^*[0]$ and this is clear since
$\widehat\delta_0$ is  the constant density
times $\frac{1}{2\pi}$.
Notice also that the Fourier transform sends
$S^*[-1]$ into itself.
\bigskip

\noindent
{\bf{B.13 The $\lambda$-maps on $\mathfrak{Db}(T)$.}}
Let $\lambda$ be a complex umber outside the set
$\{-2,-3,\ldots\}$. 
To each $\nu\in\mathfrak{Db}(T)$
we get the distribution $V_\lambda$ which belongs to
$S^*[\lambda)$.
It follows that the Fourier transform
$\widehat V_\lambda$ belongs to $S^*[-\lambda-2]$ and
this gives a unique distribution
$\nu^*$ on $T$
such that

\[
\widehat V_\lambda=V^*_{-\lambda-2}\tag{*}
\]
Keeping $\lambda$ fixed this means that we get a bijective map from
$\mathfrak{Db}(T)$ to itself defined by
\[ 
\nu\mapsto \nu^*
\]
where the rule is that (*) holds.
We denote this map by
$\mathcal H_\lambda$ and refer to this as the
$\lambda$-map on $\mathfrak{Db}(T)$.









\bigskip

\centerline{\bf  {C. The family $\int\,|P|^{2\lambda}$}}

\bigskip

\noindent
Let $P(x,y)$ be a polynomial of the two real variables. In general it
may have complex coefficients and no special assumption
is imposed on its zero set.
When $\mathfrak{Re}(\lambda)>0$
it is clear that we obtain a tempered distribution defined by

\[
\phi\mapsto \int\, |P(x,y)|^{2\lambda}\cdot\phi(x,y)\cdot dxdy\tag{*}
\]
Moreover, we see  that this gives a distribution-valued holomorphic
function in the right half-plane.
In the case when $P$ has real coefficients we can consider
a connected component $\Omega$ of the set $\{P>0\}$
and construct


\[
\phi\mapsto \int_\Omega\, P(x,y)^{2\lambda}\cdot\phi(x,y)\cdot dxdy\tag{**}
\]
\bigskip

\noindent
It turns out that both (*) and (**) extend to meromorphic
distribution valued functions in the whole complex
$\lambda$-plane and there exists a 
finite set of positive rational numbers
$\{ 0<q_\nu\}$
such that the poles are contained in the set
\[ 
\cup_{k=1}^m\, \mathcal A_k=\{-q_k-n\quad\colon n=0,1,2\ldots\}
\]
\medskip


\noindent
{\bf{Remark.}}
Special cases of this result appeared  in
work by Marcel Riesz around 1935 who used meromorphic extensions to construct
fundamental solutions to PDE-equations.
More extensive classes appear in the text-book 
[G-S] by Gelfand and Shilov. We remark that these authors also treated 
cases where $P$ depends upon more than 
two variables. But  the class of polynomials was
quite restricted. 
In the impressive work [Nilsson] it was proved 
the general fact that
distribution valued functions as above extend to meromorphic functions
to the whole complex $\lambda$-plane with poles confined to
an arithmetic progression as above. We remark that Nils Nilsson 
established this result for polynomials in any number of variables.
Even though it was not stated
explicitly
in [Nilsson] who also analyzed multi-valued penomenta caused by
non-vanishing monodromy, the meromorphic extension above 
follows from [Nilsson] when one takes a
Mellin transform.
Here we are content to treat the case of two variables and the
subsequent proof is in the spirit of
Nilsson's work which relies upon some very ingenious applications from the classic
theory about 
algebraic functions of two variables.
\medskip

\noindent
{\bf{The  functional equation.}}
Using algebraic properties of  the Weyl algebra 
of differential operators with polynomial coefficients, Joseph Bernstein gave
a  simple proof 
of the mere existence of the meromorphic
extension in [Bernstein]. Of course, just like  in Nilsson's
case the existence was established in any number of variables.
The new  point in  Bernstein's work  
is that the meromorphic
extension can be  achieved by a functional equation. Namely, with
$P$ given the required meromorphic extension in
(*) follows from the existence of a non-zero polynomial $b(\lambda)$ in
${\bf{C}}[\lambda]$ such that

\[
b(\lambda)\cdot\int\,|P(x,y)|^{2\lambda}\cdot\phi(x,y)\cdot dxdy
=
\] 
\[
\sum\,\lambda^k
\cdot\int P(x,y)\cdot |P(x,y)|^{2\lambda}\cdot Q_k(\phi)(x,y)\cdot dxdy
\]
hold for every $\phi$ in $\mathcal S({\bf{R}}^2)$
and  $\{Q_k\}$ is a finite set of differential operators 
indexed by non-negative integers
which  belong 
to the Weyl algebra $A_2({\bf{C}})$.
Above one can choose $b(\lambda)$ of smallest possible degree and it is then
referred
to as the \emph{Bernstein-Sato polynomial} of $P$.
We remark that the tribute to M. Sato stems from his early
discoveries of many functional equations of this kind from
[Sato].
The fact that the 
roots of the $b$-function always are strictly negative rational numbers was
established by Masaki Kashiwarin in the article [Kashiwara] from 1975.
















\bigskip





\centerline{\bf \large The Radon transform}

\bigskip

\noindent
In the article  [Radon] from 1917
Johann Radon 
established  an inversion formula which recaptures 
a  test-function $f(x,y)$ in ${\bf{R}}^2$
via   integrals over  affine lines
in the
$(x,y)$-plane. This family 
is parametrized by pairs $(p,\alpha)$,
where  $p\in{\bf{R}}$ and $0\leq \alpha<\pi$
give  the line
$\ell(p,\alpha)$:
\[
 t\mapsto 
 \bigl(p\cdot \text{cos}\,\alpha-
 t\cdot \text{sin}\,\alpha\, , \,p\cdot \text{sin}\,\alpha+
 t\cdot \text{cos}\,\alpha\bigr)
 \]


\noindent
The Radon  transform of $f$ is a function of the pairs $(\alpha,p)$
defined by:
\[
R_\alpha(p)=\int_{\ell(p,\alpha)}\,f\cdot dt=
\int_{-\infty}^\infty\, 
f\bigl(p\cdot \text{cos}\,\alpha-
t\cdot \text{sin}\,\alpha\, , \,p\cdot \text{sin}\,\alpha+
 t\cdot \text{cos}\,\alpha\bigr)
\cdot dt\tag{*}
 \]
Thus, for a given $\alpha$ we  take the mean value of $f$ over an affine line
which is $\perp$ to the vector $(\cos\alpha,\sin\alpha)$ and 
whose distance to the origin is $|p|$.
We give
an inversion formula which recaptures $f$ from the Radon transform.
To achieve this
we construct the partial  Fourier transform
of $R$ with respect to $p$, i.e. set
\[
 \widehat R_\alpha( \tau)=\int e^{-i\tau p}\cdot R_\alpha(p)\cdot dp\tag{1}
 \]
Consider the linear map
$(p,\tau)\mapsto (x,y)$ where
\[
x= p\cdot \text{cos}\,\alpha-
t\cdot  \text{sin}\,\alpha\quad\text{and}\quad
y=p\cdot \text{sin}\,\alpha+t\cdot \text{cos}\,\alpha\implies
\]
\[
p=\text{cos}(\alpha)\cdot x+
\text{sin}(\alpha)\cdot y\tag{2}
\]
Since  
$ \text{cos}^2\,\alpha+\text{sin}^2\,\alpha=1$ this substitution gives
$dpdt=dxdy$. So
(2) gives
\[
 \widehat R_\alpha(\tau)=
\int\, e^{-i\tau(x\cdot \text{cos}\,\alpha+
y\cdot \text{sin}\,\alpha)}
\cdot f(x,y)\cdot dxdy=
\widehat f(\tau\cdot \text{cos}\,\alpha\,,\,
\tau\cdot \text{sin}\,\alpha)\tag{3}
\]

\noindent
Next, Fourier's inversion formula applied to $f$ gives:
\[
f(x,y)=\frac{1}{(2\pi)^2}\cdot
\int\, e^{i(x\xi+y\eta)}\cdot
\widehat f(\xi,\eta)\cdot d\xi d\eta
\]
Now we use the substitution
$(\tau,\alpha)\mapsto (\xi,\eta)$ where
\[ 
\xi=\text{cos}(\alpha)\cdot \tau\quad\text{and}\quad
\eta=\text{sin}(\alpha)\cdot \tau
\]
Here $d\xi d\eta=|\tau|\cdot d\tau d\alpha$ and then
(3) gives
the equality
\[
f(x,y)=
\frac{1}{(2\pi)^2}\int_0^\pi\,\bigl[
\int_{-\infty}^\infty\,
e^{i\tau(x\cdot \text{cos}\,\alpha+
y\cdot \text{sin}\,\alpha)}
\cdot \widehat R_\alpha(\tau)\cdot |\tau|\cdot d\tau\bigr]\cdot d\alpha\tag{*}
\]
\medskip

\noindent
To get an inversion formula  where 
the partial Fourier transform
$\widehat R_f(\alpha,\tau)$
does not appear we  apply the Fouriers inversion formula in
dimension one. Namely, for each $A>0$ we set
\[ 
K_A(u)=
\frac{1}{2\pi}
\int_{-A}^A\,
e^{i\tau u}\cdot |\tau|\cdot d\tau\tag{4}
\]
This function admits a alternative
description since we have
\[
K_A(u)=
\frac{1}{\pi}
\int_0^A\,
\tau\cdot \text{cos}(\tau u)\cdot d\tau=
\frac{1}{\pi}\cdot \frac{d}{du}\bigl(
\int_0^A\,
\text{sin}(\tau u) \cdot d\tau\bigr)=
\]
\[
\frac{1}{\pi}\cdot \frac{d}{du}\bigl(
\frac{1-\text{cos}(Au)}{u}\bigr)=
\frac{1}{\pi}\cdot [A\cdot \frac{\sin Au}{u}-\frac{1-\text{cos}(Au)}{u^2}]\tag{5}
\]
\medskip

\noindent
Next, 
by the convolution formula for Fourier transforms
the right hand side in (*) becomes
\[
\lim_{A\to\infty}\,
\frac{1}{2\pi}\int_0^\pi\,\bigl[
\int_{-\infty}^\infty\,
R_\alpha(
x\cdot \text{cos}\,\alpha+
y\cdot \text{sin}\,\alpha-u)\cdot K_A(u) du\,\bigr]
\cdot d\alpha\tag{6}
\]

\medskip


\noindent
Put
\[ 
\phi_\alpha(x,y,u)=R_\alpha(
x\cdot \text{cos}\,\alpha+
y\cdot \text{sin}\,\alpha-u)\tag{7}
\]


\noindent
After the substitution
$u\to \frac{s}{A}$ and applying (5) 
the limit in (*) becomes

\[ 
\lim_{A\to\infty}\,
\int_0^\pi\,\bigl[\int_{-\infty}^\infty\, \phi_\alpha(x,y,\frac {s}{A})\cdot \bigl(
A\cdot \frac{\sin s}{s}-
A\cdot \frac{1-\cos s}{s^2}\bigr)\cdot ds\,\bigr ] \cdot d\alpha\tag{**}
\]

\bigskip


\noindent
{\bf{Remark.}}
For further material about the Radon transform
which includes inversion formulas in
every dimension $n\geq 2$ we refer to Helgason's text-book [Helgason].

\newpage


\centerline{\bf \large The Mellin transform}

\bigskip

\noindent
In many situations one encounters a
function $J(\epsilon)$ which is defined for $\epsilon>0$
and has an asymptotic expansion as $\epsilon\to 0$ by
fractional powers which  means  that there
exists a strictly increasing sequence of real numbers
$0<q_1<q_2\ldots$with $q_N\to +\infty$
and constants $c_1,c_2,\ldots$
such that for every $N$ there exists some $\delta>0$ 
which in general depends upon $N$ and:
\[
\lim_{\epsilon\to 0}\,
\frac{J(\epsilon)-(c_1e^{q_1}+\ldots +c_Ne^{q_N})}
{\epsilon^{q_N+\delta}}= 0\tag{*}
\] 

\noindent
It is clear that the constants $\{c_k\}$ are uniquely determined 
by $J$ and the $q$-numbers if (*) holds.
We are only concerned with the behavior of $J$ for small $\epsilon$
and may therefore assume that $J(\epsilon)=0$ when $\epsilon>1$.

\medskip

\noindent{\bf{Mellin's asymptotic formula.}}
Assume that  $J(\epsilon)$ be some bounded and continuous function on
$[0,1]$  and  the integral
\[ 
\int_0^1\, |J(\epsilon)|\cdot \frac{d\epsilon}{\epsilon}<\infty\tag{1}
\]
We also assume that $J(\epsilon)=0$ when $\epsilon\geq 1$.
When $\mathfrak{Re}(\lambda)>0$ we set
\[
 M(\lambda)=\lambda\cdot \int_0^1\, J(\epsilon)\cdot \epsilon^{\lambda-1}\cdot d\epsilon\tag{2}
\]

\noindent
It is clear that $M(\lambda)$ is an analytic function in the right
half-plane $\mathfrak{Re}(\lambda)>0$ which by (1) extends to a continuous function
on the closed half\vvv plane. Moreover
(*) implies  that
$ M(\lambda)$ extends to a meromorphic function in the whole complex
$\lambda$-plane whose  poles are contained in the set
$\{-q_k\}$. In fact, this follows easily via (*) since

\[
\lambda\int_0^1\, \epsilon^q\cdot  \epsilon^{\lambda-1}\epsilon=
\frac{1}{\lambda+q}\quad\text{for every}\quad q>0
\] \medskip

\noindent
Consider  $M$ along the imaginary axis where we have
\[
M(is)=is\int_0^1\, J(\epsilon)\cdot \epsilon^{is}\cdot 
\frac{d\epsilon}{\epsilon}\tag{3}
\]
Apart from the factor $i$ this is the Fourier transform of 
$J$ on  the multiplicative line ${\bf{R}}^+$.
So by XXX one has  the inversion formula
\[ 
J(\epsilon)=\lim_{R\to\infty}\, J_R(\epsilon)=
\,\int_R^R\epsilon^{-is}\cdot
\frac{ M(is)}{is}\cdot ds\tag{4}
\]
\medskip




\noindent
Mellin
found a reverse procedure where one from the star only assumes that
(1) holds and to achieve an asymptotic expansion (*)
one imposes conditions upon the $M$\vvv function.
More precisely, \emph{assume}  that
$M(\lambda)$ extends to a meromorphic function with simple vpoles confined to a set
$\{\vvv q\uuu k\}$ of strictly negative real numbers.
With $\lambda=t+is$ we consider  line integrals over rectangles
\[
\square_{R,A}=\{-A<t<0\}\cap\, \{-R<s<R\}\tag{5}
\]
where one for  an  arbitrary positive integer $N$  choose
$A$ so that 
\[
q_N<A<q_{N+1} \tag{6}
\]
With $\epsilon>0$
kept fixed we have the analytic function in
$\square_{R,A}$  defined by
\[ 
\lambda\mapsto \epsilon^{-\lambda}\cdot \frac{M(\lambda)}{\lambda}\tag{6}
\]



\newpage

\noindent
{\bf{1. Exercise.}}
Show that Cauchy's residue formula gives the equality:
\[
2\pi i\cdot J_R(\epsilon)=
2\pi i\cdot \sum_{k=1}^{k=N}\,q_k^{-1}\cdot 
\mathfrak{res}(M(\lambda):q_k)\cdot \epsilon^{q_k}
+I_1(R,A)+I_2(R,A)\tag{i}
\]
where 
\[
I_1(R)=\int_0^{-A}\, 
\bigl[\epsilon^{-t-iR}\cdot \frac{M(t+iR)}{t+iR}-
[\epsilon^{-t+iR}\cdot \frac{M(t-iR}{t-iR}\bigr]\cdot dt\tag{ii}
\]
\[
I_2(R)=-\int_{-R}^R\,  \epsilon^{A-is}\cdot 
\frac{M(-A+is)}{-A+is}\cdot ds\tag{iii}
\]


\noindent
{\bf{2. The passage to the limit.}}
The triangle inequality gives

\[
|I_2(R)|\leq  \epsilon^A\cdot 
\int_{-R}^R\,  \bigl|\frac{M(-A+is)}{-A+iRs}\bigr|\cdot ds
\leq\epsilon^A\cdot\frac{1}{R}\cdot
\int_{-R}^R\,  \bigl|M(-A+is)\bigr|\cdot ds\tag{3.1}
\]



\medskip

\noindent
Above we have  $A>q_N$ and hence $\epsilon^A$
gives an admissable error for an asymptotic expansion
up to order $N$. So from (i) in the exercise we conclude the following:

\medskip

\noindent
{\bf{3. Theorem.}}
\emph{Assume that we for every
positive integer $N$ can find $q_N<A<q_{N+1}$
such that the following two limit formulas hold:}
\[
\lim_{R\to\infty}\,
\frac{1}{R}\cdot
\int_{-R}^R\,  \bigl|M(-A+is)\bigr|\cdot ds=0\tag{i}
\]
\[
\lim_{R\to\infty}\,\int_0^{-A}\, 
\bigl[\epsilon^{-t-iR}\cdot \frac{M(t+iR)}{t+iR}-
[\epsilon^{-t+iR}\cdot \frac{M(t-iR}{t-iR}\bigr]\cdot dt=0\tag{ii}
\]

\noindent
\emph{Then the function $J(\epsilon)$ has an asymptotic expansion  (*) where
the $c$-numbers are given by}
\[ 
c_k=\mathfrak{res}(M:q_k)\quad\colon\quad k=1,2,\ldots
\]






\bigskip

\noindent
{\bf{4. The case of
multiple roots.}}
Keeping the  conditions (i-ii) in Theorem 3
we can relax the hypothesis that the poles of
$M(\lambda)$ are simple and obtain
an asymptotic expansion where 
the terms $\{c_k\epsilon^{q_k}\}$ in the expansion (*)
are replaced by finite sums of the form
\[ 
\sum_{\nu=0}^{m_k-1}\, c_{k,\nu}\cdot (\log \epsilon)^\nu\cdot \epsilon^{q_k}\tag{1}
\]
where $m_k$ is the multiplicity of the pole of $M(\lambda)$ at $q_k$.
More precisely,
Cauchy's residue formula holds in (i) when the terms
$2\pi i\cdot q_k^{-1}\mathfrak{res}(M(\lambda):q_k)$
are replaced by
\[
2\pi i\cdot \mathfrak{res}(\epsilon^{-\lambda}M(\lambda):q_k)\tag{2}
\]
For a given $k$ we set $\lambda=-q_k+\zeta$
and then the residue above can be calculated
using the expansion
\[
\epsilon ^{q_k+\zeta}=\epsilon^{q_k}\cdot
\bigl [1+\sum_{\nu=1}^\infty
\,(\log \epsilon)^\nu\cdot\zeta^\nu\bigr]
\]


\noindent
For example, assume that $M(\lambda)$ has a double pole at $-q_k$
with a local Laurent expansion
\[
M(-q_k+\zeta)= \frac{c_k}{\zeta^2}+a_0+a_1\zeta+\ldots
\]
In this case the residue in (2) becomes
\[
\mathfrak{res}(\epsilon^{-\lambda}M(\lambda):q_k)=
c_k\cdot \epsilon ^{q_k}\cdot \log\epsilon\tag{3}
\]
\medskip


\noindent
{\bf{Final remark.}}
To appreciate Mellin's result one should consider various examples where
the point is that other kind of methods to begin with prove that
the $M$\vvv function has a "nice" meromorphic extension as above.
Quite extensive classes of situations where this applies are derived via
$\mathcal d$\vvv module methods when
the $M$\vvv function
satisfies certain functional equations.
See the chapter on $\mathcal D$\vvv module theory for 
further comments and examples.
The reader may also consult the elegant article [Barlet\vvv Maire]
where Mellin's result is extended to give complex
expansions, i.e here a $J$\vvv function is defined in a punctured complex disc
and one seeks asymptotic expansions
when a complex variable $\zeta$ tends to zero instead of 
taking limits as in (*) over positive real $\epsilon$.










 




%\end{document}











 




 




