\documentclass{amsart}


\usepackage[applemac]{inputenc}

\addtolength{\hoffset}{-12mm}
\addtolength{\textwidth}{22mm}
\addtolength{\voffset}{-10mm}
\addtolength{\textheight}{20mm}

\def\uuu{_}


\def\vvv{-}

\begin{document}

\[ 
\tag{*}
\]


\newpage


\centerline{\bf\large{I. The disc algebra $A(D)$}}

\bigskip

\centerline{\emph{Contents}}
\bigskip

\noindent
\emph{0. Introduction.}

\bigskip

\noindent
\emph{1. Theorem of Brothers Riesz.}


\bigskip

\noindent
\emph{2. Ideals in the disc algebra}
\bigskip

\noindent
\emph{3. A maximality theorem for uniform algebras.}
\bigskip

\centerline{\bf{Introduction.}}

\bigskip

\noindent
Denote by $A(D)$ the algebra of continuous functions on the closed
unit disc $\bar D$ which are analytic in the open disc.
One refers to $A(D)$ as the \emph{disc-algebra}.
Since  polynomials in $z$ is a dense subalgebra of $A(D)$
it follows
that a Riesz measure $\mu$ on $T$
is $\perp$ to $A(D)$ if and only if
\[
\int_0^{2\pi}\,e^{ i n\theta}\cdot d\mu(\theta)=0\quad\colon\quad \, n=0,1,2,\ldots\tag{*}
\]
In � 1 we  show that (*) implies that $\mu$ is
absolutely continuous and deduce some facts about boundary
values of analytic functions in the open disc.
The maximum principle for analytic functions
shows that $A(D)$ can be identifies with  a closed subalgebra of
$C^0(T)$.
A result due to J. Wermer asserts that this yields a maximal closed
subalgebra. If means that if
$f\in C^0(T)\setminus A(D) $, then polynomials in
$f$ and
$e^{i\theta}$ is a dense subalgebra of
$C^0(T)$.
In � 3 we prove an extension of this result which goes as follows:
Let $K$ be a closed subset of $D$ whose planar Lebegue measure is zero
and there exists an open interval $\omega$ in $T$
such that
$K\cap \omega=\emptyset$.
Then, if $f\in C^0(K\cup T)$ is such that its restriction to
$T$ does not belong to $A(D)$, it follows that polynomials in
$z$ and $f$ is a dense subalgebra of $C^0(K)$.
\medskip

\noindent
In � 4 we study subsets on  $T$ given
zeros of functions in $A(D)$.
With no extra regularity than continuity, each
closed null set in $T$ can be realised as the zero set of
some $f$ in $A(D)$. This result was established by F. and M-Ruesz and
is proved in � 1.
If one imposes extra regularity the eventual zero sets
are more restricted.
If $\alpha>0$ we denote by
$A^\alpha (D)$ the functions in the disc-algebra which are
H�lder continuous of order $\alpha$ at least on $t$.
A result due to Beurling asserts that zero sets of functions in
$A^\alpha (D)$ must be rather thin.
More prcisely, denote by $\mathcal N*$ the family of closed subsets
$E$
on the unit circle for which
the integral
\[
\int_0^1\, \frac{\phi_E(t)}{t}\,dt=+\infty\tag{*}
\]
where  $\phi_E(t)$ denotes  the Lebesgue measure of the set of points
$e^{i\theta}$ whose distance to $E$ is $\leq t$.
With this notation the following uniqueness result holds:
\medskip

\noindent
{\bf{Theorem.}}\emph{Each set $E\in  \mathcal N^*$ is a set of uniquness for
the set of H�der continuous functions in
$A(D)$, i.e. if $\alpha>0$ and  $f\in A^\alpha (D)$ is zero on
$E$, then $f$ is identically zero.}
\medskip

\noindent
{\bf{The class $\mathcal N_*$.}}
It consists of closed null sets $E$ for which
\[
\int_0^1\, \frac{\phi_E(t)}{t}\,dt<\infty\tag{**}
\]
In � 4 we give a construction due to Carleson which shows that
if $	E\in \mathcal N_*$ and $m$ is a positive integer, then
there exists a function $f\in A^m(D)$  whose set of zeros contains
$E$ and while $f$ is not  identically zero.
A more delicate analysis occurs in � 5 where we
expose some further results from
Carleson's article 
\emph{Sets of uniqueness for functions regular in a disc}.
Here one studies 
analytic functions in ther open unit disc
with a finite Dirichlet integral, i.e.
here
\[
\iint_{|z|<1}\, |f'(z)|^2\, dxdy<\infty
\]
We refer to the introduction in � 5 for
the results which will be proved in connection with this
family of analytic functions.
Finally � 6 is devoted to
the Wiener algebra $W(T)$ which consists of functions in
$A(D)$ whose Taylor series is absolutely convergent.
Here further results from [Carelson] are exposed which
deal with sets of uniqueness for $W(T)$.




















\newpage



\centerline{\bf 1. Theorem of the Brothers Riesz}
\bigskip


\noindent
At the 4:th Scandinavian Congres held in Stockholm 1916,
Friedrich and Marcel Riesz proved the following:


\medskip

\noindent
{\bf 1.1 Theorem}
\emph{Let $E\subset T$ be a closed null set.  Then there exists
$\phi\in A(D)$ such that
$\phi(e^{i\theta})=1$ when $e^{i\theta}\in E$ while $|\phi(z)|<1$ for 
every $z\in\bar D\setminus E$.}


\bigskip


\noindent
Before the construction of  such  functions
we
draw  a consequence.


\medskip

\noindent {\bf 1.2. Theorem}
\emph{Let $\mu$ be a   Riesz-measure on $T$  
such that}
\[ 
\int_0^{2\pi}\,e^{ i n\theta}\cdot d\mu(\theta)=\quad\colon\, n=1,2,\ldots
\]
\emph{Then $\mu$ is absolutely continuous.}
\medskip





\noindent
\emph{Proof.} 
Assume the contrary. Then there exists
a closed null set $E$ in $T$ 
such that
\[ 
\int_E\, d\mu(\theta)\neq 0\tag{i}
\]
Theorem 1.1 gives   $\phi\in A(D)$ which is a peak function for 
$E$.
The hypothesis in Theorem 1.2 gives
\[
\int_0^{2\pi}\,
\phi^m(e^{i\theta})\cdot d\mu(\theta)=0\quad\colon m=1,2,\ldots\tag{ii}
\]
Next,   since $\phi$ is  a peak function for $E$ we have
\[
\lim_{m\to\infty}\, \phi^m(e^{i\theta})\to \chi_E
\] 
where $\chi_E$ is the characteristic function of $E$. Hence 
the dominated convergence theorem in general measure theory
applied to $L^1(\mu)$
gives $\int_E\,d\mu=0$. But this was not the case by (i) above
and  this
contradiction proves Theorem 1.2.
\bigskip

\centerline
{\emph{ Proof of Theorem 1.1}}

\bigskip

\noindent
Let $E\subset T$ be a closed null set and $\{(\alpha_\nu,\beta_\nu)\}$
is the family of open intervals in $T\setminus E$.
Since $b_\nu-a_\nu\to 0$ as $\nu$ increases, we can choose 
a sufficiently spare sequence of
positive numbers
$\{p_\nu\}$ such that
\[ \sum\, p_\nu(\beta_\nu-\alpha_\nu)<\infty\quad\text{and}\quad
\lim_{\nu\to\infty} p_\nu=+\infty
\]


\noindent
To each $\nu$ we define a function $g_\nu(\theta)$ on the open interval
$(\alpha_\nu,\beta_\nu)$ by
\[ 
g_\nu(\theta)=
\frac {p_\nu(\beta_\nu-\alpha_\nu)}
{\sqrt{\ell_\nu^2-(\theta-\gamma_\nu)^2}}
\colon\quad\colon
\ell_\nu=\frac{\beta_\nu-\alpha_\nu}{2}\quad\colon
\gamma_\nu=\frac{\beta_\nu+\alpha_\nu}{2}\tag{1}
\]


\noindent
For each $\nu$ a variable substitution gives:
\[
\int_{\alpha_\nu}^{\beta_\nu}
\frac{d\theta}{\sqrt{\ell_\nu^2-(\theta-\gamma_\nu)^2}}=
\int_0^1
\frac{ds}{\sqrt{\frac{1}{4}-(s-\frac{1}{2})^2}}=C\tag{2}
\]
where  $C$�is a positive constant which the reader may compute.
Next, (2) and the convergence of
$\sum\,p_\nu(\beta_\nu-\alpha_\nu)$ imply
the function 
\[
F(\theta)=\sum\, g_\nu(\theta)\tag{3}
\]
has a finite $L^1$-norm.
Here $F$ is defined outside the null set $E$ and since
each single $g_\nu$-function  restrics to
a real analytic function on $(\alpha_\nu,\beta_\nu)$ the same holds for
$F$. 
Next, we notice that
\[
\theta\mapsto
\frac{(\beta_\nu-\alpha_\nu)}{\sqrt{\ell_\nu^2-(\theta-\gamma_\nu)^2}}
\geq 2\quad\text{for all}\quad \alpha_\nu<\theta<\beta_\nu\tag{4}
\]
In addition to this the reader can verify that

\[
\frac{(\beta_\nu-\alpha_\nu)}{\sqrt{\ell_\nu^2-(\alpha+s-\gamma_\nu)^2}}
\geq \frac{\beta_\nu-\alpha_\nu}{\sqrt{s\cdot (\beta_\nu-\alpha_\nu-s)}}
\quad \colon\quad 0<s<\beta_\nu- \alpha_\nu\tag{5}
\]

\medskip

\noindent 
From (4-5) we can show that $F(\theta)$ 
gets large when we approach $E$.
Namely, let $N$ be an arbitrary positive integer.
Then we find $\nu_*$ such that
\[
\nu>\nu_*\implies p_\nu>N\tag{i}
\]
Next, let $\delta>0$ and 
consider the open set 
$E_\delta$ of points with distance $<\delta$ to $E$.
If $\theta\in E_\delta$ we have $\alpha_\nu<\theta<\beta_\nu$ for some
$\nu$.
If $\nu>\nu*$ then (i) and (4) give
\[ 
F(\theta)> 2N\tag{ii}
\]
Next, set
\[
\gamma=\min_{1\leq\nu\leq \nu_*}\, \rho_\nu\cdot
\sqrt{\beta_\nu-\alpha_\nu}
\tag{iii}
\]
Let us now consider some  $1\leq\nu\leq\nu_*$
and a point $\theta\in E_\delta$.
which  belongs to $(\alpha_\nu,\beta_\nu)$.
Since $E\cap(\alpha_\nu,\beta_\nu=\emptyset$ we see that

\[
\theta-\alpha_\nu<\delta\quad\text{or}\quad
\beta_\nu-\theta<\delta\tag{iv}
\] 
must hold.
In both cases  (4) gives:

\[ g_\nu(\theta)\geq \frac{\rho_\nu\cdot\sqrt{(\beta-\nu-\alpha-\nu}}{\sqrt{\delta}}
\geq \frac{\gamma}{\sqrt{\delta}}\tag{v}
\]
With $\gamma$ fixed we find a small $\delta$ such that
the right hand side is $>N$ and together with (ii) it follows that
\[
\theta\in E_\delta\setminus E\implies
F(\theta)>N\tag{vi}
\]

\noindent
\medskip
{\bf The construction of $\phi$}.
The Poisson kernel gives the harmonic function: 
\[ 
U(re^{i\theta})=\frac{1}{2\pi}\int_0^{2\pi}\,
\frac{1-r^2}{ 1+r^2+\text{cos}(\theta-t)}\cdot F(t) dt
\quad\colon\quad re^{i\theta}\in D
\]
Since $F\geq 0$ we have $U$ it is $\geq 0$ in $D$ and by (vi)
$U(z)$ increases to $+\infty$ as
$z$ approaces $E$. More precisely, the following companion to (vi)
holds:
\medskip

\noindent \emph{Sublemma }
\emph{To every positive integer $N$ there exists $\delta>0$ such that}
\[ 
U(z)>N\quad\colon\quad z\in D\cap E^*_\delta
\]
\emph{where  $E^*_\delta=\{z\in D\,\colon\,\,
\text{dist}(z,E)<\delta\}$.}
\medskip

\noindent
Now we construct the harmonic conjugate:

\[ 
V(re^{i\theta})=\frac{1}{\pi}\int_0^{2\pi}\,
\frac{r\cdot \text{sin}(\theta-t)}{ 1+r^2+\text{cos}(\theta-t)}\cdot F(t) dt
\quad\colon\quad re^{i\theta}\in D
\]
\medskip

\noindent
We have no control for the
limit behaviour of $V(re^{i\theta})$ as $r\to 1$
and $e^{i\theta}\in E$.
But on
the open intervals $(\alpha_\nu,\beta_\nu)$ where
$F$  restricts to a real analytic function there exists 
a limit function $V^*$:

\[
\lim_{r\to 1}\,V(re^{i\theta})=V^*(e^{i\theta)})
\quad\colon\quad \alpha_\nu<\theta<\beta_\nu
\]
Thus, $V^*$ is a function defined on $T\setminus E$.
Similarly, $U(re^{i\theta})$ has a limit function $U^*(e^{i\theta})$
defined on $T\setminus E$.
Now we set
\bigskip
\[ 
\phi(z)= \frac{U(z)+iV(z)}{U(z)+1 +iV(z)}\quad\colon\quad z\in D\tag{*}
\]
This is an analytic function in $D$. Outside $E$ we get the boundary value function
\[
\lim_{r\to 1}\,\phi(re^{i\theta})=
\frac{U^*(e^{i\theta})+iV^*(e^{i\theta})}{U^*(e^{i\theta})+1 +iV^*(e^{i\theta})}
\]
\medskip

\noindent 
\emph{The limit on $E$}.
Concerning the limit as $z\to E$ we have:

\[ 
[1-\phi(z)|=
\frac{1}{|1+U(z)+iV(z)|}\leq \frac{1}{1+U(z)}
\]
By the Sublemma the last term tends to zero as $z\to E$.
We conclude that $\phi\in A(D)$ and here $\phi=1$ on $E$ while 
$|\phi(z)|<1$
for al $z\in \bar D\setminus E$ which gives the requested peak function.

\bigskip

\noindent {\bf Remark.}
Above we have followed the orighinal proof by F. and M. Riesz.
It has the merit that it is  
quite 
constructive.
For alternative    proofs using
functional analysis and the Hilbert space $L^2(d\mu)$ attached to a
Riesz measure on $T$ we refer to 
the text-book [Koosis: p. 40-47].
\bigskip

\centerline {\bf 1.3 An application of Theorem 1.1}


\medskip

\noindent
Let $f(z)$ be analytic in the open unit disc and assume there exists
a constant $M$ such that
\[
\int_0^{2\pi}\, |f(re^{i\theta})|\cdot d\theta\leq M\quad\colon\quad
0<r<1
\]
Consider the family of measures on the unit circle
defined by 
\[
\{\mu_r=f(re^{i\theta})\cdot d\theta\,\,\colon\,\, r<1\}
\]
The uniform upper bound for their
total variation implies by compactness in the weak topology
that
there exists a sequence
$\{r_\nu\}$ with $r_\nu\to 1$
and a Riesz measure $\mu$
such that
$\mu_{r_\nu}\to\mu$ holds \emph{weakly}.
In particular we have

\[ 
\int_0^{2\pi}\, e^{in\theta}\cdot d\mu(\theta)=
\lim_{r_\nu\to 1}\int_0^{2\pi}\,e^{in\theta} f(r_\nu e^{i\theta})\cdot d\theta
\]


\noindent 
for every integer $n$.
Since $f$ is analytic the right hand side integrals  vanish whenever $n\geq 1$
and 
hence
$\mu$ is absolutely continuous by
Theorem 1.2.
So we have  $\mu=f^*(\theta)d\theta$ for an $L^1$-function $f^*$. 
Now we construct
the analytic function
\[ 
F(z)=
\frac{1}{2\pi}\int_0^{2\pi}\,
\frac{f^*(\theta)\cdot e^{i\theta}d\theta} 
{e^{i\theta}-z}
\]


\noindent
When $z\in D$ is fixed the \emph{weak} convergence applies to the
$\theta$-continuous function
$\theta\mapsto
\frac{e^{i\theta}} 
{e^{i\theta}-z}$ and hence
\[ 
F(z)=\lim_{\nu\to\infty}\,
\frac{1}{2\pi}\int_0^{2\pi}\,
\frac{ f(r_\nu e^{i\theta})e^{i\theta}d\theta}
{e^{i\theta}-z}
\]
At the same time, as soon as $|z|<r_\nu$
one has  Cauchy's formula:

\[
f(z)=
\frac{1}{2\pi}\int_0^{2\pi}\,
\frac{ f(r_\nu e^{i\theta})\cdot r_\nu e^{i\theta}\cdot d\theta}
{r_\nu\cdot e^{i\theta}-z}
\]
Since this hold for every large $\nu$  we can pass to
the limit and conclude that
$F(z)=f(z)$ olds in $D$. Hence $f(z)$ is represented by the Cauchy kernel of
the $L^1(T)$-function $f^*(\theta)$.
At this stage we apply \emph{Fatou's theorem}  to conclude that

\[
\lim_{r\to 1}\,
f(re^{i\theta})=f^*(\theta)\quad\text{holds  almost everywhere}
\]
Moreover, one has convergence in the $L^1$-norms:
\[
\lim_{r\to 1}\,
\int_0^{2\pi}\,
|f(re^{i\theta}-f^*(\theta)|=0
\]
\medskip

\noindent
Thus, thanks to Theorem 1.2 
the $L^1(T)$- sequence 
defined by the functions $\theta\mapsto f(re^{i\theta})$
converges  almost everywhere   to
a unique limit  function $f^*(\theta)\in L^1(T)$.
\medskip

\noindent
{\bf{1.4 Exercise.}}
Show that for every Lebesgue point $\theta_0$ of 
$f^*(\theta)$ there exists a radial  limit: 


\[ 
\lim_{r\to 1}\, f(re^{i\theta_0})= f^*(\theta_0)
\]


\bigskip

\noindent
{\bf{1.5 Exercise.}}
In general, let $K$ be a compact subset of $D$
and $\mu$ a Riesz measure supported by $K$ which is $\perp$ to analytic polynomials,  i.e.
\[ 
\int\, z^n\cdot d\mu(z)=0
\] 
hold for all $n\geq 0$.
Use the existence of peaking functions in $A(D)$ to conclude that if
$E\subset T$ is a null\vvv set for
linear Lebesgue measure $d\theta$, then
$E$ is a null\vvv set for $\mu$. In particular, if
$K$ contains a relatively open set given by an arc $\alpha$
on the unit circle, then the restriction of $\mu$ to $\alpha$
is absolutely continuous






\centerline{\bf{1.6  Principal ideals in the disc algebra.}}

\bigskip


\noindent
Let�$A(D)$ be the disc algebra.
The point $z=1$  gives
a maximal ideal in $A(D)$:
\[ 
\mathfrak{m}=\{f\in A(D)\quad\colon f(1)=0\}
\]
Let $f\in A(D)$  be such that $f(z)\neq 0$ for all $z$
in the closed disc except at the point $z=1$. 
The question arises if the principal ideal generated by $f$
is dense in 
$\mathfrak{m}$. This is not always true. A counterexample is given by
the function
\[
f(z)= e^{\frac{z+1}{z\vvv 1}}
\]



\noindent
Following  the appendix in [Carleman: Note 3] we shall
give  a sufficient condition on $f$ in order that its principal ideal is dense in
$\mathfrak{m}$.
To begin with there exists
the analytic function in ther open disc defined by
\[
f^*(z)=\text{exp}\bigl\{\, 
\frac{1}{2\pi}\int_0^{2\pi}\,
\frac{e^{i\theta}+z}{e^{i\theta}-z}\cdot \text{log}\, 
\bigl |\frac{1}{f(e^{i\theta})}\bigr |\cdot d\theta\bigr\}
\]
\medskip

\noindent
Since the continuous boundary function of
$f$ is $\neq 0$ except at $z=1$, it follows by a wellknoen limit 
formula  that $f^*$ extends to $D\setminus\{1\}$
a where it is equal to $f$.
We say that 
$f$ has no logarithmic residue a $z=1$ if $f=f^*$
holds everywhere on $D$.


\medskip

 \noindent
 {\bf{1.6.1  Theorem.}} \emph{If $f$ has no logarithmic 
 residue 
 then $A(D)f$ is dense in $\mathfrak{m}$.}
\medskip



\noindent
\emph {Proof}.
With  $\delta>0$ we choose a continuous
function $\rho_\delta(\theta)$ on $T$
which is equal to
$\text{log}\, |\frac{1}{f(e^{i\theta})|}$ outside the interval $(-\delta,\delta)$
while
\[ 
0<\rho_\delta(\theta)<\text{log}\, |\frac{1}{f(e^{i\theta})}\bigr|
\quad\colon
-\delta<\theta<\delta\tag{i}
\]


\noindent 
Next, let  $\phi\in\mathfrak{m}$ and set
\[
\omega_\delta(z)= \phi(z)\cdot 
\text{exp}\,\bigl\{\vvv\frac{1}{2\pi}\int_0^{2\pi}
\frac{e^{i\theta}+z}{e^{i\theta}-z}\cdot 
\rho_\delta(\theta)\cdot d\theta\bigr\}\tag{ii}
\]
It follows that
\[
\bigl|\omega_\delta(z)\cdot f(z)-\phi(z)\bigr|=|f(z)|\cdot |\phi(z)|\cdot
\bigl|\, 1- \text{exp}\,\bigl\{
\frac{1}{2\pi}\int_0^{2\pi}\,
\frac{e^{i\theta}+z}{e^{i\theta}-z}\cdot 
\bigl[\,\text{log}\, 
\frac{1}{|f(e^{i\theta})|}-
\rho_\delta(\theta)\,\bigr] \cdot d\theta\,\bigr\}\, |\tag{iii}
\]
\medskip

\noindent
{\bf{Exercise.}}
Show that the limit of the right hand side is zero when
$\delta\to 0$ and conclude that
$\phi$ belongs to the closure of the principal
ideal generated by $f$.

\newpage

\centerline{\bf{� 2. Wermer's maximality theorem.}}
\medskip

\noindent
The disc algebra $A(D)$ is a 
uniform algebra,  
where the spectral radius norm is equal to the 
maximum over the closed disc.
By the maximum principle for analytic functions in
$D$ one has $|f|_D=|f|_T$.
One therefore calls $T$ the \emph{Shilov boundary} of $A(D)$.
A notable point is that $A(D)$ is a Dirichlet algebra which means
that the linear space of real parts of functions
restricted to $T$ is a dense subspace of all real-valued and
continuous functions on $T$. In fact, using the Herglotz integral 
A $\rho(\theta)$ is real-valued function $\rho$
in 
$C^0(T)$  is equal to 
$\mathfrak{Re}(f)$ on $T$ for some
$f\in A(D)$ if and only if
the function
\[ 
z\mapsto 
\int_0^{2\pi i}\, \frac{\mathfrak{Im}(ze^{-i\theta})}
{|e^{i\theta}-z|^2}\cdot \rho(\theta)\, d\theta
\] 
extends to a continuous function on the closed disc.
For example, every $C^1$-function on $T$ belongs to
$\mathfrak{Re}(A(D))$.
\medskip

\noindent
Now we prove   Wermer's result which asserts that
$A(D)$ is a maximal uniform algebra. It means that
if 
$f\in C^0(T)$ is such that
the 
closed subalgebra of $C^0(T)$ generated by $f$ and $z$ is
not equal to 
$C^0(T)$, then $f$ must belong to $A(D)$.
\medskip

\noindent
\emph{Proof.}
Consider  the closed  algebra
$B=[z,f]_T$ of $C^0[T]$. This is a commutative
Banach algebra which gives the
maximal ideal space 
$\mathfrak{M}_B$ whose points correspond to
multiplicative functionals on $B$. If
$p\in \mathfrak{M}_B$ and $p^*$ is the corresponding
multiplicative functional it is clear that there exists
a unique point $z(p)\in D$
such that 
\[
p^*(g)=g(z(p))
\]
hold for every $g$ in the subalgebra $A(D)$ of $B$.
If
$z(p)\in T$ holds for every $p$ in the maximal ideal space
then the $B$-element $z$
is invertible. This means that $B$ regarded as a subalgebra of
$C^0(T)$ contains both
$e^{i\theta}$ and $e^{-i\theta}$ which  by the Weierstrass approximation theorem 
generate a dense subalgebra of $C^0(T)$.
Hence, since  $B\neq C^0(T)$ is assumed there
must exist at least one point
$p\in \mathfrak{M}_B$ such that $z(p)$ belongs to
the open unit disc.
In fact, \emph{every point} $z_0\in D$ is of the form $z(p)$ 
for some $p$ for otherwise
$\frac{1}{z-z_0}$
belongs to $B$ and one verifies easily that
the two functions on $T$ given by $e^{i\theta}$ and
$\frac{1}{e^{i\theta}-z_0}$ also generate a dense subalgebra of
$C^0(T)$. 
Hence 
$p\mapsto z(p)$ sends $\mathfrak{M}_B$
onto the closed disc.
\medskip

\noindent
At this stage one employs a general result from
uniform algebras. Namely, since every multiplicative functional has norm one
it follows that that for every $p\in\mathfrak{M}_B$
there exists a probability measure
$\mu_p$ on the unit circle such that
\[ 
p^*(g)=
\int_T\, g(e^{i\theta})\cdot d\mu_p(\theta)\quad\text{hold for all}\quad
g\in B\tag{*}
\]
Now we use that $A(D)$ is a Dirichlet algebra. Namely,  (*) holds
in particular  for $A(D)$-functions and since $\mu_p$
is a real measure we conclude that it must be equal to
the Poisson kernel of the point $z(p)$.
This proves to begin with that the map $p\to z(p)$ is
\emph{bijective}, and each function
$g\in B$ gives  a continuous function on the closed unit disc
defined by
\[
g^*(z(p)=p^*(g)
\] 
Here   (*) shows   that $g^*$
is the harmonic extension to the open unit disc
of the boundary
function $g$ on $T$.
since $B$ is an algebra of functiond, this easily 
entails  that each such har,onic extension actually is an
analytic function.
This means
precisely that
$B=A(D)$. In particul ar the $B$-element $f$ belongs to
$A(D)$
which is 
the assertion in Wermer's  theorem.

\bigskip


\centerline{\bf 3. Relatively maximal algebras}
\bigskip

\noindent
{\bf{Introduction.}}
An extension of Wermer's maximality theorem
was proved in [Bj�rk] and goes as follows.
Let $K$
be a closed subset of $\bar D$
whose  planar Lebesgue measure  is zero and there exists
an open interval $\omega$ in $T$
such that $K\cap\omega=\emptyset$.


\bigskip


\noindent
{\bf {3.1. Theorem.}}
\emph{Let $f\in C^0(K\cup T)$
be such that the uniform algebra $ B$ generated by $z$ and $f$ on
$K\cup T$
is a proper subalgebra of 
$C^0(K\cup T)$.
Then  the restriction of $f$ to
$T$ belongs to
$A(D)$.}
\bigskip

\noindent {\bf Remark.}
The case when
$K$ is the union of $T$ and a finite set of Jordan arcs
where each arc has one end-point on $T$ and the other
in the open disc $D$ is of special interest. If these Jordan arcs
are not too fat, then $f$ extends analytically across each arc
which means that the restriction of $f$ to $T$ must belong to the
disc-algebra.
This case was a motivation for
Theorem 3.1 since it is connected to
the problem of finding conditions on a Jordan arc
$J$ in order that it is locally a removable singularity for
continuous functions $g$ which are analytic in
open neighborhoods of $J$.
The interested reader may consult [Bj�rk:x] for a further discussion about
this problem where  comments are given by
Harold Shapiro about the connection to
between
Theorem 3.1 and 
results by Privalov concerning 
analytic extensions across a Jordan arc.
\bigskip

\noindent
{\emph{ Proof of Theorem 3.1}}.
Let $\pi$ be the projection from the maximal ideal space
$\mathfrak{M}_B$ into $D$ which means that when
$z$ is regarded as an element in $B$ then its Gelfand transform
$\widehat z$ satisfies
\[ 
\widehat z(p)=\pi(p)\quad\colon\, p\in \mathfrak{M}_B
\]
As usual $K$ is identified with a compact subset
of
$\mathfrak{M}\uuu B$
and contains the Shilov boundary.
If $e^{i\theta}\in T$
we use that it
is a peak point for
$A(D)$ and hence also for $B$. This entails
that
the fiber
$\pi^{\vvv 1}(e^{i\theta})$ is reduced to the corresponding point
$e^{i\theta}\in K$.
Next, since we assume that $K$ has planar measure zero we know from
XX that the uniform algebra on $K$ generated
by rational functions with
poles outside $K$ is equal to $C^0(K)$.
Since $z\in B$ and $B\neq C^0(K)$ is assumed, it follows that
$\pi^{\vvv 1}(D\setminus K)\neq \emptyset$.
Now we shall prove:
\medskip

\noindent
{\bf{Sublemma}}.
\emph{The fiber above every point in
$D\setminus K$ is reduced to a single point.}
\medskip

\noindent
To prove this we choose  
a non-zero Riesz measure
$\mu$ on $K$ which annihilates $B$. 
We get  
two analytic functions in the
open set $D\setminus K$:
\[
W(z)=\int_K\, \frac{f(\zeta)\cdot d\mu(\zeta}{\zeta-z}\quad\text{and}\,\,\,
R(z)=\int_K\, \frac{d\mu(\zeta)}{\zeta-z}\tag{*}
\]

\noindent
The crucial step in the proof of the sublemma is to
show that if   $z\in D\setminus K$
and $\xi\in \pi^{\vvv 1}(z)$, then 
the Gelfand transform 
$\widehat f$ satisfies:
\[
\widehat f(\xi)\cdot R(z)= W(z)\quad\colon\,\forall\,\, \xi\in \pi^{-1}(z)
\tag{**}
\]

\medskip


\noindent
To prove (**) we proceed as follows.
First, since
$\mu$  annihilates  the functions 
$z^N$ and  $z^N\cdot f(z)$ for every $N\geq 0$
we have
\medskip
\[
\int_K\, \frac{\bar z\cdot d\mu(\zeta)}{1-\bar z\cdot\zeta}=
\int_K\, \frac{\bar z\cdot f(\zeta)\cdot d\mu(\zeta)}{1-\bar z\cdot\zeta}=
0\quad\text{for every  }\,\,\, 
 z\in D
\]
Adding these
 zero-functions in (*)
 it follows that
 \medskip
\[
W(z)=
\int_K\, \frac{(1-|z|^2|\cdot f(\zeta)\cdot d\mu(\zeta)}{
(\zeta-z)(1-\bar z\zeta)}\quad\text{and}\,\,\,
R(z)=\int_K\, \frac{(1-|z|^2\cdot  d\mu(\zeta)}{
(\zeta-z)(1-\bar z\zeta)}\tag {1}
\]
\medskip

\noindent
The  assumption that the closure of $K\setminus T$
does not contain $T$ gives
some open arc
$\alpha=(\theta_0,\theta_1)$
on $T$ which is disjoint from the closure of $K\setminus T$, and
the local version of the Brother's Riesz  theorem from Exercise 1.5 implies that
the restriction of $\mu$ to $\alpha$ is absolutely continuous.
Hence, by  Fatou's theorem
there exist the two limits
\[ 
\lim_{r\to 1}
W(re^{i\phi})=W(e^{i\phi})\quad\colon
\lim_{r\to 1}
R(re^{i\phi})=R(e^{i\phi})
\tag{2}
\]
almost every on
$\theta_0<\theta<\theta_1$.
Let us fix  a pair $(\theta^*_0,\theta_1^*)$ where
$\theta_0<\theta^*_0<\theta^*_1<\theta_1$ and
the radial limits in (2) exist for $\theta^*_0$ and
$\theta^*_1$.
\medskip

\noindent
Next, consider a point  $z_0\in D\setminus K$ and choose a 
closed Jordan curve
$\Gamma$ which is the union of the interval
$[\theta^*_0,\theta_1^*]$ on $T$
a Jordan arc $\gamma$ which is disjoint to the closure of
$K\setminus T$ while $z_0$ belongs to the Jordan domain
$\Omega$ bordered by $\Gamma$. We can always   choose
a nice arc $\Gamma$ which is of class $C^1$ and hits
$T$ at $e^{i\phi_0}$ and $e^{i\phi_1}$ at right angles.
Since $\Gamma$ has a positive distance from
$K\setminus T$ there exists $r_*<1$ such that if $r_*<r<1$
then the functions
\[ 
W_r(z)=W(rz)\quad\colon\, R_r(z)=R(rz)\tag{3}
\] 
are analytic
in a neighborhood of the closure of $\Omega$.
Now we consider  the set $\pi^{-1}(\Omega)$ in
$\mathcal M_B$ whose boundary in $\mathcal M_B$
is contained in 
$\pi^{-1}(\Gamma)$.
If $Q(z)$ is an arbitrary  polynomial the\emph{ Local Maximum Principle} 
gives
\[
|Q(z_0)|\cdot[ \widehat f(\xi)\cdot R_r(z_0)-W_r(z_0)|\leq
\bigl |Q\cdot( \widehat f\cdot R_r-W_r)\bigr |_{\Gamma^*}\tag{4}
\]



\noindent
Recall that $\pi^{-1}(T)$ is a copy of $T$ 
Identifying  the subinterval
$[\theta^*_0,\theta^*_1]$
with a closed subset of $\mathcal M_B$ we can write
\[
\pi^{-1}(\Gamma)=\gamma^*\cup\,[\theta^*_0,\theta^*_1]
\quad\colon\,\gamma^*=
\pi^{-1}(\Gamma)\setminus(\theta^*_0,\theta^*_1)\tag{5}
\]


\noindent
Now (4) and the continuity of the Gelfand transform
$\widehat f$
give a constant $M$ which is independent of
$r$ such  that
the maximum norms
\[
m(r)=\bigl | \widehat f\cdot R_r-W_r)\bigr |_{\pi^{-1}(\Gamma)}\leq M\quad\colon
r_*<r<1\tag {6}
\]
\medskip
Next, since we already know that
$\widehat f(e^{i\theta})=f(e^{i\theta})$
holds on $T$ it follows from (1) that the maximum norms:
\[
m(r)=\bigl |\widehat f\cdot R_r-W_r|_{[\theta^*_0,\theta^*_1]}=0\tag{7}
\]


\noindent
tend to zero as $r\to 1$.
Next, for each  $\epsilon>0$,  Runge's theorem gives a
polynomial $Q(z)$ such that
\[
Q(z_0)=1\quad\colon\, |Q|_\gamma<\frac{\epsilon}{M}\tag{8}
\]
When $\xi\in \pi^{\vvv 1}(z\uuu 0)$
it follows from (6) that
\[
|\widehat f(\xi) R_r(z\uuu 0)\vvv W_r(z\uuu 0)|\leq
\text{Max}\,(\epsilon,\bigl |Q \bigr |_{[\theta^*_0,\theta^*_1]}\cdot\delta(r))\tag{9}
\]
Passing to the limit as $r\to 1$ we have seen that  $m(r)\to 0$, and
together with the
obvious limit formulas
$R_r(z_0)\to R(z_0)$ and $W_r(z_0)\to W(z_0)$ we conclude that
that
\[
\bigl |\widehat f(\xi)\cdot R(z_0)-W(z_0)\bigr |\leq\epsilon\tag{10}
\]
Since we can choose $\epsilon$ arbitrary small
we get
\[
\widehat f(\xi)\cdot R(z_0)=W(z_0)\quad\colon\,\xi\in \pi^{-1}(z_0)\tag{11}
\]
\medskip

\noindent
To profit upon (11)
we first notice that 
the $R(z)$  cannot be identically zero
in $D\setminus K$ for then the Riesz measure $\mu$
would be identically zero by the observation in � xx.
If $R(z)\neq 0$ for some $z\in D\setminus K$
then (11) entails that
the fiber
$\pi^{-1}(z)$ is reduced to a single point. So this
hold for all points in 
$D\setminus K$ with an eventual exception of a discrete subset.
Applying the oocal maximum principal for points in
a fiber $\pi^{-1}(z)$ where $R(z)=0$
the reader may verify that
it again is reduced to a single point and 
the Sublemma follows.
\medskip


\noindent
\emph{Final part of the proof.}
The Sublemma and (**) show that 
when the Gelfand transform of $f$ 
is restricted to $\bar D\setminus K$
then it is analytic outside the eventual zeros of
$R(z)$. At the same time the Gelfand transform is a continuous function and since
isolated points are removable singularities for
bounded analytic functions, it follows
that the Gelfand transform restricted to
$\bar D\setminus K$ is an analytic function.
The same holds of course for every $g\in B$.
At this stage we apply Wermer's theorem.
For if $f|T$ does not belongs to
$A(D)$ every continuous function on
$T$ can be approximated uniformly
by polynomials in $f|T$ and $e^{i\theta}$.







\newpage


\centerline{\bf{� 5. Sets of uniqueness for analytic functions in the unit disc.}}

\medskip

\noindent
{\bf{Introduction.}}
We shall begin with some measure theoretic considerations.
If 
$E$ is a closed subset of $T$ then  $\phi_E(t)$ is the linear measure of
the set of points on $T$ whose distance to $E$ is $\leq t$.
Denote by $\mathcal N_*$ the family of 
closed subsets
$E$ of $T$ for which the integral
\[ 
\int_0^1\,\frac{\phi_E(t)}{t}\, dt<\infty\tag{*}
\]
Similarly , we have
the family $\mathcal N^*$ of closed null sets
for the integral is divergent.
\medskip


\centerline{\emph{4.1 Some facts about  the classes $\mathcal N_*$
and $\mathcal N^*$}}



\medskip

\noindent
In general, let $E$ be a closed null set
and put
\[
E_n=\{\theta\in T \colon 2^{-n-1}<\text{dist}(\theta,E)\leq 2^{-n}\}\quad\colon n=0,1,2,\ldots
\]
It means that the Lebesgue measure
$|E_n|$ satisfies
\[ 
|E_n|= \phi(2^{-n})- \phi(2^{-n-1})
\]
Since
\[
\int_{2^{-n-1}}^{2^{-n}}\, \frac{dt}{t}=\log 2
\] 
hold for every $n$ we get
\[ 
\int_0^1\,\frac{\phi_E(t)}{t}\, dt= \sum_{n=0}^\infty\, \int_{2^{-n-1}}^{2^{-n}}\,
\frac{\phi_E(t)}{t}\,dt\leq
\log 2\cdot \sum_{n=0}^\infty\, \phi_E(2^{-n})
\]
where the last
inequality holds since
the function $\phi_E(t)$ is increasing. 
The reader may also verify the inequality
\[
\log 2\cdot \sum_{n=1}^\infty\, \phi_E(2^{-n-1})\leq \int_0^1\,\frac{\phi_E(t)}{t}\, dt
\]
Hence tghe integral in (*) is divergent if and only if
\[
\sum_{n=0}^\infty\, \phi_E(2^{-n})=+\infty\tag{4.1.1}
\]

\medskip

\noindent
{\bf{Exercise.}}
Conclude from the above  that
(*) diverges if and only if
\[
\sum\, n\cdot |E_n|=+\infty\tag{4.1.2}
\]

\medskip

\noindent
Next, let $\{\omega_\nu=(\alpha_\nu, \beta_\nu)\}$
be the open intervals in $T\setminus E$.
Adding a finite set of points to $E$ if necessary we can assume
that each interval has length $\ell_\nu\leq 1$
for every $\nu$.
\medskip

\noindent
{\bf{Exercise.}}
Show that if $E$ is a closed null set
as above then the integral (*) is finite  if and only if
\[
\sum\, \ell_\nu\cdot \log\,\frac{1}{\ell_\nu}<\infty\tag{4.1.3}
\]


\medskip

\noindent
Using the results above we can prove
Beurling's uniqueness theorem from the introduction.

\medskip

\noindent
\emph{Proof of Theorem 0.1}.
Let  $\alpha>0$ and consider  a function 
$f\in A^\alpha(D)$ which 
vanishes on a closed null set $E$ in $T$.
The H�lder continuity entails that
\[ 
|f(t)\leq C \cdot \text{dist}(t,E)^\alpha
\]
for some constant $C$. Replacing $f$ by $C^{-1}f$ we may assume that $C=1$.
With $\{E_n\}$ as above
it follows that
\[
\int_{E_n}\, \log |f(t)|\, dt\leq 
-\alpha\cdot \log 2\cdot n\cdot |E_n|\tag{1}
\]
By (4.1.2) the divergence in (*)
implies that
\[ 
\sum_{n=0}^\infty\ 
\int_{E_n}\, \log |f(t)|\, dt=-\infty\tag{i}
\]
This cannot hold  unless $f$ is identically zero. In fact,
when $f$ is not identically zero it belongs  to the Jensen-Nevannlina
class
which  implies that
\[ 
\int_0^{2\pi}\, 
 \log |f(t)|\, dt>-\infty
\]
This finishes the proof of
Theorem 0.1.

\bigskip

\noindent
{\bf{The case
when $E\in \mathcal N_*$.}}
When this holds we are going to construct smooth functions in
the disc algebra which are zero on $E$.
More precisley we prove the following result which is due to
Carleson.




\medskip

\noindent
{\bf{4.1.4  Theorem.}}
\emph{If $E\in \mathcal N_*$  there exists for every positive integer $m$
a non-zero function $f\in A^m(D)$ such that $f=0$ on $E$
while $f$ is not identically zero.}
\medskip




\medskip


\noindent
The proof relies upon a number of constructions.
We are given $E\in\mathcal N_*$ and
let $\{\omega_\nu\}$
denote the family of open intervals in $T\setminus E$.
Adding a finite set of points to $E$ if necessary
we can assume
that
each interval has length $\leq 1$.
Define a function $h(\theta)$ outside the null set
$E$
as follows:
\[ 
h(\theta)= \log\, \frac{1}{\beta_\nu-\theta}+
 \log\,\frac{1}{\theta- \alpha_\nu}
\quad\colon\, \alpha_\nu<\theta<\beta_\nu\tag{i}
\]
Notoce that $h$ is a non-negative function.We notice that
\[
\int_{\omega_\nu}\, h(\theta)\, d\theta=
xxx \tag{ii}
\]
From (ii) and the hypothesis that
$E\in\mathcal N_*$
it follows that
the almost everywhere defined and non-negative $h$-function is
integrable on $T$, i.e.
\[ 
\int_0^{2\pi}\, h(\theta)\, d\theta<\infty\tag{iii}
\]
If $K$ is a positive integer we
construct the zero-free analytic function
in ther open disc defined by the exppnential Herglotz integral:
\[ 
f(z)=xxx\tag{iv}
\]
From the construction of $h$ it follows that
when $\theta\in\omega_\nu$  for some
$\nu$, then

\[
\log |f(e^{i\theta})|
=-K\cdot h(\theta)\implies
|f(e^{i\theta})|= |\beta_\nu-\theta|^K\cdot
|\theta-\alpha_\nu|^K\tag{v}
\]
Next, from 
the construction in (iv) the analytic function $f(z)$�is bounded
in $D$
so its radial oimits exist almost evertywhere
and yields the boundary value function $f^*(\theta)$.
On the open intervals the
radial limits exist and nicely since
$h$ is a  real-analytic function on
the $\omega$-intervals. In particular the absolute
value $|f^*|$ is equal to the right hand side in (v) for every
$\omega$ interval.  Since it vanishes at the end-points we can
extend $f^*$ to $T$ where it is zero on
$E$.
Moreover, since $E$ is   a nullset we have 
\[
f(z)=Cauchy formula with f^*\tag{vi}
\]
In  (xx) we have the positive integer $K$
which obviously entials that
the boundary function $f^*(\theta)$ is Lipschitz continuous
and then the result in � xx implies that
$f(z)$ is Lipschitz continuous in the whole disc. In particular 
$f$ belongs to $A(D)$
 and the complex derivative $f'(z)$ is in the open disc
 is a bounded analytic function.
It turns out that we get more regularity when
$K$ is large. To prove this we
use the expression of the logarithmic derivative of $f$ which comes
from (xx). More precisely, by the general result in � xx we have

\[
\frac{f'(z)}{f(z)}= =xxx\cdot xxx \tag{vii}
\]
Let $\theta$ belong to an interval $\omega_\nu$.
We shall estimate the abolute values of
$|f(re^{i\theta}]|$
as $r\to 1$.
To acihve this we consider the interval on $T$ given by

\[ 
\omega_*=\{x\,\colon\, |x-\theta|\leq \frac{1}{8}\cdot \rho_\nu(\theta)\}
\]
\medskip

\noindent
{\bf{Exercise.}}
Show with the aid of a figure that
$\omega_*$ is a closed subinterval of
$\omega_\nu$.
\medskip

\noindent
If $e^{ix}\in T\setminus \omega_*$
we have
\[
|e^{ix}-re^{i\theta}|\geq\frac{1}{8}\cdot \rho_\nu(\theta)\}
\]
The triangle inequality entails that
\[
\bigl|\int_{T\setminus\omega_*}\, xxxx\bigr|\leq
\frac{64}{ \rho_\nu(\theta)^2}\cdot
\int_{T\setminus\omega_*}\, |h|
\]
Now we  estimate the integral
over $\omega_*$. 
Let us for exampe consider the integral
\[ 
\int_{\omega_*}\, 
\log\,\frac {1}{x-\alpha_\nu}\cdot \frac{e^{ix}}{(e^{ix}-re^{i\theta})^2}\, dx
\]
Partial integration  estimates this integral
by an absolute constant times
$\rho_\nu(\theta)^{-2}$.
\bigskip


\noindent
From the above
we conclude that the boundsry value function
of the complex derivtive
$f'(z)$
satisfies
\[
|f'(e^{i\theta})|\leq C\cdot \rho_\nu(\theta)^{K-2}
\]
So if $K\geq 3$  the derivative 
$f'$ is Lipschits continuous on $T$ and hence
the second order derivative $f''(z)$ is bounded in $D$.
\medskip

\noindent
{\bf{Exercise.}}
Proceed as above and show that
if $K\geq 4$ then the third order derivative
is bounded and so on. So if $m$ is
an arbitrary positive integer we can take
$K$ so  large that 
the $f$-function in (xx) belongs to
$A^m(D)$ and at the same time it vanishes on $E$ which proves
Theorem 4.1.4





\newpage

\centerline{\bf{5. Functions with finite Dirichlet integrals}}
\bigskip

\noindent
We shall consider closed sets
$E$ in $T$ with a positive logarithmic capacity.
Recall from � xx that it means that
there exists a probablity  measure $\mu$ on $E$ whose energy intergal
\[
\iint\,\log\,\frac{1}{|e^{i\theta}-e^{i\phi}|}\,
d\mu(\theta)\cdot d\mu(\phi)<\infty
\]
\medskip

\noindent
Next, denote by $\mathcal D$ the class of analytic functions
$f$ in the open unit disc
with a finite Dirichlet integral, i.e.
\[ 
\iint_D\, |f'(z)|^2\, dxdy<\infty
\]
In � XX  we proved a result due to Beurling which
asserts that
if $f\in\mathcal D$
then the radial limits
\[
\lim_{r\to 1}\, f(re^{i\theta})\tag{1}
\]
exists for all $\theta$ outside a set whose logartihmic capacity is zero.
\medskip

\noindent
{\bf{5.1 The class $\mathcal D_E$.}}
Let  $E$  be a closed subset of $T$ with 
positive logarithmic capacity and consider some
$f\in\mathcal D$.  Beurling's result gives
a set
$\mathcal N_f$
of logarthmic capactiy zero such that
the radial limits exist in
$T\setminus \mathcal N_f$.
We say that $f=0$ holds almost everywhere on
$E$
if the set of non-zero Beurling limits in $E\setminus \mathcal N_f$
has logarithmic capacity zero.
The class of these functions is denoted by
$\mathcal D_E$.
\medskip

\noindent
Now we can announce a  major results from [ibid]
where
$E$ is a closed set 
with positive capacity and in addition belongs to
$\mathcal E$. Since smooth functions in
$A(D)$ have finite Dirichlet integrals, it follows from Theorem 4.1.4
that the class
$\mathcal D_E$  contains
functions which are not identically
zero.
Let $\mathcal D_E^*$ be the family of functions
in $\mathcal D_E$ which are normalised so that $f(0)=1$.
\medskip

\noindent
{\bf{5.2  Theorem.}}
\emph{When $E$ is as above
the variational problem}
\[
\min_{f\in \mathcal D_E^*}\, D(f)
\]
\emph{has a unique solution. Moreover, the extremal function
$f_E$
extends to a continuous function in
$\bar D\setminus E$ without zeros and the complex derivative
$f'(z)$ extends to an analytic function in
${\bf{C}}\setminus E$.}
\bigskip

\noindent
{\bf{About the proof.}}
That the variational problem has a unique solution
$f_E$ which is zero-free in
the open disc is
is
fairly easy to establish.
 See � xx below.
The remaining parts of the proof are more
involved and require  several steps.
First we will 
show that
$f_E$  extends continuously to $\bar D\setminus E$
and the boundary value function $f(e^{i\theta})$
is locally Lipschitz continuous on the open
set
$T\setminus E$.
Using this regularity  the next step is to show
that
$f_E$ has no zeros on $T\setminus E$.
To prove this one uses the extremal property which entails that
if $\tau(z)$ is an arbitrary function in
$A^2(D)$
which is zero at the origin, then
\[ 
t\mapsto D(f\cdot e^{t\tau})
\]
achieves its minimum when $t$ varies over real numbers.
The final step in � xx below shows that
the previous facts imply that the derivative $f'$ extends to
be analytic in ${\bf{C}}\setminus E$.
\bigskip






\centerline{\bf{� 5.3 Preliminaries.}}
\medskip


\noindent
We 
expose some general facts which are used in the subseqent proofs.
Let $\mu$
be a non-negative Riesz measure on the unit circle which 
has  complex Fouirer coefficients, as well as the
trigonometric coefficients:
\[ 
\widehat{\mu}(n)=\frac{1}{2\pi}\int_0^{2\pi}\, e^{-in\theta}\, d\mu(\theta)
\]
\[ 
a_n=\frac{1}{\pi}\int_0^{2\pi}\int\,\cos (n\theta)\, d\mu(\theta)
\quad\colon\quad 
b_n=\frac{1}{\pi}\int_0^{2\pi}\int\,\sin (n\theta)\, d\mu(\theta)
\]
where $\{a_n\}$ and $\{b_n\}$ are defined for positive integers.
Next we have the analytic function in $D$ defined by
\[
\mathcal H_\mu(z)=\frac{1}{2\pi}\int_0^{2\pi}\,
\frac{e^{i\theta}+z}{e^{i\theta}-z}\,
d\mu(\theta)
\]
It has the Taylor series expansion
\[
\mathcal H_\mu(z)=\frac{1}{2\pi}+\frac{1}{\pi}\sum_{n=1}^\infty
\widehat{\mu}(n)
\cdot z^n
\]
where the constant term appears since
$\mu$ is a probability measure.
A computation which is left to the resder shows that
\[
\iint_D\, |\mathcal H_\mu(z)|^2\, dxdy
=++++
\]
\medskip

\noindent
Next,  the logarithmic potential defined by
\[ 
L_\mu(\theta)=\int\, \log\,\frac{1}{|e^{i\theta}-e^{is}|} \,d\mu(s)
\]
The energy integral 
\[ 
J(\mu)
=\int\, \log\,\frac{1}{|e^{i\theta}-e^{is}|} \,d\mu(s)\cdot d\mu(\theta)
\]
can be computed via the Fouier series expansion of $\mu$.
The crucial point is the Fourier formula
\[
\log \frac{1}{|e^{i\theta}-e^{is}|}\simeq
\sum_{n=1}^\infty\,n^{-1}\cdot \cos(\theta-s)
\]
The trigonometric identity
\[
\cos(\theta-s)=\cos(\theta)\cos(s)+ \sin(\theta)\sin(s)
\]
entails that
\[
J(\mu)=\pi\cdot \sum_{n=1}^\infty\, \frac{a_n^2+b_n^2}{n}
\]
So the condition that $\mu$ has a finite energy integral is expressed
by
the convergence in the right hand side.
\medskip

\noindent
Next, let $f(z)=\sum\ c_nz^n$ be the Taylor series of
an analytic function with finite Diriehlet integral.
Now
\[
\frac{1}{2\pi}\cdot \int_0^{2\pi} f(e^{i\theta})\cdot d\mu(\theta)=
f(0)+
\sum_{n=1}^\infty \, c_n\cdot \widehat{\mu}(-n)
\]
The Cauchy-Schwarz inequality majorizes the absolute value of the last sum by
\[
\sqrt{\sum_{n=1}^\infty \, n|c_n|^2}\cdot
\sqrt{\sum_{n=1}^\infty \, n^{-1}|\widehat{\mu}(-n)|^2}
\]
Now
\[
|\widehat{\mu}(-n)|^2=4(a_n^2+b_n^2)
\]

Majorize....
\medskip


\noindent
{\bf{Green's formulas.}}
Let $f$ and $g$ be a pair in $\mathcal D$.
We write $f=u+iv$ and $g=\xi+i\eta$.
The Cacuhy-Riemann equations give
\[ 
\iint_D\, f'(z)\cdot \overline{g'(z)}\, dxdy=
\iint_D\, (u_x-iu_y)(\xi_x+i\xi_y)\, dxdy
\]
Hence the real part becomes
\[
\iint_D\, (u_x\xi_x+u_y\xi_y)\, dxdy
\]
Using Stokes formua we recal from � xx that (i) is equal to
\[
\lim_{r\to 1}\,\int_0^{2\pi}\, 
u(re^{i\theta})\cdot \frac{\partial v}{\partial r} (re^{i\theta})\, d\theta
\]
Hence one has the equation

\[ 
\mathfrak{Re}\, 
\iint_D\, f'(z)\cdot \overline{g'(z)}\, dxdy=
\lim_{r\to 1}\,\int_0^{2\pi}\, 
u(re^{i\theta})\cdot \frac{\partial v}{\partial r} (re^{i\theta})\, d\theta\tag{*}
\]
\medskip

\noindent
{\bf{Integrals of $\Delta(|f|^2)$.}}
As above $f$ is a function in
$\mathcal D$.
With $f=u+iv$ one has
\[
\Delta(|f|^2)=\Delta(u^2)+
\Delta(v^2)=2(u_x^2+u_y^2+v_x^2+v_y^2)=4(u_x^2+u_y^2)
\]
Hence we have
\[
4\cdot D(f)=\iint_D\, \Delta(|f|^2)\, dxdy
\]
Keeping $f$ fixed while $g$ is a function in $A^1(D)$
and $t$ is a real number
we get the analytic function $f\cdot e^{tg}$.
Write $g=\xi+i\eta$
which gives
$|fe^{tg}|^2=|f|^2\cdot e^{2t\xi}$.
With $t$ small one has the expansion
\[
|f|^2\cdot e^{2t\xi}= |f|^2+2t\cdot |f|^2\cdot \xi+O(t^2)
\]
Here $\xi$ is a harmonic function which entails that
\[
\Delta(|f|^2\cdot \xi)=
\Delta(|f|^2\cdot \xi+2\partial_x(|f|^2)\cdot \xi_x+
2\partial_y(|f|^2)\cdot \xi_y
\]
\medskip

\noindent
{\bf{Exercise.}} Set $U=\Delta(|f|^2$ and
use Green-s formula to show that
(x) is equal to

\[
\lim_{r\to 1}\,\int_0^{2\pi}\, 
\bigl[\frac{\partial U}{\partial r}(re^{i\theta})
\cdot \xi(re^{i\theta})+
\frac{\partial \xi}{\partial r}(re^{i\theta})
\cdot U(re^{i\theta})\bigr]\,d\theta
\]
\medskip

\noindent
Now we can consider the boundsry value function
of $U$
and solving Dirichlet's problem we find
it harmonic extension in $D$ denoted by $u$.
\medskip

\noindent
{\bf{Exercise.}}
Use Green's formula to show that
\[
\lim_{r\to 1}\,\int_0^{2\pi}\, 
\frac{\partial \xi}{\partial r}(re^{i\theta})
\cdot U(re^{i\theta})\bigr]\,d\theta=
\lim_{r\to 1}\,\int_0^{2\pi}\, 
\frac{\partial u}{\partial r}(re^{i\theta})
\cdot \xi(re^{i\theta})\,d\theta
\]
\medskip

\noindent
{\bf{Conclusion.}}
With $g=\xi+i\eta$
one has
\[
4D(fe^{tg})=
4D(f)+2t\cdot 
\lim_{r\to 1}\,\int_0^{2\pi}\, (\frac{\partial U}
{\partial r}+\frac{\partial u}{\partial r})\cdot \xi\, d\theta+O(t^2)
\]





























 















\newpage


\centerline{\bf{Proof of Theorem 5.2.}}
\bigskip


\noindent
The proof requires several steps.
First we prove that the minimum in (*) form Theorem 5.2 is a posotive number
denoted by  
$D_*(E)$. to see rthis we use the assumtion that
$e$ has positive capacity  which gives
a probbility measure
$\mu$ on $E$ with a finite  energy integral
\[ 
J(\mu)=
\]
Recall from (xx) that one has the equality
\[ 
J(\mu)=\sum_{n=1}^\infty\, \frac{1}{n}\cdot |\widehat\mu(-n)|^2
\quad\colon\, 
\widehat\mu(-n)=\int_0^{2\pi}\, e^{in\theta}\cdot d\mu(\theta)\tag{i}
\]

\medskip

\noindent
If  $f=1+\sum_{n=1}^\infty \, a_nz^n$
is  a function in $\mathcal D_E$ and $r<1$ we have
\[ 
\int_0^{2\pi}\, f(re^{i\theta}\cdot d\mu(\theta)
=1+\sum\, a_nr^n\cdot \int\, e^{in\theta}d\mu(\theta)\tag{ii}
\]
The Cauchy-Schwarz inequality and (i) imply
that the absolute value of
the last sum is majorized by
\[
\sqrt{\sum\, n|a_n| ^2\cdot r^{2n}}\cdot \sqrt{J(\mu)}\tag{iii}
\]

\noindent
Next, since $f\in\mathcal D_E$
its radial limits are zero almost everywhere with respect to
$\mu$ on the closed set $E$, i.e. the left hand side in (ii) tends to zero.
A  passage to the limit as $r\to 1$ together with (ii) and (iii)
therefore give
\[
\mathcal D(f)\cdot \mathcal J(\mu)\geq 1\tag{iv}
\]
Since $f\in\mathcal D_E$ was arbitrary we get   the lower bound
\[ 
D_*(E)\geq \frac{1}{J(\mu)}
\]
\medskip

\noindent
{\bf{1.A Existence of an extremal function.}}
To prove that there exists
some
$f\in\mathcal D_E$ for which
$\mathcal D(f)=D_*(E)$ it suffices - via the strict inequality of the
inner product norm - to verify that
$\mathcal D_E$ appears as a closed subset of $\mathcal D$.
So consider a sequence $\{f_n\}$ in $\mathcal D_E$
with limit $f$ taken in
the $\mathcal D$-norm.
We must prove that the boundary values of $f$
are zero almost everywhere
with respect to an arbitrary
positive measure $\mu$ on $E$ whose energy integral is finite.

\medskip
PROVE ...+ZERO FREE

\bigskip


\centerline{\bf{1.B Properties of the extremal function.}}
\bigskip

\noindent
We have proved the existence of a unique $f\in\mathcal D_E$
for which $\mathcal D(f)= D_*(E)$ and in � x we have shown that
$f$ is zero-free in $D$. 
We will show that
$f$ 
extends to a continuous function on $\bar D\setminus E$
and that the continuous extension
has no
zeros in
$T\setminus E$.
Let us for a while
admit these properties of $f$ and
use it to establish
the  analytic extension of
the derivative $f'$ in Theorem XX.
For this purpose we  construct another analytic function in
$D$.
\medskip

\noindent
{\bf{1.C  The function $F(z)$}}.
If $p$ is a positive integer and $\lambda$ is a complex number
then
$f+\lambda\cdot z^pf$ belongs to $\mathcal D_E$. This
gives
\[ 
\mathcal D(f+\lambda\cdot z^pf)\geq \mathcal D(f)
\]
Since it holds for small non-zero $\lambda$ a standard argument in the calculus
of variation and the formuoa (xx)
give
\[
0=p a_0\bar a_p+(p+1)a_1\bar a_{p+1}+\ldots\tag{1}
\]



\noindent
Put
\[ 
c_p=\sum_{n=1}^\infty\, n\cdot \bar a_na_{n+p}\quad\colon\,
p=0,1,2,\ldots\tag{2}
\]
Since $\sum\, n\cdot |a_n|^2<\infty$
the right hand side is an absolutely convergent series for
each $p$ and
there exists the analytic function $F(z)$ in the unit disc defined by
\[ 
F(z)= \sum_{p=1}^\infty\, c_p\cdot z^p\tag{4}
\]
We shall
express $F$ as a limit of
analytic functions defined in
smaller discs than $D$.
To each $0<r<1$ we 
get the analytic function $F_r(z)$ in $\{|z|<r\}$
defined by

\[ 
F_r(z)= \frac{1}{2\pi i}\cdot
\int_{|\zeta|= r}
\bar f'(\zeta)\cdot f(\zeta)\cdot \frac{1}{\zeta-z}\cdot \frac{d\zeta}{\zeta}\tag{5}
\]
With $\zeta= re^{i\theta}$
we use the expansion
\[
\frac{1}{re^{i\theta}-z}=
\sum_{p=0}^\infty
r^{-p-1}\cdot e^{-i(p+1)\theta}z^p
\]
The coefficient of $z^p$
in the Taylor series of $F_r(z)$ becomes
(5) becomes
\[
 \frac{1}{2\pi }\cdot\sum\sum\, 
\int_0^{2\pi}
\,\bar a_n r^{n-1}e^{-i(n-1)\theta}\cdot a_m\cdot r^me^{im\theta}
\cdot r^{-p-1} \cdot e^{-i(p+1)\theta}\,d\theta=
\sum\, r^{2n+p}\, \bar a_n\cdot a_{n+p}
\]
Denote the last sum by $c_p(r)$ so that
\[ 
F_r(z)= \sum\, c_p(r)\cdot z^p\tag{6}
\]
Then it is clear that
\[ 
\lim_{r\to 1}\, F_r(z)=F(z)\tag{7}
\] 
with uniform convergence  in compact subsets of $D$.
So far we have not used the vanishing in (1).
With $|z|<r$ we set
\[ 
G_r(z)=
\frac{1}{2\pi i}\cdot
\int_{|\zeta|= r}
\bar f'(\zeta)\cdot f(\zeta)\cdot \frac{\bar z}{\bar \zeta-\bar z}\cdot\frac{d\zeta}{\zeta^2}\tag{7}
\]
\medskip

\noindent
{\bf{Exercise.}} Use (1) to show that
(7) is identically zero.
\medskip

\noindent
Now we can add the zero function $G_r(z)$ to $F_r(z)$
and use that
\[ 
\frac{\zeta}{\zeta-z}+\frac{\bar z}{\bar\zeta-\bar z}=
\frac{|\zeta|^2-|z|^2}{|\zeta-z|^2}
\]
This gives
\[ 
F_r(z)=\frac{1}{2\pi i}\int_{|\zeta|= r}\, 
\bar f'(\zeta)\cdot f(\zeta)\cdot 
\frac{|\zeta|^2-|z|^2}{|\zeta-z|^2}\cdot \frac{d\zeta}{\zeta^2}=
\frac{1}{2\pi }\int_0^{2\pi}\,
\bar f'(re^{i\theta})\cdot f(re^{i\theta})
\cdot\frac{r^2-|z|^2}{|re^{i\theta}-z|^2}\cdot
\frac{d\theta} {re^{i\theta}}
\]
\medskip

\noindent
In the last integral the Poisson kernel appears.
Consider the radial limit function:
\[ 
g^*(\theta)= \lim_{r\to 1}\, \frac{1}{re^{i\theta}}\cdot 
\bar f'(re^{i\theta})\cdot f(re^{i\theta})\tag{8}
\]
By the above
$F(z)$ is
Poisson's extension of $g^*(\theta)$ and hence
$g^*(\theta)$ is the boundary value
of an analytic function.
If we for the moment admit that 
$f$ extends to a continuous function on
$T\setminus E$
and if $\omega$ is an  open subinterval 
where  $f$ has no zeros, then
the boundary value of the complex conjugate function
$\bar f'(z)$ taken on $\omega$ coincides with those of 
\[
\frac{z\cdot F(z)}{f(z)}
\]
Schwarz's reflection principle 
implies that 
$f'(z)$ extends analytically across $\omega$
whose   the extension to the exterior disc becomes
\[
z\mapsto \frac{\bar F(\frac{1}{\bar z})}{z\cdot \bar f(\frac{1}{\bar z})}\tag{9}
\]
Since we already know that
$f(z)$ is zero-free in $D$, the function in (9)
is analytic in the whole exterior disc
$\{|z|>1\}$. This proves that
if $f$  is zero-free
on $T\setminus E$ then
the complex derivative
$f'(z)$ extends to an analytic function  in
${\bf{C}}\setminus E$.


\bigskip

\centerline{\bf{1.8 Proof of the continuous extension lemma}}

\medskip

\noindent
Theorem 0.1  enable us to construct an ample family
of smooth functions
which vanish on $E$. Applied  with $m=2$  the
reader should verify the following:

\medskip

\noindent
{\bf{1.8.1 Exercise.}}
For each point  $\xi\in T\setminus E$
there exists some
open inteval $\omega$ centered at $\xi$
and $h\in A^2(D)$
where $h(0)=0$ and
$\mathfrak{Re}\, h=0$ on $\omega$
while $\mathfrak{Im}\, h\neq 0$ on $\omega$.
\medskip

\noindent
Next, let $\omega_*$ be a compact subinterval of $\omega$
centered at $\xi$
and $\psi$ is a real-valued $C^2$-function
on $\omega_*$ which vanishes up to order two at the end-points.
Solving the Dirichlet
problem we find
a function $p(z)\in A(D)$ such that
\[
\mathfrak{Im}(p)=
\frac{\psi}{\mathfrak{Im}\,h}
\quad\text{holds on}\,\,\, \omega_*
\quad\text{and}\quad 
\mathfrak{Im}(p)|T\setminus\omega_*=0\tag{1.8.2}
\]
Set
\[ 
g= hp
\]
Since the real part of  $h$ 
is zero on $\omega$ it follows  that
\[
\mathfrak{Re}\, g=-\psi\quad \text{holds on}\quad \omega\tag{1.8.3}
\]
Next, since $h(0)=0$ we also have $g(0)=0$ and therefore
$f+\lambda\cdot g$ belongs to $\mathcal D_E$ for
each complex number $\lambda$.
Since $f$ is an extremal in the  variational problem,
a standard argument shows that
the inner product
$\langle f,g\rangle=0$. Let us write
\[
f=u+iv\quad\colon\quad g=\sigma_1+i\sigma_2\tag{1.8.3}
\]
The inner product formula (xx) gives
the equation
\[
\iint_D\, (\frac{\partial v}{\partial x}\cdot
\frac{\partial \sigma_2}{\partial x}+
\frac{\partial v}{\partial y}\cdot
\frac{\partial \sigma_2}{\partial y}\, dxdy=0\tag{1.8.4}
\]
Above $v$ and $\sigma_2$ are harmonic functions
and the integral is the limit of double integrals taken
over discs
$\{|z|\leq r\}$ as $r\to 1$. Using Green's formula we therefore get
\[
\lim_{r\to 1}\, \int_0^{2\pi}\, v(re^{i\theta})\cdot
\frac{\partial \sigma_2}{\partial r}(re^{i\theta})\, d\theta=0\tag{1.8.5}
\]
Since the harmonic functions $\sigma_1$ and $\sigma_2$
are conjugate one has: 
\[
\frac{\partial \sigma_2}{\partial r}(e^{i\theta})=-
\frac{\partial \sigma_1}{\partial \theta}(e^{i\theta})\tag{1.8.6}
\]
Hence (1.8.5) and  (1.8.3) give
\[ 
\int_{\omega_*}\,v(e^{i\theta})\cdot 
\frac{\partial \psi}{\partial \theta}(e^{i\theta})\,d\theta
=
\int_{T\setminus\omega}
\,v(e^{i\theta})\cdot 
\frac{\partial \sigma_1}{\partial \theta}(e^{i\theta})\,d\theta\tag{1.8.7}
\]
where
\[
\sigma_1=\mathfrak{Re}\, h\cdot p_1
\quad \text{holds in}\,\,  T\setminus \omega_*
\]
Here $p_1$ is the harmonic extension of $\psi$
which is supported by $\omega_*$ and
$\mathfrak{Re}\, h=0$ in $\omega$.
It follows from the generalinequality � xx that there exists
a constant $C$ which only depends upon $h$ and the pair
$\omega_*,\omega$
such that
\[
\max_{\theta\in T\setminus\omega}
|\frac{\partial \sigma_1}{\partial \theta}(e^{i\theta})|\leq
C\cdot
\int_{\omega_*}\, |\psi(\theta)|\, d\theta\tag{1.8.8}
\]
Hnece we have the inequality
\[
\bigl|\int_{\omega_*}\,v(e^{i\theta})\cdot 
\frac{\partial \psi}{\partial \theta}(e^{i\theta})\,d\theta\bigr|\leq
C\cdot
\int_{\omega_*}\, |\psi(\theta)|\, d\theta\tag{1.8.9}
\]
Above $\psi$ was an arbitrary
$C^2$-function which vanishes up to order
two at
the end-points of $\omega_*$.
By general distribution theory this entaills that
the restriction of the boundsry value function
$v$ to the open interval $\omega_*$ is Lipschitz continuous.
As explained in � xx this entials that
$v$ has a continuous extension to this interval.
Finally, starting exactly as above with a function
$h$ where we now let $\mathfrak{Im}\, h=0$ on
$\omega$ the reader may check that a similar conclusion holds 
for the real part of $f$ which finishes the proof of the Continuity Lemma.






















































\end{document}



