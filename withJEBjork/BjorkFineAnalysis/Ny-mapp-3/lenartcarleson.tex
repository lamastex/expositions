
\documentclass{amsart}
\usepackage[applemac]{inputenc}


\addtolength{\hoffset}{-12mm}
\addtolength{\textwidth}{22mm}
\addtolength{\voffset}{-10mm}
\addtolength{\textheight}{20mm}
\def\uuu{_}


\def\vvv{-}

\begin{document}


\centerline{\bf\large{6. Interpolation and solutions to the $\bar\partial$-equation.}}


\bigskip

\centerline{\emph{Contents.}}

\bigskip

\noindent
\emph{1. Carleson's interpolation theorem}
\bigskip

\noindent
\emph{2. Wolff's theorem}


\bigskip

\noindent
\emph{3. A class of Carleson measures.}


\bigskip

\noindent
\emph{4. Berndtsson's $\bar\partial$-solution}

\bigskip

\noindent
\emph{5. H�rmander's $L^2$-estimate}
\bigskip

\noindent
\emph{6. The Corona problem.}


\bigskip






\centerline{\bf I. Carleson's Interpolation Theorem}

\bigskip
\noindent
{\bf Introduction.}
Let $U=\mathfrak{Im}(z)>0$
be the upper half-plane. 
Denote by ${\bf{c}}_*$
the family of sequences of complex numbers 
$\{c_\nu\}$ where every
$|c_\nu|\leq 1$. 
A  sequence $z\uuu\bullet =\{z_\nu\}$
in $U$ has a finite interpolation norm
if there
exists a constant $K$ such that
for every 
sequence $\{c_\nu\}$ in
${\bf{c}}_*$ gives   an analytic function $f(z)$ in $U$
such that
\medskip
\[
f(z_\nu)=c_\nu\quad\colon\quad\nu=1,2,\ldots\quad\text{and}\quad |f|_U\leq K\tag{*}
\]

\noindent
The smallest  constant
$K$  
for which (*) holds is denoted by
$\mathfrak{int}(z_\bullet)$.

\medskip


\noindent
{\bf 0.1 Theorem.}
\emph{A sequence $z_\bullet$ has a finite  interpolating norm if and only if }
\medskip
\[
\min\uuu \nu\, \prod_{k\neq\nu}\, |\frac{z_\nu-z_k}{z_\nu-\bar z_k}|>0\tag{1}
\]


\noindent
\emph{Moreover, if $\delta(z\uuu\bullet)$ denotes the minimum
above then}
\[ \mathfrak{int}(z_\bullet)\leq \frac{4A}{\delta(z\uuu\bullet)}\cdot
\text{Log}
\,\frac{1}{\delta(z\uuu\bullet)}\,.\tag{2}
\]
\emph{where  $A$ is an absolute  constant.}

\medskip

\noindent
{\bf{Remark.}}
That the condition (1)  is
necessary
is easily verified. See Exercise XX below.
The proof of sufficieny is more involved.
In 
[Ca] the proof is carried out in the unit disc $D$
where (1) in
Theorem 0.1 means that a sequence 
$\{z\uuu\nu\}$ in $D$ should satisfy
\[
\min\uuu\nu \prod\uuu{k\neq \nu}\,
\frac{|z\uuu\nu\vvv z\uuu k|}{|1\vvv \bar z\uuu k\cdot z\uuu\nu|}>0
\tag{3}
\]
Carleson's result will be proved
in the upper half\vvv plane where  certain   constructions
become a bit more transparent  compared
to the unit disc.
Let $\{z\uuu\nu\}$ be a sequence in $U$ where (1) holds in Theorem 0.1.
Since a family  of analytic functions
in $U$ with a uniform upper bound for the maximum norm is a normal in Montel's sense,
it is sufficient to  prove the requested interpolation by bounded
functions
for every
finite subsequence of $\{z_\nu\}$.
The Nevanlinna-Pick theorem assigns   to each  finite
sequence $\{z_\nu\}$ and every sequence 
$\{c_\nu\}$  a unique interpolating analytic function
$F(z)$ with smallest maximum norm.
So Carleson's result  
gives a uniform bound in
the Nevannlinna-Pick interpolation expressed via the
numbers $\delta(z_\bullet)$.

\medskip





\centerline {\bf 0.1 Carleson measures.}
\medskip

\noindent
For every
$h>0$ we denote by
$\mathfrak{square}(h)$
the family of squares of the form
\[ 
\square=\{(x,y)\quad\colon\quad x_0-h/2<x<x_0+h/2\quad\colon 0<y<h\}
\quad\colon\, x_0\in{\bf{R}}
\]

\noindent
{\bf 0.2. Definition.}
\emph{A non-negative measure $\mu$ in $U$ is called a Carleson measure if
there exists a constant $K$such that}
\[
\mu(\square)\leq K\cdot h\quad\colon\quad
\square\in \mathfrak{square}(h)\quad\colon\, 0<h<\infty
\]
\emph{The least constant $K$ is denoted by
$\mathfrak{car}(\mu)$ and called the Carleson norm of
$\mu$.}
\medskip

\noindent
An essential step during   the
proof Theorem 0.1 is the following inequality:
\medskip

\noindent
{\bf 0.3 Theorem.}
\emph{Let $\{z\uuu\nu\}$ be a sequence with a positive $\delta$\vvv number.
Then}
\[
\mathfrak{car}\bigl (\, 
\sum_{\nu=1}^{\nu=\infty}\,
\mathfrak{Im}(z_\nu)\cdot \delta_{z_\nu}\bigr)\leq
 2\cdot\text{Log}\,\frac{1}{\delta(z_\bullet)}
\]
\emph{where $\{\delta\uuu{z\uuu\nu}\}$
denote  Dirac measures.}
\medskip

\noindent
{\bf {Use of duality.}}
Theorem 0.3 gives  the interpolation theorem
via a duality argument where
the Hardy space
$H^1({\bf{R}})$ appears.
Namely, to each $h\in H^1({\bf{R}})$ we associate the
maximal function
$h^*$ and
the following inequality  is proved in
� 2:

\medskip

\noindent
{\bf 0.4 Theorem.}
\emph{Let $\mu$ be a Carleson measure in the upper half-plane. Then}
\[ 
\int_U\, |h(z)|\cdot d\mu(z)\leq
\mathfrak{car}(\mu)\cdot ||h^*||_1\quad\colon\, h\in H^1({\bf{R}})
\]
\medskip


\noindent
Armed with Theorems 0.3-0.4 we derive
Theorem 0.1 in
� 3.

\bigskip


\centerline{\bf {1. Proof of Theorem 0.3}}
\medskip


\noindent
First we establish an inequality which is attributed to L. H�rmander.
\medskip



\noindent {\bf 1.1 Lemma}
\emph{Let
$z_1,\ldots,z_N$ be a finite sequence in $U$ and put $\delta=\delta(z\uuu\bullet)$.
Then}
\medskip
\[
\sum_{\nu\neq k}\, \mathfrak{Im}(z_k)\cdot
\frac{\mathfrak{Im}(z_\nu)}{|z_k-\bar z_\nu|^2}
\leq \frac{1}{2}\cdot
\log\,\frac{1}{\delta}\quad\colon\quad 1\leq k\leq N\tag{*}
\]


\noindent 
\emph{Proof.} The left hand side as well as
the $\delta$-norm of the
$z$-sequence are unchanged if we translate all points
to $z_\nu+a$ where $a$ is a real number. Similarly, the
$\delta$-norm and the left hand side in (*) are  unchanged when
the sequence is replaced by $\{A\cdot z_\nu\}$ for some $A>0$.
To prove (*) for a fixed $k$ which we may 
therefore take  $k=N$
and $z_N=i$.
Put $z_\nu=a_\nu+ib_\nu$ when $1\leq\nu\leq N-1$.
Then we must show
\[ 
\sum_{\nu=1}^{\nu=N-1}\,
\frac{b_\nu}{(1+b_\nu)^2+a_\nu^2}\leq \frac{1}{2}\cdot
\log\,\frac{1}{\delta}\tag{i}
\]

\noindent
To prove (i) we first notice that
\[
\frac{|i-\bar z_\nu|^2}{|i-z_\nu|^2}=
\frac{(1+b_\nu)^2+a_\nu^2}
{(1-b_\nu)^2+a_\nu^2}\tag{iii}
\]
By inverting the  $\delta=\delta(z_\bullet)$ we also have:
\[
 \prod\uuu{\nu=1}^{\nu=N\vvv 1}\,
\frac{(1+b_\nu)^2+a_\nu^2}
{(1-b_\nu)^2+a_\nu^2}\leq \delta^{\vvv 2}\implies
\sum_{\nu=1}^{\nu=N-1}\,
\log \bigl[\,\frac{(1+b_\nu)^2+a_\nu^2}
{(1-b_\nu)^2+a_\nu^2}\,\bigr]\leq 2\cdot\log\,\frac{1}{\delta}\tag{iii}
\]

\noindent
Next, for each $\nu$ we have the integral formula
\[
\log \, \frac{(1+b_\nu)^2+a_\nu^2}
{(1-b_\nu)^2+a_\nu^2}=\int\uuu{\vvv b\uuu\nu}^{b\uuu\nu}\,
\frac{2(1+ s)}{(1+ s)^2+a\uuu\nu^2}\, ds
\]

\noindent
Next, for every pair $(,b)$ of real numbers with $b>0$
the reader may verify that
\[
\frac{b}{(1+b)^2+a^2}\leq
\frac{1}{2}\cdot \int_b^b\, \frac{1+s}{(1+s)^2+a^2}\, ds\tag{iv}
\]
Now (iii-iv) and a summation over $\nu$ gives (i).





\newpage







\centerline{\emph{Final part of the
proof of  Theorem 0.3}}
\medskip


\noindent
If $z_\bullet\in\mathcal S(\delta)$ and $a$ is a real number
the sequence $z_\bullet+a=\{z_\nu+a\}$
also belongs to
$\mathcal S(\delta)$. Since Theorem 0.3 asserts
an
a priori estimate we may  assume that
$\square$ is centered at $x=0$, i.e. 
\[
\square=\{(x,y)\,\colon\, -h/2<x<h/2\,\,\text{and}\,\, 0<y<h\}
\]
There remains to estimate
\[
\sum_{z_\nu\in \square}\, \mathfrak{Im}\, z_\nu\tag{i}
\]
Set
\[
y^*=\text{max}\,\{ \mathfrak{Im}(z_\nu)\,\colon\,z_\nu\in\square\}
\]

\noindent
Let $k$ give the equality $y^*=\mathfrak{Im}(z_k)$.
With $z_k=x_k+iy^*$ and $z_\nu=x_\nu+iy_\nu\in\square$
we  have
\[
|z_k-\bar z_\nu|^2=(x_k-x_\nu)^2+(y^*-y_\nu)^2
\leq h^2+(y^*)^2\implies
\]
\[
\frac{\mathfrak{Im}(z_k)}{|z_k-\bar z_\nu|^2}\geq
\frac{y^*}{h^2+(y^*)^2}\quad\colon\,\nu\neq k
\]

\noindent
Next, notice that 
\[
y^*\geq h/2\implies\frac{y^*}{h^2+(y^*)^2}\geq \frac{1}{5h}\,.
\]

\noindent
Lemma 1.1. applied with $\nu=k$
gives therefore
\[
\sum_{z_\nu\in \square}\, \mathfrak{Im}(z_\nu)
\leq y^*+\frac{5h}{ 2}\cdot\text{Log}\,\frac{1}{\delta}\leq
h\cdot\bigl (1+\frac{5}{ 2}\cdot\text{Log}\,\frac{1}{\delta}\bigr)
\tag{ii}
\]


\noindent
So if $y^*\geq h/2$ we are done.
Suppose now that  $y^*<h/2$ and
regard the cubes:
\[
\square_1=\{
-h/2<x<0\,\text{and}\,\, 0<y<h/2\}\quad
\square_2=\{
0<x<h/2\,\,\text{and}\,\, 0<y<h/2\}
\]


\noindent
We want to estimate
\[
S_1+S_2=\sum_{z_\nu\in\square_1}\,\mathfrak{Im}(z_\nu)+
\sum_{z_\nu\in\square_2}\,\mathfrak{Im}(z_\nu)
\]
We have also two sequences:
\[
\{z_\nu\,\colon\,z_\nu\in\square_1\}\quad\text{and}\quad
\{z_\nu\,\colon\,z_\nu\in\square_2\}
\] 
Since all factors defining the $\delta$-norm are $\leq 1$
these two smaller sequences both belong to $\mathcal S(\delta)$.
Suppose that:
\[
y_1^*=
\max_{z_\nu\in\square_1}\,
\mathfrak{Im}(z_\nu)\geq \frac {h}{4}
\]
When this holds we obtain
exactly as above:
\[
S_1\leq \frac{h}{2}\cdot 2\cdot \text{Log}\, \frac{1}{\delta}
 \]
 
\noindent
If $y^*_1<\frac{h}{4}$ we continue to
split the cube $\square_1$. 
In a similar fashion we treat
the sequence which stays in $\square_2$.
After a finite number of steps we get the required inequality in
Theorem 0.3.


\bigskip


\centerline{\bf 2. Proof of Theorem 0.4}

\medskip


\noindent
Let $h\in H^1({\bf{R}})$ and  recall  that its maximal function is 
defined by
\[ 
h^*(t)=\max\, |h(x+iy)|\quad\colon\quad
|x-t|<y\tag{i}
\]

\noindent
To each  $\lambda>0$ we consider the open subset
on the real line defined by $\{h^*>\lambda\}$. It is some
union of disjoint intervals $\{(a_j,b_j)\}$ and
(i) gives the set-theoretic inclusion:
\[
\{|h(x+iy)|>\lambda\}\subset\,\cup \,\,T\uuu j
\quad\colon\tag{ii}
\]
where $T\uuu j$ is the  triangle side standing on the interval $(a_j,b_j)$
as explained in XXX.  (Hardy space). In particular we have
the inclusion:
\[ 
T\uuu j\subset
\square(a_j,b_j)=\{ x+iy\,\,\colon |x-\frac{a_j+b_j}{2}|< b_j-a_j\,\,
\colon\,\, 0<y< b_j-a_j\}\tag{iii}
\]

\noindent 
See figure XXX.
So if $\mu$ is a positive measure in $U$  we obtain:
\[
\mu(\{|h|>\lambda\})
\leq \sum \mu(T\uuu j)
\leq \sum \mu(\square(a_j,b_j))\tag{iv}
\]

\noindent 
If $\mu$ is a Carleson measure
the right hand side
is estimated by
\[
\mathfrak{car}(\mu)\cdot \sum\,(b_j-a_j)=
\mathfrak{car}(\mu)\cdot\mathfrak{m}(
\{h^*>\lambda\})\tag {v}
\]

\noindent
where  $\mathfrak{m}$
 refers to the 1-dimensional Lebesgue measure.
Here (v) holds for every $\lambda>0$ and  the general 
inequality for distribution functions from XXX gives:
\[
\int _U
|h|\cdot d\mu\leq \mathfrak{car}(\mu)\cdot||h^*||_1
\]
This finishes the proof of Theorem 0.4.

\bigskip



\centerline{\bf 3. Proof of Theorem 0.1.}
\bigskip

\noindent
The Banach space
$H^1({\bf{R}})$
contains a dense subspace
of functions $h(z)$ with polynomial decay at infinite, i.e.
functions  for which
\[
|h(z)|\leq C_N\cdot (1+|z|)^{-N} 
\quad\colon\text{hold for some constant}\,\,\, C_N\quad\colon\, N=1,2,\ldots
\]
This is used below to ensure that various integrals are defined
where it suffices to 
to use "nice" functions while an \emph{a priori}
inequality is established.
Consider a
finite sequence $z_1,\ldots,z_N$  in $U$
and
a finite sequence $c_1,\ldots,c_N$ in
${\bf{c}}_*$. Newton's interpolation
gives  a
unique  polynomial $P(z)$ of degree $N-1$ such that:
\[ 
P(z_k)=c_k\quad\colon\quad 1\leq k\leq N\tag{i}
\]

\noindent
Let $B(z)$ be  the Blascke product
associated to the $z$-sequence:
\[ 
B(z)=\prod_{\nu=1}^{\nu=N}\, \frac{z-z_\nu}{z-\bar z_\nu}\tag{ii}
\]

\noindent
Let $h\in H^1({\bf{R}})$ have the  polynomial
decay $\geq N+2$. Residue calculus gives
\[ 
\int_{-\infty}^\infty\,\frac{P(x)}{B(x}\cdot h(x)\cdot dx=
\sum_{k=1}^{k=N}\, \frac{c_k}{B'(z_k)}\cdot h(z_k)\tag{iii}
\]


\noindent
If $k$ is fixed we have
\[ 
\frac{1}{B'(z_k)}=\prod_{\nu\neq k}\,
\frac{z_k-\bar z_\nu}{z_k-z_\nu}\cdot 2\cdot \mathfrak{Im}(z_k)\tag{iv}
\]


\noindent
It follows that
\[
\bigl |\frac{1}{B'(z_k)}\bigr |\leq\frac{2}{\delta(z\uuu\bullet)}\cdot \mathfrak{Im}(z_k)\tag{v}
\]

\medskip
\noindent
Since $\{c_\nu\}\in{\bf{c}}_*$ we see that (v) and the 
triangle inequality applied to (iii) give:

\[ 
\bigl |\int_{-\infty}^\infty\,\frac{P(x)}{B(x}\cdot h(x)\cdot dx
\,\bigr | \leq
\frac{2}{\delta(z\uuu\bullet)}\cdot 
\sum_{k=1}^{k=N}| h(z_k)|
\cdot\mathfrak{Im}(z_k) \tag{*}
\]
\medskip


\noindent 
Now Theorem 0.4 gives the inequality
\[
\sum_{k=1}^{k=N}| h(z_k)|\cdot
\mathfrak{Im}(z_k)\leq\mathfrak{car}(\sum\,
\mathfrak{Im}(z\uuu\nu)\cdot \delta\uuu{z\uuu\nu}\cdot ||h^*||\uuu 1\tag{5}
\]

\newpage


\centerline{\emph{3.1 Use of duality.}}
 \medskip

\noindent
Let us put
\[
C\uuu\delta =\frac{2}{\delta(z\uuu\bullet)}\cdot 2\cdot \log\,\frac{1}{\delta(z\uuu\bullet)}
\]
Then (5) and Theorem 0.3
together with (*) give
\[
\bigl |\int_{-\infty}^\infty\,\frac{P(x)}{B(x}\cdot h(x)\cdot dx
\,\bigr | \leq C_\delta\cdot ||h^*||\uuu 1
\]

\noindent
Next,  the result in 
(Hardy Chapter ) gives an absolute constant $A$ such that

\[
||h^* ||_1\leq A\cdot ||h||_1
\]

\noindent
Hence the densely defined linear functional
\[
h\mapsto \int_{-\infty}^\infty\, \frac{P(x)}{B(x)}
\cdot h(x)\cdot dx
\] 
has norm $\leq C_\delta \cdot A$.
The Duality
Theorem from XXX  implies that if
$\epsilon>0$, then
there exists  some $G(z)\in\mathcal O(U)$
such that
the maximum norm
\[
\bigl |\frac{P(x)}{B(x)}-G(x)\bigr |_U<A\cdot C_\delta+\epsilon\tag{6}
\]

\noindent
Since $B(x)$ is a Blaschke product we have 
$|B(x)|=1$   almost everywhere and hence (6) gives:

\[
\bigl |P(x)-B(x)\cdot G(x)\,\bigr |<A\cdot C_\delta+\epsilon
\]

\medskip

\noindent
Now $f(z)=P(z)-B(z)G(z)$ is analytic in $U$
and  since 
$B(z_\nu)=0$ for every $\nu$ we have
\[
f(z_\nu)=P(z_\nu)=c_\nu
\]
So 
the bounded analytic function $f(z)$ interpolates and since
$\epsilon>0$ can be arbitrary small
and $c_1,\ldots,c_N$ was an arbitrary  sequence  in
${\bf{c}}_*$ we conclude that
the interpolation norm of the finite sequence $z_1,\ldots,z_N$ is at most
$A\cdot C_\delta$.  Since this uniform bound holds for all $N$
we get
\[
\mathfrak{int}(z_\bullet)\leq A\cdot C_\delta
\]
which  finishes
the proof of Theorem 0.1.
\bigskip

\bigskip


\noindent
{\bf {Exercise.}} Prove that (1) in the
Interpolation Theorem is necessary.

\bigskip

\newpage


\centerline{\bf  II. Wolff's Theorem.}

\bigskip


\noindent
{\bf Introduction.}
The Pompieu formula solves
the inhomogeneous $\bar\partial$-equation in the unit disc
$D$.
So if $h(z)$ is a $C^\infty$-function defined in some open
neighborhood of the closed disc there exists
a $C^\infty$-function $v$ such that
\[
\bar\partial(v)(z)=h(z)\quad\colon\quad z\in D\tag{*}
\]
We seek conditions 
in order that
(*) has a solution $v$ whose maximum norm over $D$
is controlled by extra properties of  $h$.
Such conditions  were imposed  in [Wolff].
To
every 
$C^\infty$-function $h$ on $\bar D$
we set
\[
\mu_h(z)=\log\,\frac{1}{|z|}\cdot
|\partial(h)(z)|\quad\colon\quad\nu_h(z)
=\text{Log}\,\frac{1}{|z|}\cdot|h(z)|^2\tag{**}
\]


\noindent
Before we announce Theorem 0.4  we recall the following.
\medskip


\noindent
{\bf{0.1 Carleson measures in $D$.}}
Consider  the family of
sector domains defined for all pairs
$0<h<1$ and $0\leq\theta\leq 2\pi$ by:
\[
S_h(\theta)=\{\,z=r\cdot e^{i\phi}\,\,\colon 1-h< r <1\,\colon
|e^{i\phi}-e^{i\theta}|\leq \frac{h}{2}\,\,\}
\]

\noindent
{\bf 0.2 Definition.}
\emph{A non-negative  measure $\mu$ in 
$D$ is called a Carleson measure if there
exists a constant $K$  such that}
\[
\iint_{S_h(\theta)}\, d\mu\leq K\cdot h
\quad\colon\quad 0< h> 1
\quad\colon\quad 0\leq\theta<2\pi 
\]
\emph{The least constant $K$ is denoted by
$\mathfrak{car}(\mu)$ and called the Carleson norm
of $\mu$.}
\medskip


\noindent
{\bf 0.3 Remark.} Using a conformal mapping between
$D$ and the upper half-plane one shows exactly as in � xx
that  there exists an absolute constant $A$
such that the following holds for every pair of a
Carleson measure $\mu$ in $D$ and 
a function $f(z)$ in the Hardy space $H^1(T)$
\[ 
\iint_D\, \bigl|f(z)\bigr|\cdot d\mu(z)\leq
A\cdot \mathfrak{car}(\mu)\cdot ||f||_1
\]

\medskip




\noindent
{\bf 0.4 Theorem.}
\emph{Let $A$ be as in (0.3).
For every $C^\infty$-function $h\in C^\infty(\bar D)$
the equation (*) has a $C^\infty$-solution
$v_*$ where}
\[ 
\max_\theta |v_*(e^{i\theta})|\leq 2\cdot A\cdot \mathfrak{car}(\mu_h)+2\cdot \sqrt{A\cdot 
 \mathfrak{car}(\nu_h)}
\]


\noindent
For the
proof
we need an   integral formula 
due to Jensen.
\medskip

\noindent
{\bf{0.5 The Fourier-Jensen formula.}}
\emph{Let $F(z)$ be an analytic function in $D$ 
with a simple zero at $z=0$ and otherwise it
is $\neq 0$. Then one has the equality:}



\[
 \iint_D\, \text{Log}\,\frac{1}{|z|}
\cdot\frac{|F'(z)|^2}{|F(z)|}\cdot dxdy=\int_0^{2\pi}\,
|F(e^{i\theta})|\cdot d\theta\tag{*}
\]
\medskip


\noindent
To prove (*) we set $F(z)=z\cdot G(z)$ where the hypothesis means that
$G$ is zero\vvv free so
we can construct a square
root function and  write
\[ 
F(z)=z\cdot \Psi^2(z)\quad\colon\,\,\Psi\in\mathcal O(D)\tag{i}
\]


\noindent
This implies that
\[
\frac{|F'(z)|^2}{|F(z)|}=\frac{|\Psi(z)+2z\cdot\Psi'(z)|^2}{|z|}
\]
Hence the left hand side in (*) becomes:


\[
 \iint_D\, \log \,\frac{1}{|z|}
\cdot\bigl |\Psi(z)+2z\cdot\Psi'(z)\bigr |^2\cdot\frac{1}{|z|}
\cdot dxdy\tag{ii}
\]
To evaluate this integral we consider the series expansion
$\Psi(z)=\sum\, a_nz^n$. In polar coordinates the double integral
becomes

\[
\int_0^1\int_0^{2\pi}\, \log \,\frac{1}{r}
\cdot\bigl |\sum\, (2n+1)\cdot a_n\cdot r^n\cdot e^{in\theta}\bigr|^2\cdot
drd\theta\tag{iii}
\]
\medskip

\noindent
{\bf{Exercise.}}
Show that (iii) is equal to
\[ 
2\pi\cdot \sum\,|a_n|^2=\int_0^{2\pi}\,|\Psi(e^{i\theta})|^2\cdot d\theta=
\int_0^{2\pi}\,|F(e^{i\theta})|\cdot d\theta
\]
which gives the requested equality (*).



\bigskip















\centerline{\bf 1. Proof of Theorem 0.4}
\medskip

\noindent
The Pompieu formula  gives
a solution $v$ to the $\bar\partial$-equation
\[ 
\bar\partial(v)=h\tag{i}
\]
We get new solutions to (i) by $v_*=v-G$ when
$G(z)$ are analytic functions in $D$.
So in order to minimize the maximum norm
of a solution to (i) we seek a  bounded analytic function
$G_*$ such that
\[
|v-G_*|_D=\min_G\,\,|v-G|_D\quad\colon\, G\in H^\infty(T)\tag{ii}
\]


\noindent
Let $m_*$
be the minimum value in (ii). 
To estimate $m_*$
we use  the duality between
$H^\infty(T)$
and 
$H_0^1(T)$ where $H_0^1(T)$
is the space of functions $F(z)$ in the Hardy space $H^1(T)$
for which $F(0)=0$.
Denote by
$S_*^(T)$ the set of functions
$F\in H_0^1(T)$ such that
\[
\int_0^{2\pi}\, \bigl|F(e^{i\theta})\bigr|\cdot d\theta=1\tag{1}
 \]


\noindent
The duality result from (XX) gives:
\[
m_*=\max_F\, \bigl|
\int_0^{2\pi}\, v(e^{i\theta})\cdot  F(e^{i\theta})\cdot d\theta\,\bigr|
\quad\colon\, F\in S_*(T)\tag{2}
\]
\medskip

\noindent
Since $F(0)=0$
Green's   formula shows that
(2) becomes
\[
\iint_D\,\text{Log}\,\frac{1}{|z|}\cdot\Delta(vF)\cdot dxdy\tag {3}
\]

\noindent
Since $\Delta=\partial\bar\partial$ and  $v$ solves
(i) while  $\bar\partial(F)=0$, we get
\[
\Delta(vF)=4\cdot\partial(hF)=4\cdot F\cdot\partial(h)+4\cdot h\cdot F'\tag{4}
\]


\noindent
Hence we have proved the following
\medskip

\noindent
{\bf 1. Lemma.} \emph{One has the equality}
\[
m_*=\max_F\,\,\bigl|
\iint_D\, \log\,\frac{1}{|z|}\cdot\,
\bigl[F\cdot\partial(h)+h\cdot F'\bigr]\cdot dxdy\bigr|
\quad\colon\quad F\in S_*^1(T)
\]
\medskip

\noindent
To profit upon this expression for $m_*$
we use the Jensen-Nevanlinna factorisation
and reduce
the
estimate to the
case when $F(z)$ has a simple zero at $z=0$ 
while it is $\neq 0$ in the punctured disc $D\setminus\{0\}$.
Thus, consider some $F$ in $S_*(T)$. Since $F(0)=0$ we 
there exists  the Jensen-Nevanlinna factorisation:
\[ 
F(z)=z\cdot B(z)\cdot G(z)\tag{i}
\]
where $B(z)$ is a Blaschke product and
$G$ has no zeros in $D$. Moreover, since
$|B|=1$ holds almost everywhere on $T$
it follows that $G$ belong to $S_*(T)$. Set:

\[ 
F_1(z)=\frac{z}{2}\bigl(B(z)-1\bigr)G(z)\quad\text{and}\quad
F_2(z)=\frac{z}{2}\bigl(B (z)+1\bigr)G(z)\tag{ii}
\]


\noindent
It follows that
$F=F_1+F_2$
and since the maximum norms of
$B(z)-1$ and $B(z)+1$ are at most 2
we have 
\[
||F_\nu||_1\leq 1\quad\colon\quad \nu=1,2\tag{iii}
\]


\noindent
From (ii) we see that $F_1$ and $F_2$ both have a simple zero at the origin
and are otherwise $\neq 0$ in the punctured disc.
Hence we can apply the Fourier-Jensen formula from (0.4)   which gives
\[
\bigl [\, \iint_D\, \text{Log}\,\frac{1}{|z|}
\cdot\frac{|F_\nu'(z)|^2}{|F_\nu(z)|}\, dxdy=\int_0^{2\pi}\,
|F_\nu(e^{i\theta})|\,d\theta\leq 1\quad\colon\,\nu=1,2 \tag{iv}
\]
\medskip


\noindent
\emph{Final part of the proof.}
For each $\nu=1,2$ we set
\[
V(F_\nu)= \iint_D\, \log\,\frac{1}{|z|}\cdot|\partial(h)|\cdot |F_1(z)|\,dxdy+
\iint_D\, \log\,\frac{1}{|z|}\cdot|h(z)|\cdot\,|F_1'(z)|\,dxdy
\tag{1}
\]
\medskip


\noindent
By the triangle inequality the right hand side in Lemma 1 is $\leq V(F_1)+V(F_2)$.
Let us for example  estimate  $V(F_1)$.
By the inequality (0.3) the first integral in (1) is estimated by
\[
A\cdot \mathfrak{car}(\text{Log}\,\frac{1}{|z|}\cdot|\partial(h)|)\cdot
||F_1||_1\tag{2}
\]


\noindent
Since $||F_1||_1\leq 1$ the definition of $\mu_h$ means that (2) is majorised by
\[
A\cdot \mathfrak{car}(\mu_h)\tag{*}
\]

\noindent
To estimate the second integral in (1) we
insert $\sqrt{|F_1|}$ as a factor 
and   Cauchy-Schwartz inequality majorises the 
second integral by the square root of




\[
\bigl [\, \iint_D\, \log\,\frac{1}{|z|}
\cdot\frac{|F_1'(z)|^2}{|F_1(z)|}\,dxdy
\bigr]
\cdot
\bigl[ \iint_D\, 
\log\,\frac{1}{|z|}\cdot\,|h(z)|^2\cdot |F_1(z)|\, dxdy\tag{3}
\]
\medskip

\noindent
In this product the first factor is given by  (iv)
and is therefore $\leq ||F\uuu 1||_1\leq 1$.
Finally,  the definition of the Carleson
norm
entails that the last factor is majorised by
$A\cdot\mathfrak{car}(\nu_h)$.
Taking the square root together with
(*) above we have proved that
\[ 
V(F_1)\leq+ A\cdot\mathfrak{car}(\mu_h)+\sqrt{A\cdot\mathfrak{car}(\nu_h)}\tag{4}
\]
The same holds for $F_2$ and  thanks to the
factor 2  the requested inequality in Wolff's theorem follows.

\bigskip


\centerline{\bf III. A class of Carleson measures}


\medskip

\noindent
Let $f(z)$ be a bounded analytic function
in $D$ and associate the non-negative measure in $D$ by:
\[
\mu_f=|f'(z)|^2\cdot\text{Log}\,\frac{1}{|z|}
\]



\medskip

\noindent {\bf {3.1 Theorem. }} \emph{There exists an absolute constant
$A_*$ such that}
\[
\sqrt{\mathfrak{car}(\mu_f)}\,\leq A_*\cdot |f|_D\quad\colon\quad
f\in H^\infty(D)
\]


\medskip


\noindent
\emph{Proof.}
To prove the a prioor inequality it suffices to treat the case when
$|f|_D=1$ and by a rotation it suffices to study integrals over sectors
of the form
\[
S_h= \{re^{i\theta}\colon\, 1-h/4<r<1\,\colon
\pi-h/4<\theta<\pi+h/4\}
\]
where $0<h<1/2$.
Consider the conformal
mapping from $\mathfrak{Im}\,\zeta>0$ to $D$ defined by

\[
z=\frac{\zeta-i}{\zeta+i}
\]
With $\zeta=\xi+i\eta$
it is easily seen that
$G^{-1}(S_h)$ is contained
in the rectangle
\[
\square=\{-h/2<\xi<h/2\}\times \{0<\eta<h/2\}
\]
Next ,  the
expansion
$\log(1-s)=-s+O(s^2)$ for small $s>0$
gives a constant $C$ which is independent of $h$
such that
\[
\log\,\frac{1}{|G(\xi,\eta)|}\leq C\cdot \eta
\quad\colon \xi+i\eta\in \square
\]
With $g(\zeta)= f(\frac{\zeta-i}{\zeta+i})$
we conclude that 
\[
\frac{1}{h}\cdot \iint _{S_h}\, |f'(z)|^2\cdot \log\,\frac{1}{|z|}\, dxdy
\leq \frac{C}{h}\cdot \int_\square\, \eta\cdot |g'(\zeta)|^2\, d\xi d\eta\tag{i}
\]




\noindent
Set  $g=u+iv$ which gives
$|g'(|^2=u_\xi^2+u_\eta^2$.
The right  hand side in (i) becomes:
\[
\frac{1}{h}\int_{\square_h}
\,\eta\cdot (u_\xi^2+u_\eta^2)
\cdot d\xi d\eta\tag{ii}
\]
Replace $\square_h$ by the larger semi-disc
\[
D_h=\{ \zeta=|xi+i\eta \quad\colon\, ||zeta|<h\quad\colon y>0\}
\]
Since $4h^2-|\zeta |^2\geq 3h^2$ when $\zeta \in D_h$ we get
a larger contribution by integrating over the larger
semi-disc $D_{2h}$ and it 
suffices to find an absolute constant $A$ such that
\[
\int_{D_{2h}}\, \eta(4h^2-|\zeta|^2)\cdot(u_\xi^2+u_\eta^2)\,d\xi d\eta\leq A\cdot h^3
\tag {*}
\]
\medskip

\noindent
To get A in (*) we
use the equality 
$\Delta(u^2)=2(u_x^2+u_y^2)$
and
since the  function $y(4h^2-|z|^2)$ is zero on the boundary
of $D_{2h}$, Green's formula shows that the integral in (*) is equal to
\[
\int_{D_{2h}}\, u^2\cdot\Delta(y(4h^2-|\zeta|^2) \,d\xi d\eta
-\int_{\partial D_{2h}}\, u^2\cdot\partial_{\bf{n}}(\eta(4h^2-|\zeta|^2)\, ds
\]
\medskip

\noindent
Notice that 
$\Delta(y(4h^2-|z|^2)=-8y<0$ in $D_{2h}$ and an easy computation gives
\[
-\int_{\partial D_{2h}}\, u^2\cdot\partial_{\bf{n}}(\eta(4h^2-||zeta|^2)\cdot ds=
\]
\[
\int_{-2h}^{2h}\, u^2(\xi,0)\cdot(4h^2-\xi^2)\,d\xi+
\int_0^\pi\, u^2(2he^{i\theta})\cdot \text{sin}\,\theta\cdot
\bigl[ -4h^2+ 3\cdot(2h)^2\bigr ]\cdot h\cdot d\theta
\]
\medskip

\noindent
Since the maximum norem of $u$ is $\leq 1$
we conclude that (*) is majorised by
\[
\int_{-2h}^{2h}\, (4h^2-\xi^2)\,d\xi+
\int_0^\pi\,  \text{sin}\,\theta\cdot
\bigl[ -4h^2+ 3\cdot(2h)^2\bigr ]\cdot h\,d\theta
\]
At this stage the reader can evaluate the requested constant $A$
which estimates(*)   by $A\cdot h^3$.



\bigskip




\centerline
{\bf IV. Berndtsson's  $\bar\partial$-estimate}
\bigskip

\noindent
Denote by $\mathcal B$
the family of bounded open sets
$\Omega$ in ${\bf{C}}$ for which there exists
a real-valued $C^2$-function
$\rho(z)$
defined in some
neighborhood of $\bar\Omega$ such that
the following  conditions hold:
\[ 
\Omega=\{\rho(z)<0\}\tag{1}
\] 
\[ 
\Delta(\rho)(z)>0
\,\,\colon\,\, z\in\Omega \quad\text{and}\quad
 \nabla(\rho)(z)\neq 0
\,\, z\in\partial\Omega\tag{2}
\]
\medskip

\noindent
{\bf{4.0 Defintion.}}\emph{
To each pair $\Omega\in\mathcal B$  and 
a bounded and subharmonic function
$\phi(z)$ in $\Omega$ a $C^\infty$-function $f$ in $\Omega$ is of the Berndtsson class
if}
\[ 
|f(z)|\leq -\rho(z)\cdot \Delta\,\phi(z)\quad\colon\quad z\in\Omega\tag{*}
\]
\emph{The family of these functions is denoted
by
$\mathfrak{Bernt}(\Omega,\phi)$}.
\medskip


\noindent
{\bf 4.1 Theorem.} \emph{To each $f\in\mathfrak{Bernt}(\Omega,\phi)$
the  equation}
\[ 
\bar\partial(u)=f\tag{*}
\] 
\emph{has a solution $u(z)$ which satisfies}
\[
\max_{z\in\partial \Omega}\, \frac{\bigl|u(z)|\cdot e^{-\phi(z)/2}\bigr|}{
|\nabla\rho(z)\bigr|}\leq
\max_{z\in\Omega}\, \frac{\bigl|f(z)|\cdot e^{-\phi(z)/2}\bigr|}{
|-\rho(z)\cdot \Delta\,\phi(z)+\nabla\rho(z)\bigr|}\tag{**}
\]
\medskip

\noindent
{\bf Remark.}
A special case occurs when
$\Omega=D$ and  $\rho(z)=|z|^2-1$. 
Then (*)   means that
a function $f$ in the Berndtsson class satisfies
\[
|f(z)|\leq (1-|z|^2)\cdot\Delta(\phi)(z)
\]

\noindent
So here $|f|$ decays as $|z|\to 1$
and when
$\Delta(\phi)$ is bounded this inequality estimates
the Carleson norm of $f$.
The reader s invited to analyze the
specific case when
\[ 
\phi(z)=\log\, 1+|z|^2\implies \Delta(\phi)=
\frac{4}{(1+x^2+y^2)^2}\geq 1\quad\colon\, z\in D
\]


\noindent
Theorem 4.1 is proved via    an
extremal problem in a Hilbert space which goes as follows:
Given the 
$\phi$-function one has the Hilbert space of 
functions $g$ in $D$ which are square integrable with
respect to $e^{-\phi}$, i.e. 
\[ 
\iint_D\, |g(z)|^2\cdot e^{-\phi(z)}\cdot dxdy<\infty\tag{1}
\]
\medskip

\noindent
In Berndtsson's article it is proved
that there exists  the unique extremal  solution $u$ to 
the equation $\bar\partial(u)=f$
whose
norm in $L^2(e^{-\phi})$
is minimal among all functions
$\psi$ in $D$ satisfying $\bar\partial\psi=f$ and
this
extremal solution 
$u$ satisfies the inequality (**) in
Theorem 4.1. 

\newpage


\centerline{\bf{V. H�rmander's  $L^2$-estimate}}

\bigskip

\noindent
Let $\Omega$ be an open  set in  ${\bf{C}}$.
If $\phi$ is a real-valued continuous and non-negative function 
we get the Hilbert space 
$\mathcal H_\phi$
whose elements are Lebesgue measurable functions $f$ in $\Omega$ such that
\[ 
\int_\Omega\, |f|^2\cdot e^{-\phi}\, dxdy<\infty
\]
The square root  yields norm and is  denoted by
$||f||_{\phi}$. Let $\psi$ be another
continuous and non-negative function 
which gives the Hilbert space 
$\mathcal H_\psi $ where the norm of an element $g$ is denoted by $||g||_{\psi}$.
We shall consider the
$\bar\partial$-operator
which sends
a function $f\in \mathcal H_\phi$  to
$\bar\partial(f)$
and want to solve the equation
\[
\bar\partial(f)= w\quad\colon w\in H_\psi\tag{5.0}
\]
where one seeks a constant $C$ such that
(5.0) has a solution $f$ with
\[
 ||f||_\phi\leq C\cdot ||w||_\psi
\]
To attain this we impose the following:

\medskip

\noindent
{\bf{5.1 H�rmander's condition.}}
The pair  $\phi,\psi$ satisfies the H�rmander condition if there exists some 
positive constant $c_0$ such that the following pointwise inequality holds in
$\Omega$:
\[ 
\Delta(\psi)\vvv 2\cdot |\nabla(\psi)|^2+
\psi\uuu x\phi\uuu x+\psi\uuu y\phi\uuu y\geq
2\cdot c_0^2\cdot e^{\psi(z)-\phi(z)}
\tag{*}
\]
where we have put $|\nabla(\psi)|^2=\psi\uuu x^2+\psi\uuu y^2$.
\medskip


\noindent
{\bf{5.2 Theorem.}}
\emph{If $(\phi,\psi)$ satisfies (*) for some $c_0>0$ then
the equation
$\bar\partial(f)=w$ has a solution for every $w\in\mathcal H_\psi$ where}
\[
||f||_\phi\leq \frac{1}{c_0}\cdot ||w||_\psi
\]


\noindent
\emph{Proof.}
Since $C_0^\infty(\Omega)$ is a dense subspace of $\mathcal H_\phi $
the linear operator $T\colon f\mapsto \bar\partial(f)$ 
from $\mathcal H_\phi$ to $\mathcal H_\psi$  is densely defined.
Let $w$ be in the domain of definition for
the adjoint operator 
$T^*$. If $f\in C_0^\infty(\Omega)$. Stokes theorem gives
\[ 
\langle T(f),w\rangle=\int \bar\partial(f)\cdot \bar w\cdot e^{-\psi}\,dxdy=
-\int\, f\cdot 
\bigl[\bar\partial(\bar w)-\bar w\cdot \bar\partial(\psi)\bigr]\cdot
 e^{-\psi}\,dxdy\tag{i}
\]


\noindent
Since $\psi$ is real\vvv valued, 
$\bar\partial(\bar w)-\bar w\cdot \bar\partial(\psi)$
is the complex conjugate of
$\partial(w)-w\cdot\partial(\psi)$ and (i) gives
\[
T^*(w)=-\bigl[\partial(w)-w\cdot\partial(\psi)\bigr]\cdot
 e^{\phi-\psi}\tag{ii}
\]
Taking the squared $L^2$-norm in $\mathcal H_\phi$
we obtain
\[ 
||T^*(w)||^2_\phi=
\int\, |\partial(w)-w\cdot\partial(\psi)|^2\cdot
e^{\phi-2\psi}=
\]
\[
\int\, \bigl(|\partial(w)|^2+|w|^2\cdot|\partial(\psi)|^2\bigr)\cdot
e^{\phi-2\psi}-2\cdot\mathfrak{Re}\bigl (
\int\, \partial(w)\cdot\bar w\cdot\bar \partial(\psi)\cdot
e^{\phi-2\psi}\bigr)\tag{iii}
\]
By partial integration
the last integral in (iii) is equal
\[
2\cdot\mathfrak{Re}\bigl(\,  \int\,w\cdot[\partial(\bar w)\cdot\bar \partial(\psi)+
\bar w\cdot \partial\bar\partial(\psi)\vvv
2\bar w\cdot \bar \partial(\psi)\cdot \partial (\psi)+
\bar w\cdot \bar\partial(\psi)\cdot \partial(\phi)]\cdot e^{\phi\vvv 2\psi}\,\bigr)\tag{iv}
\]


\noindent
Next, the Cauchy\vvv Schwarz inequality gives
\[
\bigl|2\cdot 
\int \, w\cdot  \partial(\bar w)\cdot \bar \partial(\psi)\cdot e^{\phi\vvv 2\psi}\,\bigr|\leq
\int\, \bigl(|\partial(w)|^2+|w|^2\cdot|\partial(\psi)|^2\bigr)\cdot
e^{\phi-2\psi}\tag{v}
\] 
Together (iii-v) give
\[
||T^*(w)||^2_\phi\geq 
2\cdot \mathfrak{Re}\,  \int\, |w|^2\cdot \bigl[\,\partial\bar\partial(\psi)\vvv
2\cdot \bar \partial(\psi)\cdot \partial (\psi)+
\bar\partial(\psi)\cdot \partial(\phi)\,\bigr ]\cdot e^{\phi\vvv 2\psi}=
\]
\[
2\cdot \mathfrak{Re}\,  \int\, |w|^2\cdot
\frac{1}{4}\bigl[\Delta(\psi)\vvv 2\cdot |\nabla(\psi)|^2+
\psi\uuu x\phi\uuu x+\psi\uuu y\phi\uuu y\,\bigr]\, \cdot e^{\phi\vvv 2\psi}
\tag{vi}
\]
where the last equality follows since
$\phi$ and $\psi$ are real-valued.
When (*) holds in (5.1) it follows that
\[
||T^*(w)||^2_\phi\geq c\uuu 0^2\cdot \int\, |w|^2\cdot
e^{\psi\vvv \phi}\cdot e^{\phi\vvv 2\psi}=
c\uuu 0^2\cdot ||w||^2\uuu\psi\tag{vi}
\]
This lower bound implies that the norm of $T$ is bounded by $c_0$
and Theorem 5.2 follows.




\bigskip

\noindent {\bf{5.3 Remark.}}
We refer to the article [Kiselman] by C. Kiselman for
some specific applications of  $L^2$-estimates in
${\bf{C}}$ applied to   carriers of Borel transforms.
The full strength of  $L^2$-estimate
appears in dimension $n\geq 2$ where one works with
\emph{plurisubharmonic functions}  and 
impose the condition that $\Omega$ is a
strictly pesudo-convex set in ${\bf{C}}^n$ and
solve inhomogeneous $\bar\partial$-equations for 
differential forms of bi-degree $(p,q)$. 
In addition to  H�rmander's original article [H�rmander] we refer to his text-book
[H�rmander] and  Chapter XX in [H�mander XX-Vol 2]
where   bounds for $\bar\partial$-equations are 
established with certain  relaxed assumptions which are used 
to settle the fundamental principle 
for over-determined systems of PDE-equations in the smooth case.
Working on complex manifolds 
H�rmander's $L^2$-estimates give very 
powerful tools. For a wealth of results of this nature we refer to the notes by 
Demailly
in [Dem].





\centerline{\bf{The case $n=2$}}

\bigskip

\noindent
The special case below  may help the reader to
pursue details from page xx-xx in 
H�rmander's text book (xx edition).
See also 
H�rmander's article [Acta: 1962].
Let $D^2$ be the 2-dimensional polydisc in ${\bf{C}}^2$ with
coordinates $z=(z_1,z_2)$ where $\bar\partial_1$ and $\bar\partial_2$
are pairwise commuting operators. Let 
$\phi(z)$  be a real-valued function in $D^2$
which is at least of class $C^2$.
We get the
Hilbert space
$\mathcal H$
of locally square integrab�e functions with
finite norm:
\[
||a||_\phi =\sqrt{\int\, |a(z)|^2\cdot e^{-\phi(z)}\,d\lambda(z)}
\]
where $d\lambda(z)$ is the 4-dimensional Lebesgue meaure and the integral is taken
over $D^2$.
We have the  densely defined
linear operator $T$ from
$\mathcal H$ into $\mathcal H\oplus\mathcal H$ defined by
\[ 
T(a)=\bar\partial_1(a)\oplus\,\bar\partial_2(a)
\]

\noindent
{\bf{A. Exercise.}}
Let
$T^*$ be the adjoint of $T$ which sends a pair $(f,g)\in\mathcal H\oplus \mathcal H$
to $\mathcal H$. Show that 
\[
T^*(f,g)=
-(\partial_1(f)-f\cdot\partial_1(\phi)+\partial_1(g)-g\cdot\partial_1(\phi)]\tag{A}
\]
\medskip

\noindent
{\bf{B. Exercise.}}
Put
\[
 \delta_1(f)= \partial_1(f)-f\cdot \partial_1(\phi)\quad\colon\quad 
  \delta_2(g)= \partial_2(g)-g\cdot \partial_2(\phi)
 \]
Use (A) to
show that
\[
||T^*(f,g)||^2=
||\delta_1(f)||^2+||\delta_2(g)||^2+2\cdot \mathfrak{Re}\,
\int\, \delta_1(f)\cdot \overline{ \delta_2(g)}\cdot e^{-\phi}\, d\lambda\tag{B.1}
\]
Next, use Stokes theorem to show that
\[
\int\, \delta_1(f)\cdot  \overline{ \delta_2(g)}
\cdot e^{-\phi}\, d\lambda
=
- \int f\cdot\overline{\partial_1(\delta_2(g))}\cdot
e^{-\phi}\,d\lambda\tag{B.2}
\]
\medskip

\noindent
{\bf{C. Exercise.}}
Put
\[ 
H(z)=
\frac{\partial^2\phi}{\partial z_1\bar\partial z_2}\tag{C.0}
\]
and by multiplication one identifies $H$ with a zero-order
differential operator. Show
the following
equality  in  the ring of differential  operators in
${\bf{C}}^2$:
\[ 
\partial_1\circ \delta_2=\delta_2\circ\partial_1-H\tag{C.1}
\]
Conclude that
(B.2) becomes
\[
\int\, f\cdot \bar g\cdot \bar H\cdot e^{-\phi}\,d\lambda
- \int f\cdot\overline{\delta_2\partial_1((g))}\cdot
e^{-\phi}\,d\lambda\tag{C.2}
\]
\medskip


\noindent
{\bf{D. The case $\bar\partial_1(g)= \bar\partial_2(f)$}}.
Use the above to show that this equality gives
\[
-2\cdot \mathfrak{Re}\, \int \bar f\cdot\delta_2(\bar\partial_2((f))\cdot
e^{-\phi}\,d\lambda=
-2\cdot \mathfrak{Re}\, \int\, |\partial_2(f)|^2\cdot e^{-\phi}\,d\lambda\tag{D.1}
\]
\medskip

\noindent
{\bf{E. Conclusion.}}
Show that (A), (B.1-2) and (D.1) and  the equality
$\bar\partial_1(g)= \bar\partial_1(f)$
give:
\[
||T^*(f,g)||^2=
||\delta_1(f)||^2+||\delta_2(g)||^2+2\cdot \mathfrak{Re}\,
\int\, f\cdot\bar  g\cdot H(z)\cdot e^{-\phi}\, d\lambda+
||\partial_2(f)||^2\tag{E.1}
\]


\noindent
Above the last term is always $\geq 0$.
To ensure that there exists a constant $c_0$ such that
\[
||f||^2+||g||^2\leq c_0^2\cdot ||T^*(f,g)||^2\tag{E.2}
\]

\noindent
one must impose  suitable conditions upon $\phi$.
Above the mixed derivative which defines
the
$H$-function appears while 
norms $||\delta_1(f)||^2$ and $||\delta_2(g)||^2$
can be estimated as in the case $n=1$ applied 
with  $\phi=\psi$. The reader is invited to contemplate upon conditions on
$\phi$ which give a constant $c_0$ in (E.2)
and at this stage H�rmander's work can be consulted for examples and
further dertails. Notice that above we treated a  special case
since we used the same weight  function $\phi$ and not a pair as in
the case $n=1$.


 













\newpage


\centerline{\bf VI. The Corona Theorem.}

\bigskip
\noindent
{\bf Introduction.} In  the unit disc $D$ we 
have the Banach algebra $H^\infty(D)$ of bounded analytic functions.
Let $\mathfrak{M}_\infty(D)$ denote its maximal ideal space.
Via point evaluations in $D$ we get a map

\[ 
i\colon D \mapsto \mathfrak{M}_\infty(D)
\]
The Corona problem asked if $i(D)$ is dense in
$\mathfrak{M}_\infty(D)$. The affirmative answer
to this question was found by Carleson.
It is easily seen that the density of the $i$-image is equivalent to the following
result which was proved in [Carleson]:

\medskip

\noindent{\bf{6.1 Theorem.}}
\emph{The ideal generated
by
a finite family $f_1,\ldots,f_n$ in $H^\infty(D)$ is equal to $H^\infty(D)$
if and only if there exists $\delta>0$ such that}


\[
|f_1(z)|+\ldots+|f_n(z)|\geq\delta\quad\text{hold for all}\,\, \, z\in D
\]

\bigskip

\noindent
{\bf Remark.}
Just as  in the  Interpolation Theorem an
essential ingredient of the proof
relies upon 
Carleson measures. 
An alternative to Carleson's original proof was
given by Wolff and relies upon
his result for the inhomogeneous $\bar\partial$-equation 
from XX.
The deduction from Wolff's Theorem  in XX to the solution
of the Corona problem
is  exposed a several places. See for example
Chapter XX in [Narasimhan] and  the article [Gamelin]
by T. Gamelin.
So  we refrain from giving further details.
Let us  remark that one may consider
related problems where the boundedness of the $f$-function is relaxed.
For example, using $L^2$-estimates with weight functions
one can solve  a  problem where
$f_1,\ldots,f_n$ are analytic functions in $D$ with
moderate growth,  i.e 
there is an integer $m$ and a constant $A$ such that
\[ 
(1-|z|)^m\cdot |f_\nu(z)|\leq K
\] 
holds in $D$ for every $\nu$.
Assume also that there is an integer $m_*$ and  $\delta>0$
such that
\[
|f_1(z)|+\ldots+|f_n(z)|\geq\delta\cdot (1-|z|)^{m_*}
\quad\text{hold for all}\,\, \, z\in D
\]
Then one can show that
there exists an $n$-tuple $g_1,\ldots g_n$ of analytic functions with
moderate growth such that
\[ 
g_1\cdot f_1+\ldots+g_n\cdot f_n=1\tag{1}
\] 
holds in $D$.
\medskip

\noindent{\bf{Question.}}
Find the smallest possible number $\rho=\rho(m,m_*)$ such that 
(1) has a solution where the $g$-functions satisfy

\[
|g_\nu(z(|\leq C\cdot (1-|z|)^{-\rho}
\]
for some constant $C$.
It appears that
the best constant $\rho$ is not known.
However,  upper bounds for $\rho$ can be established using
$L^2$-estimates for the $\bar\partial$-equation.
But there remains
to investigate how sharp such bounds are.


\end{document}







