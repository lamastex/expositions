%\documentclass{amsart}
%\usepackage[applemac]{inputenc}

%\def\uuu{_}


%\def\vvv{-}

%\begin{document}


\centerline{\bf\large{Special Topics}}











\bigskip


\bigskip

\noindent
1. \emph{The disc algebra}


\bigskip


\noindent
\emph{2. The Jensen-Nevanlinna class and Blaschke products.}

\bigskip

\noindent
\emph{3. The Hardy space $H^1(T)$}
\bigskip

\noindent
\emph{4. Nevanlinna\vvv Pick theory}

\bigskip


\noindent
\emph{5. Series and analytic functions.} 
\bigskip

\noindent
\emph{6. Uniqueness theorems for analytic functions.}


\bigskip

\noindent
\emph{7. Lindel�f functions.}

\bigskip




\noindent
\emph{8. Approximation theorems in complex domains}
\bigskip

\noindent
\emph{9. Radial limit of functions with  finite Dirichlet integral}

\bigskip

\noindent
\emph{10. The Denjoy conjecture and Carleman's differential inequality}
\bigskip

\noindent
\emph{11. The Dagerholm series}

\bigskip

\noindent
\emph{12. Uniform approximation by Blaschke products} 


\bigskip

\noindent
\emph{13. Interpolation and solutions to the $\bar\partial$-equation.}

\bigskip

\noindent
\emph{14. Entire functions of exponential type}
\bigskip





\noindent
\emph{15. Beurling-Wiener  algebras}
\bigskip

\noindent
\emph{16. The Robin constant and  harmonic measures}

\bigskip

\noindent
\emph{17. An automorphism of product measures}

\bigskip

\noindent
\emph{18. The Mellin expansion and the Radon transform}

\bigskip



\noindent
\emph{19. A non\vvv linear PDE\vvv equation}

\bigskip


\noindent
\emph{20. An isoperimetric problem}

\bigskip
\noindent
\emph{20. Doubly periodic meromorphic functions.}
\bigskip



\centerline{\bf{Introduction.}}

\bigskip

\noindent
Above are headlines for the sections.
There detailed contents are listed in the next pages
and further comments appear in the individual
sections.
The level of the material changes from fairly elementary
facts to results whose proofs are quite demanding.
The first three sections are essential for much of  the subsequent material, especially
facts about functions in the Jensen\vvv Nevanlinna class which together with
the Brothers Riesz Theorem from Section 1 are used to 
study
Hardy spaces an they appear frequently in later sections as well.
The topics treat
both analytic function theory and
harmonic analysis where
the interplay lead often to a powerful  theory such as
in Section 14 where Beurling\vvv Wiener algebras
are studied.
A veritable high\vvv light is the notion and properties of
Carleson
measures which are to establish
the interpolation theorem for bounded
analytic functions in section 12.
Section 19\vvv 20 are a bit apart from
analytic function theory but have been included since
the methods are of interest. For example, symmetrizations is often
used in analytic function theory and the proof in Section 20
illustrates one application of symmetrization methods
and
the
non\vvv linear solution to a PDE\vvv equation in section 19
is proved via succesive analytic series which eventually reduce
the proof of existence to solving a family of linear Neumann problems.

\newpage


\centerline{\emph{Guidance to the sections.}}

\bigskip

\noindent Below  we 
have included a more detailed list of material from some of
the extensive sections. For the shorter sections we refer to
the individual introductions.

\bigskip



\noindent{\bf{I. The disc algebra $A(D)$}}


\bigskip

\noindent
\emph{1. Theorem of Brothers Riesz.}


\bigskip

\noindent
\emph{2. Ideals in the disc algebra}
\bigskip

\noindent
\emph{3. A maximality theorem for uniform algebras.}
\bigskip

\noindent{\bf{ 2. The Jensen-Nevanlinna class and Blaschke products.}}

\bigskip


\noindent
\emph{1. The Herglotz integral.}
\bigskip


\noindent
\emph{2. The class JN(D)}

\bigskip


\noindent
\emph{3. Blaschke products}


\bigskip


\noindent
\emph{4. Invariant subspaces of $H^2(T)$}
\bigskip


\noindent
\emph{5.  Beurling's closure theorem}
\bigskip


\noindent
\emph{6. The Helson\vvv Szeg� theorem.}

\bigskip


\noindent{\bf {3.A: The Hardy\vvv Littlewood maximal function}}
\bigskip


\noindent
\emph{1. The weak type estimate}


\bigskip

\noindent
\emph{2.  An $L^2$-inequality}

\bigskip

\noindent
\emph{3 Harmonic functions  and Fatou sectors}



\bigskip

\noindent
\emph{4. Application to analytic functions}



\bigskip

\noindent
\emph { 5. Conformal maps and the Hardy space
$H^1(T)$}

\bigskip


\noindent{\bf {3:B. The Hardy space $H^1$}}



\bigskip




\noindent
\emph{1. Zygmund's inequality}
\bigskip


\noindent
\emph{2. A weak type estimate.}
\bigskip


\noindent
\emph{3. Kolmogorov's inequality.}
\bigskip


\noindent
\emph{4. The dual space of $H^1(T)$}

\bigskip


\noindent
\emph{5. The class BMO}
\bigskip




\noindent
\emph{6. The dual of $\mathfrak{Re}\, H^1\uuu 0(T)$}
\bigskip




\noindent
\emph{7. Theorem of Gundy and Silver}
\bigskip

\noindent
\emph{8. The Hardy space on ${\bf{R}}$.}

\bigskip

\noindent
\emph{9. BMO and radial norms on measures in $D$}.


\bigskip
\noindent {\bf{4. Nevanlinna-Pick theory}}

\bigskip

\noindent
\emph{0. The Nevanlinna-Pick Interpolation}

\bigskip

\noindent
\emph{1. The Lindel�f-Pick principle
with an application}

\bigskip

\noindent
\emph{2. A result by Julia}
\bigskip

\noindent
\emph{3. Geometric results by L�wner}
\bigskip



\noindent{\bf{5. Series and analytic functions.}}

\bigskip



\noindent
\emph{1. A theorem by Kronecker.}
\bigskip


\noindent
\emph{2. Newton polynomials and the disc algebra.}

\bigskip
\noindent

\noindent
\emph{3. Absolutely convergent Fourier series.}

\bigskip



\noindent
\emph{4. Harald Bohr's inequality}
\bigskip


\noindent
\emph{5. Theorem of Fatou and M. Riesz}


\bigskip

\noindent
\emph{6. On Laplace transforms}
\bigskip


\noindent
\emph{7. The Kepler equation and Lagrange series}

\bigskip

\noindent
\emph{8. An example by Bernstein}


\bigskip

\noindent
\emph{9. Almost periodic functions and additive number theory}



\bigskip


\noindent{\bf{5. Uniqueness theorems for analytic functions.}}

\bigskip


\noindent
\emph{A. A sharp version of the Phragm�n-Lindel�f theorem}

\bigskip

\noindent
\emph{B. Asymptotic series.}

\bigskip

\noindent
\emph {C. A uniqueness theorem for asymptotic series}

\bigskip









\bigskip





\noindent{\bf 8. Approximation theorems in complex domains}
\bigskip



\noindent
\emph{A. Weierstrass approximation theorem}\bigskip


\noindent
\emph{B. Polynomial approximation with bounds}

\bigskip


\noindent
\emph{C. Approximation by fractional powers}


\bigskip


\noindent
\emph{D. Theorem of M�ntz}

\bigskip



\noindent{\bf{13. Interpolation and solutions to the $\bar\partial$-equation.}}


\bigskip


\noindent
\emph{1. Carleson's interpolation theorem}
\bigskip

\noindent
\emph{2. Wolff's theorem}


\bigskip

\noindent
\emph{3. A class of Carleson measures.}


\bigskip

\noindent
\emph{4. Berndtsson's $\bar\partial$-solution}

\bigskip

\noindent
\emph{5. H�rmander's $L^2$-estimate}
\bigskip

\noindent
\emph{6. The Corona problem.}


\bigskip








\noindent{\bf {15. Beurling-Wiener  algebras}}

\bigskip

\noindent
\emph{A: Beurling-Wiener algebras on the real line.}

\medskip

\noindent
\emph{B: A Tauberian theorem}


\medskip

\noindent
\emph{C: Ikehara's theorem}

\medskip

\noindent
\emph{D: The Gelfand space of $L^1({\bf{R}}^+)$.}

\bigskip


\noindent{\bf{19. Homogeneous distributions and the Mellin transform}}

\bigskip


\noindent
\emph {A. Polar distributions}
\bigskip

\noindent
\emph {B. Homogeneous distributions}
\bigskip

\noindent
\emph {C. The family $|P(x,y)|^\lambda$}
\bigskip

\noindent
\emph{D. The Radon transform}
\bigskip

\noindent
\emph{E. The Mellin transform}
\bigskip


\newpage


\newpage











































%\end{document}