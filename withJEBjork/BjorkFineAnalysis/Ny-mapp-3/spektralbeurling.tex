

\documentclass{amsart}


\usepackage[applemac]{inputenc}

\addtolength{\hoffset}{-12mm}
\addtolength{\textwidth}{22mm}
\addtolength{\voffset}{-10mm}
\addtolength{\textheight}{20mm}

\def\uuu{_}


\def\vvv{-}

\begin{document}

\centerline{\bf{A spectral theorem for continuous functions.}}

\bigskip

\noindent
On the real $\xi$\vvv line
we denote by $C\uuu *$ the set of bounded and uniformly continuous functions
$\phi(\xi)$. So there exists a constant $m$ such that
$|\phi(\xi)\leq m$ for al $\xi$ and the uniform continuity means that if we put
\[
\omega\uuu\phi(\delta)=
\max\uuu \xi\,|\phi(\xi+\delta)\vvv \phi(|xi)|
\]
then $\omega\uuu \phi(|delta)\to 0$ as $\delta \to  0$.
When $\phi\in C\uuu *$
we get the linear space $\mathcal T\uuu\phi$
which consist of functions of the form
\[
\sum\, c\uuu k\cdot \phi(\xi+\tau\uuu k)\tag{1}
\]
where $\{\tau\uuu k\}$ is a finite set of real numbers and
$\{c\uuu k\}$ a finite tuple of complex numbers.

Following [Beurling] a function a function $f(x)$
in the class $C\uuu *$
 belongs to the 
\emph{tight closure} of $\mathcal T\uuu\phi$
if there exists a sequence
$\{f\uuu k\}$ in
$\mathcal T\uuu\phi$ such that
$f\uuu k(\xi )\to f(\xi)$ holds uniformly on every bounded
interval $\vvv A\leq \xi\leq A$ and moreover the maximum norms 
taken over the whole $\xi$\vvv line satisfy

\[ 
||f\uuu||\leq ||f||\tag{2}
\]
\medskip

\noindent
{\bf{The spectrum of $\phi$}}.
Given $\phi\in C\uuu *$ its spectrum is defined as the set of real numbers
$\lambda$ such that the exponential function $e^{i\xi\lambda}$
belongs to the tight closure of $\mathcal T\uuu\phi$.
The major result in [Beurling] asserts that the spectrum always in
$\neq\emptyset$ unless $\phi$ is identically zero.
\bigskip

\centerline{\emph{Proof that $\sigma(\phi)\neq \emptyset$}}
\medskip

\noindent
First we take some even test\vvv function $H(x)$ on the real $x$\vvv line
where $H(0)=\frac{1}{2\pi}$. With
$\zeta=\xi+i\eta$ we get
the entire function

\[ 
h(\zeta)= \int\, e^{i\zeta x}\cdot H(x)\cdot dx\tag{i}
\]
Since $H$ is a test\vvv function we know from XX
that
the functions
\[ 
\xi\mapsto h(\xi+i\beta)
\]
are rapidly decreasing for every $\beta$,  i.e. they
decrease faster than any
negative power of $|\xi|$. Moreover, Fourier's inversion formula gives
in particular
\[
\frac{1}{2\pi}= H(0)= \frac{1}{2\pi}\cdot \int h(\xi)\cdot d\xi
\]
Hence we have
\[
 \int h(\xi)\cdot d\xi=1\tag{ii}
\]

\noindent
Next, if $H(x)$ is supported by an interval $[\vvv c,c]$
the result in XXX gives  a constant
$C$ such that
\[
|h(\xi+i\eta)|\leq  C\cdot \frac{e^{c|\eta|}}{1+\xi^2+\eta^2}\tag{iii}
\]
for all $\xi+i\eta$

\noindent
Next, for each complex number $\alpha+i\beta$ we get a function of
$\xi$ defined by
\[ 
\xi\mapsto =\int\uuu{\vvv \infty}^\infty\, 
\phi(\xi\vvv s)\cdot h(s+\alpha+i\beta)\cdot ds\tag{iv}
\]

\noindent
{\bf{Exercise.}}
Show that the function in (iv) belongs to the tight closure of
$\mathcal T\uuu \phi$.
The hint is to use (iii) and the uniform continuity of $\phi$.
\medskip

\noindent
Next,  consider the function
\[
 \psi(\xi+i\eta)=\int\uuu{\vvv \infty}^\infty\, 
(\xi+i\eta\vvv s)\cdot \phi(s)\cdot ds
\]
It is easily seen that $\psi(\zeta)$ is an entire function
and using (iii) there is a constant $C$ such that
\[
|\psi(\xi+i\eta)|\leq C\cdot e^{c|\eta|}
\]
\medskip

\noindent
{\bf{Remark.}}
When $\psi$ is restricted to the real $|xi$\vvv line it is the convolution
of $h$ and $\phi$. Now $\phi(\xi)$ defines a temperate density
function and is therefore equal to the Fourier transform of a temperate
distribution $\mu$ on the real $x$\vvv line.
From the calculus with tempered distributions in XXX this means that
$\psi(\xi)$ up to a constant is the Fourier transform of
the distribution $H(x)\cdot \mu$ which has compact support on the
$x$\vvv line. This clarifies why
$\psi(\zeta)$ is an entire Fourier\vvv Laplace transform and also the
growth condition xx above.
\medskip

\noindent
At this stage we announce the following:

\medskip

\noindent
{\bf{Proposition.}}
\emph{ There exists a real number $a$ and a sequence of complex numbers
$\{alpha\uuu n+i\beta\uuu n\}$
such that the functions}

\[f\uuu n(\xi)=
e^{i\alpha\xi}\cdot \frac{\psi(\xi+\alpha\uuu n+i\beta\uuu n)}
{|\psi(\alpha\uuu n+i\beta\uuu n)|}
\] 
\emph{converge uniformly to 1 on the real $\xi$\vvv axis
and at the same time
the maximum norms
$||f\uuu n||$ converge to one. which implies that
$e^{ia\xi}$ belongs to the tight closure of $\mathcal T\uuu \phi$.}
\bigskip

\noindent
\emph{Proof.}
Introduce the function
\[
\mu\uuu f(\eta)=\max\uuu\xi\, \log\,|\psi(\xi+i\eta)|
\]
By (XX) the functions $\xi\to |\psi(\xi+i\eta)|$
are bounded in every interval�$\vvv /rho\leq y\leq \rho$ so the $\mu$\vvv
function is defined on the whole $\eta$\vvv line.
By the result in XX the $\mu$\vvv function is convex
and hence the derivative $\mu'(\eta)$ is non\vvv decreasing and the inequality
(xx) entails that
\[ 
\mu'(\eta)\leq c
\]
hold for every $\eta$.
As a consequence there exist the two limits

\[
a=\lim\uuu{\eta\to +\infty}\, \mu(\eta)\quad \colon\quad
b=\lim\uuu{\eta\to \vvv \infty}\,\mu(\eta)
\]
where $b\leq a$.
Using these limits we will show that a sequence $\{f\uuu n\}$
exists where we can take $a$ to be the first term  in (xx) above.

\bigskip

PROOF rather easy now as exposed on page 65 in Collected.

\newpage


Measures. $\{\mu\uuu n\}$ converge to limit $\mu$
meaning pointwise convergence of Fourier  weak type ...
Many ways to reach $\mu$.
Question when $\phi$ good in that sense. To entail convergence.
Meaning $\mu\uuu n\to 0$ in weak sens implies that integrals to zero.
True iff $\phi$ uniformly inverse transofrm.








\end{document}

