%\documentclass{amsart}


%\usepackage[applemac]{inputenc}

%\addtolength{\hoffset}{-12mm}
%\addtolength{\textwidth}{22mm}
%\addtolength{\voffset}{-10mm}
%\addtolength{\textheight}{20mm}

%\def\uuu{_}


%\def\vvv{-}


%\begin{document}

\centerline{\bf\large{20. A Non-Linear PDE-equation}}

\bigskip

\noindent
{\bf{Introduction.}}
We expose  Carleman's article
\emph{�ber eine nichtlineare Randwertaufgabe bei der Gleichung $\Delta u=0$}
(Mathematisches Zeitschrift vol. 9 (1921). 
Here is the equation to be considered: Let
$\Omega$ be a bounded domain in ${\bf{R}}^3$
with $C^1$-boundary and
${\bf{R}}^+$ the non-negative real line where  $u$ is the coordinate.
Let $F(u,p)$
be a real-valued and continuous function
defined on  ${\bf{R}}^+\times\partial\Omega$.
Assume that 
\[
u\mapsto F(u),p)\tag{0,1}
\]
is strictly increasing  for every $p\in\partial\Omega$
and that $F(0,p)\geq 0$. Moreover,  
\[
\lim_{u\to\infty} F(u,p)=+\infty\tag{0.2}
\] 
holds uniformly with respect to $p$.
For a given point $Q_*\in\Omega$ we seek a function $u(x)$
which is harmonic in $\Omega\setminus\{Q_*\}$
and at $Q_*$ it is locally $\frac{1}{|x-Q_*|}$ plus a harmonic function
and on $\partial\Omega$
the  inner normal derivative $\partial u/\partial n$ satisfies the 
equation
\[
\frac{\partial u}{\partial n}(p)= F(u(p),p)\quad \colon p\in\partial\Omega\tag{*}
\]

\medskip

\noindent 
Finally it is also assumed that $u$
extends to a continuous function on
$\partial\Omega$.
\medskip

\noindent {\bf {Theorem.}}
\emph{For each $F$ as above the boundary value problem has a unique solution.}
\medskip

\noindent
{\bf{Remark.}}
Apart from the general result the subsequent proof is interesting since
it teaches how to
handle a class of non\vvv linear
boundary value problems.
The strategy in Carleman's proof is to consider
the family of boundary value problems where we for each
$0\leq h\leq 1$ seek $u\uuu h$ to satisfy

\[
\frac{\partial u\uuu h}{\partial n}(p)=(1\vvv h)u\uuu h+ h\cdot  F(u\uuu h(p),p)\quad \colon p\in\partial\Omega\tag{*}
\]
where $u\uuu h$ has the same pole as $u$ above.
Starting with $h=0$ one has a classical linear Neumann problem
where a unique solution exists.
To proceed from $h=0$ to $h=1$
the idea is to use a "homotopy argument" where
one first easily reduces the proof to the case when $F$ is a real\vvv analytic function
of $u$. Then  the subsequent proof will show that if
we have found a solution $u\uuu {h\uuu 0}$ for some
$0\leq h\uuu 0<1$, we obtain solutions
$u\uuu h$ when $h\uuu 0<h<h\uuu 0+\epsilon$ for sufficiently 
small $\epsilon$ by solving an infinite system of linear boundary value problems.
It goes without saying that this method is restricted to favorable cases
as above where the uniqueness and robust properties of these solutions
are present as we show below.
But it is instructive to see how one can employ
analytic series to handle such cases.
\medskip

\noindent
Now we turn to the proof of the theorem and begin
with preliminary results concerned with uniqueness and
the reduction to the case when $F$ is real\vvv analytic.



\medskip

\noindent
{\bf{A.0. Proof of uniqueness.}}
Suppose that $u_1$ and $u_2$ are two solutions and 
notice that  $u_2-u_1$.
is harmonic in
$\Omega$.
If $u\uuu 1\neq u\uuu 2$ we may
without loss of generality we may assume that
the maximum of $u\uuu 2\vvv u\uuu 1$ is $>0$.
The maximum is attained at some $p\uuu *\in\partial\Omega$
and by the strict maximum principle for harmonic functions we have

\[
u\uuu 2(x)\vvv u\uuu 1(x)<
u\uuu 2(p\uuu *)\vvv u\uuu 1(p\uuu *)\tag{i}
\] 
for all $x\in\Omega$.With $v=u\uuu 2\vvv u\uuu 1$
we have
\[
\frac{\partial v}{\partial n}(p)=F(u_2(p),p)-F(u_1(p),p)
\]
Here (0.1) entails that
$\frac{\partial v}{\partial n}(p\uuu *)>0$
and since we have an inner normal derivative this violates
(i) which proves the uniqueness.


\medskip


\noindent
{\bf{A.1 Montonic properties.}}
Let $F_1$ and $F_2$ be two functions which both satisfy
(0.1) and (0.2) where  
\[ 
F_1(u,p)\leq F_2(u,p)
\]
hold for all $(u,p)\in{\bf{R}}^+\times\partial\Omega$.
If $u_1$, respectively $u_2$ solve (*) for $F_1$ and $F_2$
it follows that
$u_2(q)\leq u_1(q)$ for all $q\in\Omega$.
To see this we set $v=u_2-u_1$ which is harmonic in
$\Omega$.
If $p\in\partial\Omega$ we get
\[
\frac{\partial v}{\partial n}(p)=F_2(u_2(p),p)-F_1(u_1(p),p)\geq 0\tag{i}
\]
Suppose that the maximum of $v$ is $>0$ and let the maximum be attained at some 
point $p_*$. Since (i) is an inner normal it follows that we must have
$0=\frac{\partial v}{\partial n}(p)$ which would entail that

\[ 
F_2(u_2(p_*)p_*)>F_2(u_1(p_*),p_*)\geq F_1(u_1(p_*),p_*)\implies
\]
and this contradicts the strict inequality
$u\uuu 2(p\uuu *)>
u\uuu 1(p\uuu *)$
since we have an increasing function in (0.1).





\medskip


\noindent
{\bf{A.2. A bound for the maximum norm.}}
Let $u$ be a solution to (*) and $M_u$ denotes
the maximum norm of its restriction to
$\partial\Omega$. Choose
$p^*\in\partial\Omega$
such that
\[
 u(p^*)=M_u\tag{1}
\]
Let $G$ be the Green's function which has a pole
at $Q_*$ while $G=0$ on $\partial\Omega$. Now
\[
h=u-M_u-G
\] 
is a harmonic function in
$\Omega$.
On the boundary we have $h\leq 0$
and $h(p^*)=0$. So $p^*$ is a maximum point for this harmonic function
in the whole closed domain $\bar\Omega$.
It follows that
\[ 
\frac{\partial h}{\partial n}(p^*)\leq 0\implies
\] 
\[
F(u(p^*),p^*)=\frac{\partial u}{\partial n}(p^*)\leq
\frac{\partial G}{\partial n}(p^*)
\]
Set
\[ 
A^*=\max_{p\in\partial\Omega}\, \frac{\partial G}{\partial n}(p)
\]
Then we have
\[ 
F(M_u,p^*)\leq A^*\tag{*}
\]
Hence the assumption (0.2) for $F$ this gives a robust   estimate for the 
maximum norm $M_u$.
Next, let $m_u$ be the minimum of $u$ on $\partial\Omega$ and
consider the harmonic function
\[
h=u-m_u-G
\]
This time $h\geq 0$ on $\partial\Omega$
and if $u(p_*)=m_u$ we have $h(p_*)=0$
so here $p_*$ is a minimum for $h$.
It follows that
\[ 
\frac{\partial h}{\partial n}(p_*)\geq 0\implies
F(u(p_*),p)=\frac{\partial u}{\partial n}(p_*)\geq 
\frac{\partial G}{\partial n}(p_*)
\]
So with
\[
A_*=\min_{p\in\partial\Omega}\,\, \frac{\partial G}{\partial n}(p)
\]
one has the inequality
\[
F(m_u,p^*)\geq A_*\tag{**}
\]



\noindent
{\bf{Remark.}} Above $0<A_*<A^*$ are  constants which are independent of $F$.
Hence the maximum norms of 
solutions $u=u_F$ are controlled if the $F$-functions stay
in a family where  (0.2) holds  uniformly.


\medskip


\noindent
\centerline {\bf{B. The  linear equation.}}

\medskip


\noindent
Let $f(p)$ and $W(p)$
be a pair of continuous functions on the boundary
$\partial \Omega$ where  $W$ is positive, i.e. $W(p)>0$ for every
boundary point.
The classical Neumann
theorem
asserts  that there exists a unique function $U$ which is harmonic in
$\Omega$, extends to a continuous function on
the closed domain and its inner normal  derivative satisfies:




\[ 
\partial U/\partial n(p)=W(p)\cdot U(p)+f(p)\quad p\in\partial\Omega\tag{1}
\] 
The uniqueness  is a consequence of Green's formula.
For suppose that $U_1$ and $U_2$ are two solutions to (1) and set $v=U_1-U_2$.
Since $v$ is harmonic in $\Omega$ it follows that:
\[ 
\iiint_\Omega\, |\nabla(v)|^2 dxdydz+
\iint_{\partial\Omega}\, v\cdot \partial v/\partial n\cdot dS=0
\]
Here $\partial v/\partial n= W(p)v$ and since $W(p)>0$ holds on $\partial\Omega$
we conclude that
$v$ must be identically zero.
For the unique   solution to (1)  some  estimates hold.
Namely, set 
\[
M_U=\max_p\, U(p)\quad\text{and}\quad
m_U=\min_p\, U(p)
\] 

\medskip

\noindent
Since $U$ is harmonic in $\Omega$ the
the  maximum and the minimum are taken on the boundary.
If $U(p^*)= M_U$ for some $p^*\in\partial \Omega$
we have $\partial U/\partial n(p^*)\leq 0$.
Set

\[
W_*=\min_p \, W(p)
\]
By assumption $W_*>0$
and we get 
\[
M_U\cdot W(p^*)+f(p^*)=\partial U/\partial n(p^*)\leq 0\implies
M_U\leq  \frac{|f|_{\partial\Omega}}{W_*}
\]
where
$|f|_{\partial\Omega}$ is the maximum norm of $f$ on the boundary.
In the same way one verifies that
\[
m_U\geq -\frac{|f|_{\partial\Omega}}{W_*}
\]
Hence  the following inequality holds for the
 the maximum norm  $|U|_{\partial\Omega}$ :
\[
|U|_{\partial\Omega}\leq
\frac{|f|_{\partial\Omega}}{W_*}\tag{*}
\]


\noindent
{\bf{B.1 Estimates for first order derivatives.}}
Let $p\in\partial \Omega$ and denote by $N$
the inner normal at $p$. Since $\partial\Omega$ is of class $C^1$
a sufficiently small line segment from $p$ along
$N$ stays in $\Omega$. So at points $q=p+\ell \cdot N$
we can take the directional derivative of $U$ along $N_p$
This gives  a function

\[
 \ell\mapsto \partial U/\partial N(p+\ell\cdot N)
\]
Since the boundary is $C^1$
these functions are defined on a fixed interval $0\leq\ell\leq \ell^*$ for all $p$.
With these notations there exists a constant $B$ such
that 
\[
\bigl|\partial U/\partial N(p+\ell\cdot N)\bigr|\leq B\cdot 
||\partial U/\partial n||_{\partial\Omega}\quad
\colon\, p\in\partial\Omega\,\, \colon\,\, 0\leq \ell\leq \ell^*\tag{**}
\]


\noindent
where the size of $B$ is controlled by the maximum norm of $f$
on $\partial\Omega$ and the positive constant $W_*$ above.

\bigskip


\centerline {\bf{C. Proof of Theorem}} 

\bigskip

\noindent
Armed with the results above we can
begin the proof of the Theorem.
To begin with it suffices to prove
the theorem
when
$F(u,p)$ is an analytic function with respect to $u$.
For if we then take an arbitrary
$F$-function 
satisfying  (0.1) and (0.2), then $F$ is uniformly
approximated by a sequence 
$\{F_n\}$ of analytic
functions and if $\{u_n\}$ are the 
unique solutions  to $\{F_n\}$
then the  estimates in (B)  show
that there exists a limit
function $\lim_{n\to\infty}  u_n=u$ where  $u$ solves (*) for the given
$F$-function.
So let us now assume that $u\mapsto F(u,p)$ is a real-analytic function on
the positive real axis for each $p\in\partial\Omega$
where local power series converge uniformly with respect to $p$.
In this situation there remains to prove
the
\emph{existence} of  a solution $u$ to the PDE in (*) above Theorem 1.
To attain this
we  proceed as follows.


\medskip

\noindent
{\bf{C.1  The succesive solutions $\{u_h\}$.}}
To each real number $0\leq h\leq 1$ we seek a solution $u_h$ where

\[
 \frac{\partial u_h}{\partial n}(p)= h\cdot F(u_h,p)+(1-h)\cdot u_h(p)\tag{1}
\]
With $h=0$
we have a linear equation
\[
 \frac{\partial u}{\partial n}(p)=u(p)\tag{2}
\]

\noindent
which is solved by the Green's function 
with a pole at $Q_*$.
Next, suppose that $0\leq h_0<1$ and that we have found the 
solution $u_{h_0}$ in (1) above. Set $u_0=u_{h_0}$
and
with $h=h_0+\alpha$ for some small $\alpha>0$
we shall find  $u_h$ by a series
\[
u_h= u_0+\sum_{\nu=1}^\infty\, \alpha^\nu\cdot u_\nu\tag{3}
 \]

 
\noindent
The pole at $p_*$  occurs already in $u_0$. So 
$u_1,u_2,\ldots$ will be   a sequence of 
harmonic functions in  $\Omega$. There remains to find
this sequence so that
$u_h$ yields  a solution to (1). We will  show that this can
be achieved when $h-h_0=\alpha $ is sufficiently small.
To begin with the results from
(B) give  positive constants $0<c_1<c_2$
such that
\[
0<c_1\leq u_0(p)\leq c_2\quad\colon p\in\partial\Omega\tag{4}
\]



\noindent
Now we  use the analyticity of $F$ with respect to $u$
which enable us to write:
\[
 F(u_h(p),p)=
F(u_0(p)+\sum_{k=1}^\infty\, c_k(p)\cdot \bigl[\sum_{\nu=1}^\infty \alpha^\nu u_\nu(p)
 \bigr ]^k\tag{5}
 \]


\noindent
where $\{c_k(p)\}$ are continuous functions on
 $\partial\Omega$ which appear in an expansion
 \[ 
 F(u_0(p)+\xi,p)=F(u_0(p),p)+
 \sum_{k=1}^\infty\, c_k(p)\cdot \xi^k\tag{6}
 \]
 
 
\noindent
In the last series expansion we notice that
(4) and the hypothesis on $F$ entail that
the radius of convergence has a uniform bound below, i.e.
there exists $\rho>0$ which is independent of $p\in\partial\Omega$
and a constant $K$ such that
\[
\sum_{k=1}^\infty\, |c_k(p)|\cdot \rho^k\leq K\tag{7}
\] 
hold for all $p\in\partial\Omega$. Now
the equation (1) is solved 
via a system of equations for the harmonic functions
$\{u_\nu\}$ which are determined inductively
while $\alpha$-powers are identified.
The linear
$\alpha$-term gives  the equation
\[
\frac{\partial u_1}{\partial n}=F(u_0(p),p)-u_0(p)+
(1-h_0)u_1+h_0\cdot c_1(p)\cdot u_1(p)\tag{i}
\]
\medskip


\noindent
For $u_2$ we find that
\[
\frac{\partial u_2}{\partial n}=(1-h_0)u_2-u_1+h_0c_1(p)u_2+
c_1(p)u_1+c_2(p)u_1^2\tag{ii}
\]
\medskip


\noindent
In general, for $\nu\geq 3$ one has


\[
\frac{\partial u_\nu}{\partial n}=(1-h_0+h_0\cdot c_1(p))\cdot u_\nu+
R_\nu(u_0,\ldots,u_{\nu-1},p)\tag{iii}
\] 
where $\{R_\nu\}$ are polynomials in the preeceding $u$-functions
whose coefficients are
continuous functions derived via the
$c$-functions above.
Here the function $c_1(p)$ is given by

\[ 
c_1(p)= \frac{\partial F(u_0(p),p)}{\partial u}
\]
which by the hypothesis on $F$ is positive on $\partial\Omega$.
Next,
since $u_0$ is a solution we also have
a pair of  positive constants $0<c_*<c^*$ such that

\[
c_*<c_1(p)\leq c^*\quad\colon\quad
p\in\partial\Omega\tag{iv}
\]
Hence the function
\[ 
W(p)= (1-h_0)+h_0\cdot c_1(p)
\] 
is positive on $\partial\Omega$.
Now  estimates for the linear inhomogeneous  equations in
(B) can be applied where the $f$-functions are the $R$-polynomials.
Then  (7) and a majorising  positive
series expressing maximum norms show 
that if $\alpha$ is sufficiently small then the
series (3)  converges and gives the requested solution for
(1). Moreover, the positive  $\alpha$ 
can be taken  \emph{independently} of $h_0$.
So  together with the uniqueness of solutions $u_h$ whenever they exist, it follows
that we can move from $h=0$ until $h=1$ where we get the requested solution 
$u$ to the
PDE in Theorem 1.

\medskip

\noindent
{\bf{Remark.}} 
The reader may consult page 106 in [Carleman]
where the existence of a uniform constant $\alpha>0$ for which the series
(3) converge for every $h$ is
demonstrated by an
explicit majorant series.




\newpage
























%\end{document}