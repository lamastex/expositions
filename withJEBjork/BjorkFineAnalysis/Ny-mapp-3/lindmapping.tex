%\documentclass{amsart}
%\usepackage[applemac]{inputenc}



%\addtolength{\hoffset}{-12mm}
%\addtolength{\textwidth}{22mm}
%\addtolength{\voffset}{-10mm}
%\addtolength{\textheight}{20mm}
%\def\uuu{_}


%\def\vvv{-}

 
%\begin{document}

\centerline{\bf\large {VI. Lindel�f functions.}}

\bigskip

\noindent
{\bf{Introduction.}}
For each real number $0<a\leq 1$
there exists the entire  function
\[ 
Ea_(z)=\sum_{n=0}^\infty\,\frac{z^n}{\Gamma(1+na)}
\]
which for $a=1$ gives the exponential function $e^z$.
Growth properties of the $E$-functions were investigated
in a series of articles by
Mittag-Leffler between 1900-1904 using 
integral formulas  for the entire function
$\frac{1}{\Gamma}$.
This inspired
Phragm�n
to study 
entire functions $f(z)$ such that
there are constants $C$ and $0<a<1$ with:
\[
\log|f(re^{i\theta})| \leq C\cdot (1+|r|)^a)\quad\colon
-\alpha<\theta<\alpha
\] 
for some $0<\alpha<\pi/2$ while
$f(z)|\leq C$ for all $z\in{\bf{C}}\setminus S$.
When this holds  we get the entire function
\[ 
g(z)=\int_0^\infty\, f(sz)\cdot e^{-s}\cdot ds
\]
If $z$ is outside the sector $S$ it is clear that
$|g(z)|$ is bounded by $C\cdot \int_0^\infty e^{-s}ds=C$.
When $z=re^{i\theta}$ is in the sector we 
still
get a  bound from (1) since $0<a<1$
and conclude that the entire function $g$ is bounded and hence a constant.
Since the Taylor coefficients of
$f$
are recaptured from $g$ it follows that
$f$ must be  constant.
More general  results of this kind were obtained in
the joint article [PL] by Phragm�n and Lindel�f from 1908
and led to what is called the Phragm�n-Linedl�f principle.
Here we   discuss a continuation of [PL]
from the article
\emph{Remarques sur la croissance
de la fonction $\zeta(s)$} where  Lindel�f employed
results in collaboration with Phragm�n  to investigate
the growth of Riemann's $\zeta$-function along vertical lines 
in the strip $0<\mathfrak{Re}(z)<1$.
This  leads to the study of  various  indicator functions
attached to analytic functions.
Here is the  set-up:
Consider a strip domain
in the complex $s$-plane:
\[
\Omega=\{ s=\sigma+it\quad
\colon t>0\quad\text{and}\quad 0\leq a<\sigma<b\}
\]

\medskip

\noindent
An analytic function
$f(z)$ in $\Omega$ is of
\emph{finite type} if
there exists some
integer $k$, a constant  $C$ and some $t_0>0$ such that
\[ 
|f(\sigma+it)|\leq C\cdot t^k\quad\text{hold for}\quad t\geq t_0
\] 
To every such $f$ we define the Lindel�f function
\[
\mu_f(\sigma)
=
\limsup_{t\to\infty}\,
\frac{\text{Log}\, |f(\sigma+it)|}{\text{Log}\, t}\tag{*}
\]
Lindel�f and Phragm�n proved  that $\mu_f$ is a continuous and convex
function on $(a,b)$. No further restrictions
occur on the $\mu$-function because one has:
\bigskip

\noindent
{\bf{1. Theorem.}} \emph{For every convex and continuous function
$\mu(\sigma)$ defined in $[a,b]$
there exists an analytic function $f(z)$ without zeros in
$\Omega$ such that
$\mu_f=\mu$.}
\medskip

\noindent
{\bf{2. Exercise.}} Prove this result using the
$\Gamma$-function. First, to a pair of real numbers
$(\rho,\alpha)$ we set
\[
 f(s)=e^{-\frac{\pi i\cdot\rho s}{2}}\cdot\Gamma (\rho(s-a)+\frac{1}{2})\tag{i}
 \]
Use properties of the $\Gamma$-function to show that
$f$ has finite type in
$\Omega$ and its indicator function becomes 
a linear function:
\[ 
\mu_f(\sigma)= \rho\cdot (\sigma-a)
\]

\noindent
More generally one gets a function  $f$ where
$\mu_f$�is piecewise linear by:
\[ 
f= \sum_{k=1}^{k=m}\, c_ke^{-\frac{\pi i\cdot\rho_\nu s}{2}} \Gamma(\rho_k(s-a_k)+
\frac{1}{2})\tag{ii}
\]
where $\{c_k\}$, $\{\rho_k\}$ and $\{a_k\}$ are $m$-tuples of real numbers.
Finally, starting from
an arbitrary
convex curve we can 
choose  some dense and  enumerable set of enveloping tangents to
this curve. Then
an infinite series of the form above gives an analytic function
$f(s)$ such that
\[
 \sigma\mapsto\mu_f(\sigma)
\] 
yields an arbitrarily
given convex $\mu$-function on $(a,b)$.

\medskip


\centerline{\bf{1. A relation to  harmonic functions.}}
\medskip

\noindent
Let $U(x,y)$ be a bounded harmonic function in the strip domain
$\Omega$
and $V$  its harmonic conjugate. Set
\[
 f(s)=\text{exp}\,\bigl[\text{(log}(s)-\frac{\pi i}{2})(U(s)+iV(s))\bigr]\tag{*}
 \]


\noindent
It is easily see that
$f(z)$ has finite type in
$\Omega$.
With $s=\sigma+it$ we have
\[
|f(\sigma+it)|= 
\]
\[
\text{exp}(\frac{1}{2}\log(\sigma^2+t^2)\cdot U(\sigma+it)\cdot
\text{exp}(-(\frac{\pi}{2}- \text{arg}(\sigma+it))\cdot V(\sigma+it))
\]
It follows that 
\[ 
\frac{ \log\, |f(\sigma+it)|}{t}=
 \frac{\log\, \sqrt{\sigma^2+t^2}\cdot U(\sigma+it)]}{\log t}
+\frac {(\text{arg}(\sigma+it)-\frac{\pi i}{2})\cdot V(\sigma+it)}{t}
\]
 
 

\noindent{\bf{Exercise.}}
With $\sigma$ kept fixed one has
\[
\text{arg}(\sigma+it)=\tan\,\frac{t}{s}
\] 
which tends to $\pi/2$ as $t\to +\infty$.
Next,  $V(\sigma+it)$ is for large $t>0$
up to a constant the primitive of 
\[
\int\uuu 1^t\, \frac{\partial V}{\partial u}(\sigma+iu)\cdot du
\]
Here the partial derivative of $V$ is equal to the partial derivative
$\partial U/\partial\sigma(\sigma,u)$ taken along
$\mathfrak{Re}\, s=\sigma$.
Since $U$ is bounded in the strip domain
it follows from
Harnack's inequalities that
this partial derivative stays bounded when $1\leq u\leq t$
by a constant which is independent of $t$.
Putting this together the reader can verify that
\[ 
\lim_{t\to +\infty}\, 
\frac {(\text{arg}(\sigma+it)-\frac{\pi i}{2})\cdot 
V(\sigma+it)}{t}=0\tag{1}
\]

\bigskip

\noindent
From (1) in the Exercise  we obtain the equality
\[
\mu\uuu f(\sigma)=\limsup_{t\to\infty}\, U(\sigma+it)\tag{*}
\]
This suggests that we study
growth properties of bounded harmonic functions in strip domains.


\bigskip


\centerline {\bf{2. The $M$ and the $m$-functions.}}

\medskip

\noindent
To a bounded harmonic function $U$ in
$\Omega$  we associate the maximum and the minimum
functions:
\[ 
M(\sigma)=\limsup_{t\to\infty}
\, U(\sigma+it)
\quad\text{and}\quad\liminf_{t\to\infty}
\,U(\sigma+it)
\]


\noindent
{\bf{2.1 Proposition.}}
\emph{$M(\sigma)$ is a convex function while
$m(\sigma)$ is concave.}
\medskip

\noindent
We prove the convexity of $M(\sigma)$. The concavity of $m$
follows when we replace $U$ by $\vvv U$.
Consider a pair $\alpha,\beta$ with $a<\alpha<\beta<b$.
Replacing $U$ b $U+ A+\cdot Bx$ for suitable constants
$A$ and $B$ we may assume that
$M(\alpha)=M(\beta)=0$
and the requested convexity follows if we can show that
\[ 
M(\sigma)\leq 0\quad\colon\quad \alpha<\sigma<\beta
\]
To see this we consider
rectangles 
\[
\mathcal R[T\uuu *,T^*]
=\{\sigma+it\quad \alpha \leq \sigma\leq \beta
\quad\text{and}\quad  T\uuu *\leq t\leq T^*\}
\]
Let $\epsilon>0$ and start with
a large $T\uuu *$ so that
\[
t\geq T\uuu *\implies
U(\alpha+it)\leq \epsilon
\] 
and similarly with $\alpha$ replaced by $\beta$.
Next, we have a constant $M$ such that
$|U|\uuu\Omega\leq M$.
If $z=\sigma+it$ is an interior point of the rectangle above
it follows by harmonic majorisation that

\[
U(\sigma+it)\leq \epsilon+ M\cdot \mathfrak m\uuu z(J\uuu *\cup J^*)
\]
where the last term  is the harmonic measure at $z$
which evaluates the harmonic function in
the rectangle at $z$ with boundary values zero on the two verical lines
of the rectangle which it is equal to 1 on the horizontal intervals 
$J^*=(\alpha,\beta)+iT^*$ and 
$J\uuu *=(\alpha,\beta)+iT\uuu *$
\medskip

\noindent
{\bf{Exercise.}}
Show (via the aid of figure that with $T^*=2T\uuu *$
one has
\[
\lim\uuu{T\uuu *\to +\infty}\,
\mathfrak m\uuu {\sigma+3i T\uuu */2}(J\uuu *\cup J^*)=0
\] 
where this limit is uniform when $\alpha\leq \sigma\leq \beta$.
Since $\epsilon>0$ is arbitrary in (xx) the reader can now conclude that
$M(\sigma)\leq 0$ for every $\sigma\in(\alpha,\beta)$.
\bigskip

\noindent
{\bf{A special case.}}
Suppose that we have the equalities
\[
m(\alpha)=M(\alpha)\quad\text{and}\quad
m(\beta)=M(\beta)\tag{1}
\]
using rectangles as above and harmonic majorization the
reader can verify that this implies that
\[
m(\sigma)=M(\sigma)\quad\colon\quad \alpha<\sigma<\beta
\]
This result is due to Hardy and Littlewood in [H\vvv L].




\bigskip

\noindent
{\bf{The case when $M(\sigma)-m(\sigma)$ has
a tangential zero}}.
Put $\phi(\sigma)=M(\sigma)-m(\sigma)$ and suppose that this non\vvv 
negative
function in $(a,b)$
has a zero at some $a<sigma\uuu 0<b$
whose graph has a tangent at $\sigma_0$.
This means that if:
\[ 
h(r)=\max_{-r\leq|\sigma\vvv \sigma\uuu 0|\leq r}\, \phi(\sigma)
\] 
then
\[
\lim_{r\to 0}\, \frac{h(r)}{r}=0\tag{*}
\]


\noindent
Under this hypothesis the following result is proved in
[Carleman].
\medskip


\noindent
{\bf{2.2 Theorem.}} \emph{When (*) holds we have}
\[ 
m(\sigma)=M(\sigma)\quad\text{holds for all}\quad
a<\sigma<b
\]

\medskip

\noindent
The subsequent proof from [Carleman]
was given
at 
a lecture by Carleman  in Copenhagen 1931 which
has the merit that a similar reasoning 
can be applied in dimension $\geq 3$.
Adding some linear function to $U$ we may assume that
$M(\sigma_0)=m(\sigma_0)=0$ which mans that
\[
\limsup_{t\to\infty}\,
U(\sigma_0,t)=0\tag{1}
\]

\noindent
Next,
consider the function
\[ 
\phi\colon t\mapsto \partial U/\partial \sigma(\sigma_0,t)\tag{1}
\]
The assumption (*) and the result in XXX gives:
\[
\lim\uuu{t\to\infty}\,
\partial U/\partial \sigma(\sigma_0,t)=0\tag{2}
\]


\noindent
Next, consider some 
$a<\sigma<b$ and let $\epsilon>0$.
By the  result from XX there exist
finite tuples of constants $\{a_1,\ldots,a_N\}$ and
$\{b_1,\ldots,b_N\}$ and 
some $N$-tuple $\{\tau_\nu\}$ which
stays in a $[0,1]$ such that
\[
\bigl| U(\sigma,t)-\sum\, a_\nu\cdot U(\sigma_0, t_\nu+t)
-\sum\, b_\nu\cdot \partial U/\partial\sigma(\sigma_0, t_\nu+t)\bigr|<\epsilon\quad
\text{hold for all}
\quad t\geq 1\tag{5}
\]
Since $\epsilon$ is arbitrary it follows from (1\vvv 2)  
that
\[
\lim_{t\to\infty}\, U(\sigma,t)=0\tag{5}
\]
for every $a<\sigma<b$ which obviously gives
the requested equality
in
Theorem 2.2.



\bigskip

\noindent
\centerline {\bf{2.3. Integral  indicator funtions.}}
\medskip

\noindent
Let $f(s)$   be an analytic function of finite order
in the strip domain $\Omega$ and fix some $t_0>0$ which does not affect
the subsequent constructions.
For a 
pair  $(\sigma,p)$ where
$a<\sigma<b$ and $p>0$
we  associate the set of
of positive numbers $\chi$ such that the integral
\[ 
\int_{t_0}^\infty\, \frac{|f(\sigma+it)\bigl|^p}{t^\chi}\cdot dt<\infty\tag{*}
\]



\noindent
We get a critical smallest non-negative number  
$\chi_*(\sigma,p)$ such that (*) converges when
$\chi>\chi_*(\sigma,p)$.
In the case $p=1$ 
a  result due to Landau
asserts that 
$\chi(\sigma,1)$ determines the half-plane of the complex $z$-plane where
the function
\[ 
\gamma(z)=\int_{t_0}^\infty\, \frac{f(\sigma+it)}{t^z}\cdot dt
\] 
is  analytic and
$\sigma\mapsto \chi(\sigma,1)$
is a convex function on  $(a,b)$.
A more general  convexity
result holds when $p$  also varies.
\medskip

\noindent
{\bf{2.4 Theorem.}} \emph{Define the $\omega$-function by}:
\[ 
\omega(\sigma,\eta)= \eta\cdot \chi(\sigma,\frac{1}{\eta})\quad\colon
a<\sigma<b\quad\colon \eta>0
\]
\emph{Then $\omega$ is a continuous and convex function of the 
two variables $(\sigma,\eta)$
in the product set $(a,b)\times {\bf{R}}^+$.}
\medskip

\noindent
{\bf{2.5 Remark.}}
Theorem 2.4 is proved using 
H�lder inequalities and
factorisations of analytic functions which
reduces the proof to the case when
$f$ has no zeros.
The reader is invited to supply details of the proof or 
consult [Carleman].


\bigskip




\centerline{\bf {3.  Lindel�f estimates in the unit disc.}}

\bigskip

\noindent
Let $f(z)$ be analytic in the open unit disc given by a power series

\[ f(z)=\sum a_n\cdot z^n
\]
We assume that the sequence $\{a_n\}$ has temperate growth, i.e. there exists
some integer $N\geq 0$ and a constant $K$ such that
\[ 
|a_n|\leq K\cdot n^N\quad\colon\quad n=1,2,\ldots
\]
In addition we assume that the sequence
$\{a_n\}$ is not too small in the sense that
\[ 
\sum_{n=1}^\infty\, |a_n|^2\cdot n^{s}=+\infty\quad\colon
\quad\,\forall\, s>0\tag{*}
\]
Now there exists  the smallest number $s_*\geq 0$ such that
the Dirichlet series

\[ 
\sum_{n=1}^\infty\, |a_n|^2\cdot \frac{1}{n^s}<\infty,\quad
\text{for all}\,\, s>s_*
\]
\medskip

\noindent
To each
$0\leq\theta\leq 2\pi$ we set
\[ 
\chi(\theta)=\min_s\, \int_0^1\, \bigl|f(re^{i\theta})\bigr|\cdot(1-r)^{s-1}\cdot dr<\infty\tag{1}
\]
\[
\mu(\theta) =\text{Lim.sup}_{r\to 1}\,
\frac{\text{Log}\, |f(re^{i\theta})|}{ \text{Log}\,\frac{1}{1-r}}
\tag{2}
\]
\medskip

\noindent
We shall study the two functions $\chi$ and $\mu$.
The first result is left as an exercise.

\medskip


\noindent
{\bf {3.1. Theorem.}} \emph{The inequality}
\[
\chi(\theta)\leq\frac{s^*}{2}
\] 
\emph{holds almost everywhere, i.e. for all $0\leq\theta\leq 2\pi$ outside a null set on
$[0,2\pi]$.}

\medskip

\noindent
\emph{Hint.} Use the formula

\[ 
\frac{1}{2\pi}\cdot \int_=^{2\pi}\, |f(re^{i\theta})|^2\cdot d\theta=
\sum\, |a_n|^2
\]


\bigskip

\noindent 
For the $\mu$-function a corresponding result holds:



\medskip
\noindent
{\bf {3.2. Theorem.}} \emph{The inequality below holds almost everywhere.}
\[
\mu(\theta)\leq\frac{s^*}{2}
\] 


\noindent
\emph{Proof.}
Let $\epsilon>0$ and introduce the function
\[ 
\Phi(z)=\sum\, a_n\cdot \frac{\Gamma(n+1)}{\Gamma(n+1+\frac{s^*}{2}+\epsilon)}
\cdot z^n=\sum\, c_n\cdot z^n
\]


\noindent
It is clear that
the construction of
$s^*$ entails
\[ 
\sum\, |c_n|^2<\infty
\]


\noindent
Next, set $\Phi_0=\Phi$ and define inductively the sequence
$\Phi_0,\Phi_1,\ldots$ by
\[ \Phi_\nu(z)=
z^{\nu-1}\cdot \frac{d}{dz}\bigl[ z^\nu\cdot \Phi_{\nu-1}(z)\bigr)
\quad\colon\quad \nu=1,2,\ldots
\]


\noindent
{\bf 3.3 Exercise.} Show that for almost every
$0\leq\theta\leq2\pi$ there exists a constant
$K=K(\theta)$ such that
\[
|\Phi_\nu(re^{i\theta})|\leq K(\theta)\cdot \frac{1}{1-r)^\nu}\quad\colon
0<r<1
\]


\noindent
Next, with  $s^*$ and $\epsilon$ given we
define
the integers $\nu$ and $\rho$:
\medskip
\[ 
\nu=\bigl[\,\frac{s^*}{2}+\epsilon\,\bigr]+1\quad\colon\quad
\rho=\frac{s^*}{2}+\epsilon- \bigl[\,\frac{s^*}{2}+\epsilon\,\bigr]
\]
where the bracket term is the usual notation for the
smallest integer $\geq \frac{s^*}{2}+1$.
\medskip

\noindent
{\bf Exercise} Show that with
$\nu$ and $\rho$ chosen  as above one has
\[\Phi_\nu(z)=\sum\, a_n\cdot \frac{\Gamma(n+1+\nu)}{\Gamma(n+1+\rho-1)}
\cdot z^n
\]

\noindent
and use  this to show the inversion formula
\[ 
f(z)=\frac{1}{z^\nu\cdot\Gamma(1-\rho)}\cdot
\int_0^z\, (z-\zeta)^{-\rho}\dot \zeta^{\nu+\rho-1}\cdot
\Phi_\nu(\zeta)\cdot d\zeta\tag{*}
\]

\medskip

\noindent
{\bf 3.4 Exercise.} Deduce from the above that for almost every
$\theta$ there exists a constant $K(\theta)$ such that
\[
|f(re^{i\theta})|\leq K(\theta)\cdot\frac{1}{ (1-r)^{\nu+\rho-1}}\tag{**}
\]
\medskip

\noindent
{\bf Conclusion.}
From (**) and the construction of
$\nu$ and $\rho$ the reader can confirm
Theorem  3. 2.

\bigskip

\noindent
{\bf 3.5 Example.} Consider the function
\[ 
f(z)=\sum_{n=1}^\infty\, z^{n^2}
\]
Show that $s^*=\frac{1}{2}$ holds in  this case. Hence
Theorem  B.2 shows that for each $\epsilon>0$ one has 
\[
\max_r\, (1-r)^{\frac{1}{4}+\epsilon}\cdot |f(re^{i\theta}|<\infty\tag{E}
\]
for almost every $\theta$. 
\medskip

\noindent
{\bf 3.6 Exercise.}
Use the inequality above to show the following:
For a complex number
$x+iy$ with $y>0$ we set
\[
q= e^{\pi ix-\pi y}
\]
Define the function
\[
\Theta(x+iy)= 1+q+q^2+\ldots
\]
Show that when $\epsilon>0$
then there exists a constant $K=K(\epsilon,x)$ for almost all
$x$ such that
\[
y^{\frac{1}{4}+\epsilon}\cdot |\theta(x+iy)|\leq K\quad\colon\quad y>0
\]


\newpage


%\end{document}













