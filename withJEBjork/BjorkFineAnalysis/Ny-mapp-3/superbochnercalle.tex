
\documentclass{amsart}
\usepackage[applemac]{inputenc}

\def\uuu{_}

\def\vvv{-}

\begin{document}
\newpage





\centerline{\bf{Almost periodic functions.}}
\bigskip

\noindent
{\bf Introduction.} The theory about almost periodic functions is due to Harald Bohr 
and was 
developed
in the years 1910-1930. The reader may consult the text-book
[Bohr] for a detailed account of Bohr's contributions.
Here is the baisc defintion:
\medskip

\noindent
{\bf Defintion.}
\emph{A continuous function $f(x)$ on the real line
is called almost periodic if there to each $\epsilon>0$ exists some
positive number $\ell(\epsilon)$
such that every interval of length
$\ell(\epsilon)$  contains a point
$\tau$ such that the maximum norm}
\[
\max{x\in{\bf{R}}} |f(x+\tau)-f(x)|\leq\epsilon\tag{*}
\]
\emph{The class of these functions is denoted by
$\mathcal{AP}$.}
\medskip

\noindent
{\bf Trigonometric polynomials.}
They consist of functions
given by finite sums:
\[ 
A(x)=c_1e^{i\lambda x}+\ldots+c_Ne^{i\lambda_Nx}
\]
where $\lambda_1,\ldots,\lambda_N$ is some $N$-tuple of real  numbers and
$c_1,\ldots,c_N$ are complex numbers.
Every trigonometric polynomial belongs to
$\mathcal{AP}$. Indeed this follows from the following
elementary fact:
\medskip

\noindent
{\bf Proposition.}
\emph{Let $\lambda_1,\ldots,\lambda_N$
be a finite set of real numbers. To each $\epsilon>0$ there
exists some positive number $\ell(\delta)$ such that
every interval on the real line if length $\ell(\delta)$
contains a point $\tau$ such that}
\[ 
\frac{\lambda_\nu}{2\pi}\cdot\tau=\delta_\nu+b_\nu\quad\colon\quad
|\delta_\nu|<\epsilon\quad\colon\, b_\nu\in{\bf{Z}}
\]


\noindent
Apply this to a trigonometric polynomial $A(x)$. Then  we see that
\[
A(x+\tau)-A(x)=\sum\, c_\nu(\, e^{2\pi i\delta_\nu}-1)\cdot e^{i\lambda_\nu x}
\]
and conclude that $A(x)\in\mathcal{AP}$.
Conversely one has

\medskip

{\noindent
{\bf Theorem.}
\emph{Every $f\in\mathcal{SAP}$ can be uniformly
approximated by trigonometric polynomials,
i.e. there exists a sequence of trigonometric polynomials
$A_1,A_2,\ldots$ such that}

\[ \lim_{n\to\infty}\,\max_{x\in{\bf{R}}}\,
|A_n(x)-f(x)|=0
\]

\medskip

\noindent
The proof is left as an exercise. The hint is to use 
approximation for periodic functions via Fourier series.

\medskip

\noindent
{\bf The spectral function.}
Let $f(x)$ be almost periodic.
From Theorem 1  it follows easily that we get  a continuous and bounded
function $\Phi(x)$ defined by
\[ 
\Phi(x)=
\lim_{w\to\infty}\, \int_{-w}^w\,
f(x+t)\bar f(t)\cdot dt
\]
\medskip

\noindent Following Bohr we refer to
$\Phi(x)$ as the spectral function of $f$.
If $x_p$ and $x_q$ are teo real numbers we notice that
\[
 \Phi(x_p-x_q)=
\lim_{w\to\infty}\, \int_{-w}^w\,
f(x_p+t)\bar f(x_q+t)\cdot dt
\]
So of $x_1,\ldots,x_N$is an $N$-tuple of real numbers
and $a\uuu 1,\ldots,a_N$ an $N$-tuple of complex numbers we get
\[
\sum\sum a_p\bar a_q \Phi(x_p-x_q)=
\lim_{w\to\infty}\, \int_{-w}^w\,
|\sum a_pf(x_p+t)|^2\cdot dt\tag{*}
\]
Hence $\Phi(x)$ is a $\mathcal{B}$-function and Theorem X.1 
gives  a non-negative measure
$\sigma$ such that
\[ 
\Phi(x)=\int\, e^{ix\xi}\cdot d\mu(\xi)
\]
\medskip

\noindent
Following Bohr the non-negative measure $\mu$
is called the spectral measure associated to $f$.
A major result about almost periodic functions is:

\bigskip

\noindent
{\bf{Theorem.}} \emph{The spectral measure of an
almost periodic function is always discrete. Moreover one has the limit formula:}

\[ 
\lim_{w\to\infty}\, \frac{1}{2w}\int_{-w}^w\,
\bigl |\, f(x)-\sum\, c_\nu\cdot e^{i\lambda_\nu x}\bigr |^2\cdot dx=0
\]
\emph{where $\mu=\sum\, c\uuu\nu\cdot \delta\uuu{\lambda\uuu\nu}$.}







\bigskip







\centerline{\bf Operational calculus on $L^1({\bf{R}})$}


\bigskip
\noindent






Let $f(x)$ be in $L^1({\bf{R}})$ and denote its Fourier transform
by $g(\xi)$, i.e.
\[ 
g(\xi)=\int\, e^{-ix\xi}f(x)dx\tag{*}
\]
We know that
$g(\xi)$ is a continuous and complex-valued function.
Let $[a,b]$ be a closed interval on the real $\xi$-line,
Put  $w=g(\xi)$ which gives the compact subset
$g[a,b])$ of the complex $w$-plane.
Let 
$\Phi(w)$ be an analytic function
defined in some open neighborhood of
$g[a,b]$.
With these notations one has
\medskip

\noindent
{\bf Theorem.}
\emph{There exists a function
$\phi(x)\in L^1({\bf{R}})$ whose Fourier transform
satisfies}
\[ 
\hat\phi(\xi)=\Phi(g(\xi))\quad\colon\quad a\leq\xi\leq b
\]
\medskip

\noindent
\emph{Proof.}
Consider  some point $a\leq \xi_*\leq b$.
By the analyticity of $\Phi$ there
exists a  series expansion
\medskip

\[ 
\Phi(w)=\Phi(w_*)+\sum_{\nu=1}^\infty\, c_\nu(w-w^*)^\nu
\quad\colon w^*=g(\xi_*)
\]
which is convergent in some open disc centered at $w_*$. In particular we can
find  $\delta>0$ and a constant $M$
such that
\medskip
\[
|c_\nu|\leq M\cdot\delta^{-\nu}\quad\colon\quad \nu=0,1,\ldots\tag{i}
\]
Next, we consider the special function

\[
W(\xi)=1\quad\colon\quad |\xi|\leq1 \quad\colon
W(\xi)=2-|\xi|\quad\colon\quad 1\leq |\xi|\leq 2
\]
Recall from XX that $W$ is the Fourier transform of an
$L^1$-function $P(x)$. Fourier's inversion formula gives:
\[ 
P(x)=\frac{1}{2\pi}\int e^{ix\xi}\cdot W(\xi)d\xi\tag{ii}
\]
Next, when $|g(\xi)-g(\xi_*)|<\delta$
it follows from (i) that
\[ 
\Phi(g(\xi))-\Phi(g(\xi_*)=\sum\, c_\nu(g(\xi)-g(\xi_*))^\nu\tag{iii}
\]
Let $k>0$ and put
\medskip


\[ \psi_k(\xi)= W(k(\xi-\xi_*))\cdot \Phi(g(\xi_*))+
\sum\, c_\nu\cdot\bigl[ W(k(\xi-\xi_*))\cdot (g(\xi)-g(\xi_*)\,\bigr]^\nu\tag{iii}
\]
\medskip

\noindent
Rules for dilation under the Fourier transform and (ii) give
\medskip
\[
\frac{1}{k}\cdot e^{i\xi_*\cdot x}\cdot P(\frac{x}{k})=
\text{inverse Fourier transform of }\,\, W(k(\xi-\xi_*))\tag{iv}
\]
\medskip

\noindent
More precisely, we have
\[
\frac{1}{k}\cdot e^{i\xi_*\cdot x}\cdot P(\frac{x}{k})=
\frac{1}{2\pi}\cdot\int e^{ix\xi}\cdot W(k(\xi-\xi_*))\cdot d\xi\tag{*}
\]
\bigskip

\noindent
Define the function $Q_k(x)$ by:

\[
Q_k(x)= \frac{1}{k}\int e^{i\xi_*(x-y)}\bigl [ P(\frac{x-y}{k}-P(\frac{x}{k})\,\bigr]
f(y) dy\tag{v}
\]
Then (*) and Fourier's inversion formula give:

\[
W(k(\xi-\xi_*))\cdot (g(\xi)-g(\xi_*)=
\int e^{-ix\xi} \cdot Q_k(x)dx\tag{vi}
\]
\medskip

\noindent 
Next, the triangle inequality applied to the
right hand side in (vi) gives:
\medskip

\[ \int\,|Q_k(x)|\cdot dx\leq 
\frac{1}{k} \iint
| P(\frac{x-y}{k}-P(\frac{x}{k})|\cdot |f(y)|\cdot dxdy=
\]
\[
||f||_1\cdot 
\int\, \bigl |P(x-\frac{y}{k})-P(\frac{x}{k})\bigr |\cdot dx\tag{**}
\]
\medskip

\noindent
Since $P\in L^1({\bf{R}})$
the Riemann-Lebesgue theorem gives
\[ 
\lim_{k\to 0}\, 
\int\, \bigl |P(x-\frac{y}{k})-P(\frac{x}{k})\bigr |\cdot dx=0
\]
Together with the   inequality (**)
we therefore obtain
\medskip

\noindent
{\bf Sublemma.}
One has

\[ ||Q_k||_1=\lambda(k)\quad\colon\quad\lim_{k\to 0}\,\lambda(k)=0
\]
\medskip

\noindent
For each
$\nu\geq 2$ we construct the $\nu$:th fold convolution of
$Q_k$ which we denote by
$Q_k(\nu)$. The multiplicative inequality for
$L^1$-norms and the Sublemma give:
\medskip

\[ 
||Q_k(\nu)||_1\leq \lambda(k)^\nu\quad\colon\quad \nu=1,2,3,\ldots\tag{***}
\]
\medskip

\noindent
We can choose $k$ so large that
$\lambda(k)<\delta$. it follow from (xx) that
\[
G(x)= 
\sum_{\nu=1}^\infty\, c_\nu\cdot Q_k(\nu)(x)
\]
converges in the Banach space $L^1({\bf{R}})$.
Hence we obtain the $L^1({\bf{R}})$-function defined by
\[
G^*(x)= \frac{1}{k}\cdot\Phi(g(\xi_*))\cdot e^{i\xi_*\cdot x}
\cdot P\bigl[\frac{x}{k}\bigr)+ G(x)
\]
\medskip

\noindent
From the previous constructions it is clear that the
Fourier transform of $G^*(x)$ is equal to
the function $\psi_k(\xi)$ from (iii).
Finally,
from the construction of the $W$-function and the original series expansion of
$\Phi$ we have the equality
\[
\psi_k(\xi)=\Phi(g(\xi))\quad\colon\quad
|\xi-\xi_*|\leq\frac{1}{k}\tag{****}
\]
\medskip

\noindent
{\bf Remark.}
Above the integer $k$ was chosen to ensure
that $\lambda(k)<\delta$. In the
Sublemma we used the inequality (**) where
the $P$-function does not depend on $f$ or on $\Phi$.
So we conclude that when we restrict the attention to $L^1$-functions $f$ of norm
$\leq 1$, then it suffices to choose
$k$ so large that
\[
\int\, \bigl |P(x-\frac{y}{k})-P(\frac{x}{k})\bigr |\cdot dx<\delta
\]
\medskip

\noindent
{\bf Final part of the proof.}
By the remark above we find
$L^1$-functions whose Fourier transforms agree with
$\Phi(g(\xi))$ on small intervals around every point $a\leq\xi_*\leq b$.
By the Heine-Borel Lemma and a $C^\infty$-partition of the unit
we can finish the proof of Theorem xx. To be precise, we use that
if $h(\xi)$ is a test-functionon the real $\xi$-line then it is the
Fourier transform of some $L^1$-function.


\bigskip

\centerline{\bf Bochner's moment theorem}

\bigskip

\noindent
In probability theory
the frequency  of a 
stochastic variable is expressed
by a probability measure
$\mu$ on the real line where we take $t$
as the coordinate. The characteristic function
is by definition the Fourier transform
of $\mu$ and we set
\[ 
f(x)=\int\, e^{-ixt}\cdot d\mu(t)\tag{*}
\]
Let
$x_1,\ldots,x_N$ be some $N$-tuple of real numbers
and $\alpha_1,\ldots,\alpha_N$ some $N$-tuple of complex numbers.
Then 
\medskip
\[ 
\sum\sum\, f(x_p-x_q)\alpha_p\cdot\bar\alpha_q=
\int\,\bigl |\sum\, \alpha_p\cdot e^{-ix_p\cdot t}\bigr|^2d\mu(t)\tag{**}
\]
Since $\mu\geq 0$ it follows that the
right hand side is $\geq 0$.
It turns out that this inequality characterizes the family of
bounded continuous functions $f(x)$ which are Fourier
transforms of non-negative measures. First we introduce a class of functions:

\medskip

\noindent
\emph{The class $\mathcal{B}$}. It consists
of
continuous  functions $f(x)$ on the real $x$-line such that
$f(0)=1$ and }
\[
\sum\sum\, f(x_p-x_q)\alpha_p\cdot\bar\alpha_q\geq 0\quad\colon\quad
f(0)=1
\]

\noindent
hold for all
$N$\vvv tuples
$x_\bullet$ and $\alpha_\bullet$
as above.
\medskip

\noindent
{\bf Remark.}
Given $x>0$ we take $N=2$ with $x_1=0$ and $x_2=x$
and $\alpha_1=1$ while $\alpha-2=e^{i\theta}$.
Then Bochner's condition gives
\[
2\cdot f(0)+ e^{i\theta}f(x)+e^{-it\theta}f(-x)\geq 0\tag{i}
\]
With  $\theta=\pi/2$ it  follows that
$f(-x)=\bar f(x)$ and then the inequality (i)   gives
\[ 
|f(x)|\leq f(0)=1\tag{ii}
\]
So functions in $\mathcal B$ are automatically bounded.
Theorem 1 is due to Bochner and was for example
presented in his  
book [Boch]
\emph{Vorlesungen �ber Fouriersche Integrale} from 1932.
\medskip

\noindent
{\bf 1. Theorem.} \emph{For each $f\in\mathcal{B}$ there exists
a unique non-negative measure $\mu$ such that}
\[
f(x)=\int e^{-ixt}d\mu(t)
\]
\medskip


\noindent
The subsequent proof of
Theorem 1 uses   a representation formula for positive harmonic functions
in the upper half-plane which in its turn is a special case of
more general representations of harmonic functions in half\vvv planes due to 
the Brothers Nevanlinna in the article [Nev\vvv Nev] from  1920.
Let us also remark that the periodic version 
preceeded  Theorem 1 and is due 
to
G. Herglotz who proved the following
in his article   [Herg] 
from 1911:
\newpage

\noindent
{\bf 2. Theorem.} \emph{Let $\{m_n\,\colon -\infty<n<\infty\}$
be a sequence of complex numbers.
In order that there exists a non-negative Riesz measure
$\mu$ on the interval $[0,2\pi]$ such that}
\[ m_n=\int_0^{2\pi}\, e^{in\theta}\cdot d\mu(\theta)
\]
\emph{it is necessary and sufficient that}
\[
\sum_{\nu=-N}^{\nu=N}\sum_{j=-N}^{j=N}\, m_{\nu-j}\cdot \alpha_\nu\cdot\bar\alpha_j\geq 0
\]
\emph{holds for and finite sequence of complex numbers}
$\alpha_{-N}\ldots,\alpha_N$.


\bigskip

\noindent
To prove Theorem 1 we
shall need
the following result:
\medskip

\noindent
{\bf 3. Proposition.}
\emph{For each pair of real numbers
$\xi,\eta$ with $\eta>0$
there exists a function $\phi(x)\in L^1({\bf{R}})$
such that}
\[
e^{-i\xi x-\eta|x|}= \int_{-\infty}^\infty\,
\phi(x+y)\cdot\bar \phi(y)\cdot dy\quad\colon\, -\infty<x<\infty
\]
\medskip

\noindent
\emph {Proof}
Define the function
\[
g(t)=\sqrt{\frac{1}{2\pi}}\cdot 
\sqrt{\frac{\eta}{\eta^2+(\xi+t)^2}}
\tag{i}
\]
Put
\[
\phi(x)=\sqrt{\frac{1}{2\pi}}\cdot \int\, e^{itx}\cdot g(t)\cdot dt\tag{ii}
\]
\medskip

\noindent 
We leave as an exercise to the reader to verify that
$\phi(x)\in L^1({\bf{R}})$
and that
the equality in Proposition 3 holds.
\medskip

\centerline{\emph{ Proof of Theorem 1.}}

\medskip

\noindent
Consider the function
\[ 
\Phi(\xi,\eta)=\int_{-\infty}^\infty\, 
e^{-i\xi x-\eta|x|}\cdot
f(x)\cdot dx\quad\colon\, \xi\in{\bf{R}}\quad\colon\eta>0\tag{1}
\]
\medskip


\noindent
Proposition 3 gives:
\[
\Phi(\xi,\eta)=
\iint\,
\phi(x+y)\cdot\bar \phi(y)\cdot f(x)\cdot dxdy=
\]
\[
\iint\,
\phi(x)\cdot\bar \phi(y)\cdot f(x-y)\cdot dxdy
\]
Since both $f$ and $\phi$ belong to $L^1$ we can approximate
the last double integral by Riemann sums
which  take the  form

\[
\sum\, f(x_p-x_q)\cdot\alpha_p\bar \alpha_q
\]
Since 
$f\in\mathcal B$ it follows 
that
the $\Phi$-function is $\geq 0$.
Next, for each fixed $x$ we consider the function
\[ 
(\xi,\eta)\mapsto e^{-i\xi x-\eta|x|}\tag{*}
\]
Since $i^2=-1$ we see that this function is harmonic. Approximating
the integral (1) by Riemann sums we conclude that
$\Phi(\xi,\eta)$ is a harmonic function in the
upper half-plane
$\eta>0$.
Since $|f(x)|\leq 1$ for all $x$ and $|e^{-ix\xi}|=1$
the
the triangle inequality gives
\[ |\Phi(\xi,\eta)|\leq
\int_{-\infty}^\infty\, 
e^{-\eta|x|}\cdot
dx=
\frac{2}{\eta}\tag{2}
\]
\medskip

\noindent
Now $\Phi$ is harmonic and $\geq 0$
in the upper half-plane. Hence  the inequality (i) and
the general result in XX
gives a non-negative measure $\mu$ of finite total mass such that

\[ 
\Phi(\xi,\eta)=\frac{1}{\pi}\cdot
\int_{-\infty}^\infty\, \frac{\eta}{\eta^2+(\xi-t)^2}\cdot d\mu(t)\tag{3}
\]
\medskip

\noindent
With $\eta>0$ kept fixed we notice that (2) means that the function
$\xi\mapsto \Phi(\xi,\eta)$ is the Fourier transform of
$e^{-\eta|x|}f(x)$. Hence  (3) and Fourier's inversion formula 
yield:
\[
e^{-\eta|x|}f(x)=
\frac{1}{2\pi^2}\cdot\int_{-\infty}^\infty\,\, 
e^{ix\xi}\cdot \bigr[ \int_{-\infty}^\infty\, \frac{\eta}{\eta^2+(\xi-t)^2}\cdot d\mu(t)
\bigl]\cdot d\xi=
\]
\[
\frac{1}{2\pi^2}\cdot\int\,
\bigr[ \int_{-\infty}^\infty\, \frac{\eta\cdot e^{ix\xi}}{\eta^2+(\xi-t)^2}\bigl]\,
d\mu(t)\quad\colon\quad \eta>0
\tag{4}
\]
\medskip

\noindent
Now we use the limit formula:

\[
\frac{1}{\pi}\cdot \lim_{\eta\to 0}
\int_{-\infty}^ \infty e^{ix\xi}\cdot \frac{\eta}{\eta^2+(\xi-t)^2}\cdot d\xi=e^{ixt}
\quad\colon\,-\infty<t<\infty 
\]
\medskip


\noindent
At the same time we have
\[
\lim_{\epsilon\to 0}\, e^{-\epsilon|x|}\cdot f(x)\to f(x)
\]
\medskip

\noindent
So after the passage to the limit as
$\eta\to 0$ we  get:

\[ f(x)=\frac{1}{2\pi}\cdot\, \int\, e^{ixt}\cdot d\mu(t)\tag{5}
\]
\medskip

\noindent 
If we wish to use $e^{\vvv ixt}$
we just have to replace
$\mu$
by $\mu^*$ where $d\mu^*(t)=d\mu(-t)$ and 
Theorem 1 follows.

\newpage

\centerline{\bf A uniqueness for  a moment problem.}
\bigskip

{\bf Introduction.}
The first general study of
moment problems is due to Stieltjes who asked for
a non-negative Riesz measure $\mu$ defined on $x\geq 0$ such that
the moments
\[ \
\int_0^\infty\, t^n\cdot d\mu(t)=c_n\quad\colon\quad n=0,1,\ldots\tag{*}
\]
\medskip

\noindent
{\bf Remark.}
In 1880 the notion of general measure was not
Conditions for the existence of solutions to (*) as well as uniqueness
were proved by Stieltjes using continued fraction expansions
corresponding to the series
\[ 
\sum\, \frac{(-1)^\nu\cdot c_\nu}{t^{\nu+1}}
\]
We shall not discuss this any further but refer to original work
by Stieltjes. Instead we study the
moment problem where integration takes place over the whole real line, i.e. this time
we regard a non-negative measure $\mu$
with support on the real line and set
\[
c_\nu=\int_{-\infty}^\infty\, t^\nu\cdot d\mu(t)\quad\colon\quad \nu=0,1,\ldots\tag{**}
\]
This moment problem was 
studied by
Hamburger in (Math.Annalen 81-82). See also R. Nevanlinna's
article \emph{Asymptotische Entwicklungen beschr�nkter Funktionen
und das Stieltjesche Momentproblem}.
A conclusive  uniqueness result was presented 
by T. Carleman at the Scandianavian Congress of mathematics at 
Helsinki in 1922.
Before we 
announce this  uniqueness theorem
we remark that
in order that (*) has a meaning we only regard non-negative measures
$\mu$ on the real $t$-axis for which
\[
\int_{-\infty}^\infty\, t^{2n}\cdot d\mu(t)<\infty
\quad\colon\quad n=0,1,\ldots
\]
\medskip

\noindent
This means that $\mu$ does not carry much mass when
$|x|$ is large. In particular  the function
\[ 
F(z)=\int\,\frac{d\mu(t)}{t-z}
\]
exists as analytic function of the complex variable
$z$ outside the real axis, i.e. $F(z)$ is analytic in both
the upper and the lower half-plane.
Moreover, 
$F(z)$ determines $\mu$, i.e. if $F$ is identically zero
then
$\mu=0$. This is  used
to settle uniquness for the moment equation
(**)
where the 
result from  [Car: p. 80] goes as follows:


\bigskip

\noindent
{\bf Theorem.}
\emph{Let $\{c_\nu\}$ be a sequence of real numbers such that
the series}
\[ 
\sum_{n=0}^\infty\, \frac{1}{
\bigl|c_{2n}\bigr|^{\frac{1}{2n}}}=+\infty
\]
\emph{Then (*) has at most one solution.}
\medskip

\noindent{\bf Remark.}
PROOF via cauchy integralsgiven on page
80 is  elegant ...

\noindent
{\bf A case of non-uniqueness.}
Consider  a  $C^\infty$- function $f(x)$
which vanishes with all its derivatives at
$x=0$ and $x=1$.
Put
\[
m^2_p=\int_0^1\, |f^{(p)}(x)|^2\cdot dx
\]
Then there exist several measures  $\mu$ such that
\[ 
\int_0^\infty x^p\cdot d\mu(x)=m_p^2
\]
EXPLICIT constructions appear in Carleman on page 85.







\bigskip

\centerline{\bf{Another condition for  uniqueness}}
\medskip

\noindent
Let $\mu $ be as above, i.e. 
denote the class of
\[ 
\int\, t^{2p}\cdot d\mu(t)<\infty
\]
hold for every $p\geq 1$.
\medskip

\noindent
Then we can use the Gram\vvv Schmidt
procedure and construct 
a sequence of polynomials $\{P\uuu\nu\}$
which are orthogonal, i.e.
\[
\int\, P_n(t)\cdot \bar P_m(t)\cdot d\mu(t)=
\text{Kronecker's delta function}
\tag{*}
\]
\medskip

\noindent
where each $P\uuu n(t)$ is monic and of degree $n$.
So to the given measure  $\mu$ belongs
the  series:
\medskip

\[
\mathcal P^*_\mu(z)= \sum_{n=0}^\infty\, |P_n(z)|^2\quad\colon\tag{**}
z=t+is
\]
A sufficient condition for
the uniqueness is given as follows:
 
\medskip

\noindent
{\bf Theorem.}
\emph{Let $\mu\in\mathcal R$ be such that
(**) is divergent for
every $s\neq 0$. Then
$\mu$ is determined by the sequence  $\{c_j\}$.}
\medskip

\end{document}

 


