\documentclass{amsart}
\usepackage[applemac]{inputenc}
\addtolength{\hoffset}{-12mm}
\addtolength{\voffset}{-10mm}
\addtolength{\textheight}{20mm}

\def\uuu{_}


\def\vvv{-}

 
\begin{document}


\centerline{\bf{The heat equation and  Brownian motion}}
\medskip

\noindent
{\bf{Introduction.}}
The heat equation has a long history whose
physical relevance has inspired mathematicians to develop
tools to get  solutions.
A major  step was introduced  by Fourier  which gives a recipe
to solve  the heat equation
with prescribed boundary values.
We  restrict the discussion  to  the 2\vvv dimensional case 
where the basic PDE\vvv equation is
\[
\frac{\partial u}{\partial t}(t,z)= \Delta(u)(z,t)\tag{*}
\]
Here $t$ is a time variable and $z=x+iy$ moves in a domain
$\Omega$  of the complex plane
and is of class $\mathcal D(C^1)$.
The PDE\vvv equation corresponds to 
a  Brownian motion which takes place in
$\Omega$. Suppose that
the density of particles observed at time $r=0$
is observed and expressed  by a density function $g\uuu 0(z)$, i.e. $g$ is  
real\vvv valued and non\vvv negative and its area integral over
$\Omega$ is one.
When $t>0$  classical laws in physics mean that the observed density at some time
$t>0$ is expressed by 
a density function $x\mapsto u(z,t)$ where  $u$ solves (*) and
$u(0,z)= g\uuu 0(z)$ and  the following
boundary condition hold for every $t>0$:
\[ 
\frac{\partial u}{\partial n}(z,t)=0\quad
\colon \quad z\in\partial \Omega\tag{*}
\]


\medskip

\noindent
Above
(*) is an example of a parabolic 
equation for which  uniqueness and existence of solutions with
prescribed initial conditions were proved by Gevrey, Hadamard and Holmgren 
in work  around 1900.
The heat equation can be rewritten as an
integral equation and this   was   used
in pioneering work by Ivar Fredholm which 
leads to    existence theorems and
estimates for associated eigenvalues and 
eigenfunctions under various 
boundary value conditions. 
There also exists a close interplay betwee the heat equation and
the Brownian motion. This was   pointed out by 
Poincar� and   made explicit   by 
Bachelier in the article
\emph{La Bourse}  which contains the essential foundations for
the subject which nowadays bears the popular name
"mathematics of finance".
For an  account and further historic comments
about  the 
Brownian motion we refer to � XX in  \emph{Appendix Measure}.
\medskip


\noindent
Let us now discuss  the  heat
equation in ${\bf{C}}$.
Consider a domain $\Omega$ of class $\mathcal D(C^1)$
whose boundary consists of $p$ many closed and disjoint differentiable Jordan curves.
Here $p$ is some positive integer and  the case $p=1$ is not excluded.
In �� from Chapter V we constructed the Green's function $G(z,\zeta)$
defined in the closed product $\bar \Omega\times\bar \Omega$.
Denote by $C\uuu *(\Omega)$ the Banach space of continuous functions
in $\bar\Omega$ which are zero on the boundary.
There exists the linear operator from
this Banach space into itself defined by
\[
Tu(z)=\iint\uuu\Omega\, G(z,\zeta) u(\zeta) \cdot d\xi d\eta\tag{1}
\]
Recall from �� in Chapter V
that if $\phi$ is a $C^2$\vvv function with compact
support in
$\Omega$ then
\[
\iint\uuu\Omega\, \Delta(\phi)(z) \cdot u(\zeta) dx dy=
\iint\uuu\Omega\, \phi(z) \cdot Tu(\zeta) dx dy
\]
This  means that $Tu$ regarded as a distribution has a Laplacian expressed by
the continuous density $u$, i.e. 
\[
\Delta(Tu)(z)=u(z)\quad\colon\quad z\in\Omega\tag{2}
\]
Let  $u\in C^0\uuu *(\Omega)$ be an eigenfunction
where
\[
u(z)=\mu\cdot Tu(z)
\] 
holds for some non\vvv zero constant $\mu$.
From (2) it follows that
\[
\Delta(u)=\frac{1}{\mu}\cdot u\tag{3}
\] 
which  entails that $u$ is of class $C^2$ in $\Omega$.
Moreover, Green's formula gives:

\[
\mu\cdot \iint\uuu\Omega\, \Delta(u)(z) \cdot u(z) dx dy=
\iint\uuu\Omega\, \nabla(u)^2(z) \cdot dx dy=0\tag{4}
\]
where
$\nabla(u)^2= u\uuu x^2+u\uuu y^2$.
Hence the eigenvalue $\mu$ is real and strictly negative.
\medskip

\noindent
{\bf{1. The spectrum of $T$ and the function $\mathcal D(\lambda)$}}.
Set $G^{(0)}= G$ and define inductively

\[
 G^{(m)}(z,\zeta)=
\mu\cdot \iint\uuu\Omega\, G(z,\zeta) \cdot 
 G^{(m\vvv 1)}(z,\zeta)\, d\xi d\eta
\]
Let $\lambda$ be a new complex parameter and put
\[
\mathcal D(\lambda)= \sum\uuu{m=0}^\infty\,
\lambda^m\cdot G^{(m)}(z,\zeta)
\]


\noindent
We  regard $\mathcal D(\lambda)$ as a function with values
in the  Hilbert space of square integrable functions on 
the product $\Omega\times\Omega$, i.e. we use that
\[
\iint\uuu {\Omega\times \Omega}\, 
|G(z,\zeta)|^2\, d\xi d\eta dxdy<\infty
 \]
and similar finite double integrals occur for the functions
$\{ G^{(m)}\}$.
The general result in �� gives

\medskip

\noindent
{\bf{2. Theorem.}}
\emph{The function $\mathcal D(\lambda)$ extends to a meromorphic function in
the whole complex $\lambda$\vvv plane whose poles are confined
to a sequence of strictly negative real numbers.}
\bigskip

\noindent
{\bf{3. The heat equation.}}
Let $\{\lambda\uuu k\}$ be the poles of $\mathcal D$. 
If the pole has multiplicity $e\uuu k\geq 2$  the corresponding eigenspace
is $e\uuu k$\vvv dimensional.
Repeating eigenvalues with eventual multiplicities we obtain
a sequence of eigenfunctions $\{u\uuu k\}$ with eigenvalues
$\{\lambda\uuu k\}$ and for each $k$ the eigenfunction $u\uuu k$
is normalised so that
\[
\iint u\uuu k^2(x,y)\cdot dxdy=1
\]
and  chosen so that
they form an
orthonormal set in the Hilbert space $L^2(\Omega$. Notice  that
every $u$\vvv function is real\vvv valued.
Next, let $t$ be a new real parameter which
serves as a time variable.
If $\{c\uuu k\}$ is a sequence of complex numbers
we  set
\[ 
p(t,z)=
\sum\uuu{k=1}^\infty\, c\uuu k\cdot e^{\vvv\lambda\uuu k t}\cdot
u\uuu k(z)
\]
The series converges nicely when $t>0$
if $\{c\uuu k\}$ do not increase too
rapidly and
the $p$\vvv function satisfies the PDE\vvv equation
\[
\frac{\partial p}{\partial t}=p\uuu {xx}+ p\uuu {yy}= \Delta(p)
\] 
when $t>0$ and $z\in\Omega$.
Next, the sequence  $\{c\uuu k\}$
determines an initial condition which usually is interpretated via
a limit
\[ 
\lim\uuu{t\to 0}\, p(t,z)= p\uuu *(z)
\] 
where $p\uuu *(z)$ is a distribution.
If $p\uuu *$ belongs to $L^2$ we have for example
\[
c\uuu k=\iint p\uuu *(z)u\uuu k(z)\, dxdy
\]
\bigskip


\noindent {\bf{4. The Brownian  motion.}}
Solutions to the heat equation
correspond to probability densities for a particle whose 
time-dependent change of position is 
governed by a Brownian motion.
If $z\in\Omega$ is given and the particle starts at $z$ at time zero then
we consider the probability distribution:
\[ 
t\mapsto \text{prob}(z,t)
\] 
which gives the probability that the particle stays in $\Omega$ 
up to time $t$.
Since $\Omega$ is bounded the particle eventually hits the boundary where it
is absorbed.
It means that
\[ 
\lim\uuu {t\to\infty}\, \text{prob}(z,t)=0\tag{i}
\]
On the other hand the particle stays in $\Omega$ with high probability under
short time intervals, i.e.
\[
\lim\uuu {t\to 0}\, \text{prob}(z,t)=1\tag{ii}
\] 
Above (i\vvv ii) hold for every $z\in\Omega$.
The function
\[ 
p(t,z)=\text{prob}(z,t)
\]
satisfies the heat equation and by the results in � 3  
given by the series
\[
p(t,z)= \sum\, c\uuu k\cdot 
e^{\vvv\lambda\uuu k t}\cdot
u\uuu k(z)
\quad\text{where}\quad
c\uuu k= \iint\uuu\Omega\, u\uuu k(z)\cdot dxdy\tag{4.0}
\]
\medskip

\noindent
{\bf{4.1 The $E$\vvv function.}}
When the particle starts at a point $z$ we 
consider the expected time before it hits the boundary
which is expressed by the integral
\[ 
E(z)=
\vvv \int\uuu 0^\infty\, t\cdot
\frac{\partial p}{\partial t}(t,z)\cdot dt
\]
Since $p$ satsifies the heat equation and the differential operators
$\partial\uuu t$ and the Laplacian of the  $z$\vvv variable commute
it follows that
\[
\Delta(E)(z)=\vvv \int\uuu 0^\infty\, t\cdot
\frac{\partial^2 p}{\partial t^2}(t,z)\cdot dt\tag{4.1.1}
\]
A partial integration gives
\[
\Delta(E)(z)=\vvv 1
\]
Hence the function
\[ 
E(z)+\frac{|z|^2}{2}
\]
is harmonic in $\Omega$ and since $E=0$ on the boundary
we conclude that
\[
 E(z)=\int\uuu{\partial\Omega}\, P\uuu z(\zeta)\cdot \frac{|\zeta|^2}{2}
 \,d\xi d\eta\vvv \frac{|z|^2}{2}
\]
where $P\uuu z(\zeta)$ is the Poisson kernel which exhibits solutions to
the Dirichlet problem.
\medskip

\noindent
{\bf{4.2 Example.}}
Let $\Omega=\{|z|<R\}$ be a disc. Then
\[
E(z)= \frac{1}{2}(R^2\vvv |z|^2)
\]
Next, let $\Omega=\{ 1<|z|<R\}$ be an annulus.
Then the reader may verify that

\[ 
E(z)=\frac{R^2\vvv 1}{2}\cdot \frac{\log|z|}{\log R}
+\frac{1\vvv |z|^2}{2}\tag{4.2.1}
\]
Notice that $E$ takes its maxium  over the circle of radius $r^ *$
where
\[
r^*=\sqrt{\frac{R^2\vvv 1}{2\ddot \log R}}\tag{4.2.2}
\]
The reader is invited to interpretate
(4.2.1-4.2.2) by  probabilistic considerations.
\bigskip


\noindent
{\bf{4.3 Points of arrival.}}
Let $\omega$ be a  finite union of subintervals of $\partial\Omega$.
Starting the Brownian motion at a point $z\in\Omega$
we  consider the paths which at the first arrival to the boundary
hits points in $\omega$.
Again we get a $p$-function  satisfying the heat equation
and the initial condition depends
upon $\omega$.
More precisely, the probability that a Brownian path
escapes for the first time at a point in $\omega$ is equal to  the value of the 
harmonic measure
function $\mathfrak{m}(\omega,z)$. Set
\[
p\uuu \omega(t,z)=
\sum\, c\uuu k(\omega)\cdot e^{\vvv\lambda\uuu k\cdot t}\cdot u\uuu k(z)
\]
where $\{c\uuu k(\omega)\}$ are determined by
\[
c\uuu k(\omega)=\iint\uuu\Omega\,
\mathfrak{m}(\omega,z)\cdot u\uuu k(z)\cdot dxdy
\]
\medskip

\noindent
{\bf{4.4 A joint probability distribution.}}
Let $\omega\subset \partial\Omega$ be as above
and $t>0$ some fixed time\vvv value. With  $\delta t$ small
we seek the probability that the particle
which starts at some  $z$,
escapes at some point in
$\omega$ 
for the first time during the interval $[t,t+\delta t]$.
From the above this probability up to small ordo of $\delta t$
is given by:
\[
\bigl[\, \sum\, c\uuu k(\omega)\cdot \lambda\uuu k\cdot
e^{\vvv\lambda\uuu k\cdot t}\cdot u\uuu k(z)\,\bigr]\cdot \delta t
\]
\medskip

\noindent
{\bf{4.5 Example.}}
Suppose that the "open window" which  the particle 
wants  to hit on the boundary changes with time.
The probability that it will escape through the changing  window becomes
\[
\sum\, 
[\int\uuu 0^\infty\, c\uuu k(\omega\uuu t) e^{\vvv \lambda\uuu k t}\cdot dt\,]
\cdot
\lambda\uuu k\cdot u\uuu k(z)\tag{*}
\]
\medskip


\noindent
{\bf{4.6 A special case.}}
Let $\Omega$ be  the unit disc
and $z=0$ the starting point.
Let $0<a<\pi$ and suppose that the interval $\omega\uuu t$
is $(\vvv a\cdot |\sin \gamma t|,a\cdot |\sin\gamma t|)$ where 
$\gamma>0$ is a constant.
So the window is closed when $t=0$ and has maximal width at time values when
$|\sin\,\gamma t|=1$.
Here we have:
\[
c\uuu k(\omega\uuu t)=\frac{a\cdot |\sin(\gamma t|}{\pi}
\]
\medskip

\noindent
{\bf{4.7 Remark.}}
The reader may consult text\vvv books for the classic formulas which
determine the sequence of eigenvalues $\{\lambda\uuu k\}$ and
the sequence $\{u\uuu k(0)\}$ in a disc.
So above
we obtain a closed formula for the probability to escape the changing window
but a computer should be used
to obtain a numerical value.
One can  employ Monte Carlo simulations
to determine (*) above. More precisely,
instruct the computer to change
the size of the open window and now the computer
provides  accurate
approximations and there is not difficulty to extend the situation 
when one starts from an
arbitrary point in $D$. Of  course one can  replace the chosen 
"opening function" $|a|\cdot |\sin \gamma t|$ by other time dependent functions
and in  general cases one gets
numerical solutions 
via Monte Carlo simulations, i.e.  one finds
numerical values for
the probability to escape a moving window on the boundary of
an arbitrary domain in
$\mathcal D(C^1)$.



\bigskip





\noindent
Passing to  dimension  $n=3$ is of special interest
one can still employ Mone Carlo simulations and
establish numerical values for many different expected values
as well as higher moments and other 
joint distributions.






\medskip


\centerline{\bf{Schr�dinger's equation.}}
\medskip

\noindent
Quantum mechanics
raised new questions which
give rise  to  non\vvv linear equations.
A precise question was raised by Schr�dinger in the  article
\emph{Th�orie relativiste de l'�lectron et l'interpretation de la
m�canique quantique} from 1932.
In the non\vvv classical case
the previous solution $u(x,t)$   is no longer valid.
Instead another density 
$z\mapsto f\uuu 1(z)\neq u(t\uuu 1,z)$
is observed at some time
$t\uuu 1>0$.
Thus, something highly improbable
but nevertheless possible has occurred
during the time interval $[0,t\uuu 1]$.
Schr�dinger's problem was to find the most likely density a time
$t\uuu 1$ and 
he concluded that  the requested 
time\vvv dependent
density $w(t,z)$ which substitutes $u(t,z)$ above, is found
in a non\vvv linear class of functions
$\mathcal W$ formed by products 
\[ 
u\uuu 0(t,z)\cdot u\uuu 1(z,t)
\]
where $u\uuu 0(z,t)$ is a solution to the same
heat equation as in (*) while $u\uuu 1$ solves the adjoint equation
\[ 
\frac{\partial u\uuu 1}{\partial n}(z,t)=\vvv \Delta(u\uuu 1)(t,z)
\quad\text{for time values}\quad t<t\uuu 1
\]
and satisfies the same boundary condition (1).
Now one seeks a function $w(t,z)$ in the family $\mathcal W$
which satisfies the two  boundary value conditions
\[ 
w(0,z)= f\uuu 0(z)\quad\colon\quad w(t\uuu 1,z)= f\uuu 1(z)
\] 
where $f\uuu 0,f\uuu 1$ is a pair of \emph{prescribed} density functions.
The solution to this  problem  can be   transformed
into a system of non\vvv linear integral equations.
Namely,  for the given domain
$\Omega$ there exists the Poisson\vvv Greens function $K(t,z,\zeta)$
and we have

\[
u\uuu 0(t,z)= \iint\uuu\Omega\,
K(t,z,\zeta)\cdot g\uuu 0(\zeta)\cdot d\xi d\eta
\]
\[
u\uuu 1(t,z)= 
\iint\uuu\Omega\,
K(t\uuu 1\vvv t,z,\zeta)\cdot g\uuu 1(\zeta)\cdot d\xi d\eta
\]
Next, the boundary conditions for $w$ yield

\[
f\uuu 0(z)=
g\uuu 0(z)\cdot 
\iint\uuu\Omega\,
K(\vvv t\uuu 1,z,\zeta)\cdot g\uuu 1(\zeta)\cdot d\xi d\eta
\]
\[
f\uuu 1(z)=
g\uuu 1(z)\cdot 
\iint\uuu\Omega\,
K(\vvv t\uuu 1,z,\zeta)\cdot g\uuu 1(\zeta)\cdot d\xi d\eta
\]
\medskip

\noindent
{\bf{Remark.}}
The solvability of the system above was left open by Schr�dinger
and  put forward to the mathematical community at the IMU\vvv
congress in Z�rich in 1932 by Serge Bernstein.
The article \emph{R�solution d'un systeme d'�quations de Schr�dinger}
by Fortet from 1940 gave a method for succesive approximations
which led to solutions under some specific conditions imposed on the boundary data.
A general method was  introduced by Beurling in the article [Beurling]
which  is exposed in � XX from Special sections and gives
existence of solutions to the system above for smooth domains above
ad boundary data expressed by the   functions
$f\uuu 0,f\uuu 1,g\uuu 0,g\uuu 1$.
Beurling's  solution is
established
via a  non\vvv linear variational problem for product measures.

\medskip

















 





 
 











 
 


























\end{document}