\documentclass{amsart}
\usepackage[applemac]{inputenc}

\addtolength{\hoffset}{-12mm}
\addtolength{\textwidth}{22mm}
\addtolength{\voffset}{-10mm}
\addtolength{\textheight}{20mm}


\def\uuu{_}

\def\vvv{-}

\begin{document}



\centerline{\bf\large XII. A system of infinite linear equations.}

\bigskip
\noindent
{\bf Introduction.}
The main issue in this section is
the construction of a unique solution to
the system
\[ 
\sum\uuu{q\neq p}\,\frac{x\uuu q}{p\vvv q}=0\tag{*}
\] 
where (*) hold for all positive integers $p$ and
$\{x\uuu q\}$ is a sequence of real numbers for which the series
\[ 
\sum\uuu{q=1}^\infty\,\frac{x\uuu q}{q}\tag{**}
\] 
is convergent. The fact that (*) has a non\vvv trivial solution is
far from evident.
Before the study of (*)  in Section 1 we
discuss a  general situation  which was described
by Carleman in his major lecture at
the international congress in Z�rich 1932.
A homogeneous system of linear equations with an inifinite number
of variables
takes the form
\[
\sum_{q=1}^\infty\, c_{pq}x_q=a_p
\quad\colon\quad p=1,2,\ldots\tag{***}
\]
where $\{c_{pq}\}$ is a matrix with an infinite number of elements.
A sequence $\{x_q\}$
of complex numbers
is a solution to (***) if
the sum  in the left hand side converges for each $p$ and has  value $a_p$.
Notice that one does not
require that the series are  absolutely convergent.

\medskip

\noindent
{\bf{The generic case.}}
To every $p\geq 1$
we have the linear form $L_p$
defined on finite sequences
$\{x_1,x_2,\ldots\} $, i.e. where  $x_q=0$ when $q>>0$ by:
\[ 
L_p(x_\bullet)=\sum_{q\geq 1}\, c_{pq}\cdot x_q
\]
Follownig Carleman the generic case occurs when
$c_{1q}\neq 0$ for every $q$ and
the linear forms
$\{L_p\}_1^\infty$ are linearly independent.
The last condition means that for every positive integer
$M$ there exists a strictly increasing sequence
$q_1<\ldots <q_M$ so that the
$M\times M$-matrix with elements
$a_{p\nu}= c_{pq_\nu}$ has a non-zero determinant.

\medskip

\noindent
{\bf{The $R$-functions.}}
Assume that the system (***) is generic.
Given a sequence $\{a_p\}$ 
we seek a solution $\{x_q\}$.
The necessary and sufficient condition for
its  existence  goes as follows:
Consider
$n$-tuples of positive integers $m_1,\ldots,m_n$
where $n\geq 2$. For every such  $n$-tuple and  $1\leq k\leq n-1$
we set
\[ 
\mathcal D(k)=
\{\nu\colon\, m_1+\ldots+m_k<\nu\leq m_1+\ldots+m_{k+1}\}
\]
Next, with $M=m_1+\ldots+m_N$ we denote by
$\mathcal F(m_1,\ldots,m_n)$ the family of sequences $(x_1,\ldots,x_M)$ such that
the following inequalities hold for every $1\leq k\leq n-1$:
\[
\bigl|\sum_{q=1}^{q=\nu}\, c_{pq}x_q-a_q\bigr|\leq
\frac{1}{k}\quad\colon\,\nu\in\mathcal D(k)\quad\text{and}\quad
1\leq p\leq k
\]


\noindent
The generic assumption implies that the set $\mathcal F(m_1,\ldots,m_n)$
is non-empty provided that we start with a sufficiently large $m_1$
and for
every  such $n$-tuple we set
\[
R(m_1,\ldots,m_n)=\min\sum_{\nu=1}^{m_1}\, x_\nu^2
\]
where  the minimum is taken
over
sequences $x_1,\ldots,x_{m_1}$ which give the starting terms of some
sequence 
$x_1,\ldots,x_M\in \mathcal F(m_1,\ldots,m_n)$.
\medskip






\noindent
{\bf Theorem.} \emph{The necessary and sufficient condition in order that
(***) has a least one  solution is that there exists
a constant $K$ and  an infinite
sequence of positive integers $\mu_1,\mu_2,\ldots$ such that }
\[
R(\mu_1,\mu_2,\ldots,\mu_r)\leq K\quad \text{hold for every $r$}
\]

\medskip

\noindent
{\bf Remark.}
The reader may consult [Carleman] for  further remarks
about this result  and also comments upon
the more involved
criterion for non-generic linear systems.
From now on we  study  linear systems which arise
as follows:
Consider a rational function of two variables $x,y$:
\[
f(x,y)=\frac{a_0(x)+a_1(x)y+\ldots a_n(x)y^n}
{b_0(x)+b_1(x)y+\ldots b_m(x)y^m}
\]
Here $n$ and $m$ are positive integers and
$a_\nu(x)$ and $b_j(x)$ polynomials in $x$.
No special assumptions are imposed on these polynomials except
that $b_m(x)$ and $a_n(x)$ are not identically zero.
For example, it is not neccesary that the degree of $b_m$ is
$\geq\text{deg}(b_j)$ for all $0\leq j\leq m-1$.
\medskip

\noindent
{\bf 0.1 Proposition }\emph{ Let $b_m^{-1}(0)$ be the finite set of zeros of $b_m$.
Let $\zeta_0\in {\bf{C}}\setminus b_m^{-1}(0)$ be such that}
\[
\sum_{j=0}^{j=m}\, b_j(\zeta_0)\cdot q^j=0
\]
\emph{holds for some finite set of
positive integers, say $1\leq q_1<\ldots<q_k$.
Then there exists $\delta>0$ such that }
\[ 
\sum_{j=0}^{j=m}\, b_j(\zeta)q^j\neq 0\quad\colon
\text{for all positive integers}\,\, q\, \quad\colon\quad 
0<|\zeta-\zeta_0|<\delta
\]



\noindent
{\bf{Exercise.}}
Prove this result.

\medskip

\noindent
Next, consider the sequence of polynomials
of the complex $\zeta$-variable given by:
\[ 
B_q(\zeta)=b_0(\zeta)+b_1(\zeta)q+\ldots+b_m(\zeta)q^m\,\quad\colon
q=1,2,\ldots
\]

\medskip

\noindent
{\bf 0.2 Proposition} \emph{Put}
\[ 
W=b_m^{-1}(0)\,\bigcup_{q\geq 1}\, B_q^{-1}(0)
\]
\emph{Then $W$ is a discrete subset of ${\bf{C}}$, 
i.e its intersection with any
bounded disc is finite.}
\medskip

\noindent
{\bf{Exercise.}}
Prove this result where a hint
is 
to apply Rouche's theorem.
\medskip

\noindent
Next, put  $W^*=W\cup a_n^{-1}(0)$, i.e. add the zeros of the polynomial $a_n$
to  $W$.
\bigskip

\noindent
{\bf 0.3 Proposition.}
\emph{Let $\zeta_0\in C\setminus W^*$
and suppose that $\{x_q\}$ is a sequence such that the series}
\[ 
\sum_{q=1}^\infty\,
f(\zeta_0,q)\cdot x_q\tag{i}
\] 
\emph{is convergent. 
Then  the series}
\[
\sum_{q=1}^\infty\,
f(\zeta,q)\cdot x_q\quad\text{converges for every}\,\,\zeta\in
C\setminus W^*\tag{ii}
\] 
\emph{Moreover, the series sum is a meromorphic function of $\zeta$
whose poles are contained in $W^*$.}

\medskip

\noindent 
{\bf Remark.}
Proposition 0.3 gives a procedure to find solutions 
$\{x_q\}$ which is not the trivial null solution to a homogeneous system:
\[ 
\sum_{q\neq p}\, f(p,q)\cdot x_q=0\quad\colon\quad
p=1,2,\ldots\tag{*}
\]


\noindent
More precisely, assume that the rational function
$f(x,y)$ is such that
$f(p,q)\neq 0$ when $p$ and $q$ are distinct positive integers.
To get a solution $\{x_q\}$ to (iii)
it suffices to begin with to verify (i) in Proposition 0.3  for some
$\zeta_0$ and  then also try to find
$\{ x_q\}$
so that the meromorphic
function
\[ 
\zeta\mapsto 
\sum_{q=1}^\infty x_q\cdot f(\zeta,q)\tag{**}
\]
has zeros at every positive integer. Using this criterium for
a solution we can show the following:
\medskip

\noindent
{\bf{Theorem.}} \emph{For every complex number
$a\in{\bf{C}}\setminus(-\infty,0\,]$ the system}
\[ 
\sum\uuu{q=1}^\infty \frac{x\uuu q}{p+aq}=0
\quad\colon\quad p=1,2,\ldots
\]
\emph{has no non\vvv trivial solution $\{x\uuu q\}$.}
\bigskip

\noindent
There remains to analyze the case
when $a$ is real and $<0$. In this case
complete answers about possible  
when $f$ is the  rational function were 
established by K. Dagerholm in  his
Ph.D-thesis at Uppsala University from  1938
with
Beurling as  supervisor.
The hardest case occurs when $a=1$
which will be studied in the next section.





\bigskip





\centerline{\bf 1. The Dagerholm series.}

\bigskip
\noindent
Let $\mathcal F$ be the family
of all sequences of
real numbers $x_1,x_2,\ldots$ such that the series
\[ 
\sum_{q=1}^\infty\,\frac{x_q}{q}<\infty\tag{i}
\]
We only require that the series is convergent, i.e. it need  not be
absolutely convergent. 
\bigskip

\noindent {\bf 1.1 Theorem.} \emph{Up to a multiple with a real constant
there exists a unique
sequence $\{x_q\}$ in $\mathcal F$ such that}
\[
\sum_{q\neq p}\, \frac{x_q}{p-q}=0\quad\colon\quad p=1,2,\ldots
\]
\medskip

\noindent
The proof of  uniquenss relies upon Jensen's formula and
the solution to  a specific Wiener-Hopf
equation.
We begin to describe the strategy  of the proof.
For each $\{x_q\}\in\mathcal F$
there exists
the  meromorphic function 
\[ 
h(z)=\sum_{q=1}^\infty\,\frac{x_q}{z-q}\tag {ii}
\]
To see  that $h(z)$ is defined
we notice that if
$s_*$ is the series sum in (i) then
\[
h(z)+s_*=\sum_{q=1}^\infty\,x_q\cdot [\frac{1}{z-q}+\frac{1}{q}]=
z\cdot \sum_{q=1}^\infty\frac{x_q}{q(z-q)}\tag{iii}
\]
It is clear that 
the right hand side
is 
a meromorphic function
with poles confined to  the set of positive integers. 
Hence we obtain the 
entire function:

\[ 
H(z)=\frac{1}{\pi}\cdot \text{sin}(\pi z)\cdot h(z)
\]
\medskip

\noindent
{\bf{1.2 Proposition.}}
\emph{The following hold for each positive integer:}
\[ 
H(p)= (\vvv 1)^p\cdot x\uuu p\quad\colon 
H'(p)= (\vvv 1)^q\cdot 
\sum_{q\neq p}\, \frac{x_q}{p-q}=0
\]


\medskip



\noindent 
\emph{Proof.}
Let $p\geq 1$ be an integer. With $\zeta$ small we have
\[
H(p+\zeta)=\frac{1}{\pi}\cdot\text{sin}(\,\pi p+\pi\zeta)\cdot 
\bigl [\frac{x_p}{\zeta}+
\sum_{q\neq p}\,\frac{x_q}{p+\zeta-q}\bigr ]
\]
A series expansion of the complex sine-function at
$\pi p$ gives
\[
\frac{1}{\pi}\cdot\text{sin}(\,\pi p+\pi\zeta)=
[\zeta\cdot \text{cos}(\pi p)+O(\zeta^3)]\cdot \bigl [\frac{x_p}{\zeta}+
\sum_{q\neq p}\,\frac{x_q}{p+\zeta-q}\bigr ]
\]
Proposition 1.2  follow since
$\text{cos}\,\pi p=(-1)^p$.


\bigskip

\noindent
{\bf{Remark.}}
Proposition 1.2 shows that
$\{x_p\}$ solves the homogeneous system in Theorem 1.1 
if the complex derivative of the  entire $H$-function
has zeros on all positive
integers. This observation is the gateway towards the
proof of Dagerholm's Theorem. But let us   first  establish the uniqueness.


\bigskip

\centerline{\bf 2. Proof of uniqueness}

\bigskip

\noindent
Let   $\{x_q\}$ be a
sequence in $\mathcal F$.
From the constructions in above
it is clear that the  meromorphic function
$h(z)$ satisfies the following in the left half-plane
$\mathfrak{Re}(z)\leq 0$:
\[
\lim_{x\to -\infty}\,h(x)=0
\colon\quad
|h(x+iy)|\leq C_*\quad\colon\,x\leq 0\tag{i}
\]
where $C_*$ is a constant.
Moreover, in the right half-plane there exists  a constant $C^*$ such that
\[
|h(x+iy)|\leq C^*\cdot \frac{|x|}{1+|y|}\quad\colon\, |x-q|\geq \frac{1}{2}\,\,
\text{for all positive integers}\tag{ii}
\]

\noindent
To $h(z)$ we get the entire function $H(z)$ 
and  (i-ii) above give the two
the estimates below in the right half-plane:
\[ 
|H(x+iy)|\leq Ce^{\pi|y|}\,\,\colon\, x\leq 0\quad\colon\,\quad
|H(x+iy)|\leq C\frac{|x|}{1+|y|}\cdot e^{\pi|y|}\tag{iii}
\]
Moreover, the first limit formula in (i) gives
\[
\lim_{x\to-\infty}\, H(x)=0\tag{iv}
\]


\noindent
It is easily seen that the same upper bounds hold
for the entire function $H'(z)$ and a straightforward application
of
the  Phragm�n-Lindel�f theorem 
gives:
\medskip

\noindent
{\bf 2.1 Proposition.}
\emph{The complex derivative of
$H(z)$ 
satisfies the growth condition:}
\[
\lim_{r\to\infty}\,
e^{-\pi r\cdot |\text{sin}\,\phi|}\cdot |H'(re^{i\theta}\,|=0
\quad\colon\,\text{holds uniformly when}\,\,\, 0\leq\theta\leq 2\pi
\]
\medskip

\noindent
Now we are prepared to prove the uniqueness part in Theorem 0.1.
For suppose that we have two  sequences
$\{x_q\}$ and $\{x_q^*\}$ which both give solutions to 
(*) and are not equal up to a constant multiple of each other.
The two sequences  give entire functions $H\uuu 1$ and $H\uuu 2$. Since
both are constructed via real sequences 
their  Taylor coefficients  are real and there exists 
a  linear combination
\[
G=aH\uuu 1+bH\uuu 2
\] 
where $a,b$ are real numbers and
the complex derivative
$G'(0)=0$. The hypothesis that there exists
two ${\bf{R}}$-linearly independent solutions to (*)
leads to a contradiction once we have proved the following
\medskip

\noindent
{\bf{2.2 Lemma}} \emph{The entire function $G'(z)$ is identically zero.}
\medskip

\noindent
\emph{Proof.}
To simplify notations we set
$g(z)=G'(z)$ and
consider the series expansion
\[ 
g(z)= a_pz^p+a_{p+1}z^{p+1}+\ldots
\]
where $a_p$ ia the first non-vanishing coefficient. Since $g(0)=G'(0)=0$
we have $p\geq 1$ and since
the two $x$-sequences both are solutions to
(*), the second equation  in Proposition 0.2 gives
\[ 
g(p)=0\quad\colon\, p=1,2,\ldots\tag{i}
\]

\noindent
Next, $G$ is real-valued on the $x$-axis 
and since the $H$\vvv functions are zero for every integer $\leq 0$ the
same holds for $G$.
\emph{Rolle's theorem} implies
that for every $n\geq1 $
there exists
\[ 
-n<\lambda_n<-n+1\quad\colon\, g(\lambda_n)=0\tag{i}
\]


\noindent
So if $\mathcal N$ is the counting function for the
zeros of the entire $g$-function  one has
the inequality
\[ 
\mathcal N(r)\geq [2r]\tag{iii}
\]
where $[2r]$ is the largest integer $\leq 2r$.
Next, recall that  $a_p$ is the first non-zero term in the series
expansion of $g$. Hence Jensen's formula gives:


\[
\text{log}\,|a_p|+p\cdot\log\, r+
\int_0^r\,\frac{\mathcal N(t)\cdot dt}{t}=
\frac{1}{2\pi}\int_0^{2\pi}\, \text{Log}\, |g(re^{i\theta}\,|\,\cdot d\theta
\tag{*}
\]


\noindent
Proposition 2.1 applied to $g(z)$ gives:
\[
\int_0^{2\pi}\, \text{Log}\, |g(re^{i\theta}\,|\,\cdot d\theta
=2r-m(r)\quad\text{where}\,\lim_{r\to\infty}\, m(r)=+\infty\tag{iv}
\]


\noindent
At this stage we get the contradiction 
as follows. First (iii) gives
\[
\int_0^r\,\frac{\mathcal N(t)\cdot dt}{t}
\geq 2r-\text{Log}(r)-1
\]
Now (*) and   (iii) give the inequality
\[
\text{log}\,|a_p|+p\cdot\log\, r+2r-1-\text{log}\, r\leq
2r-m(r)\quad\colon\,r\geq 1\tag{vi}
\]


\noindent
Here $p\geq 1$ which therefore would give:
\[
\text{log}\,|a_p|-1+m(r)\leq 0
\]
But this is impossible since
we have seen that
$m(r)\to+\infty$.

\bigskip

\centerline{\bf 3. Proof of existence}
\bigskip

\noindent
We start with a general construction.
Let $\phi(z)$ be analytic in the unit disc $D$ which
extends to a continuous function on
$T$ except at the point $z=1$.
We also assume that
there exists some $0<\beta<2$ and a constant $C$ such that
\[ 
|\phi(\zeta)|\leq C|1-\zeta|^{-\beta}\tag{1}
\]
This implies that
the function
\[ 
\theta\mapsto \theta\cdot \phi(e^{i\theta})
\]
is integrable on the unit circle. Hence
there exists the entire function
\[
f(z)=\frac{-i}{2\pi}\int_{-\pi}^{\pi}
e^{-i\theta z}\cdot \theta\cdot \phi(e^{i\theta})\cdot d\theta\tag{2}
\]
\medskip

\noindent 
Next, with $\epsilon>0$ small we let $\gamma_\epsilon$ be the interval of
the circle $|z-1|=\epsilon$ with  end-points at the intersection with
$|z|=1$. So on $\gamma_\epsilon$ we have 
\[
z= 1+\epsilon\cdot e^{i\theta}\quad\colon\,
-\frac{\pi}{2}+\epsilon\uuu *<\theta<\frac{\pi}{2}-\epsilon\uuu *
\]
where $\epsilon_*$ is small with $\epsilon$.
We obtain
the entire function
\medskip
\[
F(z)=  \frac{1}{2\pi}\,\int_{\epsilon}^\pi
e^{-i\theta\cdot z}\cdot \phi(e^{i\theta})d\theta+
\frac{1}{2\pi}\, \int_{-\pi}^{-\epsilon}
e^{-i\theta\cdot z}\cdot \phi(e^{i\theta})d\theta+
\frac{1}{2\pi i}\int_{\gamma_\epsilon}\,
\frac{e^{-z\cdot\text{Log}\zeta}\cdot 
\phi(\zeta)d\zeta}{\zeta}
\]
\medskip

\noindent
If $z=n$ is an integer we
have 
\[
e^{-in\theta}=\zeta^{-n}\quad\colon\quad
e^{-n\cdot\text{Log}\zeta}=\zeta^{-n}
\]

\noindent 
Hence we get
\[
F(n)=\frac{1}{2\pi i}\cdot
\int_{\Gamma_\epsilon}\,\frac{\phi(\zeta)\cdot d\zeta}{\zeta^{n+1}}\tag{*}
\]
\medskip

\noindent
where $\Gamma_\epsilon$ is the closed curve
given as the union of $\gamma_\epsilon$ and the
interval of $T$ where
$|\theta|\geq\epsilon$.
Cauchy's formula applied to $\phi$ gives:

\medskip

\noindent {\bf 2.1 Proposition.} \emph{Let $\phi(\zeta)=\sum c_n\zeta^n$.
Then} 
\[
F(n)=c_n\quad\colon\,n\geq 0\quad\text{and}\quad  F(n)=0
\quad\, n\leq -1
\]


\noindent
Next, using (i) above we also have:
\medskip


\noindent 
{\bf 2.2 Proposition.} \emph{The complex derivative of $F$ is equal to $f$.}

\medskip

\noindent
\emph{Proof.}
With $\epsilon>0$ the derivative of the sum of first
two terms from the construction of $F(z)$ above
become
\[
\frac{1}{2\pi}\,\int_{|\theta|\geq\epsilon}\,
-i\theta\cdot e^{-iz\theta} \phi(e^{i\theta})d\theta\tag{i}
\]
\medskip

\noindent
In the last integral derivation with respect to $z$ gives
\[
-\frac{1}{2\pi i}\int_{\gamma_\epsilon}\,
\frac{e^{-z\cdot\text{Log}\zeta}\cdot\phi(\zeta)d\zeta}{\zeta}
 \tag{ii}
\]
Now $\zeta=1+\epsilon\cdot e^{i\theta}$ during the
integration along $\gamma_\epsilon$ which gives:
\[ 
|\text{Log}\,(1+\epsilon\cdot e^{i\theta})|\leq\epsilon
\]
At the same time the circle interval $\gamma_\epsilon$ has length
$\leq\epsilon$
and hence the growth condition  (i)
shows that the integral  (iii) tends to zero when
$\epsilon\to 0$.
Finally, since we assumed that 
the function $\theta\mapsto \theta\cdot\phi(e^{i\theta})$ 
is absoutely integrable on $T$ a passage to the limit as
$\epsilon\to 0$ gives $F'=f$ as requested.


\medskip

\noindent 
{\bf 2.3 Conclusion.}
If $n$ is a positive integer in Proposition 2.3 we have:
\[ 
F'(n)=
\frac{-i}{2\pi}\int_{-\pi}^{\pi}
e^{-in \theta }\cdot \theta\cdot \phi(e^{i\theta})\cdot d\theta\tag{**}
\]


\noindent
These integrals are zero for every
$n\geq 1$ if and only if
$\theta\cdot \phi(e^{i\theta})$ is the boundary value function of
some $\psi(z)$ which is analytic in the exterior disc
$|z|>1$. In 2.X we will show that
this is true for a specific $\phi$\vvv function 
satisfying the growth condition (1) above and in addition
the series

\[ 
\sum\uuu{n=1}^\infty\, (\vvv 1)^n\cdot \frac{c\uuu n}{n}
\]
converges.
\bigskip

\noindent
{\bf{2.4 How to deduce a solution $\{x\uuu p\}$.}}
Suppose we have found $\phi$ satisfying the conditions above
which gives the entire function $F(z)$
whose derivatives are zero for all $n\geq 1$.
Now we set

\[ 
x\uuu p=(\vvv 1)^p\cdot c\uuu p
\]
By (***) this sequence belongs to $\mathcal F$ and we construct the 
associated entire function $H(z)$.
From (i) in Proposition  0.1 and Proposition 2.1 we get

\[ 
H(p)=(\vvv 1)^p\cdot x\uuu p=c\uuu p=F(p)
\]
In addition both $H$ and $F$ have zeros at all integers $\leq 0$.
Next, by the construction of $F$ it is clear that this 
is an entire function of exponential
type and by the above the entire function $G=H\vvv F$
has zeros at all integers.
We leave as an exercise to the reader to show that
$G$ must be identically zero. The hint is to use similar
methods as in the proof of
the uniqueness.
It follows that
\[ 
H'(q)= F'(q)=0
\]
for all $q\geq 1$.
By (ii) in Proposition 0.2 this means precisely that
$\{x\uuu p\}$ is a solution to the requested equations in (*)
which gives the existence in Dagerholm's  Theorem.

\bigskip


\centerline {\bf{2.5 The construction of $\phi$.}}
\medskip


\noindent
There remains to find $\phi$ such that the conditions above hold.
To obtain $\phi$ we start with the integrable function
on $T$ defined by:
\[ 
u(\theta)=\frac{1}{2}\cdot \log \,\frac{1}{|\theta|}
\quad\colon\, -\pi<\theta<\pi
\]
We get the analytic function
\[ 
g(\zeta)=\frac{1}{2\pi}\int_{-\pi}^\pi\,
\frac{e^{i\theta}-\zeta}{e^{i\theta}+\zeta}\cdot
u(\theta)\cdot d\theta
\]


\noindent
In the exterior disc
we find the analytic function
\[
\psi(\zeta)=\text{exp}\,\, -\bar g(\frac{1}{\bar\zeta})
\]


\noindent
Let us also put
$\phi\uuu *(\zeta)=e^{g(\zeta)}$.
Now we have

\[
\log |\phi(e^{i\theta})|=\mathfrak{Re}\, g(e^{i\theta})=
u(\theta)
\]
In the same way we see that
\[
\log |\psi(e^{i\theta})|=\vvv \mathfrak{Re}\, g(e^{i\theta})=
\vvv u(\theta)
\]
Since $2u(\theta)= \vvv \log\,|\theta|$
it follows that
\[
\log\,|\theta|+\log |\phi\uuu *(e^{i\theta})|=\log |\psi(e^{i\theta})|
\]
Taking exponentials  we obtain
\[
|\theta|\cdot|\phi\uuu *(e^{i\theta})|= |\psi(e^{i\theta})|
\]
\medskip

\noindent
{\bf{Exercise.}}
Check also arguments and verify that
we can remove
absolute values in the last equality to attain
\[
|\theta|\cdot \phi\uuu *(e^{i\theta})= \psi(e^{i\theta})\tag{*}
\]

\medskip

\noindent
Here (*) is not precisely what we want since
our aim was to construct $\phi$ so that
$\theta|\cdot \phi(e^{i\theta})$ is equal to the boundary
value of an analytic function in
$|z|>1$.
So in order to get rid of the absolute value for $\theta$ in
(*) we modify $\phi\uuu *$ as follows: Set
\medskip

\noindent 
\[
\rho(\theta)=\frac{\pi i}{2}\cdot
\text{sign}\,\theta\cdot e^{-i\theta}\quad\colon 
-\pi<\theta<\pi
\]
\medskip

\noindent 
Next, consider the two analytic functions
in $D$, respectively in $|z|>1$ defined by:

\[ \phi\uuu 1(z)=
\frac{1}{\sqrt{1-z^2}}\quad\text{and}\quad \, \psi\uuu 1(z)=
\frac{1}{\sqrt{1-z^{-2}}}
\]
\medskip

\noindent {\bf {Exercise.}}
Show that one has the equality

\[
\rho(\theta)=\frac{\phi\uuu 1(e^{i\theta})}
{\psi\uuu 1(e^{i\theta})}
\] 
when $\vvv \pi<\theta <\pi$ and $\theta\neq 0$.

\medskip

\noindent
{\bf{The $\phi$\vvv function.}}
it is defined by
\[ 
\phi(z)=\frac{z}{\sqrt{1\vvv z^2}}\cdot \phi\uuu *(z)
\]
From (*) above and the construction of $\rho$
it follows that
\[ 
\theta\cdot \phi(e^{i\theta})=
\frac{\pi}{2}\cdot  \psi\uuu 1(e^{i\theta})\cdot\psi(e^{i\theta})
\]
The right hand side is the boundary function of an
analytic function in $|z|>1$ and hence $\phi$ satisfies (**) from
XX.
Consider its Taylor expansion
\[ 
\phi(z)=\sum\, c\uuu n\cdot z^n
\]
There remains to verify that the series (**) converges and
that $\phi$ satisfies the growth condition in XX.
To prove this
we begin to analyze the function
\[ 
\phi\uuu *(z)= e^{g(z)}\tag{i}
\]







\noindent
Rewrite the $u$ function as a sum
\[
u(\theta)=
\frac{1}{2}\log\,\bigl |\frac{1}{1-e^{i\theta}}\bigr |+
k(\theta)\quad\text{where}\quad \, k(\theta)=
\frac{1}{2}\log\,\bigl  |\frac{1-e^{i\theta}}{\theta}\bigr |\tag{ii}
\]


\noindent
When $\theta$ is small we have an expansion
\[
\frac{1-e^{i\theta}}{\theta}=-i+\theta/2+\ldots\tag{iii}
\] 
From this we conclude that the $k$-function is
at least twice differentiable as a function of $\theta$. So  the Fourier
coefficients in the expansion
\[ 
k(e^{i\theta})=\sum\, b_\nu e^{i\nu \theta}\tag{iv}
\]
have a good decay. For example, there is a constant $C$ such that
\[
 |b_\nu|\leq \frac{C}{\nu^2}\quad\colon\, \nu\neq 0\tag{v}
\]
This implies that the analytic function
\[
\mathcal{K}(z)=
\frac{1}{2\pi}\cdot\int_{-\pi}^{\pi}\,
\frac{e^{i\theta}+z}{e^{i\theta}-z}\cdot k(e^{i\theta})\cdot d\theta\tag{vi}
\]
yields a 
bounded analytic function in the unit disc.
Next, the construction of the $g$-function gives:

\[
g(z)= \frac{1}{2}\cdot \log\,\frac{1}{1-z}+
\sum\, b_\nu z^\nu\implies\tag{vii}
\]
\[ 
\phi\uuu *(z)=\frac{1}{\sqrt{1-z)}}\cdot e^{\mathcal K(z)}
\]
\medskip

\noindent
We conclude that

\[
\phi(z)= 
\frac{z}{1\vvv z}\cdot
\frac{1}{\sqrt{1+z)}}\cdot e^{\mathcal K(z)}
\]
Since $\mathcal K(z)$ extends to a continuous function on the closed disc
it follows that $\phi$ satisfies the growth condition (1) with $\beta=1$.
Moreover, the function $\theta\cdot \phi(e^{i\theta})$ belongs to $L^1(T)$
since
$\frac{1}{\sqrt{1+e^{i\theta}}}$ is integrable. 
There remains only to prove:

\medskip

\noindent
{\bf{Lemma.}} The series
\[
\sum\uuu{p=1}^\infty\, (\vvv 1)^n\cdot\frac{c\uuu n}{n}
\]
is convergent.
\medskip

\noindent
\emph{Proof.}
let us put

\[
A(z)= 
\frac{z}{\sqrt{1+z)}}\cdot e^{\mathcal K(z)}
\]
This gives

\[
\phi(z)=\frac{A(1)}{1\vvv z}+\frac{A(z)\vvv A(1)}{1\vvv z}
\]
From (v) it follows that $K(z)$, and hence also
$e^{K(z)}$ is differentiable at $z01$
which gives the existence of a constant $C$ such that

\[
\bigl|\frac{A(z)\vvv 1}{1\vvv z}\bigr|\leq
C\cdot \frac{1}{|\sqrt{1+z}|}
\]
Here the function $\theta\mapsto \frac{1}{|\sqrt{1+e^{i\theta}}|}$
belongs to $L^p(T)$ for each $p<2$ which by the  inequality for 
$L^p$\vvv norms between functions and their Fourier coefficients
in XX
for example implies that
if $\{c\uuu \nu^*\}$ give the Taylor series for
$\frac{A(z)\vvv 1}{1\vvv z}$ then
\[
\sum\, |c^*\uuu\nu|^3<\infty
\]
Now H�lder's inequality gives





\[
\sum\, \frac{|c^*_\nu|}{\nu}\leq \bigl(\sum\,|c_\nu^*|^3\bigr)^{\frac{1}{3}}\cdot
\bigl(\sum\,\nu^{-3/2}\bigr)^{\frac{2}{3}}<\infty\tag{8}
\]

\noindent
We conclude that the Taylor series for $\phi$ becomes


\[
A(1)\cdot(1+z+z^2+\ldots)+\sum c_\nu^*z^\nu
\]
Hence $c\uuu n=A(1)+c^*\uuu\nu$ and now
Lemma xx follows since the alternating series
$\sum\,(-1)^n\frac{1}{n}$ is convergent
and we have the absolute convergence in XX above.


\newpage

\centerline{\bf{A theorem by Kjellberg.}}
\bigskip


\noindent
{\bf{Introduction.}}
We  expose an article by Bo Kjellberg - former Ph.D-student of Beurling -
which deals with a
comparison between integrals and certain sums
of analytic functions of exponential type in a half-space.
Here is the situation: Let $f(z)$ be analytic in
the closed half-spae
$\mathfrak{Re}\, z\geq 0$.
Assume that the function $x\mapsto |f(x)|$ is bounded on the real $x$-axis
and there exists a positive real number
$c$ such that
\[
 \limsup_{|z|\to\infty}\, \frac{\log\,|f(z)|}{|z|}= c\tag{1}
\]
Next, let
$\phi(t)$ be a $C^2$-functionon $t\geq $ with $\phi(0)=0$ for which
\[
\phi'(t)\geq 0\quad\colon\quad  t\cdot \phi''(t)+\phi'(t)\geq 0
\]
Thus, $\phi$ is non-decreasing and the last inequality means that it is
convex
as a function of $\log t$.
\medskip

\noindent
{\bf{ Gap sequences.}}
A strictly increasing sequence $\{\lambda_n\}$ has gaps of order $\geq \delta$ 
if
\[
\lambda_{n+1}-\lambda_n\geq 2\cdot\delta\quad\colon\, n=1,2,\ldots
\]
\medskip

\noindent
{\bf{0.1 Theorem.}}
\emph{Let $f$ and $\phi$ be as above.
Then, for every $\delta>0$ and each
$\lambda$-sequence with gaps of order $\geq \delta$
one has the implication}

\[
\int_0^\infty\, \phi(|f(x)|)\, dx<\infty\implies
\sum_{n=1}^\infty\, \phi(e^{-\delta\cdot c}\cdot |f(\lambda_n)|)<\infty\tag{*}
\]
\medskip

\noindent
{\bf{Remark.}}
The result above gave an affirmative anser to a question posed by Boas
in the article \emph{Inequalities between
series and integrals involving entire functions}.
The proof of Theorem 0.1 has two ingredients. First one has 
a uniqueness result   which goes as follows:

\medskip

\noindent
{\bf{0.2 Theorem.}}
\emph{Let $f$ and $\phi$
be as above where (1) holds for $f$ and
the integral in the
left hand side of Theorem 0.1 is finite. Then, if}
\[
\int_0^{1/2}\, \frac{\phi(t)}{t\cdot \log\,\frac{1}{t}}\,dt=+\infty
\]
\emph{it follows that $f$ is identically zero.}


\medskip

\noindent
{\bf{Remark.}} We shall first prove Theorem 0.2 and notice that
it then is suffcient to prove Theorem 0.1 under the
\emph{additional assumption} that the integral in Theorem 0.2 is finite.
In both theorems the convexity of $\phi$ with respect to $\log t$
will
be used. More precisely we first notice that if
$H(x,y)$ is a harmonic function defined in some open
set in
${\bf{C}}$ then
\[
u(x,y)= \phi(e^{H(x,y)})
\] 
is subharmonic. Indeed , this follows since
\[
u_x= H_x\cdot e^H\cdot \phi'(e^H)\implies
u_{xx}=H_{xx}e^H\cdot \phi'(e^H)+
H_x^2e^H(e^H\cdot \phi''(e^H)+ \phi'(e^H))
\]
A similar equation holds for $u_{yy}$ and adding the result we use that
$\Delta(H)=0$ and
see that (2) above entails that
$\Delta(u)\geq 0$.
We shall need the following prelimninary result:
\medskip


\noindent
{\bf{1.1 Proposition.}}
\emph{Let $u(x)$ be a continuous and
non-negtive function on $x\geq 0$ where}
\[ 
\limsup_{x\to+\infty}\, u(x)=+\infty
\]
\emph{and 
$g$ is a stricty increasing function on $x>0$.
Then the implication below holds:}
\[
\int_0^\infty\, g(u(x))\, dx<\infty
\quad\text{and}\quad
\int_1^\infty\, \frac{g(u)}{u}\, du=+\infty
\implies
\int_1^\infty\, \frac{u(x)}{x^3}\, dx=+\infty
\]
\medskip

\noindent
\emph{Proof.}
Let $v(x)$ be the non-decreasing funftion which is equi-distributed with
$u$ as explained in ��. Then we have
\[
\int_1^\infty\, \frac{v(x)}{x^3}\, dx\leq
\int_1^\infty\, \frac{u(x)}{x^3}\, dx
\]
while the left and side integrals are unchanged when
$u$ is replaced by $v$. Hence it suffices to prove
the result for $v$, i.e. from now on we  assume
that $u$ is non-decreasing.
If $0<a<A$ a partial integration gives
\[
\int_a^A\, g(u(x))\, dx=
\quad\text{and}\quad
\]

Easy finish ....


\bigskip

\noindent
{\bf{1.2 Majorisations in quarter-planes.}}
We are given the constnst $c$ which implies that if $c'>c$ then there is a 
constant $B$ such that
the inequality below holds in the right half-plane:
\[ 
|f(x+iy)|\leq B\dot e^{c'\dot |y|}
\]
Let us then choose $c^*>c'$ snd set

\[ 
\psi(z)= \phi(e^{ic^*z}\cdot f(z)|)
\] 
By (xx) $\psi$
is a subharmonic function in the quarter plane
$U_*=\{x>0\}\times \{y>0\}$ and here
\[ 
\psi(x+iy)=\phi(e^{-c^*\cdot y}\cdot |f(x+iy)|)
\]
The subharmonicity of $\psi$ entails that
\[
\psi(z)\leq h_1(z)+h_2(z)
\] 
where
$h_1(z)$ is the harmonic extension of the boundary value function
which is
$\phi|f(x)|$ on the positive real axis and zero on
the positive $y$-axis, while $h_2(x,0)=0$ and
\[
h_2(0,y)= \phi(e^{-c^*\cdot y}\cdot |f(iy)|)
\]
Since $c^*>c'$ it follows from (xx) that $h_2(0,y)$ is bounded and 
tends to zero as $y\to\infty$
The growth properties of the two $h$-functions on
the boundsry of $U_*$ entail that
both are represented as in � XX.
Thus, one has

\[ 
h_1(x+iy)=\frac{1}{\pi}\int_0^\infty\,
y\cdot \psi(\xi)\cdot\bigl[\frac{1}{(\xi-x)^2+y^2}-
\frac{1}{(\xi+x)^2+y^2}\bigr]\, d\xi
\]

\[ 
h_2(x+iy)=\frac{1}{\pi}\int_0^\infty\,
x\cdot \psi(i\eta )\cdot\bigl[\frac{1}{(\xi-x)^2+y^2}-
\frac{1}{(\xi+x)^2+y^2}\bigr]\, d\eta
\]
From (i) we obtain
\[
\int_0^\infty\,h_1(x+iy)\,dx=
\frac{2}{\pi}\int_0^\infty\,
\psi(\xi)\cdot \arctan\,\frac{\xi}{y}\, d\xi\leq
\int_0^\infty\,\psi(\xi)\,d\xi\tag{1}
\]
For $h_2$ we obtain
\[
\int_0^\infty\,h_1(x+iy)\,dx=\frac{1}{\pi}\int_0^\infty\,
\psi(i\eta)\cdot \log\, \bigl|\frac{\eta+y}{\eta-y}\bigr|\, d\eta
\leq C+ A\cdot
\int_{2\delta}^\infty\,
\frac{\psi(i\eta)}{\eta}\, d\eta\tag{2}
\]
where $A$ and $C$ are two constants which the reader may derive from
(x-x) above,
Let us now estaimte the last integral in (2).
Since $\phi$ is non-decreasing we see that (xx) and (xx) give
have
\[ 
\psi(i\eta)\leq\phi(B\cdot e^{(c'-c^*)\eta})
\]
Consider the variable substitution
\[ 
t= B\cdot e^{(c'-c^*)\eta})\implies \frac{dt}{t}=-Bc^*\cdot d\eta
\]
Notice that it gives

\[
\eta=\frac{1}{c^*-c'}\cdot \log \frac{B}{t}
\]
It follows that
\[
\int_{2\delta}^\infty\,
\frac{\phi(B\cdot e^{(c'-c^*)\eta}}{\eta})\, d\eta=
(c^*-c')\cdot \int_{t_0}^\infty\, \frac{\phi(t)}{t\cdot \log\,\frac{B}{t}}\, dt
\]
\medskip


\noindent
{\bf{Conclusion.}}
Under the hypothesis that the integral (**) in Theorem 0.2 converges and 
integral in the left hand side in Theorem 0.1
also converges, it follows that for every $\delta>0$
there exis  constants $C_1,C_2$
which may depend upon $\delta$ such that
\[
\int_{C_1}^\infty\, (h_1(x+iy)+ h_2(x+iy))\, dy\leq C_2
\quad\colon\,\,\, 0\leq y\leq \delta
\]
Together with
the majorization in (xx) it follows that with another constant $C_3$ we have
\[ 
\iint_{\square_\delta}\, \psi(x+iy)\, dxdy\leq C_3
\]
where $\square_\delta= \{0<x<\infty\}\times \{0<y<\delta\}$.

\bigskip

NOW easy to finish the proof of theorems....






\end{document}