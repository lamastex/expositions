\documentclass{amsart}
\usepackage[applemac]{inputenc}
\addtolength{\hoffset}{-12mm}
\addtolength{\textwidth}{22mm}
\addtolength{\voffset}{-10mm}
\addtolength{\textheight}{20mm}


\def\uuu{_}

\def\vvv{-}










\begin{document}




\centerline
{\bf{On spectra of some compact operators.}}
\bigskip

\noindent
{\bf{Introduction.}}
We shall establish some scattered results aobut
spectra of linear operators.
A crucual tool is Carleman's inequality for norms of resolvents
of matrices, expressed by their Hilbert-Schmidt norm.
Since this important result is seldom treated in text-books
devoted to "abstract linear algebra" we have included
details in a separate appendix.
In � xx we give an application where one regards
compact operators on Hlbert spaces.
In � xx we study a specific case where 
the positviity of a kernel for an integral and compact operator on
$L^2[0,1)$ implies that
the spectral values are qyite ample.
In � 3 we prove  the Fredolm's resolvents
yiled entire analytic function whose growth
is controlled in an optimal fashgion. Here analytic function theory
intervenes, i.e. the proof relies upon several
basic results, foremost due to Lindel�f, Phragm�n and Wiman.


\bigskip

\centerline{\bf{� 1. On the spectra of compact operators on Hilbert spaces.}}
\bigskip

\noindent
Let $\mathcal H$ be a Hilbert space.
Recall that if  $T$ is a compact operator on 
$\mathcal H$, then there exists the compact self-adjoint operator
$\sqrt{T^*T}$. Denote by
$\{\mu_n(T)\}$ its discrete spectrum
arranged so that sequence 
$\{\mu_n(T)\}$ is non-increasing, where eventual 
multiplicitiesare counted as usual.
\medskip

\noindent
{\bf{1.1 Definition.}} \emph{For each $p>0$ we denote by
$\mathcal C_p$ the class of compact operators on
$\mathcal H$ such that}
\[
\bigl(\sum_{n=1}^\infty \, \mu_n(T)^p\,\bigr) ^{\frac{1}{p}}<\infty
\]
\medskip

\noindent
Next, denote by $\text{sp}(T)$ the set of all vectors
$x\in\mathcal H$ for which there
exists some non-zero complex number
$\lambda$ and some integer $n\geq 1$ such that
\[
(\lambda E-T)^n(x)=0\tag{*}
\]
\medskip

\noindent
For  every positive integer $N$ we have operator $T^N$
and get the image space
 $T^N(\mathcal H)$.
We shall find sufficient conditions in order that
\[
T^N\mathcal H\subset\text{sp}(T)\tag{**}
\] 
holds for some positive integer $N$. To achieve this we shall study 
the resolvent operator
\[ 
R(\lambda)=(\lambda\cdot E-T)^{-1}
\]
defined outside the discrete spectrum of $T$.
Let $\gamma$ be a simple Jordan arc
which has the origin as one end-point while
$\gamma^*=\gamma\setminus\{0\}$ stays outside the spectrum of $T$.
Now  $R(\lambda)$ exists for every $\lambda\in \gamma^*$
and we can compute  operator norms which leads to:
\medskip

\noindent
{\bf{1.2 Definition.}}
\emph{A Jordan arc $\gamma$ as above is called $T$-escaping of order
$N$ if there exists a constant $C$ such that}
\[
||R(\lambda)||\leq C\cdot |\lambda|^{-N}\quad\colon\quad 
\lambda\in \gamma_*
\]
\medskip

\noindent
Next, let 
$\gamma_1,\ldots,\gamma_s$ be a finite family of Jordan arcs as above
whose  intersections 
with a small  punctured disc $D^*(\delta)=\{0<|z|<\delta\}$
gives a disjoint famlily of curves
$\{\gamma^*_\nu\}$.
Then 
$D^*(\delta)$
is decomposed into  $s$ many pairwise disjoint
Jordan domains, each of which is bordered by a pair of
$\gamma^*$-curves.
Let  $\rho>0$ be some positive number.
We impose the geometric condition that every  Jordan
domain above is contained
in a sector where $\text{arg}(z)$ stays in an interval of
length
$<\rho$ as $z$ varies in the Jordan domain.
Denote by $\mathcal J(\rho)$ the class of all finite families of
Jordan curves for which
these sector conditions hold.
\medskip


\noindent
{\bf{1.3 Theorem.}}
\emph{Let $T$  be a  compact
operator of class $C_p$ for some $p>0$ and
suppose there exists
a family  $\{\gamma_\nu\}$  
which belongs to $\mathcal J(\pi/p)$
where  each $\gamma_\nu$ is $T$-escaping
order $N$. Then  one has the inclusion}

\[ 
T^N(\mathcal H)\subset \text{sp}(T)
\]
\bigskip


\noindent
{\bf{Remark.}}
This theorem  is announced and proved in
[Dunford-Schwartz: Theorem XI.9.29 on page 1115].
The first step in the proof  is  straightforward
reduces the proof to the case
when the compact operator $T$ is of Hilbert-Schmidt type.
We remark only that for this reduction one uses
the fact that $T$ belongs to $C_p$ for some $p>0$.
The major step is
to extend 
Carleman's inequality for matrices from the appendix 
to  Hilbert-Schmidt operators acting on infinite-dimensional 
Hilbert spaces. After this one easily finishes 
the proof is by standard applications of the
Phragm�n-Lindel�f inequalities.

\medskip

\noindent
{\bf{Example and a question.}}
Let
$\mathcal H=L^2[0,1]$ and consider the operator
$T$  defined by
\[ 
T_u(x)=\int_0^1\,
\frac{k(x,y)\cdot u(y)}{|x-y|^\alpha}\,dy
\]
where $1/2<\alpha<1$
and $k(x,y)$ is a real-valued continuous function on
the closed unit square.
Then  $T$ is
compact and we leave to the reader to verify
the existence of a $p$-number which depend upon $\alpha$
such that
$T$ belongs to the class $C_p$.
It would be interesting to investigate
how one can produce integers $N$ to get the inclusion
in
Theorem 1.3.
In such an investigations one can   try 
specified $k$-functions and in particular regard  the symmetric 
case when $k(x,y)= k(y,x)$. 









\newpage









\centerline{\bf{� 2. Distribution of eigenvalues for a class of singular operators.}}
\bigskip


\noindent
Let $f(x,y)$ be a continuous function on the unit square
$0\leq x,y\leq1 $ which is symmetric, i.e.
$f(x,y)=f(y,x)$.
When $0<\alpha<1$
we set
\[
k(x,y)= 
\frac{f(x,y)}{|x-y|^\alpha}
\]
and get the linear operator
\[
K_\alpha(u)(x)=\int_0^1\, k(x,y)u(y)\, dy
\]
Obne easily verifies that 
$K_\alpha$ is a compact operator on the
Hilbert space
$L^2[0,1]$, and 
since $k$ is symmetric the eigenvalues are real and non-zero.
Let $\{\lambda_n^+\}$ be the positive eigenvalues arranged in a non-decreasing order.
Similarly $\{\lambda_n^-\}$ is the set of negative eigenvalues where
the sequence $\{-\lambda^-_n\}$ is non-decreasing. 
The eigenvalues correspond eigenfunctions
$\{\phi_n^+\}$ and $\phi_n^-\}$. In particular 
\[ 
K_\alpha(\phi_n^+)=\lambda_n^+\cdot \phi_n^+
\]


\medskip

\noindent
{\bf{2.1  Theorem.}} \emph{If $f(x,x)>0$ hold on some open interval $x_0<x<x_1$
 it follows that}
 \[ 
 \sum_{n=1}^\infty\,\bigl(\frac{1}{\lambda_n^+}\bigr)^{\frac{1}{1-\alpha}}=
 +\infty
\]
\medskip

\noindent
During the proof
we use the following notation for real-valued functions $u$ in $L^2[0,1]$:
\[
\langle K_\alpha u,u\rangle=
\iint f(x,y)u(x)u(y)\, dxdy
\]
and for  a pair of real-valued $L^2$-functions $u,v$ we set
\[
\langle u,v\rangle=\int_0^1\, u(x)v(x)\, dx
\]
 
\medskip

\noindent
We shall need the following result whose proof is left as an exercise to the reader:
\bigskip

\noindent
{\bf{2.2 Proposition.}}
\emph{For each $u\in L^2[0,1]$ one has the equality}
\[
\langle Ku,u\rangle=
\sum\, \frac{1}{\lambda_n^+}\cdot
\langle u,\phi_n^+\rangle^2+
\sum\, \frac{1}{\lambda_n^-}\cdot
\langle u,\phi_n^-\rangle^2\tag{*}
\]
\medskip

\noindent
\emph{Proof of Theorem 2.1.}
Let $m$ be a positive integer. Since $\{\lambda_n^-\}$ are negative
(*) gives:
\[
\langle Ku,u\rangle\leq
\sum_{n=1}^m\, 
\frac{1}{\lambda_n^+}\cdot
\langle u,\phi_n^+\rangle^2+
\sum_{n=m+1}^\infty\, 
\frac{1}{\lambda_n^+}\cdot
\langle u,\phi_n^+\rangle^2
\]


\noindent
Since $\{\lambda_n^+\}$ is non-decreasing the last sum above is majorized by
\[
\frac{1}{\lambda_m^+}\cdot
\sum_{n=m+1}^\infty\, 
\langle u,\phi_n^+\rangle^2\leq \frac{1}{\lambda_{m+1}^+}\cdot \langle u,u\rangle
\]
where the last inequality follows from Bessel's inequality since
the eigenfunctions $\{\phi_n^+\}$ form an orthonormal family.
Hence the following inequality holds for every positive integer $m$:
\[
\langle Ku,u\rangle\leq
\sum_{n=1}^m\, 
\frac{1}{\lambda_n^+}\cdot
\langle u,\phi_n^+\rangle^2+\frac{\langle u,u\rangle}{\lambda_{m+1}^+}\tag{i}
\]
\medskip

\noindent
Let $\psi_1,\ldots,\psi_m$ be some
orthonormal $m$-tuple in $L^2[0,1]$.
We can apply (i) to each $\psi$-function and a summation over 
$1\leq k\leq m$ gives:

\[
\sum_{k=1}^{k=m}\, 
\langle K\psi_k,\psi_k\rangle\leq
\sum_{k=1}^{k=m}  \sum_{n=1}^{n=m}\, 
\frac{1}{\lambda_n^+}\cdot
\langle \psi_k,\phi_n^+\rangle^2+
\frac{m}{\lambda_{m+1}^+} \tag{ii}
\]
Another application of Bessel's inequality gives for each
$1\leq n\leq m$:

\[
\sum_{k=1}^{k=m}\,
 \langle \psi_k,\phi_n^+\rangle^2\leq \langle\phi_n^+,\phi_n^+\rangle=1
\] 
Hence (ii) entails that

\[
\sum_{k=1}^{k=m}\, 
\langle K\psi_k,\psi_k\rangle\leq
\sum_{n=1}^{n=m}\, \frac{1}{\lambda_n^+}+\frac{m}{\lambda_{m+1}^+}\tag{iii}
\]
\medskip

\noindent
\emph{A choice of $\psi$-functions.}
By assumption we find an interval $[x_0,x_1]$ where
$f(x,x)>0$. Set $d=x_1-x_0$ and when
$m$ is a  positive integer we 
define $\psi_1,\ldots,\psi_m$ where
\[
\psi_k(x)=\sqrt{\frac{m}{d}} \quad\text{when}\quad
x_0+(k-1)\frac{d}{m}<x<x_0+k\frac{d}{m}
\]
\medskip

\noindent
while $\psi_k=0$ outside the intervals above.
The  continuity of $f$ gives
some large integer $m_*$
and a positive constant $\delta$ such that
if $m\geq m_*$ then
$f(x,y)\geq \delta$ on each  small square
where 
$\psi_k(y)\cdot\psi_k(x)\neq 0$. Denote this small square by
$\square_k$ which gives  the inequality below 
for each $1\leq k\leq m$:
\[
\langle K\psi_k,\psi_k\rangle
\geq \delta\cdot\frac{m}{d} \cdot 
\iint_{\square_k} \frac{dxdy}{|x-y|^\alpha}
\]
An easy calculation shows that the double integral
over $\square_k$ becomes
\[
\frac{2}{(1-\alpha)(2-\alpha)}\cdot \bigl( \frac{d}{m}\bigr)^{2-\alpha}
\]
So with $A=\delta\cdot \frac{2}{(1-\alpha)(2-\alpha)}$ one has the inequality
\[
\sum_{k=1}^{k=m}\, 
\langle K\psi_k,\psi_k\rangle
\geq A\cdot \bigl( \frac{d}{m}\bigr)^{1-\alpha}\cdot m= Ad^{1-\alpha}\cdot m^\alpha
\tag{iv}\]
\medskip

\noindent
By  construction  $\psi_1,\dots,\psi_m$ is an orthonormal family
and hence (iii) holds which together with (iv)
gives the inequality below for every  $m\geq m_*$:
\[
Ad^{1-\alpha}\cdot m^\alpha\leq 
\sum_{n=1}^{n=m}\, \frac{1}{\lambda_n^+}+\frac{m}{\lambda_{m+1}^+}\tag{v}
\]


\noindent
At this stage we shall argue by a contradiction, i.e. we  prove that
(v) prevents that the positive  series in Theorem 1 converges.
Namely, suppose that
\[ 
\sum\, \bigl(\frac{1}{\lambda_n^+}\,\bigr)^{\frac{1}{1-\alpha}}<\infty
\]
Since the terms in this positive series decrease with $n$ it follows 
that
\[
\lim_{m\to \infty}\, 
m\cdot \bigl(\frac{1}{\lambda_m^+}\,\bigr)^{\frac{1}{1-\alpha}}=0
\]
So if $\epsilon>0$ we can find $m^*\geq m_*$ such that

\[
m\cdot \bigl(\frac{1}{\lambda_m^+}\,\bigr)^{\frac{1}{1-\alpha}}<\epsilon
\implies \frac{1}{\lambda_m^+}<\bigl(\frac{\epsilon}{m}\bigr)^{1-\alpha}
\]
Hence (v) gives the following when $m>m^*$:

\[
Ad^{1-\alpha}\cdot m^\alpha\leq
\sum_{n=1}^{n=m^*}\, \frac{1}{\lambda_n^+}+
\sum_{\nu=m^*+1}^{n=m}
(\frac{\epsilon}{\nu}\bigr)^{1-\alpha}+m\cdot (\frac{\epsilon}{m+1}\bigr)^{1-\alpha}
\]

\noindent
The middle sum above is majorized by 

\[
\epsilon^{1-\alpha}\cdot\int_{m^*}^m\, \frac{dx}{x^{1-\alpha}}
=\frac{\epsilon^{1-\alpha}}{\alpha}\cdot m^\alpha
\]
At the same time we notice that the last term is $\leq \epsilon^{1-\alpha}\cdot m^\alpha$
and after a division with $m^\alpha$ we  obtain
\[
Ad^{1-\alpha}\leq m^{-\alpha}\cdot 
\sum_{n=1}^{n=m^*}\, \frac{1}{\lambda_n^+}+\epsilon^{1-\alpha}(
\frac{1}{\alpha}+1)
\]
Above $A$ and $d$ are fixed positive constants 
while we can choose arbitary large $m$ and arbitrary small $\epsilon$. 
This gives a contradiction
and Theorem 2.1 is proved.







\newpage

\centerline{\bf{� 3. An entire spectral function.}}




\bigskip


\noindent
{\bf{Introduction.}} Theorem 1 was proved by 
Carleman in the article
\emph{Sur le genre du d�nominateur $D(\lambda)$
de Fredholm}. The proof uses some basic results about entire functions 
due to Poincar�, Lindel�f, Phragm�n  and Wiman and offers an instructive lesson
in analytic function theory.
Let $k(x,y)$ be a continuous function on the unit square
$\{0\leq x,y\leq 1\}$.
We do not assume that $k$ is symmetric, i.e. $k(x,y)\neq k(y,x)$
can hold.
To each $n$-tuple of points $\{s_\nu\}$ on
$[0,1]$ we assign the determinant function
\[ 
K(s_1 ,\ldots,s_n)
=\text{det}
\begin{pmatrix}
k(s\uuu 1,s_1)&\cdots&k(s_1,s_n)\\
\cdots &\cdots&\cdots \\
\cdots &\cdots&\cdots\\
k(s_n,s_1)&\cdots &k(s_n,s_n)\\
\end{pmatrix}
\]
Set
\[ 
c_n= \int_{\square_n}\,
K(s_1 ,\ldots,s_n)\cdot ds_1\cdots ds_n
\]

\medskip

\noindent
where the integral is taken over the $n$-dimensional unit cube.

\medskip

\noindent
{\bf{3.1. Theorem.}} \emph{Put}
\[ 
D(\lambda)=1+ \sum_{n=1}^ \infty\, \frac{(-1)^n}{n!}\cdot c_n\cdot \lambda^n
\]
\emph{Then $D$ is an entire function of the form}
\[ 
D(\lambda)= e^{a\lambda}\prod\,(1-\frac{\lambda}{\lambda_\nu})
\cdot e^{\frac{\lambda_\nu}{\lambda}}
\] 
\emph{where $a$ is some complex constant and}
\[ 
\sum\, \frac{1}{|\lambda_\nu|^2}<\infty
\]



\noindent
{\bf{Remark.}} 
Prior to Carleman's result above, 
Schur proved that $D(\lambda)$ is an entire function
of the form
\[
D(\lambda)= e^{a\lambda+b\lambda^2}\prod\,(1-\frac{\lambda}{\lambda_\nu})
\cdot e^{\frac{\lambda_\nu}{\lambda}}
\]
for some second constant $b$.
The novelty in [Carleman] is that $b=0$ always holds.
Above we assumed that $k$ is a continuous kernel.
This condition was later relaxed by Carelan in  the article [� xx. 1919]
which gives
Theorem 1 when 
$k(x,y)$  is a kernel of the Hilbert-Schmidt type, i.e. it suffices to assume that 
\[
\iint\, |k(x,y)|^2\, dxdy<\infty
\]

\newpage



\centerline{\bf{Proof of Theorem 3.1 }}

\medskip

\noindent
First we 
approximate $k$ by polynomials. If $\epsilon>0$ we find a 
polynomial $P(x,y)$ such that the maximum norm of $k-P$ over
the unit square is $<\epsilon$. Write
\[ 
k(x,y)= P(x,y)+B(x,y)
\] 
So now $|B(x,y)|<\epsilon$ for all $0\leq x,y\leq 1$.
To each pair $0\leq p\leq n$ we set

\[ 
B_p(s_1,\ldots,s_n)
=\text{det}
\begin{pmatrix}
P(s\uuu 1,s_1)&\cdots&k(s_1,s_n)\\
\cdots &\cdots&\cdots \\
P(s_p,s_1)&\cdot&P(s_p,s_n)\\
B(s_{p+1},s_1)&\cdots& B(s_{p+1},s_n)\\
\cdots &\cdots&\cdots\\
B(s_n,s_1)&\cdots &B(s_n,s_n)\\
\end{pmatrix}
\]
It is easily seen that
\[ 
c_n=
\sum_{p=0}^{p=n}\,
\binom {n}{p}\cdot  \int_{\square_n}\, B_p(s_1,\ldots,s_n)\cdot ds_1\cdots ds_n\tag{i}
\]
\medskip

\noindent
Next, let $N$ be the degree of the polynomial $P(x,y)$.
The reader can verify that
the first $p$ row vectors in the matrix which
defines
$B_p(s_1,\dots,s_n)$ are linearly independent as soon as $p>N$
which therefore gives $B_p=0$. So for every $n\geq N$ one has the equality
\[
c_n=\sum_{p=0}^{p=N}\,
\binom {n}{p}\cdot  \int_{\square_n}\, B_p(s_1,\ldots,s_n)\cdot ds_1\cdots ds_n\tag{ii}
\]
Next, if $M$ is the maximum norm of $k(x,y)$,
and $\epsilon<M$  the maximum norm of
$P$ is $\leq 2M$.  Hadamard's
determinant inequality in � xx gives
\[
|B_p(s_1,\ldots,s_n)|\leq (2M)^p\epsilon^{n-p}\cdot n^{\frac{n}{2}}\tag{iii}
\]
Next, when $n\geq p$ and $0\leq p\leq n$ we set

\[
c_n(p)=
\binom {n}{p}\cdot  \int_{\square_n}\, B_p(s_1,\ldots,s_n)\cdot ds_1\cdots ds_n
\]
Then (iii) gives:
\[ 
|c_n(p)|\leq\binom {n}{p}\cdot(2M)^p\epsilon^{n-p}\cdot n^{\frac{n}{2}}\tag{iv}
\]
Next, recall that
$\binom{n}{p}\leq \frac{n^p}{p !}$ and hence (iv) gives
\[ 
|c_n(p)|\leq \frac{(2M)^p}{p !}\cdot \epsilon^{n-p}\cdot n^{\frac{n}{2}+p}\tag{v}
\]
\medskip

\noindent
At this stage we return to the $D$-function. To each $0\leq p\leq N$
we set
\[ 
D_p(\lambda)=\sum_{n=p}^\infty\, \frac{-1)^n}{n !}\cdot c_n(p)\lambda^n
\]



\medskip

\noindent
{\bf{Sublemma.}}
\emph{For each $p$  we have}
\[ 
\lim_{|\lambda|\to +\infty}\,
e^{-4\epsilon|\lambda|^2}\cdot D_p(\lambda)=0\tag{*}
\]

\medskip

\noindent
{\bf{Exercise.}} Prove (*). The hint is to 
use (v) above and Lindel�f's wellknown 
asymptotic formula for entire functions. See my notes in analytic 
function theory if necessary.

\medskip

\noindent
Next, we notice that
(ii) gives an equation
\[ 
D(\lambda)=q(\lambda)+ \sum_{p=0}^{p=N}\,D_p(\lambda)\tag{vi}
\] 
where $q(\lambda)$ is a polynomial of degre $N-1$ at most.
This 
entails that the entire function
$D(\lambda)$ also satisfies (*) in the Sublemma.
Let $\{\lambda\uuu\nu\}$ be the zeros of $D(\lambda)$. Then (*) and 
a classic result due to Poincar�  gives  the entire function
\[ 
F(\lambda)=
\prod\,(1-\frac{\lambda}{\lambda_\nu})\cdot e^{\frac{\lambda}{\lambda_\nu}}
\quad\&\quad
\lim_{|\lambda|\to +\infty}\,
e^{-\delta|\lambda|^2}\cdot F(\lambda)=0
\quad\text{for all}\quad \delta>0\tag{**}
\]


\noindent
To profit upon (**) we use a device introduced by Lindel�f.
Let $\omega= e^{2\pi i/5}$ which  gives
the entire function 
\[ 
G(\zeta)= F(\zeta^5)\cdot F(\omega\zeta^5)\cdots F(\omega^4\zeta^5)\tag{vii}
\]
\medskip



\noindent
The right hand side in  (**) entails  
that the entire function $G$ has order
$<1/2$. Then a result due to   Wiman  gives the existence of 
an increasing  sequence 
$\{R_k\}$ which tends to $+\infty$  such that
\[
\min_\theta\, |G(R_ke^{i\theta})|\geq 1
\quad\colon\,k=1,2,\ldots
\tag{viii}
\]
<Choose  $\lambda$-circles with $r_k^5=R_k$
and  Poincar�'s limit  from (**). Then it is clear
that (vii) and (viii)
entail that for every $\delta>0$ there exist some  $k_*$ such that
\[ 
k\geq k_*\implies
\max_\theta\, \frac{1}{|F(r_ke^{i\theta})}|\leq e^{\delta r_k^2}\tag{ix}
\]
Finally, we have the zero-free entire function
\[ 
H(\lambda)= \frac{D(\lambda)}{F(\lambda)}
\]
In (ix) we can  take $\delta=\epsilon$ which gives
\[
\limsup_{k\to +\infty}\, e^{-5\epsilon\cdot r_k^2}\cdot 
\max_\theta\, |H(r_ke^{i\theta})|=0
\]
\medskip

\noindent
Liouville's theorem entails that the entire function
$\log H(z)$ must be a linear polynomial and since
$D(0)=1$ we
conclude that
\[ 
D(\lambda)= e^{a\lambda}\cdot F(\lambda)
\] 
for a constant $a$ which finishes the proof of Theorem 1.
\bigskip


\noindent
{\bf{Remark.}} Above we took several facts from analytic function
theory for granted.
At the end we appealed to the classic Liouiville Theorem.
A much stronger version was proved by Carleman and
for readers who are not so familiar with analytic function
theory
we include this result together with some details of the proof.
\medskip


\centerline{\bf{The general Liouville theorem.}}











\end{document}

\medskip

\noindent
{\bf{9.1 Theorem.}}
\emph{The function $\lambda\mapsto u\uuu\lambda(x)$ with values in the Banach space
$B=C^0[0,1]$ extends to a meromorphic $B$\vvv valued 
function in the whole
$\lambda$\vvv plane.}
\bigskip

\noindent
This result is due to Carleman in[Carleman].
Notice that we have only assumed that the maximum norm of
$k$ is $<1$ while the kernel in general can be non\vvv symmetric, i.e.
$k(x,y)=k(y,x)$ is not assumed.


To establish Theorem 9.1 we use Hadamard's theorem, i.e for each
$0\leq x\leq 1$ we take $\{f\uuu n(x)\}$ and obtain
for a given $p\geq 2$ the matrix $C\uuu n^{(p)}(x)$ as above.
With these notations we shall prove

\medskip

\noindent
{\bf{Proposition 9.2}} \emph{For every $p\geq 2$ and $0\leq x\leq 1$ onr has
the inequality}

\[
\bigl |\,  \text{det}( C\uuu n^{(p)}(x))\,\bigr|\leq 
(p\, !)^{\vvv n}\cdot \bigl( p^{\frac{p}{2}}) ^n\cdot \frac{p^p}{p\,!}
\]
\medskip

\noindent
The inequality (*) entails that

\[ \limsup\uuu{n\to \infty}\, \bigl[\text{det}( C\uuu n^{(p)}(x))\,\bigr|\,\bigr )^{1/n}
\leq 
\frac{p^{p/2}}{p\,!}
\]
Passing to the $p$:th roots the reader may using Stirling's formula that
\[
\lim\uuu{p\to \infty}\frac{p^{1/2}}{p\,!}^{\vvv 1/p}=0
\]
Hence Hadamard's theorem gives
Theorem 8.2.
\bigskip

\noindent{\bf{8.4 Proof of Proposition 8.3.}}
To prove this inequality we shall use some 
general determinant formulas of independent interest.
Let $\phi\uuu x),\ldots,\phi\uuu p(x)$ and
 $\psi\uuu x),\ldots,\psi\uuu p(x)$
be a pair of $p$\vvv tuples of continuous functions on
$[0,1]$.
For each point $(s\uuu 1,\ldots,s\uuu p)$ in
$[0,1]^p$ we put

\[ 
D\uuu\phi(s\uuu 1,\ldots,s\uuu p)=
\Phi(x\uuu 1,\ldots,x\uuu p)
=\text{det}
\begin{pmatrix}
\phi\uuu 1(x\uuu 1)&\cdots&\phi\uuu 1(x\uuu p)\\
\cdots &\cdots&\cdots \\
\cdots &\cdots&\cdots\\
\phi\uuu p(x\uuu 1)&\cdots &\phi\uuu 1(x\uuu p)\\
\end{pmatrix}\quad\colon\quad
\]
In the same way we define $D\uuu\psi(s\uuu 1,\ldots,s\uuu p)$.
We also get the $p\times p$\vvv matrix with elements

\[
a\uuu{jk}= \int\uuu 0^1\, \phi\uuu j(x)\cdot \psi\uuu k(x)\cdot dx
\]
With these notations we have

\medskip

\noindent {\bf{Lemma.}}\emph{One has the equality}
\[
\text{det}(a\uuu{jk})=
\frac{1}{p\,!}\int\uuu {[0,1] ^p}\,
D\uuu\phi(s\uuu 1,\ldots,s\uuu p)\cdot 
D\uuu\psi(s\uuu 1,\ldots,s\uuu p)\cdot ds\uuu 1\cdots ds\uuu p
\]

\medskip

\noindent
{\bf{Exercise.}} Prove this result using standard formulas for
determinants.

\medskip

\noindent
The next step towards the proof of Proposition xx 
uses the following equality below. First, inductively
we get the sequence $\{k^{(m)}(x)\}$
starting with $k=k^{(1)}$ we put:
\[
k^{(m)}(x)= \int\uuu 0^1\, k^{(m\vvv 1)}(x,s)\ddot k(s)\cdot ds
\quad\colon\quad m\geq 2
\]
Now the reader can verify that

\[
f\uuu{n+m}(x)= \int\uuu 0^1\, k^{m)}(x,s)\cdot f\uuu n(s)\cdot ds
\]
hold for all pairs $m\geq 1$ and $n\geq 0$.
\medskip

\noindent
On $[0,1]^{p+1}$ we
define the function

\[ 
\mathcal K(x,s\uuu 1,\ldots,s\uuu p)
=\text{det}(XXXX)
\]
next, on $[0,1]^p$ we define for every pair $n,p$ the function
\[
\mathcal F\uuu n^{(p)}(s\uuu 1,\ldots,s\uuu p)
=\text{det}(XXXX)
\]
\bigskip

\noindent
{\bf{Exercise.}}
Use Lemma XX and (xx) above to show the two equalities:

\[
\mathcal F\uuu n^{(p)}(s\uuu 1,\ldots,s\uuu p)=
\frac{1}{p\,!}\cdot
\cdot \int\uuu{[0,1]^p}\,{\bf {K}}(s\uuu 1,\ldots,s\uuu p,t\uuu 1,
\ldots ,t\uuu p)\cdot \mathcal F\uuu {n\vvv 1}^{(p)}(t\uuu 1,\ldots,t\uuu p)
\cdot dt\uuu 1\cdots d t\uuu p
\]


\[
\text{det}(\mathcal C\uuu n^{(p)}(x))=
\frac{1}{p\,!}\cdot \int\uuu{[0,1]^p}\, 
\mathcal F\uuu n^{(p)}(s\uuu 1,\ldots,s\uuu p)\cdot 
\mathcal K(s\uuu 1,\ldots,s\uuu p)
\cdot ds\uuu 1\cdots ds\uuu p
\]
\medskip

NOW easy to finish ....






\end{document}











