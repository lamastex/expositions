\documentclass{amsart}
\usepackage[applemac]{inputenc}
\addtolength{\hoffset}{-12mm}
\addtolength{\textwidth}{22mm}
\addtolength{\voffset}{-10mm}
\addtolength{\textheight}{20mm}


\def\uuu{_}

\def\vvv{-}

\begin{document}



\centerline{\bf{9. Neumann-Poincar�  boundary value problems}}

\bigskip

\noindent
{\bf{Introduction.}}
Several fundamental  results were
achieved by Carl  Neumann in his pioneering  article
emph{xxx} from 1877.
Apart from an elegant solution to the
Dirichlet problem, Neumann solved boundary
valued problems where  double-layer potentials occur.
His results were carried out in dimension 3.
The  strategy was to  use resolvents and 
prove
that
certain poles are absent while
meromorphic extensions of  operator-valued  series are constructed
which after has become
a widespread tool to study elliptic differential equations.
Neumann's studies were   confined to convex
domains
where
certain majorizations become straighforward since one deals with 
positive integral kernels. 
The extension  to  general domains with a $C^2$-boundary  
was achieved by Poincar� in the article \emph{xxx} from 1897
where  new methods were introduced to overcome
the failure of positivity for the
non-symmetric kernel defining the double-layer potential.
In � xx we expose Poincar�'s solution to the Neumann problem in
dimension two and mention only that
a similar proof is available in higher dimension.
In these proof analytic function theory appears which
to establish existence results for elliptic
boundary value problems.
For example, a crucial point during the proof of Theorem
xx is that
one first has found a certain meromorphic operator-valued function
and following Poincar� one shows that uniqueness entails that
a pole is absent at the critical value $\lambda=1$.

\medskip

\noindent
Here we shall  restrict the  discussion  to planar domains
$\Omega$ borded by a finite union of closed Jordan arcs
of class $C^2$. 
Let $\Gamma$ denote $\Omega$ and 
following Poincar� in his article
\emph{La m�thode de Neumann et le probl�me de Dirhclet}
one introduces the certain  linear operators acting on
continuous functions on
$\Gamma$ as follows:
\medskip

\noindent
 Denote by ${\bf{n}}$  the
\emph{inner normal} to $\Gamma$, i.e. the unit normal along
$\partial\Omega$ which is directed  into the domain, and 
let $ds$ be   the arc-length.
When $\psi$ and $\phi$ are continuous functions on
$\Gamma$ we put:
\[
\mathcal N_\psi(x)= \int_\Gamma\, \frac{\langle x-p,{\bf{n}}(x)\rangle}
{|x-p|^2}\cdot \psi(p)\, ds(p)
\quad\&\quad 
\mathcal D_\phi(x)= \int_\Gamma\, \frac{\langle x-p,{\bf{n}}(p)\rangle}
{|x-p|^2}\cdot \phi(p)\, ds(p)\tag{*}
\]
We refer to $\mathcal N$ as the Neumann operator and
$\mathcal D$ as the Dirichlet operator.
Poincar� proved that both these linear operators acting on the
Banach space $C^0(\Gamma)$ 
have a real spectrum whose sole cluster point is at the origin
and the eigenspace at
every spectral point is 1-dimensional.
This  can be expressed by saying that the operator valued functions
\[
\lambda\mapsto(\mathcal N-\lambda\cdot E)
\quad\&\quad
\lambda\mapsto(\mathcal D-\lambda\cdot E)
\tag{**}
\]
are  holomorphic in a disc centered at the origin and extend
to  meromorphic functions in  the complex parameter $\lambda$
with a discrete set of simple real poles whose absolute values tend to
$\infty$.



\medskip


\centerline{\bf{The case when $\Gamma$ has corner points.}}

\medskip

\noindent
Certain existence theorems  for planar domains where
isolated corner points are allowed were 
established by Zarmela in 1904.
Further results which led  to
quite precise facts about the spectra for operators as above 
were obtained 
in Carleman's thesis \emph{xxx} from 1916.
A novelty in this work was  that solutions to equations of the form
\[
\mathcal N_\psi=\lambda\cdot \psi
\quad\&\quad
\mathcal D_\phi= \lambda\cdot \phi\tag{0.1}
\]
were stided for  functions which only 
are  integrable on the boundary.
This leads to new phenomena for the spectra of the operators in (*),
i.e. they need not  
be  confined to discrete subsets of the
complex $\lambda$-plane since the equations (0.1) can have 
$L^1$-solutions for values of $\lambda$ where no
continous function appears.
The reason is that the oprecence of corner pints
means that the operators in (*) are unbounded
when they act on a space like $L^1(\Gamma)$.
Let me  recall some  results from Carleman's 
thesis which I personally find to be
worth studying even today  by  readers  interested in
elliptic boundary value problems.
As far as I know  the sharp results from [ibid] for planar domains
have not been improved  in more recent literature
and  methods initiated in [ibid]
have benn adopted by generations of authors later n.


\medskip

\noindent
Suppose that $\Gamma$ has a finite set of corner points
where interior angles $\{\alpha_1,\ldots,\alpha_M$ appear
and give numbers
\[
R_\nu= \frac{\pi}{|\pi-\alpha_\nu|}>1\tag{i}
\]
Notice that a small $R$-value means that the 
corresponding corner point is rather  accute.
When the corner points appear 
a first result in [ibid] goes as follows.

\medskip

\noindent
{\bf{Theorem.}}
\emph{Let $R_*$ be the smallest number from  (i),
Then each of 
the operator-valued functions in (**) acting on $C^0(\Gamma)$ 
are  meromorphic in the open disc
$|\lambda|<R_*$ with a finite number of poles
which are all real and simple.}
\medskip


\noindent
A more involved analysis appears in the second chapter from
[ibid] where Carleman studies the spectrum in the closed complement of
the disc of radius $R_*$.
Consider for example a given continuous function $f$ on
$\Gamma$ and the equation
\[
\psi-\lambda\cdot \mathcal N_\psi=f\tag{i}
\]
Under the hypothesis that $f$ is H�lder continuous of some order
$\delta>0$ close to the corner point where
the minimal $R$-value is attained,
the solutions to (i) as $\lambda$ is close to $R_*$ can be
expressed via an algebraic function of $\lambda$ with
a branch  at $\lambda=R_*$.
For the precise
expansions which occur  we refer to [ibid].
\medskip

\noindent
One may also consider the homogenous equations
\[
\psi-\lambda\cdot \mathcal N_\psi=0
\quad\&\quad
\phi-\lambda\cdot \mathcal D_\phi=0
\tag{ii}
\]
Carleman  proved non-zero 
continuous solutions only occurs at
a discrete set of real numbers which
tends to $\infty$  and  the spectra  are the same for 
Neumann's and Dirichlet's operators.
Moreover, the dimesion  of the solution spaces in
(ii) are equal at every
spectral $\lambda$-value.


\medskip

\noindent
One can also 
solutions to (ii) where $\psi$ or $\phi$  only are integrable with
respect to the arc-length meaure,.
In that case Carleman  proved that the spectrum 
becomes  quite extensive, i.e. it has positive area and  contains
an assigned geometrically  define planar domain
which is described explicityl in [ibid].
It   would briung us too far to discuss this further since
it requires  familarity about unbounded linear operstors.
But  readers  interested to
learn about examples where peculiar
spectra  of linear operators appear
should consult Carleamn's thesis
which provides "concrete" and yet
highly non-trivial examples.
A    survey of  (ibid]  appears in
the article \emph{Sur les recherches de M. Carelab relatives fonctions harmoniques}
(Comptes Rendus  vol. 43 1920) by Erik Holmgen.
\medskip

\noindent
{\bf{PDE-equations in unbounded domains.}}
At the Scandinavian congress in matheamtics 1925, Niles Bohr
put forward the phyciual signiciance of solutions in
$L^({\bf{R}}^3)$ to equations of th form
\[
\Delta(u)+W\cdot u+\lambda\cdot u=0
\]
where $W(p)$ is a potentia, function 
of the form
\[
W(p)=\sum\, \frac{m_\nu}{|p-q_\nu|}
\]
with the sum taken over a
finite set of pints
$\{q_\nu\}$ in
${\bf{R}}^3$ and $\{m_\nu\}$ are postivie real numbers.
It turns out that (*) has non-zero $L^2$-solutions for
a discrete set of real numbers
which tend to $+\infty$.
The first demonstration of this result is due to carelan who
gave an affirative answer to Bohr's
raides questi on via an immediate appication of this general theory
about 
\emph{unbounded self-adjoint operators on Hilbert spaces}.
which was created in his 
monograph  \emph{Sur les �quations singuliers
� noyeau r�el et symmetrique} 
published by  Uppsala University is 1923, a work which can be seen as a historic landmark
in
operator theory. Here one finds the 
contructions of spectral resolutions  and their properties
assigned to an  unbounded self-ajoint operator
whuch has
a wide range of applications.
Hundreds  - ot reven thousands - of articles
have later exploited Carleman's original work.
Concerning the equation (*) it is a special case of
more general equations where one seeks $L^2$-functions $u$ which satisfy

\[
\Delta(u)+c(x,y,z)\cdot u+\lambda\cdot u=0
\]
where $c$ is a real-valued and locally square integrable function
in
${\bf{R}}^3$.
The operator
\[
L=\Delta+c(x,y,z)
\]
is ab example of a denseyl defined and symmetric operator on
the Hilbert space
$L^2({\bf{R}}^3)$.
There remains to find conditions on $c$ in order that
$L$ is self-adjoint. In Carelan's cited work
this is equivspent to the condition that
for every pair $u,v$ in $L^2$ for which
$L(u)$ and $L(v)$ also are square integrable over
${\bf{R}}^3$, one has the equation
\[
\int\, u\cdot L(v)\, dm=
\int\, L(u)\cdot v\, dm
\]
where $dm$ is the Lebesgue measure in
${\bf{R}}^3$.
No general \emph{sufficient  and necessary} condfitions for
$L$ to be self-adjoin t are known.
However, one can seek for sufficient con ditions on the $c$-functions
in order that (xx) holds.
The following  result of  was
presented by Carelan in his lectures at instuut Pincar� helt in May 1930.
It assets that (xx) holds under the condition that
the locally suare integrable function $c$ satisfies
\[
\limsup\, c(x,y,z)<+\infty
\]
where the limes superior is taken as
$|x|^2+|y|^2+|z|^2\to +\infty$.
Actually the more precise
condition to be imposed upon $c$ is that
for everu $u$ siuch that both $u$ and $L(u)$ belong to
$L^2({\bf{R}}^3$, it follows that
\[
\liminf_{r\to \infty}\,
\int_{B(r)}\, c\cdot u^2\, dm<\infty
\]
where $B(r)$ denote the open balls of radius $r$ centered at the origin.
As far as I know this is still the shapest known sufficency  resut in order that
$L$ is self-adjoint.
As expected Carelan's sufficency theorem
cannot be derived by
"abstract reasoning". However, the proof is not too cumeersome and is presented
with all details in Carleman's article
where systemtic use of Newton's fundaemtnal solution to the Laplace operator in
${\bf{R}}^3$ is used.
\bigskip

\noindent
{\bf{Exercise.}}
To each $\rho>0$ and every pair of points $p,q$ in
${\bf{R}}^3$ we set
\[
A_\rho(p,q)= \frac{1}{|p-q|}+\frac{|p-q|}{\rho^2}-\frac{2}{\rho}
\]
Consider an open domain
$\Omega$ in ${\bf{R}}^3$ and a Lebesgue measurable function
$\phi$ in $\Omega$ with the property that
for  every point $p\in \Omega$ there exists a small
ball centered at $p$ such that
\[
\int_{B_p(\epsilon)}\, \frac{\phi(q)}{|p-q|}\,dm<\infty
\]
So in particular $\phi$ is locally integrable.
Apply Newton's classic formulas to show that
a function $u$ satisfies the equation
\[
\Delta(u)=\phi
\] 
in $\Omega$ if and only if the following is valid: For 
each  point $p\in \Omega$ and every 
$\rho>0 $ such that   the ball $B_p(\rho)$ is a relatively compact subset of
$\Omega$, one has the equation
\[
u(p)-\frac{1}{2\pi\cdot \rho^2}\cdot
\int_{B_p(\rho)}\, \frac{u(q)}{|p-q|}\, dm(q)+
\frac{1}{4\pi}\cdot \int_{B_\rho(p)}\, A_\rho(p.q)\cdot u(q)\, dm(q)=0
\]
\medskip


\noindent
Let us  remark that Newton's criterion for solutions to
the equation (xx) is a gateway to analyze 
operators such that $L=\delta+c$.
See also � xxx where we expose Carelan's proof of
the sufficency in order that the $L$-operator above is self-adjoint.
For another  intructive lesson  dealing with
spectral resolutions of unbounded self-adjoint operators
we refer to � xx which treats Carleman's solution to
equations related to propagation of sound in from the cited monograph [xxx].
Here one is led to more refined questions
about  the spectral resolution, namely if
they are exhibited by \emph{aboslutely continuous}
spectral functions.
In � xx we follow Carleamns' original wojk and prove the
absolute contoinuity for
a class of boundsry value probelms
of the Neumann type where solutions are $L^2$.functions defined over
unbounded domains in
${\bf{R}}^3$ which leads to
spectra whch in general contain
unbounded real intervals.

\medskip

\noindent






\newpage

\centerline{\bf{Solutions for regukar boundary curves.}}



\bigskip

\noindent
Let $\mathcal C$ be a closed Jordan curve of class $C^2$ whose 
arc-lengt measure is denoted by $\sigma$.
If  $g$ is a continuous function on
$\mathcal C$ the logarithmic  potential
\[
L_g(z)=
\frac{1}{\pi}\int_{\mathcal C}\, \log\,\frac{1}{|z-q|} \cdot g(q)\, d\sigma(q)
\]
yields a harmonic function
in open the complement of  $\mathcal C$. Notice that we one gets  a pair of
harmonic functions, defined  in the bounded Jordan domain and the
the exterior domain respectively.
Since $\log\,|z|$ is locally integrable in
${\bf{C}}$ and  $L_g(z)$  the convolution of
this log-function and the compactly supported Riesz measure
$g\cdot \sigma$, it follows that $L_g$ extends to a continuous function.
In particular  the pair of harmonic functions in the 
inner respectively outer component  are equal
on $\mathcal C$.
Moreover,  by the result in  � xx
the  Laplacian of $L_g$ taken in the distribution sense is equal to
the measure $g\cdot\sigma$. Of course, this was known to 
Poincar� even if one did not speak about
distriobutins in those dys, but rather
"Grundl�sungen" and  the fact that the $log$-function
is a findfsem,tnal solution to the Lpace operator in
${\bf{R}}^2$ is classic and already used by Isaac Newton,

\medskip

\noindent
Let us now consider 
the inner normal derivative as $z$ approaches points
$p\in\mathcal C$ from the inside to be denoted by
${\bf{n}}_*$.
We get 
the function  on $\mathcal C$ defined by:
\[
p\mapsto \frac{\partial L_g}{\partial {\bf{n}}_*}(p)\tag{1.1}
\]
It turns out that (1.1)
is recaptured by an integral kernel
function $K(p,q)$ defined on
the product $\mathcal C\times\mathcal C$.
With  $p\neq q$
we consider the vector $p-q$ and  the unit vector
${\bf{n}}_*(q)$ and put
\[ 
K(p,q)=
\frac{\langle p-q,{\bf{n}}_*(q)\rangle}{|p-q|^2}\tag{*}
\]
\medskip


\noindent
Let  analyze the behaviour of $K$ close to a point on
the diagonal.
Working in local coordinates we can take $p=q=(0,0)$ and close 
the this boundary point 
the $C^2$-curve $\mathcal C$ is locally defined by a function
\[ 
y=f(x)
\] 
where a point 
$(x,y)$ belongs to the bounded Jordan domain when
$y>f(x)$.
By drawing a figure the reader can verify that
\[
{\bf{n}}_*(x,f(x))\cdot d\sigma=(-f'(x), 1)\dot dx
\]
So with $p=(t,f(t))$ and $q=(x,f(x))$ we have

\[
K(p,q)\cdot d\sigma(q)=
\frac{f(t)-f(x)- f'(x)(t-x)}{(t-x)^2+(f(t)-f(x))^2}\cdot dx
\]
By hypothesis $f$ is of class $C^2$ which implies that
the right hand side stays bounded as $t$ and $x$ independently
of each other approach zero.
Now one has:

\medskip

\noindent
{\bf{Theorem.}}
\emph{For each $p\in\mathcal C$ one has the equality}

\[
\frac{\partial U_g}{\partial {\bf{n}}_*}(p)=
 g(p)+\int_{\mathcal C}\, K(p,q)\cdot g(q)\,d\sigma(q)\tag{0.2}
 \]
 \medskip
 
 \noindent
 {\bf{Exercise.}} Prove (0.2).
A hint is that by additivity it suffices to take $g$-functions with 
supports confined to small sub-intervals of $\mathcal C$ and profit upon
local coordinates and  parametrizations as above when we study 
$\mathcal C$ close to the support of $g$.


\newpage


\centerline {\bf{1. Neumann's boundary value problem.}}
\bigskip

\noindent
Let $\Omega$ be a bounded domain
where $\partial\Omega$ consists of a finite set of closed Jordan
curves of class $C^2$. 
Let $h$ and $f$
be a pair of real-valued continuous functions on
$\partial\Omega$ where  $h$ is positive.
We seek a function $U$ which is harmonic in $\Omega$
and on the boundary satisfies
\[
\frac{\partial U}{\partial {\bf{n}}_*}(p)=
 h(p)\cdot U(p)+ f(p)
\]
\medskip

\noindent
Following Poincar� we  announce and prove the following result.

\medskip

\noindent
{\bf{1.1 Theorem.}}
\emph{The boundary value problem
above has a unique solution $U$.}

\medskip

\noindent
The uniqueness amounts to show that if $V$ is harmonic in $\Omega$ and
\[
\frac{\partial V}{\partial {\bf{n}}_*}(p)=
 h(p)V(p)
 \] 
 on $\partial\Omega$, then $V=0$.
Since 
$h$ is  positive 
this follows from  the maximum porinciple
for harmonic  functions. 
\medskip

\noindent 
\emph{Proof of existence.}
For each $g\in C^0(\partial \Omega)$ we construct 
$L_g$
which by Theorem 0.1 solves the Neumann problem if
$g$  satisfies the integral equation
\[
g(p)+\int_{\mathcal C}\, K(p,q)\cdot g(q)\,d\sigma(q)=
h(p)\cdot \frac{1}{\pi}\cdot\int_{\partial\Omega}\, \log\,\frac{1}{|p-q|}\cdot g(q)\,d\sigma(q)
+f(p)\tag{1}
\]
With $h$ kept fixed we introduce the kernel
\[ 
K_h(p,q)=h(p)\cdot \frac{1}{\pi}\cdot\log\, \frac{1}{|p-q|}- K(p.q)
\]
and  (1) reduces to the equation
\[
g(p)-\int_{\partial\Omega}\,K_h(p,q) g(q)\,d\sigma(q)= f(p)\tag{2}
\]
\medskip
To solve (2) we regard
 the linear operator on
the Banach space $C^0(\partial\Omega)$
defined by
\[ 
\mathcal K_h(f)=\int_{\partial\Omega}\,K_h(p,q) f(q)\,d\sigma(q)
\quad\colon\quad
f\in C^0(\partial\Omega)\tag{3}
\]
With this notation a $g$-function satisfies (2) if
\[
(E-\mathcal K_h)(g)=f\tag{4}
\] 
where $E$ is the identity operator on
$C^0(\partial\Omega)$.
Next, from the general result in �� 
$\mathcal K_h$ is a compact linear operator and  by
another general result from � xx  each
$f\in C^0(\partial\Omega)$
yields a meromorphic function of the complex parameter
$\lambda$ given  by
\[ 
N_f(\lambda)=
f+\sum_{n=1}^\infty\, \lambda^n\cdot \mathcal K_h^n(f)
\]
If $\delta>0$ is so small that
$||\mathcal K_h||<\delta^{-1}$ it is clear that
\[
(E-\lambda \mathcal K_h)(N_f(\lambda)=f
\]
If $N_f(\lambda)$ has no pole at $\lambda=1$ it follows by analyticity that
\[
(E-\mathcal K_h)(N_f(1)=f
\] 
which means that $g=N_f(1)$
solves (4) and the existence part  follows.
So there remains only to show:
\medskip

\noindent
\emph {The absence of a pole at $\lambda=1$.}
Suppose that $N_f(\lambda)$ has a pole at $\lambda=1$
which entails that there is a positive integer $m$ such that 
\[ 
N_f(\lambda)=
\sum_{k=1} ^{k=m}\, \frac{a_k}{(1-\lambda)^k}+ b(\lambda)
\] 
hold when $|\lambda-1|$ is small where $a_m\neq 0$ in 
$C^0(\partial\Omega)$
and $b(\lambda)$ is analytic in some disc centered at $\lambda=1$.
It follows that
\[
(1-\lambda)^m N_f(\lambda)=a_m+(1-\lambda)\beta(\lambda)
\] 
where 
$\beta(\lambda)$ again is an analytic $C^0(\partial\Omega)$-valued function
close to 1.
Apply $E-\mathcal K_h$ on both sides which gives
\[
(1-\lambda)^m (E- \mathcal K_h)(N_f(\lambda))
=(E-\mathcal K_h)(a_m)+(1-\lambda)
+(E-\mathcal K_h)(\beta(\lambda))
\]
Now  $\lambda=1$  gives
\[
(E-\mathcal K_h)(a_m)=0\implies
a_m=\mathcal K_h(a_m)
\]

\medskip

\noindent
This  contradicts  the uniqueness part which already has been proved
and hence the proof of Theiorem 1 is finished.



\bigskip

\centerline{\bf{2. The case when
$\mathcal C$ has corner points.}}

\bigskip


\noindent
In the previous  section we  found a unique solution to 
Neumann's boundary problem where
the inner normal derivative of $U$
along $\partial\Omega$ is a continuous function.
If corner points appear 
this will no longer be true. But 
stated in an appropriate way one can  extend
Theorem 1.1.
Let us analyze the specific case when
the boundary curves are piecewise linear, i.e. each
closed Jordan curve in
$\partial\Omega$
is a simple polygon with a finite number of corner points.
Given one of these curves $\mathcal C$
we shall study the
$K$-function locally.
Let $\xi_1,\ldots,\xi_N$ be the corner points on
$\mathcal C$.
On the linear interval $\ell_i$ which joints two succescive
corner
points $\xi_i$ and $\xi_{i+1}$ we notice that
${\bf{n}}_*$ is constant and 
\[ 
K(p,q)=0\quad\colon\quad p,q\in \ell_i
\]
Indeed, this is obvious for if $p$ and $q$ both belong to $\ell_i$ then
the vector $p-q$ is parallell to $\ell_i$ and hence $\perp$ to
the normal of this line.
Next, keeping $q$ fixed on the open interval $\ell_i$
while $p$ varies on $\mathcal C\setminus \ell_i$
the behaviour of the function
\[ 
p\mapsto \langle p-q,{\bf{n}}_*(q)\rangle\tag{*}
\] 
is can be understood via a picture and it is
clear that (i) is a continuous function.
By a picture the reader should discover the different behaviour in the case
when $\mathcal C$ is convex or not.
For example, in the non-convex case it is in general not true 
that $\mathcal C\setminus \ell_i$
stays in the half-space bordered by the line passing $\ell_i$ and then
(*) can change sign, i.e. take both positive and negative values.
In the special case when
$\mathcal C$ is a convex polygon the reader should confirm that
(*) is a positive function of $p$ because we have taken the
\emph{inner} normal 
${\bf{n}}_*(q)$.
\medskip


\noindent
{\bf{2.1 Local behaviour at a corner point.}}
After a linear change of coordinates
we take a corner point $\xi_*$ placed at the origin
and one of the $\ell$-lines with end-point at the origin
is defined by
the equation $\{y=0\}$ to the left of $\xi_*$ where
$x<0$ while $y=Ax$ hold to the right for some $A\neq 0$.
If $A>0$ it means that
the angle $\alpha$ at the corner point is detemined by
\[ 
\alpha=\pi-\text{arctg}(A)
\]
If $A<0$ the inner angle is between 0 and $\pi/2$
which the reader should illustrate by a picture.
Next, consider
a pair of points $p=(-x,0)$ and $q=(t,At)$
where $x,t>0$. So $p$ and $q$ belong to opposite sides of the
corner point.
To be specific, suppose that $A>0$ which entails that
\[ 
{\bf{n}}_*(q)=\frac{(-A,1)}{\sqrt{1+A^2}}\implies
K(p,q)\cdot d\sigma(q)=  \frac{Ax+t}{(x+t)^2+A^2t^2}
\]
When $x$ and $t$ decrease to the origin the order of magnitude 
is
$\frac{1}{x+t}$ so the kernel function is unbounded
and the order of magnitude is $\frac{1}{x+t}$.
If $\ell_+$ denotes the boundary interval to the right of the origin
where $q$ are placed we conclude that
\[
\int_{\ell_+}\, K(p,q)\cdot d\sigma(p)\simeq
\int_0^1\, \frac{dt}{x+t}\simeq \log\,\frac{1}{x}
\]
The last function is integrable with respect to $x$. This local computation shows
that the kernel function $K$ is not too large in 
the average. In particular
\[
\iint_{\mathcal C\times\mathcal C}\,
|K(p,q)|\cdot d\sigma(p)d\sigma(q)<\infty
\]
But
the
growth of $K$ near corner points prevail 
a finite $L^2$-integral, i.e. the reader may verify that
\[
\iint_{\mathcal C\times\mathcal C}\,
|K(p,q)|^2\cdot d\sigma(p)d\sigma(q)=+\infty
\]
\medskip


\noindent
{\bf{2.2 The integral operator $\mathcal K_h$}}.
Let $h$ be a positive continuous function on
$\partial\Omega$.
Now we define the kernel function $K_h(p,q)$ exactly as in
� 1 and obtain the 
linear operator
\[ 
g\mapsto\int_{\partial\Omega}\, K_h(p,q)g(q)d\sigma(q)
\]
It has a natural domain of definition.
Namely,  introduce the space
$L^1_*$ which consists of functions on $g$
on $\partial\Omega$ for which
\[
\iint\, \log\frac {R}{|p-q|}\cdot |g(p)|\cdot d\sigma(q)d\sigma(p)<\infty\tag{*}
\] 
where  $R>0$ is so large that
$\frac{R}{|p-q|}>1$  hold for  pairs $p,q$ on $\partial\Omega$.
Return to the local situation in
(2.1) and consider a $g$-function in  $L^1_*$. Locally 
we encounter an integral of the form
\[
\iint_{\square_+}\, \frac{1}{x+t}\cdot |g(t,At))|\, dt
\]
where $0\leq x,t\leq 1$ hold in
$\square_+$.
In this double integral  integration with respect to $x$
is finite since the inclusion $g\in L^1_*$ entails that
\[
\int_0^1\,\log\,\frac{1}{t}\cdot |g(t,At)|\, dt<\infty
\]


\noindent
From the above we obtain the following:

\medskip

\noindent
{\bf{2.3 Theorem.}}
\emph{The kernel function $K_h$ yields a continuous linear operator from
$L^1_*$ into $L^1(\partial\Omega)$, i.e. there exists a constant
$C$ such that}
\[
\int_{\partial \Omega} |\mathcal K_h(g)|\cdot d\sigma
\leq C\cdot 
\iint
_{\partial \Omega\times \partial \Omega} 
\, \log\frac {R}{|p-q|}\cdot |g(p)|\cdot d\sigma(q)d\sigma(p)
\]
\bigskip

\noindent
Armed with Theorem 2.3 
one  can solve Neumann's boundary value problem for domains 
whose boundary curves are polygons.

\medskip
\noindent
{\bf{2.4 Theorem.}}
\emph{For each $f\in L^1(\partial\Omega$
there exists
a unique harmonic function $U$ in $\Omega$ such that}
\[
\frac{\partial U}{\partial {\bf{n}}_*}(p)=
 h(p)U(p)+ f(p)
\]
\emph{holds on $\partial \Omega$. Moreover, $U=L_g$ where
$g\in L^1_*(\partial\Omega)$ solves the integral equation}
\[
g-\mathcal K_h(g)=f
\]

\newpage


\noindent
\emph{Proof of  uniqueness.}
At corner points the inner normal of $U$ has no limit and
to establish the uniqueness we use  instead
an integral formula:


\medskip

\noindent{\bf{2.5 Proposition.}}
\emph{For each $g\in L^1_*$ the potential function $U=L_g$ satisfies}

\[
\iint_\Omega\ [
(\frac{\partial U}{\partial x})^2
+(\frac{\partial U}{\partial y})^2]\, dxdy+
\int_{\partial\Omega}\, U\cdot 
\frac{\partial U}{\partial {\bf{n}}_*}\,
d\sigma=0\tag{2.5.1}
\]
\medskip


\noindent
{\bf{Exercise.}} Prove this or consult Carleman's Phd-thesis for details.

\bigskip 

\noindent
The requested uniqueness in Theorem 2..4 follows from (2.5.1).  For if
$\frac{\partial U}{\partial {\bf{n}}_*}= h\cdot U$
holds on the boundary we get 

\[
0=\iint_\Omega\ [
(\frac{\partial U}{\partial x})^2
+(\frac{\partial U}{\partial y})^2]\, dxdy=\int_{\partial\Omega}\, 
h\cdot U^2\,
d\sigma\implies g=0
\] 
\medskip

\centerline
{\emph{2.6 Proof of existence.}}
\medskip

\noindent
It  is carried out by the same mehod as in
� X. The crucial  point is  that
the kernel function $K_h$ is sufficiently well-behaved in order that
every $f\in L^1(\Omega)$
yields a meromorphic function $N_f(\lambda)$
where $\mathcal K_h$-powers are applied to $f$ exactly as in XX.
\medskip


\noindent
{\bf{Exercise.}} Supply details which prove that
$N_f(\lambda)$ is meromorphic.
\medskip


\noindent
{\bf{Remark.}}
In [Carleman: Part 3] it is proved that
the unique solution $g$ to the integral equation
is represented in a canonical fashion using a certain orthonormal family of functions with 
respect to the $L^2$-function $\log\,\frac{1}{|p-q|}$ 
wtih respect to the arc-length measure $\sigma\times\sigma$. Moreover, there exists
a representation formula expressed by convergent series for the 
equation
\[ 
g+\lambda\cdot \mathcal K_h(g)=f
\] 
where poles of $N_f(\lambda)$ are taken into the account.






















































\end{document}