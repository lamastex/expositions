


\documentclass{amsart}
\usepackage[applemac]{inputenc}


\addtolength{\hoffset}{-12mm}
\addtolength{\textwidth}{22mm}
\addtolength{\voffset}{-10mm}
\addtolength{\textheight}{20mm}

\def\uuu{_}


\def\vvv{-}

\begin{document}


\newpage



\noindent
{\bf{Example.}}
Let $n=3$ and consider a PDE-operator
\[
L=
\sum_{p=1}^{p=3}\sum_{q=1}^{q=3}\, a_{pq}(x)\cdot \frac{\partial ^2}{\partial x_p\partial x_q}+
\sum_{p=1}^{p=3}\, a_p(x)
\frac{\partial }{\partial x_p}+a_0(x)
\]
The $a$-functions are real-valued and
defined in a neighborhood of the closure of a bounded
domain
$\Omega$ with a $C^1$-boundary.
Here one has the symmetry $a_{pq}=a_{qp}$, and 
$\{a_{pq}\}$ are
of class $C^2$, $\{a_p\}$  of class $C^1$ and $a_0$ is continuous.
The elliptic property of
 $L$ means that
for
every $x\in\Omega$ the eigenvalues of the symmetric
matrix
$A(x)$ with elements $\{a_{pq}(x)\}$
are positive.
Under these conditions, a  result which goes back to work by 
Neumann and Poincar�,
gives
a positive constant
$\kappa_0$ such that
if $\kappa\geq \kappa_0$ then
the inhomogeneous equation
\[
L(u)-\kappa^2\cdot u=f\quad\colon f\in L^2(\Omega)
\]
has a unique solution $u$ which is a $C^2$.-function 
in
$\Omega$ and  extends to the closure where it is zero on
$\partial\Omega$.
Moreover, there exists some $\kappa_0$
and for each $\kappa\geq \kappa_0$ a
Green's function
$G(x,y;\kappa)$ such that
\[
(L-\kappa^2)(\frac{1}{4\pi}\cdot \int_\Omega\, 
G(x,y;\kappa)\, f(y)\, dy )= -f(x)\quad\colon f\in L^2(\Omega)\tag{i}
\]
This  means  that the bounded linear operator on
$L^2(\Omega$ defined by
\[
f\mapsto 
-\frac{1}{4\pi}\cdot \int_\Omega\, 
G(x,y;\kappa)\, f(y)\, dy\tag{ii}
\]
is Neumann's resolvent
to the densely defined operator
$L-\kappa^2$ on the 
Hilbert space $L^2(\Omega)$.
After a detailed study of these $G$-functions, Carleman established
an asymptotic formula for
the discrete sequence of eigenvalues $\{\lambda_n\}$. In general they
are complex but arranged so that
the absolute values increase. To begin with one proves rather easily that they
are "almost real" in the sense that there exist positive constants
$C$ and $c$ such that
\[
|\mathfrak{Im}(\lambda_n)|\leq
C\cdot(\mathfrak{Re}(\lambda_n)+c)
\] 
hold for every $n$.
Next, the elliptic
hypothesis means that
the determinant function
\[
D(x)=\det(a_{p,q}(x))
\]
is positive in $\Omega$. With these notations one has

\medskip



\noindent
{\bf{ Theorem.}}
\emph{The following limit formula holds:}
\[
\lim_{n\to\infty}\, \frac{\mathfrak{Re}(\lambda_n)}{n^{\frac{2}{3}}}
=\frac{1}{6\pi^2}\cdot \int_\Omega\, 
\frac{1}{\sqrt{D(x)}}\, dx\tag{*}
\]
\medskip

\noindent
{\bf{Remark.}}
The formula above is due to Courant and Weyl  when
$P$ is symmetric and was extended to
non-symmetric operators during 
Carleman 's lectures at Institute Mittag-Leffler in  1935.
Weyl and Courant used calculus of variation
in the symmetric case 
while Carleman employed  different methods which
have the merit that the passage to the non-symmetric case
does not cause any  trouble. 
As pointed out by
Carleman the methods in the  proof  
give similar asymptotic formulas 
in   other boundary value problems such as
those considered by Neumann where one imposes boundary value conditions on
outer normals, and so on.
A crucial step during the  proof of the theorem above
is to
construct a fundamental solution $\Phi(x,\xi;\kappa)$
to the PDE-operators
$L-\kappa^2$ which is exposed in � E.





\newpage

\centerline {\bf{2. Green's functions.}}
\bigskip


\noindent
Let $\Omega$ be a bounded domain in
${\bf{R}}^3$.
A Green's function $G(x,y;\kappa)$ attached to this domain
and the PDE-operator $P(x,\partial_x;\kappa)$
is a function which for fixed $\kappa$ is a function in 
$\Omega\times\Omega$
with the following properties:
\[ 
G(x,y;\kappa)=0\quad\text{when}\quad x\in\partial \Omega\quad \text{and}\quad
y\in\Omega\tag{*}
\]
\[
\psi(y)=\int_\Omega\, (P^*(x,\partial_x)-\kappa^2)
(\psi(x))\cdot G(x,y;\kappa)\, dx\quad\colon\quad y\in\Omega\tag{**}
\]


\noindent
hold for all $C^2$-functions $\psi$ with compact support in
$\Omega$.
To find $G$ one 
solves a   Dirchlet problem.
With
$\xi\in\Omega$ kept fixed one has the continuous function on
$\partial\Omega$:
\[ 
x\mapsto \Phi^*(x,\xi;\kappa)
\]



\noindent
Solving Dirchlet's problem gives  a unique $C^2$-function
$w(x)$ which satisfies:
\[ 
P(x,\partial_x)(w)+\kappa^2\cdot w=0 \quad\text{holds  in}\quad \Omega
\quad\text{and}\quad w(x)= \Phi(x,\xi;\kappa)\quad\colon x\in\partial\Omega=0
\]

\medskip

\noindent
From the above it is clear that
this gives the requested $G$-function, i.e. 

\medskip

\noindent
{\bf{2.1 Proposition.}} \emph{The 
the function }
\[ 
G(x,\xi;\kappa)=
\Phi(x,\xi;\kappa)-w(x)
\quad \text{satisfies}\quad (*-**)
\]


 
\noindent
Using the estimates for the $\phi$-function from � 1
we shall establish 
estimates for the $G$-function above where
we start with  a sufficiently large
$\kappa_0$ so  that 
$\Phi^*(x,\xi;\kappa_0)$ is a positive function of
$(x,\xi)$. 
\medskip


\noindent
{\bf{2.2 Theorem.}}
\emph{One has}
\[ 
G(x,\xi;\kappa_0)=
\frac{1}{\sqrt{\Delta(x)}\cdot\sqrt{\Phi(x,\xi;\kappa_0)}}
+R(x,\xi)
\]

\noindent
\emph{where the remainder function satisfies the following for all pairs
$(x,\xi)$ in $\Omega$:}
\[ 
|R(x,\xi)|\leq C\cdot |x-\xi|^{-\frac{1}{4}}
\]


\noindent
\emph{with a    constant $C$ which   depends on the  domain 
$\Omega$ and
the PDE-operator $P$.}
\bigskip

\noindent
{\bf{2.3 Remark.}}
Above the negative power  of 
$|x-\xi|$ is  a fourth-root which means that  the remainder term $R$ 
is more regular  compared
to the first term which behaves like $|x-\xi|^{-1}$ on the diagonal $x=\xi$.
A detailed proof of Theroem 2.2 is given in
[Carleman: page 125-127]. 

\bigskip

\centerline {\bf{2.4 The integral operator
$\mathcal J$}}. 
\medskip


\noindent
From now on we admit Theorem 2.2 and
consider the integral operator
which sends a function
$u$ in $\Omega$ to 
\[
\mathcal J_u(x)=\int_\Omega\, G(x,\xi;\kappa_0)\cdot u(\xi)\, d\xi
\]


\noindent
The construction of the Green's function gives:
\[
(P-\kappa_0^2)(\mathcal J_u)(x)=u(x)\quad\colon\quad x\in\Omega\tag{2.4.1}
\] 
This , if $E$ denotes the identity we have
the operator equality

\[
P(x,\partial_x)\circ \mathcal J_u=\kappa_0^2\cdot  \mathcal J+E\tag{2.4.2}
\]



\noindent
Consider a pair $(u$ and a real number $\gamma)$ such that 
\[
u(x)+ \gamma\cdot \mathcal J_u(x)=0\quad\colon\quad x\in\Omega\tag{2.4.3}
\] 
The vanishing from (*) for the $G$-function 
implies that $J_u(x)=0$ on $\partial\Omega$. Hence  every
$u$-function which satisfies
in (2.4.3) for some  constant $\gamma$
vanishes   on $\partial\Omega$.
Next, when $P$ is applied to
(2.4.3) 
the operator equation  (2.4.2) gives

\[
0=P(u)+\gamma \kappa_0^2\cdot \mathcal J_u+\gamma\cdot u\implies
P(u)+(\gamma-\kappa_0^2)u=0
\]
\medskip

\noindent
{\bf{2.4.4 Conclusion.}}
The boundary value problem (*) from 0.B
is equivalent to find eigenfunctions of 
$\mathcal J$ via (2.4.3) above.

\medskip

\noindent
\centerline {\bf{3. Almost reality of eigenvalues.}}
\medskip

\noindent
Consider the set of eigenvalues $\lambda$ to (*) in (0.B).
Then we have:

\medskip

\noindent
{\bf{3.1 Proposition.}}
\emph{There exist positive constants $C_*$ and $c_*$ such that
every eigenvalue  $\lambda$ to  (*) in (0.B) satisfies}

\[
|\mathfrak{Im}\,\lambda|^2\leq C_*(\mathfrak{Re}\,\lambda)+c_*)
\]
\medskip

\noindent
\emph{Proof.}
Let $u$ be an eigenfunction where
$P(u)+\lambda\cdot u=0$.
Stokes theorem and the vanishing of $u|\partial\Omega$
give:
\[
0=\int_\Omega\, \bar u\cdot (P+\lambda)(u)\,dx
=-\int_\Omega \, \sum_{p,q}\, a_{pq}(x)\cdot \frac{\partial u}{\partial x_p}
\frac {\partial \bar u}{\partial x_q}\, dx+
\int_\Omega\, \bar u\cdot( \sum \, a_p(x)
\frac{\partial u}{\partial x_p}\,)\, dx+
\] 
\[
\int_\Omega\, |u(x)|^2\cdot b(x)\, dx+
\lambda\cdot \int\, |u(x)|^2\, dx
\]
\medskip


\noindent
Write $\lambda=\xi+i\eta$.
Separating real and imaginary parts we find the two equations:
\[
\xi\int\, |u|^2\, dx=
\int\, \sum_{p,q} a_{p,q}(x)\,\frac{\partial u}{\partial x_p}\cdot 
\frac{\partial \bar u}{\partial x_q}\, dx+
\int\, \bigl(\frac{1}{2}\cdot \sum\, \frac{\partial a_p}{\partial x_p}- b\,\bigr )
\cdot |u|^2\, dx\tag{i}
\]
\[
\eta\int\, |u|^2\, dx=\frac{1}{2i}\int \sum\, a_p\bigl(
u\frac{\partial \bar u}{\partial x_p}-
\bar u \frac{\partial u}{\partial x_p}\,\bigl )\, dx\tag{ii}
\]


\noindent
Set
\[ A= \int\, |u|^2\,dx\quad\colon\quad
B= \int\, |\nabla(u)|^2\,dx
\]












Since $P$ is elliptic there exists a positive constant $k$ such that
\[
\sum_{p,q} a_{p,q}(x)\,\frac{\partial u}{\partial x_p}>
k\cdot |\nabla(u)|^2
\]
From this we see that (i-ii) gives positive constants $c_1,c_2,c_3$ such that
\[
A\xi>c_1B-c_2B\quad\colon\quad A|\eta|<c_3\cdot \sqrt{AB}\tag{iii}
\]
\medskip

\noindent
Here (iii) implies that $\xi>-c_2$ and the reader can also confirm that
\[ 
B<\frac{A}{c-1}(\xi+c-2)\quad\colon\quad
A|\eta|< A\cdot c_2\cdot \sqrt{\frac{\xi+c_2}{c_1}}\quad\colon\quad
|\eta|< c_3\cdot \sqrt{\frac{\xi+c_2}{c_1}}\tag{iv}
\]


\noindent Finally it is obvious that (iv) above gives the requested
inequality in Proposition 3.1.

\bigskip

\centerline{\bf{4. The asymptotic  formula.}}
\bigskip


\noindent
Using the results above
where we have found  a good control of the integral operator
$\mathcal J$ and the identification of eigenvalues to $\mathcal j$ and those from (*) in (0.B), one can proceed and apply Tauberian theorems to derive
the asymptotic formula  in Theorem 1 using 
similar methods as described in � XX where we treated
the Laplace operator. 
For details the reader may consult [Carleman:p age xx-xx].


\end{document}












