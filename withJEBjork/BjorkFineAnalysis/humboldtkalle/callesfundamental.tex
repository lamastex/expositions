



\documentclass{amsart}
\usepackage[applemac]{inputenc}


\addtolength{\hoffset}{-12mm}
\addtolength{\textwidth}{22mm}
\addtolength{\voffset}{-10mm}
\addtolength{\textheight}{20mm}

\def\uuu{_}


\def\vvv{-}

\begin{document}





\centerline {\bf{Fundamental solutions to second order
Elliptic operators.}}
\bigskip


\noindent
{\bf{Introduction.}}
Elliptic partial differential operators appear frequently and 
one often employs  fundamental  solutions.
The subsequent material is devoted to
contruct fundamental solutions with
best possible regularity conditions
for second order elliptic operators with variable coefficients in
${\bf{R}}^3$.
For PDE-operators with constant coefficients one 
employs Fouriers' inversion fourmula
and we  refer to Chapter X in vol.2 in H�rmander's text-book series on linear 
partial differential operators for a detailed account
about constructions of fundamental  solutions with optimal regularity
In the case of second  order elliptic operstors with
constant coefficients the best funfsrmtnal solutions were alreafy
found by Newton in pioneering eork
from hids famous text-books published in 1666.
Passing to the case of variable coefficients one profits Newton's constructions
and
following Carleman we will show that
one can obtain fundamental  solutions to second order elliptic operators 
with variable coefficientd in
a canonical fashion.
In contrast to the case  of constant coefficients  the subsequent constructions do not 
use the Fourier transform. Insetad one finds fundamental 
solutions by solving integral equations of the Neumann-Fredholm type.
Let us  remark that one  does not need any concepts from distribution
theory since we will find
fundamental solutions are locally integrable and 
in such situations
the
notion of fundamental solutions were well understood at an early stage after pioneering work by
Weyl and Zeilon prior to 1925. In his article Carleman refers to
\emph{Grundl�sungen} 
and their requested properties are derived via Green's formula.

\medskip


\noindent
{\bf{Remark.}}
For students interested in PDE-theory the material below offers 
an instructive lesson and  suggests further investigations.
We restrict the study to ${\bf{R}}^3$ and  remark only that
similar constructions can be performed when $n\geq 4$ starting from
Newton's potential 
$|x-\xi|^{-n+2}$. Here it would be  interesting to clarify the
precise estimates  when $n\geq 4$ and
establish similar inequalities as in the Main Theorem.
It is also tempting to  try to extend the whole constructions to elliptic operators of
order $\geq 3$. Recall that for elliptic operators 
with constant coefficiebts having an even order
$2m$ with $m\geq 2$
there exist fuyndsrmtnal solutions constructed by Fritz John
which like Newton's are found in a can onical fashion
and therefore should be useful
to treat the case of vartiable coefficiebts.
This appears to be a "profitable research problem" for 
ph.d-students.
Let us also remark that
one does not assume that the elliptic operators are symmetric, i.e.
both the constructions as well as estimates for the fundamental solutions do
not rely upon symmetry conditions.
Recall finally  that fundamental solutions are used to construct various Greens' functions
and here a priori estiamates are valuable 
We illustrate  this in � xx where we expose some further results by Carleman concerned with
asymptotic distributions of eigenvalues to elliptic boundary value problems.

\medskip


\noindent
{\bf{An asymptotic formula for the spectrum.}}
Let $n=3$ and consider a second order PDE-operator
\[
L=
\sum_{p=1}^{p=3}\sum_{q=1}^{q=3}\, a_{pq}(x)\cdot \frac{\partial ^2}{\partial x_p\partial x_q}+
\sum_{p=1}^{p=3}\, a_p(x)
\frac{\partial }{\partial x_p}+a_0(x)
\]
The $a$-functions are real-valued and
defined in a neighborhood of the closure of a bounded
domain
$\Omega$ in ${\bf{R}}^3$
with a $C^1$-boundary.
Here one has the symmetry $a_{pq}=a_{qp}$, and 
$\{a_{pq}\}$ are
of class $C^2$, $\{a_p\}$  of class $C^1$ and $a_0$ is continuous.
The elliptic property of
 $L$ means that
for
every $x\in\Omega$ the eigenvalues of the symmetric
matrix
$A(x)$ with elements $\{a_{pq}(x)\}$
are positive.
Under these conditions, a  result which goes back to work by 
Neumann and Poincar�,
gives
a positive constant
$\kappa_0$ such that
if $\kappa\geq \kappa_0$ then
the inhomogeneous equation
\[
L(u)-\kappa^2\cdot u=f\quad\colon f\in L^2(\Omega)
\]
has a unique solution $u$ which is a $C^2$-function 
in
$\Omega$ and  extends to the closure where it is zero on
$\partial\Omega$.
Moreover, there exists some $\kappa_0$
and for each $\kappa\geq \kappa_0$ a
Green's function
$G(x,y;\kappa)$ such that
\[
(L-\kappa^2)(\frac{1}{4\pi}\cdot \int_\Omega\, 
G(x,y;\kappa)\, f(y)\, dy )= -f(x)\quad\colon f\in L^2(\Omega)\tag{i}
\]
This  means  that the bounded linear operator on
$L^2(\Omega$ defined by
\[
f\mapsto 
-\frac{1}{4\pi}\cdot \int_\Omega\, 
G(x,y;\kappa)\, f(y)\, dy\tag{ii}
\]
is Neumann's resolvent
to the densely defined operator
$L-\kappa^2$ on the 
Hilbert space $L^2(\Omega)$. 
Next, one seeks pairs $(u_n,\lambda_n)$
where $u_n$ are $L^2$-functions in $\Omega$ which 
extend to be zero on
$\partial\Omega$ and satisfy
\[ 
L(u_n)+\lambda_n\cdot u_n=0
\]
It turns out that the set of eigenvalues is discrete and
moreover their real parts tend to $+\infty$.
They are arranged with non-decreasing absolute values and in � xx we prove
that there exist positive constants
$C$ and $c$ such that
\[
|\mathfrak{Im}(\lambda_n)|^2\leq
C\cdot(\mathfrak{Re}(\lambda_n)+c)
\] 
hold for every $n$.
Next, the elliptic
hypothesis means that
the determinant function
\[
D(x)=\det(a_{p,q}(x))
\]
is positive in $\Omega$. With these notations one has


\medskip



\noindent
{\bf{ Theorem.}}
\emph{The following limit formula holds:}
\[
\lim_{n\to\infty}\, \frac{\mathfrak{Re}(\lambda_n)}{n^{\frac{2}{3}}}
=\frac{1}{6\pi^2}\cdot \int_\Omega\, 
\frac{1}{\sqrt{D(x)}}\, dx\tag{*}
\]
\medskip

\noindent
{\bf{Remark.}}
The formula above is due to Courant and Weyl  when
$P$ is symmetric and  was extended to
non-symmetric operators during 
Carleman 's lectures at Institute Mittag-Leffler in  1935.
Weyl and Courant used calculus of variation
in the symmetric case 
while Carleman employed  different methods which
have the merit that the passage to the non-symmetric case
does not cause any  trouble. 
A crucial step during the  proof of the theorem above
is to
construct a fundamental solution $\Phi(x,\xi;\kappa)$
to the PDE-operators
$L-\kappa^2$ which done in � 1 while � 2 treats the asymptotic formula above.
As pointed out by
Carleman the methods in the  proof  
give similar asymptotic formulas 
in   other boundary value problems such as
those considered by Neumann where one imposes boundary value conditions on
outer normals.
As an example we consider an elliptic
operator
of the form
\[
L=\Delta+
\sum_{p=1}^{p=3}\, a_p(x)
\frac{\partial }{\partial x_p}+a_0(x)
\]
where $\Delta$ is the Laplace operator.
Given a positive real-valued continuous function
$\rho(x)$ on $\partial\Omega$
we obtain the Neumann-Poincar� operator
$\mathcal{NP}$ which sends each $u\in C^0(\partial\Omega)$ to
\[
\mathcal{NP}(u)=
\frac{\partial u^*}{\partial{\bf{n}}_i}-\rho\cdot u
\]
Here $u^*$ is the Dirichlet extension of $u$ to $\Omega$ which
is equal to $u$ on $\partial\Omega$ and satisfies $L(u)=0$ in
$\Omega$, while
$\frac{\partial u^*}{\partial{\bf{n}}_i}$ is the inner normal along the boundary.
In the special case when
$L=\Delta$ this  boundary value problem has unique solutions, i.e.
for every
$f\in C^0(\partial\Omega)$ there exists a unique $u$ such that
\[
\mathcal{NP}(u)=f
\]
This was proved
by Poincar� in 1897 for
domains in ${\bf{R}}^3$ whose boundaries are of class
$C^2$ and the extension to domains with a $C^1$-boundary
is also classic.
Passing to general operators $L$ as above which are not neccesarily symmetric
one encounters spectral problems, i.e. above
$\mathcal{NP}$ regared as a linear operator on the
Banach space $C^0(\partial\Omega)$ is densely defined and
one seeks its spectrum, i.e,  complex numbers
$\lambda$ for which there exists a non-zero
$u$ such that
\[
\mathcal{NP}(u)+\lambda\cdot u=0
\]
I do not
know if there exists
an  analytic formula for these eigenvalues. Notice that a new feauture is that
the $\rho$-function affects the spectrum.










\newpage 



\centerline{\bf{Fundamental solutions.}}
\bigskip

\noindent
In
${\bf{R}}^3$ with coordinates $x=(x_1,x_2,x_3)$ we consider 
 a second order PDE-operator
\[
L=
\sum_{p=1}^{p=3}\sum_{q=1}^{q=3}\, a_{pq}(x)\cdot \frac{\partial ^2}{\partial x_p\partial x_q}+
\sum_{p=1}^{p=3}\, a_p(x)
\frac{\partial }{\partial x_p}+a_0(x)
\]
where 
$a$-functions are real-valued and
one has the symmetry $a_{pq}=a_{qp}$.
To ensure existence of a globally defined fundamental solutions we
suppose the the following limit formulas hold
as $(x,y,z)\to \infty$:
\[
\lim a_\nu(x,y,z)=0 \colon 0\leq p\leq 3\quad\colon\,
\lim a_{pq}(x,y,z)= \text{Kronecker's delta function}\tag{0.0}
\]
Thus, $L$ approaches the Laplace operator as $(x,y,z)$ tends to
infinity. Moreover
$L$ is elliptic which means that
the eigenvalues of the symmetric
matrix with elements $\{a_{pq}(x)\}$
are positive for every $x$.
Recall the  notion of  fundamental solutions.
Consider
the adjoint
operator:
\[ 
L^*(x,\partial _x)=P-2\cdot \bigl(
\sum_{p=1}^{p=3}\, \bigl(\sum_{q=1}^{q=3}\, 
\frac{\partial a_{pq}}{\partial x_q}\bigr)\cdot \frac{\partial}{\partial x\uuu p}
-\sum_{p=1}^{p=3}\, \frac{\partial a_p}{\partial x_p}
+2\cdot \sum\sum\, \frac{\partial^2 a_{pq}}{\partial x_p\partial x_q}\tag{0.1}
\]
Partial integration gives  the equation  below for every pair of
$C^2$-functions $\phi,\psi $ in ${\bf{R}}^3$ with compact support:
\[
\int\, L(\phi)\cdot \psi\, dx=
\int\, \phi\cdot L^*(\psi)\, dx\tag{0.2}
\] 
where
the volume integrals are taken over
${\bf{R}}^3$.
A locally integrable function $\Phi(x)$ in ${\bf{R}}^3$
is  a fundamental solution to $L(x,\partial _x)$
if
\[
\psi(0)=\int\, \Phi\cdot L^*(\psi)\, dx\tag{0.3}
\] 
hold for every $C^2$-function  $\psi$ with compact support.
Next, to each 
positive number
$\kappa$ we  get  the PDE-operator $L-\kappa^2$ and a
function $x\mapsto \Phi(x;\kappa)$ is a fundamental solution to $L-\kappa^2$
if
\[
\psi(0)=\int\, \Phi(x:\kappa)\cdot (L^*-\kappa^2)(\psi(x))\, dx\tag{0.4}
\] 
hold for compactly supported $C^2$-functions $\psi$.
Next, the  origin can  replaced by a variable point $\xi$ in
${\bf{R}}^3$ and then one seeks
a function
$\Phi^*(x,\xi;\kappa)$ with the property that
\[
\psi(\xi)=\int\, \Phi(x,\xi;\kappa)\cdot (L^*(x,\partial_x)-\kappa^2)(\psi(x))\, dx\tag{*}
\] 
hold for all  $\xi\in{\bf{R}}^3$
and every $C^2$-function $\psi$ with compact support.
Keeping $\kappa$ fixed this means that  
$\Phi(x,\xi;\kappa)$ is a function of six variables  defined in 
${\bf{R}}^3\times {\bf{R}}^3$.
With these notations we announce the main result:


\bigskip

\noindent
{\bf{Main Theorem.}} \emph{There exists a  constant
$\kappa_*$ such that for every $\kappa\geq\kappa_*$ one can find a fundaemtal
solution
$\Phi(x,\xi;\kappa)$ which is locally integrable
in the 6-dimensional $(x,\xi)$-space.
Moreover,
there exist positive constants $C$ and $k$ and
for each $0<\gamma\leq 2$ a constant $C_\gamma$ such that}
\[
|\Phi(x,\xi;\kappa)|\leq
C\cdot \frac{e^{-k\kappa|x-\xi|}}{|x-\xi|}+
 \frac{C_\gamma}{(\kappa|x-\xi|)^{\gamma}}
\]
\emph{hold for all pairs $(x,\xi)$ in ${\bf{R}}^3$ and every
where the constants $k$ and $C$ do not depend upon $\kappa$.}

\bigskip



\centerline {\bf{1. The construction of $\Phi(x,\xi;\kappa)$.}}
\medskip



\noindent
The subsequent constructions 
are 
based upon a
classic formula due to Newton together with  solutions to
integral equations
found by a convergent Neumann series.
When $L$ has
constant coefficients 
the construction of   fundamental solutions 
given by
Newton
goes as follows:
Consider a 
positive  and symmetric $3\times 3$-matrix
$A= \{a_{pq}\}$. Let
$\{b_{pq}\}$ be the elements of the inverse matrix which gives 
the quadratic form

\[ 
 B(x)= \sum_{p,q}\, b_{pq} a_px_q
\]
Put
\[
\alpha=\sqrt{\kappa^2+\frac{1}{2}\, \sum_{p,q}\, b_{pq} a_pa_q
-a_0}
\]
where $\kappa$ is chosen so large that
the term under the square-root is $>0$. Finally, put
\[
\Delta=\det(A)
\]
With these notations we get a function:
\[
H(x;\kappa)= \frac{1}{4\pi\cdot \sqrt{\Delta\cdot B(x)}}
\cdot e^{-\alpha \sqrt{B(x)}-
\frac{1}{2}\sum_{p,q}b_{pq} a_p\cdot x_q}\tag{1.1} 
\]


\medskip




\noindent
{\bf{Exercise.}} Verify by Stokes  formula
that $H(x;\kappa)$  yields  a fundamental solution
to the PDE-operator $L(\partial_x)-\kappa^2$.



\bigskip



\noindent
{\bf{1.2 The case with variable coefficients.}}
Now $L$ has variable coefficients.
For each $\xi\in{\bf{R}}^3$ the elements of the inverse matrix
to $\{a_{pq}(\xi)$
are denoted by $\{b_{pq}(\xi)\}$.
Choose $\kappa_0>0$ such that
\[
\kappa_0^2+\frac{1}{2}\, \sum_{p,q}\, b_{pq}(\xi) a_p(\xi)a_q(\xi)
-b(\xi)>0\quad\text{hold for all}\quad  \xi\in{\bf{R}}^3
\] 
and for every $\kappa\geq \kappa_0$ we set
\[
\alpha_\kappa(\xi)=
\sqrt{\kappa^2+\frac{1}{2}\, \sum_{p,q}\, b_{pq}(\xi) a_p(\xi)a_q(\xi)
-b(\xi)}\tag{i}
\]
Following Newton's construction in (1.1) we
put:
\[
H(x,\xi;\kappa)=\frac{1}{4\pi}\cdot
 \frac{\sqrt{\Delta(\xi)}^{-\frac{1}{2}}}{
\sqrt{ \sum_{p,q}\, b_{pq}(\xi)\cdot x_px_q}}
\cdot e^{-\alpha_\kappa(\xi) \sqrt{B(x)}-
\frac{1}{2}\sum_{p,q}b_{pq}(\xi) a_p(\xi)\cdot x_q} \tag{ii}
\]

\noindent
When $\xi$ is kept the  function of
$x\to H(x,\xi;\kappa)$ is locally integrable
as a function of $x$ in a neighborhood of the origin.
We are going to construct  a fundamental solution
which takes the form
\[
\Phi(x,\xi;\kappa)=
H(x-\xi,\xi;\kappa)+\int_{{\bf{R}}^3}\, 
H(x-y,\xi;\kappa)\cdot\Psi(y,\xi;\kappa)\, dy\tag{iii}
\]
where the $\Psi$-function is the solution to an integral equation
which we define  in (1.5). But first we need another construction.

\medskip

\noindent
{\bf{1.3 The function $F(x,\xi;\kappa)$.}}
For every fixed $\xi$ we get  the  differential operator in the
$x$-space:
\[ 
L_*(x,\partial_x,\xi;\kappa)=
\]
\[\sum_{p=1}^{p=3}\sum_{q=1}^{q=3}\, (a\uuu{pq}(x)-
(a\uuu{pq}(\xi))\cdot 
\frac{\partial^2}{\partial x_p\partial x_q}+
\sum_{p=1}^{p=3}\,
(a_p(x)-a_p(\xi))\frac{\partial}{\partial x_p}+ (b(x)-b(\xi))
\]

\medskip

\noindent
Apply $L_*$ to the function
$x\mapsto H(x-\xi,\xi;\kappa)$
and put
\[ 
F(x,\xi;\kappa)=\frac{1}{4\pi}\cdot L_*(x,\partial_x,\xi;\kappa)(H(x-\xi,\xi,\kappa)) \tag{1.3.1}
\]

\medskip

\noindent
{\bf{1.4 Two  estimates.}}
The limit conditions  in (0.0)     give  positive constants
$C,C_1$ and $k$ such that the following hold when $\kappa\geq\kappa_0$:
\[
|H(x-\xi,\xi;\kappa)|\leq C\cdot \frac{e^{-k\kappa|x-\xi|}}{|x-\xi|}
\quad\colon\,
[F(x,\xi;\kappa)|\leq C_1\cdot 
\frac{e^{-k\kappa|x-\xi|}}{|x-\xi|^2}
\tag{1.4.1}
\]


\noindent
The verification of (1.4.1) is left as an exercise.



\bigskip





\noindent
{\bf{1.5 An integral equation.}}
We  seek
$\Psi(x,\xi;\kappa)$ which satisfies the equation:
\[ 
\Psi(x,\xi;\kappa)= \int_{{\bf{R}}^3}\,  F(x,y;\kappa)\cdot \Psi(y,\xi;\kappa)\,dy+F(x,\xi;\kappa)\tag{1.5.1}
\]
To solve (1.5.1 )
we construct the Neumann series of $F$.
Thus, starting with $F^{(1)}=F$ we set
\[
F^{(k)}(x,\xi;\kappa)=\int_{{\bf{R}}^3}\, F(x,y;\kappa)\cdot
F^{(k-1)}(y,\xi;\kappa)\, dy\quad\colon\quad k\geq 2\tag{1.5.2}
\]
Then (1.4.1 ) gives  the inequality
\[
|F^{(2)}(x,\xi;\kappa)|
\leq C_1^2\iiint
\frac{e^{-k\kappa|\xi-y|}}{|x-y|^2\cdot |\xi-y|^2}\cdot dy\tag{i}
\]


\noindent
To estimate (i) we  notice that the triple integral after
the substitution $y-\xi\to u$
becomes
\[
C_1^2\iiint
\frac{e^{-k\kappa|u|^2}}{|x-u-\xi|^2\cdot |u|^2}\cdot du\tag{ii}
\]


\noindent
In  (ii)  the volume integral can be integrated in polar
coordinates
and becomes
\[
C_1^2\cdot \int_0^\infty\int_{S^2}\, 
\frac{e^{-k\kappa r^2}}{|x-r\cdot w-\xi|^2}\cdot dwdr\tag{iii}
\]
where $S^2$ is the unit sphere and $dw$ the area measure on
$S^2$. It follows that
(iii) becomes
\[
2\pi C_1^2\cdot
\int_0^\infty\int_0^\pi\, 
\frac{e^{-k\kappa r}}{(x-\xi)^2+r^2-
2r\cdot |x-\xi|\cdot \sin\theta}\cdot d\theta dr=
\]
\[
\frac{2\pi C_1^2}{|x-\xi|}\cdot\int_0^\infty\, e^{-k\kappa |x-\xi|t}\cdot
\log\, |\frac{1+t}{1-t}|\cdot \frac{dt}{t}\tag{iv}
\]
where the last equality follows by a straightforward computation.

\medskip


\noindent
{\bf{1.6 Exercise.}}
Show that (iv) gives the estimate
\[
|F^{(2)}(x,\xi;\kappa)|\leq \frac{2\pi\cdot  C_1^2\cdot C_1^*}{\kappa\cdot |x-\xi|^2}
\]
where $C_1^*$ is a fixed positive constant
which is independent of $x$ and $\xi$ and 
show by an induction over $n$
that one has:
\[
|F^{(n)}(x,\xi;\kappa)|\leq\frac{C_1}{|x-\xi|^2}\cdot 
\bigl[\frac{2\pi C_1^2\cdot C_1^*}{\kappa}\bigr]^{n-1}
\quad\text{for every}\quad  n\geq 2\tag{*}
\]
\medskip

\noindent
{\bf{1.6 Conclusion.}}
Choose  $\kappa_0^*$ so large that
\[
2\pi C_1^2\cdot C_1^*<\kappa_0^*\tag{1.6.1}
\]
Then (*) implies that 
the Neumann series
\[
\sum_{n=1}^\infty F^{(n)}(x,\xi;\kappa)
\]
converges when 
$\kappa\geq \kappa_0^*$ 
and
gives the requested solution $\Psi(x,\xi;\kappa)$ in (1.5.1).
\bigskip


\noindent
{\bf{1.7 Exercise.}}
We  have found 
$\Psi$ which satisfies the integral equation in � 1.5.1.
Next, since the $H$-function in (ii) from � 1.2 is everywhere positive
the integral equation (iii)in � 1.2 has a unique solution
$\Phi(x,\xi;\kappa)$. Using Green's formula the reader can check that
$\Phi(x,\xi;\kappa)$ yields   a fundamental solution
of $L(x,\partial_x)-\kappa^2$.

\medskip


\noindent{\bf {1.8 Some estimates.}}
The  constructions above show that
the  functions
\[
x\mapsto \Phi(x,\xi;\kappa)\quad\text{and}\quad 
x\mapsto H(x-\xi,\xi;\kappa)
\]


\noindent
have the same  singularities at $x=\xi$.
Consider the difference
\[
Q(x,\xi;\kappa)=\Phi(x,\xi;\kappa)-
H(x-\xi,\xi;\kappa)\tag{1.8.1}
\]
\medskip


\noindent
{\bf{1.8.2 Exercise.}}
Use the previous constructions to show
that for every $0<\gamma\leq 2$
there is a constant $C_\gamma$
such that
\[
\bigl |Q(x,\xi;\kappa)\,\bigr|\leq \frac{C_\gamma}{(\kappa|x-\xi|)^{\gamma}}
\]
hold for every pair $(x,\xi)$ and every $\kappa\geq \kappa_0$.
Finally, the reader can apply 
the inequality for the
$H$-function in (1.4.1) to conclude the results in the Main Theorem.



\newpage

\centerline {\bf{� 2. Green's functions.}}
\bigskip


\noindent
Let $\Omega$ be a bounded domain in
${\bf{R}}^3$, and $L$ an elliptic differential operator as in � 1.
Let $\kappa>0$ and suppose we have found a function
$G(x,y;\kappa)$ defined when $(x,y)\in \Omega\times\Omega$
with the property that
$G(x,y)=0$   if $x\in\partial\Omega$ and  $y\in\Omega$.
Moreover
\[
(L(x,\partial_x)-\kappa^2)(G(x,y;\kappa))=\delta(x-y)
\]
where $\delta$ denotes the usual Dirac distribution. Taking  $G$ as a  kernel we get the integral operator
\[
\mathcal G(f)(x)=
\int_\Omega\, 
G(x,y;\kappa)\, f(y)\, dy 
\]
Then $\mathcal G(f)(x)=0$ on $\partial\Omega$
and the composed operator 
\[
(L(x,\partial_x)-\kappa^2)\circ \mathcal G=E
\]
To construct $G$ we use the fundamental solution
$\Phi(x,y;\kappa)$ from � 1 which satisfies
\[
(L(x,\partial_x)-\kappa^2)(\Phi(x,y;\kappa))=\delta(x-y)\tag{2.0.0}
\]
Next, with $y\in\Omega$ kept fixed we have the 
continuous boundsry functon
\[
x\mapsto \Phi(x,y;\kappa)
\]
Solving the Dirchlet problem we find
$w(x)$ such that
$w(x)= \Phi(x,y;\kappa)$ on the boundary while
$ (L(x,\partial_x)-\kappa^2)(w)=0$ holds in $\Omega$.
Then we can take
\[
G(x,y;\kappa)=\Phi(x,y;\kappa)-w(x)\tag{2.0.1}
\]


\noindent
Using the estimates for the $\Phi$-function from � 1
we get
estimates for this $G$-function.
We can also choose    a sufficiently large
$\kappa_0$ so  that 
$\Phi(x,\xi;\kappa_0)$ is a positive function of
$(x,\xi)$. Then the following hold:
\medskip


\noindent
{\bf{2.1 Theorem.}}
\emph{One has}
\[ 
G(x,\xi;\kappa_0)=
\frac{1}{\sqrt{\Delta(x)}\cdot\sqrt{\Phi(x,\xi;\kappa_0)}}
+R(x,\xi)
\]

\noindent
\emph{where the remainder function satisfies the following for all pairs
$(x,\xi)$ in $\Omega$:}
\[ 
|R(x,\xi)|\leq C\cdot |x-\xi|^{-\frac{1}{4}}
\]


\noindent
\emph{and the   constant $C$ only  depends on the  domain 
$\Omega$ and
the PDE-operator $L$.}
\medskip

\noindent
{\bf{Remark.}}
Above the negative power  of 
$|x-\xi|$ is  a fourth-root which means that  the remainder term $R$ 
is more regular  compared
to the first term which behaves like $|x-\xi|^{-1}$ on the diagonal $x=\xi$.

\bigskip

\noindent
{\bf{2.2 Exercise.}} Prove Theorem 2.1 
If necessary, consult [Carleman: page xx-xx] for details.

\bigskip

\medskip

\noindent
\centerline {\bf{2.3. Almost reality of eigenvalues.}}
\medskip

\noindent
Consider the set of eigenvalues $\lambda$ for which there exists a function $u$
in $\Omega$ which is zero on
$\partial\Omega$ while
\[
L(u)+\lambda\cdot u=0
\]
holds in $\Omega$.


\medskip

\noindent
{\bf{2.3.1  Proposition.}}
\emph{There exist positive constants $C_*$ and $c_*$ such that
every eigenvalue  $\lambda$ above   satisfies}
\[
|\mathfrak{Im}\,\lambda|^2\leq C_*(\mathfrak{Re}\,\lambda)+c_*)
\]
\medskip

\noindent
\emph{Proof.}
Let $u$ be an eigenfunction where
$L(u)+\lambda\cdot u=0$.
Stokes theorem and the vanishing of $u|\partial\Omega$
give:
\[
0=\int_\Omega\, \bar u\cdot (L+\lambda)(u)\,dx
=-\int_\Omega \, \sum_{p,q}\, a_{pq}(x)\cdot \frac{\partial u}{\partial x_p}
\frac {\partial \bar u}{\partial x_q}\, dx+
\int_\Omega\, \bar u\cdot( \sum \, a_p(x)
\frac{\partial u}{\partial x_p}\,)\, dx+
\] 
\[
\int_\Omega\, |u(x)|^2\cdot b(x)\, dx+
\lambda\cdot \int\, |u(x)|^2\, dx
\]
\medskip


\noindent
Write $\lambda=\xi+i\eta$.
Separating real and imaginary parts we find the two equations:
\[
\xi\int\, |u|^2\, dx=
\int\, \sum_{p,q} a_{p,q}(x)\,\frac{\partial u}{\partial x_p}\cdot 
\frac{\partial \bar u}{\partial x_q}\, dx+
\int\, \bigl(\frac{1}{2}\cdot \sum\, \frac{\partial a_p}{\partial x_p}- b\,\bigr )
\cdot |u|^2\, dx\tag{i}
\]
\[
\eta\int\, |u|^2\, dx=\frac{1}{2i}\int \sum\, a_p\bigl(
u\frac{\partial \bar u}{\partial x_p}-
\bar u \frac{\partial u}{\partial x_p}\,\bigl )\, dx\tag{ii}
\]


\noindent
Set
\[ A= \int\, |u|^2\,dx\quad\colon\quad
B= \int\, |\nabla(u)|^2\,dx
\]
Since $L$ is elliptic there exists a positive constant $k$ such that
\[
\sum_{p,q} a_{p,q}(x)\,\frac{\partial u}{\partial x_p}>
k\cdot |\nabla(u)|^2
\]
From this we see that (i-ii) gives positive constants $c_1,c_2,c_3$ such that
\[
A\xi>c_1B-c_2B\quad\colon\quad A|\eta|<c_3\cdot \sqrt{AB}\tag{iii}
\]
\medskip

\noindent
Here (iii) implies that $\xi>-c_2$ and the reader can also confirm that
\[ 
B<\frac{A}{c-1}(\xi+c-2)\quad\colon\quad
A|\eta|< A\cdot c_2\cdot \sqrt{\frac{\xi+c_2}{c_1}}\quad\colon\quad
|\eta|< c_3\cdot \sqrt{\frac{\xi+c_2}{c_1}}\tag{iv}
\]


\noindent Finally it is obvious that (iv) above gives the requested
inequality in Proposition 2.3.1.

\bigskip

\centerline{\bf{2.4. Asymptotic  formula for eigenvalues}}
\bigskip


\noindent
Consider a function $f$ which satisfies
\[
\mathcal G(f)=-\frac{1}{\lambda}\cdot f
\]
for some non-zero complex number $\lambda$.
With $u=\mathcal G(f)$ it follows from the previous material  that
\[
(L-\kappa^2)(u)=f=-\lambda\cdot u
\]
Hence
\[
L(u)+(\lambda-\kappa^2)u=0
\]
\medskip


\noindent
From  the above the asymptotic formula in the Main Theorem from the introduction
can be derived from asymptotic properties of eigenvalues to
the integral operator $\mathcal G$. More precisely, 
using Theorem 2.1 and the estimates for the fundamental  solution
$\Phi$ in � 1, one can proceed as in the
the next section and employ a  Tauberian theorem.
The  reader may try to supply details or  consult 
[Carleman: page xx-xx] for details
after reading the proof of another asymptotic formula in
the next section.














\newpage














\newpage

\centerline{\bf{� 3. A study of $\Delta(\phi)+\lambda\cdot \phi$.}}
\bigskip


\noindent
{\bf{Introduction.}}
We expose material from Carleman's article \emph{xxx}
presented at the Scandinavian Congress in Stockholm 1934.
In
${\bf{R}}^2$ we 
consider a bounded Dirichlet regular domain
$\Omega$, i.e. every $f\in C^0(\partial\Omega)$
has a harmonic extension to $\Omega$.
A wellknown  
fact
established by
G. Neumann and H. Poincar�
during the years 1879-1895
gives the following: First there
exists the Greens' function
\[
G(p,q)= \log\,\frac{1}{|p-q|}+H(p,q)
\]
where $H(p,q)= H(q,p)$ is continuous in
the product set
$\overline{\Omega}\times\overline{\Omega}$ with the property that
the operator $\mathcal G$ defined on $L^2(\Omega)$ by
\[
f\mapsto \mathcal G_f(p)= \frac{1}{2\pi}\dot \iint\, G(p,q)f(q)\,dq
\]
satisfies
\[ 
\Delta\circ \mathcal G_f=-f\quad\colon f\in L^2(\Omega)
\]
Moreover,  $\mathcal G$ is a compact operator on
the Hilbert space $L^2(\Omega)$ and there 
exists a sequence $\{f_n\}$ in $L^2(\Omega$ such that
$\{\phi_n=\mathcal G_{f_n}\}$
is an orthonormal basis in $L^2(\Omega$ and
\[ 
\Delta(\phi_n)=-\lambda_n\cdot \phi_n\quad\colon\, n=1,2,\ldots
\] 
where
$0<\lambda_1\leq \lambda_2\leq \ldots\}$.
When  eigenspaces have  dimension $\geq 2$, the 
eigenvalues are repeated by their multiplicity.
\medskip

\noindent
{\bf{Main Theorem. }}\emph{For every Dirichlet regular domain
$\Omega$ and each $p\in\Omega$�one has the limit formula}
\[ 
\lim_{N\to\infty}\, \lambda_N^{-1}\cdot \sum_{n=1}^{n=N}\, \phi_n(p)^2= \frac{1}{4\pi}
\]
\medskip

\noindent
The strategy in the proof is to consider the function of a complex variable $s$
defined by
\[
\Phi(s)=\sum_{n=1}^\infty \frac{\phi_n(p)^2}{\lambda_n^s}
\]
and show that it
is a meromorphic function in the whole complex $s$-plane with
a simple pole at $s=1$ whose residue is $\frac{1}{4\pi}$.
More precisely we shall prove:
\medskip


\noindent
{\bf{3.1 Theorem.}}
\emph{There exists an entire function
$\Psi_p(s)$ such that}
\[
\Phi_p(s)=\Psi_p(s)+\frac{1}{4\pi(s-1)}
\]
\medskip

\noindent
Let us first remark that Theorem 3.1   gives ther Main Theorem 
by a result  due to Wiener in the article
\emph{Tauberian theorem} [Annals of Math.1932].
Wiener's theorem 
asserts that if $\{\lambda_n\}$ is a non-decreasing sequence of
positive numbers which tends to infinity and
$\{a_n\}$ are non-negative real numbers such that
there exists the limit
\[ 
\lim_{s\to 1}\,(s-1)\cdot \sum\, \frac{a_n}{\lambda_n^s}=A
\] 
then it follows that
\[
\lim_{n\to \infty}\,\lambda_n^{-1}\cdot
\sum_{k=1}^{k=n}\, a_k=A
\]
\medskip

\noindent
{\bf{Exercise.}}
Derive the main theorem from Wiener's result and Theorem 3.1.
\medskip

\noindent
{\bf{About Wiener's result.}}
It is a version of
a famous   theorem proved by
Hardy and Littlewood in 1913 which goes as follows:
\medskip

\noindent
{\bf{0.2 The Hardy-Littlewood theorem.}}
\emph{Let $\{a_n\}$ be a sequence of non-negative real numbers
such that}
\[ 
A=
\lim_{r\to 1}\, (1-r)\cdot \sum\,a_nr^n\tag{*}
\]
\emph{exists. Then  there also exists the limit}
\[
A=\lim_{N\to \infty}\, \frac{a_1+\ldots+a_N}{N}\tag{**}
\]
\medskip

\noindent
Notice that no growth condition is imposed on  the sequence 
$\{a_n\}$, i.e. the sole assumption
is the existing limit (*). The proof is quite demanding and does not follow by
"abstract nonsense" from functional analysis.
For the reader's convenience we include details of the proof in a 
separate appendix since
courses devoted to such a basic and very delicate topic as the study of series rarely appear in contemporary
education where too much attention often is given to more "soft snalysis".



\bigskip









\centerline{\bf{� 1. Proof of Theorem 3.1.}}
\bigskip

\noindent
Let
$\Omega$ be a bounded and Dirichlet regular domain.
For ech fixed point  $p\in\Omega$
we 
get  the continuous function on
$\partial\Omega$ defined by
\[ 
q\mapsto  \log \frac{1}{|p-q|}
\]
We find the harmonic function
$u_p(q)$ in $\Omega$ such that
$u_p(q)=\log \frac{1}{|p-q|}\,\colon\, q\in\partial\Omega$.
Green's function is defined for pairs $p\neq q$ in
$\Omega\times\Omega$ 
by
\[ 
G(p,q)= \log\,\frac{1}{|p-q|}-u_p(q)\tag{1}
\]
Keeping
if $p\in\Omega$  fixed, the function
$q\mapsto G(p,q)$ extends to the 
closure of $\Omega$ where it vanishes if
$q\in\partial\Omega$.
If  $f\in L^2(\Omega$
we set
\[ 
\mathcal G_f(p)=\frac{1}{2\pi}\cdot \int_\Omega\, G(p,q)\cdot f(q)\, dq\tag{2}
\]
where $q=(x,y)$ so that $dq=dxdy$ when the double integral is evaluated.
From (1) we see that
\[
 \iint_{\Omega\times \Omega}\, |G(p,q)|^2\, dpdq<\infty
\]
Hence
$\mathcal G$ is of the Hilbert-Schmidt type and
therefore a compact operator on 
$L^2(\Omega)$.
Next,  recall that
$\frac{1}{2\pi}\cdot \log\sqrt{x^2+y^2}$ is
a fundamental solution to the Laplace operator.
From this the reader can   deduce the following:

\medskip

\noindent{\bf{1.1 Theorem.}}
\emph{For each $f\in L^2(\Omega)$
the Lapacian of $\mathcal G_f$
taken in the distribution sense belongs to
$L^2(\Omega)$ and one has the equality}
\[ 
\Delta(\mathcal G_f)=-f\tag{*}
\] 
\medskip


\noindent
The equation (*) means that
the composed operator
$\Delta\circ \mathcal G$ is minus the identity on
$L^2(\Omega)$.
We are  led
to introduce the 
linear operator $S$ on $L^2(\Omega)$ 
defined by $\Delta$, where  
$\mathcal D(S)$ is
the range of $\mathcal G$.
If $g\in C^2_0(\Omega)$, i.e. twice differentiable and with
compact support, it follows via Greens' formula that
\[
\frac{1}{2\pi}\cdot \int_\Omega\, G(p,q)\cdot \Delta(g)(q)\, dq=-g(p)
\]
In particular $C_0^2(\Omega)\subset\mathcal D(S)$
which  implies that
$S$ is densely defined and we  leave it to the reader to verify that
\[
\mathcal G(\Delta(f))=-f\quad\colon f\in\mathcal D(S)
\]


\medskip
\noindent
{\bf{Remark.}}
By Carl Neumann's classic construction of
resolvent operators from 1879, the result above   means that
$-\mathcal G$ is Neumann's inverse  of 
$S$. Since
$-\mathcal G$ is compact it follows by Neumann's formula for spectra that
$S$ has a discrete spectrum,and we recall the following wellknown fact which
goes back to work by Poincar�: 
\medskip

\noindent
{\bf{1.2 Proposition.}}\emph{
There exists an orthonormal basis $\{\phi_n\}$ in $L^2(\Omega)$
where each $\phi_n\in\mathcal D(S)$ is an eigenfunction, and
a non-decreasing sequence of positive real numbers
$\{\lambda_n\}$ such that}
\[ 
\Delta(\phi_n)+\lambda_n\cdot \phi_n=0\quad\colon n=1,2,\ldots\tag{1.2.1}
\]
\medskip

\noindent
{\bf{Remark. }} Above (1.2.1) means 
that
\[
\mathcal G(\phi_n)= \frac{1}{\lambda_n}\, \cdot \phi_n
\]
This, 
$\{\lambda_n^{-1}\}$
are  eigenvalues of the compact operator $\mathcal G$ whose sole cluster
point is $\lambda=0$.
As usual eigenvalues whose eigenspaces have
dimension $e>1$ are repeated $e$ times.



\medskip

\noindent
After these preliminaries we embark upon the proof
of Theorem 0.1. 
First, since $\mathcal G$ is a Hilbert-Schmidt operator a wellknown result due to Schur
gives
\[
\sum\, \lambda_n^{-2}<\infty \tag{i}
\]
This convergence entails that various constructions below are defined.
For each complex number $\lambda$ outside $\{\lambda_n\}$ we set
\[
G(p,q;\lambda)=
G(p,q)+
2\pi\lambda\cdot \sum_{n=1}^\infty\,
\frac{\phi_n(p)\phi_n(q)}{\lambda_n(\lambda-\lambda_n)}\tag{ii}
\]
This gives the integral operator
$\mathcal G_\lambda$ defined on $L^2(\Omega)$ by
 \[ 
 \mathcal G_\lambda(f)(p)
 =\frac{1}{2\pi}\cdot \iint_\Omega\, G(p,q;\lambda )\cdot f(q)\, dq\tag{iii}
\]

\medskip

\noindent
{\bf{A. Exercise.}} Use that the eigenfunctions $\{\phi_n\}$ is an orthonormal basis in
$L^2(\Omega)$ to show that
\[
(\Delta+\lambda)\cdot \mathcal G_\lambda=-E
\]


\noindent{\bf{B. The function $F(p,\lambda)$}}.
Set
\[ 
F(p,q,\lambda)= G(p,q;\lambda)- G(p,q)
\]
Keeping $p$ fixed we see that (ii) gives
\[
\lim_{q\to p}\, F(p,q,\lambda)=
2\pi\lambda\cdot \sum_{n=1}^\infty\,
\frac{\phi_n(p)^2}{\lambda_n(\lambda-\lambda_n)}\tag{B.1}
\]
Set
\[
F(p,\lambda)=
\lim_{q\to p}\, F(p,q,\lambda)
\]
From (i) and (B.1) it follows that it is a meromorphic function in
the complex $\lambda$-plane with at most simple poles
at $\{\lambda_n\}$.
\medskip

\noindent{\bf{C. Exercise.}}
Let $0<a<\lambda_1$. Show via residue calculus that
one has the equality below in a half-space
$\mathfrak{Re}\, s>2$:
\[ 
\Phi(s)=
\frac{1}{4\pi^2 \cdot i}\cdot \int_{a-i\infty}^{a+i\infty}\, 
F(p,\lambda)\cdot \lambda^{-s}\, d\lambda\tag{C.1}
\]
where the line integral  is taken on the vertical  line
$\mathfrak{Re}\,\lambda=a$.

\medskip

\noindent
{\bf{D. Change of contour integrals.}}
At this stage we employ a device which goes to
Riemann and
move the integration into the half-space
$\mathfrak{Re}(\lambda)<a$.
Consider  the curve $\gamma_+$
defined as the union of the
negative real interval $(-\infty,a]$ followed by
the upper
half-circle $\{\lambda= ae^{i\theta}\,\colon 0\leq\theta\leq \pi \}$
and the 
half-line $\{\lambda= a+it\,\colon t\geq 0\}$.
Cauchy's theorem entails that 
\[ 
\int_{\gamma_+}\, F(p,\lambda)\cdot \lambda^{-s}\, d\lambda=0
\]
We leave it to the reader to contruct the
similar
curve
$\gamma_-=\bar \gamma_+$. Using 
the vanishing of these line integrals and taking the branches of the 
multi-valued function
$\lambda^s$ into the account the reader should verify the following:

\medskip


\noindent
{\bf{E. Lemma.}}
\emph{One has the equality}
\[ 
\Phi(s)=\frac{a^{s-1}}{4\pi}\cdot \int_{-\pi}^\pi\,
F(ae^{i\theta})\cdot e^{(i(1-s)\theta}\,d\theta
+
\frac{\sin \pi s}{2\pi^2}\cdot \int_a^\infty\, F(p,-x)\cdot x^{-s}\,dx\tag{E.1}
\]
\medskip

\noindent
The first term in the sum of the right hand side of (E.1)
is obviously an entire function of $s$. So there remains to
prove that
\[
 s\mapsto  \frac{\sin \pi s}{2\pi^2}\cdot \
 \int_a^\infty\, F(p,-x)\cdot x^{-s}\,dx\tag{E.2}
\]
is meromorphic with
a single pole at $s=1$ whose residue is $\frac{1}{4\pi}$.
To attain this we  express $F(p,-x)$ when $x$ are real and positive in another way.
\medskip

\noindent
{\bf{F.  The $K$-function.}}
In the half-space $\mathfrak{Re}\,z>0$ there exists the analytic function
\[
K(z)= \int_1^\infty\, \frac{e^{-zt}}{\sqrt{t^2-1}}\,dt
\]
\medskip

\noindent
{\bf{Exercise.}}
Show that $K$ extends to a multi-valued analytic function outside
$\{z=0\}$ given by
\[
K(z)=-I_0(z)\cdot \log z+ I_1(z)\tag{F.1}
\] 
where $I_0$ and $I_1$ are entire functions
with series expansions
\[
I_0(z)=\sum_{m=0}^\infty\, \frac{2^{-2m}}{(m!)^2}\cdot
z^{2m}\tag{i}
\]
\[ 
I_1(z)= \sum_{m=0}^\infty\, \rho(m)\cdot
\frac{2^{-2m}} {(m!)^2} \cdot z^{2m}\quad
\colon \rho(m)=1+\frac{1}{2}+\ldots+\frac{1}{m}-\gamma\tag{ii}
\]
where $\gamma$ is the usual Euler constant.

\bigskip


\noindent
With  $p$ kept fixed and $\kappa>0$ 
we solve the Dirichlet problem and find
a  function $q\mapsto H(p,q;\kappa)$ which satisfies  the
equation
\[
 \Delta(H)-\kappa\cdot H=0\tag{F.2}
\] 
in $\Omega$ with boundary values
\[ 
H(p,q;\kappa)=K(\sqrt{\kappa}|p-q|)\quad\colon q\in \partial\Omega
\]


\noindent
{\bf{G. Exercise.}}
Verify the equation
\[ 
G(p,q;-\kappa)=K(\sqrt{\kappa}\cdot |p-q|)- H(q;\kappa)\quad\colon \kappa>0
\]



\noindent
Next,   the construction of $G(p,q)$ gives
\[
 F(p,-\kappa)=
 \lim_{q\to p}\,
 [K(\sqrt{\kappa}\cdot |p-q|)+\log\,|p-q|]+
 \lim_{q\to p}\,[u_p(q)+ H(p,q,\kappa)]\tag{G.1}
\]
The last term above has the  "nice limit" 
$u_p(p)+H(p,p,\kappa)$ and from  (F.1)  the reader can  verify the limit formula:
\[
 \lim_{q\to p}\,
 [K(\sqrt{\kappa}\cdot |p-q|)+\log\,|p-q|]=
 -\frac{1}{2}\cdot \log \kappa +\log 2-\gamma\tag{G.2}
\]
where $\gamma$ is  Euler's constant.

\bigskip

\noindent
{\bf{H. Final part of the proof.}}.
Set $A=  +\log 2-\gamma+u_p(p)$. Then (G.1) and (G.2)
give
\[
F(p,-\kappa)= -\frac{1}{2}\cdot \log \kappa +A+H(p,p;-\kappa)
\]
With $x=\kappa$ in (E.2 ) we  proceed  as follows.
To  begin with it is clear that
\[
s\mapsto A\cdot 
\frac{\sin \pi s}{2\pi^2}\cdot \int_a^\infty\,  x^{-s}\,dx
\]
is an entire function of $s$.
Next,  consider the function
\[ 
\rho(s)=
 -\frac{1}{2}\cdot 
\frac{\sin \pi s}{2\pi^2}\cdot \int_a^\infty\,  \log x\cdot x^{-s}\,dx
\]
Notice that the complex derivative
\[
\frac{d}{ds}\,  \int_a^\infty\,  x^{-s}\,dx=
- \int_a^\infty\,  \log x\cdot x^{-s}\,dx
\]

\medskip
\noindent
{\bf{H.1 Exercise.}}
Use the  above to show that
\[
\rho(s)-\frac{1}{4\pi(s-1)}
 \]
is an entire function.
\medskip


\noindent
From the above we see that Theorem 0.1  follows if we have proved
\medskip

\noindent
 {\bf{H.2 Lemma.}}
\emph{The following function  is entire}:
\[
s\mapsto \frac{\sin\,\pi s}{2\pi^2}\cdot
\int_a^\infty\, H(p,p,\kappa)\cdot  \kappa^{-s}\,d\kappa
\]
\medskip

\noindent
\emph{Proof.}
When $\kappa>0$
the equation (F.1) shows that $q\mapsto H(p,q;\kappa)$
is subharmonic  in $\Omega$ and the maximum principle gives
\[
0\leq  H(p,q;\kappa)\leq \max_{q\in\partial\Omega}\,K(\kappa|p-q|)\tag{i}
\]
With  $p\in\Omega$ fixed there is 
a positive number
$\delta$ such that
$|p-q|\geq\delta\,\colon q\in \partial\Omega$ which  
gives
positive constants
$B$ and  $\alpha$  such that
\[
H(p,p;\kappa)\leq e^{-\alpha\kappa}\quad\colon \kappa>0\tag{ii}
\]
The reader may now check that this
exponential decay gives Lemma H.2.

\newpage



\centerline{\bf {Appendix. Theorems by Abel, Tauber, Hardy and Littlewood}}
\bigskip


\noindent
{\bf Introduction.}
Consider a power series
$f(z)=\sum\, a_nz^n$ whose radius of convergence is one.
If $r<1$ and $0\leq\theta\leq 2\pi$
we are sure that the series
\[ 
f(re^{i\theta})= 
\sum\, a_nr^ne^{in\theta}
\]
is convergent. In fact, it is even absolutely convergent since
the assumption implies that
\[\sum\, |a_n|\cdot r^n<\infty\quad\text{for all}\quad r<1
\]
Passing to $r=1$ it is in general not true that the series
$\sum\, a_ne^{in\theta}$ is convergent. An example arises if we consider
the geometric series
\[ 
\frac{1}{1-z}= 1+z+z^2+\ldots
\] 
This leads to the following  problem where we without loss of
generality can take 
$\theta=0$.
Consider as above a
convergent power series and assume that there exists the limit
\[
\lim_{r\to 1}\, \sum\, a_nr^n\tag{*}
\]
When can we conclude that the series
$\sum\, a_n$ also is convergent and that
one has the equality
\[
 \sum\, a_n=
\lim_{r\to 1}\, \sum\, a_n r^n\tag{**}
\]
The first result in this direction was established by Abel in a work from 1823:
\medskip

\noindent
{\bf{A. Theorem}} \emph{Let $\{a_n\}$ be a sequence such that
$\frac{a_n}{n}\to 0$
as $n\to \infty$
and there exists}
\[
A=\lim_{r\to 1}\, \sum\, a_nr^n
\] 
\emph{Then  
$\sum\, a_n$ is convergent and the  sum is $A$.}

\medskip

\noindent
An extension of Abel's result was
established by  Tauber in 
1897. 
\medskip

\noindent
{\bf B. Theorem.}
\emph{Let $\{a_n\}$ be a sequence of real numbers such that
there exists the limit}
\[
A=\lim_{r\to 1}\, \sum\, a_nr^n
\]
Set
\[ 
\omega_n=a_1+2a_2+\ldots+na_n\quad\colon\, n\geq 1
\]
\emph{If $\lim_{n\to\infty}\,\omega_n=0$ it follows that
the series $\sum\, a_n$ is convergent and the sum is $A$.}


\bigskip

\noindent
{\bf C. Results by 
Hardy and Littlewood.}
In their joint article \emph{xxx} from 1913
the following extension of Abel's  result was proved by Hardy and Littlewood:

\medskip

\noindent
{\bf {C. Theorem.}}
\emph{Let $\{a_n\}$ be a sequence of real numbers such that there 
exists a constant $C$ so that
$\frac{a_n}{n}\leq C$ for all $n\geq 1$. Assume also that the
power series $\sum\, a_nz^n$ converges when
$|z|<1$. Then the same conclusion as in Abel's theorem holds.}
\medskip

\noindent
{\bf {Remark.}}
In addition to this they proved
a result about
positive series from the cited article which has independent interest.

\medskip

\noindent
{\bf {D. Theorem.}}
\emph{Assume that
each $a_n\geq 0$ and that there exists the limit:}
\[ 
A=
\lim_{r\to 1}\, (1-r)\cdot \sum\,a_nr^n\tag{*}
\]
\emph{Then  there exists the limit}
\[
A=\lim_{N\to \infty}\, \frac{a_1+\ldots+a_N}{N}\tag{**}
\]
\medskip

\medskip

\noindent
{\bf{Remark.}} The proofs of Abel's and Tauber's results are  easy
while C and D require more effort and rely upon 
results from calculus in one variable.
So before we enter the proofs of the  theorems above insert
some  preliminaries.
\newpage

\centerline{\bf{1. Results from calculus}}

\medskip

\noindent
Below $g(x)$ is a real-valued function defined on $(0,1)$ and
of class $C^2$ at least.
\medskip


\noindent
{\bf 1.1 Lemma } \emph{Assume that there exists a constant $C>0$ such that}
\[
g''(x)\leq C(1-x)^{-2}\quad\colon\, 0<x<1\quad\text{and}\quad
\lim_{x\to 1}\, g(x)=0
\] 
\emph{Then one  has the limit formula}:
\[
\lim_{x\to 1}\, (1-x)\cdot g'(x)=0
\]

\medskip

\noindent
{\bf 1.2 Lemma } \emph{Assume that the second order derivative
$g''(x)>0$.
Then the following implication holds for each $\alpha>0$:}

\[
\lim_{x\to 1}\, (1-x)^\alpha\cdot g(x)=1\implies
\lim_{x\to 1}\, (1-x)^{\alpha+1}\cdot g'(x)=\alpha
\]




\noindent
{\bf{Remark.}}
If $g(x)$ has higher order derivatives
which all are
$>0$ on $(0,1)$
we can iterate the conclusion in Lemma 1.2 where 
we take $\alpha$ to be positive integers.
More precisely, by an induction over
$\nu$ the reader may verify that if
\[
\lim_{x\to 1}\, (1-x)\cdot g(x)=1
\]
exists and  if 
$\{g^{(\nu)}(x)>0\}$ for all 
every $\nu\geq 2$ then 
\[
\lim_{x\to 1}\, (1-x)^{\nu+1}\cdot g^{(\nu)}(x)=\nu\,!
\quad\colon\, \nu\geq 2\tag{*}
\]



\bigskip

\noindent
Next, to each integer $\nu\geq 1$ we denote by $[\nu-\nu^{2/3}]$
the largest integer $\leq\,(\nu-\nu^{2/3}).$ Set
\[
J_*(\nu)=\sum_{n\leq [\nu-\nu^{2/3}]}\,
n^\nu e^{-\nu}
\quad\colon\quad J^*(\nu)=\sum_{n\geq [\nu+\nu^{2/3}]}\,
n^\nu e^{-\nu}
\]

\medskip

\noindent
{\bf 1.3 Lemma }
\emph{There exists a constant $C$
such that}
\[
\frac{J^*(\nu)+J_*(\nu)}{\nu\,!}\leq \delta(\nu)\quad\colon\quad
\delta(\nu)=C\cdot \text{exp}\, \bigl(-\frac{1}{2}\cdot \nu^{\frac{1}{{3}}}\bigr )
\quad\colon\,\nu=1,2,\ldots
\]

\medskip

\centerline{\emph{Proofs}}

\bigskip

\noindent
We prove only  Lemma 1.1 which is a bit tricky 
while the proofs of Lemma 1.2 and 1.3 are
left as  exercises to the reader.
Fix $0<\theta<1$. Let
$0<x<1$ and set
\[
x_1=x+(1-x)\theta
\]
The mean-value theorem in calculus gives
\[ 
g(x_1)-g(x)=\theta(1-x)g'(x)+\frac{\theta^2}{2}(1-x)^2\cdot g''(\xi)\quad
\text{for some}\quad \, x<\xi<x_1\tag{i}
\]
By the hypothesis
\[
g''(\xi)\leq C(1-\xi)^{-2}\leq C)1-x_1)^{-2}
\]
Hence (i) gives
\[
(1-x)g'(x)\geq
\frac{1}{\theta}(g(x_1)-g(x))-
C\cdot \frac{\theta}{2}\frac{(1-x)^2}{1-x_1)^2}=
\]
\[
\frac{1}{\theta}(g(x_1)-g(x))-
\frac{C\cdot \theta}{2(1-\theta)^2}
\]
Keeping $\theta$ fixed we have by assumption
\[
\lim_{x\to 1}\, g(x)=0
\]
Notice also that $x\to 1\implies x_1\to 1$. It follows that



\[
\liminf_{x\to 1}\,\,
(1-x)g'(x)\geq -\frac{C\cdot \theta}{2(1-\theta)^2}
\]
Above  $0<\theta<1$ is arbitrary, i.e. we can choose 
small $\theta>0$ and hence we have proved that
\[
\liminf_{x\to 1}\,
(1-x)g'(x)\geq 0\tag{*}
\]
\medskip

\noindent
Next we  prove the opposed inequality

\[
\limsup_{x\to 1}\,
(1-x)g'(x)\leq 0\tag{**}
\]
To get (**) we apply the mean value theorem in the form
\[
g(x_1)-g(x)=\theta(1-x)g'(x_1)-\frac{\theta^2}{2}(1-x)^2\cdot g''(\eta)\quad
\colon\, x<\eta<x_1\tag{ii}
\]
Since $(1-x_1)=\theta(1-x)(1-\theta)$ we get
\[ 
(1-x_1)g'(x_1)=\frac{1-\theta}{\theta}\cdot (g(x_1)-g(x))+
\frac{(1-\theta)\theta}{2}\cdot(1-x)^2g''(\eta)\tag{iii}
\]
Now $g''(\eta)\leq C(1-\eta)^{-2}\leq C(1-x_1)^{-2}$ so
the right hand side in (iii) is majorized by
\[
\frac{1-\theta}{\theta}\cdot (g(x_1)-g(x))+
C\cdot \frac{(1-\theta)\theta}{2}\cdot(1-x)^2(1-x_1)^2=
\]
 \[
\frac{1-\theta}{\theta}\cdot (g(x_1)-g(x))+
C\cdot \frac{\theta}{2(1-\theta}\tag{iv}
\]
Keeping $\theta$ fixed while $x\to 1$ we obtain:
\[
\liminf_{x\to 1}\, (1-x)g'(x)\leq 
C\cdot \frac{\theta}{2(1-\theta}
\]
Again we can choose arbitrary small $\theta$ and hence (**) 
holds which finishes the proof of 
Lemma 1.1.













\bigskip


\centerline{\bf{2. Proof of Abel's theorem.}}

\medskip

\noindent
Without loss of generality we can assume that
$a_0=0$ and set  $S_N=a_1+\ldots+a_N$.
Given $0<r<1$ we let $f(r)=\sum\, a_nr^n$. For every positive integer $N$
the triangle inequality gives:
\[
\bigl| S_N-f(r)\bigr|\leq
\sum_{n=1}^{n=N}\, |a_n|(1-r^n)+
\sum_{n\geq N+1}\, |a_n|r^n
\]
Set $\delta(N)=\max_{n\geq N}\,\frac{|a_n|}{n}$.
Since  $1-r^n)=(1-r)(1+\ldots+r^{n-1}\leq (1-r)n$
the last sum is majorised by
\[
(1-r)\cdot \sum_{n=1}^{n=N}\, n\cdot |a_n|
+
\delta(N+1)\cdot
 \sum_{n\geq N+1}\, \frac{r^n}{n}
\]
Next, the obvious inequality 
$\sum_{n\geq N+1}\, \frac{r^n}{n}\leq\frac{1}{N+1}\cdot \frac{1}{1-r}$
gives the new majorisation
\[
(1-r)\cdot \sum_{n=1}^{n=N}\, \frac{|a_n|}{n}
+\frac{\delta(N+1)}{N+1}\cdot  \frac{1}{1-r}\tag{1}
\]
This hold for all pairs $N$ and $r$.
To each $N\geq 2$ we take $r=1-\frac{1}{N}$
and hence the right hand side in (1)
is majorised by

\[
\frac{1}{N}\cdot \sum_{n=1}^{n=N}\, \frac{|a_n|}{n}
+\delta(N+1)\cdot \frac{N}{N+1}
\]
Here both terms tend to zero as $N\to\infty$. Indeed, Abel's condition 
$\frac{a_n}{n}\to 0$   implies that
$\frac{1}{N}\cdot \sum_{n=1}^{n=N}\, \frac{|a_n|}{n}$ 
tends to zero as $N\to \infty$. Hence
we have proved the limit formula:
\[ 
\lim_{N\to\infty}\,\bigl |s_N-f(1-\frac{1}{N})\bigr|=0\tag{*}
\]
Finally it is clear that (*)  gives Abel's result.

\bigskip

\centerline{\bf{3. Proof of Tauber's theorem.}}
\medskip

\noindent
We may assume that $a_0=0$. Notice that
\[
a_n=\frac{\omega_n-\omega_{n-1}}{n}\quad\colon\, n\geq 1
\]
It follows that
\[ 
f(r)=\sum\, \frac{\omega_n-\omega_{n-1}}{n}\cdot r^n
=\sum\,\omega_n\bigl(\frac{r^n}{n}-\frac{r^{n+1}}{n+1}\bigr )
\]
Using  the equality
$\frac{1}{n}=
\frac{1}{n+1}=
\frac{1}{n(n+1)}$ we can  rewrite the right hand side as follows:
\[
\sum\,\omega_n\bigl(\frac{r^n-r^{n+1}}{n+1}+\frac{r^n}{n(n+1)}\bigr )
\]
Set
\[ g_1(r)=\sum\,\omega_n\cdot\frac{r^n-r^{n+1}}{n+1}
=(1-r)\cdot \sum\, \frac{\omega_n}{n+1}\cdot r^n
\]
By the hypothesis 
$\lim_{n\to\infty}\,  \frac{\omega_n}{n+1}=0$ and then it is
clear that we get
\[
\lim_{r\to 1}\, g_1(r)=0
\]
Since we also have $f(r)\to 0$ as $r\to 1$ we conclude that
\[
\lim_{r\to 1} \sum\, \frac{\omega_n}{n(n+1)}\cdot r^n=0\tag{1}
\]
Next, with
$b_n= \frac{\omega_n}{n(n+1)}$ we have
$nb_n= \frac{\omega_n}{n+1}\to 0$.
Hence Abel's theorem applies so (1) gives  convergent series
\[
\sum\, \frac{\omega_n}{n(n+1)}=0\tag{2}
\]
If $N\geq 1$ we have the partial sum
\[
S_N=\sum_{n=1}^{n=N} 
\, \frac{\omega_n}{n(n+1)}=
\sum_{n=1}^{n=N}, \omega_n\cdot\bigl(\frac{1}{n}-\frac{1}{n+1}\bigr)
\]
The last term becomes
\[
\sum_{n=1}^{n=N}\,\frac{1}{n}(\omega_n-\omega_{n-1})-
\frac{\omega_N}{N+1}=
\sum_{n=1}^{n=N}\,a_n-\frac{\omega_N}{N+1}
\]
Again, since
$\frac{\omega_N}{N+1}\to 0$ as $N\to\infty$ we conclude that the convergent series
from (2) implies that the series
$\sum\, a_n$ also is converges and has sum equal to zero.
This finishes the proof of Tauber's result.
\bigskip

\centerline{\bf{4. Proof of Theorem D. }}
\medskip

\noindent
Set $f(x)=\sum\, a_nx^n$ which is defined when $0<x<1$.
Notice that 
\[
(1-x)f(x)=\sum\, s_nx^n\quad\text{where}\quad s_n=a_1+\ldots+a_n
\]
Set $g(x)= \sum\, s_nx^n$ which is defined when
$0<x<1$.
Since $s_n\geq 0$ for all $n$  all the higher order derivatives
\[
g^{(p)}(x)= \sum_{n=p}^\infty\, n(n-1)\cdots (n-p+1)a_nx^{n-p}>0
\]
when $0<x<1$.
The hypothesis that
$\lim_{x\to 1}\, g(x)=A$ and Lemma 1.1 and the inductive result in the
remark after Lemma 1.2 give:
\[
\lim_{x\to 1}\, (1-x)^{\nu+2}\cdot
\sum\, s_n\cdot n^\nu x^n=(\nu+1)!\quad\colon\,\nu\geq 1\tag{1}
\]
We shall use the substitution $e^{-t}=x$ where $t>0$.
Since $t\simeq 1-x$ when
$x\to 1$ we see that
(1) gives
\[
\lim_{t\to 0}\, t^{\nu+2}\cdot 
\sum\, s_n\cdot n^\nu e^{-nt}=(\nu+1)!\quad\colon\,\nu\geq 1\tag{2}
\]
Let us put
\[ J_*(\nu,t)=\frac{t^{\nu+2}}{(\nu+1)!}\cdot 
\sum_{n=1}^\infty\, s_n\cdot n^\nu e^{-nt}
\]
So for each fixed $\nu$ one has 
\[
\lim_{t\to 0}\, J_*(\nu,t)=1\tag{3}
\]

\medskip

\noindent
Next, for each pair $\nu\geq 2$ and $0<t<1$ we define the integer
\[ 
N=\bigl[\frac{\nu-\nu^{2/3}}{t}\bigr]\tag{*}
\]
Since the sequence $\{s_n\}$ is non-decreasing we get
\[
s_N\cdot\sum_{n\geq N}\, n^\nu e^{-nt}\leq
\sum_{n\geq N}\, s_n\cdot n^\nu e^{-nt}\leq\frac{(\nu+1)!\cdot J_*(\nu,t)}{t^{\nu+2}}
\tag{i}
\]
Next,  the construction of $N$ and Lemma 1.3 give:

\[
\sum_{n\geq N}\, n^\nu e^{-nt}\geq \frac{\nu !}{t^{\nu+1}}\cdot (1-\delta(\nu))\tag{ii}
\]
where the $\delta$ function is independent of $\nu$ and
tends to zero as $\nu\to\infty$. Hence (i-ii) give
\[
s_N\leq \frac{(\nu+1)}{t}\cdot \frac{1}{1-\delta(\nu)}\cdot
J_*(\nu,t)\tag{iii}
\]
Next, by the construction of $N$ one has
\[ 
N+1\geq \frac{\nu-\nu^{2/3}}{t}=\frac{\nu}{t}\cdot(1-\nu^{-1/3})
\]
It follows that (iii) gives
\[
\frac{s_N}{N+1}\leq 
\frac{\nu+1}{\nu}
\cdot\frac{1}{1-\nu^{-1/3}}\cdot\frac{1}{1-\delta(\nu)}\cdot 
J_*(\nu,t)\tag{iv}
\]
Since $\delta(\nu)\to 0$
it follows that for any $\epsilon>0$ there exists some
$\nu_*$ such that
\[
\frac{\nu_*+1}{\nu_*}
\cdot\frac{1}{1-\nu_*^{-1/3}}\cdot\frac{1}{1-\delta(\nu_*)}<1+\epsilon \tag{v}
\]
\medskip

\noindent
Keeping $\nu_*$ fixed we now consider pairs $t_n,N$ such that
(*) above hold with $\nu=\nu_*$.
Notice that
\[
 N\to+\infty\implies t_N\to 0\tag{vi}
\]

\medskip
\noindent
It follows from (iv) and (v) that we have:
\[
\frac{s_N}{N+1}<(1+\epsilon)\cdot J_*(\nu_*,t_N)\quad\colon\, N\geq 2\tag{vii}
\]
Now (vi) and the limit in (3) which applies with
$\nu_*$ while $t_N\to 0$
entail that
\[
\lim_{N\to \infty}\, J(\nu_*,t_N)=1
\]
We have also that
$\frac{N}{N+1}\to 1$ and since $\epsilon>0$ was arbitrary
we see that (vii) proves  the inequality
\[ 
\limsup_{N\to\infty}\, 
\frac{s_N}{N}\leq 1\tag{1}
\]
So Theorem 2 follows if we also prove that
\[
\liminf_{N\to\infty}\, 
\frac{s_N}{N}\geq 1\tag{2}
\]
The proof of (II) is similar where we now define the integers $N$ by:
\[ 
N=\bigl[\frac{\nu+\nu^{2/3}}{t}\bigr]
\]
Then we have
\[ 
S_N\cdot\sum_{n\leq N}\, n^\nu e^{-nt}\geq
\frac{(\nu+1)!\cdot J_*(\nu,t)}{t^{\nu+2}}-
\sum_{n> N}\,s_n\cdot n^\nu e^{-nt}
\]
Here the last term can be estimated above since the Lim.sup
inequality (I) gives a constant $C$ such that
$s_n\leq Cn$ for all $n$ and then
\[
\sum_{n> N}\,s_n\cdot n^\nu e^{-nt}
\leq C\cdot \sum_{n> N}\,n^{\nu+1} e^{-nt}
\leq C\cdot \delta^*(\nu)\cdot 
\frac{(\nu+1)!}{t^{\nu+2}}
\]
where Lemma 1.3  entails that
$\delta^*(\nu)\to 0$ as $\nu$ increases. At the same time Lemma 1.3 also gives
\[
\sum_{n\leq N}\, n^\nu \cdot e^{-nt}=
\frac{\nu !}{t^{\nu+1}}\cdot (1-\delta_*(\nu)
\] 
where $\delta(\nu_*)\to 0$. At this stage the reader can verify that (2) 
by  similar methods as in the proof of (I).


\bigskip
\centerline{\bf{5. Proof of Theorem C}}
\medskip

\noindent
Set $f(x)=\sum\, a_nx^n$. Notice that it suffices to prove
Theorem C when
the limit value
\[
\lim_{x\to 1}\, \sum\,a_nx^n=0
\]
Next,
the assumption that $a_n\leq\frac{c}{n}$ for a constant $c$ gives
\[
f''(x)=\sum\, n(n-1)a_nx^{n-2}\leq c\sum\, (n-1)x^{n-2}=\frac{c}{1-x)^2}
\]
The hypothesis $\lim_{x\to 1}\, f(x)=0$ and Lemma xx therefore gives
\[
\lim_{x\to 1}\, (1-x)f'(x)=0\tag{i}
\]
Next, notice the equality
\[
\sum_{n=1}^\infty\, \frac{na_n}{c} x^n=
\frac{x}{c}\cdot f'(x)\tag{ii}
\]
At the same time
$\sum_{n=1}^\infty\,x^n=\frac{x}{1-x}$
and hence (i-ii)  together give:

\[
\lim_{x\to 1}\, (1-x)\cdot\sum\, (1-\frac{na_n}{c} )\cdot x^n=1
\]
Here $1-\frac{na-n}{c}\geq 0$ so Theorem 2 gives
\[
\lim_{N\to \infty}\,
\frac{1}{N} \sum_{n=1}^{n=N}\, (1-\frac{na_n}{c})=1
\]
It follows that
\[
\lim_{N\to \infty}\,\frac{1}{N} \cdot \sum_{n=1}^{n=N}\, na_n=0
\]
This means precisely that the condition in Tauber's Theorem holds and hence 
$\sum\, a_n$ converges and has series sum
equal to 0 which finishes the proof of Theorem C.











\newpage



















 
 \end{document}


