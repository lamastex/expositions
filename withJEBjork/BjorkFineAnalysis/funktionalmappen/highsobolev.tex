


\documentclass[12pt]{amsart}


\usepackage[applemac]{inputenc}



\addtolength{\hoffset}{-12mm}
\addtolength{\textwidth}{22 mm}
\addtolength{\voffset}{-10mm}
\addtolength{\textheight}{20mm}

\def\uuu{_}

\def\vvv{-}

\begin{document}




\centerline{\bf{Sobolev inequalities}}


\bigskip



\noindent

\noindent
Let $n\geq 2$ abnd $F(x)$ is a funvtion with compact support contaimed in a
ball $\{|x|\leq K\}$ fopr some $K>0$ whose partial derivtives
$\{\partial_j(F)\}$ belong to $L^1({\bf{R}}^n)$.
Let $\omega$ denote points on the unit sphere $S^{n-1}$.
For each fixed $\omega$ we set
\[ 
F_\omega(x)=
\int_0^\infty\, \frac{\partial}{\partial r}(
F(x-r\omega)\cdot b(r)\, dr
\]
A psrtial integration shows that the right hand side becomes
\[ 
F(x)+\int_0^\infty\, 
F(x-r\omega)\cdot b'(r)\, dr
\]
Now $b'(r)\neq 0$ only occurs if $2K<r<3K$ so if
$|x|\leq K$ it is clear that the last integral is zero because $F$
vanishes when
$|x|>K$.
Hence $F_\omega(x)= F(x)$ when $|x|\leq K$.
This implies that the $L^q$-norm of $F$ is majorized by
the $L^q$-norm of $F_\omega$ for every $q\geq 1$ and
every $\omega\in S^{n-1}$.
Since $L^q$-norms satisfy the triangle inequality we conclude that if 
$a(\omega)$ is some non-negaitve function on $S^{n-1}$ such that
\[
\int_{S^{n-1}}\, a(\omega)\, d\omega=1
\]
then every $L^q$-norm of $F$ is majorized
by that of
\[
F_a(x)= \int_{S^{n-1}}\int_0^\infty\, 
\int_0^\infty\,
 \frac{\partial}{\partial r}(
F(x-r\omega)\cdot b(r) a(\omega) \, dr d\omega
\]
Notice that
\[
 \frac{\partial}{\partial r}(
F(x-r\omega)=\sum_{j=1}^{j=n}\, \omega_j\cdot 
\frac{\partial}{\partial x_j}( F(x-rw)
\]
Define the functions $h_1,\ldots,h_n$ in
${\bf{R}}^n$ by
\[ 
h_j(r\omega)=\frac{b(r)\cdot \omega_j \cdot a(\omega)}{r^{n-1}}
\]
Since $dx= r^{n-1}d\omega$
holds when we pass to polar coordinates in ${\bf{R}}^n$, it follows 
from (xx) that
\[
F_a(x)=\sum\, \int_{{\bf{R}}^n}\, \partial_j(F(x-y))\cdot h_j(y)\, dy
\]
The individual $h$-functions satisfy 
\[
|h_j(r\omega)|\leq \frac{b(r)a(\omega)}{r^{n-1}}
\leq C\cdot\frac{b(r)}{r^{n-1}}
\]
where $C$ is the maximum norm of $a$.
So if $\sigma_{n-1}$ denotes the $n-1$-dimensional volume of 
$S^{n-1}$ 
and $s>1$ it follows that

\[
\int_{{\bf{R}}^n}\, |h_j(x)|^s\, dx\leq \sigma_{n-1}
\int_0^\infty\, \frac {b(r)^s}{r^{(n-1)(s-1)}}
\]
The last integral is convergent provided that
\[
(n-1)(s-1)<1\implies 1\leq s< \frac{n}{n-1}
 \]
\medskip

\noindent
{\bf{Conclusion.}}
If $p\geq 1$ and each $\partial_j(F)$ belongs to $L^p({\bf{R}}^n)$
then H�lder's inequality entails that
$F_a$ belongs to $ L^{p_*}$ when
\[
\frac{1}{p_*}>\frac{1}{p}-\frac{1}{n}\tag{*}
\]
\medskip

\noindent
{\bf{The case of equality.}}
The inequality (*) holds for every $p\geq 1$, i.r. even if $p=1$.
To get (*) in the critical case when equality holds
one must appeal to the Calderon-Zygmund inequality
and use the rather spcial properties of the $h$-functions above. 
More precisely, one
should choose $a(\omega)$ so that
not only (xx) above holds, but also 
\[ 
\int_{S^{n-1}}\, \omega_j\cdot a(\omega)\, d\omega=0
\quad\colon\quad 1\leq j\leq n
\]
The fact that (xx) entails that Theorem xx also holds in
the critical case when
$\frac{1}{p_*}= \frac{1}{p}-\frac{1}{n}$ follows by
general facts about convolution operators. More precsiely,
(xx) entails that convolution by the $h$-functions satisfy a
certain weak-type estimate in the critical case when one takes $p=1$ and after 
Thorin's interpolation theorem is applied.
We leave this to the reader who may consult text-books
for details. See
in particular [Stein-Fourier analysis] and the reader may also consult
Chapter XIV: � 4 in [Dunford-Schwarz] for a further 
discussion of Sobolev inequalities.
\bigskip

\noindent
{\bf{Passage to higher order derivatives.}}
By repeated use of Theorem XX
it follows that if $F(x)$ has bounded support and
$k\geq 2$ is an integer such that
the partial derivatives
$\frac{\partial^\alpha}{\partial x^\alpha}(F)$ belong to
$L^p$ for some $p>1$, then
\[ 
F\in L^{p_*}({\bf{R}}^n) \quad\text{where}\quad
\frac{1}{p_*}=\frac{1}{p}-\frac{k}{n}
\]
\medskip

\noindent
Finally, if it happens that $\frac{1}{p}-\frac{1}{n}<0$
one cann establish a continuity result
which goes as follows:

\medskip

\noindent
Consider a bounded open set
$\Omega$  in
${\bf{R}}^n$ with a smooth boundary, i.e. of class
$C^\infty$. Let $p\geq 1$ and $k$ is a positive integer
which yields the largest integer $m$ such that
\[ 
\frac{1}{m}<k-\frac{n}{p}
\]
Then the folloewing hold:

\medskip
\noindent
{\bf{Theorem.}}
\emph{Let $F(x)$ be a function in $\Omega$
whose partial derivatives up to order $k$
belong to $L^p(\Omega)$.
Then every derivative of order $\leq m$ exists and is even
a continuous function defined on the closure of $\Omega$.}

\newpage









\centerline{\bf{Sobolev inequalities}}


\bigskip



\noindent
{\bf{Theorem.}}
\emph{Let $p>1$ and assume that
each $\partial_j(F)$ belongs to $L^p({\bf{R}}^n)$. Then it follows that
$F\in L^{p_*}({\bf{R}}^n)$ where}
\[ 
\frac{1}{p_*}= \frac{1}{p}-\frac{1}{n}
\]
\medskip

\noindent
The proof relies upon an interesting construction.
With $K$ given as above we cohoose some $C^\infty$-function
$b(r)$ on the real line where
$b(r)=1$ if $0\leq r\leq 2K$ and zero if $r> 3K$.
Let $\omega$ denote points on the unit sphere $S^{n-1}$.
For each fixed $\omega$ we set
\[ 
F_\omega(x)=
\int_0^\infty\, \frac{\partial}{\partial r}(
F(x-r\omega)\cdot b(r)\, dr
\]
A psrtial integration shows that the right hand side becomes
\[ 
F(x)+\int_0^\infty\, 
F(x-r\omega)\cdot b'(r)\, dr
\]
Now $b'(r)\neq 0$ only occurs if $2K<r<3K$ so if
$|x|\leq K/2$ it is clear that the last integral is zero because $F$
vanishes when
$|x|>K$.
Hence $F_\omega(x)= F(x)$ when $|x|\leq K/2$.
This implies that the $L^q$-norm of $F$ is majorized by
the $L^q$-norm of $F_\omega$ for every $q\geq 1$ and
every $\omega\in S^{n-1}$.
Since $L^q$-norms satisfy the triangle inequality we conclude that if 
$a(\omega)$ is some non-negaitve function on $S^{n-1}$ such that
\[
\int_{S^{n-1}}\, a(\omega)\, d\omega=1
\]
then every $L^q$-norm of $F$ is majorized
by that of
\[
F_a(x)= \int_{S^{n-1}}\int_0^\infty\, 
\int_0^\infty\,
 \frac{\partial}{\partial r}(
F(x-r\omega)\cdot b(r) a(\omega) \, dr d\omega
\]
Next, we notice that
\[
 \frac{\partial}{\partial r}(
F(x-r\omega)=\sum_{j=1}^{j=n}\, \omega_j\cdot 
\frac{\partial}{\partial x_j}( F(x-rw)
\]
Let us now define the functions $h_1,\ldots,h_n$ in
${\bf{R}}^n$ by

\[ 
h_j(r\omega)=\frac{b(r)\cdot \omega_j \cdot a(\omega)}{r^{n-1}}
\]
Since $dx= r^{n-1}d\omega$
holds when we pass to polar coordinates in ${\bf{R}}^n$, it follows 
from (xx) that

\[
F_a(x)=\int_{{\bf{R}}^n}\, \partial_j(F(x-y))\cdot h_j(y)\, dy
\]
Next,  consider the individual $h$-functions and notice that
\[
|h_j(r\omega)|\leq \frac{b(r)a(\omega)}{r^{n-1}}
\leq C\cdot\frac{b(r)}{r^{n-1}}
\]
where $C$ is the maximum norm of $a$.
So if $\sigma_{n-1}$ denotes the $n-1$-dimensional volume of 
$S^{n-1}$ 
and $s>1$ it follows that

\[
\int_{{\bf{R}}^n}\, |h_j(x)|^s\, dx\leq \sigma_{n-1}
\int_0^\infty\, \frac {b(r)^s}{r^{(n-1)(s-1)}}
\]
The last integral is convergent provided that
\[
(n-1)(s-1)<1\implies 1\leq s< \frac{n}{n-1}
 \]
\medskip

\noindent
{\bf{Conclusion.}}
If $p\geq 1$ and each $\partial_j(F)$ belongs to $L^p({\bf{R}}^n)$
then H�lder's inequality entails that
$F_a$ belongs to $ L^{p_*}$ when
\[
\frac{1}{p_*}>\frac{1}{p}-\frac{1}{n}\tag{*}
\]
\medskip

\noindent
{\bf{The case of equality.}}
The inequality (*) holds for every $p\geq 1$, i.r. even if $p=1$.
To get (*) in the critical case when equality holds
one must appeal to the Calderon-Zygmund inequality
and use the rather spcial properties of the $h$-functions above. 
More precisely, one
should choose $a(\omega)$ so that
not only (xx) above holds, but also 
\[ 
\int_{S^{n-1}}\, \omega_j\cdot a(\omega)\, d\omega=0
\quad\colon\quad 1\leq j\leq n
\]
The fact that (xx) entails that Theorem xx also holds in
the critical case when
$\frac{1}{p_*}= \frac{1}{p}-\frac{1}{n}$ follows by
general facts about convolution operators. More precsiely,
(xx) entails that convolution by the $h$-functions satisfy a
certain weak-type estimate in the critical case when one takes $p=1$ and after 
Thorin's interpolation theorem is applied.
We leave this to the reader who may consult text-books
for details. See
in particular [Stein-Fourier analysis] and the reader may also consult
Chapter XIV: � 4 in [Dunford-Schwarz] for a further 
discussion of Sobolev inequalities.
\bigskip

\noindent
{\bf{Passage to higher order derivatives.}}
By repeated use of Theorem XX
it follows that if $F(x)$ has bounded support and
$k\geq 2$ is an integer such that
the partial derivatives
$\frac{\partial^\alpha}{\partial x^\alpha}(F)$ belong to
$L^p$ for some $p>1$, then
\[ 
F\in L^{p_*}({\bf{R}}^n) \quad\text{where}\quad
\frac{1}{p_*}=\frac{1}{p}-\frac{k}{n}
\]
\medskip

\noindent
Finally, if it happens that $\frac{1}{p}-\frac{1}{n}<0$
one cann establish a continuity result
which goes as follows:

\medskip

\noindent
Consider a bounded open set
$\Omega$  in
${\bf{R}}^n$ with a smooth boundary, i.e. of class
$C^\infty$. Let $p\geq 1$ and $k$ is a positive integer
which yields the largest integer $m$ such that
\[ 
\frac{1}{m}<k-\frac{n}{p}
\]
Then the folloewing hold:

\medskip
\noindent
{\bf{Theorem.}}
\emph{Let $F(x)$ be a function in $\Omega$
whose partial derivatives up to order $k$
belong to $L^p(\Omega)$.
Then every derivative of order $\leq m$ exists and is even
a continuous function defined on the closure of $\Omega$.}

\end{document}







