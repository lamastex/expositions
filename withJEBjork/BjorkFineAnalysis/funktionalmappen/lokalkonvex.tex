

\documentclass{amsart}

\usepackage[applemac]{inputenc}

\addtolength{\hoffset}{-12mm}
\addtolength{\textwidth}{22mm}
\addtolength{\voffset}{-10mm}
\addtolength{\textheight}{20mm}

\def\uuu{_}

\def\vvv{-}

\begin{document}






\centerline {\bf{ Fixed point theorems.}}
\medskip

\noindent
A compact topological space $S$ has the fixed point property
if every continuous map $T\colon S\to S$
has at least one fixed point.
An example is the closed unit ball in ${\bf{R}}^n$ whose  fixed-point
property is proved in � 5.xx.
More generally, consider
a locally convex vector space $X$ whose dual space is denoted by
$X^*$.
Now one equips $X$  with the weak
topology whose open sets are generated
by pairs
$x^*\in X^*$ and positive numbers
$\delta)$ of the form:
\[
B_\delta(x^*)= \{ x\in X\,\colon\, |x^*(x)|\leq \delta\}
\]
Denote by $\mathcal K(X)$   the family of
convex subsets  of $X$ which are
compact
with respect to
the $X^*$-topology. 

\medskip

\noindent
{\bf{The Schauder-Tychonoff fixed point theorem.}}\emph{
Each $K$ in $\mathcal K(X)$ has the fixed point property.}

\medskip

\noindent
The merit of this result is of course that one allows
non-linear maps. The next
result is  due to Kakutani and  goes as follows:
By a group of linear transformations
$\bf{G}$ on a real vector space $X$ we mean a family of bijective linear maps
$g\colon X\to X$ such that composed maps $g_2\circ g_1$
again belong to the group as well as  the inverse of every $g$.
\medskip

\noindent
{\bf{Kakutani's theorem.}} \emph{Let $K\in\mathcal K(X)$ be 
$\bf {G}$-invariant,  i.e. $g(K)\subset K$ hold for every
$g\in{\bf {G}}$. 
Assume in addition that the family of the restricted
${\bf{G}}$-maps to $K$ is equicontinuous.
Then there exists at least some
vector $k\in K$ such that $g(k)=k$ for every
$g\in{\bf{G}}$.}
\medskip

\noindent
{\bf{Remark.}}
The equicontinuous assumption means that to each pair
every $(x^*,\epsilon)$ with $x^*\in X^*$ and $\epsilon>0$ , there exists a finite
family $x^*_1,\ldots,x^*_M$ and some $\delta>0$
such
that the following hold: If
$p$ and $q$ is a pair of points in $K$ such that
$p-q$ belongs to $\cap\, B_\delta(x^*_\nu)$, then
\[
g(p)-g(q)\in B_\epsilon(x^*)
\] 
hold for all $g\in {\bf{G}}$.
\medskip

\noindent
{\bf{Haar measures.}}
Let $G$ be a compact topological group which
means that the  group is equipped with a Hausdorff topology where
the group operations are continuous, i.e the map from
$G\times G$ into $G$ which sends a pair of group elements $g,h$ to
the product $gh$ is continuous, and  the inverse map $g\mapsto g^{-1}$
is bi-continuous.
Now there exists  the  Banach apace
$C^0(G)$ of continuous real-valued functions on $G$.
Recall from basic measure theory that
the dual space
consists of Riesz measures. Denote by $P(G)$ the family of non-negative measures with
total mass one, i.e. probability measures on�$G$.
If $\phi\in C^0(G)$
and $g\in G$ we get the new continuous function
$S_g(\phi)$ defined by
\[ 
S_g(\phi)(h)=\phi(gh)\quad\colon h\in G
\]
Next, if  $\mu\in P(G)$ we get the new probability measure
$T_g(\mu)$ given by the linear functional
\[
\phi\mapsto \int_G\, S_g(\phi)\, d\mu
\]
In this way $G$ is identified with a group of
transformations on $P(G)$.
Next,  $P(G)$ is equipped with the weak-star topology
where  open neighborhoods of
a given $\mu\in P(G)$ consists of
finite intersections of sets
$\{\gamma\in P(G)\,\colon |\gamma(\phi)-\mu(\phi)|<\delta\}$
for pairs $\delta>0$ and $\phi\in C^0(G)$.
The  uniform continuity of every $\phi \in C^0(G)$
entails that
that the group action on
$P(G)$ is equi-continuous on $P(G)$
with respect to the weak-star topology.
As explained in � xx, Kakutani's theorem also applies 
in this situation which yields
a fixed point. Hence  there is a probability measure
$\mu$ such that
\[
\int_G\, \phi(gh)\, d\mu(h)=\int_G\,\phi(h)\, d\mu(h)\tag{*}
\] 
hold for every pair $g\in G$ and
$\phi \in C^0(G)$.
In � xx we show
that $\mu$ is uniquely determined by (*), i.e.
only one probability measure enjoys the invariance above.
Moreover,  starting with the operators
\[ 
S^*_g(\phi)(h)=\phi(hg)\quad\colon h\in G
\]
one finds a probability measure $\mu^*$ such that
\[
\int_G\,\phi(hg)\, d\mu(h)=\int_G\,\phi(h)\, d\mu(h)\tag{**}
\] 
hold for every pair $g\in G$ and
$\phi \in C^0(G)$.
In � xx we prove that $\mu=\mu^*$ which 
means that the unique Haar measure
is both left and right invariant.


\bigskip

\centerline{\bf{� 1. Convex sets and their $\rho$-functions.}}
\medskip


\noindent
Let $E$ be a real vector space.
A convex  set $U$ which contains the origin is said to be
absorbing if there for  
each vector
$x\in E$  exists some
real $s>0$ such that
$s\cdot x \in U$.
It may occur that
the whole line
${\bf{R}}x$ is contained in $U$, and then  we say that
$x$ is fully absorbed by $U$.
The convexity of $U$ entails that
the set of fully absorbed vectors is a linear subspace of $E$ which we denote by
$\mathcal L_U$.
\medskip


\noindent
{\bf{1.1 The function
$\rho_U$.}}
Let $x$ be a  non-absorbed vector $x$. Then there exists a positive real number 
\[
\mu(x)=\max\{ s\,\colon sx\in U\}
\]
If $x$ is absorbed we put $\mu(x)=+\infty$ and
for every non-zero vector $x$
we set
\[ 
\rho_U(x)=\frac{1}{\mu(x)}
\]
it is clear that if $x\in U$ then
$\mu(x)\geq 1$ and hence $\rho_U(x)\leq 1$.
Notice that we also have
\[
\rho_U(x)=\min\{s\, \colon\,x\in s^{-1}U\}
\]
\medskip



\noindent
{\bf{1.2 Exercise.}}
Show that the convexity of $U$ entails that
$\rho_U$ satisfies the triangle inequlity
\[
\rho_U(x_1+x_2)\leq \rho_U(x_1)+\rho_U(x_2)\tag{1.2.1}
\]
for all pairs of vectors in $E$.
Moreover, check also that
$\rho_U(x)=0$ if and only if
$x$ belongs to
$\mathcal L_U$ and that $\rho_U$ is positively homogeneous, i.e.
the equality below holds when $a$ is real and positive:
\[
\rho_U(ax)= a\rho_U(x)\quad\colon a>0\tag{1.2.2}
\]

\noindent
{\bf{1.3 The Hahn-Banach theorem.}}
Keeping $U$ fixed we set $\rho(x)= \rho_U(x)$.
An ${\bf{R}}$-linear map
$\lambda$ from $E$ to the 1-dimensional real line
is majorised by $\rho$ if
\[
\lambda(x)\leq \rho(x)\tag{1.3.1}
\]
hold for every vector $x$.
More generally, let $E_0$ be a subspace of $E$ and
$\lambda_0\colon E_0\to {\bf{R}}$ a linear map such that
(1.3.1) hold for vectors in $E_0$.
Then there exists a linear map
$\lambda\colon E\to {\bf{R}}$ which extends 
$\lambda_0$ and is again majorised by $\rho$.
To prove this we use Zorn's Lemma which
gives a maximal extension, i.e. we find
a subspace $E_*$ which contains $E_0$ and
a linear map $\lambda_*\colon E_*\to {\bf{R}}$
which is majorised by $\rho$ and extends $\lambda_0$.
If $E_*\neq E$ we pick a non-zero vector
$y\in E\setminus E_*$ and consider the linear space
$E_{**}= E_*+{\bf{R}}y$.
For each pair of vectors $x$ and $\xi$ in $E_*$
the triangle inequality from (1.2.1) gives
\[
\rho(\xi+x)\leq  \rho(y+x)+\rho(\xi-y)
\]
Next, put
\[
\alpha=\min_{x\in E}\, \rho(x+y)-\lambda(x)
\quad\colon \beta=\max_{\xi\in E}\,\lambda(\xi)-\rho(\xi-y)
\]
It follows that when $x$ and $\xi$ are vectors in $E$, then
\[
\alpha-\beta\geq \rho(x+y)-\lambda(x)-\lambda(\xi)+\rho(\xi-y)
=\rho(x+y)+\rho(\xi-y)- \lambda(x+\xi)\tag{i}
\]
Next,
the triangle inequality applied to the vectors $x-y$ and $\xi+y$ gives
\[
\rho(x+\xi)=\rho((x+y)+(\xi-y))\leq \rho(x+y)+\rho(\xi-y)\tag{ii}
\]
Moreover, since
$\rho$ majorises $\lambda$ on $E_*$ we have
\[
 \lambda(x+\xi)\leq \rho(x+\xi)\tag{iii}
\]
 it follows from (i-iii) that
\[
 \alpha\geq \beta\tag{iv}
\]
Let us then take a real number $a$ such that
$\beta\leq a\leq \alpha$ and extend $\lambda$ to $E_{**}$ by
\[
\lambda(x+sy)=\lambda(x)+ sa
\]
where $x\in E_*$ and $s$ are real numbers.
if $s>0$ we have
\[
\lambda(x+sy)= s\lambda(s^{-1}x+y)=s((\lambda(s^{-1}x+a)\leq
s((\lambda(s^{-1}x+\alpha)\leq s\rho(s^{-1}x+y)
\]
where the last inequality follos from the defintion of $\alpha$ in
(x) applied to the vector $s^{-1}x\in E_*$.
Finally, since $\rho$ is positively homogeneous the last term is
equal to $\rho(x+sy)$. Hence we have proved the inequality
\[
\lambda(x+sy)\leq \rho(x+sy)
\] 
when $s>0$.
Next, since $a\geq\beta$, the reader can check by a similar method as above that
\[
\lambda(x+sy)\leq \rho(x+sy)
\] 
hold when $s<0$.
This  proves that the choice of $a$ above gives an extension of $\lambda$ to
$E_{**}$ which again is majorised by
$\rho$ and via Zorn's Lemma we get the Hahn-Banach theorem.




\bigskip



\centerline {\bf{2 Locally convex topologies.}}
\bigskip


\noindent
Denote by $\mathcal C_E$ the family of convex sets $U$ as in � 1.
Let
$\mathfrak{U}= \{U_\alpha\}$ be a family in  $\mathcal C_E$ such that
\[
\bigcap\,\mathcal L_{U_\alpha}=\{0\}
\]
i.e. the intersction is reduced to the origin.
Now there exists a topology on $E$
where a basic for open neighborhoods of the origin
consists of sets:
\[
\cap\, \{\rho_{U_{\alpha_i}}(x)<\epsilon\}\tag{1}
\] 
where $\epsilon>0$ and $\{\alpha_1,\ldots,\alpha_k\}$ is a finite set
of indices defining the  $U$-family.
If $x_0$ is a vector in $XS$, then
a basis for its open neihghborhoods
are given by
sets of the for $x_0+U$ where $U$ is a set from (1).
In general, a subset $\Omega$ in $E$ is open if
there to eah $x_0\in\Omega$ exists some
$U$ from (1) such that
$x_0+U\subset \Omega$.
it is clear that this gives a topology and (1) entaills that it is separated, i.e. a Huasdorff topology on
$E$.
Notice also that eqch set in 82) is convex. One therefore
refers to a locally con vex
topology on $E$.
\medskip
 
 \noindent
 {\bf{2.1  Remark.}}
 The locally convex topology above depends upon the family
 $\mathfrak{U}$-toplogy. Its topology  is not changed if we enlarge 
 the family to consist of all finite intersection of its
 convex subsets. When this has been done
 we notice that if $U_1,\ldots, U_n$ is a finite family in
 $\mathfrak{U}$ then the norm defined by
 $U=U_1|\cap\ldots\,\cap U_n$ is stronger than
 the individual $\rho_{U_i}$-norms. Hence
 a fundamental system of neighborhoods 
 consists of single $\rho$-balls:
  \[
 \Omega=\{\rho_U<\epsilon \}    \quad\colon U\in\mathfrak {U}
 \]
 
 \medskip
 
 \noindent
 {\bf{2.2 The dual space $E^*$}}.
Let $E$ be equipped with a locally convex $\mathfrak{U}$-topology.
As above $\mathfrak{U}$ has been enlarged so that
the balls above  give a basis for neighborhoods of the origin.
A linear functional  $\phi$ on $E$
is $\mathfrak{U}$-continuous if  there
exists some
$U\in\mathfrak{U}$ and a
constant $C$ such that
\[
|\phi(x)||\leq C\cdot \rho_U(x)
\]
\medskip


\noindent
{\bf{2.3 Closed half-spaces.}}
To each pair 
$\phi\in E^*$ and a real number $a$ one assigns the
closed half-space
\[
H=\{x\in X\,\colon\, \phi(x)\leq a\}
\]
Notice that $a<0$ can occur in which case
$H$ does not contain the origin.
\medskip

\noindent
{\bf{2.4 The separation  theorem.}}
\emph{Each closed convex set $K$ in $E$ is the intersection of closed half-spaces.}



\medskip
\noindent
{\bf{2.5 Exercise.}}
Show (2.4)  using the Hahn-Banach theorem.
\medskip


\noindent
Next, let $K_1$ and $K_2$ be a pair of closed and disjoint convex sets.
Then they can be spearated by a hyperplane. More precisley, there
exists some $\phi\in E^*$ and a positive number $\delta$ such that
\[
\max_{x\in K_1}\, \phi(x)+\delta\leq
\min_{x\in K_2}\, \phi(x)\tag{2.6}
\]
Again we leave the proof as an exercise to the reader.

\bigskip


\centerline {\bf{3. Support functions of convex sets.}}
\medskip


\noindent
Let $E$ be a locally convex space as above.
Vectors in $E$ are denoted by  $x$, while $y$ denote  vectors in $E^*$.
To each closed and convex subset $K$ of $E$
we define a function $\mathcal H_K$ on the dual $E^*$ by:
\[ 
\mathcal H_K(y)=\sup_{x\in K}\,y(x)
\]


\noindent
Notice that  $\mathcal H_K$
take values in $(-\infty,+\infty]$, i.e. it may be $+\infty$ 
for some vectors $y\in E^*$.
For example, let $K=\{{\bf{R}}^+x_0$ be a half-line.
Then $\mathcal H_K(y)=+\infty$ when
$y(x_0)>0$
and otherwise zero. So here the range consisists of 0 and $+\infty$.
It is clear that
\[
\mathcal H_K(sy)= s\mathcal H_K(y)
\] 
hold when $s$ is a positive real number, i.e
$\mathcal H_K$ is positively homogeneous.



\medskip

\noindent
{\bf{3.1 Exercise.}}
Show that the convexity of $K$ entails that
\[
\mathcal H_K(y_1+y_2)\leq 
\mathcal H_K(y_1)+
\mathcal H_K(y_2)
\]
for each pair of vectors in $E^*$.
\medskip




\noindent
{\bf{3.2 Upper semi-continuity.}}
For each fixed vector $x\in E$
the function 
\[ 
y\mapsto y(x)
\]
is weak-star continuous on $E^*$.
Since
the supremum function attached to
an arbitrary family of weak-star continuoes functions is upper
semi-continuous, it follows that $\mathcal H_K$ is upper semi-contiuous.
\medskip

\noindent
{\bf{3.4 Exercise.}}
Let $K$ and $K_1$ be a pair of closed convex sets such that
$\mathcal H_K=\mathcal H_{K_1}$. Show that this entails that $K=K_1$.
The hint is to use the separation theorem.
\medskip

\noindent
{\bf{3.5 The class $\mathcal S(E)$}}.
It consids of all  
all upper semi-continuous functions
$G$ on $E^*$ with values in $(-\infty,+\infty]$
which satisfy (x) and (xx).
The next result 
was proved by   H�rmander 
in the article
\emph{Sur la fonction d'appui des ensembles convexes dans un espaces
localementt convexe} [Arkiv f�r mat. Vol 3: 1954].

\medskip

\noindent
{\bf{3.6 Theorem.}}
\emph{Each $G\in\mathcal S(E)$ is of the form
$\mathcal H_K$ for a unique closed convex subset $K$ in $E$.}
\medskip

\noindent
{\bf{Remark.}}
As pointed out by H�rmander in [ibid] this result
is closely related to  earlier studies by
Fenchel in the article \emph{On conjugate convex functions}
Canadian Journ. of math. Vol 1 p. 73-77) 
where  Legendre transforms are
studied in infinite dimensional topological vector spaces.
The novely in Theorem 3.3 is the generality and we
remark that various separation theorems 
in text-books dealing with notions of convexity
are easy consequences of Theorem 5.C.2.


\bigskip



\noindent{\emph{Proof of Theorem 3.6 }}
Put $F=E\oplus{\bf{R}}$ which is a new vector space where
the 1-dimensional real line is added. It dual space
$F^*=E^*\oplus{\bf{R}}$.
We are given $G\in\mathcal S(E)$ and 
put
\[
G_*=\{(y,\eta)\in E^*\oplus {\bf{R}}\quad\colon\, G(y)\leq \eta\}\tag{i}
\]
Condition in (*) entails that $G_*$ is a convex cone
in $F^*$ and the semi-continuous hypothesis on $G$
implies that  $G_*$ is closed with respect to the weak-star toplogy on $F^*$.
Next, in $F$ we define the set
\[ 
G_{**}=\{(x,t)\in E\oplus{\bf{R}}^+\,\colon\, y(x)\leq \eta t\,\colon\, (y,\eta)\in G_*\}\tag{ii}
\]
This gives a set
$\widehat C$ in 
$F^*$ which consists of vectors  $(y,\eta)$ such that
\[
\max_{(x,t)\in 
G_{**}}\,y(x)- \eta t\leq 0
\]
It is clear that
$G_*\subset \widehat C$. Now we prove the equality

\[
G_*= \widehat C\tag{*}
\]
To get (*) we use Theorem 2.4.
Namely, since
the two sets in (*)  are weak-star closed a strict inequality
gives a separating
vector $(x_*,t_*)\in E$, i.e.  there exists $(y_*,\eta_*)\in \widehat C$
and a real number $\alpha$ such that 
\[
y_*(x_*)-\eta _* t_*>\alpha
\quad\text{and}\quad (y,\eta)\in D_K\implies y(x_*)-\eta t_*\leq \alpha\tag{iv}
\]
Since $G_*$ contains $(0,0$ we  have $\alpha\leq 0$.
and since it also is a cone the last implication  gives
$(x_*,t_*)\in G_{**}$. Now
the construction of $\widehat{C}$ in (iii) contradicts the strict inequality in
the left hand side of (iv).
Hence there cannot exist
a separating vector and  (*) follows.
\medskip

\noindent
Next, in  $E$ we consider  the convex set
\[
K=\{ x\,\colon (x,1)\in G_{**}\}
\]
Using (*) the reader can check that
\[
\mathcal H_K(y)=G(y)
\]
 for all $y\in E^*$ which proves that $G$ has the requested form.
 The uniqueness of $K$ follows from Exercise 3.4.
 
 

\bigskip

\noindent
{\bf{3.7  The case of normed spaces.}}
If $X$ is a normed vector space Theorem 3.6 
leads to a certain isomorphism of two families.
Denote by $\mathcal K$ the family of all convex
subsets of $E$ which are closed with respect to the norm topology.
A topology on $\mathcal K$ is defined
when we for each $K_0\in\mathcal K$ and $\epsilon>0$
declare an open neighborhod
\[ 
U_\epsilon(K_0)=\{ K\in\mathcal K\,\colon\,
\text{dist}(K,K_0)<\epsilon\}
\] 
where the norm defines the distance between
$K$ and $K_0$ in the usual way.
Denote by $\mathfrak{H}$ the family of all
functions $G$ on $E^*$ which satisfy (*) in 5.B.1 and are continuous
with respect to the norm topology on $E^*$.
A subset $M$ of $\mathfrak{H}$ is equi-continuous if there
to
each $\epsilon>0$ exists $\delta>0$ such that
\[
||y_2-y_1||<\delta \implies ||G(y_2)-G(y_1)||<\epsilon
\] 
for every $G\in M$ and 
all pairs $y_1,y_2$ in $E^*$.
The topology on $\mathfrak{H}$ is  defined
by  uniform convergence on equi-continuous subsets.

\medskip




\noindent
{\bf{3.8 Theorem.}}
\emph{If $E$ is a normed vector space
the set-theoretic bijective map $K\to \mathcal H_K$ is
a homeomorphism when
$\mathcal K$ and
$\mathfrak {H}$ are equipped with the described topologies.}
\medskip

\noindent
{\bf{3.9 Exercise.}} Deduce this  result 
from     Theorem 3.6
\newpage


\centerline
{\bf{4. The Krein-Smulian theorem.}}
\bigskip


\noindent
Let $X$ be a Banach space and $X^*$ its dual.
The weak-star topology on $X^*$ was defined in � 5.4.
We have also the bounded weak-star topology desrcibed in � xx from
the introduction.
So now we have a pair of  locally convex spaces 
$X^*_w$ and $X^*_{bw}$.
Open sets in the weak-star topology are
by definition also open in
the bounded weak-star topology, i.e. the latter topology 
contains more open sets and is therefore stronger than the ordinary weak-star topology.
Hence  there is a natural inclusion of   dual spaces:
\[
(X_w^*)^*\subset
(X_{bw}^*)^*\tag{*}
\]
The Krein-Smulian theoren assers that eqaulity holds in (*).
To prove this we proceed as follows.
For  each finite
subset $A$ of  $X$ we put
\[ 
\widehat{A}=\{x^*\,\colon\, \max_{x\in A}\,|x^*(x)|\leq 1\}
\]
Let $U$ be an open set in $X^*_{bw}$
which contains the origin
and $S^*$ is the closed unit ball in
$X^*$.
The construction of the bounded weak-star topology gives
a finite set $A_1$ in $X$ such that
\[
S^*\,\cap \widehat{A}^0\subset U\tag{i}
\]
Next, let $n\geq 1$ and suppose we have constructed
a finite set $A_n$ where
\[
nS^*\,\cap \,\widehat{A_n}\subset U\tag{ii}
\]
To each finite set $B$ of vectors 
in $X$ with norm $\leq n^{-1}$
we notice that   
\[
 \widehat{A_n\cup\, B}\subset \widehat{A_n}\tag{iii}
\]
Put
\[
 F(B)=(n+1)S^*\cap \,\widehat{A_n\cup\, B} \,\cap (X^*\setminus U)
\]
It is clear that $F(B)$  is weak-star closed
for
every finite set $B$ as above. If these sets are non-empty for all $B$,
it follows from 
the weak-star compactness of $(n+1)S^*$
 that the whole intersection is non-empty. So we find a vector
 \[
 x^*\in \bigcap_B\, F(B)
 \]
Notice that $F(B)\subset \widehat{B}$ for every finite set $B$ as above
which means that  
$|x^*(x)||leq 1$ for every  vector $x$ in 
in $X$ of norm $\leq n^{-1}$. 
Hence the norm
\[
||x^*||\leq n
\]
 But then (iii) gives the inclusion
\[
 x^*\in nS^*\,\widehat{ A_n}\bigr)\, \cap (X\setminus U)\tag{iv}
\]
This contradicts (ii) and hence we have proved that there exists a finite set
$B$ of vectors with norm
$\leq n^{-1}$ such that 
$F(B)=\emptyset$.
\medskip

\noindent
From the above it is clear that an induction over $n$ gives
a sequence of sets $\{A_n\}$ such that  (ii) hold
for each $n$ and
\[ 
A_{n+1}=A_n\,\cup B_n\tag{v}
\]
where
$B_n$ is a finite set of vectors of norm
$\leq n^{-1}$.
\bigskip

\centerline{\emph{Proof of the Krein-Smulian theorem.}}

\bigskip

\noindent
Let $\theta$ be a linear functional on $X^*$ which is
continuous with respect to the bounded weak-star topology.
This gives an open neighborhood
$U$ in
$X^*_{bw}$ such that
\[
|\theta(x^*)\leq 1\quad\colon\,  x^*\in U\tag{i}
\]
To the set $U$
we find a sequence $\{A_n\}$ as above.
Let us enumerate the vectors in this sequence of finite sets by
$x_1,x_2,\ldots$, i.e.
start with the finite string of vectors in $A_1$, and so on.
By the inductive construction of the $A$-sets we have
$||x_n||\to 0$ as $n\to \infty$.
If $x^*$ is a vector in $X^*$ we 
associate the complex sequence
\[
\ell(x^*)= \{x^*(x_n)\}
\]
which tends to zero since
$||x_n||\to 0$ as $n\to \infty$.
Then
\[ 
x^*\mapsto \ell(x^*)
\]
is a linear map from $X^*$ into the 
Banach space ${\bf{c}}_0$.
If
\[
\max_n\, |x^*(x_n)|\leq 1
\]
we have by definition $x^*\in A_n^0$ for each $n$.
Choose  a positive integer $N$ so  that $||x^*||\leq n$. Thus entails that
\[
x^*\in NS^*\cap A_N^0
\]
From (ii) during the inductive construction of the $A$-.sets,
the last set is contained in $U$. Hence
$x^*\in U$
which by (i) gives
$\theta(x^*)|\leq 1$.
We conclude that $\theta$ yields
a linear functional on
on the image space of the $\rho$-map  with norm one at most.
The 
Hahn-Banach theoren gives
$\lambda\in{\bf{c}}_0^*$
of norm one at most such that
\[
\theta(x^*)= \lambda(\ell(x^*))
\]
Next,  by a wellknown result due to
Banach the dual of ${\bf{c}}_0$ is $\ell^1$. Hence there exists
a sequence $\{\alpha_n\}$ in $\ell^1$ such that
\[
\theta(x^*)= \sum\,\alpha_n\cdot x^*(x_n)
\]
In $X$ we find the vector $x=\sum\, \alpha_n\cdot x_n$
and conclude that
$\theta=\widehat{x}$ which proves the Krein-Smulian theorem.




\newpage



\centerline{\bf{5. Fixed point theorems.}}
\bigskip


\medskip
\noindent
A topological space $S$ has the fixed-point property if every
continuous map $f\colon S\to S$ has at least one fixed point.
\medskip


\noindent
{\bf{5.1 Theorem.}}
\emph{The closed unit ball in ${\bf{R}}^n$ has the fixed point property
for every $n\geq 2$.}


\medskip


\noindent
\emph{Proof.}
By Weierstrass approximation theorem
every continuous map from $B$ into itself can be approximated unifomly by
a $C^\infty$-map. Together with the compactness of $B$ the reader
should conclude that it suffices to prove every $C^\infty$-map, 
$\phi\colon B\to B$  has at least one fixed point.
We  argue by  contradiction, i.e suppose that
$\phi(x)\neq x$ for all $x\in B$.
Each fixed  $x\in B$ gives a quadratic equation  of  the variable $a$
\[
1= |x+a(x-\phi(x)|^2=|x|^2+2a(1-\langle x,\phi(x)\rangle )+a^2|x-\phi(x)|^2\tag{i}
\]


\noindent
{\bf{Exercise 1.}}
Use that $\phi(x)\neq x$,
to show that
(i) has two simple real roots for
each $x\in S$, and if $a(x)$ is the larger then
the function $x\mapsto a(x)$
belongs to $C^\infty(B)$. Moreover
\[
a(x)=0\quad\colon x\in S\tag{E.1}
\]
Next,  for each real number $t$ we set
\[
f(x,t)= x+ta(x)(x-\phi(x))
\]
This is a vector-valued function of the $n+1$ variables $t,x_1,\dots,x_n$
where $x$ varies in $B$.
Put
\[
g_i(x)=a(x)(x_i-\phi_i(x)\implies  f_i)x,t)= x_i+tg_i(x)
\]
Taking partial derivatives
with respect to $x$ we get
\[
\frac{\partial f_i}{\partial x_k}= e_{ik}+t\frac{\partial g_i}{\partial x_k}\tag{ii}
\]
where $e_{ii}=1$ and $e_{ik}=0$ if $i\neq k$.
Let $D(x;t)$ be the determinant of the $n\times n$-matrix
whose elements are the partial derivatives in (ii) and put
\[ 
J(t)=\int_B\, D(x;t)\, dx\tag{iii}
\]
When $t=0$ we notice that the $n\times n$-matrix above
is the identy matrix and hence
$D(x;0)$ has constant value one so that  $J(0)$ is the volume of $B$.
Next, (i) entails that 
$x\mapsto f(x;1)$ satisfies the functional equation
\[
|f(x;1)|^2=1\implies
\sum_{i=1}^{i=n}\, f_i(x,1)\cdot
\partial f_i(x,1)/\partial x_k(x,1)=0
\] 
for every $1\leq k\leq n$.
Since the $n$-vector $f(x,1)\neq 0$
it means that
the columns of the matrix are linearly independent for every $x\in B$ and hence
$D(x;1)=0$ holds in $B$ so that   
$J(1)=0$.
The  contradiction follows if we show that 
$t\mapsto J(t)$
is  a constant function of $t$. 
\medskip

\noindent
{\bf{2. Exercise.}}
Use Leibniz's rule and that determinants of matrices with
two equal columns are zero to conclude that
\[
\frac{d}{dt}(D(x;t)= \sum\sum\, (-1)^{j+k}\cdot
\frac{\partial g_i}{\partial x_k}\tag{E.2}
\] 
where the double sum extends over all pairs $\leq j,k\leq n$.
Next, Stokes Theorem gives
\[
\int_B\, \frac{\partial g_i}{\partial x_k}\, dx=
\int_S\, g_i\cdot {\bf{n}}_k\, d\omega\quad\colon\quad 1\leq i,k\leq n\tag{iv}
\]
where  $\omega$ is the area measure on $S$.
From (E.1)  we have $g_i=0$ on $S$ for each $i$.
Hence (E.2)   gives
\[
\frac{dJ}{dt}=\int_B\, \frac{d}{dt}(D(x;t)\, dx=0
\]
So $t\mapsto J(t)$ is constant which is impossible  because
$J(0)=1$ and $J(1)=0$ which  finishes  the proof.



\newpage



\bigskip


\noindent
{\bf{5.2 The Hilbert cube $\mathcal H_\square$}}.
It is the closed subset of
the Hilbert space $\ell^2$ 
which consists of vectors $x=(x_1,x_2,\ldots)$ such that
$|x_k|\leq 1/k$ for each $k$.
\medskip



\noindent
{\bf{5.3 Proposition.}}
\emph{Every closed and convex subset of
$\mathcal C$ has the fixed point property.}
\medskip

\noindent
{\bf{5.4 Exercise.}} Deduce this result from Theorem 5.1.

\bigskip

\noindent
Next, let $X$ be a locally convex vector space and $X^*$ its dual.
Denote by  $\mathcal K(X)$ the family of convex subsets 
which are compact with respect to the  weak topology on $X$. 
Let $K\in \mathcal K(X)$ and $T\colon K\to K$
a continuous map with respect to the weak topology.
For each fixed  $f\in X^*$,  it follows from our assumptions that 
the complex-valued function on $K$ defined by
\[
p\mapsto f(T(p))
\]
is uniformly contnuus with respect to the weak topology.
So for each positive integer $n$ there exists a finite set
$G_n=(x^*_1,\ldots,x^*_N)$ and some $\delta>0$
such that the following implication holds for each pair of points
$p,q$  in $K$:
\[
p-q\in\cap \, B_\delta(x^*_\nu)\implies 
|f(T(p))-f(T(q))|\leq n^{-1}\tag{i}
\]
We can attain this for each positive integer  $n$ and get a denumerable set
\[
G=\,\cup \, G_n
\]
From (i)  it is clear that if
$p,q$ is a pair in $K$ and $g(p)=g(q)$ hold for every $g\in G$, then
$x^*(T(p))=x^*(T(q)$.
We refer to $G$ as a determining set for the map $T$.
In a similar way we find a denumerable
determining set
$G^{1)}$
for $g_1$, By a standard diagonal argument the
reader may verify the following:
\medskip


\noindent
{\bf{5.5 Proposition.}}
\emph{There exists a denumerable subset
$G$ in $X^*$ which contains $f$ and  is self-determining in the sense that
it determines each of its vectors as above}.

\bigskip


\noindent
{\bf{5.6 An embedding into the Hilbert cube.}}
During the construction of the finite $G_m$-sets which
give (i), we can choose small $\delta$-numbers and
take $\{x^*_\nu\}$  such that the maximum values
\[
\max_{p\in K}\, |x^*_\nu(p)|
\] 
are small.
From this observation the reader should confirm that
in Proposition 5.5 we can construct 
the sequence
 $G=(g_1,g_2,\ldots)$  in such a way that
 \[
 \max_{p\in K}\, |g_n(p)|\leq n^{-1}
\]
hold for every $n$.
Hence each $p\in K$ gives the vector
$\xi(p)=(g_1(p),g_2(p),\ldots)$ in the Hilbert cube
and now 
\[
K_*=\{\xi(p)\,\colon\, p\in K\}
\]
yields a convex subset of $\mathcal C$.
Since $G$ is self-determininig we have
$T(p)=T(q)$ whenever $\xi(p)=\xi(q)$. Hence
there exists a  map from
$K_*$ into itself defined by
\[
T_*(\xi(p))= \xi(T(p)\tag{4.1}
\]
\medskip

\noindent
{\bf{5.7 Exercise.}}
Use the compact property of $K$ to show that
$K_*$ is closed in the Hibert cube and that
$T_*$ is a continuous map with respect to the induced 
strong norm topology on $K_*$ derived from the
complete norm on $\ell^2$.
\medskip

\noindent
{\bf{5.8 Consequence.}}
Suppose from the start that
we are given a pair of points $p_1,p_2$  in $K$ and some
$f\in X^*$ where $f(p_1)\neq f(p_2)$.
From the above $f$ appears in the $G$-sequence and put
\[
K_0=\{p\in K\colon\, \xi(p)=\xi (p_1)\}
\]
Then $K_0$ is a 
convex subset of $K$, and since
$f$ appears in the $G$-sequence it follows that
$p_2$ does not belong to $K_0$.
Moreover, since $G$ is self-determining with respect to $T$ it 
is clear that 
\[
T(K_0)\subset K_0
\]
Hence we have proved:

\medskip

\noindent
{\bf{5.9 Proposition.}} \emph{For each pair
$K$ and $T$  as above where $K$ is not reduced to a single point, there
exists a proper and $X^*$-closed convex subset $K_0$ of $K$ such that
$T(K_0)\subset K_0$.}
\bigskip


\centerline {\bf{5.10 Proof of the Schauder-Tychonoff theorem.}}
\bigskip

\noindent
Let $T\colon K\to K$
be a continuous map where $K$ belongs to $\mathcal K(X)$.
Consider the family  $\mathcal F$
of all closed and convex subsets
which are $T$-invariant. It is clear that
intersections of such sets enjoy the same property. So we find
the minimal set
\[
K_*= \bigcap\, K_0
\]
given by the intersection of all sets $K_0$
in
$\mathcal F$.
If $K_*$ is not reduced to a single point then Proposition  5.9
gives a proper closed subset which again belongs to
$\mathcal F$. This is contradicts the minimal property.
Hence $K_*=\{p\}$ is a singleton set and $p$ 
gives the requested fixed point for $T$.
\bigskip


\centerline{\bf{5.11 Proof of Kakutani's theorem.}}
\bigskip

\noindent
With the notations from the introduction we are given a group
${\bf{G}}$ where each element $g$ preserves the convex set
$K$ in $\mathcal K(X)$.
Zorn's lemma gives a minimal 
closed and convex subset $K_*$ of $K$ which again is invariant under the group.
Kakutani's theorem follows if $K_*$ is a singleton set.
To prove that this indeed is the case we  argue by contradiction.
For  $K_*$ is not a singleton set then
\[
K_*-K_*=\{p-q\colon p,q\in K_*\}
\] 
contains points outside the origin, and we find  
a convex open neighborhood $V$ of the origin such that
\[
(K_*-K_*)\setminus\overline{V}\neq\emptyset\tag{i}
\]
Since
${\bf{G}}$ is equicontinuous on $K$ and hence also on $K_*$ there
exists an open convex neighborhood $U$ of the origin such that
whenever $k_1,k_2$ is a pair in $K_*$
such that $k_1-k_2\in U$, then  the  orbit
\[
{\bf{G}}(k_1-k_2)\subset V\tag{ii}
\]
Set
\[
U^*=\text{convex hull of }\,\,{\bf{G}}(U)
\]
Since the ${\bf{G}}$-maps are linear, the set 
$U^*$ is invariant and  continuity entails  the equality
\[
{\bf{G}}\overline{U^*})=\overline{U^*}\tag{iii}
\]
We find the unique positive number $\delta$ such that
the following hold for every $\epsilon>0$:
\[
K_*-K_*\subset (1+\epsilon)\cdot U^*
\quad\colon\, (K_*-K_*)\setminus (1-\epsilon)\cdot \delta\cdot\overline{U^*}\neq\emptyset\tag{iv}
\]
Next,  $\{k+\frac{\delta}{2}\cdot U\colon\, k\in K_*\}$ 
is an open covering of the compact set $K_*$.
Hence Heine-Borel's Lemma gives 
 a finite set 
$k_1,\ldots,k_n$ in $K_*$ such that
\[
K_*\subset\ \cup\, (k_\nu+\frac{\delta}{2}\cdot U)\tag{v}
\]
Put
\[
K_{**}= K_*\cap\, \bigcup_{k\in K_*}
(k+(1-1/4n)\delta\cdot \overline{U})\tag{vi}
\]
Since
 $\overline{U}$ is 
${\bf{G}}$-invariant and the intersection above is taken over all $k$ in the
invariant  set $K_*$, we see that 
$K_{**}$ is  a closed convex  and ${\bf{G}}$-invariant set.
The requested contradiction follows if we prove that
$K_{**}$ is non-empty
and
is strictly contained in $K_*$.
To get the  strict inclusion follows  we take some
$0< \epsilon<1/4n$. Then   (iv)  gives
a pair $k_1,k_2$ in $K_*$ such that
$k_1-k_2$ does not belong to
$(1-\epsilon) \delta\cdot\overline{U^*}$.
At the same time the inclusion $k_1\in K_{**}$ entails that
\[
k_1\in (k_2+(1-1/4n)\delta\cdot \overline{U})
\implies k_1-k_2\in  (1-1/4n)\delta\cdot \overline{U}\tag{v}
\]
which cannot hold since   $1-1/4n<1-\epsilon$.
The proof of Kakutani's theorem is therefore finished if we have shown that
\[
p\in K_{**}\tag{vi}
\]
To see this
we take an arbitrary $k\in K_*$.
From (v) we find some $1\leq i\leq n$ such that
\[
k_i-k\in \frac{\delta}{2}\cdot U\tag{vii}
\]
Without loss of generality we can assume that $i=1$
and get  a vector $u\in U$ such that
\[
k_1= k+\frac{\delta}{2}\cdot u\tag {viii}
\]
It follows that
\[
p=\frac{k_1+\ldots+k_n}{n}=k+\frac{\delta}{2n}\cdot u+
\sum_{i=2}^{i=n}\, \frac{1}{n}(k_i-k)\tag {ix}
\]
Next, for  each $\epsilon>0$ the left hand inclusion in (iv) and the 
convexity of $U$ give
\[
\sum_{i=2}^{i=n}\, \frac{1}{n}(k_i-k)\subset
\frac{n-1}{n}(1+\epsilon)\cdot \delta \cdot U\tag{x}
\]
It follows   that
\[
\frac{\delta}{2n}\cdot u+
\sum_{i=2}^{i=n}\, \frac{1}{n}(k_i-k)\in
(\frac{n-1}{n}(1+\epsilon)\delta +\frac{\delta}{2n})\cdot U\tag{xi}
\]
Above we can choose $\epsilon $ so small that
\[
\frac{n-1}{n}(1+\epsilon)+\frac{1}{2n})<1-1/4n
\]
and then we see that
\[
p\in k+(1-1/4n)\delta\cdot \overline{U}
\]
Since $k\in K_*$ was arbitrary the requested inclusion
$p\in K_{**}$ follows.



\end{document}






