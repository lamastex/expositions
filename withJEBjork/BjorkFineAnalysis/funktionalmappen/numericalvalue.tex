

\documentclass[12pt]{amsart}


\usepackage[applemac]{inputenc}



\addtolength{\hoffset}{-12mm}
\addtolength{\textwidth}{22 mm}
\addtolength{\voffset}{-10mm}
\addtolength{\textheight}{20mm}

\def\uuu{_}

\def\vvv{-}

\begin{document}



\centerline{\bf{Contractions}}


\bigskip


\noindent
A bounded linear operator $A$ on the Hilbert space $\mathcal H$ is called a contraction if
its operator norm is $\leq 1$, i.e.
if
\[
 ||Ax||\leq ||x||\quad\colon\quad x\in\mathcal H\tag{1}
\] 
This means that if $x\in\mathcal H$ then

\[
\langle Ax,Ax\rangle \leq ||x||^2=\langle x,x\rangle\tag{2}
\]
Let $E$ be the identity operator on $\mathcal H$.
Now $E-A^*A$ is a self-adjoint operator and we see that (2) entails
\[
\langle x-A^*Ax,x\rangle=||x||^2-||Ax||^2\geq 0
\]
So this self-adjoint operator is non-negative and by � xx it has a square root, i.e.
there exists the self-adjoint operator
\[ 
B_1=\sqrt{E-A^*A}
\]
Next, recall that the operator norms of $A$ and its adjoint $A^*$ are the same so
$A^*$ is also a contraction and the bidai�ity formula $A^{**}=A$
entails that we get another self-adjoint operator
\[
B_2=\sqrt{E-AA^*}
\]
Since we have not assumed that $AA^*=A^*A$ the two self-adjoint operators
$B_1,B_2$ need not be equal.
However, the follwonig hold:

\medskip

\noindent
{\bf{Propostion.}}
\emph{One has the equalities}

\[ 
AB_1=B_2A\quad\text{and}\quad A^*B_2= B_1A^*
\]
\medskip


\noindent
\emph{Proof.}
If $n$ is a positive integer we notice that
\[
A(A^*A)^n=(AA^*)^nA\tag{i}
\]
Now $A^*A$ is a self-adjoint operator whose compact spectrum is confined to
the closed unit interval $[0,1]$.
if $f\in C^0[0,1]$
is a real-valued continuous function it can be approximated uniformly
by a sequence of polynomials $\{p_n\}$
and the operational calculus from � XX yields an operator
$f(A^*A)$ such that the perator norms tend to zero, i.e.
\[
\lim\, ||p_n(A^*A)-f(A^*A)||=0
\]
Since the spectrum of $AA^*$ also is confined to $[0,1]$
it follows that when ewe take the same polynomial sequence
$\{p_n\}$ then
we get an operator $f(AA^*)$ and

\[
\lim\, ||p_n(AA^*)-f(AA^*)||=0
\]
Now (i) and the teo limit formulas above entail that

\[
Af(A^*A)= f(AA^*)A\tag{ii}
\]
In particular we can use the continuous function
$f(t)= \sqrt{1-t}$
and then Proposition XX follows.
\medskip

\noindent{\bf{The unitary operator $U_A$}}.
On the Hilbert space $\mathcal H\times \mathcal H$
we define a linear operator $U_A$ represented by the block matrix

\[
U_A=
\begin{pmatrix} A& B_2\\
B_1&-A^*
\end{pmatrix}
\]
\medskip

\noindent
{\bf{Proposition.}}
\emph{$U_A$ is a unitary operator on
 $\mathcal H\times \mathcal H$}.
 
 \medskip
 
 \noindent
 \emph{Proof.}
 For a pair of vectors $x,y$ in $\mathcal H$
 we must prove the equality
\[
|| U_A(x\oplus y)||^2=||x||^2+||y||^2\tag{i}
\]
To prove this we first notice that for every vector $h\in\mathcal H$ 
the self-adjointness of $B_1$ gives

\[
||B_1h||^2=\langle B_1h,B_1h\rangle=
\langle B_1^2h,h\rangle=\langle h-A^*Ah,h\rangle=
||h||^2-||Ah||^2\tag{ii}
\]
Above the last equality holds since
$\langle A^*Ah,h\rangle =\langle Ah,A^{**}h\rangle =||Ah||^2$
where we used the biduality formula $A=A^{**}$.
In the same way we find that

\[
||B_2h||^2=||h||^2-||A^*h||^2\tag{iii}
\]
\medskip

\noindent
Next, by the construction of $U_A$ the left hand side in (i) becomes

\[
||Ax+B_2y||^2+||B_1x-A^*y||^2\tag{iv}
\]
Using (iii) the first term in (iv)becomes
\[
||Ax+B_2y||^2=||Ax||^2+||y||^2-||A^*y||^2+
\langle Ax,B_2y\rangle+ \langle B_2 y,Ax\rangle
\]
By (ii) the second term becomes
\[
||B_1x-A^*y||^2=||x||^2-||Ax||^2+||A^*y||^2-
\langle B_1x,A^*y\rangle- \langle A^*y,B_ x\rangle
\]
Adding this we conclude that (i) follows from the equality

\[
\langle Ax,B_2y\rangle+ 
\langle B_2 y,Ax\rangle=
\langle B_1x,A^*y\rangle+ \langle A^*y,B_ x\rangle\tag{v}
\]
To get (v) we use Proposition XX which for example gives
\[
\langle Ax,B_2y\rangle=
\langle x,A^*B_2y\rangle=
\langle x,B_1A^*y\rangle=
\langle B_1x,A^*y\rangle
\]
where the last equality used that $B_1$ is self-adjoint. In the same way one verifies that
\[
\langle B_2 y,Ax\rangle=
 \langle A^*y,B_ x\rangle
\]
and (v) follows.

\newpage


\noindent
\emph{The Nagy-Szeg� theorem.}
The constructions above yield the following result which is 
due to Nagy and Szeg� 
\medskip

\noindent
{\bf{Theorem}}
\emph{For every bounded linear operator $A$ on a Hilbert space
$\mathcal H$ there exists a Hilbert space
$\mathcal H^*$ which contains $\mathcal H$ and a unitary operator $U$
on $\mathcal U_1$ such that}
\[
A^n=\mathcal P\cdot U^n\quad\colon\quad n=1,2,\ldots
\]
where $\mathcal P$ is the orthogonal projection from
$\mathcal H_1$ onto $\mathcal H$.
\medskip


\noindent
\emph{Proof.}
Take $\mathcal H_1=\mathcal H\times\mathcal H$ where we have the unitary
operator $U_A$ above and let
$\mathcal P(x,y)= x$ be the projection onto the first
factor.
By the boock from of $U_A$ we have
$A=\mathcal PU_A$ and we leave it to the reader to show that
the previous constructions imply that
$A^n=\mathcal P\cdot U^n$ hold for every $n\geq 1$.
\bigskip

\noindent
{\bf{A general norm inequality.}}
The Nahy-Szeg� resut has
an important conseqeunce. 
Let $A$ as above be a contraction. If
$p(z)=
c_0+c_1<+\ldots+c_nz^n$ is an arbitrary polynomial with
complex coefficients
we get the 
operator $p(A)=\sum\,c_\nu A^\nu$ and 
with these notations one has:

\medskip

\noindent
{\bf{Theorem}}
\emph{For every pair $A,p(z)$ as above one has}

\[
||p(A)||\leq \max_{z\in D}\, |p(z)|
\] 
\emph{where the the maximum in the right hand side is taken on the unit disc.}
\medskip

\noindent
\emph{Proof.} Theroem X gives $p(A)= \mathcal P\cdot p(U_A)$.
Since the orthogonal $\mathcal P$-projection is norm decreasing we get

\[
||p(A)(\xi)||^2\leq ||p(U_A)(\xi,0)||^2
\]
Now the operational calculus from � 7 is applied to the unitaty operator
$U_A$ which yields a
probablity measure $\mu_\xi$ on
the unit circle such that
\[
 ||p(U_A)(\xi,0)||^2=
 \int_0^{2\pi}\, |p(e^{i\theta})|^2\cdot d\mu_\xi(\theta)
 \]
The right hand side is majorized by $|p|_D^2$ and Theorem XX follows.

\medskip

\noindent
{\bf{Corollary.}}
Let $A(D)$ be the disc algebra. Since each $f\in A(D)$ can be 
uniformly approximated by analytic 
polynomials, Theorem X entails that
there exists a bounded linear operator $f(A)$ for every contraction $A$.
\bigskip













\end{document}