
\documentclass{amsart}
\usepackage[applemac]{inputenc}


\addtolength{\hoffset}{-12mm}
\addtolength{\textwidth}{22mm}
\addtolength{\voffset}{-10mm}
\addtolength{\textheight}{20mm}

\def\uuu{_}


\def\vvv{-}

\def\bbb{{\bf{R}}}

\def\bbc{{\bf{C}}}

\def\bbi{\int_{xxx}}


\begin{document}





\centerline{\bf{Taylor series and real-analytic functions.}}
\bigskip


\noindent
The study of Taylor series of differentiable functions on
the real line
were investigated  by Borel and Denjoy who 
established several results during the years 1910-1922.
Among these we recall the following result.
Consider a real-valued $C^\infty$-function $f$
on
a bounded open interval $(a,b)$ whose derivatives  have
have finite maximum norms. For each non-negative integer $k$
we put
\[
C_k(f)=\bigl(|f^{(k)}|\bigr)^{\frac{1}{k}}\tag{*}
\]
where $|f^{(k)}|$ is the  maximum of the 
$k$:th order derivative of $f$ taken over $(a,b)$.
The question posed by Borel and Denjoy was to find
conditions expressed  by  estimates on the
sequence $\{C_k(f)\}$ which 
impliy that $f$ cannot be flat at a point
$x_0\in(a,b)$ , unless it is identically zero.
To say that $f$ is flat at $x_0$ means that
\[
f^{(k)}(x_0)=0\quad\colon\quad k=0,1,2,\ldots
 \]
Denjoy 
proved that if
\[
\sum_{k=0}^\infty\, \frac{1}{C_k(f)}=+\infty\tag{**}
\]
then $f$ cannot be flat at a point $x_0\in (a,b)$
unless it is identically zero.
In 1923 Carleman established necessary and sufficient conditions
for quasi.analyticity in his lectures at Sorbonnen which goes as follows:

\bigskip

\noindent
Let $\mathcal A=\{\alpha_\nu\}$
be a non-decreasing sequence
of positive real numbers.
Denote by $\mathcal C_\mathcal A$ the family of
all
$f\in C^\infty[0,1]$
for which there exists a constants $ M$ and $k$ which may depend on
$f$  such that
\[
\max\uuu {0\leq x\leq 1}\, |f^{(\nu)}(x)|\leq M\cdot k^\nu\cdot \alpha\uuu\nu^\nu
\quad\colon\quad \nu=0,1,\ldots\tag{0.4.1}
\]
One says that $\mathcal C_\mathcal A$ is a quasi-analytic class
if every $f\in C_ \mathcal A$ whose Taylor series is identically
zero at $x=0$ vanishes identically on $[0,1]$.
The following 
was proved by Carleman in 1922:
  \medskip
 
\noindent
{\bf{Theorem.}}
\emph{
The class  $C_\mathcal A$ is quasi-analytic if and only if}
\[
 \int\uuu 1^\infty\, \log\,\bigl[\,\sum_{\nu=1}^\infty\,
\frac{r^{2\nu}}{a_ \nu^{2\nu}} \,\bigr] \cdot \frac{dr}{r^2}=+\infty
\]
\medskip


\noindent
We shall expose material from Carelan's wok
where the major results appear in [Ca:xx.].
First we shall derive the result above by
Denjoy via a remarkable 
theorem from Carleman's article [xxx].





\newpage




\centerline{\bf{An inequality for differentiable functions.}}
\bigskip


\noindent
Let $n$ be a positive integer and denote by
$\mathcal F_n$ the family of
$n$  times continuously differentiable and real-valued functions
$f$ on the closed unit interval such that
\[
f^{(k)}(0)=
f^{(k)}(1)=0\quad\colon 0\leq k\leq n-1\tag{0.1}
\]
with a normalized  $L^2$-integral:
\[
\int_0^1\, f(t)^2\, dt=1\tag{0.2}
\]
Next, for each $1\leq k\leq n$
we can consider the $L^2$-norm of the $k$:th order derivative, i.e. set
\[
||f^{(k)}||_2)= \sqrt{\int_0^1\, f^{(k)}(t)^2\, dt}
\]



\noindent
{\bf{Main Theorem}}.
\emph{For each $n\geq 1$ and every $f\in \mathcal F_n$ one has
the inequality}
\[
\sum_{k=1}^{k=n}\, \frac{1}{||f^{(k)}||_2)^{\frac{2}{k}}}
\leq \pi\cdot e
\]
\emph{where $e$ is Neper's constant.}


\bigskip

\noindent
We shall first establish a general inequality of
independent interest.
Let $0<b_1<\ldots<b_n$ be a strictly increasing sequence of 
positive real numbers where $n\geq 1$ is some integer.
Let $\phi(z)$ be an analytic function
in the right half-plane $\mathfrak{Re} z>0$ which in addition extends to a 
continuous function on the imaginary axis.
Assume that its maximun norm over the right half-plane is
$\leq 1$  and in addition satisfies
\[
|z|^k\cdot \phi(z)\leq b^k_k\quad \colon k=1,\ldots,n\tag{1.1}
\]
\[
\phi(a)\geq e^{-a}\quad\colon\quad a>0\tag{1.2}
\]
\medskip

\noindent
Here (1.2) means that the restriction of $\phi$ to the non-negative real axis is
real-valued and satisfies the inequalities expressed by (1.2).
\medskip


\noindent
{\bf{1.3 Theorem.}} \emph{For each $\phi$ as above one has 
the inequality}
\[
\sum_{k=1}^{k=n}\,\frac{1}{b_k}\leq  \frac{e\pi}{2}\tag{1.3.1}
\]



\medskip

\noindent
We prove Theorem 1.3 in � 2 and proceed to show how it gives the Main Theorem.
We are given $f\in\mathcal F_n$ and put
\[
\phi(z)=\int_0^1\, e^{-zt}\cdot f(t)^2\, dt
\]
When $\mathfrak{Re} \,z\geq 0$ the absolute value $|e^{-zt}|\leq 1$
for all $t$ on the unit interval and hence (0.2) 
implies that the maximum norm of $\phi$ is $\leq 1$.
Next, if $1\leq k\leq n$
the vanishing in (0.1) and partial integration give
\[
z^k\cdot \phi(z)=
\sum_{\nu=0}^{\nu=k}\, \binom{k}{\nu}\,\int_0^1\, f^{(\nu)}(t)
\cdot f^{(k-\nu)}(t)(t)\, dt\tag{i}
\]
The Cauchy-Schwarz inequality estimates  the absolute value of the right hand side by
\[
\sum_{\nu=0}^{\nu=k}\, \binom{k}{\nu}\cdot ||f^{(\nu)}||_2\cdot 
||f^{(k-\nu)}||_2\tag{ii}
\]
At this stage we use a wellknown resut from calculus  which
entails that
\[
||f^{(\nu)}||_2\leq ||f^{(k)}||_k\quad\colon 0\leq \nu\leq k
\]
and from this the reader can check that (ii) is majorised by
$2^k\cdot ||f^{(\nu)}||^2_k$.
Hence 
\[
|z|^k\cdot |\phi(z)|\leq 2^k\cdot (||f^{(k)}||_2)^2\quad\colon\, k=1,2,\ldots\tag{iii}
\]
Put
\[
b_k= 2\cdot (||f^{(k)}||_2)^{\frac{2}{k}}\implies
|z|^k\cdot |\phi(z)|\leq b_k^k\tag{iv}
\]
Next, if $a>0$ we have
\[
\phi(a)= 
\int_0^1\, e^{-at}\cdot f(t)^2\, dt\geq e^{-a}\cdot 
\int_0^1\, f(t)^2\, dt=e^{-a}
\]
where the last equality holds by (0.2).
Hence we can apply Theorem 1.3 to $\phi$ and conclude that
\[
\sum_{k=1}^{k=n}\,\frac{1}{b_k}\leq \frac{e\pi}{2}\tag{v}
\]
Here  the $b$-numbers are given by (iv) which gives the Main Theorem.


\bigskip
\centerline{\bf{Proof of Theorem1.2}}

\medskip

\noindent
First we establish an inequality where condition (1.2) does not appear.

\medskip


\noindent
{\bf{2 Theorem.}} \emph{For each $\phi(z)$ which satisfies (1.1) and 
every real $a>0$ one has
the inequality}
\[
\frac{2a}{e\pi\cdot (1+\frac{a^2}{e^2b_1^2})}\cdot
\sum_{k=1}^{k=n}\,\frac{1}{b_k}\leq \log\, \frac{1}{\phi(a)}\tag{2.1}
\]

\medskip

\noindent
\emph{Proof.}
On the imaginary axis we consider the intervals
\[
\ell_k=[e\cdot  b_k,e\cdot eb_{k+1}]\quad\colon \, k=1,\ldots,n-1
\quad\& \quad\, \ell_n=[eb_n,+\infty)\tag{i}
\]
Since $\log e^{-1}=-1$ it is clear that
(1.1) gives the following for each $1\leq k\leq n$:
\[
\log|\phi(iy)|\leq -k\quad\colon \quad y\in \ell_k \tag{ii}
\]
Taking the negative intervals
$-\ell_k=[-e\cdot  b_{k+1},-e\cdot b_k]$ and
$-\ell_n=(-\infty, -eb_n$ we also have
\[
\log|\phi(iy)|\leq -k\quad\colon y\in -\ell_k \tag{iii}
\]
Moreover, since the maximum norm of $\phi$ is $\leq 1$ one has
\[
\log|\phi(iy)|\leq  0\quad\colon -b_1\leq y\leq b_1\tag{iv}
\]
Next, solving the Dirichlet problem we find the harmonic function $u$ in the open right
half-plane whose boundary values on
$(-b_1,b_1)$ is zero ,while
$u=-k$ in the the open intervals $\ell_k$ and $-\ell_k$ for every $k$.
The principle of harmonic majorisation applied to the subharmonic function 
$\log\,|\phi(z)|$ entails that
\[
\log\,|\phi(a)\leq u(a)\tag{v}
\]
Now we  evaluate $u(a)$ using Poisson's formula to represent
harmonic functions in the right half-plane.
For each $1\leq k\leq n-1$ we denote by $\theta_a(k)$ the
angle between the two vectors which join $a$ to the end-points
$ieb_k$ and $ieb_{k+1}$.
Computing
the area of the triangle with corner points
at $a,ieb_k,ieb_{k+1}$ the reader may check that
\[
\sqrt{a^2+e^2b_k^2}\cdot 
\sqrt{a^2+e^2b_{k+1}^2}\cdot \sin \theta_a(k)=
a\cdot e\cdot (b_{k+1}-b_k)\tag{vi}
\]
Finally, let $\theta_a(n)$ be the angle between the vector which joins $a$ with $ieb_n$ and
the vertical line $\{x=a{}$. The reader may check  with the aid of a figure that
\[
\sin \theta_a(n)=\frac{a}{\sqrt{a^2+e^2b_n^2}}\tag{vii}
\]
\medskip

\noindent
Poisson's formula gives
\[
u(a)=-\frac{2}{\pi}\cdot \sum_{k=1}^{k=n}\,k\cdot \theta_a(k)
\]

\noindent
Together with (v) it follows that
\[
\frac{2}{\pi}\cdot \sum_{k=1}^{k=n}\,k\cdot \theta_a(k)\leq 
\log\,\frac{1}{\phi(a)|}\tag{viii}
\]
The inequality $\sin t\leq t$ for every $t>0$ implies that
\[
\frac{2}{\pi}\cdot \sum_{k=1}^{k=n}\,k\cdot \sin \,\theta_a(k)\leq 
\log\,\frac{1}{\phi(a)|}\tag{ix}
\]
Next we use (vi-vii) to
estimate 
$\{ \sin \,\theta_a(k)\}$.
When $1\leq k\leq n-1$ we have from (vi)

\[
e^2\cdot  b_k\cdot b_{k+1}\cdot \sqrt{1+\frac{a^2}{e^2b_k^2}}
\cdot 
\sqrt{1+\frac{a^2}{e^2b_{k+1}^2}}\cdot \sin \theta_a(k)=
a\cdot e\cdot (b_{k+1}-b_k)\implies
\]
\[
e\cdot (1+\frac{a^2}{e^2b_1^2})\cdot \sin\,\theta_a(k)\leq a\cdot (\frac{1}{b_k}-\frac{1}{b_{k+1}})
\]
where the last inequality follows since $b_k\geq b_1$ for every $k$.
We conclude that the left hand side in (ix) majorizes
\[
\frac{2a}{e\pi\cdot (1+\frac{a^2}{e^2b_1^2})}\cdot
\sum_{k=1}^{k=n-1}\, k\cdot (\frac{1}{b_k}-\frac{1}{b_{k+1}})
+\frac{2}{\pi}\cdot n\cdot \sin \theta_a(n)
\]
Finally, (vii) gives
\[
\sin \theta_a(n)=\frac{a}{eb_n}\cdot \frac{1}{\sqrt{1+\frac{a^2}{e^2b_n^2}}}
\geq\frac{a}{eb_n}\cdot \frac{1}{1+\frac{a^2}{e^2b_1^2}}
\]
From this we conclude  that the left hand side in (ix) majorizes
\[
\frac{2a}{e\pi\cdot (1+\frac{a^2}{e^2b_1^2})}\cdot
\bigl(\sum_{k=1}^{k=n-1}\, k\cdot (\frac{1}{b_k}-\frac{1}{b_{k+1}})+
n\cdot \frac{1}{b_n}\bigr)
\]
Abel's summation formula identifies the last term with
$\sum_{k=1}^{k=n}\,\frac{1}{b_k}$.
Hence we have proved the requested inequality
\[
\frac{2a}{e\pi\cdot (1+\frac{a^2}{e^2b_1^2})}\cdot
\sum_{k=1}^{k=n}\,\frac{1}{b_k}\leq \log\, \frac{1}{\phi(a)}\tag{x}
\]
\bigskip

\noindent
{\bf{2.3 A special case.}}
Assume in addition to (1.1) that (1.2) holds.
\[
\phi(a)\geq e^{-a}\implies
\log\, \frac{1}{\phi(a)}\leq a\quad\colon a>0
\]
So after division with $a$ we see that  Theorem 1.2  gives
\[
\frac{2}{e\pi\cdot (1+\frac{a^2}{e^2b_1^2})}\cdot
\sum_{k=1}^{k=n}\,\frac{1}{b_k}\leq 1\tag{2.3.1}
\]
Passing to the limit as $a\to 0$ it follows that
\[
\sum_{k=1}^{k=n}\,\frac{1}{b_k}\leq \frac{e\pi}{2}\tag{2.3.2}
\]
which proves Theorem 1.3.


\newpage



\centerline{\bf{Carleman's reconstruction theorem for
real-analytic functions.}}
\medskip

\noindent
A  real-valued $C^\infty$-function $f$ on 
the closed unit interval is real analytic if and only if there
exist constant $C$ and $M$ such that
\[
\max_{0\leq x\leq 1}
|f^{(k)}(x)|\leq M\cdot k !\cdot C^k
\quad\colon k=1,2,\ldots\tag{0.1}
\]
The analyticity implies that
$f$ is determined by its derivatives at the origin.
However, the Taylor series
\[
\sum_{k\geq 0}\, f^{(k)}(0)\cdot \frac{x^k}{k !}
\] 
is in general only convergent for
in a small interval�$[0\leq x<\delta$.
In 1921 Borel posed the question how on
determines $f(x)$ on the whole interval from
the sequence $\{ f^{(k)}(0)\}$.
An affirmative answer was given by Carleman in 1923
via solutions to a family of variational problems
which 
goes as follows:
Put $\alpha_k=f^{(k)}(0)$ for each $k\geq 0$.
If $N$ is a positive integer 
we denote by $\mathcal H_N$ the Hilbert space 
whose elements are $N-1$-times
continuous differentiable functions $g$ on
$[0,1]$, and in addition
$g^{(N)}$ is square integrable, i.e. it belongs to
$L^2[0,1]$.
In "contemporary mathematics" this  means that $H_N$ is a Sobolev space. But of course the notion of weak $L^2$-derivatives was perfectly well understood long
before and for example used extensively in work by
Weyl before 1910.
Inside $\mathcal H_N$ we have the subspace
$\mathcal H_N(f)$ which  consists of functions
$g$ such that
\[ 
g^{(k)}(0)=
f^{(k)}(0)\quad\colon k=0,\ldots,N-1\tag{0.2}
\]
With these notations one regards the variational problem
\[
\min_{g\in \mathcal H_N(f)}\,
J_N(g)= \sum_{k=0}^{k=N}\, (\log (k+2))^{-2k}\cdot (k !)^{-2k}
\cdot \int_0^1\, g^{(k)}(x)^2\, dx\tag{0.3}
\]
Elementary  Hilbert space methods
yield a unique minimzing function
denoted by $f_N$.
These  successive solutions give
a sequence $\{f_N\} $ where each 
$f_N$ has at least $N-1$ continuous derivatives.
Less obvious is the following:

\medskip

\noindent
{\bf{Main Theorem.}} \emph{For each real-analytic function $f$
the sequence 
$\{f_N\}$ converges uniformly together with all derivatives
to $f$, i.e. for every $m\geq 0$ it holds that}
\[
\lim_{N\to \infty}\,|f_N^{(m)}-f^{(m)}|_{0,1}=0
\] 
\medskip

\noindent
{\bf{Remark.}} 
Since every
$f_N$ is determined by derivative of $f$ up to order $N-1$
at $x=0$ it means that
one has  a reconstruction of the real-analytic
function $f$
via these derivatives.


\medskip

\centerline{\bf{Proof of the Main Theorem}}


\bigskip

\noindent
For each $N$ we denote by $J_*(N)$ the minimum in the variational problem (0.3).
Among the competing functions we can choose
$f$  and hence
\[
J_*(N)\leq J_N(f)
\]
Now there exist constants $C$ and $M$ from (0.1)
which entails that
\[
 J_N(f)\leq M\cdot \sum_{k=0}^N\, (\log (k+2))^{-k} C^{2k}
\]
Since
$\log (k+2)$ tends to $+\infty$, it is clear that the series
\[
\sum_{k=0}^\infty\, (\log (k+2))^{-k} \cdot C^{2k}<\infty
\]
We conclude that there exists a constant
$J_*$ such that
\[
J_*(N)\leq J_*\quad\colon\, N=1,2,\ldots\tag{i}
\]
So if $m$ is some positive integer and $N\geq m$ we have
\[
\sum_{k=0}^{k=m}\, (\log (k+2))^{-2}\int_0^1\, f_N^{(k)}(x)^2\, dx
\leq J_*\tag{ii}
\]
Now we recall the classic resut due to Arzela-Ascoli
which implies that bounded sets in $H_m$
give relatively compact subsets of
$C^{m-1}[0,1]$.
Since (ii) hold for each $m$, it follows
by a standard diagonal procedure which is left to the reader 
that we can find a subsequence $\{g_\nu=f_{N_\nu}\}$
such that
the sequence of derivatives
$\{g_\nu^{(m)}\}$ converge uniformly for every $m$, i.e
$g_\nu\to g_*$ holds in the space $C^\infty[0,1]$.
Next, by (0.2) we have for each fixed integer $k\geq 0$:
\[
f^{(k)}(0)=f_N^{(k)}(0)\quad\colon N\geq k+1
\]
From this it follows that
\[
f^{(k)}(0)=g_*^{(k}(0)\quad\colon k=0,1,2\dots\tag{iii}
\]
Hence the $C^\infty$-function
\[
\phi=f-g_*
\]
 is flat at $x=0$.
Next, for a fixed integer $k$
the uniform bound in
(ii) gives
\[
\int_0^1\, \phi^{(k)}(x)^2\, dx\leq
 J_*\cdot (\log (k+2))^{2k}\cdot (k !)^2\tag{iv}
\]
Moreover, for each $0<x\leq 1$ the Cauchy-Schwartz inequality gives
\[
\phi^{(k)}(x)= \int_0^x\, \phi^{(k+1)}(t)\,dt\leq
\sqrt{\int_0^1\, \phi^{(k)}(x)^2\, dx}
\]
and since (iv) hold for every $k$ it follows that
\[
\max_x\, |\phi^{(k)}(x)|\leq  J_*\cdot (\log (k+2))^k\cdot k !
\]
Since $k !\leq k^k$
this entails that
\[ 
\mathcal C_k(\phi)\leq J_*^{\frac{1}{k}}\cdot  k\cdot (\log (k+2))
\]
Since the series $\sum_{k=1}^\infty\, \frac{1}{k\log k}$
is divergent we conclude that
\[
\sum_{k=1}^\infty\, \frac{1}{\mathcal C_k(\phi)}=+\infty
\]
Hence Denjoy's result in xxx proves that $\phi$ is identically zero.
which means that

\[
\lim _{k\to\infty}\, f_{N_k}=f\tag{*}
\]
where the convergence holds in the space
$C^\infty[0,1]$.
Finally, the reader may check that (*)
holds for 
an arbitrary convergent subsequence which by the previous compactness
by Arzela-Ascoli entails  that
the whole sequence $\{f_N\}$ converges to $f$. This
finishes the proof of the Main Theorem.










\end{document}
