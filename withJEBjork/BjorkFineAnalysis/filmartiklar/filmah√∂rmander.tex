\documentclass{amsart}
\usepackage[applemac]{inputenc}

 
\begin{document}

\centerline{\bf{II. Null solutions of PDE:s
with constant coefficients.}}

\bigskip

\noindent
{\bf{Introduction.}}
We  expose material from
the  article 
\emph{Null solutions to partial differential operators}
[Arkiv f�r matematik. 1959] by Lars H�rmander.
The Main  Theorem to be  announced below
contains  an instructive proof where
complex line integrals taken over contours 
adapted to the complex zeros of the polynomial  $P(\zeta)$
of
$n$ independent complex variables which corresponds  to
a PDE-operator $P(D)$ are used in order to  get an ample family of null solutions, i.e. functions $u(x)$ for which $P(D)u=0$.
H�rmander 
employs  Puiseux series constructed via embedded curves in the
zeros of $P(\zeta)$ to get such $u$-functions
supported by the half-space $\{x_n\geq 0\}$
in the case when the hyperplane $\{x_n=0\}$ is characterstic to the
differential operator $P(D)$.
The remaining part of the proof  of the Main Theorem 
is based upon
the Paley-Wiener theorem
and  duality results  from  general distribution theory.
Here one crucial   point appears. Namely, thanks to
constructions due to   Gevrey, there exists test-functions 
whose higher order derivatives have 
a good control which entail that their Fourier transforms enjoy certain decay conditions.
So the subsequent material offers  an instructive mixture of algebra and analysis.
For example, one ingredient employs
a
density
result which goes back to
Pusieeux which can be considered  as a sharp version of the standard Nullstellen Satz.
Namely.
for 
every  algebraic hypersurface
$S=\{P(\zeta)=0\}$ in the $n$-dimensional complex $\zeta$-space there exists
an ample family of   curves of two independent complex
 contained in
$S$ with the property
that if $g(\zeta)$ is an entire function which vanishes on all these curves then
is is identically zero on $S$. This entails 
that
\[
g(\zeta)= P(\zeta)h(\zeta)
\] 
for another entire function $h$.
Moreover,  when
$g$�is the Fourier-Laplace transform of a distribution $\mu$
with compact support, the  Paley-Wiener theorem entails that
$h=\widehat{\gamma}$ for another distribution whose compact support is
contained in the convex hull of $\mu$
which is used during the final step in the proof of the Main Theorem.
\bigskip

\noindent
Before we  announce our Main Theorem 
we need some notations.
Let $n\geq 2$ and in ${\bf{R}}^n$ we consider the hyperplane
$H=\{x_n=0\}$. Let $P(D)$ be a differential operator with
constant coefficients. Here $D_k=-i\cdot \partial/\partial x_k$
and by Fourier's inversion formula 
\[
P(D)f(x)= (2\pi )^{-n}\cdot \int\, e^{i\langle x,\xi\rangle}\, \widehat{f}(\xi)\, d\xi
\]
for test-functions $f(x)$. Let $m$ be the order of $P(D)$ which means that
\[
P(D)=\sum\, c_\alpha\cdot D^\alpha
\] 
where the sum is taken over multi-indices $\alpha$ for which
$|\alpha |= \alpha_1+\ldots+\alpha_n\leq m$.
The leading form is defined by
\[
P_m(D)=\sum_{|\alpha |=m}\, c_\alpha\cdot D^\alpha
\]
The hyperplane $H$ is characteristic  if $P_m(N)=0$ where
$N=(0,\ldots,1)$, i.e. the term $D_n^m$ does not appear in 
$P_m(D)$ with a non-zero coefficient.
Put $H_+=\{x_n>0\}$ and 
\[
\mathcal N_+=\{ g\in C^\infty(H_+)\,\colon\, P(D)(g)=0
\]
Thus, we consider $C^\infty$-functions in the open half-plane $H_+$ which are
null solutions to $P(D)$ in this open half-plane.
A smaller space is given by
\[
\mathcal N_*=\{ g\in C^\infty({\bf{R}}^n)\,\colon\, P(D)(g)=0\,\,\text{and}\,\,
\text{Supp}(g)\subset \overline{H_+}\}
\]
Denote by
$\mathcal N_*^\perp$  the family of  distributions $\mu$ with compact support in
$H_+$ which are zero on $\mathcal N_*$.
\medskip

\noindent
{\bf{Main Theorem.}} \emph{Every distribution $\mu$ in
$\mathcal N_*^\perp$  is zero on $\mathcal N_+$}
\medskip

\noindent
The proof requires several steps.
The crucial step is to 
construct functions in $\mathcal N_*$ and after prove that they 
give a dense subspace of $\mathcal N_+$.
So we begin with:
\bigskip


\centerline {\bf{1. A construction of null solutions.}}
\medskip

\noindent
Let $\xi_0$ be a real $n$-vector such that $P_m(\xi_0)\neq 0$ and
$\zeta_0$ some complex $n$-vector.
Let $s$ and $t$ be independent complex variables and set
\[
p(s,t)= P(s\cdot N+t\xi_0+\zeta_0)
\]
This gives a polynomial  where the term $t^m$ appears since
$P_m(\xi_0)\neq 0$. At the same time $s^m$ does not appear because
$P_m(N)=$ is assumed.
A classic result due to Pusieux from 1852 shows that
there exists a positive  integer $p$ and a series
\[
t(s)=s^{k/p}\cdot \sum_{j=0}^\infty\, c_j\cdot s^{-j/p}\tag{1.1}
\]
where $0\leq k<p$ which converges when $|s|$ is large, i.e. there
exists some $M>0$ such that
\[
 \sum_{j=0}^\infty\, |c_j|\cdot M^{-j/p}<\infty
\]
Moreover,
\[
P(s\cdot N+t(s)\xi_0+\zeta_0)=0\quad\colon\, |s|\geq M\tag{1.2}
\]
In the lower half-plane $\mathfrak{Im}(s)<0$ 
we
choose a single valued branch of $s^{1/p}$ where
\[
s= |s|\cdot e^{i\phi}\implies s^{1/p}= |s|^{1/p}\cdot e^{i\phi/p}\quad
\colon\quad -\pi<\phi<0
\]
Next, choose a number
\[
1-1/p<\rho<1
\]
Now $(is)^\rho$ has a single valued branch for which 
\[
\mathfrak{Re}((is)^\rho)= \cos\,\frac{\rho\pi}{2}\cdot |s|^\rho\cdot \cos(\rho\cdot(\pi/2+ \phi))\tag{1.3}
\] 
So if $\epsilon>0$ we have
\[
|e^{-\epsilon(is)^\rho}|=
e^{-\epsilon\cdot \mathfrak{Re}(is)^\rho}=
e^{-\epsilon \cdot |s|^\rho\cdot \cos\,(\rho(\pi/2+\phi))}\tag{1.4}
\]
Since $\rho<1$ we notice that
\[
\cos\,(\rho(\pi/2+\phi))\geq \cos\, \rho\pi/2= a
\]
for all $-\pi<\phi<0$�where  a is positive constant $a$.
It  follows that
\[
|e^{-\epsilon(is)^\rho}|\leq e^{-a\epsilon\cdot |s|^\rho}\tag{1.5}
\]
for all $s$ in the lower half-plane, arm also when $s$ is real.
\medskip


\noindent
Let  $M$ be as above and denote by $C_*$ the
circle in the lower half-pane which consists of the two 
real intervals
$(-\infty,-M)$ and $(M,+\infty)$ and the lower
half-circle 
where $|s|=M$. For each $x\in{\bf{R}}^n$ and every non-negative integer $\nu$
we get the complex line integral
\[
\int_{C_*}\, e^{i\langle x,sN+t(s)\xi_0+\zeta_0\rangle}
\cdot s^{\nu/p}\cdot e^{-\epsilon(is)^\rho}\, ds\tag{*}
\]
This integral is absolutely convergent. Namely,
during the integration on the real interval�$(-\infty,-M)$ or the real interval $[M,+\infty)$
we see that (1.5) gives  estimates the absolute value of the integrand by
\[
|s|^{\nu/p}\cdot \cdot
|e^{it(s)\langle x,\xi_0\rangle}|\cdot
e^{-a\epsilon\cdot |s|^\rho}\tag{1.6}
\]
Next, the Puiseux expansion for $t(s)$ entails that
\[
 |t(s)\leq A|s|^{1-1/p}
 \]
hold for some constant $A$. Since $\rho>1-1/p$
It follows that (.6) is majorised by
\[
|s|^{\nu/p}\cdot 
e^{A\cdot |\langle x,\xi_0\rangle|\cdot |s|^{1-1/p}}\cdot
e^{-a\epsilon\cdot |s|^\rho}\tag{1.6}
\]
Since $\rho>1-1/p$
we conclude that
the line integral (*) converges absolutely
for each  positive integers�$\nu$.
\medskip

\noindent
{\bf{Exercise.}}
Show by Cauchy's theorem in analytic function theory that
the line integral  (*) does not depend on $M$ as soon as it has been
chosen so that
the Puiseux series definiing $t(s)$ exists. The resulting value of (*) is 
therefore a function of $x$ and $\epsilon$ and gives a function $u_\epsilon(x)$
defined for all $x$ in ${\bf{R}}^n$.
Moreover, the reader should check
that when  $\epsilon>0$ kept fixed this yields a $C^\infty$-function of
$x$.
In particular

\[
P(D)(u_\epsilon)(x)=
\int_{C_*}\, P(sN+t(s)\xi_0+\zeta_0)\cdot
e^{i\langle x,sN+t(s)\xi_0+\zeta_0\rangle}
\cdot s^{\nu/p}\cdot e^{-\epsilon(is)^\rho}\, ds\tag{**}
\]
Since $P(sN+t(s)\xi_0+\zeta_0)=0$ when $|s|\geq M$ we conclude that
$P(D)(u_\epsilon)=0$, i.e. $u_\epsilon$ is a null solution.
\medskip

\noindent
{\bf{The inclusion $\text{Supp}(u)\subset \overline{H}_+$.}}
In (*) we perform a line integral whose integrand is an analytic function in
the lower half-plane.
Using Cauchy's theorem the reader can check that for any
$M^*>M$ we have
\[
u_\epsilon(x)=\int_{\mathfrak{Im}(s)=-M^*}\,
e^{i\langle x,sN+t(s)\xi_0+\zeta_0\rangle}
\cdot s^{\nu/p}\cdot e^{-\epsilon(is)^\rho}\, ds\tag{**}
\]
With $s=t-iM^*$ we have
\[
|e^{i\langle x,sN\rangle}|= e^{M^*\langle x,N\rangle}
\]
If
$\langle x,N\rangle<0$ this decreases exponentially to zero as
$M^*\to+\infty$ and then the reader can  check that the limit of (**)  as
$M^*\to+\infty$ is zero.
This proves that the null solution $u_\epsilon$ is supported by
the half-plane $\overline{H}_+$
and hence belongs to $\mathcal N_*$.


\newpage


\centerline{\bf{� 2. A study of $\mathcal N_*^\perp$.}}
\medskip

\noindent
Consider a test-function
$\phi $ with a compact support in $H_+$ such that
$\phi(\mathcal N_*)=0$. It gives the
entire function in the $n$-dimensional complex $\zeta$-space:
\[
\Phi(\zeta)=\int\, e^{i\langle x,\zeta\rangle}\, \phi(x)\, dx\tag{2.0}
\]
Using the convergence of the line integrals in (*) the reader
should verify that  Fubini's theorem gives 
the equation
\[
\int\, u_\epsilon(x)\phi(x)\, dx=
\int_{C_*}\,
\Phi(sN+t(s)\xi_0+\zeta_0)
\cdot s^{\nu/p}\cdot e^{-\epsilon(is)^\rho}\, ds\tag{2.1}
\]
Since $\phi(\mathcal N_*)=0$ is assumed it follows that the last integral is zero
for all non-negative integers $\nu$ and each $\epsilon>0$.
\medskip

\noindent
{\bf{2.2 Another vanishing integral.}}
In the upper half-plane $\mathfrak{Im}(s)>0$
we can also choose single-valued branches of $s^{1/p}$ and 
$(-is)^\rho$, where the last branch is chosen so that
the value is $a^\rho>0$ when $s=ai$ for $a>0$.
Then  we construct the contour $C^*$ given by the  real intervals $(\infty,-M)$ and 
$(M,+\infty)$ together with
the upper half circle of radius $M$, which  for each non-negative integer $\nu$
gives  the function
\[
v_\epsilon(x)=
\int_{C^*}\, e^{i\langle x,sN+t(s)\xi_0+\zeta_0\rangle}
\cdot s^{\nu/p}\cdot e^{-\epsilon(-is)^\rho}\, ds\tag{*}
\]
Exactly as in � 1  one verifies that this gives a $C^\infty$-function of $x$ 
supported by the right half space $\{x_n\leq 0\}$.
Since $\phi$ has compact support in
$H_+$ it follows that
\[
0=\int\, v_\epsilon(x)\phi(x)\, dx=
\int_{C^*}\,
\Phi(sN+t(s)\xi_0+\zeta_0)
\cdot s^{\nu/p}\cdot e^{-\epsilon(-is)^\rho}\, ds\tag{2.2.1}
\]
\medskip

\noindent
{\bf{2.3 The limit as $\epsilon\to 0$}}.
In (2.2.1) we have vanshing integrals for each  $\epsilon>0$.
If the test-function $\phi(x)$
belongs to a suitable Gevrey class with more regularity than
an arbitrary test-function, then
the entire function $\Phi(\zeta)$ enjoys a decay condition which enable us to pass
to the limit as $\epsilon\to 0$ in   (2.2.1).
To find a sufficient decay condition 
we set $\zeta=\xi+i\eta$, and 
with $M$ kept fixed
we study the function
\[ 
s\mapsto \Phi(sN+t(s)\xi_0+\zeta_0)
\]
We already know that there is a constant $C$ such that
$|t(s)|\leq C|s|^{1-1/p}$ when $|s|\geq M$.
Since $\xi_0$ and $\zeta_0$ are fixed this gives a constant $C_1$ such that
\[
|\mathfrak{Im}(sN+t(s)\xi_0+\zeta_0)|\leq C_1(1+|s|)^{1-1/p}\tag{2.3.1}
\]
At the same time we have the unit vector $N$ and get a positive  constant 
$C_2$ such that
\[
|\mathfrak{Re}(sN+t(s)\xi_0+\zeta_0)|\geq C_1(1+|s|)\tag{2.3.2}
\]
when $|s|$ is large. Suppose now that
the test-function $\phi$ has been chosen so that
\[
|\Phi(\xi+i\eta)\leq C\cdot e^{A|\eta|)-B|\xi|^{b}}\tag{2.3.3}
\]
hold for some constants $C,A,B,a$ where $b<1$.
From (2.3.1-2.3.2) this gives with other positive constants
\[
|\Phi(sN+t(s)\xi_0+\zeta_0)\leq C_1e^{A_1|s|^{1-1/p}-B_1|s|^{b}}\tag{2.3.4}
\]
With $\rho$ chosen as in � 1 where the equality (1.3) is used,
it follows that as sson as
\[
a>\rho
\]
then
we get absolutely convergent integrals 
\[
\int_{|s|\geq M}\,|\Phi(sN+t(s)\xi_0+\zeta_0)\cdot |s|^w|\, ds<\infty
\]
for every positive integer $w$.
This enable us to pass to the limit in (2.2) and conclude that
\[
\int_{C^*}\,
\Phi(sN+t(s)\xi_0+\zeta_0)
\cdot s^{\nu/p}\, ds=0\tag{2.3.5}
\]
for every non-negative  integer $\nu$.
In the same fashion we find vanishing integrals with
$C^*$ replaced by $C_*$.
The vanishing of these integrals for all $\nu\geq 0$ entails
by the classic result  due to Puiseux
that
that
$\frac{\Phi}{P}$ is an entire function.
Then a  division theorem with bounds due to Lindel�f, together with the
Paley-Wiener theorem imply  that
the entire quotient
\[
\frac{\Phi}{P}=\Psi\tag{i}
\]
where $\Psi$ is given as in (2.0) for some test-function
$\psi$ supported by the convex hull of the support of
$\phi$.
Moreover, (i) entails that
\[
P(-D)(\psi)=\phi
\]
and then it is obvious that $\phi$ annihilates
$\mathcal N_+$.
Hence we have proved the implication in Theorem 0 for distributions
which
are defined by test-functions $\phi$ whose  associated
entire $\Phi$-function satisfies (2.3.3) with some
$a>1-1/p$.
But this finishes the proof of  the Main Theorem.
Namely, fix
$a$ as above and put
\[
\delta=1/a
\]
Now $\delta>1$ which by a classic construction due to Gevrey
enable us to construct an ample family of test-functions
$\phi$ for which (2.3.3) hold and at the same time
this family is weak-star dense in the space of distributions
with compact support in $H_+$ which gives the Main Theorem.
For details about this density the reader can consult H�rmander's article
or his text-book [H�:xx] if necessary.
See also the article [Bj] by G�ran Bj�rck which offers a very detailed
study of distributions arising from Gevrey classes.






\end{document}











