\documentclass{amsart}

\usepackage[applemac]{inputenc}

\addtolength{\hoffset}{-12mm}
\addtolength{\textwidth}{22mm}
\addtolength{\voffset}{-10mm}
\addtolength{\textheight}{20mm}

\def\uuu{_}

\def\vvv{-}

\begin{document}






\bigskip

\noindent
Following constructions due to Cantor  we
shall exhibit a family of closed subsets of the unit interval $[0.1]$.
In the first step we remove an open interval $(a,b]$ where
$0<a<b<1$ and the complement
$[0,1]\setminus (a,b)$ becomes a union of disjoint closed interval $[0,a]$ and $[b,1]$.
In the second steep we remove open intervals $(a-1,a_2)$ with
$0<a_1<a_2<a$ and $(b_1,b_2)$ with $b<b_1<b_2<1$.
Then there remains four disjoint closed intervals
$[0,a_1), [a_2,a], [b.b_1], [b_2,1]$.
We can continue in this way, and at step $N\geq 2$
one has removed $2^N-1$  pairwise  disjoint open intervals, while three remain $2^N$ many disjoint closed intervals
whose union is a compact set denoted by $\mathcal E_N$. These closed sets decrease with $N$ and the intersection
\[
\mathcal C= \cap \, E_N
\] 
is called  a Cantor  set.
One is foremost interested in
constructions as above where
$\mathcal C$ is meager, i.e. without interior ppints.
When this holds we can construct a non-decreasing and continuous function
$L$ which is constant on each of the removed open intervals
and the range is $[0.1]$.
Namely, at step $N$
we find a strictly increasing sequence
\[
0=a_0<a_1<\ldots <a_{2^N-1}<a_{2^N}=1
\]
where $\{[a_k.,a_{k+1}]\}$
are the disjoint closed intervals whose union is $E_N$.
For a fixed $n$ wwe find he
piecewise linear function $f_N(x)$
which  is constant on the removed open intervals at step $N$
while
\[ 
f(a_{k+1})= f(a_k)+ 2^{-N}\quad\colon\quad 0\leq k\leq 2^N_1
\]
and $f_N(0)=0$ and $f_N(1)=1$.
The reader may check that
the maximum norms
\[
\max_x\, |f_{N+1}(x)-f_N(x)|\leq 2^{-N}
\]
hold for every $N$.
It follows that there exists a continuous limit function
\[
L(x)= \lim_{n\to \infty}\, f_N(x)
\]
It is clear that $L$ is non-decreasing and by construction constant on
the removed open intervals. In particular the derivative
$f'(x)$ exists and is zero in the open complement
$[0,1]\setminus\mathcal C$.
The question arises at which points $x\in\mathcal C$  there  exists a derivative.
Lebesgue considered the case when
$\mathcal C$ is a null set which means that
the sum of the lengths of removed open intervals is equal to one.
In this case the $L$-function is quite remarkable since
it is non-constant and yet its derivative is zero almost everywhere.
One cannot therefore recapture $f$ via its derivative as in ordinary cal cues
and the meaning of a differential $df$ is obscure.
This led to general measure theory where
the differential $dL$ has a meaning 
as a  so called singular measure whose primitive  is $L$.
\medskip

\noindent
A special  Cantor  set  arises when one removes triadic open intervals,  i.e first
$(1/3,2/3)$and in the second step
$(1/9,2/9)($ and $(7/9,8/9)$ and so on. Since
\[
1/3+\sum_{k=1}^{\infty}\, 2^k\cdot 3^{-k-1}=1
\]
it follows that $\mathcal C$ is a null set.
\medskip


\noindent{\bf{Remark.}}
With the special $\mathcal C$ above we have the $L$-function. as far as I know one does not 
know exactly at which points $x\in\mathcal C$ the $L$-function has a finite ordinary derivative.
So in general measure theory one is content with
the
less precise assertion that $L$ has a vanishing derivative almost everywhere.































\end{document}

