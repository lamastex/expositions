
\documentclass{amsart}
\usepackage[applemac]{inputenc}


\addtolength{\hoffset}{-12mm}
\addtolength{\textwidth}{22mm}
\addtolength{\voffset}{-10mm}
\addtolength{\textheight}{20mm}

\def\uuu{_}


\def\vvv{-}

\def\bbb{{\bf{R}}}

\def\bbc{{\bf{C}}}

\def\bbi{\int_{xxx}}


\begin{document}



\centerline{\bf{0.6 Taylor series and quasi\vvv analytic functions.}}
\bigskip


\noindent
The study of Taylor series of differentiable functions on
the real line
was  considered in a general setting by Borel and Denjoy.
These authors established several 
remarkable facts  during the years 1910-1922.
Among these one should mention Denjoy's result from 1921.
He considered  a real-valued $C^\infty$-function $f$
on
a bounded open interval $(a,b)$ whose derivatives  have
have finite maximun norms. For each non-negative integer $k$
we put
\[
C_k(f)=\bigl(|f^{(k)}|_{a,b}\bigr)^{\frac{1}{k}}\tag{*}
\]
where $|f^{(k)}|_{a,b}$ is the  maximum of the 
$k$:th order derivative of $f$ taken over $(a,b)$.
The question posed by Borel and Denjoy was to find
coinditions in order that some a prioro estimates on the
sequence $\{C_k(f)\}$
implies that $f$ cannot be flat at a point
$x_0\in(a,b)$ , unless it is identically zero.
To say that $f$ is flat at $x_0$ means that
$f^{(k)}(x_0)=$ for all non-negative integers $k$.
Denjoy proved that if
\[
\sum_{k=0}^\infty\, \frac{1}{C_k(f)}=+\infty\tag{**}
\]
then $f$ cannot be flat at a point $x_0\in (a,b)$
unless it is identically zero.
In 1923 Carleman established necessary and sufficient conditions
for quasi.analyticity in  lectures held at Sorbonne.
They are exposed in � xx.
A crucial result in Carleman's work  goes as follows:
\medskip

\noindent
{\bf{0.1 Carleman's  inequality.}}
Let $n$ be a positive integer and denote by
$\mathcal F_n$ the family of
$n$  times continuously differentiable functions
$f$ on the closed unit interval such that
\[
f^{(k)}(0)=
f^{(k)}(1)=0\quad\colon 0\leq k\leq n-1\tag{0.1.1}
\]
and   the $L^2$-integral is normalised so that
\[
\int_0^1\, f(t)^2\, dt=1\tag{0.1.2}
\]
\medskip

\noindent
{\bf{0.2 Theorem}}.
\emph{For each $n\geq 1$ and every $f\in \mathcal F_n$ one has
the inequality}
\[
\sum_{k=1}^{k=n}\, \frac{1}{C_k(f)^2}\leq \pi\cdot e
\]
\emph{where $e$ is Neper's constant.}
\medskip

\noindent
We prove this result in � xx.
It is very instructive since it
demonstrates the usefulness of harmonic majorisations of
subharmonic functions, i.e. the proof relies upon a detour into the complex domain.
In � xx we also demonstrate how Theorem 0.2 gives a simple proof of' Denjoy's result.


\medskip

\centerline{\bf{0.3 Carleman's reconstruction theorem for
real-analytic functions.}}
\medskip

\noindent
A  real-valued $C^\infty$-function $f$ on 
the closed unit interval is real analytic if and only if there
exist constant $C$ and $M$ such that
\[
\max_{0\leq x\leq 1}
|f^{(k)}(x)|\leq M\cdot k !\cdot C^k
\quad\colon k=1,2,\ldots\tag{0.3.1}
\]
The analyticity implies that
$f$ is determined by its derivatives at the origin.
However, the Taylor series
\[
\sum_{k\geq 0}\, f^{(k)}(0)\cdot \frac{x^k}{k !}
\] 
is in general only convergent for
in a small interval�$[0\leq x<\delta$.
In 1921 Borel posed the question how on
determines $f(x)$ on the whole interval from
the sequence $\{ f^{(k)}(0)\}$.
An affimative answer was given by Carleman in 1923
via solutions to a family of variational problems
which 
goes as follows:
Put $\alpha_k=f^{(k)}(0)$ for each $k\geq 0$.
If $N$ is a positive integer 
we denote by $\mathcal H_N$ the Hilbert space 
whose elements are $N-1$-times
continuous differentiabke functions $g$ on
$[0,1]$, and in addition
$g^{(N)}$ is square integrable, i.e. it belongs to
$L^2[0,1]$.
In "contemporary mathematics" this  means that $H_N$ is a Sobolevs space. But of course the notion of weak $L^2$-derivatives was perfectly well understood long
before and for example used extensively in work by
Weyl before 1910.
Inside $\mathcal H_N$ we have the subspace
$\mathcal H_N(f)$ which  coinsists of functions
$g$ such that
\[ 
g^{(k)}(0)=
f^{(k)}(0)\quad\colon k=0,\ldots,N-1\tag{0.3.2}
\]
With these notatons one regards the variational problem
\[
\min_{g\in \mathcal H_N(f)}\,
\sum_{k=0}^{k=N}\, (\log (k+2))^{-2k}\cdot (k !)^{-2k}
\cdot \int_0^1\, g^{(k)}(x)^2\, dx\tag{0.3.3}
\]
Elementary  Hilbert space methods
yield a unique minimziing function
denoted by $f_N$.
These  succesive solutions give
a sequence $\{f_N\} $ where each 
$f_N)$ has at least $N-1$ cointinous derivatives.
Less obvious is the following:

\medskip

\noindent
{\bf{0.3.4 Theorem.}} \emph{For each real-analytic function $f$
the sequence 
$\{f_N\}$ converges uniformly together with all derivatives
to $f$, i.e. for every $m\geq 0$ it holds that}
\[
\lim_{N\to \infty}\,|f_N^{(m)}-f^{(m)}|_{0,1}=0
\] 
\medskip

\noindent
{\bf{Remark.}} 
Since every indivdual funtion
$f_n$ is determined by derivative of $f$ up to order $N-1$
at $x=0$ it means that
one has attained a reconstruction of the real-analytic
function $f$
via these derivatives.




\centerline {\bf{� 0.4 General quasi-analytic classes.}}

\bigskip

\noindent
Let $\mathcal A=\{\alpha_\nu\}$
be a non-decreasing sequence
of positive real numbers.
Denote by $\mathcal C_\mathcal A$ the family of
all
$f\in C^\infty[0,1)$
for which there exists a constants $ M$ and $k$ which may depend on
$f$  such that
\[
\max\uuu {0\leq x\leq 1}\, |f^{(\nu)}(x)|\leq M\cdot k^\nu\cdot \alpha\uuu\nu^\nu
\quad\colon\quad \nu=0,1,\ldots\tag{0.4.1}
\]
One says that $\mathcal C_\mathcal A$ is a quasi-analytic class
if every $f\in C_ \mathcal A$ whose Taylor series is identically
zero at $x=0$ vanishes identically on $[0,1)$.
A conclusive result about quasi-analyticy
was proved by Carleman in 1922 and  asserts the following:
  \medskip
 
\noindent
{\bf{0.4.1Theorem.}}
\emph{
The class  $C\uuu\mathcal A$ is quasi\vvv analytic if and only if}
\[
 \int\uuu 1^\infty\, \log\,\bigl[\,\sum\uuu{\nu=1}^\infty\,
\frac{r^{2\nu}}{a\uuu \nu^{2\nu}} \,\bigr] \cdot \frac{dr}{r^2}=+\infty
\]

\medskip

\noindent
'The 
proof of Theorem 0.4.1 
is rather involved 
 It is presented in the separate 
section � xx.
\bigskip

\centerline{\bf {� 0.5 . Quasi-analytic boundary values.}}
\medskip


\noindent
Another  problem 
is concerned with  boundary values of analytic functions
whose
set of non\vvv zero Taylor\vvv coefficients is sparse.
In general, consider a power series $\sum\, a\uuu nz^n$
whose radius of convergence equal to one. Assume that there exists an interval $\ell$
on the unit circle such that
the analytic function $f(z)$ defined by the series extends to
a continuous function in the closed sector where
$\text{arg}(z)\in\ell$. So on $\ell$ we get a continuous boundary
value function $f^*(\theta)$.
Let $f$ be given by the series
\[
f=\sum\, a\uuu n\cdot z^n
\]
Suppose that gaps occur
and write  the sequence of non\vvv zero coefficients
as $\{ a\uuu{n\uuu 1}, a\uuu{n\uuu 2}\ldots\}$
where $k\mapsto n\uuu k$ is a  strictly increasing sequence.
With these notations the following result is due to Hadamard:
\medskip

\noindent
{\bf{0.5.1  Theorem.}}\emph{
Let $f(z)$ be as above
and assume it has a continuous extension to some open  interval on the 
unit circle where 
the boundary function $f^*(\theta)$ is real\vvv analytic. Then there exists an integer 
$M$ such that}
\[ 
n\uuu{k+1}\vvv n\uuu k\leq M\quad\colon k=1,2,\ldots
\]
\bigskip

\noindent
Hadamard's result was extended to the quasi\vvv analytic case in
[Carleman] . In particukar we may consider
the case when
$f^*$ belongs to a 
Denjoy class $\mathcal D_\mathcal A$
for a sequence
$\mathcal A= \{\alpha\uuu\nu\}$ where the series in � xx diverges.
\[ 
\sum\, \frac{1}{\alpha_\nu}=+\infty
\]
In [ibid] it is proved that
when this hold and $f(z)$ is not identically zero, then
the gaps of its Taylor coefficients
cannot be too sparse. However, in contrast to Hadamard's theorem the
result is more involved and
and the
rate of increase 
depends upon $\{\alpha\uuu\nu\}$.
Up to the present date
it appears that no precise descriptions of
the growth of $k\mapsto n\uuu k$ which would ensure unicity
is known   while one regards arbitrary Denjoy classes as above.
So there remains many basic  questions concerned with quasi-analyticity, and
readers who would like to pursue this should first
consult 
the
subtle analysis which appears in Carleman's original work










\newpage



\centerline {\bf{� 1. Proof of Theorem 0.2}}
\bigskip

\noindent
We shall first establish a general inequality of
independent interest.
Let $0<b_1<\ldots<b_n$ be a strictly increasing sequence of 
positive real numbers where $n\geq 1$ is some integer.
Let $\phi(z)$ be an analytic function
in the right half-plane $\mathfrak{Re} z>0$ which in addition extends to a 
continuous function on the imaginary axis.
Assume that its maximun norm over the right half-plane is
$\leq 1$  and in addition
\[
|z|^k\cdot \phi(z)\leq b^k_k\quad \colon k=1,\ldots,n\tag{1.1}
\]

\medskip

\noindent
{\bf{1.2 Theorem.}} \emph{For each $\phi$ as above and
every real $a>0$ one has
the inequality}
\[
\frac{2a}{e\pi\cdot (1+\frac{a^2}{e^2b_1^2})}\cdot
\sum_{k=1}^{k=n}\,\frac{1}{b_k}\leq \log\, \frac{1}{\phi(a)}\tag{1.2.1}
\]

\medskip

\noindent
\emph{Proof.}
On the imaginary axis we consider the intervals
\[
\ell_k=[e\cdot  b_k,e\cdot eb_{k+1}]\quad\colon \, k=1,\ldots,n-1
\quad\colon \, \ell_n=[eb_n,+\infty)\tag{i}
\]
Since $\log e^{-1}=-1$ it is clear that
(1.1) gives
\[
\log|\phi(iy)|\leq -k\quad\colon y\in \ell_k \tag{ii}
\]
Taking the negative intervals
$-\ell_k=[-e\cdot  b_{k+1},-e\cdot b_k]$ and
$-\ell_n=(-\infty, -eb_n$ we also have
\[
\log|\phi(iy)|\leq -k\quad\colon y\in -\ell_k \tag{iii}
\]
Moreover, since the maximum norm of $\phi$ is $\leq 1$ one has
\[
\log|\phi(iy)|\leq  0\quad\colon -b_1\leq y\leq b_1\tag{iv}
\]
Next, solving the Dirichlet problem we find the harmonic function $u$ in the open right
half-plane whose boundary values on
$(-b_1,b_1)$ is zero ,while
$u=-k$ in the the open intervals $\ell_k$ and $-\ell_k$ for every $k$.
The principle of harmonic majorisation applied to the subharmonic function 
$\log\,|\phi(z)|$ entails that
\[
\log\,|\phi(a)\leq u(a)\tag{v}
\]
Now we  evaluate $u(a)$ using Poisson's formula to represent
harmonic functions in the right half-plane.
For each $1\leq k\leq n-1$ we denote by $\theta_a(k)$ the
angle between the two vectors which join $a$ to the end-points
$ieb_k$ and $ieb_{k+1}$.
Computing
the area of the triangle with corner points
at $a,ieb_k,ieb_{k+1}$ the reader may check that
\[
\sqrt{a^2+e^2b_k^2}\cdot 
\sqrt{a^2+e^2b_{k+1}^2}\cdot \sin \theta_a(k)=
a\cdot e\cdot (b_{k+1}-b_k)\tag{vi}
\]
Finally, let $\theta_a(n)$ be the angle between the vector which joins $a$ with $ieb_n$ and
the vertical line $\{x=a{}$. The reader may check  with the aid of a figure that
\[
\sin \theta_a(n)=\frac{a}{\sqrt{a^2+e^2b_n^2}}\tag{vii}
\]
\medskip

\noindent
{{\bf{Exercise.}}
Confirm via Poisson's formula that
\[
u(a)=-\frac{2}{\pi}\cdot \sum_{k=1}^{k=n}\,k\cdot \theta_a(k)
\]

\noindent
Together with (v) it follows that
\[
\frac{2}{\pi}\cdot \sum_{k=1}^{k=n}\,k\cdot \theta_a(k)\leq 
\log\,\frac{1}{\phi(a)|}\tag{viii}
\]
The inequality $\sin t\leq t$ for every $t>0$ implies that
\[
\frac{2}{\pi}\cdot \sum_{k=1}^{k=n}\,k\cdot \sin (\theta_a(k))\leq 
\log\,\frac{1}{\phi(a)|}\tag{ix}
\]
Next we use (vi-vii) to
estimate 
$\{ \sin (\theta_a(k))\}$.
When $1\leq k\leq n-1$ we have from (vi)

\[
e^2\cdot  b_k\cdot b_{k+1}\cdot \sqrt{1+\frac{a^2}{e^2b_k^2}}
\cdot 
\sqrt{1+\frac{a^2}{e^2b_{k+1}^2}}\cdot \sin \theta_a(k)=
a\cdot e\cdot (b_{k+1}-b_k)\implies
\]
\[
e\cdot (1+\frac{a^2}{e^2b_1^2})\cdot \sin\,\theta_a(k)\leq a\cdot (\frac{1}{b_k}-\frac{1}{b_{k+1}})
\]
where the last inequality follows since $b_k\geq b_1$ for every $k$.
We conclude that the left hand side in (ix) majorizes
\[
\frac{2a}{e\pi\cdot (1+\frac{a^2}{e^2b_1^2})}\cdot
\sum_{k=1}^{k=n-1}\, k\cdot (\frac{1}{b_k}-\frac{1}{b_{k+1}})
+\frac{2}{\pi}\cdot n\cdot \sin \theta_a(n)
\]
Finally, (vii) gives
\[
\sin \theta_a(n)=\frac{a}{eb_n}\cdot \frac{1}{\sqrt{1+\frac{a^2}{e^2b_n^2}}}
\geq\frac{a}{eb_n}\cdot \frac{1}{1+\frac{a^2}{e^2b_1^2}}
\]
From this we see that the left hand side in (ix) majorizes
\[
\frac{2a}{e\pi\cdot (1+\frac{a^2}{e^2b_1^2})}\cdot
\bigl(\sum_{k=1}^{k=n-1}\, k\cdot (\frac{1}{b_k}-\frac{1}{b_{k+1}})+
n\cdot \frac{1}{b_n}\bigr)
\]
Abel's summation formula identifies the last term with
$\sum_{k=1}^{k=n}\,\frac{1}{b_k}$.
Hence we have proved the requested inequality
\[
\frac{2a}{e\pi\cdot (1+\frac{a^2}{e^2b_1^2})}\cdot
\sum_{k=1}^{k=n}\,\frac{1}{b_k}\leq \log\, \frac{1}{\phi(a)}\tag{x}
\]
\bigskip

\noindent
{\bf{1.3 A special case.}}
Assume in addition to (1.1) that
\[
\phi(a)\geq e^{-a}\quad\colon a>0\tag{1.3.1}
\]
This gives
\[
\log\, \frac{1}{\phi(a)}\leq a
\]
So here Theorem 1.2 after division with $a$ gives
\[
\frac{2}{e\pi\cdot (1+\frac{a^2}{e^2b_1^2})}\cdot
\sum_{k=1}^{k=n}\,\frac{1}{b_k}\leq 1\tag{1.3.2}
\]
Passing to the limit as $a\to 0$ it follows that
\[
\sum_{k=1}^{k=n}\,\frac{1}{b_k}\leq \frac{e\pi}{2}\tag{1.3.3}
\]

\newpage

\centerline{\emph{Proof of Theorem 0.2. Final part.}}


\bigskip

\noindent
We are given $f\in\mathcal F_n$ and put
\[
\phi(z)=\int_0^1\, e^{-zt}\cdot f(t)^2\, dt
\]
When $\mathfrak{Re} z\geq 0$ the absolute value $|e^{-zt}|\leq 1$
for all $t$ on the unit interval. The normalisation in (0.1.2)
implies that the maximum norm of $\phi$ is $\leq 1$.
Next, if $1\leq k\leq n$
the vanishing in (0.1.1) and partial integration give
\[
z^k\cdot \phi(z)=
\sum_{\nu=0}^{\nu=k}\, \binom{k}{\nu}\,\int_0^1\, f^{(\nu)}(t)
\cdot f^{(k-\nu)}(t)(t)\, dt\tag{i}
\]
The Cauchy-Schwarz inequality estimates  the absolute value of the right hand side by
\[
\sum_{\nu=0}^{\nu=k}\, \binom{k}{\nu}\cdot ||f^{(\nu)}||_2\cdot 
||f^{(k-\nu)}||_2\tag{ii}
\]
At this stage we use a wellknown resut from calculus  which
entails that
\[
||f^{(\nu)}||_2\leq ||f^{(k)}||_k\quad\colon 0\leq \nu\leq k
\]
and from this the reader can check that (ii) is majorised by
$2^k\cdot ||f^{(\nu)}||^2_k$.
Hence 
\[
|z|^k\cdot |\phi(z)|\leq 2^k\cdot (||f^{(k)}||_2)^2\quad\colon\, k=1,2,\ldots\tag{iii}
\]
Put
\[
b_k= 2\cdot (||f^{(k)}||_2)^{\frac{2}{k}}\implies
|z|^k\cdot |\phi(z)|\leq b_k^k\tag{iv}
\]
Next, if $a>0$ we have
\[
\phi(t)= 
\phi(z)=\int_0^1\, e^{-at}\cdot f(t)^2\, dt\geq e^{-a}\cdot 
\int_0^1\, f(t)^2\, dt=e^{-a}
\]
where the last equality holds by the normalsation in (0,xx).
Now the  special case in (1.3.3)  gives
\[
\sum_{k=1}^{k=n}\,\frac{1}{b_k}\leq \frac{e\pi}{2}\tag{v}
\]
Finally, we have the trivial inequality
\[
||f^{(k)}||_2\leq
\max_{0\leq t\leq 1}\,|f^{(k)}(t)|
\]
It follows from (v) that
\[ 
\frac{1}{b_k}\geq \frac{1}{ 2\cdot C_k(f)}
\]
and hence (v) gives the requested inequality in Theorem 0.2


\newpage

\centerline{\bf{Proof of Theorem 0.3.4}}


\bigskip

\noindent
For each $N$ we denote by $J_*(N)$ the minimum in the variational problem from (xxx).
Among the competing functions we can choose
$f$  and hence
\[
J_*(N)\leq J_N(f)
\]
Now there exist constants $C$ and $M$ from (0.3.1)
which entails that
\[
 J_N(f)\leq M\cdot \sum_{k=0}^N\, (\log (k+2))^{-k} C^{2k}
\]
Since
$\log (k+2)$ tends to $+\infty$, it is clear that the series
\[
\sum_{k=0}^\infty\, (\log (k+2))^{-k} C^{2k}<\infty
\]
We conclude that there exists a constant
$J_*$ such that
\[
J_*(N)\leq J_*\quad\colon\, N=1,2,\ldots\tag{i}
\]
So if $m$ is some positive integer and $N\geq m$ we have
\[
\sum_{k=0}^{k=m}\, (\log (k+2))^{-2}\int_0^1\, f_N^{(k)}(x)^2\, dx
\leq J_N\leq J_*\tag{ii}
\]
Now we recall the classic resut due to Arzela-Ascoli
which implies that bounded sets in $H_m$
give relatively compact subsets of
$C^{m-1}[0,1]$.
Since (ii) hold for each $m$, it follows
by a standard diagonal procedure which is left to the reader 
that we can find a subsequence $\{g_\nu=f_{N_\nu}\}$
such that
the sequence of derivatives
$\{g_\nu^{(m)}\}$ converge uniformly for every $m$, i.e
$g_\nu\to g_*$ holds in the space $C^\infty[0,1]$.
Next, by (0.3.2) we have for each fixed integer $k\geq 0$:
\[
f^{(k)}(0)=f_N^{(k)}(0)\quad\colon N\geq k+1
\]
From this it follows that
\[
f^{(k)}(0)=g_*^{(k})(0)\quad\colon k=0,1,2\dots\tag{iii}
\]
Hence the $C^\infty$-function
\[
\phi=f-g_*
\]
 is flat at $x=0$.
Next, for a fixed integer $k$
the uniform bound in
(ii) gives
\[
\int_0^1\, \phi^{(k)}(x)^2\, dx\leq
 J_*\cdot (\log (k+2))^{2k}\cdot (k !)^2\tag{iv}
\]
Moreover, for each $0<x\leq 1$ the Cauchy-Schwartz inequality gives
\[
\phi^{(k)}(x)= \int_0^x\, \phi^{(k+1)}(t)\,dt\leq
\sqrt{\int_0^1\, \phi^{(k)}(x)^2\, dx}
\]
and since (iv) hold for every $k$ it follows that
\[
\max_x\, |\phi^{(k)}(x)|\leq  J_*\cdot (\log (k+2))^k\cdot k !
\]
Since $k !\leq k^k$
this entails that
\[ 
\mathcal C_k(\phi)=\leq J_*^{\frac{1}{k}}\cdot  k\cdot (\log (k+2))
\]
Since the series $\sum_{k=1}^\infty\, \frac{1}{k\log k}$
is divergent we conclude that
\[
\sum_{k=1}^\infty\, \frac{1}{\mathcal C_k(\phi)}=+\infty
\]
Hence Denjoy's result in xxx proves that $\phi$ is identically zero.
\medskip

\noindent
\emph{Final part of the proof.}}
We have proved that
$\phi=0$ which means that

\[
\lim _{k\to\infty}\, f_{N_k}=f
\]
where the convergence holds in the space
$C^\infty[0,1]$.
Finally, by the compactness above this limit for 
an arbitrary convergent subsequence entails  that
the whole sequence $\{f_N\}$ converges to $f$ which finishes the proof
of 
Theorem 0,xx.









\newpage







\end{document}
