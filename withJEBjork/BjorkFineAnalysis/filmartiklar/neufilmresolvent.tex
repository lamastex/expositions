


\documentclass{amsart}

\usepackage[applemac]{inputenc}

\addtolength{\hoffset}{-12mm}
\addtolength{\textwidth}{22mm}
\addtolength{\voffset}{-10mm}
\addtolength{\textheight}{20mm}

\def\uuu{_}

\def\vvv{-}

\begin{document}



\centerline{\bf{0.5.5 Adjoint operators and closed extensions.}}
\medskip


\noindent
Let $T$ be densely defined. 
But for the monent we do not assune that it is closed.
In the dual space $X^*$ we have   the family 
of  vectors $y$ for which   there exists a constant
$C(y)$ such that
\[
|y(Tx)|\leq C(y)\cdot ||x||\quad\colon x\in\mathcal D(T)\tag{i}
\]
It is clear that the set of such $y$-vectors is a subspace of $X^*$.
Moreover, when (i) holds the density of $\mathcal D(T)$
gives  a unique vector 
$T^*(y)$ in $X^*$ 
such that
\[
y(Tx)= T^*(y)(x)\quad\colon x\in\mathcal D(T)\tag{ii}
\]
One refers to $T^*$ as the adjoint operator of $T$ whose
domain of definition is denoted by
$\mathcal D(T^*)$.



\noindent
{\bf{Exercise.}}
Show that the graph of $T^*$ is closed in
$X^*\times X^*$.
However,
$\mathcal D^*(T)$
is in general not a dense subspace of $X^*$.
See � xx for an example.



\bigskip

\noindent
{\bf{ Closed extensions.}}
There may exist  several closed operators $S$
with the property that
\[
\Gamma(T)\subset \Gamma(S)
\]
When this holds
we refer to
$S$ as  a closed extension of $T$. Notice that
the inclusion above is strict if and only if
$\mathcal D(S)$ is strictly larger than
$\mathcal D(T)$.

\medskip

\noindent
{\bf{Exercise.}}
Use the density of
$\mathcal D(T)$ to show that 
\[ 
T^*= S^*
\]
hold for every closed extension $S$ of $T$.
\medskip

\noindent
{\bf{The case when
$\mathcal D(T^*)$ is dense.}}
Let $T$ be densely defined and assume that its adjoint has a dense domain of definition.
In this situation the following holds:


\medskip




\noindent
{\bf{0.5.6 Theorem.}}
\emph{If $\mathcal D(T^*)$ is dense then there exists
a closed operator $\widehat{T}$ whose graph is the closure of
$\Gamma(T)$.}
\medskip

\noindent
\emph{Proof.}
Consider the graph 
 $\Gamma(T)$
and let 
$\{x_n\}$ and $\{\xi_n\}$
be two sequences in $\mathcal D(T)$ which both converge to a point
$p\in X$ while 
$T(x_n)\to y_1$ and $T(\xi_n)\to y_2$ hold 
for some pair $y_1,y_2$. We must rove that $y_1=y_2$.
To achieve this we take some
$x^*\in \mathcal D(T^*)$ which gives
\[
x^*(y_1)= \lim \,x^*(Tx_n)= \lim\,T^*(x^*)(x_n)= T^*(x^*)(p)
\]
In the same way we get $x^*(y_2)=T^*(x^*)(p)$. 
Now the density of $\mathcal D(T^*)$ gives $y_1=y_2$
which proves that
the closure of $\Gamma(T)$ is a graphic subset of $X\times X$
and gives the closed   operator $\widehat{T}$
with 
\[
\Gamma(\widehat{T})=
\overline{\Gamma(T)}
\]
\bigskip


\noindent
{\bf{ The case when $X$ is reflexive.}}
Assume this and
let $T$ be a densely defined and closed operator. Suppose in addition
that $T^*$ also is  densely defined. 
Now we can construct the adjoint of $T^*$ which is denoted by $T^{**}$.
Since $X$ is reflexive it follows that $T^{**}$ is a 
closed and densely  defined operator on $X$.
If $x\in\mathcal D(T)$ and $y\in\mathcal D(T^*)$
we have the vector $\widehat{x}\in X^{**}$
and
\[
\widehat{x}(T^*(y))=T^*(y)(x)= y(T(x))
\]
From this it is clear that
$\widehat{x}\in\mathcal D(T^{**})$
and one has the equality
\[
T^{**}(\widehat{x})= T(x)
\]
Hence 
the graph of $T$ is contained in that of $T^{**}$, i.e.
$T^{**}$ is a closed extension of $T$.

\bigskip

\noindent
{\bf{0.5.7 The spectrum of $T^*$.}}
Let $X$ and $T$ be as above. Then
one has the inclusion
\[
 \rho(T) \subset \rho(T^*)\tag{*}
\]
\medskip

\noindent
\emph{Proof.}
By translations it suffices to show that if
the origin belongs to $\rho(T)$ then it also belongs to
$\rho(T^*)$. So now  the resolvent 
$R_T(0)$ exists which means  that
$T$ is surjective
and there is a constant $c>0$ such that
\[
||x||\leq c^{-1}\cdot ||Tx||\quad\colon x\in\mathcal D(T)\tag{i}
\]
Consider some
$y\in\mathcal D(T^*)$ of unit norm. Since $T$ is surjective
we find 
$x\in\mathcal D(T)$ with 
$||Tx||=1$ and
\[ 
|y(Tx)|\geq 1/2\tag{ii}
\]
Now
\[ 
y(Tx)= T^*(y)(x)\tag{iii}
\]
and from (i) we have
\[
||x||\leq c^{-1}\cdot ||Tx||= c^{-1}\tag{iv}
\]
Then (ii) and (iv) entail that 
\[ 
||T^*(y)||\geq c/2
\]
This proves that
\[ 
||T^*(y)||\geq  c/2\cdot ||y||\quad\colon y\in\mathcal D(T^*)\tag{v}
\]
Hence the origin belongs to $\rho(T^*)$ if we prove that
$T^*$ has a dense range.
If the density  fails  there exists a non-zero linear functional
$\xi\in X^{**}$ such that
\[
\xi(T^*(y))=0\quad\colon y\in\mathcal D(T^*)
\]
Since $X$ is reflexive we have $\xi=\mathfrak{i}_X(x)$ for some
vector $x$ and obtain
\[
y(Tx)=0\quad\colon y\in\mathcal D(T^*)
\]
The density of $\mathcal D(T^*)$ gives $Tx=0$ which contradicts the hypothesis that
$T$ is injective and (*) follows.
\medskip

\noindent
{\bf{0.5.8 The case when $X$ is a Hilbert space.}}
In this case we shall prove that
when both $T$ and $T^*$ are closed and densely defined, then one has the equality
\[
\sigma(T)\subset\sigma(T^*)
\]
We refer to  � xx for the proof.

\bigskip






\noindent
{\bf{0.5.3 The case when resolvent operators are compact.}}
Let $T$ be such that
$R_T(\lambda_0)$ is a compact operator for some
resolvent value. We assume of course that
the Banach space $X$ is not finite dimensional.
In �� we shall learn that the spectrum of a compact operator
always contains zero and outside the origin the spectrum is a discrete set
with a sole cluster point at the origin.
From (0.5.1) it follows that
$\sigma(T)$ is a discrete set in
${\bf{C}}$, i.e. its intersection with
every disc
$\{|\lambda|\leq R\}$ is finite.
\bigskip

\noindent
{\bf{0.5.4 Remark.}}
In � xx we shall learn that if $S$ is a compact operator
then $S\circ  U$ and $U\circ S$ are compact for every bounded
operator $U$.
Applying  Neumann's equation (*) in (0.3)
it follows  that
if one resolvent operator
$R_T(\lambda_0)$ is  compact, then all resolvents  of $T$ are compact.
 
\bigskip








\medskip


\noindent
{\bf{0.1 The class $\mathcal I(X)$.}}
It consists
of bounded linear operators
$R$ on $X$ with the property that
$R$ is injective and the range
$R(X)$ is a dense subspace of $X$.
We do not exclude the possibility that
$R$ is surjective.
Each such operator
$R$ gives  a densely defined
operator $T$ as follows:
If $x\in R(X)$ the injectivity of $R$ gives a unique
vector
$\xi\in X$ such that
$R(\xi)=x$ and we set
\[
T(x)=\xi\tag{i}
\]
It means that the composed operator $T\circ R=E$, where
$E$ is the identity operator on $X$.
Here the domain of defintion for $T$ is equal to the range $R(X)$ 
and this dense subspace of $X$ is denoted by
$\mathcal D(T)$. By construction we have
\[ 
R\circ T(x)=x\quad\colon x\in\mathcal D(T)
\]
Next, the bounded operator $R$ has a finite operator norm
$||R||$ and (i) entails that
\[
||x|||\leq ||R||\cdot ||T(x)||\tag{ii}
\]
Thus, with $c= ||R||^{-1}$ one has
\[
||T(x)||\geq c\cdot ||x|||\quad\colon x\in\mathcal D(T)\tag{iii}
\]
\medskip

\noindent
{\bf{The graph $\Gamma(T)$}}.
It is  the subset of $X\times X$ given by
$\{(x,Tx)\,\colon x\in\mathcal D(T)\}$.
The construction of $T$ gives
\[
\Gamma(T)=\{(Rx,x)\colon x\in X\}
\]
Since
$R$ is a bounded inear operator it is clear  that
the last set is closed in $X\times X$, i.e.
$\Gamma(T)$ is closed which means
that
$T$ is a densely defined and closed
linear operator on $X$.
The inequality (iii) shows that $T$ is injective and
since
\[ 
T(Rx))=x\quad\colon x\in X
\]
the range of $T$ is equal to $X$.
\medskip


\noindent
{\bf{A converse result.}}
Assume that $T$ is a densely defined and closed operator
such that
(iii) holds and in addition the range of $T$ is dense in $X$.
It turns out that this gives the equality
\[
T(\mathcal D(T))=X\tag{1}
\]
For if $y\in X$ the density of the range  gives
a sequence
$\{x_n\}$ in $\mathcal D(T)$ such that
\[
\lim_{n\to \infty}\, ||T(x_n)-y||=0\tag{2}
\]
Now
\[
||x_n-x_m||\leq c^{-1}\cdot ||T(x_n)-T(x_m)||
\] 
and (2) entails that
$\{T(x_n)\}$ is a Cauchy sequence. Since $X$ is a Banach space
it follows that
$\{x_n\}$ converges to a limit vector $x$.
Now $\Gamma(T)$ is closed which implies that 
$(x,y)$ belongs to the graph, i.e. $x\mathcal D(T)$ and $T(x)=y$
which proves (1).
\medskip


\noindent
{\bf{Exercise.}}
Let $T$ be densely defined and closed where
(iii) holds and $T(\mathcal D(T))=X$. 
Show that there exists a unique bounded operator
$R\in\mathcal I(X)$
such that $T$ is the attached operator as in 0.1 above.

\newpage





\centerline
{\bf{0.2 Spectra of densely defined operators.}}
\medskip




\noindent
Let $T$ be a densely defined and closed linear operator.
Each complex number
$\lambda$ gives  the densely defined
operator $\lambda\cdot E-T$.
We say that
$\lambda$ is a resolvent value of $T$
if
$\lambda\cdot E-T$ is surjective and there exists 
a positive constant $c$ such that
\[
||\lambda\cdot x-T(x)||\geq c\cdot ||x||
\]
The set of resolvent values is denoted by
$\rho(T)$. Its closed complement is called the spectrum of $T$ and we put
\[
\sigma(T)={\bf{C}}\setminus \rho(T)
\]
Each
$\lambda\in\rho (T)$   gives a unique bounded operator
$R_T(\lambda)\in\mathcal I(X)$ such that
\[
(\lambda\cdot E-T)\circ R_T(\lambda)(x)=x
\] 
Since $\mathcal D(T)= \mathcal D(\lambda\cdot E-T)$
it follows that the range of $R_T(\lambda)$ is equal to $\mathcal D(T)$.
\medskip


\noindent
{\bf{0.2.1 Definition.}}
\emph{Tbe family
$\{R_T(\lambda)\,\colon\,\lambda\in \rho(T)\}$
are called Neumann's resolvents  of $T$.}

\medskip

\noindent
{\bf{An example.}}
Let $X$ be the Hilbert space $\ell^2$ whose vectors are  complex sequences
$\{c_1,c_2,\ldots\}$ for which $\sum\ |c_n|^2<\infty$.
We have  the dense subspace $\ell^2_*$ vectors such that
$c_n\neq 0$ only occurs for finitely many integers $n$.
If $\{\xi_n\}$ is an arbitrary sequence of complex numbers
there exists the densely defined operator $T$ on $\ell^2$
which
sends every sequence  vector $\{c_n\}\in \ell_*^2$ to
the vector $\{\xi_n\cdot c_n\}$.
If $\lambda$ is a complex number 
the reader may check that
(i) holds in (0.0.1) if and only if
there exists a  constant $C$ such that
\[
|\lambda-\xi_n|\geq C\quad\colon n=1,2,\ldots\tag{v}
\]
Thus, $\lambda\cdot E-T$ has a bounded left inverse if and ony if
$\lambda$ belongs to the open complement of the closure of the
set
$\{\xi_n\}$ taken in the complex plane.
Moreover,   if (v) holds then
$R_T(\lambda)$ is the bounded
linear operator on $\ell^2$
which sends $\{c_n\}$ to 
$\{\frac{1}{\lambda-\xi_n}\cdot c_n\}$.
Since
every closed subset of ${\bf{C}}$
is equal to the closure of a denumerable set of points our
construction  shows that the spectrum of a 
densely defined
operator $\sigma(T)$ can be an arbitrary closed set in
${\bf{C}}$. 

\medskip








\newpage




\centerline {\bf{� 9. Neumann's  resolvent operators}}
\medskip



\noindent
Throughout $X$ denotes a complex Banach space.
 densely defined linear operator on $X$
 isa liner map
 \[ T\colon\mathcal D(T)\to X
 \] 
 where $\mathcal D(T)$ is a dense subspace of $X$, called the domain of definition for $T$.
To each such $T$ we associate the graph
\[ 
\Gamma(T)= \{(x,Rx)\,\colon\, x\in \mathcal D(T)\}
\]
So $\Gamma(T)$ is a subspace of the product $X\times X$ and 
if this graph is a closed subspace we say for brevity that
$T$ is closed. From now on $T$ is densely defined and closed.
Let $E$ be the identity operator on $X$. Then  each complex number
$\lambda$ gives the  operator
\[
\lambda\cdot E-T
\]
whose domain of definition is equal to $\mathcal D(T)$.
\medskip


\noindent
{\bf{Definition.}}
\emph{A complex number $\lambda$ is called a resolvent value of $T$
if there exists a constant $c>0$ such that}
\[ 
||\lambda\cdot x-Tx||\geq c\cdot ||x||\quad\colon\quad x\in\mathcal D(T)\tag{*}
\]
\emph{and in addition the range of $\lambda\cdot E-T$ is equal to $X$.}
\medskip



\noindent
The set of resilient values is denoted by $\rho(T)$.
There exist ugly operators for which
$\rho(T)=\emptyset$. We shall ignore this car and assume that
$T$ has at least one resolvent value.
\medskip

\noindent
{\bf{The resolvent operator $R_T(\lambda)$.}}
Let $\lambda\in \rho(T)$ be given.
Sinc the range of $\lambda\cdot E-T$ is equal to $X$ we find for
every $y\in X$ some $x(y)\in\mathcal D(T)$ such that
\[ 
\lambda\cdot x(y)-Tx(y)=y
\]
and here (*) entails that $x$ is unique and
\[
||x||\leq c^{-1}||y||
\]
Hence there exists a bounded liner operator $S$ on $X$ with operator norm
$\leq c^{-1}$ such that
\[ 
Sy= x(y)
\]
We put $S=R_T(\lambda)$ and it is called Neumannn's resovlent operator
of $T$ for at the given resolvent value $\lambda$.
The construction shows that the range
\[ 
R_T(\lambda)(X)= \mathcal D(T)
\]
and passing to composed operators we have
\[
R\circ T(x)=x\,\colon\, x\in \mathcal D(T)\,\,\&\,\,
T\circ R(x)=x\,\colon\, x\in X
\]
\medskip

\centerline{\bf{0.3 Neumann's equation.}}

\medskip


\noindent
Let $T$ as above be densely defined and closed where $\rho(T)$ is non-empty. Then one has:
\medskip


\noindent
\emph{For each pair 
$\lambda\neq\mu$  in $\rho(T)$ the operators 
$R_T(\lambda)$ and $R_T(\mu)$ commute
and}
\[
R_T(\mu)R_T(\lambda)=\frac{R_T(\mu)-R_T(\lambda)}{\lambda-\mu}\tag{*}
\]


\noindent
\emph{Proof.}
Notice that
\[
(\mu\cdot E-T)\cdot \frac{R_T(\mu)-R_T(\lambda)} {\lambda-\mu}
=
\]
\[
\frac{E}{\lambda-\mu}-(\mu-\lambda)\cdot\frac{R_T(\lambda)}{\lambda-\mu}-
(\lambda\cdot E-T)\cdot \frac{R_T(\lambda)}{\lambda-\mu}=R_T(\lambda)\tag{i}
\]
Multiplying to the left by
$R_T(\mu)$
gives (*).


\newpage


\centerline {\bf{0.4 Neumann series.}}
\medskip


\noindent
If  $\lambda_0\in\rho(T)$  we
construct the operator valued series
\[
S(\zeta)= R_T(\lambda_0)+ \sum_{n=1}^\infty
(-1)^n\cdot \zeta^n\cdot R_T(\lambda_0)^{n+1}\tag{1}
\]
where the series 
converges in the Banach space of bounded linear operators when
\[
|\zeta|<\frac{1}{||R_T(\lambda_0)||}
\]
Next, one has 
\[ 
(\lambda_0+\zeta-T)\cdot S(\zeta)=
(\lambda_0-T)\cdot S(\zeta)+\zeta S(\zeta)=
\] 
\[
E+ \sum_{n=1}^\infty
(-1)^n\cdot \zeta^n\cdot R_T(\lambda_0)^{n+1}+\zeta S(\zeta)
\]
and thanks to the alternating signs the last term is reduced to $E$.
We conclude that
\[
S(\zeta)=R_T(\lambda_0+\zeta)
\] 
give resolvent operators. This the open disc of radius 
$\frac{1}{||R_T(\lambda_0)||}$ centered at $\lambda_0$ stays in $\rho(T)$. Hence 
the set of reovlent values for $T$ is open.
Moreover, the
 operator-valued  function
\[
\lambda\mapsto R_T(\lambda)
\]
is an
analytic function in
$\rho(T)$.
For if  $\lambda\in \rho(T)$ we can pass to the limit as 
$\mu\to \lambda$ in Neumann's equation
and conclude that the complex derivative exists and is given by
\[
\frac{d}{d\lambda}(R_T(\lambda)=-R^2_T(\lambda)\tag{**}
\]
Thus, Neumann's resolvent operator satisfies a specific differential equation for
every densely defined and closed operator $T$
with a non-empty resolvent set.

\medskip




\centerline{\bf{0.5 The position of $\sigma(T)$.}}

\medskip



\noindent
Assume that $\rho(T)\neq\emptyset$.
For a pair of resolvent values of $T$ we
can  write Neumann's equation in the form
\[
R_T(\lambda)(E+(\lambda-\mu)R_T(\mu))= R_T(\mu)\tag{1}
\]
Keeping $\mu$ fixed we conclude that
$R_T(\lambda)$ exists if and only if
$E+(\lambda-\mu)R_T(\mu))$ is invertible. This gives 
the set-theoretic equality
\[
 \sigma(T)= \{\lambda\,\colon \frac{1}{\mu-\lambda}\in
\sigma(R_T(\mu))\} \tag{0.5.1}
 \]

 
 \noindent
Hence one  recovers $\sigma(T)$ via the spectrum of
any given  resolvent operator. Notice that (0.5.1) holds  even
when
the open component
of $\sigma(T)$ has several connected components.

\medskip

\noindent
{\bf{0.5.2 Example.}}
Suppose that $\mu=i$ and that 
$\sigma(R_T(i))$ is contained in a circle
$\{|\lambda+i/2|=1/2\}$.
If $\lambda\in\sigma(T)$ the inclusion (0.0.5.1) gives some
$0\leq\theta\leq 2\pi$ such that 
\[
\frac{1}{i-\lambda}=-i/2+ 1/2\cdot e^{i\theta}\implies
1-i\cdot e^{i\theta}=\lambda(e^{i\theta}-i)
\]
The last equation entails that
\[
\lambda=\frac{2\cdot \cos \theta}{|e^{i\theta}-i|^2}
\]
and hence $\lambda$ is real.

\newpage





\centerline{\bf{0.6 Operational calculus.}}
\medskip


\noindent
Let $T$ be a densely defined and closed operator on a Banach space $X$.
To each pair $(\gamma,f)$ where
$\gamma$ is a rectifiable Jordan
arc contained in
${\bf{C}}\setminus \sigma(T)$
and $f\in C^0(\gamma)$, there exists the bounded linear operator
\[ 
T_{(\gamma,f)}=\int_\gamma\, f(z)R_T(z)\, dz\tag{0.6.1}
\]
The integral is calculated
via a Riemann sum
where the integrand has values in the Banach space of bounded
linear operators on $X$. 
More precisely, let $s\mapsto z(s)$ be a parametrisation with respec to
arc-length. If $L$ is the arc-length of $\gamma$ we 
get Riemann sums
\[
\sum_{k=0}^{k=N-1} \, f(z(s_k))\cdot (z(s_{k+1})-z(s_k))\cdot 
(s_{k+1}-s_k)\cdot R_T(z(s_k))
\]
where $0=s_0<s_1<\ldots s_N=L$ is a partition of
$[0,L]$.
These Riemann sums converge
to a limit when
$\{\max\, (s_{k+1}-s_k\}\to 0$ with respect to the operator norm
and give the $T$-operator in (0.6.1). The triangle inequality entails that
\[
T_{(\gamma,f)}\leq L\cdot |f|_\gamma\cdot \max_{z\in\gamma}\, ||R_T(z)||
\]
where $|f|_\gamma$ is the maximum norm of $f$ on $\gamma$..
\medskip

\noindent
Neumann's equation in (0.3)
entails that
$R_T(z_1)$ and $R_T(z_2)$
commute for all pairs $z_1,z_2$ on $\gamma$. It follows that
if 
$g$ is another function in
 $C^0(\gamma)$
then  the operators $T_{f,\gamma}$
and $T_{g,\gamma}$
commute. Moreover, for each $f\in C^0(\gamma)$
the reader may verify that
the closedeness of $T$
implies that
the range of
$T_{f,\gamma}$ is contained in
$\mathcal D(T)$ and one has
\[
T_{f,\gamma}\circ T(x)= T\circ T_{f,\gamma}(x)
\quad\colon x\in \mathcal D(T)
\]


 

\medskip

\noindent
Next, let $\Omega$ be an open set of class $\mathcal D(C^1)$, i.e.
$\partial\Omega$ is a finite union of
closed differentiable Jordan curves. 
When
$\partial\Omega\,\cap \sigma(T)=\emptyset$
we construct line integrals as in (0.6.1)
for continuous functions on
the boundary.
Consider the algebra $\mathcal A(\Omega)$
of analytic functions in
$\Omega$ which extend to be continuous on the closure.
Each $f\in \mathcal A(\Omega)$
gives the operator
\[ 
T_f=
\int_{\partial\Omega}\, f(z)R_T(z)\, dz\tag{0.6.2}
\]
\medskip

\noindent
{\bf{0.6.3 Theorem.}}
\emph{The map $f\mapsto T_f$ is an algebra homomorphism from
$\mathcal A(\Omega)$ into a commutative algebra of bounded
linear operators on $X$ whose  image is  a commutative algebra of 
bounded linear operators
denoted by $T(\Omega)$.}

\medskip

\noindent
{\emph{Proof.}}
Let $f,g$ be a pair in $\mathcal A(\Omega)$.
To show that
$T_{gf}= T_f\circ T_g$
we consider a slightly  smaller open
set $\Omega_*$ which again is of class $\mathcal D(C^1)$
and each of it  bounding Jordan curve is close to one 
boundary curve in
$\partial\Omega$ and $\Omega\setminus \Omega_*$ does not
intersect $\sigma(T)$.
By Cauchy's theorem we can shift the integration to
$\partial\Omega_*$ and get
\[
T_g=
\int_{\partial\Omega_*}\, g(z)R_T(z_*)\, dz_*\tag{i}
\]
where we use  $z_*$ to indicate that integration takes place
along
$\partial\Omega_*$. Now
\[
T_f\circ T_g=
\iint_{\partial\Omega_*\times\partial\Omega}\, f(z)g(z_*)R_T(z)\circ
R_T(z_*)\, dz_*dz\tag{ii}
\]
Neumann's equation
(*) from  (0.0.3) entails that  the right hand side in
(ii) becomes
\[
\iint_{\partial\Omega_*\times\partial\Omega}\, \frac{f(z)g(z_*)R_T(z_*)}{z-z_*}
\, dz_*dz+
\iint_{\partial\Omega_*\times\partial\Omega}\, \frac{f(z)g(z_*)R_T(z)}{z-z_*}
\, dz_*dz=A+B
\tag{iii}
\]
\medskip

\noindent
Here $A$ is  evaluated
by first integrating with respect to $z$ and 
Cauchy's theorem gives
\[
f(z_*)=\frac{1}{2\pi i}\cdot 
\iint_{\partial\Omega }\frac{f(z)}{z-z_*}\quad\colon z_*\in\partial\Omega_*
\, dz
\]
It follows that
\[ 
A=\frac{1}{2\pi i}\cdot
\iint_{\partial\Omega_*\times\partial\Omega}\, f(z_*)g(z_*)R_T(z_*)\, dz_*
=T_{fg}
\]
Next, $B$ is evaluated when we first integrate with respect to $z_*$. Here
\[
\iint_{\partial\Omega}\frac{g(z_*)}{z-z_*}\quad\colon z\in\partial\Omega
\]
which entails that $B=0$ and the theorem  follows.
\medskip

\centerline {\bf{0.7 Spectral gap sets.}}
\medskip


\noindent
Let
$K$ be a compact subset
of $\sigma(T)$ 
such that
$\sigma(T)\setminus K$ is a closed
set in ${\bf{C}}$.
This implies that if $V$ is an open  neighborhood of $K$, then
there exists a relatively compact subdomain
$U\in \mathcal D(C^1)$ which contains
$K$ as a compact subset.
To every such domain $\Omega$ we can apply Theorem 0.0.6.3.
If $U_*\subset U$ for a pair of such domains
we can restrict functions in
$\mathcal A(U)$ to $U_*$
which  yields an algebra homomorphism
\[
\mathcal T(U)\to \mathcal T(U_*)
\]
Next, denote by
$\mathcal O(K)$ the algebra of germs of analytic functions
on $K$.
So each $f\in \mathcal O(K)$ 
comes from
some analytic function in a domain $U$ as above.
The resulting operator $T_U(f)$ depends on the germ $f$ only.
In fact, this follows because if
$f\in \mathcal A(U)$ and
$U_*\subset U$ is a similar 
$\mathcal D(C^1)$-domain which again contains
$K$, then
Cauchy's vanishing theorem from � xxx is applied
to $f(z)R_T(z)$ in $U\setminus \bar U_*$ and entails that
\[
\int_{\partial U_*}\, f(z)R_T(z)\, dz=
\int_{\partial U}\, f(z)R_T(z)\, dz
\]
Hence there exists an algebra homorphism
from $\mathcal O(K)$ into
bounded linear operators on $X$
whose image is denoted by $\mathcal T(K)$.
The identity in $\mathcal T_K$
is denoted
by $E_K$ and called 
the spectral projection operator
attached to the compact set $K$ in
$\sigma(T)$. By this construction one has
\[
E_K=\frac{1}{2\pi i}\cdot \int_{\partial U}\, z\cdot R_T(z)\, dz
\] 
for every open domain $U$ around $K$ as above.


\medskip

\noindent
{\bf{0.0.6.4 The operator $T_K$.}} When $K$ is a compact spectral gap set of $T$
we set
\[
 T_K=TE_K
\]
This   bounded linear operator is given
by
\[
\frac{1}{2\pi i}\cdot \int_{\partial U}\, z\cdot R_T(z)\, dz
\] 
where $U$ is a domain as above containing $K$.
\medskip

\noindent
{\bf{0.0.6.4.1}} \emph{Identify
$T_K$ with a densely defined operator on
the space $E_K(X)$. Then one has the equality}
\[
\sigma(T_K)=K
\]
\emph{Proof.}
If $\lambda_0$ is outside $K$ we can choose
$U$ so that $\lambda_0$ is outside $\bar U$
and construct the operator
\[
S=\frac{1}{2\pi i}\cdot \int_{\partial U}\, \frac{1}{\lambda_0-z}\cdot R_T(z)\, dz
\]
The operational calculus gives
\[
S(\lambda_0 E_K-T)= E_K
\]
here $E_K$ is the identity operator on
$E_K(X)$ which shows that
$\sigma(T_K)\subset K$.


\bigskip


\noindent
{\bf{0.0.6.5  Discrete spectra.}}
Consider 
a spectral set  reduced to a singleton set
$\{\lambda_0\}$, i.e.
$\lambda_0$ is an isolated point in
$\sigma(T)$. The associated spectral projection is denoted by
$E_T(\lambda_0)$ and
expressed 
\[
E_T(\lambda_0)=
\frac{1}{2\pi i}\cdot \int_{|\lambda-\lambda_0|=\epsilon}\,
R(\lambda)\, d\lambda
\]
for all sufficiently small $\epsilon$.
Now
$R_T(\lambda)$ is an analytic function defined
in some punctured disc
$\{0<\lambda-\lambda_0|<\delta\}$
with  a Laurent expansion
\[
R_T(\lambda)=\sum_{-\infty}^\infty\, 
(\lambda-\lambda_0)^k\cdot B_k
\]
where $\{B_k\}$ are bounded linear operators
obtained  by residue formulas:
\[
B_k= \frac{1}{2\pi i}\cdot
\int_{|\lambda|=\epsilon}\,  
\frac{R_T(\lambda)}{(\lambda-\lambda_0)^{k+1}}\, d\lambda
\quad\colon\,\epsilon<\delta
\]

\noindent
{\bf{Exercise.}}
Show
that
$R_T(\lambda)$ is meromorphic, i.e. 
$B_k=0$ hold when $k<<0$,
if and only if there
exists a constant $C$ and some integer $M\geq 0$ such that
the operator norms satisfy
\[
||R_T(\lambda)||\leq C\cdot |\lambda-\lambda_0)^{-M}
\]
\medskip

\noindent
Suppose now that
$R_T$ has a pole of some order $M\geq 1$ which gives
an  expansion
\[
R_T(\lambda)=\sum_1^M\,
\frac{B_{-k}}{(\lambda-\lambda_0)^k}+
\sum_0^\infty\, 
(\lambda-\lambda_0)^k\cdot B_k
\]
Here $B_{-1}=E_T(\lambda_0)$ and if $M\geq 2$
the negative indexed operators satisfy
\[
B_{-k}= B_{-k}E_T(\lambda_0)\quad 2\leq k\leq M
\]
In the case of a simple pole, i.e.
when $M=1$ the operational calculus gives
\[
(\lambda_0E-T)E_T(\lambda_0)=
\lim_{\epsilon\to 0}
\frac{1}{2\pi i}\cdot \int_{|\lambda-\lambda_0|=\epsilon}\,
(\lambda_0-\lambda)R(\lambda)\, d\lambda=0
\]
which implies
hat the range of the projection operator
$E_T(\lambda_0)$ is equal to the kernel of
$\lambda_0\cdot E-T$.
\medskip

\noindent
{\bf{0.0.6.6 The case $M\geq 2$}}.
Now one has 
a
a non-decreasing family of subspaces
\[
N_k(\lambda_0)= \{ x\colon (\lambda_0E-T)^k(x)=0\}
\quad\colon 1\leq k\leq M
\]


\noindent
Let us analyzie the special case
when the range of 
$E_T(\lambda_0)$ has   finite dimension.
Here  the operator $T(\lambda_0)= TE_T(\lambda_0)$
acts on this finite dimensional vector space 
and the $B$-matrices with negative indices can be expressed 
as in linear algebra via 
a Jordan decomposition of $T(\lambda_0)$.
More precisely Jordan blocks of size
$>1$ may occur which
occurs of the smallest positive integer $m$ such that
\[
(\lambda_0E-T)^m(x)= \quad\colon
x\in E_T(\lambda_0)(X)
\]
is strictly larger than one.
Moreover, $E-E_T(\lambda_0)$ is a projection operator
and
one has a direct sum decompostion
\[
X=E_T(\lambda_0)(X)\oplus E-E_T(\lambda_0)
\]
Here $V=E-E_T(\lambda_0)$ is a closed subspace of $X$
which is invariant under $T$ and  there exists
some $c>0$ such that
\[
||\lambda_0-Tx||\geq ||x||\quad x\in V\cap \mathcal D(T)
\]
\medskip

\noindent
{\bf{Remark.}}
In applications it is often an important issue to decide when
$E_T(\lambda_0)$ has a finite dimensional range for
an isolated point in $\sigma(T)$.
The Kakutani-Yosida theorem 
in � 11.9 is an example where
this finite dimensionality will
be established for certain
operators $T$.



\end{document}

