\documentclass{amsart}

\usepackage[applemac]{inputenc}

\addtolength{\hoffset}{-12mm}
\addtolength{\textwidth}{22mm}
\addtolength{\voffset}{-10mm}
\addtolength{\textheight}{20mm}

\def\uuu{_}

\def\vvv{-}

\begin{document}




\centerline{\bf{4.    Frechet spaces.}}

\bigskip


\noindent
Recall from � xx that a  pseudo-norm on a  vector space $E$ is  a map $\rho$ from $E$ into
the non-negative real numbers
with the properties
\[
\rho(\alpha\cdot x)= |\alpha|\cdot \rho(x)
\quad\colon
\rho(x_1+x_2)\leq \rho(x_1)+\rho(x_2)\tag{*}
\]
where $x,x_1,x_2$ are vectors in
$X$ and $\alpha\in {\bf{R}}$. 
Let 
 $\{\rho_n\}$ is a denumerable sequence of pseudo-norms and put
\[ 
d (x,y)=\sum_{n=1}^\infty\, 2^{-n}\cdot 
\frac{\rho_n(x-y)}{1+\rho_n(x-y)}\tag{**}
\]
Assume that the intersection
\[
\bigcap\,\, \text{ker}(\rho_n)= \{0\}
\]
Then $d$ yilds a metric on $E$.
iIf every Cauchy-sequence with respect to $d$ converges to a vector in
$E$ one refers to   
$E$ as a Frechet space  defined by the sequence $\{\rho_n\}$.
Denote by
$\mathcal K_E$ he family of
convex, symmetric and  absorbing sets defined as in � xx.
\medskip


\noindent
{\bf{4.2 Theorem.}}
\emph{Let $(E,d)$ be a Frechet space. 
Then every 
$d$-closed  set $K\in \mathcal K_E$
contains an open neighborhood of the origin.}
\medskip

\noindent
\emph{Proof.}
To each integer $N\geq 1$ we have the closed set
\[
F_N= N\cdot K
\]
Since $K$ is absorbing
the union of these $F$-sets is equal to $E$.
Baire's theorem applied to the complete metric space
$(E,d)$ yields some $N$ and $\epsilon>0$ such
that $F_N$ contains
an open ball of radius $\epsilon$ centered at some
$x_0\in F_N$. If $d(x)<\epsilon$ we write
\[ 
x=\frac{x_0+x}{2}-\frac{x_0-x}{2}
\]
Now $F_N$ contains $x_0+x$ and $x_0-x$. By symmetry it also contains
$-(x_0-x)$ and the convexity entails that
$x\in F_N$.
Hence one has  the implication
\[
d(x,0)<\epsilon\implies x\in N\cdot K\tag{i}
\]
Choose an integer $M$ where $2^{-M}<\epsilon/2$.
The construction of $d$  shows that
if
$\rho_n(x)<\frac{\epsilon}{2M}$ hold for $1\leq n\leq M$, then
$d_\rho(x)<\epsilon$. Hence (i) gives
\[ 
\max_{1\leq n\leq M}\, \rho_n(x)<
\frac{\epsilon}{2M}\implies x\in N\cdot K\tag{ii}
\]
After a  scaling one has
\[
\max_{1\leq n\leq M}\, \rho_n(x)<
\frac{\epsilon}{2M N}\implies x\in K\tag{iii}
\]
Finally, for each $1\leq n\leq M$ and every vector $x$ one has
\[
\frac{\rho_n(x)}{1+\rho_n(x)}\leq 2^n\cdot d(x,0)\leq 2^M\cdot d(x,0)
\]
Now
the  reader can check that there exists $\epsilon_*>0$ such that
\[
d(x,0)<\epsilon_*\implies
\max_{1\leq n\leq M}\, \rho_n(x)<
\frac{\epsilon}{2M N}
\]
and Iii) implies  that  $K$ contains an
open neighborhood of  the origin.


\medskip

\noindent
Now $d$ and $\delta$ define
Frechet on same $E$.
We get
\[
\{d\leq \epsilon_N\}\subset \overline{\{\delta\leq 2^{-N}\}}
\]
where $\epsilon_N\to 0$ decrease.
shrinking so that
$\sum\,\epsilon_N<\epsilon_*$.
Now start $d(x_1)<\epsilon_1$.
Pick $\delta(y_1)< 1/2$ and 
\[
d(x-1-y_1)< \epsilon_2
'\]
Find $\delta(y_2)<1/4$ and
\[
d(x-1-y_1-y_2)< \epsilon_3
\]
Continue and so on.


\bigskip












{\bf{4.3 The open mapping theorem.}}
\emph{Consider  a pair of
Frechet spaces $X$ and $Y$ and let}
\[
u\colon X\to Y
\]
\emph{be  a continuous linear and surjective map.
Then $u$ is an open mapping, i.e.
for every $\epsilon>0$
the $u$-image of the $\epsilon$-ball with respect to
the Frechet metric on $X$ contains
an open neighborhood of the origin in $Y$.}
\medskip

\noindent
Proof via above !!!

\medskip


\noindent
{\bf{4.4 Closed Graph Theorem.}}
\emph{Let $X$ and $Y$ be Frechet spaces and 
$u\colon X\to Y$ is 
a linear map. Set}
\[ 
\Gamma(u)=\{ (x,u(x)\}
\]
\emph{and suppose it is a closed subspace of the Frechet space $X\times Y$.
Under this hypothesis it follows that
$u$ is continuous.}
\bigskip

\noindent
{\bf{  Exercise.}}
Verify the closed graph theorem. A hint is that
one has  the bijective linear map
from $X$ onto $\Gamma(u)$ defined by
$x\mapsto  (x,u(x))$.
Since $\Gamma(u)$ is closed in
$X\times Y$ it is a Frechet
space so the  mapping above
is open and from this one easily checks 
that $u$ is continuous.

\bigskip





\centerline {\bf{5.7  The Krein-Smulian theorem.}}
\bigskip


\noindent
Articles by these authors
from the years around 1940 contain
a wealth of results. Here we expose one of them.
Let $X$ be a Banach space and
$X^*$ its dual on which we have  constructed the weak star topology.
The  
\emph{bounded weak-star topology}
is defined as follows.
Let $S^*$ be the open 
ball  of vectors in $X^*$ with norm
$<1$. If $n$ is a positive integer we get
the ball $nS^*$ of vectors with norm $<n$.
A subset $V$ of  $X^*$ is open in the bounded weak-star
topology if and only
if
the interesections $V\cap nS^*$ are weak-star open
for every positive integer $n$.
In this way we get a new topology on $X^*$
whose corresponding
topological vector space is denoted 
by
$X^*_{bw}$,  while $X^*_w$ denotes
the topological  vector space when
$X^*$ is equipped with the weak-star topology.
Notice that  the family of open sets in
$X^*_{bw}$
contains the open sets in
$X^*_w$, i.e. the bounded weak-star topology is stronger.
Examples show that the topologies in general are not equal.
\medskip

\noindent
Next, let 
$\lambda$ be  a linear functional on $X^*$ which is continuous with
respect to the weak-star  topology.
This gives by
definition  a finite set
$x_1,\ldots,x_M$ in $X$ such that
if $|x^*(x_\nu)|<1$ for each $\nu$, then
$\lambda(x^*)|<1$.
This  implies that the 
subspace of $X^*$ given by
the common kernels of
$\widehat{x}_1,\ldots\widehat{x}_M$
contains the $\lambda$-kernel and   linear algebra gives
an $M$-tuple of complex numbers
such that
\[
\lambda=\sum\, c_\nu\cdot \widehat{x}_\nu
\]
We can express this by saying that 
the dual space of $X^*_w$
is equal to $\widehat{X}$, i.e. every linear functional on $X^*$ which is continuous 
with respect to the
weak-star topology is of the form
$\widehat{x}$ for a unique $x\in X$.
Less obvious is the following:

\medskip

\noindent
{\bf{5.7.1. Theorem.}}
\emph{The dual of $X^*_{bw}$ is equal to $\widehat{X}$.}
\medskip

\noindent
\emph{Proof.}
For  each finite
subset $A$ of  $X$ we put
\[ 
\widehat{A}=\{x^*\,\colon\, \max_{x\in A}\,|x^*(x)|\leq 1\}
\]
Let $U$ be an open set in $X^*_{bw}$
which contains the origin
and $S^*$ is the closed unit ball in
$X^*$.
The construction of the bounded weak-star topology gives
a finite set $A_1$ in $X$ such that
\[
S^*\,\cap \widehat{A}^0\subset U\tag{i}
\]
Next, let $n\geq 1$ and suppose we have constructed
a finite set $A_n$ where
\[
nS^*\,\cap \,\widehat{A_n}\subset U\tag{ii}
\]
To each finite set $B$ of vectors 
in $X$ with norm $\leq n^{-1}$
we notice that   
\[
 \widehat{A_n\cup\, B}\subset \widehat{A_n}\tag{iii}
\]
Put
\[
 F(B)=(n+1)S^*\cap \,\widehat{A_n\cup\, B} \,\cap (X^*\setminus U)
\]
It is clear that $F(B)$  is weak-star closed
for
every finite set $B$ as above. If these sets are non-empty for all $B$,
it follows from 
the weak-star compactness of $(n+1)S^*$
 that the whole intersection is non-empty. So we find a vector
 \[
 x^*\in \bigcap_B\, F(B)
 \]
Notice that $F(B)\subset \widehat{B}$ for every finite set $B$ as above
which means that  
$|x^*(x)||leq 1$ for every  vector $x$ in 
in $X$ of norm $\leq n^{-1}$. 
Hence the norm
\[
||x^*||\leq n
\]
 But then (iii) gives the inclusion
\[
 x^*\in nS^*\,\widehat{ A_n}\bigr)\, \cap (X\setminus U)\tag{iv}
\]
This contradicts (ii) and hence we have proved that there exists a finite set
$B$ of vectors with norm
$\leq n^{-1}$ such that 
$F(B)=\emptyset$.
\medskip

\noindent
From the above it is clear that an induction over $n$ gives
a sequence of sets $\{A_n\}$ such that  (ii) hold
for each $n$ and
\[ 
A_{n+1}=A_n\,\cup B_n\tag{v}
\]
where
$B_n$ is a finite set of vectors of norm
$\leq n^{-1}$.
\medskip

\noindent{\emph{Final part of the  proof.}}
Let $\theta$ be a linear functional on $X^*$ which is
continuous with respect to the bounded weak-star topology.
This gives an open neighborhood
$U$ in
$X^*_{bw}$ such that
\[
|\theta(x^*)\leq 1\quad\colon\,  x^*\in U\tag{i}
\]
To the set $U$
we find a sequence $\{A_n\}$ as above.
Let us enumerate the vectors in this sequence of finite sets by
$x_1,x_2,\ldots$, i.e.
start with the finite string of vectors in $A_1$, and so on.
By the inductive construction of the $A$-sets we have
$||x_n||\to 0$ as $n\to \infty$.
If $x^*$ is a vector in $X^*$ we 
associate the complex sequence
\[
\ell(x^*)= \{x^*(x_n)\}
\]
which tends to zero since
$||x_n||\to 0$ as $n\to \infty$.
Then
\[ 
x^*\mapsto \ell(x^*)
\]
is a linear map from $X^*$ into the 
Banach space ${\bf{c}}_0$.
If
\[
\max_n\, |x^*(x_n)|\leq 1
\]
we have by definition $x^*\in A_n^0$ for each $n$.
Choose  a positive integer $N$ so  that $||x^*||\leq n$. Thus entails that
\[
x^*\in NS^*\cap A_N^0
\]
From (ii) during the inductive construction of the $A$-.sets,
the last set is contained in $U$. Hence
$x^*\in U$
which by (i) gives
$\theta(x^*)|\leq 1$.
We conclude that $\theta$ yields
a linear functional on
on the image space of the $\rho$-map  with norm one at most.
The 
Hahn-Banach theoren gives
$\lambda\in{\bf{c}}_0^*$
of norm one at most such that
\[
\theta(x^*)= \lambda(\ell(x^*))
\]
Next,  by a wellknown result due to
Banach the dual of ${\bf{c}}_0$ is $\ell^1$. Hence there exists
a sequence $\{\alpha_n\}$ in $\ell^1$ such that
\[
\theta(x^*)= \sum\,\alpha_n\cdot x^*(x_n)
\]
In $X$ we find the vector $x=\sum\, \alpha_n\cdot x_n$
and conclude that
$\theta=\widehat{x}$ which proves the Krein-Smulian theorem.




\end{document}



\newpage
