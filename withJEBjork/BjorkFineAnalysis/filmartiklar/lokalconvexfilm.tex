\documentclass{amsart}

\usepackage[applemac]{inputenc}

\addtolength{\hoffset}{-12mm}
\addtolength{\textwidth}{22mm}
\addtolength{\voffset}{-10mm}
\addtolength{\textheight}{20mm}

\def\uuu{_}

\def\vvv{-}

\begin{document}



\centerline
{\bf{The support function of conves sets in locally context
spaces}}


\bigskip

\noindent
In � 2 we expose a theorem due to Lars H�rmander from his article
\emph{Sur la fonction d'appui des ensembles convexes dans un espaces
localement convexe} [Arkiv f�r mat. Vol 3: 1954].
As pointed out by H�rmander in his cited article, Theorem 2 which will be proved in � 2
is related to  earlier studies by
Fenchel in the article \emph{On conjugate convex functions}
Canadian Journ. of math. Vol 1 p. 73-77) 
where  Legendre transforms are
studied in infinite dimensional topological vector spaces.
The novelty in Theorem 2   is the generality  and we
remark that various separation theorems 
in text-books dealing with notions of convexity
are easy consequences of H�rmander's result.
In � 1 we collect preliminary facts about locally convex vector spaces over the real numbers
which are used in � 2. Of course, the material in �1 has independent interest and teaches the beginner
basic facts about locally convex vector spaces
which is a starting point for further study in the subject
called functional analysis.


\bigskip











\centerline {\bf {Topological vector  spaces}}
\bigskip



\noindent
Throughout $E$ denotes a vector space over the real numbers.



\medskip

\noindent {\bf{Convex sets and their $\rho$-functions.}}
A convex  set $U$ in $E$ which contains the origin is said to be
absorbing if there for  
each vector
$x\in E$  exists some
real $s>0$ such that
$s\cdot x \in U$.
The  vector  is fully absorbed  if we have the inclusion
\[
{\bf{R}}^+\cdot x\subset U
\]
{\bf{The function
$\rho_U$.}} 
If $x$ is  a  vector which is not fully absorbed we put
\[
\mu(x)=\max\{ s\,\colon sx\in U\}\quad\&\quad
\rho_U(x)=\frac{1}{\mu(x)}
\]
If $x$ is fully absorbed we put $\mu(x)=+\infty$ so that $\rho_U(x)=0$.
Notice that
\[
x\in U\implies \mu(x)\geq 1\implies \rho_U(x)\leq 1
\]
\medskip



\noindent
{\bf{0.1 Exercise.}}
Show that the convexity of $U$ entails that
$\rho_U$ satisfies the triangle inequality
\[
\rho_U(x_1+x_2)\leq \rho_U(x_1)+\rho_U(x_2)\tag{0.1.1}
\]
for all pairs of vectors in $E$,
and that $\rho_U$ is positively homogeneous, i.e.
\[
\rho_U(sx)= s\rho_U(x)\quad\colon s>0\tag{0.1.2}
\]
Conversely,, let $\rho\colon E\to {\bf{R}}^+$
satisfy (0.1.1) and (0.1.2).
Put
\[ 
U=\{\rho\leq 1\}
\]
and show that $\rho_U=\rho$.
\medskip


\noindent
A $\rho$-map as above is called
a subadditive and  positively homogeneous function on $E$.
If  $\rho$ is given we get
the convex and absorbing sets
\[ 
U_*=\{\rho<1\}\quad\&\quad U^*=\{\rho\leq 1\}
\]
The reader can check that
\[
 \rho= \rho_{U^*}= \rho_{U_*}
\]
Moreover, for every convex set $U$ such that $\rho_U=\rho$ one has
\[ 
U_*\subset U\subset U^*
\]
One refers to $U_*$ as the minimal absorbing convex set of $\rho$, and $U^*$ is the maximal  associated
convex set.
So $U\mapsto \rho_U$ is surjective from the family of absorbing
convex sets but not injective. The
failure is expressed  via the two associated minimal and maximal convex sets for a given $\rho$.


\newpage




\centerline{\bf{The Hahn-Banach theorem.}}
\medskip


\noindent
Let $\rho$ be subadditive and positively homogeneous.
An ${\bf{R}}$-linear map
$\lambda$ from $E$ to the 1-dimensional real line
is majorised by $\rho$ if
\[
\lambda(x)\leq \rho(x)\tag{*}
\]
hold for every vector $x$.
Let $E_0$ be a subspace of $E$ and
$\lambda_0\colon E_0\to {\bf{R}}$ a linear map such that
(*) hold for vectors in $E_0$.
Then there exists a linear map
$\lambda\colon E\to {\bf{R}}$ which extends 
$\lambda_0$ and is  majorised by $\rho$. 
\medskip


\noindent
{\bf{Exercise.}}
Prove  the Hahn-Banach Theorem using the following hint.
Zorn's lemma gives a maximal subspace $E^*$ which contains $E_0$  such that
$\lambda_0$ can be extend to a linear map $\lambda^*$
on $E^*$ which is majored by $\rho$.
There remains to show that $E^*=E$.
Assume the contrary and pick a non-zero vector
$\xi\in E\setminus E^*$.
For every real number
$\alpha$ we get an extension of $\lambda^*$ to a linear
functional on $E^*+{\bf{R}}\xi$
by
\[
\Lambda(x+s\xi)= \lambda^*(x)+ s\alpha
\]
with $x\in E^*$ and $s$  is a real number.
Since $\rho$ is positively homogeneous we see that
it majorises $\Lambda$ if and only if
\[
\Lambda(x+\xi)\leq \rho(x+\xi)\,\&\, \Lambda(x-\xi)\leq \rho(x-\xi)
\]
hold for all $x\in E^*$.
It means that
\[ 
\alpha\leq \rho(x+\xi)-\lambda^*(x)\,\&\,
\alpha\geq \lambda^*(x)-\rho(x-\xi)
\]
The existence of $\alpha$ for which the two inequalities hold follow if
\[
\rho(x_1+\xi)-\lambda^*(x_1)\geq \lambda^*(x_2)-\rho(x_2-\xi)
\]
or equivalently
\[
\rho(x_1+\xi)+\rho(x_2-\xi)\geq \lambda^*(x_2)+\lambda^*(x_1)= \lambda^*(x_1+x_2)\tag{i}
\]
Now (i) holds since
$\lambda^*(x_1+x_2)\leq \rho^*(x_1+x_2)$ and since $\rho$ is subadditive we have
\[
\rho(x_1+x_2)\leq
\rho(x_1+\xi)+\rho(x_2-\xi)
\]





\medskip


\centerline{\bf{Pseudo-norms.}}
\medskip


\noindent
Denote by $\mathcal C_E$ the family of absorbing convex sets $U$
which  in addition are symmetric, i.e.
\[ 
x\in U\implies -x\in U
\]
The symmetry entails that $\rho_U(-x)= \rho_U(x)$ and in general
\[
\rho_U(sx)= |s|\cdot \rho_U(x)\tag{i}
\] 
hold for every real $s$.
If $\rho\colon E\to {\bf{R}}^+$ is a sub additive and (i)  holds we say that it is a pseudo-norm.
The reader can check that
$\{\rho=0\}$
becomes a subspace of $E$.
The Hahn-Banach theorem for pseudo-norms asserts that if $\rho$ is
a given pseudo-norm and $\lambda$ a linear map on a subspace $E_0$  for which
\[
 |\lambda(x)|\leq \rho(x)\quad\colon\,\, x\in E_0
\]
then it can be extended to a linear map $\Lambda$ 
 for which 
\[
 |\Lambda(x)|\leq \rho(x)\quad\colon\,\, x\in E
\]
The proof of this symmetric version of the Hahn-Banach theorem is left as an exercise to the reader.



\newpage




\centerline {\bf{1.  Locally convex topologies.}}
\bigskip


\noindent
Denote by $\mathcal C_E$ the family of symmetric and absorbing convex sets $U$.
To each such $U$ we denote by $\mathcal L(U)$ the set of fully absorbed vectors. The reader can check that the symmetry entails that
this gives a subspace of $E$.
Next,
let
$\mathfrak{U}= \{U_\alpha\}$ be a family in  $\mathcal C_E$ such that
\[
\bigcap\,\mathcal L(U_\alpha)=\{0\}\tag{1.1}
\]
Now there exists a topology on $E$
where a base for open neighborhoods of the origin
consists of sets:
\[
\cap\, \{\rho_{U_{\alpha_i}}(x)<\epsilon\}\tag{1.2}
\] 
where $\epsilon>0$ and $\{\alpha_1,\ldots,\alpha_k\}$ is a finite set
of indices from  the  $\mathfrak{U}$-family.
If $x_0$ is a vector in $E$, then
a basis for its open neighborhoods
are given by
sets of the for $x_0+U$ where $U=\cap\, U_{\alpha_i}$.
In general, a subset $\Omega$ in $E$ is open if
there to each $x_0\in\Omega$ exists some
$U$ from (1.2) such that
$x_0+U\subset \Omega$.
This gives a topology and (1.1) entails that it is  a Hausdorff topology.
The sets  in (1.2) are  convex and therefore one 
refers to a locally convex
topology on $E$.
\medskip
 
 \noindent
 {\bf{Remark.}}
 The locally convex topology above depends upon the chosen family
 $\mathfrak{U}$. It is unchanged if we enlarge 
 the family to consist of all finite intersection of its
 sets. When this has been done
 we notice that if $U_1,\ldots, U_n$ is a finite family in
 $\mathfrak{U}$ then the norm defined by
 $U=U_1\cap\ldots\,\cap U_n$ is stronger than
 the individual $\rho_{U_i}$-norms. Hence
 a fundamental system of neighborhoods 
 consists of single $\rho$-balls:
  \[
\{\rho_U<\epsilon \}    \quad\colon U\in\mathfrak {U}
 \]
 


 
 
 
 
 \bigskip
 
 
 \noindent{\bf{The dual space $E^*$}}.
Let $E$ be equipped with a locally convex $\mathfrak{U}$-topology where
$\mathfrak{U}$ has been enlarged so that
the balls above  give a basis for neighborhoods of the origin.
A linear functional  $\phi$ on $E$
is $\mathfrak{U}$-continuous if  there
exists some
$U\in\mathfrak{U}$ and a
constant $C$ such that
\[
|\phi(x)|\leq C\cdot \rho_U(x)
\]
The family of such $\phi$-maps give vectors in a  space denoted by $E^*$ which is 
called the dual space of $E$.

\medskip


\noindent
{\bf{The weak topology on $E$.}}
It is by definition the coarsest topology for which
the functions
\[
x\mapsto \phi(x)
\]
become continuous  on $E$ for every fixed $\phi\in E^*$.
A fundamental system of open neighborhoods of the origin in the
weak topology consist of sets
\[
\cap\, \{|\phi_k(x)|<\epsilon\}
\] 
where $\epsilon>0$ and $\{\phi_k\}$ is a finite family in $E^*$.
It is clear that every weakly open set in $E$ is open with respect to the given locally convex topology.

\medskip

\noindent
{\bf{1.3 The weak-star topology on $E^*$.}}
This is the locally  convex topology on the vector space $E^*$
where a base for open neighborhoods of the
zero-vectors consist of sets defined as finite intersections of
sets defined by
\[
\{\phi\,\colon\, -\delta<\phi(x)<\delta\}\quad\colon\quad x\in X\quad\&\quad \delta>0
\]










\medskip

\noindent
{\bf{The separation  theorem.}}
To each pair 
$\phi\in E^*$ and a real number $a$ one assigns the
set
\[
H=\{x\in X\,\colon\, \phi(x)\leq a\}
\]
Notice that $a<0$ can occur in which case
$H$ does not contain the origin.
\medskip

\noindent
{\bf{1. Theorem}}. \emph{Each closed convex set $K$ in $E$ is the intersection of closed half-spaces.}

\medskip

\noindent
\emph{Proof.}
Assume  first that $K$ contains the origin and 
consider a vector $x_0\in E\setminus K$. Since $K$  is closed we find
a pseudo-norm $\rho_U$ with $U$ in the defining family $\mathfrak{U}$ such that
\[
\{x_0\}+\{\rho_U<\epsilon\}\cap K=\emptyset
\]
Put
\[ 
V= K+\{\rho_U<\epsilon\}
\]
This yields an open a convex set
in $E$ and we construct  $\rho_V$.
If $s>0$ and
$x_0\in sV$ we have $k\in K$ and a vector $\xi$ with $\rho(\xi)<\epsilon$ such that 
\[ 
x_0=sk+s\xi\implies x_0+\{\rho_U<s\epsilon\}\in sK
\]
Since $K$ is convex and contains the origin we see that
(xx) implies that $s\geq 1$.
Hence we have
\[
\rho_V(x_0)\geq 1
\]
Now we apply the Hahn-Banch Therem to the absorbing convex set $V$ and find
a linear functional $\lambda$ such that
Get $\lambda$ and
\[ 
\lambda(x_0)=\rho_V(x_0)\geq 1
\]
and at the the same time the range 
\[ 
\lambda(V)\leq 1
\]
Here  $\lambda$ belongs to $E^*$ and is not
identically zero and therefore its restriction to the open
ball $\{\rho_U<\epsilon\}$ cannot vanish identically.
So we choose 
\[
\xi\in \{\rho<\epsilon\}\,\&\,  \lambda(\xi)>0
\]
Now $k+\xi\in V$ hold for every $k\in K$ and (xx) gives 
\[ 
\lambda(k)+\lambda(\xi)\leq 1\implies \lambda(k)\leq 1-\lambda(\xi)
\]
So the half-pace
\[ 
H= \{x\colon \lambda(x)\leq 1-\lambda(\xi)\}
\]
contains $K$ while $x_0$ is outside.






\medskip










\noindent
{\bf{Remark.}}
The half-spaces in Theorem   are closed in the weak topology.
Hence every 
a closed convex set in the original topology   is also closed in the weak topology.

\medskip


 
 \noindent
{\bf{Normed spaces.}}
A pseudo-norm $\rho$ on a vector space
$E$  is called a norm
of
\[ x\neq 0\implies \rho(x)>0
\]
This gives the $\rho$-topology on $E$  whee the open balls
$\{\rho(x)<\epsilon\}$
 is a fundamental system of open neighborhoods of the origin.
One often uses the notation
\[
||x||= \rho(x)
\]
and refer to $E$ as a normed space.




\newpage


\centerline {\bf{2. Support functions of convex sets.}}
\medskip


\noindent
Let $E$ be a locally convex space.
Vectors in $E$ are denoted by  $x$, while $y$ denote  vectors in $E^*$.
To each closed and convex subset $K$ of $E$
we define a function $\mathcal H_K$ on  $E^*$ by:
\[ 
\mathcal H_K(y)=\sup_{x\in K}\,y(x)
\]


\noindent
Notice that  $\mathcal H_K$
take values in $(-\infty,+\infty]$, i.e. it may be $+\infty$ 
for some vectors $y\in E^*$.
For example, let $K={\bf{R}}^+x_0$ be a half-line.
Then $\mathcal H_K(y)=+\infty$ when
$y(x_0)>0$
and otherwise zero.
It is clear that
\[
\mathcal H_K(sy)= s\mathcal H_K(y)\tag{i}
\] 
hold when $s$ is a positive real number, i.e
$\mathcal H_K$ is positively homogeneous.



\medskip

\noindent
{\bf{2.0 Exercise.}}
Show that the convexity of $K$ entails that
\[
\mathcal H_K(y_1+y_2)\leq 
\mathcal H_K(y_1)+
\mathcal H_K(y_2)\tag{ii}
\]
for each pair of vectors in $E^*$.
Show also that if 
$K$ and $K_1$ is  a pair of closed convex sets such that
$\mathcal H_K=\mathcal H_{K_1}$ then  $K=K_1$.
\medskip






\noindent
{\bf{2.1 Upper semi-continuity.}}
For each fixed vector $x\in E$
the function 
\[ 
y\mapsto y(x)
\]
is weak-star continuous on $E^*$.
Since
the supremum function attached to
an arbitrary family of weak-star continuoes functions is upper
semi-continuous, it follows that $\mathcal H_K$ is upper semi-continuous.
\medskip


\noindent
{\bf{2.3   The class $\mathcal S(E)$}}.
It consists of all  
all upper semi-continuous functions
$G$ on $E^*$ with values in $(-\infty,+\infty]$
which satisfy (i) and (ii).

\medskip

\noindent
{\bf{2.4 Theorem.}}
\emph{Each $G\in\mathcal S(E)$ is of the form
$\mathcal H_K$ for a unique closed convex subset $K$ in $E$.}
\medskip



\noindent
\emph{Proof}
Put $F=E\oplus{\bf{R}}$ which is a new vector space where
the 1-dimensional real line is added. Its dual space
$F^*=E^*\oplus{\bf{R}}$.
Let $G\in\mathcal S(E)$ and 
put
\[
G_*=\{(y,\eta)\in E^*\oplus {\bf{R}}\quad\colon\, G(y)\leq \eta\}\tag{i}
\]
It is clear that   $G_*$ is a convex cone
in $F^*$ and the semi-continuous hypothesis on $G$
implies that  $G_*$ is closed with respect to the weak-star toplogy on $F^*$.
Next, in $F$ we define the set
\[ 
G_{**}=\{(x,t)\in E\oplus{\bf{R}}^+\,\colon\, y(x)\leq \eta t\,\colon\, (y,\eta)\in G_*\}\tag{ii}
\]
This gives a set
$\widehat C$ in 
$F^*$ which consists of vectors  $(y,\eta)$ such that
\[
\max_{(x,t)\in 
G_{**}}\,y(x)- \eta t\leq 0
\]
It is clear that
$G_*\subset \widehat C$. Now we prove the equality
\[
G_*= \widehat C\tag{*}
\]
To get (*) we use that
the two sets in (*)  are weak-star closed. If the quality fails we find
$(y_*,\eta_*)\in \widehat C\setminus G_*$
and  vector
 $(x_*,t_*)\in F$,
and a real number $\alpha$ such that 
\[
y_*(x_*)-\eta _* t_*>\alpha
\quad\&\quad (y,\eta)\in G_*\implies y(x_*)-\eta t_*\leq \alpha\tag{iii}
\]
Since $G_*$ contains $(0,0$ we  have $\alpha\leq 0$.
and since it also is a cone the last implication  gives
$(x_*,t_*)\in G_{**}$. Now
the construction of $\widehat{C}$  contradicts the strict inequality in
the left hand side of (iii).
Hence there cannot exist
a separating vector and  (*) follows.
Next, in  $E$ we consider  the convex set
\[
K=\{ x\,\colon (x,1)\in G_{**}\}
\]
Using (*) the reader can check that
\[
\mathcal H_K(y)=G(y)\quad\colon\quad y\in E^*
\]
 Hence $G$ has the requested form and the uniqueness of $K$
 follows easily via the   Exercise 0.1.
 



\newpage




\medskip


\noindent
{\bf{ 2.5 The case of normed spaces.}}
If $X$ is a normed vector space Theorem 2.4 
leads to a certain isomorphism of two families.
Denote by $\mathcal K$ the family of all convex
subsets of $E$ which are closed with respect to the norm topology.
A topology on $\mathcal K$ is defined
when we for each $K_0\in\mathcal K$ and $\epsilon>0$
declare an open neighborhood
\[ 
U_\epsilon(K_0)=\{ K\in\mathcal K\,\colon\,
\text{dist}(K,K_0)<\epsilon\}
\] 
where the norm defines the distance between
$K$ and $K_0$ in the usual way.
Denote by $\mathfrak{H}$ the family of all
functions $G$ on $E^*$ which satisfy (*) in 5.B.1 and are continuous
with respect to the norm topology on $E^*$.
A subset $M$ of $\mathfrak{H}$ is equi-continuous if there
to
each $\epsilon>0$ exists $\delta>0$ such that
\[
||y_2-y_1||<\delta \implies ||G(y_2)-G(y_1)||<\epsilon
\] 
for every $G\in M$ and 
all pairs $y_1,y_2$ in $E^*$.
The topology on $\mathfrak{H}$ is  defined
by  uniform convergence on equi-continuous subsets.

\medskip




\noindent
{\bf{2.5.1 Theorem.}}
\emph{If $E$ is a normed vector space
the set-theoretic bijective map $K\to \mathcal H_K$ is
a homeomorphism when
$\mathcal K$ and
$\mathfrak {H}$ are equipped with the described topologies.}
\medskip

\noindent
{\bf{Exercise.}} Deduce this  result 
from     Theorem 2.4. If necessary, consult H�rmander's cited article.





\end{document}





