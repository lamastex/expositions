\documentclass{amsart}
\usepackage[applemac]{inputenc}

\addtolength{\hoffset}{-12mm}
\addtolength{\textwidth}{22mm}
\addtolength{\voffset}{-10mm}
\addtolength{\textheight}{20mm}



\def\uuu{_}

\def\vvv{-}

\begin{document}




\centerline{\bf\large{Chapter VIII  The Gamma function and Riemann's $\zeta$-function}}

\bigskip
\noindent

\centerline{\emph{Contents}}
\bigskip

\noindent
\emph{1. The Gamma function}
\bigskip


\noindent
\emph{2. The $\zeta$-function}



\bigskip
\noindent
\emph{3. The Riemann hypothesis}
\bigskip


\noindent
\emph{4. The prime number theorem.}


\bigskip

\noindent
\emph{5. A uniqueness
result for the $\zeta$\vvv function.}
\bigskip

\noindent
\emph {6. A theorem on functions defined by a semi\vvv group}
\bigskip

\noindent
\emph{7. Beurling's criterion for the Rieman hypothesis}


\bigskip

\bigskip

\noindent
{\bf{Introduction.}}
The results in this chapter
stem from early work by Euler and later
studies by Gauss and Riemann.
The $\Gamma$\vvv function is
defined and analyzed in � I.
A major result is that
$\frac{1}{\Gamma(z)}$ is an entire function.
More precisely one has the Gauss representation
\[
 \frac{1}{\Gamma(z)}=z\cdot e^{\frac{\gamma}{z}}\cdot
 \prod_{m=1}^\infty\,
 (1+\frac{z}{m})\cdot e^{-\frac{z}{m}}
 \]
 where $\gamma$ is the Euler constant.
Section II is devoted to Riemann's $\zeta$-function 
defined by the Dirichlet series
\[ 
\zeta(s)=\sum_{n=1}^\infty \frac{1}{n^s} 
\quad\colon\quad
\mathfrak{Re}\, s>1\tag{*}
\]
Euler proved that the $\zeta$-function extends to
the whole complex $s$-plane with a simple  pole
at $s=1$ whose residue is one which gives
the entire function
\[
\zeta(s)- \frac{1}{s- 1}\tag{**}
\]


\noindent
In 1894 it was proved (independently) by Hadamard and
de Vall� Poussin 
that 
the zeta-function has no zeros on
$\mathfrak{Re}\, s=1$, i.e.
\[
\zeta(1+it)\neq 0
\quad\text{for all real}\quad t\neq 0\tag{1}
\]
From (1) we  shall deduce the Prime Number Theorem
in � 4.
We proceed to discuss 
properties of  the $\zeta$-function which
are due to 
Riemann. His first major result is the functional equation:
\[
\zeta(1-s) =\frac{2}{(2\pi)^s}\cdot
\text{cos}\,\frac{\pi}{2}\, s\cdot
\Gamma(s)\cdot\zeta(s)\tag{2}
\]
where the $\Gamma$\vvv function appears with
simple poles at
$(0,-1,-2,\ldots)$. Hence this
functional equation implies that
\[
\zeta(-2n)=0\quad\colon\quad n=1,2,\ldots\tag{3}
\]
The \emph{Riemann hypothesis} states that all other zeros belong to
the critical line $\mathfrak{Re}\, s=1/2$. Further comments about this conjecture
are given in � 0.4
\medskip

\noindent
{\bf{Remark about Riemann's work.}}
An account of Riemann's original work appears in
the article [xxx] by Siegel where a wealth of
involved analytic formulas related to
the $\zeta$-function appear. So it was not by a mere guessing that
Riemann arrived at his famous conjecture.
Examples of his brilliant mastery establish various
convergence formulas for
the $\zeta$-function which later were  adopted by many authors.
So even if we will prove a number of results about the
$\zeta$-function i
this material in this chapter does not give a full tribute to Riemann's work.
An example from Riemann's original studies of the
zeta-function is the following inequality:
\medskip

\noindent
{\bf{0.1 Theorem.}}
\emph{There exists a constant $C$ such that}
\[
\max_ {|s|=r}\, |\zeta(s)|\leq
C\cdot \frac{\Gamma(r)}{(2\pi )^r}\quad
\colon
r\geq 2\tag{i}
\]
\emph{Moreover, in the half-space
$\mathfrak{Re}\, s>0$ one has the equation}
\[
\zeta (z)= s\cdot \sin\,\frac{\pi s}{2}\cdot
\frac{1}{\pi}\cdot
\int_ 0^\infty\, 
\log \frac{e^{\pi x}-e^{-\pi x}}{2\pi x}\cdot \frac{dx}{x^{s+1}}\tag{ii}
\]
\medskip

\noindent
The fact that (*) gives  
(i-ii) is far from obvious 
and illustrates the
depth in  Riemann's work.


\bigskip


\noindent
{\bf{0.2  A class of Dirichlet series.}}
Let $\mathcal F$
be the family of all 
non-decreasing  increasing sequence of
positive numbers $\lambda_1\leq \lambda_2\leq \ldots$ 
for which there exists some $\delta>0$ such that
\[ 
\lambda_n\geq \delta\cdot n\tag{i}
\]
hold for every $n$ where $\delta$ can depend on the given sequence.
General facts about entire functions of exponential
type 
entail that (i)  
the Hadamard product
\[ 
f(z)=\prod\, (1+\frac{z^2}{\lambda_n^2})\tag{ii}
\]
is an entire  function of exponential type, i.e. belongs to the class $\mathcal E$.
Next, we construct
the Dirichlet series
\[
\Lambda(s)=\sum_{n=1}^\infty \frac{1}{\lambda\uuu n^s}\tag{iii} 
\]
By (i) this gives
an analytic function in the half-plane
$\mathfrak{Re}\, s>1$.
By a classic result known as the Mellin's inversion formula
it follows that $\Lambda(s)$ is obtained
from $f(z)$ by the equation
\[
\Lambda(s)= s\cdot \sin\,\frac{\pi s}{2}\cdot
\frac{1}{\pi}\cdot
\int_0^\infty\, \log f(x)\cdot \frac{dx}{x^{s+1}}\tag{iv}
\]
Using this we  prove the following conclusive result in � XX:

\medskip

\noindent
{\bf{0.2.1 Theorem.}}
\emph{Every  $\Lambda$-function obtained from 
a seqence in $\mathcal F$
extends to a meromorphic function in
the complex $s$-plane.}



\medskip

\noindent
{\bf{Example.}}
By (ii) in Riemann's cited result above
the assocaited $f$-function to the
Direchlet series defining
$\zeta(s)$ is equal to
\[
f(z)=
\frac{e^{\pi z}-e^{-\pi z}}{2\pi z}
\]



\medskip


\noindent
{\bf{0.2.2  An extremal property of  $\zeta(s)$.}}
In a lecture at Harvard University in 1949, Beurling proved
that Riemann's  zeta\vvv function
has a distinguished  position in a class of
functions defined by Dirichlet series.
For each  positive number $k$
we denote by $\mathcal C\uuu k$
the class of 
series  $\Lambda(s)$ from (ii) 
with the properties:
\[
\Lambda(s)\vvv \frac{1}{s\vvv 1}\quad\text{is entire}\tag{a}
\]
\[
\Lambda(\vvv 2n)=0 \quad\text{for all positive integers}\tag{b}
\]
\[ 
\max\uuu {|s|=r}\, |\Lambda(s)|\leq
C\cdot \frac{\Gamma(r)}{(2\pi k)^r}\quad\text{hold for a constant}\quad
C\quad\text{and 
all}\quad r\geq 2\tag{c}
\]

\medskip

\noindent
Notice that the class $\mathcal C\uuu k$ becomes more
restrictive as $k$ increases.
We shall learn that $\zeta(s)$ satisfies (a-b) and (i) in
Theorem 0.1 entails that
the zeta-function belongs to $\mathcal D_1$.
In  � 5 we prove that for every 
$1/2<k\leq 1$ the class $\mathcal C\uuu k$ only consists of constants times
the zeta-function while $\mathcal C_k=\emptyset$ if $k>1$.
This illustrates
the special role of the
$\zeta$\vvv function. 
\medskip


\noindent
{\bf{0.2.3 Beurling's closure theorem.}}
In Theorem 7.1
we prove another  result by  Beurling which
gives a \emph{necessary and sufficient condition}
for the validity of the
Riemann hypothesis expressed by a certain $L^2$\vvv closure 
on the interval $(0,1)$ generated by a specific family of functions.
Theorem 7.1 is based upon
a  closure theorem in
� 6 whose proof  illustrates the efficiency of mixing
functional analysis with analytic function theory.
Let us remark that the study of distributions of primes and the
Riemann hypothesis was one of  the main issues in
Beurling's research.
His first extensive article on this subject  from 1937
is  entitled \emph{Analyse de loi asymptotique
de la distribution des nombres premier generalis�s}.
Even though this work does not settle the Riemann hypothesis the reader
will find a number of interesting results concerned with
Dirichlet series.


\bigskip

\centerline{\bf 0.3. The distribution of prime numbers}

\medskip

\noindent
A motivation to consider the
$\zeta$-function is Euler's product formula:
\[
\zeta(s)=\prod\,\frac{1}{1-p^{-s}}
\colon\, \text{product over all prime numbers}\,\,\, \geq 2\tag{*}
\]


\noindent 
Indeed, (*)   follows since
every integer $n\geq 2$ can be factorised in a unique way
as a product of  prime numbers. Let us introduce the counting function:
\[ 
\mathcal N(x)=
\text{number of primes}\,\,\leq [x]\quad\colon
[x]=\text{least integer}\,\,\,\leq x
\]
Thus $\mathcal N(x)$ is the primitive of the discrete
measure supported by $[2,+\infty)$ which
assigns a unit point mass at every prime.
Integration by parts
gives:
\[ 
\log\,\,\zeta(s)=
s\cdot\int_2^\infty\,\frac{ \mathcal N(x)\cdot dx}{(x^s-1)x} 
\quad\colon\,s\,\,\,\text{real and}\,\,>1\tag{i}
\]


\noindent
By Euler's result we can write
\[ 
\zeta(s)=\frac{1}{s-1}+g(s)\quad\colon\, g(s)
\,\,\text{analytic in a disc}\,\,|s-1|<\delta\tag{ii}
\]


\noindent
This gives the  limit formula:
\[ 
\lim_{\epsilon\to 0}\,\,\,
\frac{1}{\text{Log}\,[\frac{1}{\epsilon}\,]}
\cdot\int_2^\infty
\frac{\mathcal N(x)\cdot dx}{x^{2+\epsilon}}=1\tag{**}
\]

\medskip

\noindent
\emph{The Prime Number Theorem.}
Notice that
\[
\lim_{\epsilon\to 0}\,\,\,
\frac{1}{\log  \,[\frac{1}{\epsilon}\,]}
\cdot\int_2^\infty
\frac{x\cdot dx}{x^{2+\epsilon}
\cdot\log\,x}=1
\]
which follows by the variable substitution
$x\mapsto e^t$
and the observation that
\[
\lim_{\epsilon\to 0}\,\,\,
\frac{1}{\text{Log}\,[\frac{1}{\epsilon}\,]}
\cdot\int_1^\infty
\,e^{-\epsilon t}\cdot\frac{dt}{t}=1
\]
\medskip

\noindent
In view of (**)  it
is therefore no surprise that
the following limit formula holds for
$\mathcal N$.
\bigskip

\noindent
{\bf 0.3.1 Theorem.} \emph{One has}
\[
\lim_{x\to\infty}
\frac{\text{Log}\, (x)\cdot \mathcal N(x)}{x}=1
\]
\medskip

\noindent
{\bf Remark.}
This limit formula was known
by heuristic considerations long before
Riemann's study of the $\zeta$-function. 
By elementary arithmetic 
one  can show
the prime number theorem holds
under the
\emph{extra hypothesis} that
$\mathcal N(x)$ has a "regular growth".
But 
both Riemann and  Gauss  knew that
$\mathcal N(x)$
does \emph{not} increase in a  regular way.
Hence a solid proof of the prime number theorem  was requested
and it was finally established by Hadamard and 
de Valle Poussin. The detailed proof is given
in  � 4.





\newpage





\centerline {\bf {0.4. The Riemann Hypothesis.}}
\bigskip

\noindent
The  conjecture   by Riemann  states that  the zeros
of $\zeta(s)$ in the critical strip $0<\mathfrak{Re}(s)<1$
belong to  the line
$\mathfrak{Re}(s)=1/2$.
This line is special since
the functional equation yields an entire function
$\xi(s)$ defined by
\[ 
\xi(s)=\frac{s(s-1)}{2}\cdot \Gamma(\frac{s}{2})\cdot \pi^{-\frac{s}{2}}
\cdot\zeta(s)\tag{*}
\]
Moreover, the $\xi$-function satisfies
\[
\xi(s)=\xi(1-s)\quad\colon\quad \xi(s)=\bar\xi(\bar s)
\]
From this it follows that the function
\[ 
t\mapsto 
\Gamma(\frac{1/2+it)}{2})\cdot
\pi^{-\frac{1/2+it}{2}}\cdot\zeta(\frac{1}{2}+it)
\]
is real-valued.
Here  $\Gamma(\frac{1/2+it)}{2})\cdot
\pi^{-\frac{1/2+it}{2}}\neq 0$
for all real $t$
and we can define the
function
\[
\theta(t)=\text{arg}\bigl [\,\Gamma(\frac{1/2+it)}{2})\cdot
\pi^{-\frac{1/2+it}{2}}\,\bigr ]\quad\colon\quad 0\leq t \leq\infty\tag{**}
\]
where we take $\theta(0)=0$.
So now we have the real-valued function
\[
X(t)= e^{i\theta(t)}\cdot\zeta(\frac{1}{2}+it)\tag{***}
\]


\noindent
Hence zeros  on the critical line
$\mathfrak{Re}(s)=\frac{1}{2}$
correspond to zeros of this real valued function.
An  asymptotic formula  of $X(t)$ was established
by Siegel based upon unpublished work
by Riemann:
\medskip


\noindent
{\bf 0.4.1 The Riemann-Siegel formula.}
In [Sie] 
the following asymptotic limit formula is proved:
\[
X(t)=\sum_{n\leq\sqrt{\frac{t}{2\pi}}}\,
\frac{\text{cos}\,\theta(t)-t\cdot\text{log}\, n}{\sqrt{n}}
+O(t^{-\frac{1}{2}})\tag{*}
\]
Starting from this  Julius Gram and Arvid B�cklund
established   numerical
results. Namely, from (*) one
derives 
the asymptotic formula:
\[
\theta(t)=\frac{t}{2}\cdot\bigl[\, \text{Log}\frac{t}{2\pi} -\frac{1}{2}\,\bigr ]
-\frac{\pi}{8}+O(\frac{1}{t})
\]


\noindent
Consider the increasing sequence of real numbers
$\{t_\nu\}$
for which
\[
\theta(t_\nu)=(\nu-1)\cdot\pi\quad\colon\quad \theta'(t_\nu)>0
\]
The Riemann-Siegel formula gives:
\[
\zeta(\frac{1}{2}+it_\nu)=1+
\sum_{n\leq\sqrt{\frac{t}{2\pi}}}\,
\frac{\text{cos}(\,t_\nu\cdot\text{log}\, n)}{\sqrt{n}}
+O(t_\nu^{-\frac{1}{2}})\tag{1}
\]
This  suggests that
$\zeta(\frac{1}{2}+it_\nu)$ in general is positive
and that $X(t_{\nu-1})$ and $X(t_\nu)$ will have different sign
which therefore gives a zero for the
$\zeta$-function in the interval $(t_{\nu-1}-t_\nu)$.
The \emph{Law of Gram}
asserts that all zeros of $X(t)$ should appear
in this fashion, i.e. one zero is produced
in $(t_{\nu-1}-t_\nu)$
for every $\nu$. Of course, this "law" was
presented  as an asymptotic formula only. A "weak
asymptotic 
law"  was confirmed
in work by Hutchinson and Titchmarsh. But
the situation is not so easy. For consider the
actual zeros on the critical line:
\[
0<\gamma_1<\gamma_2<\ldots\quad\colon\quad 
\,\zeta(\frac{1}{2}+i\gamma_\nu)=0
\]


\noindent
In  an article from 1942,
Atle Selberg proved  that the $\gamma$-sequence
increases in a certain   \emph{irregular} fashion. 
\medskip

\noindent
{\bf 0.4.2 Theorem.} \emph{There exists an absolute  constant
$0<C_*<1$
such that for every positive integer $r$
one has}
\[
\limsup_{n\to\infty}\,
\frac{\gamma_{n+r}-\gamma_n}{2\pi r}\cdot\text{Log}\,\gamma_n>1+C_*
\quad\text{and}\quad
\liminf_{n\to\infty}\,
\frac{\gamma_{n+r}-\gamma_n}{2\pi r}\cdot\text{Log}\,\gamma_n<1-C_*
\]
\medskip

\noindent
Let us finish by a citation from A. Selberg's 
lecture  on the Zeta Function and the Riemann Hypothesis
at the Scandinavian Congress in mathematics held at Copenhagen in 1946,
where he gave some comments about the eventual validity of the
Riemann Hypothesis:
\medskip

\noindent
\emph{In spite of the numerical evidence
by which it  is supported there are still reasons
to regard the Riemann Hypothesis with suspicion. For in the range
covered by calculations. the exceptions from Gram's Law are few
and they are of the simplest kind, farther out more severe
departures from Gram's law must occur and it seems
likely that the irregularities
in the variation of
$\zeta(s)$ which should be necessary
for producing zeros outside
$\mathfrak{Re}(s)=\frac{1}{2}$, should be
far more remote than the first exceptions from
Gram's Law.}
\medskip

\noindent{\bf Note.}
Atle Selberg (191x-2007)
received the Fields medal at the IMU-congress in 1950 for his outstanding contribution
in number theory and deep studies of the $\zeta$-function.


 







\bigskip


\centerline{\bf 0.4.3 Hardy's inversion formula.}
\bigskip

\noindent
Another  contribution in the study of
zeros on the critical line was achieved by Hardy in
[Har]. His  method was to regard the function
\[
\omega(x)=
\sum_{n=1}^\infty\, e^{-n^2\pi x}\quad\colon\,x>0
\]



\noindent
The construction of the $\Gamma$ function
gives the equality
\[ 
\frac{1}{n^s}\cdot \Gamma(\frac{s}{2})
\cdot {\pi}^{-\frac{s}{2}}=
\int_0^\infty\, e^{-n^2\pi x}\cdot x^{\frac{s}{2}}\cdot\frac{dx}{x}\tag{*}
\]
for each positive integer $n$.
A
summation
over $n$ gives:

\[
\zeta(s)\cdot \Gamma(\frac{s}{2})\cdot\pi^{-\frac{s}{2}}=
\int_0^\infty\, \omega(x)
\cdot x^{\frac{s}{2}}\cdot\frac{dx}{x}\quad\colon\mathfrak{Re}(s)>1
\]



\noindent 
Using Fourier's inversion formula and a shift of certain complex
line integrals, Hardy established the following result:
\bigskip

\noindent 
{\bf {0.4.4 Theorem}.}
\emph{When $\mathfrak{Re}\,x>0$ one
has the equality}
\[
\omega(x)=\frac{1}{2\sqrt x}+
\frac{1}{4\pi }\cdot\int_{-\infty}^\infty\,
\zeta(1/2+it)\cdot \Gamma(1/4+it/2)\pi^{-it/2-1/4}
x^{-1/4-it}\cdot dt
\]
\medskip

\noindent
{\bf Remark.}
The right hand side yields a nicely convergent integral.
The reason is that one has an exponential decay
for the $\Gamma$-function, i.e. there are constants
$A$ and $k$ such that
\[
|\Gamma(1/4+it/2)|\leq A\cdot e^{-k|t|}\quad\colon\, -\infty<t<\infty
\]
At the same time the $\zeta$-function 
does not increase too fast. Namely, in � XX we show that
there is constant $B$ such that
\[
|\zeta(1/2+it)|\leq B\cdot t^2\quad\colon\,t\geq 1
\]


\noindent
We refer to Hardy's original work  how the inversion formula
is used to
produce zeros of the zeta-function on the critical line.
See also  the text-book by Titchmarsch
devoted  Riemann's $\zeta$-function 
where results from  
analytic function theory and  Fourier analysis
are used to the "bitter end" in the search for a
positive answer to the Riemann Hypothesis.
\bigskip

\noindent
{\bf Final remark.}
The literature about the Riemann hypothesis is
extensive and there exist  alternative conjectures, some of them
even more general than Riemann's.  
We shall not enter into a  discussion about  this. 
But let   recall that Andr� Weil  solved
the  Riemann hypothesis in characteristic $p$. 
This gives some
support for  the "optimistic point of view point "
that Riemann's hypothesis is true.
But until an answer is found  the Riemann Hypothesis remains as
an outstanding open problem in mathematics.



\bigskip



















\newpage



\centerline {\bf {1. The Gamma function}}
\bigskip


\noindent
The $\Gamma$-function has from the start a simple definition:
\[ 
\Gamma(z)=
\int_0^\infty\,e^{-t}t^{z-1}dt\quad\colon\,
\mathfrak{Re}(z)>0\tag{*}
\]

\medskip

\noindent
With $z=x+iy$ the absolute value $|t^{z-1}|= t^{x-1}$ when $t$ is real and
positive. Moreover,  $t^\alpha\cdot \text{log}(t)$ is locally integrable
on intervals $(0,t_*)$ with $t_*>0$ and we have the exponential decay from $e^{-t}$.
Hence (*) converges when $\mathfrak{Re}(z)>0$ and gives a holomorphic
function with the
complex derivative 
\[
\Gamma'(z)=\int_0^\infty\,e^{-t}\cdot \,\text{Log}(t)\cdot t^{z-1}dt\tag{0.1}
\]



\noindent
Partial integration gives:

\[
\Gamma(z)= e^{-t}\cdot \frac{t^z}{z}\,|_0^\infty+
\frac{1}{z}\int_0^\infty e^{-t} t^z\,dt
=\frac{1}{z}\cdot\Gamma(1+z)
\]
Hence we have the equality
\[
 z\Gamma(z)=\Gamma(z+1)\quad\colon\quad\mathfrak{Re}(z)>0\tag{0.2}
\]
Replacing  $z$ by $z+1$ and so on we obtain
\[
z(z+1)\cdots (z+m)\Gamma(z)= \Gamma(z+m+1)\quad\colon\quad
m=0,1,2,\ldots\tag{0.3}
\]
Since (0.3) hold for all non-negative integers, 
$\Gamma(z)$ extends to a meromorphic function defined in
the whole complex plane and for every positive integer $m$ one has:
\[
\Gamma(z)=\frac{1}{z(z+1)\ldots(z+m)}
\cdot \Gamma(z+m)\quad\colon\quad
\mathfrak{Re}(z)>-m\tag{0.4}
\]
\medskip


\noindent
Here (0.4) shows that the
poles of 
$\Gamma(z)$  are contained in the set of 
non-negative integers.
We shall later prove that
a simple pole exists for every
such integer. Moreover, we will show that
$\frac{1}{\Gamma(z)}]$ is an entire
function and establish  the functional equation:
\[
\frac{1}{\Gamma(z)\cdot\Gamma(1-z)}=
\frac{\text{sin}\,\pi z}{\pi}\tag{**}
\]

\bigskip

\centerline {\bf 1. The Gauss representation.}
\medskip

\noindent
Consider the
Hadamard product:
\[ 
H(z)=\prod_{m=1}^\infty\,(1+\frac{z}{m})e^{-\frac{z}{m}}\tag{1.1}
\]
Here $H(z)$ is entire with simple zeros at negative integers and 
(0.4) gives
the entire  function
\[
z H(z)\cdot \Gamma(z)\in\mathcal O(\bf C)\tag{1.2}
\]
\medskip

\noindent
{\bf 1.3 Theorem} \emph{Let $\gamma$ be the Euler constant defined by}
\[
\gamma=Lim_{n\to\infty}\, 1+\frac{1}{2}+\ldots+\frac{1}{n}
-\text{Log}\, n
\]
\emph{Then one has}
\[
z H(z)\cdot \Gamma(z)=e^{-\gamma z}\tag{**}
\]

\medskip

\noindent 
\emph{Proof.}
When $n\geq 2$ we consider the partial product
\[
H_n(z)=\prod_{m=1}^{m=n}\,(1+\frac{z}{m})e^{-\frac{z}{m}}
\]
A computation gives the identity
\[
\frac{z(z+1)\cdots(z+n)}{n\,!\cdot n^z}
=
e^{z\cdot [1+\frac{1}{2}+\ldots+\frac{1}{n}-\text{Log}\,n]}
\cdot z\cdot H_n(z)	
\]
If $\mathfrak{Re}(z)>0$ the right hand side
converges to the limit $e^{\gamma z}\cdot z\cdot H(z)$.
Hence  there exists the entire limit function
\[ 
G(z)=\lim_{n\to\infty}\,\frac{z(z+1)\cdots(z+n)}{n\,!\cdot n^z}\tag{i}
\]
It is clear that (**) holds in Theorem 1.3
if we have proved:

\[ 
G(z)\cdot\Gamma(z)=1\tag{ii}
\]
To prove (ii) we regard the meromorphic function $\mathcal G=
\frac{1}{G}$ so that
\[ 
\mathcal G(z)=
\lim_{n\to\infty}\,
\frac{n\,!\cdot n^z}
{z(z+1)\cdots(z+n)}\tag{iii}
\]
Let us put
\[
\mathcal G_n(z)=
\frac{n\,!\cdot n^z}
{z(z+1)\cdots(z+n)}\tag{iv}
\]
Then we have


\[
\mathcal{G}_n(z+1)=\frac{n\,!\cdot n^{1+z}}
{(z+1)(z+2)\cdots(z+n+1)}=
\]
\[
\frac{n\,!\cdot n^z}
{z(z+1)\cdots(z+n)}\cdot\frac{nz}{n+1+z}=z\mathcal{G}_n(z)\cdot\frac
{n}{n+z+1}
\]
The last quotient tends to $z$ and we already know that
$\mathcal{G}_n(z)\to \mathcal {G}(z)$. We conclude 
that the $\mathcal G$-function satisfies
\[ 
\mathcal G(z+1)=z\mathcal G(z)\tag{v}
\]
Hence $\mathcal G$ satisfies the same functional equation as the
$\Gamma$-function and there remains only to show that
$\Gamma=\mathcal G$. Since we have two meromorphic functions
it suffices that they are equal on the positive real axis.
To show this  we first regard complex derrivatives
of $\text{Log}\,\mathcal G$ which is holomorphic in the right
half-plane. Since  the limit in (iii)  defines
$\mathcal G$ in $\mathfrak{Re}(z)>0$
we  have
\[
\log\,\mathcal G(z)=
\lim\,[z\cdot \log(n)+\sum_{\nu=1}^{\nu=n}\,
\log\,\nu -
\sum_{\nu=0}^{\nu=n}\,
\log(z+\nu) \,]\tag{vi}
\]
In the right hand side we take the second order derivative
which becomes
\[
\sum_{\nu=0}^{\nu=n}\,\frac{1}{z+\nu)^2}
\]
Passing to limit as $n\to\infty$ it follows that
\[
\frac{d^2\,\log\,\mathcal G(z)}{dz^2}=
\lim_{n\to\infty}\,-\sum_{\nu=0}^n\,\frac{1}{(z+\nu)^2}=
-\sum_{\nu=0}^\infty\,\frac{1}{(z+\nu)^2}\tag{vii}
\]
Let us now regard the function defined for $x>0$ by
\[  
\delta(x)=\log\,\Gamma(x)-
\log\,\mathcal G(x)\tag{vii}
\]
If we prove that $\delta(x)=0$ for all $x>0$ then (ii) follows by analyticity.
To show that $\delta(x)=0$
we first notice that (v) gives:
\[
\delta(x+1)=\delta(x)\quad\colon\quad x>0\tag{viii}
\]
Next, we shall regard the second derivative
of $\delta$. First, put
\[ 
\psi(x)=\frac {d^2\log\,\Gamma(x)}{dx^2}
=\frac{\Gamma(x)\Gamma''(x)-\Gamma'(x)^2}{\Gamma'(x)^2}\tag{ix}
\]



\noindent
Next we have
\[ \Gamma'(x)=
\int_0^\infty\, e^{-t}\cdot \log\,t\cdot t^{x-1}\cdot dt
\quad\colon\quad\Gamma''(x)= \int_0^\infty\, e^{-t}(\log\,t)^2\cdot t^{x-1}\cdot dt\tag{x}
\]


\noindent
From these two expressions 
the Cauchy Schwarz inequality gives
\[
\Gamma(x)\Gamma''(x)-\Gamma'(x)^2\geq 0\tag{xi}
\]
At the same time (vii) shows that
the second order derivative of
$\log\,\mathcal G(x)$ is $<0$.
We conclude that 
$\delta''(x)\geq 0$, i.e. the $\delta$-function is strictly convex when $x>0$.
Next, the periodicity remains valid for the first order derivative, i.e.
\[ 
\delta'(x+1)=\delta'(x)\quad\colon\quad x>0\tag{xii}
\]
Finally,  $\delta$ is convex  the derivative
$\delta'(x)$ is a non-increasing  function and we notice that
every non-increasing and
1-peridoic function on $x>0$
is a constant. Hence $\delta'(x)$ is a constant which gives
\[
\delta(x)=ax+b\quad\colon\, a,b \,\,\text{real constants}\tag{xiii}
\]
Since  $\delta(x+1)=\delta(x)$ it follows that
$a=0$. Hence 
$\delta(x)=b$ is a constant But $b=0$ since it is clear that
the functions $\Gamma$ and $\mathcal G$ are equal at all positive integers. This proves that
$\delta(x)$ is identically zero on $x>0$ and the proof of Theorem 1.3 is finished.

\bigskip

\noindent
{\bf 1.4 Remark.}
The limit of products which defined $\mathcal G(z)$
was considered by Gauss. So one  refers to 
$\mathcal G$ as the \emph{Gauss representation}
of the $\Gamma$-function.
Theorem 1.3 is   due to Weierstrass.
The proof above using the $\delta$-function
was  discovered by  Erhard Schmidt.
His method was later extended by Emil Artin who established a remarkable
uniqueness property of the $\Gamma$-function in an article from
1931. More precisely he proved
\medskip

\noindent 
{\bf 1.5 Artin's Theorem}
\emph{Let $f(z)$ be an entire function  with $f(0)=1$
satisfying:}
\[
f(z+1)=f(z)\quad\colon\,\colon\,
\frac{d^2\log\,f(x)}{dx^2}\geq 0\quad\colon\,x>0
\]
\emph{Then $f(z)=\frac {1}{z\cdot \Gamma(z)}$.}
\bigskip

\noindent
We refer  [Artin] for details of proof and
a further discussions about the $\Gamma$-function.


\bigskip


\noindent
\centerline {\bf  2. A functional equation.}



\noindent
Consider the product
\[ 
\Gamma(z)\cdot\Gamma(1-z)\tag{2.1}
\]
This is a meromorphic function with simple poles at all integers.
Next, we have the entire function $\text{sin}\,\pi z$
with simple zeros at all integers. Hence the function
\[ 
F(z)=
\Gamma(z)\cdot\Gamma(1-z)\cdot
\text{sin}\,\pi z
\] 
has no poles and is therefore  entire.
It turns out that this function is constant. 
\bigskip

\noindent {\bf 2.1 Theorem.}
One has the equality
\[
\frac{1}{\Gamma(z)\cdot\Gamma(1-z)}=
\frac{\text{sin}\,\pi z}{\pi}
\]

\medskip

\noindent \emph{Proof.}
The Gauss representation from (iii) in the proof of Theorem 1.3
gives
\[ 
\frac{1}{\Gamma(z)}=
\lim_{n\to\infty}\,\frac{z(z+1)\cdots(z+n)}{n\,!\cdot n^z}=
\lim_{n\to\infty}\,
\frac{z (z+1)\cdot(\frac{z}{2}+1))\cdots
(\frac {z}{n}+1)}{n^z}
\]
Similarly we obtain
\[
\frac{1}{\Gamma(1-z)}=
\lim_{n\to\infty}\,
\frac{(1-z) \cdot(1-\frac{z}{2})\cdots
(1-\frac {z}{n})}{n^{-z}}\cdot \frac{n+1-z}{n}
\]
Since $\lim_{n\to\infty}\, \frac{n+1-z}{n}=1$ we conclude that
\[
\frac{1}{\Gamma(z)\cdot\Gamma(1-z)}=
\lim_{n\to\infty}\,
z \cdot (1-z^2)\cdot(1-(\frac{z}{2})^2)\cdots
(1-(\frac {z}{n})^2)
\]
But the last term is the Hadarmard product for
$\frac{\text{sin}\,\pi z}{\pi}$
and Theorem 2.1 follows.


\bigskip


\centerline {\bf 3. The integral formula.}
\medskip

\noindent
Let $z$ be a non-zero complex number
such that $\mathfrak{Re}\, z<1$.
Keeping $z$ fixed we  define the complex powers
$s^{-z}$ for all $s$ in ${\bf{C}}\setminus (-\infty,0]$.
More precisely, when the negative real axis is removed we have a unique polar
form
\[ 
s= re^{i\theta}\, -\pi<\theta<\pi\tag{i}
\]
Then we can write
\[ 
s^{-z}= r^{-z}\cdot e^{-i\theta z}\tag{ii}
\]
For every fixed $z$
the  function $s\mapsto s^{-z}$ is analytic in the simply connected
domain
$\Omega={\bf{C}}\setminus (-\infty,0]$ where it is equal to
$e^{-z\cdot\log \,s}$ and (i) determines the value of
$\log s$.
Multiplying with $e^s$ we get
\[
 f(s)=
s^{-z}\cdot e^s\in\mathcal O(\Omega)\tag{1}
\]
With $s=\sigma+i\tau$ we get
\[ 
|f(\sigma+i\tau)|=
|r|^{-x}\cdot e^{-\theta y}\cdot e^\sigma\quad\colon\quad
z=x+iy\quad\colon\,
\tag {2}
\]


\noindent
Hence $f(s)$ has exponential decay in the right
half-plane $\mathfrak{Re}(s)<0$.
We profit upon this to construct two absolutely convergent integrals.
Given $\epsilon>0$ we consider the two half-lines
\[ 
\ell^*(\epsilon)=
\{ s=\sigma+i\epsilon\,\quad\colon\, \sigma\leq 0\}\quad\colon\quad
\ell_*(\epsilon)=
\{ s=\sigma-i\epsilon\,\quad\colon\, \sigma\leq 0\}\tag{3}
\]
\medskip

\noindent
Let $T_+(\epsilon)=\{ s= \epsilon e^{i\theta}
\,\colon\, -\pi/2\leq\theta<\pi/2\}$ and put:
\medskip
\[\gamma(\epsilon)=\ell^*(\epsilon)\cup\,T_+(\epsilon)\,\cup
\ell_*(\epsilon)\tag{4}
\]
We choose the \emph{negative} orientation along
$\gamma(\epsilon)$ which means that
we first integrate along
$\ell^*(\epsilon)$ as $\sigma$ increases from
$-\infty$ to 0 and  and so on. This gives:

\[ \int_{\gamma(\epsilon)}f(s)ds=
\int_{-\infty}^0\, f(\sigma+i\epsilon)d\sigma
-i\epsilon \cdot \int_{-\pi/2}^{\pi/2}\,
f(\epsilon\cdot e^{i\theta})e^{i\theta}\cdot d\theta-
\int_{-\infty}^0\, f(\sigma-i\epsilon)d\sigma\tag{5}
\]


\noindent
With $\mathfrak{Re}(z)=x<1$ we see from (2) that
\[
\lim_{\epsilon\to 0}\,\epsilon \cdot \int_{-\pi/2}^{\pi/2}\,
f(\epsilon\cdot e^{i\theta})e^{i\theta}\cdot d\theta=0\tag{6}
\]
Next, when $\sigma<0$ we see that (ii) and (1) give
\[ 
\lim_{\epsilon\to 0}\,
f(\sigma+i\epsilon)-f(\sigma-i\epsilon)=
|\sigma|^{-x-iy}[e^{i\pi\cdot z}-
e^{-i\pi\cdot z}]\tag{7}
\]


\noindent
Hence we have proved

\medskip

\noindent
{\bf 3.1 Proposition.}
\emph{One has}
\[ 
\lim_{\epsilon\to 0}\,
\int_{\gamma(\epsilon)}f(s)ds=2i\cdot\text{sin}(\pi z)\cdot \int_{-\infty}^0\, |\sigma|^{-x-iy}\cdot
e^\sigma d\sigma
\]


\noindent
Dividing by $2\pi i$
and making the variable substitution $t=-\sigma$ we get
\[
\lim_{\epsilon\to 0}\,\frac{1}{2\pi i}\cdot \int_{\gamma(\epsilon)}f(s)ds=
\frac{\text{sin}(\pi z)}{\pi}\cdot
\int_0^\infty  t^{-x-iy}\cdot e^{-t}dt\tag{8}
\]
\medskip

\noindent
The last integral is $\Gamma(1-z)$. Together with
Theorem 2.1 we conclude the following:
\medskip

\noindent
{\bf 3.2 Theorem.}
\emph{When $\mathfrak{Re}(z)<1$ one has the equality}
\[
\frac{1}{\Gamma(z)}=
\lim_{\epsilon\to 0}\,\frac{1}{2\pi i}\cdot \int_{\gamma(\epsilon)}e^s\cdot s^{-z}ds
\]
\emph{where $\gamma(\epsilon)$ has the negative orientation
as described in (4).}
\bigskip

\noindent
{\bf 3.3 Remark.}
In the right hand side we can change the contour
$\gamma\uuu\epsilon$ where convergence
holds as long as the real part of $s$ tends to
$\vvv\infty$
along the
end\vvv tails. For example, the integral representation holds for
every $\epsilon>0$, i.e. even for \emph{large} $\epsilon$.
This flexible manner to represent $\frac{1}{\Gamma(z)}$
is used in many formulas where $\Gamma$\vvv functions appear.
An  example are  the  integral formulas due to Barnes
for hypergeometric functions.


\newpage



\centerline{\bf \large II. Riemann's $\zeta$-function}

\bigskip

\noindent
{\bf Introduction.}
The 
zeta-function is defined
by the series:
\medskip
\[ 
\zeta(s)=\sum_{n=1}^\infty \frac{1}{n^s} 
\quad\colon\quad
\mathfrak{Re}\, s>1\tag{0.1}
\]




\noindent 
It is clear that
$\zeta(s)$ is an analytic function in the half-space
$\mathfrak{Re}(s)>1$ whose complex derivative is found
by termwise differentiation, i.e.
\[
\zeta'(s)=-\sum_{n=1}^\infty \frac{\log(n)}{n^s} 
\quad\colon\quad
\mathfrak{Re}\, s>1\tag{0.1}
\]
In the subsequent sections we establish  results about
the $\zeta$\vvv functions concerned with
growth properties and distribution of its zeros.
\bigskip


\centerline{\bf{1. The meromorphic extension.}}
\medskip

\noindent
The fact  that  $\zeta(s)$ has a meromorphic extension
with a simple pole a $s=1$
goes back to  Euler and is presented below.

\medskip

\noindent
{\bf{1.1 Euler's summation formula.}}
When
$x\geq 1$ we let $[x]$
denote the largest integer which is $\leq x$.
Set
\[
P(x)= [x]-x+\frac{1}{2}\,,\quad x\geq 1
\]
The differential  $dP$ is  the counting function at
positive integers. So if $\mathfrak{Re}\, s>1$
we have

\[ 
\zeta(s)-\int_1^\infty\, \frac{dx}{x^s}=
\int_1^\infty\, \frac{dP(x)}{x^s}=
\frac{P(x)}{x^s}\bigl|_1^\infty-
\int_1^\infty\, \frac{P(x)}{x^{s+1}}\cdot dx
\]
\medskip

\noindent
Since $P(1)=\frac{1}{2}$
and $\int_1^\infty\, \frac{dx}{x^s}=\frac{1}{s-1}$ we obtain
\medskip

\noindent
{\bf 1.2 Euler's integral formula.}
One has
\[
\zeta(s)= \frac{1}{s-1}+\frac{1}{2}+
s\cdot \int_1^\infty\, \frac{P(x)}{x^{s+1}}\cdot dx
\]
\medskip

\noindent
{\bf Remark.}
Since the function $P(x)$ is bounded  the last integral
extends to an analytic function 
in the half plane
$\mathfrak{Re}\, s>0$.
So Euler's integral formula  shows that
the zeta-function extends to a meromorphic function in
$\mathfrak{Re}\, s>0$ with a simple pole at
$s=1$.
\medskip

\noindent
{\bf 1.3 Further integral formulas.}
Notice  that $P(x)$ is 1-periodic:
\[
P(x+1)=P(x)\,\quad\, x\geq 1\,.
\]
Moreover it has
the Fourier series expansion

\[ 
P(x)=\sum_{n=1}^\infty\,
\frac{\text{sin}\, (2n\pi x)}{n\pi}
\]
whose primitive function becomes
\[
P_1(x)=-
\sum_{n=1}^\infty\,
\frac{\text{cos}\,( 2n\pi x)}{2n^2\pi^2}
\]
A partial integration gives
\[
s\cdot \int_1^\infty\, \frac{P(x)}{x^{s+1}}\cdot dx
=s\cdot \frac{P_1(x)}{x^{s+1}}\bigl|_1^\infty-
s(s+1)\int_1^\infty\,\frac{P_1(x)}{x^{s+2}}\cdot dx
\]
Next, one has the summation formula
\[
P_1(1)=\sum_{n=1}^\infty\,
\frac{1}{2n^2\pi^2}=\frac{1}{12}
\]
It follows that
\[
\zeta(s)= \frac{1}{s-1}+\frac{1}{2}+\frac{s}{12}-
s(s+1)\cdot \int_1^\infty\, \frac{P_1(x)}{x^{s+2}}\cdot dx
\]
\medskip

\noindent
The last integral is analytic when
$\mathfrak{Re}\, s>-1$ which gives a further
meromorphic extension of $\zeta(s)$.
Repeating the process by taking the primitive function of
$P_1$ one shows that $\zeta(s)$ extends to the whole complex plane with
a  simple pole at $s=1$.
Let us also notice that the boundedness of $P_1(x)$
and the formula above gives
\medskip

\noindent
{\bf 1.4. Proposition.} \emph{Let $0<\delta<1$. Then
there exists a constant $C(\delta)$
such that}

\[ |\zeta(-1+\delta+it)|\leq
C(\delta)\cdot t^2\,,\quad\text{for all}\,\,|t|\geq 1
\]

\bigskip





\centerline{\bf 2. Riemann's functional equation.}
\bigskip

\noindent
The next result consolidates Euler's results from section 1.
\medskip

\noindent
{\bf 2.1. Theorem} 
\emph{The $\zeta$-function extends to a meromorphic
function in the whole complex $s$-plane where
it satisfies the functional equation}
\[
\zeta(1-s) =\frac{2}{(2\pi)^s}\cdot
\text{cos}(\,\frac{\pi s}{2})\, \cdot
\Gamma(s)\cdot\zeta(s)\quad\colon\, s\in {\bf{C}}
\]

\medskip

\noindent
The proof requires several steps. To begin with we have
the integral formula in Theorem 3.2 from � I: 
\[
\frac{1}{\Gamma(s)}=
\frac{1}{2\pi i}\cdot
\int_{L\uuu \epsilon}\,
e^z\cdot z^{-s} dz\tag{1}
\]
Notice that the role of $z$ and $s$ are interchanged
in (1) above as compared to the formula in  � 3 about the
$\Gamma$-function. 
Since both sides in (1) are entire functions of $s$
we can replace $s$ by $1-s$ and write
\[
\frac{1}{\Gamma(1-s)}=\frac{1}{2\pi i}\cdot
\int_{\gamma(\epsilon)}\,
e^z\cdot z^s \cdot \frac{dz}{z}\tag{2}
\]
If $n$ is a positive integer the variable substitution $z\mapsto nz$
gives
\[
\frac{1}{\Gamma(1-s)}=\frac{1}{2\pi i}\cdot
n^s\cdot \int_{L_\epsilon}\,
e^{nz}\cdot z^s \cdot \frac{dz}{z}\implies
\]
\[
n^{-s}\cdot \frac{1}{\Gamma(1-s)}=
\frac{1}{2\pi i}\cdot
\int_{L_\epsilon}\,
e^{nz}\cdot z^s \cdot \frac{dz}{z}
\tag{3}
\]


\noindent
Next, notice that
\[ 
\sum_{n=1}^\infty\, 
e^{nz}=
\frac{e^{z}}{1-e^{z}}\quad\colon\, \mathfrak{Re}\, z<0
\]


\noindent
Taking the sum over $n$ in (3)
we obtain
\medskip



\noindent
{\bf 2.2 Proposition.} \emph{One has the equality}
\[ 
\frac{\zeta(s)}{\Gamma(1-s)}=
\lim_{\epsilon\to 0}\frac{1}{2\pi i}\cdot
\int_{L_\epsilon}\,
\frac{e^{z}}{1-e^z}\cdot z^s \cdot \frac{dz}{z}
\]
\newpage

\centerline{\emph{Proof of  Theorem 2.1}}


\medskip

\noindent
Let $R=(2N+1)\pi$ where $N$ is a positive integer.
With $0<\epsilon<1$ we obtain a simply connected
domain
$\Omega$ bounded by the circle $|z|=R$
and the portion of  of $L_\epsilon$. See figure !
With $s$ fixed we find a single valued branch of
$z^s$ in $\Omega$ and regard the analytic function
\[ 
g_s(z)=\frac{e^{z}}{1-e^{z}}\cdot z^{s-1}\tag{ii}
\]
In $\Omega$ we encounter poles when
$e^z=1$, i.e. when $z=2\pi i\nu$ when $1\leq\nu\leq N$.
Residue calculus gives - see figure !!:
\[ 
\int_{-\pi+\delta}^{\pi-\delta}
\, g_s(Re^{i\theta})\cdot d\theta
-\int_{L_\epsilon}\, g_s(z)\cdot dz=
2\pi i\cdot\sum_{\nu=1}^{\nu=N}\,\mathfrak{res}\,
[g_s(2\pi i\nu)+g_s(-2\pi i\nu)]\tag{iii}
\]


\noindent
The residue sum is easily found, i.e. one has
\medskip

\noindent \emph{Sublemma} The right hand side in
(iii) becomes:
\[
2\cdot\sum_{\nu=1}^{\nu=N}\,
\frac{1}{\nu^{1-s}}\cdot (2\pi)^{s-1}\cdot
\text{cos}\frac{\pi}{2}(s-1)\tag{*}
\]


\noindent
So far we have allowed any $s$. Let us now specify $s$ to be real and negative.
A straightforward calculation which is left to the reader gives:
\[
\lim_{N\to\infty}\,
\int_{-\pi+\delta}^{\pi-\delta}
\, g_s(Re^{i\theta})\cdot d\theta
=0\colon\quad\, s\,\,\text{real and}\,\,<0\tag{iv}
\]


\noindent
Next,  the  definition of
the $\zeta$-function gives
\[ 
\zeta(1-s)=
\lim_{N\to\infty}\, \sum_{\nu=1}^{\nu=N}\,
\frac{1}{\nu^{1-s}}\quad\colon\, s<0\tag{**}
\]
\medskip

\noindent Hence (iii-iv) together with
Proposition 2.2, (*) and (**) give:

\medskip





\noindent
{\bf{Sublemma}}.
\emph{When $s$ is real and $<0$ one has
the equality:}
\[
2\cdot (2\pi)^{s-1}\cdot \zeta(1-s)\cdot 
\text{cos}\,\frac{\pi}{2}(s-1)=
\frac{\zeta(s)}{\Gamma(1-s)}
\]


\noindent
\emph{Final part of the proof }.
Notice  that
\[
\text{cos}\,\frac{\pi}{2}(s-1)=\text{sin}\,\frac{\pi s}{2}=
\frac{\text{sin}\,\pi s}{2\cdot 
\text{cos}\,\frac{\pi s}{2}}\tag{1}
\]





\noindent 
Hence the Sublemma gives
\[
\frac{\zeta(s)}{\Gamma(1-s)}=
(2\pi)^{s-1}\cdot\zeta(1-s)\cdot 
\text{sin}\,\pi s\cdot 
\frac{1}{\text{cos}\,\frac{\pi s}{2}}\implies
\]
\[
\text{cos}\,\frac{\pi s}{2}\cdot\zeta(s)=
(2\pi)^{s-1}\cdot\zeta(1-s)\cdot 
\text{sin}\,\pi s\cdot\Gamma(1-s)\tag{1}
\]
\medskip

\noindent
At this stage we recall that
\[
\text{sin}\,\pi s\cdot\Gamma(1-s)=
\frac{\pi}{\Gamma(s)}\tag{2}
\]


\noindent 
So (1-2) give together:

\[
\text{cos}\,\frac{\pi s}{2}\cdot\zeta(s)=
\frac{1}{2}\cdot (2\pi)^s\cdot\zeta(1-s)\cdot 
\frac{1}{\Gamma(s)}\tag{3}
\]
\medskip

\noindent
Expressing $\zeta(1-s)$ alone we get the requested formula in
Theorem 2.1 where analyticity gives equality for all $s$.

\bigskip


\centerline {\bf 3. The asymptotic formula for
$\mathcal N(T)$ }
\bigskip



\noindent
Using the functional equation and the
fact that the $\Gamma$-function has simple poles
at all non-negative integers we will show
that $\zeta(-2m)=0$ for every positive integer $m$.
Apart from these zeros    it turns out
that
the remaining zeros of $\zeta(s)$ belong to the strip
\[ 
\mathcal S=\{0<\mathfrak{Re}\, s<1 \}\tag{0.1}
\]

\noindent
To find  zeros in $\mathcal S$
we  notice that 
the $\zeta$-function is real when $s$ is real and $>1$. 
By analytic continuation it follows that
\[ 
\zeta(s)=\bar\zeta(\bar s)\tag{0.2}
\]
hold for all $s$. Hence  zeros of $\zeta$ appear with conjugate pairs 
in $\mathcal S$   and it suffices to
study the counting function
\[ 
\mathcal N(T)=\text{number of zeros of}\,\,\zeta(s)\quad
\colon\, s\in\mathcal S\cap\,\{\mathfrak{Im}\,s>0\}\tag{0.3}
\]

\bigskip



\noindent
{\bf 3.1 Riemann's asymptotic formula.}
\emph{There exists a constant $C_0$ such that the following hold when $T\geq 1$:}
\[
\mathcal N(T)=\frac{1}{2\pi} T\cdot\log\,T-
\frac{1+\log\,2\pi}{2\pi}\cdot T+\rho(T)\cdot \log\,T\quad\colon
\,|\rho(T)|\leq C_0
\]
\medskip

\noindent
{\bf Remark.}
Riemann announced this asymptotic formula in [Rie]. It was later
proved
by von Mangoldt and here we shall 
present the elegant    proof due to
B�cklund.
Before we begin the proof of the asymptotic formula above
we draw some conclusions from
Riemann's functional equation.
Let $n\geq 1$ be a positive  integer and put $s=2n+1$.
Here the cosine function has a simple zero, i.e.
$\text{cos}\,\pi n+\frac{\pi}{2}=0$ and at  the same time 
$\zeta(2n+1)$ and $\Gamma(2n+1)$  are real and positive.
Hence Theorem 2.2  implies that the $\zeta$-function has a simple
zero at $1-s=-2n$. So we have proved:
\bigskip

\noindent
{\bf 3.2 Proposition.}
\emph{The $\zeta$-function has simple zeros at all even negative integers.}
\bigskip















\noindent{\bf 3.3 The entire $\xi$-function.} Let us define the function
\[ 
\xi(s)= \frac{s(s-1)}{2}\cdot \Gamma(\frac{s}{2}) \cdot \pi^{-\frac{s}{2}}
\cdot\zeta(s)
\]


\noindent {\bf 3.4 Proposition.} \emph{The function $\xi(s)$ is entire and satisfies}
\[
\xi(s)=\xi(1-s)\quad\colon\,\xi(s)=\bar\xi(\bar s)
\]


\noindent 
\emph{Proof.} That $\xi(s)$ is entire is clear from the
construction since
the zeros of the
$\zeta$-function at even and negative integers
compensate the simple poles of the
$\Gamma$-function at negative integers. Moreover, the
factor $s(s-1)$
takes care of the simple pole of
$\zeta(s)$ at $s=1$
and the pole of the $\Gamma$-function at
$s=0$.
The equality $\xi(s)=\bar\xi(\bar s)$ follows
since the same conjugation property  hold for 
the three factors defining $\xi(s)$ and
the equality $\xi(s)=\xi(1-s)$ follows from Riemann's 
functional equation.

\bigskip


\centerline {\bf 3.5. Zeros in the critical strip.}
\medskip

\noindent
The construction of the $\xi$-function in
3.3. and the fact that
$\Gamma(\frac{s}{2})$ has no zeros in the critical strip give:
\medskip

\noindent {\bf 3.6 Proposition.}
\emph{The zero sets of $\zeta$  and $\xi$
in the critical strip are equal.}


\medskip

\noindent
Now we shall count zeros of $\xi$.
For this purpose we consider
the rectangle
\[ 
\square_T=\{s=\sigma+it\quad\colon\, -1/2<\sigma<3/2\,\colon
-T<t<T\}\quad\colon\, T\geq 1
\]
Notice that $\square_T$ is symmetric around $\mathfrak{Re}(s)=1/2$.
We choose $T$ so that $\xi(s)$
has no zeros occur when
$\mathfrak{Im}(s)=T$.
This gives
\[ 
2\cdot \mathcal N(T)=\frac{1}{2\pi i}
\int_{\partial\square_T}\, \frac{\xi'(s)ds}{\xi(s)}\tag{*}
\]
\medskip

\noindent 
To estimate the line integral in
(*) we recall that
\[ 
\xi(s)=\xi(1-s)\quad\colon\,\xi(s)=\bar\xi(\bar s)\tag{i}
\] 

\noindent
These two equations  entail
that the  line integral
over $\partial\square_T$  is \emph{four times}
the line integral over the the "quarter part" given by the union of
the two lines
\[
\ell_*(T)=\{2+it\quad\colon\,0<t<T\}\quad\colon\,
\ell^*(T)=\{\sigma+iT\quad\colon\,1/2<\sigma<2\}\tag{ii}
\]




\noindent
Next,  the line integral is the real number
$2\cdot\mathcal N(T)$ and
taking the factor 4 into the account we get:
\medskip

\noindent
{\bf 3.7 Lemma.} \emph{One has}
\[
\mathcal N(T)=
\frac{1}{\pi i}
\cdot\int_{\ell_*}\,
 \frac{\xi'(s)ds}{\xi(s)}+
\frac{1}{\pi i}
\cdot\int_{\ell^*}\,
 \frac{\xi'(s)ds}{\xi(s)}
\]


\medskip






\noindent
Following Backlund we  decompose  the logarithmic
derivative of $\xi(s)$ which gives   a sum of
five terms:
\medskip
\[
\cdot  \frac{\xi'(s)ds}{\xi(s)}=\frac{1}{s}+\frac{1}{s-1}+
\frac{1}{2}\cdot  \frac{\Gamma'(\frac{s}{2})}{\Gamma(\frac{s}{2})}
-\frac{\text{Log}\,\pi}{2}+
\frac{\zeta'(s)}{\zeta(s)} \tag{iii}
\]

\noindent
There remains to study the line integrals of
each of these separate terms. 
The reader can verify that
the contribution from  the four
first terms  give
the sum of the four first terms
which appear in Riemann's asymptotic formula.
There remains to investigate
the contribution of:

\[
\frac{1}{\pi i}
\cdot\int_{\ell_*(T)}\,
 \frac{\zeta'(s)ds}{\zeta(s)}+
\frac{1}{\pi i}
\cdot\int_{\ell^*(T)}\,
 \frac{\zeta'(s)ds}{\zeta(s)}\tag{*}
\]
Here the  requested  estimate for the remainder  term
follows if we have proved
\medskip

\noindent
{\bf{3.8 Lemma.}} \emph{There exists a constant $C$ such that the absolute value
of (*) is bounded above by
$C\cdot\log(T)$ for every $T\geq e$.}


\medskip
\noindent
{\emph{Proof.}}
First we
consider the
the line integral over the vertical line $\ell_*(T)$.
To pursue the logarithmic derivative 
of $\zeta$ along
$\ell_*$
which  by assumption is $\neq 0$
we choose choose a branch of its Log-function and
write
\[
\log\,\zeta(2+it)=
\log\,|\zeta(2+it)|+ i\phi(t)\tag{1}
\]
where $\phi(t)$  is a real valued argument function which gives
\[ 
\frac{\zeta'(2+it)}{\zeta(2+it)}=
\frac{1}{i}\frac{d}{dt}\,[
\log\,\zeta(2+it)\,]=
\frac{1}{i}\frac{d}{dt}\,
|\log\,\zeta(2+it)|\,]+\frac{d\phi}{dt}\tag{2}
\]
\medskip

\noindent
Along $\ell_*$ we  have $ds=idt$ and
since $\mathcal N(T)$ is real our sole concern is to study:
\[
\mathfrak{Re}\,[
\frac{1}{\pi i}
\cdot\int_{\ell^*}\,
 \frac{\zeta'(s)ds}{\zeta(s)}\,]=
 \frac{1}{\pi}\int_0^T\,
\frac{d\phi}{dt}\cdot dt \tag{3}
\]


\noindent
To estimate  (3)
we shall need
the following inequality
\[ 
\mathfrak{Re}[\,\zeta(2+it)\,]>\frac{1}{3}\quad\colon\,t>0\tag{4}
\]



\noindent
To verify (4) we use that
$n^{it}=e^{it\text{log}\,n}$ and get
$\mathfrak{Re}\, n^{it}=
\text{cos}\, (t\cdot \text{log}\, n)$. It follows that
\[
\mathfrak{Re}[\,\zeta(2+it)\,]=
1+\sum_{n=2}^\infty\, \frac{\text{cos}\, (t\cdot \text{log}\, n)}{n^2}
\geq 1-\sum_{n=2}^\infty\, \frac{1}{n^2}
= 2-\frac{\pi^2}{6}>\frac{1}{3}\tag{5}
\]

\medskip


\noindent
which proves (4) and it shows
that 
$\zeta(2+it)$ belongs to   the right half-plane when
$0\leq t\leq T$.Hence the argument of the $\phi$-function satisfies:
\[
-\pi/2<\phi(t)<\pi/2\,\,\quad\colon\,\text{along}\,\,\ell_*(T)
\] 
In particular we get
\[
-1< \frac{1}{\pi}\int_0^T\,
\frac{d\phi}{dt}\cdot dt <1
\]


\noindent 
Thus, the contribution along $\ell_*(T)$ has absolute value $<1$ for all
$T$ and is therefore  harmless for the asymptotic estimate
in Lemma 3.8.
\bigskip

\noindent
{\emph{3.9
The line integral over $\ell^*(T)$.}} Along $\ell^*(T)$
we have $ds=d\sigma$. So this time our concern is to estimate
\[
\frac{1}{\pi}\cdot\int_{1/2}^2\,
\mathfrak{Im}\,[\frac{\zeta'(\sigma+iT)}{\zeta(\sigma+iT)}\,]\cdot
d\sigma \tag{1}
\]



\noindent
By the result in XXX this amounts to find an upper bound for
the zeros of the function
\[
\sigma\mapsto
\mathfrak{Re}(\zeta(\sigma+iT)\quad\colon
1/2< \sigma<2\tag{2}
\]



\noindent
Since
$\zeta(s)=\bar\zeta(\bar s)$
this amounts to consider zeros of the function
\[
\sigma\mapsto
\zeta(\sigma+iT)+\zeta(\sigma-iT)\quad\colon\,
1/2< \sigma< 2\tag{3}
\]


\noindent
We seek an upper bound of zeros for large
$T$. So from now on we assume that
\[ 
T\geq \frac{7}{2}
\]
To obtain an upper bound for the zeros in (3)
we consider the 
analytic function of
a new complex variable $w$:
\[
\phi_T(w)=\zeta(w+iT)+\zeta(w-iT)\quad\colon\,|w-3|\leq 5/2\tag{4}
\]


\noindent
Let $\frac{1}{2}< \sigma_1\leq\ldots\leq \sigma_m<2$
be the $m$-tuple which yield all the zeros counted with 
multiplicities in (3).
They also give zeros of the analytic  function $\phi$ in the disc
$|w\vvv 3|\leq 5/2$
and by the general inequality from XXXX we have
\[
\sum_{\nu=1}^m\,\text{log}\,\bigl[\frac{5}{2\sigma_\nu}\,\bigr]
+\text{log}\,|\phi_T(3)|\leq
\frac{1}{2\pi}\cdot\int_0^{2\pi}\,
\text{log}\,\bigl|\phi_T(3+\frac{5}{2}e^{i\theta})\,\bigl|\cdot d\theta\tag{5}
\]
\medskip

\noindent
By XX we have
$|\phi_T(3)|\geq 1/3$ and since
$1/2\leq\sigma\uuu \nu\leq 2$ hold we get
\[
m\cdot \text{log}\,\frac{5}{4}\leq
\text{log}\, 3+\max_{|w-3|= 5/2}\,\text{log}\, |\phi_T(w)|\tag{6}
\]
\medskip

\noindent
Finally, Proposition xx gives the constant $C(1)$ such that
\[
\phi_T(w)|\leq 2\cdot C(1)\cdot T^2\quad\colon\quad  T\geq 7/2\tag{7}
\]
\medskip

\noindent
Passing to $\text{Log}\,|\phi_T(w)|$
we conclude that the $m$-number which counts the zeros
is bounded above by an absolute constant times
$\text{log}\, T$ for all $T\geq 7/2$
which  finishes the proof of
Lemma 3.8.









\newpage


\centerline{\bf  4. The prime number theorem}

\bigskip

\noindent
The counting function for prime numbers is defined by
\[ 
\Pi(x)=\text{number of primes}\,\,\,\leq x\tag{i}
\]
This is an increasing jump function where
we for example have
$\Pi(8)=\Pi(9)=\Pi(10)=4$ and 
$\Pi(11)=5$. 
\medskip

\noindent
{\bf{4.0 Exercise.}}
Show that prime numbers are sufficiently sparse in order that
\[
\lim_{x\to\infty}\, \frac{\Pi(x)}{x}=0
\]

\medskip

\noindent
{\bf{4.1 Theorem}}
\emph{There exists the limit formula}
\[
\lim_{x\to\infty}\,\,\frac{\text{log}\,x\cdot \Pi(x)}{x}=1\tag{*}
\]
\medskip

\noindent
To prove this  we introduce the function
defined for $x>0$ by
\[
\gamma(x)=\sum_{p\leq x}\,\log p
\] 
where the sum as indicated extends over all prime numbers $\leq x$.
Next, let $d\Pi$ be the discrete measure which assigns a unit point mass
at every prime number.
A
partial integration gives the equation:
\[
\gamma(x)=\int_2^x \,\log x\cdot d\Pi(x)=
\log x\cdot \Pi(x)-\int_2^x \,\frac{\Pi(x)}{x}\cdot dx\tag{i}
\]
Afer a division by $x$  on both sides it follows from
Exercise 0.4 that the limit formula (*) in Theorem 4.1. is equivalent to 
\[ 
\lim_{x\to\infty}\, \frac{\gamma(x)}{x}=1\tag{**}
\]

\noindent
There  remains to prove (**) where a first step
 is the following:

\medskip

\noindent
{\bf Lemma 4.2.} \emph{The function $\frac{\gamma(x)}{x}$ is bounded.}
\medskip

\noindent
\emph{Proof.}
The idea is to use that if $N\geq 2$ is a positive integer
then all terms in  the binomial expansion of $(1+1)^{2N}$
are integers. In particular we have the integer
\[
\xi_N=\frac{(2N)\,!}{N\,!\cdot N\,!}\tag{i}
\]
Let $q_1,\ldots,q_m$ be the distinct primes
in $[N+1,2N-1]$.
By (i) each $q_\nu$ is a prime divisor of $\xi_N$. So we have trivially
\[ 
q_1\cdots q_m\leq\xi_N
\]
Taking the logarithm the definition of the
$\gamma$-function gives
\[ 
\gamma(2N)-\psi(N)\leq
\log\,\xi_N\leq N\cdot \log\, 2\tag{ii}
\]
where the reader may confirm the last inequality by
a trivial calculation.
Now we perform the usual trick using 2-powers, i.e. given
$K\geq 2$ we apply (ii) with
$N=2^k\quad\colon\,1\leq k\leq K-1$. After a
summation over $k$ we get

\[
\gamma(2^{K}-\psi(2)\leq \log\,2\cdot [1+\ldots+2^{K-1}]
\leq \log\, 2\cdot 2^K
\]
Since this hold for all $K$ and the $\gamma$-function is increasing we see that
Lemma 4.2 holds.

\bigskip

\noindent
Next, Lemma 4.2 and an elementary fact about convergent integrals 
give
(**) if we have proved
that the  integral below exists:
\[
\int_2^\infty\, \frac{\gamma(x)-x}{x^2}\cdot dx\tag{***}
\]
Notice that we only request that
there exists
a limit
\[
\lim_{M\to \infty}\,\int_2^M\, \frac{\gamma(x)-x}{x^2}\cdot dx
\]
whose  the actual limit value
is of no concern for us !


\medskip

\noindent
{\bf{4.3 The $\Phi$-function}}.
To establish (***) we introduce the function
\[
\Phi(z)=\sum\,\log p\cdot p^{-z}
\]
where $z$ is a complex variable. To begin  with $\Phi(z)$ is defined
in the half-plane $\mathfrak{Re}(z)>1$ and
since $d\gamma$ is the discrete measure which assigns the mass
$\log p$ at every prime number we have:
\[
\Phi(z)=\int_1^\infty\,\,x^{-z}\cdot d\gamma(x)=
z\cdot 
\int_1^\infty\, x^{-z-1}\cdot \gamma(x)\cdot dx\tag{1}
\]
With $z=1+\zeta$ and $\mathfrak{Re}(\zeta)>0$
we can write (1) as
\[
\frac{1}{1+\zeta}\cdot \Phi(1+\zeta)=
\int_1^\infty\, x^{-\zeta-2}\cdot \gamma(x)\cdot dx\tag{2}
\]
and the substitution
$x\to e^{t}$ identifies the last integral by
\[
\int_0^\infty\, e^{-\zeta t}\cdot e^{-t}\cdot \gamma(e^t)\cdot dy\tag{3}
\]
Next, let us introduce the function
\[ 
f(t)=e^{-t}\gamma(e^t)-1
\]


\noindent
Lemma 4.2 shows that $f$ is a bounded function
and the substitution $x\to e^t$ shows that
the integral (***) converges if there exists the limit
\[ 
\lim_{T\to\infty}\, \int_0^T\, f(t)\cdot dt\tag{****}
\]
Hence there only remains to prove (****). For this purpose we introduce the 
Laplace transform
\[
F(z)=\int_0^\infty\, e^{-z t}\cdot f(t)dt\tag{4}
\]
Using the equality

\[ 
\int_0^\infty\, e^{-z t}\cdot dt=
\frac{1}{z}
\] 
we see that (2-3) above give

\[
\frac{1}{1+z}\cdot \Phi(1+z)=F(z)-\frac{1}{z}\tag{5}
\]
At this stage we can apply Ikehara's Tauberian Theorem which
applied to the bounded
function $f(t)$ (****) if $F(\zeta)$ extends to an analytic function in some
open set
$\Omega$ which contains the closed half-space
$\mathfrak{Re}(z)\geq 0$.
Hence, using (5) we conclude that the Prime Number theorem
follows from the following:
\bigskip

\noindent
{\bf{4.4 Lemma}}\emph{
The function 
$\frac{1}{1+z}\cdot \Phi(1+z)+\frac{1}{z}$ extends to
be analytic in an open set which contains
$\mathfrak{Re}(z)\geq 0$.}
\medskip

\noindent
\emph{Proof.}
When $\mathfrak{Re}\, z<1$
Euler's product formula gives
\[ 
\log \zeta(z)=\sum\,\log(1-p^{-z})
\]
Passing to the logarithmic derivative we get

\[
-\frac{\zeta'(z)}{\zeta(z)}=
\sum\,\frac{\log p}{p^z-1}
\]
The last sum is rewritten as
\[
\sum\,\frac{\log p}{p^z}-
\sum\,\frac{\log p}{p^z\cdot (p^z-1)}
\]
So the construction of $\Phi$ gives
the equality

\[ 
\Phi(z)=-\frac{\zeta'(z)}{\zeta(z)}
-\sum\,\frac{\log p}{p^z\cdot (p^z-1)}
\]
With $z=1+w$ we can write
\[
\Phi(1+w)-\frac{\zeta'(1+w)}{\zeta(1+w)}=
-\sum\,\frac{\log p}{p^{1+w}\cdot (p^{1+w}-1)}\tag{1}
\]
It is clear the  right hand side is an analytic function of
$w$ in the half-plane $\mathfrak{Re}(w)>-1/2$.
Finally, since the $\zeta$-function has a simple pole at
$z=1$ while $\zeta(1+it)\neq 0$
for all real $t$ we get Lemma 4.4.





























\newpage

\centerline {\bf{5. A uniqueness result for the $\zeta$\vvv function}}
\bigskip

\noindent
{\bf{Introduction.}}
In the introduction  we defined
a class $\mathcal D\uuu k$
of Dirichlet series for every $k>0$ in � 0.1.
The proof of Beurling's  result in (0.0.2)
requires  several steps and 
we  describe some steps before the details of the proof start.
Each  Dirichlet series $\Lambda (s)$ in the family $\mathcal D\uuu k$
 gives an even and entire function of exponential type defined by 
an Hadamard product:
\[ 
f(z)=\prod\, (1+ \frac {z^2}{\lambda\uuu n^2})
\]
Using 
Phragm�n\vvv Lindel�f inequalities together with
properties of the $\Gamma$\vvv function and Mellin's inversion formula,
we shall prove that for each $\epsilon>0$ there exists a constant
$C\uuu\epsilon$
such that the following hold for each real $x>0$:
\[ 
\bigl |f(x)\vvv ax^p\cdot e^{\pi x}\bigr | \leq
C\uuu\epsilon\cdot e^{\pi(1\vvv 2k+\epsilon\cdot x}
\tag{*}
\] 
where
\[
a= e^{2\Lambda(0)}\quad\text{and}\quad  p=2\Lambda'(0)
\]


\noindent
When $k>1/2$ we can choose $\epsilon$ small so that
$1\vvv k+\epsilon=\vvv \delta$ for some $\delta>0$
which 
means that
$f(x) \vvv ax^p\cdot e^{\pi x}$ has exponential decay as $x\to+\infty$.
From this we shall deduce that $f(z)$ is of a 
special form and after deduce Theorem 0.1 
via an  inversion formula for Dirichlet series. First  we shall 
establish a  uniqueness result for entire functions in the class
$\mathcal E$.
 
\bigskip
\centerline{\bf{5.A. On a  uniqueness result in $\mathcal E$}}.
\medskip

\noindent
Let $a$ and $\delta$ be positive real numbers and $p$ some real number.
Consider an even entire function $f(z)$ of exponential type
for which there exists a constant $C$ such that
\[ 
\bigl|f(x)\vvv ax^p\cdot e^{\pi x}\bigr|\leq Ce^{\vvv \delta x}
\quad\colon\,   x\geq 1\tag{*}
\] 


\noindent
We shall prove that
$f$ is of a special form.
In  the half-space $\mathfrak{Re}(z)>0$
we have the analytic  function
\[
h(z)= f(z)\vvv az^p\cdot e^{\pi z}\tag{i}
\]
where the branch of $z^p$ is taken so that $x^p>0$ when
$z=x$ is real and $>0$.
Consider  the domain
\[ 
\Omega=\{z=x+iy\quad y>0\quad\text{and}\quad x>1\}
\]
Since $f\in \mathcal E$ there is a  constant $A$ such that
$e^{\vvv A|z|}\cdot f(z)$ is bounded which gives
constants $C$ and  $B$ such that
\[
 |h(1+iy)|\leq C\cdot e^{By}\tag{ii}
 \] 
for all $y>0$.
At the same time (*) gives
\[
 |h(x)|\leq C\cdot e^{\vvv \delta x}\tag{iii}
\]
The Phragm�n\vvv Lindel�f theorem applied to the quarter planer
$\Omega$ therefore gives a constant $C$ such that
\[ 
|h(x+iy)|\leq Ce^{Ay\vvv \delta x}\quad\text{for all}\quad
x+iy\in\Omega\tag{iv}
\]
In exactly the same way one proves (iv) with $y$ replaced by $\vvv y$
in the quarter\vvv plane where $x>0$ and $y<0$.
Let us then consider the strip domain
\[
S=\{ x+iy\quad |y|\leq 1\quad \text{and}\quad x>1\}
\]
Then we see that there is a constant such that
\[
 h(x+iy)|\leq C\cdot e^{\vvv \delta x}\quad\colon\quad  x+iy\in S\tag{v}
\]
If $n\geq 1$ we consider the complex derivative $h^{(n)}$
and with $x>2$   Cauchy's inequality and (*)
give a constant $C\uuu n$ such that
\[ 
|h^{(n)}(x)|\leq C\uuu n\cdot e^{\vvv \delta x}\quad\colon x\geq 2\tag{vi}
\]
Next, consider the second order
differential operator
\[
L=x^2\partial\uuu x^2\vvv 2p\cdot x\partial\uuu x\vvv \pi^2x^2+
p(p+1)
\]
The  functions $x^pe^{\pi x}$ and $x^pe^{\vvv \pi x}$
are  solutions to
the homogeneous equation $L=0$ when $x> 0$ and hence 
\[ 
L(f)=L(h)
\]
holds on $x>0$.
Now $L$ also yields the holomorphic differential operator
where $\partial\uuu x$ is replaced by $\partial\uuu z$
and here  $g=L(f)$ is en entire function exponential type.
Now (*) above  and the estimates (vi) for $n=0,1,2$ give a constant
$C$ such that
\[
|g(x)|\leq  C(1+x^2)\cdot e^{\vvv \delta|x|}\quad x>0\tag{vii}
\]
Moreover, since $f$ is even it follows that
$g$
is so and hence (vii) hold for all real $x$.
Then the $\mathcal E$\vvv function $g$ is identically zero by
the result in � XXX.
Hence  $f$ satisfies the differential equation
\[ 
L(f)=0\tag{viii}
\]
The uniquenss for solutions of the  in ODE-equation
(viii) gives contants $c_1,c_2$ such that
\[
f(x)= 
c_1x^pe^{\pi x}+
c_2x^pe^{-\pi x}\quad x>0
\]
Since $f$ is an even entire  function it is clear that this entails that
$p$ must be an integer and if we moreover assume that
$f(0)=1$ then the reader may verify that
we have $p=0$ or $p=-1$
which yield corresponding
$f$-functions
\[
f_1(z)=
\frac{e^{\pi z}+
e^{-\pi z}}{2}\quad\colon\,\,
f_2(z)=\frac{e^{\pi z}-
e^{-\pi z}}{2\pi z}
\]











\bigskip

\centerline{\bf{B. Dirichlet series and their transforms.}}


\bigskip

\noindent
Let $0<\lambda\uuu 1\leq\lambda\uuu 2\leq\ldots$
be a non\vvv decreasing sequence of positive real numbers in the family $\mathcal F$
(0.0.2).
It is clear that
the
Dirichlet series
\[
\Lambda(s)=\sum_{n=1}^\infty \frac{1}{\lambda\uuu n^s}\tag{1} 
\]
is analytic in the half\vvv space
$\mathfrak{Re}\, s>1$
and  the results by Hadamard and Lindel�f in
�� XX give the entire function 
\[ 
f(z)=\prod\, \bigl(1+\frac{z^2}{\lambda^2\uuu n}\bigr)\tag{2}
\] 
of exponential type.

\bigskip

\noindent
{\bf{B.1 Inversion formula.}} \emph{One has the equation}
\[
\int\uuu 0^\infty\log f(x)\cdot  \frac{dx}{x^{s+1}}= \frac{\pi\cdot \Lambda(s)}
{s\sin\,\frac{\pi s}{2}}\quad\colon\quad \mathfrak{Re}\, s>1\tag{B.1.}
\]
\medskip

\noindent
\emph{Proof.}
When $0<\mathfrak{Re}\, s<2$ and $a>0$ is real
the reader may verify the
equality:
\[
\int\uuu 0^\infty\, \log\,(1+\frac{x^2}{a^2})\cdot \frac{dx}{x^{s+1}}
=\frac{1}{a^s}\cdot \frac{\pi}{\cdot \sin\,\frac{\pi s}{2}}\tag{i}
\]
\medskip


\noindent
Apply (i) with 
$a=\lambda\uuu n$ and then (B.1) follows
after a summation over
$n$.
\medskip


\noindent
{\bf{B.2 Meromorphic extensions.}}
The inversion formula  (B.1) entails that
$\Lambda$ extends to a meromorphic function in the complex $s$-plane.
To see this we notice that if $x>0$ then the logartihmic derivative
\[
\frac{f'(x)}{f(x)}=\sum\, \frac{2x}{\lambda_n^2+x^2}\tag{i}
\]
Next, a   partial integration gives
\[
(s+1)\cdot
\int\uuu 0^\infty\, \log\,f(x)\cdot \frac{dx}{x^{s+1}}
=
\int\uuu 0^\infty\, \frac{f'(x)}{f(x)}\cdot 
\frac{dx}{x^s}\tag{ii}
\]
Since $\sum\, \lambda_n^{-2}$ is convergent it is clear
that
the right hand side is analytic in
the half-space $\mathfrak{Re}\, s>0$
and by further integrations by parts the reader
may verify that it extends to
a meromorphic function in
the $s$-plane. Together with
the inversion for ua we conclude that
$\Lambda(s)$ extends to a meromorphic
function in the $s$-plane.


\bigskip



\noindent 
{\bf{The case when
$\Lambda\in\mathcal D_k$}}.
Suppose this holds for some
$k>1/2$.
In particular $\Lambda(-2n)=0$ for every positive integer 
and then 
the right hand side in (B.1)
is a meromorphic function whose poles 
are confined to $s=0$ and $s=1$.
Denote this function with
$\Phi(s)$.
Now we shall  estimate  certain $L^1$\vvv integrals.
\medskip

\noindent
{\bf{B.3 Proposition.}}
\emph{There exists a constant $C$ such that}
\[
\int\uuu{\vvv\infty}^\infty\, 
|\Phi(\vvv \sigma+it)|\cdot dt\leq C\cdot
\frac{\sigma^3\cdot \Gamma(\sigma)}{(2\pi k)^\sigma}
 \quad\colon\quad \sigma\geq 2
 \]
 \medskip
 
 \noindent
\emph {Proof.}
Consider the function
\[ 
\psi\uuu *(s)=
(2\pi k)^s\cdot \Gamma(2\vvv s)
\]
The  series expression for 
$\Lambda (s)$ gives a constant $C$ such that
\[ 
\Lambda(3/2+it)|\leq C\quad\colon\quad \vvv \infty <t<+\infty\tag{i}
\] 
It follows that
\[ 
\bigl|\frac{\Phi(3/2+it)}{\psi\uuu *(3/2+it)}\bigr |
 \leq \frac{C\pi}{|3/2+it|}\cdot \frac{1}{2\pi k)^{3/2}} \cdot 
 \frac{1} { \sin\,(\pi(3/4+it/2)\cdot \Gamma(1/2\vvv it)}\tag{ii}
\]
\medskip

\noindent
The complex sine\vvv function increases along this
vertical line, i.e. there is  a constant $c>0$ such that
 \[
 |\sin\,(\pi(3/4+it/2)|\geq c\cdot e^{\pi |t|/2}\tag{iii}
\]
At the same time the result in (� xx) gives the lower bound
\[
| \Gamma(1/2\vvv it)|\geq \sqrt{\pi}\cdot e^{\vvv \pi|t|/2}\tag{iv}
\]
From (iii\vvv iv ) we conclude that
the function $\frac{\Phi}{\psi\uuu *}$ is bounded
on the line $\mathfrak{Re}(s)= 3/2$.
\medskip

\noindent
\emph{Sublemma 1.} \emph{The function
$\frac{\Phi}{\psi\uuu *}$ is a bounded function in the domain}
\[ 
\Omega=\{ \mathfrak{Re}(s)<3/2\}\cap\, \{|s| >2\}
\]

\medskip

\noindent\emph{Proof.}
Follows easily via
the Phragm�n\vvv Lindel�f theorem and the bound above on
$\mathfrak{Re}(s)= 3/2$.

\medskip

\noindent
Next, Sublemma  1 gives a constant $C$ such that
\[ 
|\Phi(s)|\leq C\cdot \bigl|(2\pi k)^s\cdot \Gamma(2\vvv s)|
\quad\colon\quad s\in\Omega\tag{v}
\]

\medskip

\noindent
We shall also need an inequality for the
$\Gamma$\vvv function which
asserts that there
exists a constant $C$ such that
\[
\int\uuu {\vvv \infty}^\infty\, 
 \bigl|\, \Gamma(\sigma+2+it)\,\bigr|\cdot dt\leq 
 C\cdot \sigma^3\cdot \Gamma(\sigma)
 \quad\colon\quad \sigma\geq 2\tag{vi}
 \]
\medskip

\noindent
The verification of (vi) is left to the reader. Together (v) and (vi)
give the inequality in Proposition B.3.

\newpage




\centerline {\bf{B.4 Mellin's inversion formula.}}
\bigskip

\noindent
The integral inequality in Proposition B.3  enable us to apply
the Fourier\vvv Mellin inversion formula via (**) from XX. This gives
\[
\log f(x)= \frac{1}{2\pi i}\cdot
\int\uuu {c\vvv  i\infty }^{c+i\infty}\,
\Phi(s)\cdot x^s\cdot ds\quad\colon\quad 1<c<2
\]
\medskip

\noindent
Using Proposition B.3  we can shift the contour the left and perform
integrals over
lines
$\mathfrak{Re}\, s= \vvv c$ where $c>0$.
During such a shift we pass the poles of
$\Phi$ which appear at $s=0$ and $s=1$. Using condition (a) from (0.2)
in the introduction the
reader can deduce  the integral formula:
\[
\log f(x)\vvv \pi x\vvv
2\phi(0)\cdot \log x\vvv 2\phi'(0) = \frac{1}{2\pi i}\cdot
\int\uuu {\vvv c \vvv  i\infty}^{\vvv c+i\infty}\,
\Phi(s)\cdot x^s\cdot ds \quad \text{for all}\quad c>0\tag{*}
\]
\medskip

\noindent
{\bf{B.4.1 A clever estimate.}}
To profit upon (*)
we  adapt the  $c$\vvv values when $x$ are real and large.
With $x\geq 2$ we take $c=x$ and notice that
\[
|x^{(\vvv x+it)}|= x^{\vvv x}
\]
Proposition B.3  and the triangle inequality show that
the absolute value of the right hand side integral in (*)  is majorized by
\[
2\pi\cdot x^{\vvv x}\cdot C\cdot \frac{x^3\cdot \Gamma(x)}{(2\pi k)^x}
\quad\colon\quad x\geq 2\tag{**}
\]
\medskip

\noindent{\bf{B.4.2 Exercise.}}
Recall that $\Gamma(N)= N !$ for positive integers. Use this and
Stirling's formula to conclude that
for every $\epsilon>0$ there is a constant $C\uuu\epsilon$
such that (**)  is majorized by
\[
C\uuu\epsilon\cdot e^{\vvv 2\pi(k\vvv \epsilon)x}
\]


\noindent
{\bf{B.4.3 Consequences.}}
With $x\geq 2$, 
$p=2\phi(0)$ and $a=2\phi'(0)$ we obtain from above:

\[
\bigl|\log f(x)\vvv \pi x\vvv
p\cdot \log x\vvv a\bigr| \leq
C\uuu\epsilon\cdot e^{\vvv 2\pi(k\vvv \epsilon)x}
\]

\medskip

\noindent
Now
(B.4.3) gives
a constant
$C^*\uuu\epsilon$ such that
\[ 
|f(x)\vvv e^a\cdot x^p\cdot e^{\pi x}|\leq 
C^*\uuu\epsilon\cdot e^{\pi(1\vvv 2k+2\epsilon)\cdot x}
\quad\colon\quad x\geq 2\tag{B.4.4}
\]
\medskip

\noindent
Since $k>1/2$ and $\epsilon$ can be arbitrary small
we conclude the foillowing

\medskip

\noindent
{\bf{B.4.5 Theorem.}}
\emph{When $\Lambda\in \mathcal D_k$ for some $k>1/2$ there
exists $\delta>0$ and constants $a,p$ such that}
\[
|f(x)\vvv e^a\cdot x^p\cdot e^{\pi x}|\leq 
C^*\uuu\epsilon\cdot e^{-\delta\cdot x}
\quad\colon\quad x\geq 2
\]




\newpage







STRUL to be deleted



\bigskip









\noindent
{\bf{B.2 A condition on $f$}}.
Suppose there exist real constants $a,p,k$  where $k>1/2$
such that 
\[
f(x)= ax^pe^{\pi x}+O(e^{\pi(1+\epsilon\vvv 2k)x}
\] 
hold for each $\epsilon>0$ as $x\to +\infty$.
In other words, for every $\epsilon>0$ there exists a constant
$C\uuu \epsilon$ such that
\[
|f(x)\vvv ax^pe^{\pi x}|\leq C\uuu\epsilon\cdot
e^{\pi(1+\epsilon\vvv 2k)x}\quad\text{for all}\quad x\geq 1\tag{*}
\]
\medskip

\noindent
{\bf{Exercise.}}
Show that (*) entails that
\[
\log f(x)=\log a+p\log x+\pi\cdot x+
O(e^{\pi(1+\epsilon\vvv 2k)x}\quad\text{for all}\quad x\geq 1\tag{**}
\]
where $\epsilon>0$ can be arbitrary small.
\medskip

\noindent
Next, set
\[
\psi(s)=\int\uuu 0^\infty\, \log\,f(x)\cdot \frac{dx}{x^{s+1}}
\]
Notice that $f$ is an even entire function where
the Hadamard product gives $f(0)=1$ which implies that
\[ 
\log f(x)\simeq x^2
\] 
when $x$ is close to zero.
Therefore it is only the behaviour of $f(x)$ when $x$ is large
which determines whether the $\psi$\vvv function
is nice or not in the half\vvv space $\mathfrak{Re}\, s<2$.
Notice that

\[
\int\uuu 0^\infty\,[\log a+p\log x+\pi\cdot x]\cdot \frac{dx}{x^{s+1}}
=\frac{\log a}{s}+\frac{p}{s^2}+\frac{\pi}{s\vvv 1}
\]
Moreover, we have 
$k>1/2$ so when $\epsilon$ is small it follows that
$1+\epsilon\vvv 2k=\vvv \delta$ for some $\delta>0$ and
the function
\[ 
g\uuu\delta(s)=  \int\uuu 0^\infty e^ {\vvv \pi\delta x}\cdot \frac{dx}{x^{s+1}}
\]
is analytic in the half\vvv space $\mathfrak{Re}\, s< 2$.
\medskip

\noindent
{\bf{Exercise.}}
Use the definition of the $\Gamma$\vvv function to conclude that
\[
|g\uuu\delta(\vvv \sigma+it)|\leq 
\frac{\Gamma(\sigma)}{(\pi \delta)^\sigma}
\quad\text{hold for every}\quad  \sigma>0
\]






\bigskip



\noindent Let us now add the assumption that
\[
\phi(\vvv 2n)=0\quad\text{hold  for  every positive integer}\tag{*}
\]

\medskip

\noindent
When (*) holds we see that
the right hand side in (B.1.2)
is a meromorphic function whose poles in the half\vvv plane
$\mathfrak{Re} s<2$ are confined to $s=0$ and $s=1$.
Denote this function with
$\Phi(s)$.
Next, by assumption $\phi$ also belongs  $\mathcal D\uuu k$
for some $k>1/2$ and  using  the growth condition (xx) from � XX
we shall  estimate  certain $L^1$\vvv integrals.
\medskip

\noindent
{\bf{B.3 Proposition.}}
\emph{There exists a constant $C$ such that}

\[
\int\uuu{\vvv\infty}^\infty\, 
|\Phi(\vvv \sigma+it)|\cdot dt\leq C\cdot
\frac{\sigma^3\cdot \Gamma(\sigma)}{(2\pi k)^\sigma}
 \quad\colon\quad \sigma\geq 2
 \]
 \medskip
 
 \noindent
\emph {Proof.}
Consider the function
\[ 
\psi\uuu *(s)=
(2\pi k)^s\cdot \Gamma(2\vvv s)
\]
Next, the convergent series expression for the given Dirichlet series
$\phi(s)$ gives a constant $C$ such that
\[ 
\phi(3/2+it)|\leq C\quad\colon\quad \vvv \infty <t<+\infty\tag{i}
\] 
It follows that
\[ 
\bigl|\frac{\Phi(3/2+it)}{\psi\uuu *(3/2+it)}\bigr |
 \leq \frac{C\pi}{|3/2+it|}\cdot \frac{1}{2\pi k)^{3/2}} \cdot 
 \frac{1} { \sin\,(\pi(3/4+it/2)\cdot \Gamma(1/2\vvv it)}\tag{ii}
\]
\medskip

\noindent
The complex sine\vvv function increases along this
vertical line,i.e. there is  a constant $c>0$ such that
 \[
 |\sin\,(\pi(3/4+it/2)|\geq c\cdot e^{\pi |t|/2}\tag{iii}
\]
At the same time the result in (� xx) gives the lower bound
\[
| \Gamma(1/2\vvv it)|\geq \sqrt{\pi}\cdot e^{\vvv \pi|t|/2}\tag{iv}
\]
From (iii\vvv iv ) we conclude that
the function $\frac{\Phi}{\psi\uuu *}$ is bounded
on the line $\mathfrak{Re}(s)= 3/2$.
\medskip

\noindent
\emph{Sublemma 1.} \emph{The function
$\frac{\Phi}{\psi\uuu *}$ is a bounded function in the domain}
\[ 
\Omega=\{ \mathfrak{Re}(s)<3/2\}\cap\, \{|s| >2\}
\]

\medskip

\noindent\emph{Proof.}
Follows easily via
the Phragm�n\vvv Lindel�f theorem and the bound above on
$\mathfrak{Re}(s)= 3/2$.

\medskip

\noindent
Next, Sublemma  1 gives a constant $C$ such that
\[ 
|\Phi(s)|\leq C\cdot \bigl|(2\pi k)^s\cdot \Gamma(2\vvv s)|
\quad\colon\quad s\in\Omega\tag{v}
\]

\medskip

\noindent
We shall also need an inequality for the
$\Gamma$\vvv function which
asserts that there
exists a constant $C$ such that
\[
\int\uuu {\vvv \infty}^\infty\, 
 \bigl|\, \Gamma(\sigma+2+it)\,\bigr|\cdot dt\leq 
 C\cdot \sigma^3\cdot \Gamma(\sigma)
 \quad\colon\quad \sigma\geq 2\tag{vi}
 \]
\medskip

\noindent
The verification of (vi) is left to the reader. Together (v) and (vi)
give the inequality in Proposition B.3.

\bigskip

\medskip


\centerline {\bf{B.4 Mellin's inversion formula.}}
\medskip

\noindent
The integral inequality in Proposition B.3  enable us to apply
the Fourier\vvv Mellin inversion formula via (**) from XX. This gives
\[
\log f(x)= \frac{1}{2\pi i}\cdot
\int\uuu {c\vvv  i\infty }^{c+i\infty}\,
\Phi(s)\cdot x^s\cdot ds\quad\colon\quad 1<c<2
\]
\medskip

\noindent
Using Proposition B.3  we can shift the contour the left and perform
integrals over
lines
$\mathfrak{Re}\, s= \vvv c$ where $c>0$.
During such a shift we pass the poles of
$\psi$ which appear at $s=0$ and $s=1$
with residues described in �� XX - at start - above.
From this the reader can deduce  the integral formula:
\[
\log f(x)\vvv \pi x\vvv
2\phi(0)\cdot \log x\vvv 2\phi'(0) = \frac{1}{2\pi i}\cdot
\int\uuu {\vvv c \vvv  i\infty}^{\vvv c+i\infty}\,
\Phi(s)\cdot x^s\cdot ds \quad \text{for all}\quad c>0\tag{*}
\]
\medskip

\noindent
{\bf{B.6 A clever estimate.}}
To profit upon (*)
we shall adapt the  $c$\vvv values when $x$ are real and large.
More precisely, with $x\geq 2$ we take $c=x$ and notice that
\[
|x^{(\vvv x+it)}|= x^{\vvv x}
\]
Then Proposition B.4  and the triangle inequality show that
the absolute value of the right hand side integral in (*)  is majorized by
\[
2\pi\cdot x^{\vvv x}\cdot C\cdot \frac{x^3\cdot \Gamma(x)}{(2\pi k)^x}
\quad\colon\quad x\geq 2\tag{**}
\]
\medskip

\noindent{\bf{B.7 Exercise.}}
Recall that $\Gamma(N)= N !$ for positive integers. Use this and
Stirling's formula to conclude that
for every $\epsilon>0$ there is a constant $C\uuu\epsilon$
such that (**)  is majorized by
\[
C\uuu\epsilon\cdot e^{\vvv 2\pi(k\vvv \epsilon)x}
\]
\medskip

\noindent
{\bf{B.8 Consequences.}}
With $x\geq 2$, 
$p=2\phi(0)$ and $a=2\phi'(0)$ we obtain from above:

\[
\bigl|\log f(x)\vvv \pi x\vvv
p\cdot \log x\vvv a\bigr| \leq
C\uuu\epsilon\cdot e^{\vvv 2\pi(k\vvv \epsilon)x}
\]

\medskip

\noindent
{\bf{B.9 Exercise.}}
Deduce from this estimate
that there is a constant
$C^*\uuu\epsilon$ such that
\[ 
|f(x)\vvv e^a\cdot x^p\cdot e^{\pi x}|\leq 
C^*\uuu\epsilon\cdot e^{\pi(1\vvv 2k+2\epsilon)\cdot x}
\quad\colon\quad x\geq 2
\]







\newpage


\centerline{\bf{6. A theorem on functions defined by a semi\vvv group}}
\bigskip

\noindent
Let $f(x)$ be a complex-valued function in $L^2(0,1)$ which is 
not identically zero on any interval $(0,\delta)$ with
$0<\delta<1$, i.e.
for each $\delta>0$ one has
\[
\int\uuu 0^\delta\, |f(x)|\cdot dx>0\tag{*}
\]
Next, for each $0<a<1$ we set
\[ 
f\uuu a(x)=f(ax)
\] 
We restrict each  $f\uuu a$ to  $(0,1)$
and denote by $\mathcal C\uuu f$
the linear space generated by $\{f\uuu a\}$ as $0<a<1$.
Thus, a function in $\mathcal C\uuu f$
is expressed as a finite sum
\[
\sum\, c\uuu k\cdot f\uuu{a\uuu k}(x)
\]
where $\{c\uuu k\}$ are complex numbers and 
$0<a\uuu 1<\ldots<a\uuu n<1$ some finite 
tuple of points in $(0,1)$.
Consider  some  $1<p<2$.
The inclusion $L^2(0,1)\subset L^p(0,1)$
identifies $\mathcal C\uuu f$ with a subspace of
$L^p(0,1)$ and 
its closure  in the Banach space $L^p(0,1)$.
is denoted by $\mathcal C\uuu f(p)$.



\medskip


\noindent
{\bf{0.1 The function $F(s)$}}. It is defined by
\[
F(s)=\int\uuu 0^1\, f(x)\cdot x^{s\vvv 1}\cdot ds\tag{1}
\]
Here  $F$ is analytic
in  the
half\vvv plane 
$\mathfrak{Re}(s)>1/2$. Indeed, with
$\sigma= \mathfrak{Re}(s)>1/2$
the Cauchy\vvv Schwarz inequality gives
\[ 
|F(\sigma+it)|\leq
\sqrt{\int\uuu 0^1\, |f(x)|^2\cdot dx}\cdot
\sqrt{\int\uuu 0^1\, |x|^{2\sigma\vvv 2}\cdot dx}=||f||\uuu 2\cdot
\sqrt{\frac{1}{2\sigma\vvv 1}}\tag{2}
\]
\medskip


\noindent
{\bf{6.1 Theorem}} \emph{If there exists some
$1<p<1$ such that
$\mathcal C\uuu f(p)$ is a proper subspace of
$L^p[0,1]$, then
$F(s)$ extends to a meromorphic function in the whole complex
$s$\vvv plane whose poles are confined to the open half\vvv plane
$\mathfrak{Re}(s)<1/2$. Moreover, for every  pole
$\lambda$ the function
$x^{\vvv\lambda}$ belongs to   $\mathcal C\uuu f(p)$.}

\medskip

\noindent
\emph{Proof.}
Recall that $L^q(0,1)$ is the dual of $L^p(0,1)$ where
$\frac{1}{q}= 1\vvv \frac{1}{p}$. The assumption that
$\mathcal C\uuu f(p)\neq L^p(0,1)$ gives 
a non\vvv zero $k(x)\in L^q(0,1)$ such that
\[
\int\uuu 0^1\, k(x) f(ax)\cdot dx=0
\quad\colon\quad 0<a<1\tag{1}
\]
To the $k$\vvv function we associate
the transform
\[
 K(s)= \int\uuu 0^1\, k(x)\cdot x^{\vvv s}\cdot dx\tag{2}
\]
\medskip

\noindent
H�lder's inequality implies that
$K(s)$ is analytic in the half\vvv plane $\mathfrak{Re} s<\frac{1}{p}$.
Next, we define a function $g(\xi)$ for every real $\xi>1$ by
\[
 g(\xi)=\int\uuu 0^1\, k(x)\cdot f(\xi x)\cdot dx
\]
H�lder's inequality gives
\[ 
|g(\xi)|\leq 
\bigl[\int\, |k(x)|^q\cdot dx\bigr]^{\frac{1}{q}}\cdot
\bigl[\int\uuu 0^{1/\xi}\, |f(\xi x)|^p\cdot dx\bigr]^{\frac{1}{p}}
\]
With $\xi>1$ we notice that the last factor after a variable
substitution is equal to
\[
||f||\uuu p\cdot |\xi|^{\vvv 1/p}
\]
Since the $L^p$\vvv norm of $f$ is majorized by its $L^2$\vvv norm
we conclude that
\[
|g(\xi)\leq ||k||\uuu q\cdot ||f||\uuu p\cdot \xi^{\vvv 1/p}\quad\colon\quad \xi>1\tag{3}
\]
Next, put
\[
G(s)= \int\uuu  1^\infty\, g(\xi)\cdot \xi^{s\vvv 1}\cdot d\xi\tag{4}
\]


\noindent
From (3) it follows that
$G(s)$ is analytic in the half\vvv space
$\mathfrak{Re}\, s<1/p$.
Consider  the strip domain:
\[
\square=1/2<\mathfrak{Re}\, s< 1/p
\]

\noindent
Variable substitutions of  double integrals show that the following holds
in $\square$:
\[
 G(s)= F(s)\cdot K(s)\tag{*}
\]


\noindent
\emph{Conclusion.}
If follows from (*) that
$F$ extends to a meromorphic function in the whole $s$\vvv plane.
The
inequality (2) from 0.1 which
shows that
no poles appear during the meromorphic continuation across 
$\mathfrak{Re}\, s=1/2$. Hence
$F$ either is an entire function or else  it has a
non\vvv empty set of
poles where each pole
$\lambda$ has  real part  $<1/2$.
At this stage we are prepared to finish the proof of Theorem 6.1
and begin with:

\medskip




\noindent{\bf{Existence of at least one pole.}}
There remains to prove that $F$ has at least one pole.
We prove this by a contradiction, i.e. suppose that
$F$  is  an entire function and 
consider 
a real number 
$1/2<\alpha<1/p$.
The construction of $F$ shows  that its
restriction to the half\vvv space
$\mathfrak{Re}\, s\geq \alpha$ is bounded and it is also clear that
\[
\lim\uuu{\sigma\to +\infty}
F(\sigma+it)=0\tag{i}
\]



\noindent
Next, in the half\vvv space 
$\mathfrak{Re}\,s\leq\alpha$
we know that
$F=\frac{G}{K}$
where  $G$ and $K$ both are  bounded and at the same time their quotient is
analytic in this half\vvv space.
Moreover their constructions imply that
\[ 
\lim\uuu{\sigma\to \vvv \infty}\, G(\sigma+it)=0\quad\text{and}|\quad
K(\sigma+it)=0
\] 
Next,   the result by F. and R. Nevanlinna from XX
gives some 
$M>0$ and  a real number  $c$ such that
\[
F(\sigma+it)|\leq M\cdot e^{c(\sigma\vvv \alpha})
\quad\text{holds when}\quad \sigma\leq \alpha\tag{iii} 
\]


\noindent
If $c\geq 0$  we see  that
the entire function $F$ is bounded and
(i) implies  that $F=0$. But this is impossible since it
entails that $f=0$.
\medskip

\noindent
{\bf{The case $c<0$.}}
When this holds we set $a=e^c$ so that $0<a<1$
and define
\[ 
F\uuu 1(s)= \int\uuu 0^1\, f(ax)x^{s\vvv 1}\cdot ds\tag{iv}
\]



\noindent 
Here a  variable substitution  gives
\[
F\uuu 1(s)=a^s\bigl( F(s)\vvv
\int\uuu a^1\, f(x)x^{s\vvv 1}\cdot ds\bigr)\tag{v}
\]
It follows that the entire function $F\uuu 1(s)$ is bounded
so  by Liouville's theorem it is identically zero.
Si by (iv) the
the transform of the function
$f\uuu a(x)=f(ax)$ is identically zero. This means precisely that
$f$ vanishes on the interval $[0,a]$. But this was
excluded by condition (*)  above Theorem 6.1. which 
shows that $F$ cannot be an entire function.

\medskip

\noindent
{\bf{The case at pole}}
Suppose that $F$ as a pole at some
$\lambda$ with  real part  $<1/2$.
Since $G$ is analytic in $\mathfrak{Re}(s)<1/2$
the equality (*)
implies that $\lambda$ is  a zero of $K$.
Notice also that the presence  of the 
pole of $F$ at $\lambda$ is \emph{independent}
of the chosen 
$L^q$\vvv function $k$
which is $\perp$ to $\mathcal C\uuu f$. Hence the following 
implication holds:
\[
k\perp \mathcal C\uuu f(p)\implies 
K(\lambda)= \int\uuu 0^1\, k(x) x^{\vvv \lambda}\cdot dx=0
\]
The Hahn\vvv Banach theorem  entails
that
the $L^p$\vvv function $x^{\vvv\lambda}$ belongs to
$\mathcal C\uuu h(p)$ which proves  the last claim in
Theorem 6.1





\newpage

\centerline {\bf{7. Beurlings criterion for the Riemann hypothesis}}
\bigskip

\noindent
Let $\rho(x)$ denote the 1\vvv periodic function on the positive real $x$\vvv line
with $\rho(x)=x$ if $0<x<1$.
So if $\{x\}$ is the integral part of $x$ then
\[
\rho(x)= x\vvv\{x\}
\]
To each $0<\theta<1$ we get the function
\[ 
\rho\uuu\theta(x)= \rho(\theta/x)
\]
whose  restriction   to $(0,1)$
gives a non\vvv negative function 
with  jump\vvv discontinuites at the discrete set 
of $x$\vvv values  where $\theta/x$ is an integer.
Denote by $\mathcal D$
the linear space 
of functions on $(0,1)$ of the form
\[
f(x)=\sum\, c\uuu\nu\cdot \rho(\theta\uuu \nu/x)
\]

\noindent
where $0<\theta\uuu 1<\ldots<\theta\uuu N<1$ is a finite set
and $\{c\uuu\nu\}$ complex numbers such that
\[
\sum\, c\uuu\nu\cdot \theta\uuu\nu=0
\]


\noindent
{\bf{7.1 Theorem.}} \emph{The Riemann hypothesis is valid if and only if
the identity function 1 belongs to the closure of
$\mathcal D$ in
$L^2(0,1)$.}
\medskip

\noindent
The proof will use the following formula:

\medskip

\noindent
{\bf{7.2 Proposition.}}
\emph{ For each $0<\theta<1$ one has the equality}
\[
 \int\uuu 0^1\,\rho(\theta/x)x^{s\vvv 1}\cdot dx=
 \frac{\theta}{s\vvv 1}\vvv 
 \frac{\theta^s\cdot \zeta(s)}{s}
\quad\text{when}\quad \mathfrak{Re}s >1\tag{*}
\]
\medskip
 
\noindent
\emph{Proof.}
The variable substitutions $x\to \theta\cdot y$ and $y\to 1/u$ 
identifies the left hand side with
\[
\theta^s\cdot \int\uuu  0^{1\theta}\, \rho(1/y)\cdot y^{s\vvv 1} \cdot dy=
\theta^s\cdot 
\int\uuu\theta^\infty\,\rho(u)\cdot u^{\vvv s\vvv 1}\cdot du\tag{i}
\]
Next, since $\rho$ is periodic we have
\[
\int\uuu1^\infty\,\rho(u)\cdot u^{\vvv s\vvv 1}\cdot du=
\sum\uuu{n=1}^\infty\, \int\uuu 0^1\,\frac{u}{(u+n)^{s+1}}\cdot du\tag{ii}
\]
An integration by parts gives for each $n\geq 1$:
\[
\int\uuu 0^1\,\frac{u}{(u+n)^{s+1}}\cdot du=
\vvv \frac{1}{s}(n+1)^{\vvv s}+\frac{1}{s}\int\uuu 0^1\, \frac{du}{(n+u)^s}
\]
After a summation over $n$ we see that (ii) becomes
\[
\vvv \frac{\zeta(s)}{s}+\frac{1}{s}+\frac{1}{s}\int\uuu 1^\infty\, u^{\vvv s}\cdot du
=\vvv \frac{\zeta(s)}{s}+\frac{1}{s}+\frac{1}{s(s\vvv 1)}=
\vvv \frac{\zeta(s)}{s}+\frac{1}{s\vvv 1}
\]
It follows that the left hand side in (*) is equal to
\[
\theta^s\cdot \bigl [\int\uuu \theta^1\, u\cdot u^{\vvv s\vvv 1}\cdot du
\vvv \frac{\zeta(s)}{s}+\frac{1}{s\vvv 1}\,\bigr]=
\]
\[
\theta^s\cdot \bigl[ \frac{\theta^{\vvv s+1} \vvv 1}{s\vvv1}
\vvv \frac{\zeta(s)}{s}+\frac{1}{s\vvv 1}\,\bigr]=\frac{\theta}{s\vvv 1}\vvv
\frac{\theta^s\cdot \zeta(s)}{s}
\]

\newpage



\noindent Now we are prepared to begin the proof of Theorem 7.1
and begin with the sufficiency.


\medskip

\centerline{\emph{2. The case when $1$ in the $L^2$\vvv closure of $\mathcal D$.}}
\medskip


\noindent
If   $\epsilon>0$ this assumption
gives some 
$f\in\mathcal D$ such that the $L^2$\vvv norm
of $1+f$ is $<\epsilon$. 
Since $\sum\, c\uuu \nu\cdot\theta\uuu\nu =0$,
Proposition 7.2 gives:
\[
\int\uuu 0^1\,(1+f(x))\cdot x^{s\vvv 1}\cdot dx=
\frac{1}{s}
 \vvv \frac{\zeta(s)}{s}\cdot \sum\, c\uuu\nu\cdot \theta\uuu\nu ^s
\] 
With $s=\sigma+it$ and  $\sigma>1/2$
we have $x^{s\vvv 1}$ in $L^2$ and Cauchy\vvv Schwarz inequality gives:
\[
\bigl|\int\uuu 0^1\,(1+f(x))\cdot x^{s\vvv 1}\cdot dx\,\bigr|\leq
||f||\uuu 2\cdot \sqrt{\int\uuu 0^1\, x^{2\sigma\vvv 2}\cdot dx}=
||f||\uuu 2\cdot \frac{1}{\sqrt{2\sigma\vvv 1}}
\]
Hence we obtain
\[
|\frac{1}{s}
\vvv \frac{\zeta(s)}{s}\cdot \sum\, c\uuu\nu\cdot \theta\uuu\nu ^s|\leq
 \epsilon\cdot \frac{1}{\sqrt{2\sigma\vvv 1}}\quad\colon\quad
 \sigma>1/2
\]
If $\zeta(s\uuu *)=0$ holds for some $s\uuu *=\sigma\uuu *+it\uuu *$
with $\sigma\uuu *>1/2$, 
the left hand side is reduced to $\frac{1}{|s\uuu *|}$. 
Since
we can find $f$ as above for every $\epsilon>0$ it would follow that
\[
\frac{1}{|s\uuu *|}\leq \epsilon\cdot 
\frac{1}{\sqrt{2\sigma\uuu *\vvv 1}}\quad\text{for every}\quad 
\epsilon>0
\]
But this is clearly impossible so
if 1 belongs to the $L^2$\vvv closure of $\mathcal D$ then
the Riemann\vvv Hypothesis holds.
\bigskip

\noindent
\centerline{\emph{3. Proof of necessity.}}
\medskip

\noindent
There remains to show that if
1 is outside the $L^2$\vvv closure of $\mathcal D$
then the $\zeta$\vvv function has a zero in
the half\vvv plane $\mathfrak{Re}\, s>1/2$.
To show this we 
introduce a family of linear operators $\{T\uuu a\}$
as follows:
If $0<a<1$ and $g(x)$ is a function on $(0,1)$
we set
\[ 
T\uuu a(g)(x)= g(x/a)\quad\colon\quad 0<x<a
\]
while $T\uuu a(g)=0$ when $x\geq a$.

\medskip

\noindent
{\bf{Exercise.}}
Show that each $T\uuu a$ maps $\mathcal D$ into itself
and one has the inequality
\[
||T\uuu a(f)||\uuu 2\leq ||f||\uuu 2
\]


\noindent
Since 
$1$ is outside the $L^2$\vvv closure of $\mathcal D$
its orthogonal complement in the Hilbert space
is $\neq 0$ which gives
a non\vvv zero
$g\in L^2(0,1)$ such that
\[
\int\uuu 0^1\, f(x)\cdot g(x)\cdot dx=0
\quad\colon\quad f\in\mathcal D\tag{*}
\]
Since $\mathcal D$ is invariant under the $T$\vvv operators it follows that
if $0<a<1$ then we also have
\[
0=\int\uuu 0^a\, f(x/a)\cdot g(x)\cdot dx=
a\cdot \int\uuu 0^1\, f(x)\cdot g(ax)\cdot dx\tag{1}
\]


\noindent
At this stage we  apply Theorem 6.1.
To begin with we show that the $g$\vvv function satisfies
(*) in Theorem 6.1.
For suppose that $g=0$ on some interval $(0,a)$ with $a>0$.
Choose some $b$ where
\[
a<b<\min (1,2a)
\]
Now $\mathcal D$ contains the function
$f(x)= b\rho(x/a)\vvv a\rho(x/b)$.
The reader may verify that $f(x)=0$ for $x>b$ 
and is equal to the constant $a$ on $(a,b)$.
With (1) applied to $f$ we therefore get
\[
\int\uuu a^b\, g(x)\cdot dx=0
\]
This 
means that the primitive function
\[
 G(x)=\int\uuu 0^x\, g(u)\cdot du
 \]
has a vanishing derivative on the interval $(a,b)$. The derivative is also zero  on
$(0,a)$ where $g=0$.
We conclude that $G=0$ on the interval $(0,b)$ so the
$L^2$\vvv function $g$ is almost everywhere a fixed constant on this interval.
But this constant is zero since $g=0$ on $(0,a)$. Hence we have shown that
$g=0$ on the whole interval $(0,b)$.
We can repeat this with $a$ replaced by $b$
and conclude that $g$ also is zero on the interval
\[ 
0<x<\min(1,2b) =\min(1,4a)
\]
After a finite number of steps $2^m a\geq 1$ and 
hence $g$ would be identically zero on $(0,1)$ which is not the case.
Hence  Theorem 6.1 applies to  $g$
and gives
some $\lambda\uuu *$ with $\mathfrak{Re}\, \lambda\uuu *<1/2$
such that $x^{\vvv \lambda\uuu *}$
belongs to $\mathcal C\uuu g(p)$ for every
$p<2$.
Next, for each $\theta>0$ we get
the $\mathcal D$\vvv function
\[
x\mapsto \rho(1/x)\vvv \frac{1}{\theta}\cdot \rho(\theta/x)
\]
Since (1) holds for all $0<a<1$ it follows that
\[
\int\uuu 0^1\, [\rho(1/x)\vvv \frac{1}{\theta}
\cdot \rho(\theta/x)]\cdot x^{\vvv\lambda\uuu *}
\cdot dx=0\tag{2}
\]
Put
$s\uuu *=1\vvv\lambda\uuu *$.
The formula in Proposition 7.2 
shows  that the vanishing in
(2)  gives
\[
\frac{\theta^{s\uuu *}\vvv 1}{s\uuu *}\cdot \zeta(s\uuu *)=0
\]
This hold for every $0<\theta<1$ and  we can choose $\theta$
 so that
$\theta^{s\uuu *}\vvv 1\neq 0$ which would give a  zero
$\zeta(s\uuu *)=0$
where  $\mathfrak {Re}(s\uuu *)=1
\vvv \mathfrak {Re}(\lambda \uuu *)>1/2$.



\newpage

 









\end{document}













