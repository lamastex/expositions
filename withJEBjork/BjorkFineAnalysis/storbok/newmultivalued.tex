
\documentclass{amsart}


\usepackage[applemac]{inputenc}


\def\uuu{_}

\def\vvv{-}





\begin{document}



\centerline {\bf 7. The $p^*$-function.}
\medskip

\noindent
We  construct a special harmonic function which 
will be  used
to get solutions to the Dirichlet problem in
XXX.
Let $\Omega$ be an open and connected set in
${\bf{C}}$.
Its closed complement has connected components.
Let $E$ be
such a connected component.
To each $a\in E$ we get the winding number
$\mathfrak{w}_a(\gamma)$. If $b$  is another point in
$E$ which is sufficiently close to  $a$
it is clear that
\[
\bigl|\frac{1}{\gamma(t)-a}-\frac{1}{\gamma(t)-b}\bigr|<
\bigl|\frac{1}{\gamma(t)-a}\bigr |
\]


\noindent
Rouche's theorem from 1.4 implies that
$\mathfrak{w}_a(\gamma)=\mathfrak{w}_b(\gamma)$, i.e. 
for every closed curve $\gamma$ in $\Omega$,
the winding number
stays constant in each connected component of
${\bf{C}}\setminus\Omega$. 
This  enable us to construct single valued  Log-functions
in $\Omega$.
Namely, let $a\in E$ where $E$ is a connected
componen in the complement of
$\Omega$.
Consider 
$f=\text{Log}\,(z-a)$ and choose a single valued branch
$f_*$ at some point $z_0\in\Omega$. 
If $\gamma\subset\Omega $ is a closed curve with initial point at
$z_0$ the analytic continuation along $\gamma$ of the Log-function
gives:
\[
T_\gamma(f_*)= f_*+2\pi i\cdot \mathfrak{w}_a(\gamma)\tag {1}
\]


\noindent
Next, if $b$ is another point in $E$ we consider
$g_*=\text{Log}(z-b)$ and obtain
\[
T_\gamma(g_*)= g_*+2\pi i\cdot \mathfrak{w}_b(\gamma)\tag {2}
\]



\noindent
Since
$\mathfrak{w}_b(\gamma)=\mathfrak{w}_b(\gamma)$ it follows that
\[
T_\gamma(f_*)-T_\gamma(g_*)=f_*-g_*\tag{3}
\]
Hence the difference $\text{Log}(z-a)-\text{Log}(z-b)$
is a \emph{single valued} analytic function in $\Omega$. Taking  
is exponential we
find $\Psi(z)\in\mathcal O(\Omega)$ such that
\[
e^{\Psi(z)}=
\frac{z-a}{z-b}\tag{4}
\]


\noindent
Since $a\neq b$
we see that
$\Psi(z)\neq 0$
for all $z\in\Omega$. Next, we get the
harmonic function defined in 
$\Omega$
by
\[
p(z)=\mathfrak{Re}\bigl(\frac{1}{\Psi(z))}\bigr)=
\frac{\mathfrak{Re}(\Psi(z)}{|\Psi(z)|^2}\tag{*}
\]


\noindent
Notice that
$\mathfrak{Re}(\Psi(z))=\text{Log}\, |z-a|-\text{Log}|z-b|$
and since
$\text{Log}\, |z-a|\to-\infty$ as $z\to a$
we see from (*) that
\[
\lim_{z\to a}\, p(z)=0\tag{**}
\]


\noindent
Notice also that
$\Psi(z)$ exends to a continuous function on
$\bar\Omega\setminus (a,b)$ and we can
choose a small $\delta>0$ such that
\[ 
\text{Log}|z-a|-\text{Log}\,|z-b|<-1\quad\colon\quad
|z-a|\leq\delta\tag{ii}
\]

\noindent Then (i) and (ii) give

\medskip

\noindent {\bf 7.1 Theorem.}
\emph{Let $a\in\partial\Omega$ be such that
the connected component of ${\bf{C}}\setminus \Omega$
which contains $a$ is not reduced to the single point $a$.
Then there exists a harmonic function
$p^*(z)$
in $\Omega$
for which}
\[
\lim_{z\to a}\, p^*(z)=0
\]
\emph{and there exists $\delta>0$ such that}
\[
\max_{\{|z-a|=r\}\,\cap\Omega}\, p^*(z)<0\quad\colon\quad
z\in D_a(r)\,\cap\,\Omega
\]


\newpage

\centerline{\bf\large  Chapter 4: Multi-valued analytic functions}

\bigskip

\noindent
0. Introduction
\medskip




\noindent
1. Angular variation and Winding numbers
\medskip

\noindent
2. The argument principle
\medskip

\noindent
3. Multi-valued functions

\medskip


\noindent
4. The monodromy theorem
\medskip

\noindent
5. Homotopy and Covering spaces
\medskip

\noindent
6. The uniformisation theorem
\medskip


\noindent
7. The $\mathfrak{p}^*$-function


\medskip


\noindent
8. Reflections across a boundary


\medskip

\noindent
9. The elliptic modular function
\medskip


\noindent 
10. Work by Henri Poincar�





\bigskip



\noindent
\centerline {\bf\large{ Introduction.}}
\medskip


\noindent
In section 1 we define winding numbers of curves
in ${\bf{R^2}}$
where no complex variables appear.
After this we study
complex line integrals where
Theorem 1.7  is the first main result in this chapter.
The second main result is
Theorem 2.1, referred to as the
\emph{argument principle}. It gives  a gateway to
find zeros of an analytic function since the counting function of its
zeros is expressed by winding numbers arising from
the image under the given function of closed boundary curves to a domain where
we seek the number of zeros of $f$.
\medskip

\noindent Section 3 starts with 
a construction due to  Weierstrass
and leads to  analytic continuation of   germs of
analytic functions. To grasp his  construction 
one 
considers the \emph{total sheaf space} 
$\widehat{\mathcal O}$
whose stalks are germs of analytic functions 
at points in ${\bf{C}}$.
This topological manifold  is locally homeomorphic to
small discs in $\bf C$ which   express 
germs of multi-valued
functions $f$.
If  $\rho$ denotes the local homemorphism from 
$\mathcal {\widehat O}$ onto $\bf C$.  Weierstrass's
construction 
gives for every open and connected subset
$\Omega$ of ${\bf{C}}$ 
a 1-1 correspondence between connected open subsets of
$\mathcal {\widehat O}$ whose
$\rho$\vvv image
is equal to $\Omega$,  and the class of multi-valued analytic functions in
$\Omega$. In this correspondence
one does not exclude  functions  which may
have analytic continuation to larger sets.
In � 4 we prove the 
\emph{Monodromy Theorem}  and describe
multi-valued functions in a punctured disc.
In � 7 we construct the
${\bf{p}}^*$-function which will be used
to solve 
the Dirichlet problem in Chapter 5.
In � 8 we recall the fundamental 
\emph{Spiegelungsprinzip}
by  Hermann Schwarz which 
is applied
in  �9 to construct the
\emph{modular function}.
Finally
Section 10 gives a brief exposition about
some of the contributions by Poincar�
which  has
played a major role for the theory
of Fuchsian groups and automorphic functions.




\newpage



\centerline{\bf 1. Angular variation and winding numbers}



\bigskip

\noindent
Consider a vector-valued function of the real parameter $t$:
\[ 
t\mapsto (x(t),y(t))\quad\colon\quad 0\leq t\leq T
\]
whose  image does not contain the origin, i.e. 
$x^2(t)+y^2(t)>0$ hold for
each $t$. Moreover we assume that
the functions $x(t)$ and $y(t)$  are 
continuously differentiable. Set
\[
\dot x(t)=\frac{dx}{dt}\quad\colon \quad
\dot y(t)=\frac{dy}{dt}\quad\colon\quad 
r(t)=\sqrt{x(t)^2+y(t)^2}
\]
\medskip

\noindent 
{\bf{1.1 Proposition.}}
\emph{There exists a unique continuous map $t\mapsto \phi(t)$ such that}
\[ 
x(t)=r(t)\cdot \text{cos}\,\phi(t)\quad\colon\quad
y(t)=r(t)\cdot \text{sin}\,\phi(t)\quad\colon 0\leq t\leq T\tag{*}
\]
\emph{where the $\phi$-function satisfies
the initial condition:}
\[
x(0)=r(0)\cdot \text{cos}\,\phi(0)\quad\text{and}\quad
y(0)=r(0)\cdot \text{sin}\,\phi(0
\]



\medskip
\noindent
\emph{Proof.}
We  solve the first order  ODE-equation.
\[
\dot\phi=
\frac{x\dot y-y\dot x}{x^2+y^2}
\]
with initial condition $\phi(0)$ as above.
There remains to show that (*) holds for all $t$.
Let us for example verify the
$x(t)=r(t)\cdot \text{cos}\,\phi(t)$.
It suffices to prove that
\[ 
\frac{d}{dt}( r(t)\cdot \text{cos}\,\phi(t))=\dot x\tag{1}
\]
To prove  this we notice that the left hand side in (1) is equal to 
\[
\dot r\cdot\text{cos}(\phi)-
r\cdot\text{sin}(\phi)\dot\phi=
\frac{\dot r\cdot x}{r}-y\cdot\dot\phi
\tag{2}
\]
Next, since $r=\sqrt{x^2+y^2}$ we have
\[ 
r\dot r=x\dot x+y\dot y
\]
Hence (2) is equal to
\[
\frac{x^2\dot x+xy\dot y}{r^2}-
\frac{y(x\dot y-y\dot x)}{r^2}=\frac{(x^2+y^2)\dot x}{r^2}=\dot x
\]
This proves Proposition 1.1

\bigskip

\noindent
{\bf {1.2 The angular variation.}}
Since the sine- and the cosine    functions have period  $2\pi$,
the $\phi$-function   is only  determined up to
integer multiples of $2\pi$. 
However, we get an
\emph{intrinsic number}  by the difference
\[ 
\phi(T)-\phi(0)\tag{1.2.1}
\] 
This number is  called the  \emph{angular variation}
of the  function
$t\mapsto (x(t),y(t))$.
If we  choose another parametrization where $t=t(\tau)$ is 
non-decreasing and $0\leq \tau\leq T^*$, then 
we start with the vector valued function $\tau\mapsto x(t(\tau),y(t(\tau))$
and find $\phi(\tau)$. Calculus shows that
(1.2.1) is the same. Thus, the  angular variation
of an oriented  parametrized $C^1$-curve is intrinsically defined.


\bigskip

\noindent
{\bf A notation.} The angular  variation along
a parametrized curve
$\gamma$ is denoted by
$\mathfrak{a}(\gamma)$.
If $\gamma$ is a curve we can construct the
curve $\gamma^*$ with the opposite direction:
\[ \gamma^*(t)=\gamma(T-t)
\]
It is clear that
\[
\mathfrak{a}(\gamma^*)=-
\mathfrak{a}(\gamma)
\]
In other words,  up to a sign the angular variation is
determined by the orientation of the curve.


\bigskip


\centerline {\bf 1.3 The case of closed curves.}
\medskip

\noindent
If $x(0)=x(T)$ and $y(0)=y(T)$
the   variation is an integer multiple of $2\pi$.
So if $\gamma$ is a closed  parametrized curve then
$\mathfrak{a}(\gamma)$ is an integer times  $2\pi$ and we set
\[
\mathfrak{w}(\gamma)=\frac{\mathfrak{a}(\gamma)}{2\pi}
\]
The integer $\mathfrak{w}(\gamma)$  is called
the 
\emph{winding number} of $\gamma$.
The construction gives 
\[ 
\mathfrak{w}(\gamma)=
\frac{1}{2\pi}\int_0^T\,
\frac{x\dot y-y\dot x}{x^2+y^2}\cdot dt\tag{1.3.1}
\]
\medskip

\noindent
{\bf Example.}
Let  $m$ be a positive integer and 
\[
x(t)=\text{cos}\, mt\quad\colon\quad
y(t)=\text{sin}\, mt
\]
Notice that $x^2+y^2=1$. It follows that
\[
\frac{x\dot y-y\dot x}{x^2+y^2}\cdot dt=
\text{cos}\,mt\cdot m\cdot \text{cos}\, mt-
\text{sin}\, mt\cdot (-m\cdot\text{sin}\, mt)=m
\]
Hence the winding number is equal to $m$.






\bigskip

\noindent
{\bf 1.4 Homotopy invariance.}
Consider a family of closed curves
\[
\{\gamma_s\colon\,0\leq s\leq 1\}\quad\colon\,\gamma_s(0)=\gamma_s(T)
\,\colon\quad\,0\leq s\leq 1
\]
Let $t\mapsto (x_s(t),y_s(t))$ be the parametrization of
$\gamma_s$. 
For  each fixed $s$ the curve
$t\mapsto \gamma_s(t)$ 
has a winding number $\mathfrak{w}(\gamma_s)$.
If
these  $C^1$-functions 
depend continuously upon $s$ it follows that 
$s\mapsto\mathfrak{w}(\gamma_s)$ is a continuous function.
Since it is  integer-valued it must be a constant.
Hence we have proved
\bigskip

\noindent {\bf 1.5 Theorem} \emph{Let $\{\gamma_s\}$
In a homotopic  family of closed
curves, the individual curves have a common
winding number.}
\medskip

\noindent {\bf  Remark.} In topology one refers to
this by saying that the winding number is
the same in each homotopy class of closed 
parametrized curves which
surround the origin. A parametrized curve
$\gamma$ is defined by a map
$t\mapsto \gamma(t)$
which  need not be 1-1, i.e. we only assume that
$\gamma(0)=\gamma(T)$.
One may think of an insect which takes a walk on the
horisontal $(x,y)$-plane starting at point
$p$ at time $t=0$ and  returns to $p$ after a certain time
interval $T$. During this walk the
insect may
cross an earlier path several times and even walk in the same path
but in opposed direction for a while. The sole constraint is that the insect
never attains the origin, i.e. 
$x(t)^2+y(t)^2>0$ must hold in order to construct the winding number.



\noindent
{\bf 1.6 The case of non-closed curves.}
Let $p$ and $q$ be two points outside the origin.
Consider two curves
$\gamma_1$ and $\gamma_2$ where $p$ is the common initial point and $q$ the common end-point.
Now we get the closed curve
$\rho$ defined by
\[ 
\rho(s)=\gamma_1(2s)\quad\colon 0\leq s\leq T/2\text{and}\quad \rho(s)=
\gamma_2(2T-2s)
\quad\colon T/2\leq s\leq T
\]
Here we find that
\[
\mathfrak{w}(\rho)=
\mathfrak{a}(\gamma_1)-
\mathfrak{a}(\gamma_2)
\]
Next,
keeping $p$ and $q$ fixed we consider a continuous family of curves
$\{\gamma_s\}$ where
$\gamma_s(0)=p$ and $\gamma_s(T)=q$ for all $0\leq s\leq 1$.
To each $s$ we get the two curves
$\gamma_0$ and $\gamma_s$
and construct the closed curve
$\rho$ as above. Theorem 1.5  implies 
that 
\[
s\mapsto \mathfrak{a}(\gamma_0)-
\mathfrak{a}(\gamma_s)
\] 
is a constant function of $s$.
Since the difference  is zero when $s=0$
we conclude that
\[
\mathfrak{a}(\gamma_0)=
\mathfrak{a}(\gamma_s)\quad\colon\, 0\leq s\leq 1
\]
Thus, the angular variation is constant in a homotopic family of curves
which join a pair of
points $p$ and $q$.







\medskip


\noindent{\bf 1.7 Rouche's principle.}
Let $\gamma_*$ be a  parametrized closed curve, and
$\gamma$  another closed curve such that
\[
|\gamma_*(t)-\gamma(t)|<|\gamma_*(t)|\quad
\colon\quad 0\leq t\leq T\tag{i}
\]
To each $0\leq s\leq 1$ we  obtain the closed curve
$\gamma_s(t)=s\cdot \gamma_*+(1-s)
(\gamma_*(t)-\gamma(t))$ which by (i)
also the  surrounds the origin.
This gives  a homotopic family and   Theorem 1.5 gives:
\[ 
\mathfrak{w}(\gamma_*)=
\mathfrak{w}(\gamma)\tag{*}
\]


\bigskip


\centerline {\bf { Variation of vector-valued functions}}
\medskip

\noindent
Let $\gamma$ be a  parametrized $C^1$-curve. Here we do not exclude
that
$\gamma(t)=(0,0)$ for some values of $t$, i.e. $\gamma$ is an arbitrary
$C^1$-curve.
Consider a pair of $C^1$-functions $u(x,y)$ and
$v(x,y)$ defined in some neighborhood of the compact
image set $\Gamma=\gamma([0,T])$.
Assume that $u^2+v^2\neq0$ on $\Gamma$.
This gives a  curve $\gamma^*$ which surrounds the
origin
defined by
\[ 
t\mapsto (u(\gamma(t)),v(\gamma(t))\tag{i}
\]
Write $\gamma(t)=(x(t),y(t))$ and set
\[ 
\xi(t)=u(x(t),y(t))\quad\colon\quad
\eta(t)=v(x(t),y(t))
\]
Then we have
\[ 
\mathfrak{a}(\gamma^*)=
\int_0^T\,\frac{\xi\dot\eta-\eta\dot\xi}{\xi^2+\eta^2}\cdot dt\tag{ii}
\]
Now $\dot\xi= u_x\dot x+u_y\dot y$ and similarly for
$\dot\eta$.
So the last integral becomes
\[
\int_0^T\,\frac{u(v_x\dot x+v_y\dot y)-v(u_x\dot x+u_y\dot y)}
{u^2+v^2}\cdot dt \tag{*}
\]
This yields an integer called
the
variation 
of the vector valued function $(u,v)$ along the closed curve
$\gamma$.
We denote this integer by a subscript notation and write
$\mathfrak{a}_{(u,v)}(\gamma)$. When
$\gamma$ is a  closed curve we define the winding number
\[
\mathfrak{w}_{(u,v)}(\gamma)=
\frac{1}{2\pi}\cdot \mathfrak{a}_{(u,v)}(\gamma)
\]
Notice that this integer depends upon the pair $(u,v)$ while $\gamma$ is kept fixed. 



\bigskip


\centerline {\bf 1.8 The case of CR-pairs}
\medskip

\noindent
Let $\gamma$ be a  curve
and $f(z)=u+iv$  an  analytic in a neighborhood of
$\gamma(T)$ where 
$f(\gamma(t))\neq 0$ for all $t$.
Hence $u^2+v^2\neq 0$ on $\gamma$
so we can define
$\mathfrak{a}_{(u,v)}(\gamma)$.
Since  $(u,v)$ satisfy the Cauchy-Riemann equations we can
express  
$\mathfrak{a}_{(u,v)}(\gamma)$ 
in an elegant way.
Namely let $t\mapsto 
(x(t),y(t))$ be a parametrization of $\gamma$ and write
$z(t)=x(t)+iy(t))$. Then
\[ 
\dot z=\dot x+i\dot y
\]
Regard  the function
\[
t\mapsto \mathfrak{Im}\,[\frac{f'(z(t))}{f(z(t))}\cdot\dot z(t)\,]\tag{i}
\]
Since the complex derivative
$f'(z)=u_x+iv_x$ we obtain
\[
\frac{f'(z(t))}{f(z(t))}\cdot\dot z(t)=
\frac{(u_x+iv_x)(u-iv)(\dot x+i\cdot \dot y)}{u^2+v^2}\tag{ii}
\]
The imaginary part becomes


\[
\frac{u_x u\dot y-u_xv\dot x+v_x u\dot x+v_xv\dot y}{u^2+v^2}=
\frac{u(u_x\dot y+v_x\dot x)-v(u_x\dot x-v_x\dot y)}{u^2+v^2}\tag{iii}
\]


\noindent
Next, we can apply the Cauchy-Riemann equations and replace
$u_x$ with $v_y$ and $-v_x$ by $u_y$. Then
we see that (iii) is equal to the integrand which appears
in (*) in 1.7.
Hence we have proved the following:.
\bigskip

\noindent {\bf 1.9 Theorem} \emph{Let $f(z)=u+iv$ be holomorphic in a neighborhood of
$\gamma$ and set $\mathfrak{a}_f(\gamma)=\mathfrak{a}_{(u,v)}(\gamma)$.
Then}
\[ 
\mathfrak{a}_f(\gamma)=
\int_0^T\,
\mathfrak{Im}\,\bigl [\frac{f'(z(t))}{f(z(t))}\cdot\dot z(t)\,\bigr ]\cdot dt\tag{*}
\]
\bigskip

\noindent
{\bf 1.10 Remark} By the construction of complex line integrals, the
integral (*) above
can  be written
as

\[
\frac{1}{i}\cdot\int_\gamma
\mathfrak{Im}\,[\frac{f'(z)}{f(z)}\,]\cdot dz
\]
This complex notation is often used.
When
$\gamma$ is a closed curve we get the winding number
\[
\mathfrak{w}_f(\gamma)=
\frac{1}{2\pi i}\cdot\int_\gamma
\mathfrak{Im}\,[\frac{f'(z)}{f(z)}\,]\cdot dz
\]
So this complex line integral always is an integer whenever
$f(z)$ is analytic and $\neq 0$ in some open neighborhood of
the compact set
$\gamma([0,T])$.






\bigskip

\noindent
\centerline
{\bf 1.11 Jordan's curve theorem  }
\medskip

\noindent
Let $\gamma$ be 
a closed $C^1$-curve and set $\Gamma=\gamma([0,T])$.
To each $a\in {\bf{ C}}\setminus \Gamma $
the closed curve 
\[
t\mapsto \frac{1}{\gamma(t)-a}
\]
surrounds the origin. Its   winding number
denoted by $\mathfrak{w}_a(\gamma)$.
From (*) in 1.4 we see
that
this winding number is constant in every connected
component if $ {\bf {C}}\setminus \Gamma$.

\bigskip


\noindent
{\bf 1.12 The case when $\gamma$ is 1-1}
Assume that $\gamma(t)$ is 1-1  except for the 
common end-values. This means that the image curve
$t\mapsto \gamma(t)$ is a 
\emph{closed Jordan curve}.
For each $a\in{\bf{ C}}\setminus\Gamma$ 
we notice that $t\mapsto \gamma(t)-a$
is  1-1.
In  the equation from XX which
determines the $\phi$-function for a given $a$
where we may take $\phi(0)=0$
as initial value shows that $t\mapsto\gamma(t)$ is 1-1 on the open interval $(0,T)$.
Hence $\phi(t)$ cannot be an integer multiple of $2\pi$ when
$0<t<T$.
Starting with $\phi(0)=0$ it  follows that
\[ -2\pi<\phi(t)<2\pi\quad\colon\quad
0<t<2\pi
\]
Hence $\phi(T)$ can only attain one of the values $-2\pi,0,2\pi$.
The \emph{Jordan curve theorem}  tells us that the value zero is never
attained. Moreover, 
the set of points $a$ for which the winding number equals 1 is a conneced
open set, called the Jordan domain bounded by $\Gamma$. The complementary set is also
connected and here $\mathfrak{w}_a(\gamma)=0$.
This can be expressed by saying
that the closed Jordan curve $\Gamma$ divides $\bf C$ into two component.
This topological  result was proved by   
Camille Jordan in 1850 and it is 
is actually  valid under the relaxed assumption that the $\gamma $-function is only continuous.
In that case the proof of Jordan's Curve Theorem
is  more demanding.
For a detailed proof 
of the continuous version of Jordan's Curve 
Theorem we refer  [Newmann] where methods of algebraic topology are used.
We remark that Jordan's theorem in the plane is subtle in view of a quite 
remarkable  discovery in dimension 3 due to 
X. Alexander who constructed 
a \emph{homeomorphic copy} of the unit sphere in $R^3$ where the analogue of Jordan's theorem is not valid. This goes beyond
the scope of these notes.
A recommended text-book in
algebraic topology is Alexander's classic text-book [Al] 
which gives an excellent introduction to the  subtle parts of the theory.
\medskip


\noindent
{\bf 1.13 The case of a simple polygon.}
Let $p_1,\ldots,p_N$ be distinct points in $\bf C$
where $N\geq 3$.
To each 
$1\leq\nu\leq N-1$ we get a line segment
$\ell_\nu=[p_\nu,p_{\nu+1}]$ and we also get the line segment 
$\ell_N=[p_N,p_1]$.
Assume that they do not intersect.Then
they give sides of a simple closed curve $\Gamma$ whose corner points are
$p_1,\ldots,p_N$. The circle $\Gamma$ is oriented where one travels in 
the positive direction from $p_\nu$ to $p_{\nu+1}$ 
when $1\leq\nu\leq N-1$ and makes the final positive travel from
$p_N$ to $p_1$.
We can imagine a narrow channel $\mathcal C_+$ which 
surrounds $\Gamma$ and from this one can
"escape" to the point at infinity. For example at a corner point
$p_\nu$ where $|p_\nu|$ is maximal the channel contains
points of absolute value $>1$.
From this picture it is clear the the \emph{outer component}
$\Omega_\infty$
of
$\Gamma$ is connected - and even simply connected if one
adds the point at infinity. Rouche's principle shows
that the winding number is zero for
all points in the exterior component.
If we instead construct a narrow channel $\mathcal C_*$
which moves "just inside" $\Gamma$ 
then the channel itself is obviously connected.
But their remains to see why the whole interior is connected and that
the common winding number is  equal to one.
This, if $\Omega_*$ is the open complement of
$\Gamma\cup\Omega_\infty$ we must first prove that
$\Omega_*$ is connected. Since the narrow channel $\mathcal C_*$
is connected it suffices to show that when $p\in\Omega_*$ then
there exists  some curve $\gamma$
from $p$ which reaches $\mathcal C_*$.
To obtain $\gamma$ we consider a point $p^*\in\Gamma$
such that $|p-p^*|$ is the distance of $p$ to $\Gamma$, i.e.
we pick a point nearest to $p$.
Now we draw the straight line $L$ through $p$ and $p^*$ and by a picture
the reader discovers that if we travel along $L$ from
$p$ towards $p^*$ then we  reach $\mathcal C_*$ prior to the
arrival at $p^*$.
This proves that $\Omega_*$ is connected. The proof that the
common winding number for points in $\Omega_*$ is equal to one
is left as an \emph{exercise} to the reader.



\newpage




\centerline{\bf\large  2. The argument principle}
\bigskip

\noindent
Let $\Omega\in \mathcal D(C^1)$ and $f(z)$ is an analytic function
in $\Omega$ which extends to a $C^1$-function on its closure.
Denote by $\mathcal N_\Omega(f)$ the number of zeros of $f$ in
$\Omega$ and assume that $f\neq 0$ on $\partial\Omega$.
\medskip


\noindent {\bf 2.1 Theorem.} \emph{Let
$\Gamma_1,\ldots,\Gamma_k$
be the  oriented boundary curves of $\Omega$.
Then}
\[
N_\Omega(f)=\sum_{\nu=1}^{\nu=k}\,\mathfrak{w}_f(\Gamma_\nu)
\]



\noindent \emph{Proof.}
By the result in � III one has
\[
N_\Omega(f)=\sum\,\frac{1}{2\pi i}\cdot
\int_{\Gamma_\nu}\, \frac{f'(z)dz}{f(z)}
\]
Since $N_\Omega(f)$ is an integer and hence a real number
it follows that
\[
N_\Omega(f)=\sum\,\frac{1}{2\pi }\cdot
\int_{\Gamma_\nu}\, \mathfrak{Im}\,[\frac{f'(z)dz}{f(z)}\,]
\]
By Theorem 1.9 expressed in the complex notation
each
term of the sum above is equal to
$\mathfrak{w}_f(\Gamma_\nu)$ and Theorem 2.1 follows.




\medskip



\noindent 
{\bf 2.2 Rouche's Theorem.}
\emph{Let $\Omega$ and $f$ be as above and let $g$ be another
holomorphic function in $\Omega$ which extends to be $C^1$�on the closure.
If $|g|<|f|$ holds on $\partial\Omega$, it follows that}
\[
\mathcal N_{f+g}(\Omega)=\mathcal N_f(\Omega)
\]


\noindent 
\emph{Proof.} Apply the result in 1.6.


\bigskip


\noindent
{\bf 2.3 An application to trigonometric series}.
Let $1\leq m<n$ be a pair of positive integers. Consider a trigonometric polynomial
\[ 
P(\theta)=
\sum_{\nu=m}^{\nu=n}\,
a_\nu\text{cos}(\nu\theta)+
b_\nu\text{sin}(\nu\theta)\quad\colon\quad a_\nu,b_\nu\in\bf R
\]
We assume that at least one of the coefficients $a_m$ or $b_m$ is $\neq 0$, and similarly at least one of the numbers  $a_n$ or $b_n$ is $\neq 0$.
Then one has 
\medskip

\noindent {\bf 2.4 Theorem} 
\emph{$P$ has at least $2m$ zeros on $[0,2\pi]$ 
counted with multiplicity.}
\medskip

\noindent
\emph{Proof}
Consider the polynomial 
\[
Q(z)=(a_m-ib_m)z^m+\ldots+(a_n-ib_n)z^n
\]
Notice that $\mathfrak{Re}(Q(e^{i\theta})=P(\theta)$.
The polynomial $Q$ has a zero of multiplicity $m$ at the origin.
Consider some $r<1$  chosen so that
$Q\neq 0$ on the circle $T_r=\{|z|=r\}$.
Since $Q(z)$ has at least $m$ zeros counted with
multiplicity in the disc $D_r$, it follows from
Theorem 2.2 that
\[
\mathfrak{w}_Q(T_r)\geq m
\]
Regarding a picture the reader discovers that  the curve
$\theta\mapsto Q(re^{i\theta})$ must intersect the imaginary axis  line at least $2m$
times which means that the function
\[ 
\theta\mapsto \sum_{\nu=m}^{\nu=n}\,
r^\nu\cdot a_\nu\text{cos}(\nu\theta)+
r^\nu\cdot b_\nu\text{sin}(\nu\theta)
\]
as at least $2m$ distinct zeros on $[0,2\pi]$.
Passing to the limit as $r\to 1$ we get
Theorem 2.4.

\bigskip

\noindent
{\bf 2.5 A special estimate.}
Theorem 2.1 can be used to give upper bounds for the counting function
$\mathcal N_\Omega(f)$.
Suppose  that $\Omega$ is a rectangle
\[ \{z=x+iy\colon\, a<x<b\colon\, 0<y<T\}
\]
Here $\partial\Omega$ contains the vertical line
$\ell=\{x=b\,\colon\,0<y<T\}$.
The line integral along $\ell$ contributes to the evaluation of
$\mathcal N_\Omega(f)$ by
\[
\frac{1}{2\pi}\cdot
\int_\ell\,\mathfrak{Im}\,|\frac{f'(z)dz}{f(z)}]\tag{2.5.1}
\]


\noindent
Now $dz=idy$ along $\ell$ and therefore
the integral above is equal to
\[
\frac{1}{2\pi}\cdot\int_0^T\,
\mathfrak{Re}[\frac{f'(b+iy)}{f(b+iy)}]\cdot dy
\]
\medskip

\noindent Let us now assume that $\mathfrak{Re}\, f(b+iy)\geq c_0>0$
for all $0\leq y\leq T$.
Then there exists a single valued branch of the complex Log-function, i.e.
\[ 
\log\, f(b+iy)=
\log\, |f(b+iy)|+
i\cdot \text{arg}(f(b+iy))\,\,\colon\,
-\pi/2<\text{arg}(f(b+iy))<\pi/2
\]
Since $f'(z)=\frac{1}{i}\cdot\partial_y(f)$ it follows that
\[
\frac{f'(b+iy)}{fb+iy)}=
\frac{1}{i}\cdot [\partial_y(\log\, |f(b+iy)|)+
i\cdot\partial_y(\text{arg}(f(b+iy)))]
\]
Hence we obtain
\bigskip

\noindent {\bf 2.6 Proposition.} \emph{One has the equality}
\[
\mathfrak{Re}\,\frac{f'(b+iy)}{f(b+iy)}=
\partial_y(\text{arg}(f(b+iy)))
\]
\medskip

\noindent
{\bf 2.7 Remark.} Proposition 2.6  gives therefore
\[
\frac{1}{2\pi}\cdot
\int_\ell\,\mathfrak{Im}\,|\frac{f'(z)dz}{f(z)}]=
\frac{1}{2\pi}\cdot
\text{arg}(f(b+iT)))-\text{arg}(f(b)))\tag{*}
\]
The right hand side is a real number in $(-1/4,1/4)$ and 
hence we get a small contribution from the line integral in the left
hand side when we regard  whole line integral over $\partial\Omega$
which
evaluates $\mathcal N_f(\Omega)$. This will be used to study the
zeros of Riemann's $\zeta$-function.
\bigskip


\noindent
{\bf{2.8 A local implicit function theorem.}}
Let $m\geq 2$ and $g_2(z),\ldots,g_m(z)$
are analytic functions defined in an open disc
$D$ centered at $z=0$  where $g_\nu(0)=0$ for every
$\nu$. Let  $\phi(z)$ be another analytic function in
$D$ with $\phi(0)=0$ and
consider the  equation
\[ 
y+g_2(z)y^2+\ldots+g_m(z)y^m=\phi(z)\tag{2.8.1}
\]
Thus, we seek $y(z)$ so that (2.8.1) holds.
It turns out that there exists a unique analytic function
$y(z)$ defined in some open disc $D_*$ centered at $z=0$
where $y(0)=0$ and
(2.8.1) hold for every $z\in D_*$. 
To prove this we set
\[
P(y,z)=
y+g_2(z)y^2+\ldots+g_m(z)y^m
\]
Since $\phi$ and the $g$-functions are zero at
$z=0$ there exists
some 
$\delta>0$
such that if $z\in D(\delta)$ then
\[
|\phi(z)|<|P(e^{i\theta},z)|
\quad\text{for all}\quad 0\leq\theta\leq 2\pi\tag{i}
\]
Next, let us put
\[
P'_y(y,z)= 1+2g_2(z)y+\ldots+mg_m(z)y^{m-1}
\]
From (i)  there exists the integral
\[
\frac{1}{2\pi i}\cdot \int_{|y|=1}\,
\frac{P'_y(y,z)}{P(y,z)-\phi(z)}\cdot dy\quad
\colon\, z\in D(\delta)\tag{1}
\]
By Rouche's Theorem this integer-valued function
is constant when $z$ varies in $D(\delta)$.
When $z=0$  the integrand is $\frac{1}{y}$ and hence
the constant integer is 1.
This means  that when $z\in D(\delta)$ is fixed, then
the analytic function
\[ 
y\mapsto P(y,z)-\phi(z)
\] 
has exactly one simple zero in $|y|<1$.
Denote this zero by $y(z)$.
The residue formula gives:
\[ 
y(z)=\frac{1}{2\pi i}\cdot \int_{|y|=1}\,
\frac{y\cdot P'_y(y,z)}{P(y,z)-\phi(z)}\cdot dy\quad
\tag{2}
\]
It is clear that $y(z)$ is analytic in $D(\delta)$ and by the construction
$P(y(z),z)=0$.
Thus, $y(z)$ is the required solution. 
\bigskip

\noindent
{\bf{2.9 The case of higher multiplicity.}}
This time we consider the equation
\[
P(z,y)= g\uuu 1(z)y+g_2(z)y^2+\ldots+g_m(z)y^m=0
\] 
where $g\uuu k(0)\neq 0$ for some
$k\geq 2$ while
$g\uuu m(0)=\ldots=g\uuu{k+1}(0)=0$.
No special, assumption is imposed on $g\uuu 1(0),\ldots, g\uuu {k\vvv 1}(0)$, i.e.
some of these numbers may be $\neq 0$.
Since $g_k(z)\neq 0$ in some disc around $z=0$ and 
we study a homogeneous equation we can divide out $g\uuu k$ and assume that it is 1 from the start.
Let us then consider the polynomial
\[ 
Q(y)= y^k+g\uuu {k\vvv 1}(0)y^{k\vvv 1}+ g\uuu 1(0)y
\]
It has $k$ zeros counted with multiplicity and we choose $R$ so large that these zeros all
belong to $|y|<R$. With $R$ kept fixed it is clear that
there exists $\delta\uuu *>0$ such that
\[ 
|z|<\delta\uuu *\implies
|g\uuu m(z)y^m+\ldots+ g\uuu{k+1}(z)y^{m+1}|< 
|y^k+g\uuu{k\vvv 1}(z)y^{k1}+\ldots+g\uuu 0(z)|
\]
when $|y|=R$.
Rouche's theorem implies that $y\mapsto P(z,y)$ has
$m$ zeros in $|y|<R$ for each $|z|<\delta\uuu *$.
This $m$\vvv tuple of zeros need  be single\vvv valued analytic functions
in $|z|<\delta\uuu *$.
In XX we  describe the multi\vvv valued behaviour of these root functions in more detail.

\bigskip

\noindent
{\bf{2.10 Images of closed curves.}}
Let $\gamma$ be a closed Jordan curve
parametrized by arc\vvv length. So we have a map 
$s\to z(s)$
where $\gamma(0)=\gamma(L)$
and we  assume that this function is $C^1$
with a non\vvv zero derivative,  i.e.
with $z=x+iy$ the two functions $x(s)$ and $y(s)$
are both of class $C^1$ and $x'(s)^2+y'(s)^2>0$ hold when
$0\leq s\leq L$.
Let $f(z)$ be analytic in some
open neighborhood of $\gamma$ and assume that 
the absolute value $|f(z)|=c$ for some  constant $c>0$ holds on $\gamma$.
This gives  a  continuous function $\theta(s)$
such that
\[ 
f(z(s))=c\cdot e^{i\theta(s)}\tag{i}
\]
where $\theta(0)$�is determined by the equality
$f(z(0))= c\cdot e^{i\theta(0)}$.
Taking the derivative with respect to $s$ we get
\[ 
f'(z(s))\cdot z'(s)= c\cdot i \theta'(s)\cdot e^{i\theta(s)}\implies
\frac{f'}{f}\cdot z'(s)= i\cdot \theta'(s)\tag{i}
\]
It follows that
\[
\frac{1}{2\pi i}\cdot \int\uuu\gamma\, \frac{f'\cdot dz}{f}=
\frac{1}{2\pi} \int\uuu 0^{2\pi}\, \theta'(s)\cdot ds=
\frac{\theta(L)\vvv\theta(0)}{2\pi}\tag{*}
\]
Let us now assume now that
the left hand side is 1 which gives
\[ 
\theta(L)=\theta(0)+2\pi\tag{ii}
\]
In addition  we assume that the complex derivative $f'\neq 0$ on 
$\gamma$.
Then (i) shows that the $s$\vvv derivative of the real\vvv valued $\theta$\vvv function
is always $\neq 0$ and since the value increases by (ii) we have
$\theta'(s)>0$ for all $s$.
From this we conclude

\medskip

\noindent
{\bf{2.10 Theorem.}}
\emph{Assume that $f'\neq 0$ on $\gamma$ and that
the left hand side in (*) is one.
Then  $f$ yields a bijective map from
$\gamma$ onto the circle of radius $c$ centered at the origin.}
\medskip

\noindent
Next, with the assumptions in Theorem 2.10
we consider a complex number
$w$ of absolute value $<1$. Since $\gamma$ is a closed curve we know that
\[
\frac{1}{2\pi i}\cdot \int\uuu\gamma\, \frac{f'\cdot dz}{f\vvv w}\tag{1}
\]
is an integer and just as in Rouche's theorem we conclude that
(1) is equal to one for every $|w|<1$.
The reader may also verify that
\[ 
|w|>1\implies
\frac{1}{2\pi i}\cdot 
\int\uuu\gamma\, \frac{f'\cdot dz}{f\vvv w}=0\tag{2}
\]

\noindent
{\bf{Remark.}}
Notice that (1\vvv2) hold without the hypothesis that
$f$ extends to be analytic
in the Jordan domain $\Omega$
bordered by $\gamma$.
If we in addition assume
that $f$ from the start is analytic in
a neighborhood of $\bar\Omega$ then
(1) shows that $f(z)\vvv w$ has exactly one zero in
$\Omega$ for each $|w|<1$ which means that
$f$ yields a conformal map from $\Omega$ onto the disc
$|w|<c$.
Thus, to check when a given
$f\in\mathcal O(\Omega)$ which extends to $\gamma$
where $|f|=c$ and $f'\neq 0$, it suffices to check that
the left hand side in (*) is equal to one in order that
$f$ yields a conformal map.
\medskip

\noindent
{\bf{2.11 A conformal condition.}}
Let $\gamma$ and $f$ be as above where $f$ extends to be analytic
in a neighborhood of $\gamma$ and  assume that
the curve $\gamma$ is real\vvv analytic, i.e. 
$z(s)$ is a real\vvv analytic function of $s$.
We can find a closed Jordan curve $\gamma\uuu *$
contained in the Jordan domain $\Omega$
which together with $\gamma$ borders a doubly\vvv connected domain
$U$. Here $\gamma\uuu *$ can be close to $\gamma$ as illustrated by figure XXX,


\medskip

\noindent
{\bf{2.12 Theorem.}}\emph{ Assume that the restriction of $f$ to $U$ is 1\vvv 1. Then
$f$ yields a bijective map from $\gamma$ onto the circle of radius $c$.}
\medskip

\noindent
\emph{Proof.}
We may assume that
$\theta'(0)>0$ and Theorem 2.12 follows if we prove that
$\theta'(s)>0$ for all $s$.
If $\theta'(s\uuu 0)=0$ for some $0<s\uuu 0<L$
and $z\uuu 0=z(s\uuu 0)$ we have $f'(z\uuu 0)=0$.
Hence we have a Taylor series
with
\[ 
f(z\uuu 0+\zeta)= f(z\uuu 0)+c\uuu 2\zeta^2+c\uuu 3\zeta^3+\ldots
\]
Let $m$ be the smallest integer such that $c\uuu m\neq 0$.
We claim that $m=2$ must hold. 
To see this we use tha $\gamma$ is of class $C^1$
which means that if $\epsilon>0$ is sn�mall then
\[
\{|z\vvv z\uuu 0|<\epsilon\}\cap U= V
\]
is almost a small half\vvv disc.
\medskip


\noindent
{\bf{Exercise.}}
Show that if $m\geq 3$ with $c\uuu m\neq 0$
then the restriction of $f$ to $
V$ cannot be 1\vvv1.
\medskip

\noindent
We conclude that if $\theta'(s\uuu 0)=0$ then
the second order derivative $f''(z\uuu 0)\neq 0$. At the same time we notice that
another derivation in (ii) above at $s=s\uuu 0$ where 
$f'(z\uuu 0)=0$
holds
\[
\frac{f''(z\uuu 0)}{f(z\uuu 0)}\cdot z'(s\uuu 0(^2)=i\cdot \theta''(s\uuu 0)
\]
So whenever $\theta'(s\uuu 0)=0$ it follows that $\theta''(s\uuu 0)\neq 0$.
\medskip

\noindent
{\bf{Exercise.}}
Show that when $f|U$ is 1\vvv 1 then
$\theta'(s)\neq 0 $ for all $s$ and hence $f|\gamma$ is bijective.
\medskip

\noindent
{\bf{2.13 The case when the winding number is zero.}}
We keep the assumption that $\gamma$ is a real\vvv analytic closed Jordan  curve
and $f$ extends to an analytic function which is 1\vvv 1 in 
a domain $U$ as above while $|f(z)|=c$ has constant absolute value along
$\gamma$.
But this time we suppose that
\[
\int\uuu\gamma\, \frac{f'}{f}\, dz=0\tag{*}
\]
So here $\theta(L)=\theta(0)$ and after 
a rotation we may assume that this common value is zero.
while $\theta'(0)>0$.
It follows that $\theta(s)$ takes a maximum $>0$ for some $s^*$
where we therefore get $\theta'(s^*)$. 
So we always find the smallest $0<s\uuu 0<L$ where
$\theta'(s\uuu 0)=0$.
So $s\to \theta(s)$ is strictly increasing on $[0,s\uuu 0)$
and it is clear that the hypothesis that $f|U$ is 1\vvv 1
entails that
the range $\theta[0,s\uuu 0]$ is an interval
$[0,\theta\uuu 0|$ where $\theta\uuu 0<2\pi$, i.e. 
the $\theta$\vvv function has not made a full turn 
so the image set $f(\gamma[0,s\uuu 0])$ is
an interval on the circe $|w|=c$ where we write $w=f(z)$.
\medskip

\noindent
{\bf{Exercise.}}
Show by a similar reasoning as in XxX that $\theta''(s\uuu 0)\neq 0$
which means that $\theta(s)$ starts to decrease on some interval 
$s\uuu 0\leq s\leq
s\uuu 1$ until $\theta'(s\uuu 1)=0$ which must occur for some
$s\uuu 1<L$ since $\theta'(L)\neq 0$ was assumed.
Keeping this in mind the reader should supply a figure and 
verify the details of the following result:

\medskip

\noindent
{\bf{2.14 Theorem.}} \emph{$f$ maps $\gamma$ onto a
proper circular interval of $|w|=c$
where it yields a double cover except at the two end\vvv points of the interval given
by
$f(\theta(s\uuu 0)$ and $f(\theta(s\uuu 1))$ where the
$\theta$\vvv function achieves its maxium respectively  its minium.}
\medskip

\noindent
{\bf{2.15 Koebe's function.}}
Consider the analytic function
\[ 
f(z)=z+\frac{1}{z}
\]
When $\gamma$ is the unit circle we have
$f(e^{i\theta})= 2\cdot \cos\;\theta$.
This yields a double\vvv cover onto the interval $[\vvv 2,2]$.
At the same time one easily verifies 
$f$ restricts to a 1\vvv 1 map in the punctured disc $0<|z|<1$
and it is also 1\vvv 1 in the exterior disc $|z|>1$.
In fact, in the extended $w$\vvv plane we have the simple connected
domain $\Omega^*={\bf{C}}\cup\infty\setminus [-2,2]$
and $f$ is a conformal map from the 
union of the exterior disc $|z|>1$ and $z=\infty$ onto $\Omega^*$.
The double cover of $[\vvv 2,2]$ arises from this conformal
map and $\Omega^*$ is an example of a so called slit\vvv domain.
\medskip

\noindent
{\bf{2.16 A counting of fixed points.}}
In the open unit disc $D$ we remove a finite number of
disjoint Jordan domains 
$\{U_\nu\}$
bordered by closed Jordan curves
$\gamma_1,\ldots\gamma_k$ where each $\gamma$-curve is of class
$C^1$.
Let
$\Omega=D\setminus \cup\, \bar U_\nu$, i.e. we remove the
family of Jordan domains.
Now $\partial\Omega$ is th union of
the $\gamma$-curves and the unit circle
$T$.
Let $f(z)$ be an analytic function in $\Omega$ which extends
continuously to
the $\gamma$-curves and in addition across the unit circle, i.e. 
$f$ is analytic in a domain
$ \{|z|<1+\delta\}\setminus \cup\, \bar U_\nu$ for some
$\delta>0$.
We also assume that
$|f(z)|<1$ in the annulus $\{1<|z|<1+\delta\}$. Set
\[
\phi(z)=z-f(z)
\]
and assume that it has a finite number of zeros on
the unit circle and a finite number of zeros in
the domain $\Omega$, while $\phi\neq 0$ on each
$\gamma$-curve.
Without essential loss of generality we suppose that
$f(1)\neq 1$ and now the zeros of $\phi$
appear at points $e^{i\theta_\nu}$
for a finite set  $0<\theta_1<\ldots<\theta_k< 2\pi$.
Let $e_\nu$ be the multiplicity of the zero $\theta_\nu$.
With a small $\epsilon>0$ we get the simple closed curve
$T_\epsilon$ by replacing intervals along $T$ centered at
the points 
$\theta_\nu$ by the circular
arcs which belong to
the intersection of $D$ and the circle 
$\{ |z-e^{i\theta_\nu}|=\epsilon\}$.
Under the conditions above we have:

\medskip

\noindent
{\bf{Theorem.}} \emph{One has the equality}
\[
\frac{1}{2\pi i}\cdot \int_{T_\epsilon}\,
\frac{\phi'(z)}{\phi (z)}\, dz=
\]
\medskip

\noindent
\emph{Proof.}
Let $N$ be the number of zeros of 
$\phi $ in the domain $omega$.
with $\epsilon$ small it follows that
the integral in star is equal to $N$ plus the line integral over gamma-curves.
now we take the larger 
$T$ upper $\epsilon$ curve. line integral over that
is enlarged with sum of multiplicities. 
at the same time a direct computation
shows that the integral is 2 pi.
residue calculus shows also that outer integral minus  that over inner
$T_\epsilon$ is equal to the sum of multiplicities.
so in all the integral over inner part is
a half-sum of multiplicities minus 2 pi.
and at the same time the global formula ....












\bigskip









\newpage



\centerline{\bf  3. Multi-valued functions}
\bigskip

\noindent
Let $\Omega$ be an open connected subset of ${\bf{ C}}$
and 
$D\subset\Omega$ is an open disc of some radius $r$ centered at a point $z_0$.
The material about  power series in Chapter XX
shows that
$\mathcal O(D)$ is identified with  convergent power series
\[
\sum\,c_\nu(z-z_0)^\nu\quad\colon\quad
\text{radius of convergenece}\,\,\geq r
\]
So if $f\in\mathcal O(\Omega)$ its restriction to any disc $D\subset\Omega$
determines a convergent power series.
These power series must be matching when two discs have a
non-empty intersection which 
is the starting point for a general construction
due to
Weierstrass.

\medskip

\noindent
{\bf 3.1 Analytic continuation along paths}
Let $s\mapsto \gamma(s)$ be a continuous and
complex valued function with values in
$\Omega$. We do not require that $\gamma(0)=\gamma(1)$
or that $\gamma$ is 1-1.
The points $p=\gamma(0)$ and $q=\gamma(1)$ are called the 
terminal points of $\gamma$. Let $f_0\in\mathcal O( D_r(p))$
for some $r>0$, i.e. $f$ is analytic in a small disc centered at $p$.
Consider a
strictly increasing sequence
$0=s_0<s_1<\ldots<s_N=1$ and to each $p_\nu=\gamma(s_\nu)$
we choose a small  disc $D_{p_\nu}(r_\nu)$ such that:
\[
D_{p_\nu}(r_\nu)\cap
D_{p_{\nu+1}}(r_{\nu+1})\neq\emptyset\quad\,\quad 0\leq\nu\leq N-1
\]
Assume that
for  each $1\leq\nu\leq N$
exists  $f_\nu\in\mathcal O( D_{r_\nu}(p_\nu)) $ such that
\[
f_\nu=f_{\nu+1}\,\,\text{holds in}\,\,\,
D_{p_\nu}(r_\nu)\cap
D_{p_{\nu+1}}(r_{\nu+1})
\]
After $N$ many   \emph{direct analytic continuations over  pairs of intersecting discs}
we arrive at
$f_N$ which is analytic in an open disc centered at $\gamma(1)$.
The  uniqeness of each direct analytic continuation entails
that
the locally defined analytic function $f_N$
at $\gamma(1)$ is the same  if we  have chosen
a \emph{refined} partition  of $[0,1]$.
Since two coverings of $\gamma$ via  finite families of discs
have a common refinement, we conclude
that  locally defined analytic function at the end-point is unique.
Thus, the  construction   yields a map
$T_\gamma$ map which sends an analytic function
$f$ defined in a disc around $\gamma(0)$ to an analytic
function $T_\gamma(f)$ defined is some disc centered at  $\gamma(1)$. Of course, here
$T_\gamma$ is only defined on those $f$ at $\gamma(0)$ which have an analytic continuation along $\gamma$ in the sense of Weierstrass.




\bigskip




\noindent{\bf 3.2 The class $M_\Omega (\mathcal O)$.}
Let $\Omega$ be an open subset of $\bf C$. At each point
$z\in\Omega$ we denote by $\mathcal O(z_0)$
the germs of analytic functions at $z_0$ and recall that
this set is identified with  power series
$\sum\,c_\nu(z-z_0)^\nu$ with some positive radius of
convergence. 
In  $\mathcal O(z_0)$ we can consider those germs which have
analytic continuation along \emph{every}
curve in $\Omega$ whose initial point is $z_0$ while the end-point is arbitrary.
This leads to:

\bigskip

\noindent
{\bf 3.3 Definition}
\emph{A germ $f\in\mathcal O(z_0)$ generates a multi-valued analytic function
in $\Omega$ if it can be extended in the sense of Weierstrass along every
curve $\gamma\subset\Omega$ which has $z_0$ as initial point.
The set of all these germs is denoted by
$M_\Omega(\mathcal O)(z_0)$}.


\bigskip

\noindent{\bf 3.4 Remark.} Notice that
$M_\Omega( \mathcal O)(z_0)$ contains those germs at $z_0$ which
are induced by \emph{single-valued} analytic functions in
$\Omega$.
If $f\in M_\Omega(\mathcal O)(z_0)$ and $\gamma$ is a curve in 
$\Omega$ with $z_0$ as initial point and $z_1$ as end-point, then
the germ $T_\gamma(f)$ at $z_1$ belongs to
$M_\Omega( \mathcal O)(z_1)$.
This is obviuos since if $\gamma_1$
is a curve starting at $z_1$ with end-point at $z_2$, then $f$ extends along the composed curve $\gamma_1\circ \gamma_1$
and one has the composition formula:
\[ 
T_{\gamma_1}(T_\gamma(f))= T_{\gamma_1\circ\gamma}(f)\tag{*}
\]
\medskip

\noindent
{\bf 3.5 The total sheaf space $\mathcal {\widehat O}$.}
We construct
a big topological space
$\mathcal {\widehat O}$ as follows: One has a map
$\rho$ from $\mathcal {\widehat O}$ onto $\bf C$. The inverse fiber
$\rho^{-1}(z)=\mathcal O(z)$  for each $z\in\bf C$.
An open neighborhood of a "point"  $f\in\rho^{-1}(z_0)$
consists of 
a pair $(f,D)$ where $D$ is a small disc centered
at $z_0$ such that the germ $f$ extends to an analytic function in $D$. 
Then its induced germ at a point $z\in D$ belongs to
$\rho^{-1}(z)$. The set of points in
$\mathcal {\widehat O}$ obtained in this way yields the subset $(f,D)$
and as $D$ shrinks to $z_0$ they give by definition a fundamental system of open
neighborhoods of the point $f$ in $\mathcal {\widehat O}$.
With this topology on $\mathcal{\widehat O}$ 
the map $\rho$ is a \emph{local homeomorphism} 
and each inverse fiber $\rho^{-1}(z_0)$ appears as a 
\emph{discrete} subset of
$\mathcal {\widehat O}$.
\medskip

\noindent
{\bf Remark} $\mathcal{\widehat O}$ is the first example of a sheaf
which later 
led
to the general construction of sheaves.
The construction of the sheaf topology 
on $\mathcal{\widehat O}$ yields the following elegant
description of multi-valued functions.
\bigskip

\noindent
{\bf 3.6 Proposition.} \emph{Let $\Omega$ be an open and connected
subset of $\bf C$.
Let $z_0\in\Omega$ and  $f\in M_\Omega(\mathcal O)(z_0)$.
Then  $f$ appears in the inverse fiber $\rho^{-1}(z_0)$
of  an open and connected
set $\mathcal W(f)$ of $\rho^{-1}(\Omega)$ called 
Weierstrass Analytische Gebilde of the germ of this multi-valued
funtion. For each $z\in \Omega$ the set
$\mathcal W(f)\cap\rho^{-1}(z)$ consists of all germs at $z$ obtained by
analytic continuation of $f$ along some curve with end-point at $z$.}
\bigskip

\noindent
{\bf{Some notations.}}
Let $f$ be as above.
If $z\in\Omega$ we denote by $W(f:z)$ the set of germs at $z$ which arise via 
all analytic continuations  of $f$. Thus, $W(f:z)$ is equal to
$\mathcal W(f)\cap\rho^{-1}(z)$. In addition to this we can consider the
set of values at $z$ which are attained by these germs.
We have also the set
\[ 
R_f(z)= \{T_\gamma(f)(z)\quad\colon\,\, T_\gamma(f)\in W(f:z)\}
\]
\medskip

\noindent
{\bf{Exercise.}}
Let $\Omega$ be an open and connected subset of
${\bf{C}}$.
Conclude from ther above that there exists a 1-1 correspondence between
open and connected subsets of $\rho^{-1}(\Omega)$
whose $\rho$-image is $\Omega$,  and the family of
multi-valued analytic functions in
$\Omega$.







\noindent
{\bf Example}
Let $\Omega=\bf C$ minus the origin, i.e. the punctured complex plane.
Then we have the multi-valued Log-function. At each point
$z\in\Omega$ it has an infinite set of local branches which differ by integer
multiples of $2\pi i$. The resulting connected set
$\mathcal W(\log (z))$ can be regarded as a 2-dimensional connected manifold.
In topology one learns that this is the \emph{universal covering space} of 
$\Omega$. In particular  
$\mathcal W(\log(z))$  is a  \emph{simply connected} manifold.
\bigskip


\centerline {\bf 3.7 Normal families.}
\medskip

\noindent
Let $\Omega$ be some connected open set in
${\bf{C}}$.  Let $x_0\in \Omega$ and consider
some germ $f\in M\mathcal O(\Omega)(x_0)$.
We say that $f$ yields a  bounded multi-valued function if there exists a constant
$K$ such that
\[
\bigl|T_\gamma(f)(x)\bigr |\leq K\tag{*}
\] 
holds for all pairs $x,\gamma$ where $x\in\Omega$ and
$\gamma$ is any curve from $x_0$ to $x$.
Suppose that
$\{f_\nu\}$ is a sequence of germs
in $M\mathcal O(\Omega)(x_0)$ which are uniformly bounded, i.e.
(*) above holds for some constant $K$ and every $\nu$.
If we to begin with consider a small open disc
$D$ centered at $x_0$ we get the unique single-valued branches of
each $f_\nu$ in
$\mathcal O(D)$. This family in
$\mathcal O(D)$ is normal by the results in
XXX. Passing to a subsequence we may assume that
there exists a limit function
$g\in\mathcal O(D)$, i.e. shrinking $D$ if necessary we may assume that
\[
\lim_{\nu\to\infty}\, ||f_\nu-g||_D\to 0\tag{i}
\]
Next, if $\gamma$ is a curve in
which starts at $x_0$ and has some end-point $x$
we cover $\gamma$ with a finite number of open discs and
each $f_\nu$ has its analytic continuation along $\gamma$
by the Weierstrass procedure.
From the material in XXX it is clear that
during these analytic continuations the local series expansions of
the sequence $\{f_\nu\}$ converge uniformly
and as a result we find that
$g$ has an analytic extension along
$\gamma$. Hence the germ of $g$ at $x_0$
belongs to $M\mathcal O(\Omega)(x_0)$.
Moreover, the uniform convergence "propagates".
For example, if $\gamma$ is a closed curve at
$x_0$ we get the sequence
$\{T_\gamma(f_\nu)\}$
after the analytic continuation along $\gamma$, and similarly
$T_\gamma(g)$. Then 
\[
\lim_{\nu\to\infty}\, ||T_\gamma(f_\nu)-T_\gamma(g)||_D\to 0\tag{ii}
\]
holds for a small disc $D$ centered at the end point of $\gamma$.


\bigskip



\centerline
{\bf 4. The Monodromy Theorem}
\medskip

\noindent
Let $f\in M\mathcal O(\Omega)$.
If $z_0\in\Omega$ and $\gamma$ is a curve starting at $z_0$ we obtain 
the germ $T_\gamma(f)$ at the end-point $z_1$ of $\gamma$.
The analytic continuation is obtained by the Weierstrass procedure and
since $\gamma$ is a compact subset of $\Omega$
it can be covered by a finite set of discs $D_0,D_1,\ldots,D_N$ where 
$D_\nu\cap\,D_{\nu+1}$ are non-empty and the analytic continuation of 
$f$ is achieved by succesive direct continuations of analytic functions
$\{g_\nu\in\mathcal O(D_\nu)\}$
where $g_\nu=g_{\nu+1}$ holds in
$D_\nu\cap\,D_{\nu+1}$.
The discs are  chosen 
so small that they are 
relatively compact in $\Omega$.
If $\gamma_1$  is another curve from $z_0$ to $z_1$ which stays
so close to $\gamma$ that the discs $D_0,\ldots,D_N$
again can be used to perform the analytic continuation of $f$ along
$\gamma_1$, then it is clear that  $T_\gamma(f)=T_{\gamma_1}$.
This observation gives:
\medskip

\noindent {\bf 4.1 Theorem} \emph{Let $(z_0,z_1)$ be a pair in $\Omega$ and
$\Gamma(s,t)$ 
a continuous map from the unit square in the $(s,t)$-space into $\Omega$ where}
\[ 
\Gamma(s,0)=z_0\quad\colon,\gamma(s,1)=z_1\quad\colon\quad
0\leq s\leq 1 
\]
\emph{Then, if  $\{\gamma_s\}$ is the family of curves defined by
 $t\mapsto \Gamma(s,t)$, it follows that}
 \[ 
 T_{\gamma_s}(f)=T_{\gamma_0}(f)\quad\colon\,\quad 0\leq s\leq 1
 \]
 \medskip
 
 \noindent {\bf Remark} This is called the monodromy theorem and can be
 expressed by saying that  analytic continuation along a curve 
 which joins  a given pair of points only  depends on the
 \emph{homotopy} class of the curve, taken in the family of all
 curves which joint
 the two given points.
Of course, when we deal with some multi-valued function in an open set
$\Omega$ we are obliged to use curves inside $\Omega$ only.

\bigskip

\noindent
{\bf 4.2 The case of finite determination.}
Let $f\in M\mathcal(\Omega)$.
If $z_0\in\Omega$ we get the set of germs   $W(f:z_0)$ at $z_0$.
This is a subset of $\mathcal O(z_0)$ and we can regard the complex vector space
it generates. It is denoted by $\mathcal H_f(z_0)$.
Suppose that this complex vector space has a finite dimension $k$.
Then we can choose  a $k$-tuple of germs $g_1,\ldots,g_k$
in $W(f:z_0)$  which give a basis of
$\mathcal H_f(z_0)$. Thus, one has to begin with

\[ 
\mathcal H_f(z_0)=Cg_1+\ldots+Cg_k
\]
Let $z_1$ be another point in $\Omega$ and fix 
some curve $\gamma$ which joins $z_0$ and $z_1$.
At $z_1$ we get the germs $T_\gamma(g_1),\ldots,T_\gamma(g_k)$.
By the remark in XXX
$T_\gamma$ is a \emph{bijective map} from $W(f:z_0)$ 
to $W(f:z_1)$.
Moreover, if  $\phi=c_1g_1+\ldots+c_kg_k$ belongs to $\mathcal H_f(z_0)$ we have

\[
T_\gamma(\phi)=c_1T_\gamma(g_1)+\ldots+c_kT_\gamma(g_k)
\]
Hence the $k$-tuple $\{T_\gamma(g_\nu)\}$ generates the vector space 
$\mathcal H_f(z_1)$.
Since we also can use  the inverse map $T_{\gamma^{-1}}$
it follows that the $k$-tuple
$\{T_\gamma(g_\nu)\}$ yields a basis of $\mathcal H_f(z_1)$.
In particular the vector space $\mathcal H_f(z)$ have common dimension 
$k$ as $z$ varies in the connected open set
$\Omega$.
\emph{Summing up}, we can conclude the following:

\medskip


\noindent
{\bf 4.3 Proposition}
\emph{If $f\in M\mathcal (\Omega)$ has finite determination the complex vector spaces
$\mathcal H_f(z)$ have  common dimension. Moreover, one gets
a basis of these by  starting at any point $z_0$ and choose some $k$-tuple
of $C$-linearly germs $g_1,\ldots,g_k$ in $W(f:z_0)$. 
Then
we obtain a basis in $\mathcal H_f(z)$ for any point $z\in\Omega$ by a $k$-tuple
$\{T_\gamma(g_\nu)\}$ where $\gamma$ is any curve which joins  
$z_0$ and $z$.}
\bigskip


\noindent
{\bf 4.4 The case of a punctured disc}
Let $\dot D=\{0<|z|<R\}$
be a punctured disc centered at the origin.
Consider some $f\in\mathcal M\mathcal O(\dot D)$ of finite determination and let 
$k$ be its rank.
In a punctured open disc every closed curve is homotopic to
a closed circle parametrized by  $\theta\mapsto re^{i\theta}$.
Another way to express this is that the fundamental group
$\pi_1(\dot D)$ is isomorphic to the abelian group of
integers.
Thus, the multi-valuedness is determined by a sole $T$-operator
which arises when we let $\gamma$ be a circle surrounding the origin in
the positive sense.
Given $z_0\in\dot D$ we  consider the  ${\bf{C}}$-linear operator
\[
T_\gamma\colon\,\mathcal H_f(z_0)\mapsto\mathcal H_f(z_0)
\]
By Jordan's decomposition theorem we can choose a basis
in $\mathcal H_f(z_0)$ such that the matrix representing
$T_\gamma$ is of Jordan's  normal form.
This means that we have a direct sum
\[ 
\mathcal H_f(z_0)=\oplus\,
\mathcal K_\nu(z_0)
\]
where $\{\mathcal K_\nu(z_0)\}$ are $T_\gamma$-invariant subspaces and the
restriction of $T_\gamma$
$\mathcal K_\nu(z_0)$ is represented by an elementary
Jordan matrix $J(m,\lambda)$
for some complex number $\lambda$ and $m\geq 1$.
Given the pair $m,\lambda$  we consider
a local branch of the function
\[ 
f(z)=z^\alpha\cdot\,[ \text{Log}\,z\,]^{m-1}\quad\colon\,
e^{2\pi i\alpha}=\lambda
\]
which for example is defined close to $z=1$ where 
$f(1)=0$. After one turn around the origin we get a new local branch
of the form
\[ 
f_1(z)=\lambda\cdot  z^\alpha\cdot
[\text{Log}\,z\,+2\pi i]^{m-1}
\]
Continuing in this way $m$ times we see that 
the local branches of $f$ 
generate an $m$-dimensional complex vector space
whose  monodromy is  determined by the matrix $J(m,\lambda)$.
Using this fact it follows that if $f(z)$ is any local branch of a multi-valued function of
finite determination, then it can be expressed as:
\[ 
f(z)=\sum_{\nu=1}^k\sum_j\, \, g_{\alpha_\nu,j}(z)\cdot
z^{\alpha_\nu}\cdot\,[\text{Log}\, z\,]^j\tag{*}
\] 
Here 
$0\leq\mathfrak{Re}(\alpha_1)<\ldots<\mathfrak{Re}(\alpha_k)<1$
and $\{j\}$ is a finite set of  non-negative integers
and the $g$-functions 
are \emph{single-valued} in the punctured disc $\dot D$.
Moreover these $g$-functions
are
uniquely determined provided a specific local 
branch of the Log-function is chosen. 
For example, when $f$ is a local branch at some real point
$0<a<R$ where  $\text{Log}\,a$ is chosen to  be real.
\bigskip





\centerline { \bf\large 5. Homotopy and Covering spaces}
\bigskip

\noindent
Let $X$ be a metric space, i.e. the topology is defined by some
distance function. By a  curve in $X$ we  mean a
continuous map $\gamma$ from the closed unit interval
$[0,1]$ into $X$.
In general $\gamma$ need not be 1-1.
The initial point is $\gamma(0)$ and the end point is $\gamma(1)$.
If $\gamma(0)=\gamma(1)$ we 
say that $\gamma$ is a \emph{closed} curve.
We say that $X$ is \emph{arcwise connected}
if there to each pair of points $x_0,x_1$ exists some curve
$\gamma$ with
$x_0=\gamma(0)$ and
$x_1=\gamma(1)$.

\medskip


\noindent
{\bf A notation.} Given a point $x_0\in X$ we denote by
$\mathcal C(x_0)$ the family of all closed curves
$\gamma$ where
$\gamma(0)=\gamma(1)=x_0$.
\medskip

\noindent
{\bf 5.1 Definition.}
\emph{A pair of closed curves
$\gamma_0$ and $\gamma_1$
in $\mathcal C(x_0)$ are homotopic if there
exists a continuos map $\Gamma$ from the
unit square
$\square=\{(t,s)\quad\colon\, 0\leq t,s\leq 1\}$ into $X$
such that}
\[ 
\Gamma(t,0)=\gamma_0(t)
\,\,\text{and}\,\, 
\Gamma(t,1)=\gamma_1(t)\quad
\, \Gamma(0,s)=\Gamma(1,s)=x_0\,\,\,
\colon\,\, 0\leq s\leq 1
\]
\medskip

\noindent
It is clear that homotopy yields an equivalence relation on 
$\mathcal C(x_0)$.
If $\gamma\in \mathcal C(x_0)$ then
$\{\gamma\}$ denotes its homotopy class.
Next, 
if
$\gamma_0$ and $\gamma_1$ are two closed curves
at $x_0$ we get a new closed curve
$\gamma_2$ defined by
\[ 
\gamma_2(t)=\gamma_1(2t)\,\colon\, 0\leq t\leq \frac{1}{2}\,
\,\text{and}\,\, 
\gamma_2(t)=\gamma_2(2t-1)\,\colon\, \frac{1}{2}\leq t\leq 1\,
\]
We refer to $\gamma_2$ as the composed curve
and it is denoted by $\gamma_1\circ\gamma_0$. One verifies easily that
the homotopy class of $\gamma_2$
depends upon $\{\gamma_1\}$
and $\{\gamma_1\}$ only.
In this way we obtain a composition law on the 
set of homotopy classes of closed curves at $x_0$ defined by
\[
\{\gamma_1\}\cdot \{\gamma_0\}=
\{\gamma_1\circ\gamma_0\}
\]
One verifies easily that this composition satisfies the associative law.
A neutral element is the closed curve
$\gamma_*$ for which $\gamma_*(t)=x_0$ for every $t$.
Finally, if
$\gamma(t)$ is any closed curve at
$x_0$ we get a new closed curve by reversing
the direction, i.e. set
\[ 
\gamma^{-1}(t)= \gamma(1-t)
\]


\noindent
{\bf Exercise.} Show that
the composed curve $\gamma^{-1}\circ\gamma$ is homotopic to $\gamma_*$.
\medskip

\noindent
{\bf 5.2 The fundamental group}. The construction of
composed closed curves and the exercise above show
that 
homotopy classes of closed curves at $x_0$ give elements of a group 
to be denoted
by $\pi_1(X:x_0)$.
\medskip

\noindent
{\bf Remark.} The group
$\pi_1(X:x_0)$ is intrinsic in  the sense that it does not depend upon the chosen point $x_0$. Namely, let $x_1$ be another point in $X$ and fix
a curve
$\lambda$ with $\lambda(0)=x_0$ and $\lambda(1)=x_1$.
Then we obtain a map from
$\mathcal C(x_1)$ to $\mathcal C(x_0)$ defined by
\[
\gamma\mapsto \lambda^{-1}\circ\gamma\circ\lambda\tag{i}
\]
One verifies that (i) sends homotopic curves to homotopic curves
and by considering homotopy classes we obtain an 
isomorphism between 
$\pi_1(X:x_0)$ and $\pi_1(X:x_1)$.
Hence there exists an intrinsically defined group
denoted by $\pi_1(X)$. It is called the fundamental group of
the metric space $X$. If 
$\pi_1(X)$ is reduced to a single element,  i.e. when all closed curves
in $\mathcal C(x_0)$ are homotopic we say that
$X$ is \emph{simply connected}.
\medskip

\noindent
{\bf Exercise.}
Let $x_0$ and $x_1$ be two distinct points in $X$. Denote by
$\mathcal C(x_0,x_1)$ the family of curves
$\gamma$ for which
$\gamma(0)=x_0$ and $\gamma(1)=x_1$.
Two such curves
$\gamma_0$ and $\gamma_1$ are homotopic if there exists
a continuos map $\Gamma$ from the square $\square$ such that
\[
\Gamma(t,0)=\gamma_0(t)
\,\,\text{and}\,\, 
\Gamma(t,1)=\gamma_1(t)\quad
\, \Gamma(0,s)=x_0\,\,\text{and}\,\, \Gamma(1,s)=x_1\,\,\,
\colon\,\, 0\leq s\leq 1
\]
\medskip
\noindent
Show that a pair $\gamma_0$ and $\gamma_1$ are homotopic in
$\mathcal C(x_0,x_1)$ if and only if the closed curve
$\gamma_1^{-1}\circ\gamma_0$ is homotopic to
$\gamma^*$ in $\mathcal C(x_0)$ where
\[ 
\gamma_1^{-1}=\gamma_1(1-t)
\]
In particular each pair of curves in $\mathcal C(x_0,x_1)$
are homotopic if $X$ is simply connected.
\bigskip


\centerline {\bf 5.3 Covering maps.}
\medskip

\noindent
Let $X$ and $Y$ be two
arcwise connected metric
spaces.
A continuous map $\phi$ from $X$ onto $Y$
is a local homeomorphism if the following hold:
For each $y_0\in Y$ there exists an open neighborhood
$U$ such that the inverse image
$\phi^{-1}(U)$ is a union of  pairwise disjoint open
sets $\{U_\alpha^*\}$ and the restriction of $\phi$ to each
$U_\alpha^*$
is a homeomorphism from
this set onto $U$.
\medskip

\noindent
{\bf 5.4 Lifting of curves.} Let
$\phi\colon X\to Y$ be a local homeomorphism where we assume that
$\phi(X)=Y$.
Let $\gamma$ be a curve in $Y$ defined by
a continuous map $t\to\gamma(t)$ from the closed unit interval $[0,1]$
into $Y$ with 
some initial point
$y_0=\gamma(0)$ and some end-point $y_1=\gamma(1)$.
The case  $y_0=y_1$ is not excluded, i.e. $\gamma$ may
be a closed curve.
Next, in $X$ we chose a point $x_0$ such that
$\phi(x_0)=y_0$.
By assumption there exists an open neighborhood $U$ of $y_0$
in $X$ a unique open neighborhood $U^*$ of $x_0$
such that
$\phi\colon\, U^*\to U$ is a homeomorphism.
Since $t\to\gamma(t)$is continuous there exists
some $\delta>0$ such that
\[
\gamma(t)\in U\,,\quad 0\leq t\leq \delta\tag{i}
\]
Then we get a \emph{unique} curve
$\gamma^*$ in $X$
defined for $0\leq t\leq\delta$ such that 
\[
\phi(\gamma^*(t))=\gamma(t)\,,\quad 0\leq t\leq \delta\quad
\text{and}\,\, \gamma^*(0)=x_0\,.\tag{ii}
\]
If this lifting process can continued for
all $0\leq t\leq 1$ we say that
$\gamma$ has a lifted curve
$\gamma^*$. This  means that there exists a curve
$t\mapsto\gamma^*(t)$ from $[0,1]$ into $X$
such that
\[
\phi(\gamma^*(t))=\gamma(t)\,,\quad 0\leq t\leq 1\quad
\text{and}\,\, \gamma^*(0)=x_0\,.\tag{*}
\]
\medskip

\noindent
{\bf Exercise.}
Show  that
the curve
$\gamma^*$ is unique if it exists.
The hint is to use that
$\phi$ is a local homeomorphism.
\medskip

\noindent
The whole discussion above leads to
\medskip

\noindent 
{\bf 5.5 Definition.}\emph{ A local homeomorphism
$\phi\colon X\to Y$ is called a covering map of  $\mathcal L$-type
if the following hold: For each pair of points
$y_0\in Y$ and 
$x_0\in \phi^{-1}(0)$, every 
curve $\gamma$ in $Y$ with initial 
point $y_0$ and arbitrary end-point $y$
can be lifted to a curve in
$X$ with initial point $x_0$.}
\bigskip

\noindent
{\bf 5.6 The case when
$X$ is simply connected.}
Assume this and let
$\phi\colon X\to Y$ be a covering map of 
$\mathcal L$-type.
Let $y_0\in Y$ and choose some point $x_0\in\phi^{-1}(y_0)$.
Next, let
$\gamma$ be a closed curve in $Y$ with
$\gamma(0)=\gamma(1)=y_0$.
By assumption there exists a unique lifted curve
$\gamma^*$ in $X$ with
$\gamma^*(0)=x_0$.
Suppose  that
$\gamma^*(1)=x_0$, i.e.  the lifted curve is closed.
Since $X$ is simply connected it is homotopic to
the trivial curve which stays at $x_0$, i.e. there exists a continuos map
$\Gamma^*$ from $\square$ into $X$ 
such that
\[
\Gamma^*(t,0)=\gamma^*(t)\,
\text{and}\,\Gamma^*(t,1)=x_0\quad
\Gamma^*(0,s)=\Gamma(1,s)=x_0\,\,\colon\,0\leq s\leq 1\tag{i}
\]
Now $\Gamma(t,s)=\phi(\Gamma^*(t,s)$  is a continuous map from
$\square$ into $Y$ and from (i) we see that
$\Gamma$ yields a homotopy between
$\gamma_0$ and $\gamma_1$.
Using this observation we arrive at:
\medskip

\noindent
{\bf 5.7 Proposition.} \emph{Let
$\gamma_0$ and $\gamma_1$ be two closed curves at
$y_0$. Then  they are homotopic if and only if
$\gamma^*(1)=\gamma^*(1)$.}
\medskip

\noindent
\emph{Proof.}
We have already seen that if
$\gamma^*(1)=\gamma^*(1)$
then the two curves are homotopic.
Conversely, if they are homotopic we get a continus map
$\Gamma(s,t)$ from $\square$ into $Y$ and for each
$0\leq s\leq 1$ we have the closed curve
$\gamma_s(t)=\Gamma(t,s)$
at $y_0$.
Since the inverse fiber
$\phi^1{_1}(y_0)$ by assumption is a discrete set in
$X$, it follows by 
continuity and the unique path lifting
that $s\mapsto \gamma_s^*(1)$ 
is constant and hence
$\gamma_0^*(1)=\gamma^*_1(1)$.
\bigskip

\noindent
{\bf 5.8 Conclusion.}
Proposition 5.7 shows that homotopy classes of closed curves
$\gamma$ at $y_0$ are in a 1-1 correspondence with
their end-points in $X$.
Notice also that if $x$ belongs to $\phi^{-1}(y_0)$ then
the arc-wise connectivity of $X$ gives a curve
$\rho$ where
$\rho(0)=x_0$ and $\rho(1)=x$.
Now $\gamma(t)=\phi(\rho(t))$ is a closed curve at
$y_0$ and here $\gamma^*(t)=\rho(t)$
and hence $x$ appears as an end-point for at least one closed curve
at $y_0$. Identifying $\pi_1(Y)$ with homotopy classes of closed
curves at $y_0$ we have therefore proved the following:
\medskip

\noindent
{\bf 5.9 Theorem.}
\emph{The map $\gamma\to\gamma^*(1)$
yields a bijective correspondence between the fundamental group
$\pi_1(Y)$ and the inverse fiber
$\phi^{-1}(y_0)$.}

\medskip

\noindent
{\bf Exercise.}
Let $X$ and $Z$ be two simply connected metric spaces.
Suppose that  $\phi\colon X\to Y$ and $\psi\colon Z\to Y$
are two covering maps which both belong to the class $\mathcal L$.
Fix some $y_0\in Y$. Choose
$x_0\in \phi^{-1}(y_0)$ and 
$z_0\in \psi^{-1}(y_0)$.
Next, let $y\in Y$ and consider some curve
$\gamma$ in $Y$ with $\gamma(0)=y_0$ and $\gamma(1)=y$.
Its unique lifted curve to
$X$ is denoted by $\gamma^*$ and we get the end-point
\[
\gamma^*(1)\in \phi^{-1}(y)
\]
Similarly. we get a unique lifted curve
$\gamma^{**}$ in $Z$ and the end-point

\[
\gamma^{**}(1)\in \psi^{-1}(y)
\]
From the above these two end-points only depend on  the homotopy class
of $\gamma$. Use this to conclude that
we obtain a \emph{unique and  bijective} map
from the discrete fiber
$\phi^{-1}(y)$ to $\psi^{-1}(y)$. Moreover, as $y$ varies
in $Y$ this gives a unique homeomorphism $G$ from $X$ onto $Z$
with $G(x_0)=z_0$.



\newpage


\centerline{\bf 6. The uniformisation theorem.}
\bigskip

\noindent
{\bf Introduction.}
Let $\Omega$ be a connected open subset
of
${\bf{C}}$. If
the closed complement contains at least two points there
exists a a covering map $f\colon\,D\to \Omega$
of $\mathcal L$-type given by an analytic function.
This will be proved in Chapter 6.
Here we  take this existence for granted and analyze some consequences.
More precisely,
in Chapter VI
we prove Riemann's mapping theorem for connected domains
which asserts the following:
\medskip

\noindent
{\bf 6.1 Theorem.} \emph{For every  $z_0\in\Omega$ 
there exists a unique analytic covering map
$f$ of $\mathcal L$-type where $f(0)=z_0$ and $f'(0)$ is real and positive.}
\bigskip


\noindent
{\bf 6.2 The multi-valued inverse to $f$.}
We take the theorem above for granted and discuss some consequences.
Let $f$ be a an analytic covering map as above.
To distinguish  the
$z$-coordinate in $\Omega$ from $D$ we let $w$ be the complex
coordinate in $D$.
To begin with $f$ yields a biholomorphic map from a small open
disc $D_*$ centered at the origin in $D$
to a small open neighborhood $U_0$ of
$z_0$. It gives the inverse analytic function
$F(z)$ defined in $U_0$ such that
\[ 
F(f(w))=w\quad\, w\in D\,.
\]
Next, let $\gamma$ be a curve in
$\Omega$ with
$\gamma(0)=z_0$.
Since $f$ is of $\mathcal L$-type there exists a unique lifted
curve
$\gamma^*$ in $D$ with
$\gamma^*(0)=0$.
Now the germ of $F$ at $z_0$ can be continued analytically along
$\gamma$ where
\[ 
T_{\gamma(t)}(F(\gamma(t))=\gamma^*(t)\quad\colon\,
0\leq t\leq 1\tag{i}
\]
Hence we get a multi-valued analytic function $F$ in $\Omega$.
It gives an inverse to $f$ in the following sense: Let
$w\in D$ and consider the curve
$t\mapsto t\cdot w$ in $D$. Now $t\mapsto f(t\cdot w)$
is a curve $\gamma$ in
$\Omega$ and by the construction (i) we have
\[
T_{\gamma(t)}(F(f(t\cdot w))=tw\quad\colon\,0\leq t\leq 1\tag{ii}
\]
We may express this by saying that the composed function
$F\circ f$ is the identity on $D$.
\medskip



\noindent
{\bf 6.3 Constructing single-valued functions.}
Consider the situation in Theorem 6.2, i.e. $f$ is a covering map
of $\mathcal L$-type  from
$D$ onto $\Omega$.
Let $g(w)$ be some analytic function in
$D$ whose range $g(D)\subset \Omega$ and $g(0)=z_0$.
We  use $F$ to construct a 
single-valued analytic function
$F\circ g$  in $D$. Namely, let $w\in D$
which gives  the curve $\gamma_w$ parametrized by
$t\mapsto g(t\cdot w)$ in $\Omega$
where $\gamma_w(0)=z_0$.
We can continue $F$ along
this curve
and when $t=1$ we 
get the value
\[ 
T_{\gamma_w(1)}(F(g(w))
\]
It is clear that this gives an analytic function in
$D$ defined by
\[ 
F\circ g(w)= T_{\gamma_w(1)}(F(g(w))
\]
This construction can be performed for every
$g\in\mathcal O(D)$ such that $g(0)=z_0$ and $g(D)\subset \Omega$.
Hence we have
proved the following:
\medskip

\noindent
{\bf 6.4 Proposition.}
\emph{Let $\mathcal O_*(D:\Omega)$  denote the family of analytic functions $g$
in
$D$ where $g(0)=z_0$ and $g(D)\subset \Omega$.
Then there exists
a map from $\mathcal O_*(D:\Omega)$ into $\mathcal O(D)$
given by:}
\[ 
g\mapsto F\circ g\,.
\] 
\emph{Here
$F\circ g(0)=0$ and the range
$(F\circ g)(D)\subset D$.}
\bigskip



\noindent
{\bf 6.5 M�bius transforms.}
Let $f$ be a as in Theorem 6.1 and
identify the fundamental group
$\pi_1(\Omega)$ with
homotopy classes of closed curves at $z_0$.
Theorem xx gives a bijective map between
elements in the group $\pi_1(\Omega)$ and the discrete subset 
$f^{-1}(z_0)$ of $D$.
Let $a$ be a point in this  fiber, i.e.
here $f(a)=z_0$. For each
$0\leq\theta<2\pi$
we get an analytic function in $D$ defined by
\[ 
g(w)=f(e^{i\theta}\cdot \frac{w+a}{1+\bar a\cdot w})
\]
Here $g(0)=f(a)=z_0$ and
the complex derivative at $w=0$ becomes
\[ 
g'(0)=f'(a)\cdot  e^{i\theta}\cdot (1-|a|^2)
\]
We can choose $\theta$ so that
$f'(a)\cdot  e^{i\theta}$ is real and positive.
With this choice of $\theta$ it follows from
the uniqueness in Theorem 6.1
that $g=f$.
Hence the function $f$ satisfies

\[ 
f(w)=f(e^{i\theta}\cdot \frac{w+a}{1+\bar a\cdot w})\tag{*}
\]
\bigskip

\noindent
This means that $f$ enjoys certain invariance
properties.
We return to a discussion at the end of section xx.



\medskip





\centerline {\bf 6.6 Inverse multi-valued functions.}
\bigskip




\noindent
Let $\phi$ be an analytic function 
defined in some
open and connected subset $U$
of ${\bf{C}}$. We assume that the derivative
is $\neq 0$ at every point and get the open image domain
$\Omega=\phi(U)$.
Since $\phi$ is locally conformal it is in particular a 
local homeomorphism.
We add the hypothesis that $\phi$ yields a covering map of
$\mathcal L$-type.
Consider some $\zeta_0\in\Omega$ and
put $x_0=\phi(\zeta_0)$.
We get a germ $f(x)$ of an analytic function at
$x_0$ using the local inverse of $\phi$, i.e. since
$\phi'(\zeta_0)\neq 0$
there exists a small open disc
$D_\delta(\zeta_0)$ such that
\[ 
f(\phi(\zeta)=\zeta\quad\colon\quad
|\zeta-\zeta_0|<\delta
\]
In fact, we simply find the  convergent power series
\[ 
f(x)=
\sum\, c_\nu(x-x_0)^\nu
\]
where $c_0,c_1,\ldots$ are determined so that
\[
\sum\, c_\nu\bigl (\phi(\zeta)-\phi(\zeta_0)\bigr)^\nu=\zeta
\]
\medskip

\noindent
{\bf 6.7 Proposition.}
\emph{The germ $f$ at $x_0$ extends to a multi-valued analytic function in
$U$.}
\medskip

\noindent
\emph{Proof.} 
Let $\gamma$ be a curve in
$U$ having $x_0$ as initial point.
The lifting lemma gives a unique curve
$\gamma^*$ in $\Omega$.
The required analytic continuation of $f$
along
$\gamma$ now follows when we apply the Heine-Borel Lemma cover
the compact set $\gamma$ with a finite set of discs which
are homemorphic images of discs
in $\Omega$ whose consequtive union covers
$\gamma^*$. Then we use that
$\phi$ is everywhere analytic. The result is that
the germ $T_\gamma(f)$ at the end-point
$\zeta_1=\gamma(1)$ satisfies
\[
T_\gamma(f)(\phi(x))=x
\]
where $x$ is close to the point $\phi(\gamma^*(1))$
in $\Omega$. So in particular
\[ T_\gamma(f)(\gamma(1))= \gamma^*(1)
\]
which clarifies how to determine  values of
the multi-valued  analytic function.
\medskip

\noindent
{\bf 6.8 Remark.}
It is instructive to consider some specific cases.
Consider the entire function
$\phi(\zeta)=e^\zeta$. With $\Omega={\bf{C}}$
the image domain $U$
is the punctured complex plane. If we take
$x_0=1$ and $\zeta_0=0$
we find that $f$ is the multi-valued Log-function
where we start with the local branch at $x_0=1$ for which
$\log\,1=0$.
Next, let
us regard the polynomial $\phi(\zeta)= \zeta^2$.
in order to get a covering we must exclude the origin to
ensure that
$\phi'(\zeta)\neq 0$. So if $\Omega={\bf{C}}\setminus\{0\}$
we get a covering whose image set $U$
also becomes the punctured complex plane.
In this case the inverse fiber consists of two points and
the function $f(z)$ is the multi-valued
square-root of $z$.

\medskip

\centerline{\bf 6.9 Constructing  single-valued functions.}

\medskip

\noindent
Let $\Omega$ be a connected open set and consider some
multi-valued analytic function
$F$ in $\Omega$.
Let $U$ be some open and 
\emph{simply connected} set.
Consider some $h\in\mathcal O(U)$ whose image set $h(U)$
is contained in $\Omega$. No further conditions on $h$ are imposed, i.e.
the inclusion
$h(U)\subset\Omega$ may be strict and
the derivative of $h$ may have zeros.
Using $F$ we produce single valued analytic functions
in $U$ by the following procedure.
Let us fix a point $\zeta_0\in U$
and put $x_0=h(\zeta_0)$.
At $x_0$ we haver the family of local branches of
$F$. Let $f_*$ be one such local branch.
Next, let $\gamma$
be a curve in $U$ where $x_0$ is the initial point
and $x=\gamma(1)$ denotes the end-point. 
In $\Omega$ we get the image curve
\[ 
t\mapsto  h(\gamma(t))\tag{i}
\]
Now $f_*$ has an analytic continuation along
the curve in (i).
When $t=1$ we arrive at the endpoint $\gamma(1)$
which we denote by $x$.
At $x$ 
we can evaluate the local branch $T_\gamma(f_*)$.
Next, let $\gamma_1(t)$ be another curve
in $U$ with  the same end-point $x$ as $\gamma$.
By assumption $U$ is simply connected which means that
the  curves
$\gamma$ and $\gamma_1$ are homotopic.
It is clear that the homotopy in $U$ implies that
the two image curves
obtained via (i) are homotopic in  the curve family in
$\Omega$ which joint $x_0$ and $x$.
It follows that
the image curves constructed via (i) are homotopic 
The monodromy theorem applied to $F$ implies that
\[
T_\gamma(f_*)(x)=T_{\gamma_1}(f_*)(x)\tag{ii}
\]


\noindent
We conclude that (ii) gives an analytic function in $U$.
Denote by
$\mathcal O(U)_\Omega$
the family of analytic functions in
$U$ whose image is contained in
$\Omega$.
With these notations the discussion above gives:

\medskip

\noindent
{\bf 6.10 Proposition.}
For each point $\zeta_0\in U$ there exists a map
\[
\rho\colon\, \mathcal O(U)_\Omega\times M\mathcal O(\Omega)(x_0)
\to \mathcal O(U)
\]
where $x_0=h(\zeta_0)$
and  for a  pair
$h\in \mathcal O(U)_\Omega$ and $f_*\in M\mathcal O(\Omega)(x_0)$
the analytic function $\rho(h,f_*)$ 
satisfies
\[ 
\rho(h,f_*)(\zeta)= T_\gamma(f_*)(h(\zeta))\quad\colon\quad \zeta\in U
\]
where $\gamma$  is the $h$-image of any curve
in $U$ which joins
$\zeta_0$ with
$\zeta$.
\bigskip

\noindent
{\bf Remark.}
Keeping $h$ fixed we notice that
the map $f_*\to \rho(h,f_*)$ is a ${\bf{C}}$-algebra homomorphism from
the complex ${\bf{C}}$-algebra 
$M\mathcal O(\Omega)(x_0)$ into $\mathcal O(U)$.








\newpage


\centerline {\bf 7. The $p^*$-function.}
\medskip

\noindent
We  construct a special harmonic function which 
will be  used
to get solutions to the Dirichlet problem in
XXX.
Let $\Omega$ be an open and connected set in
${\bf{C}}$.
Its closed complement has connected components.
Let $E$ be
such a connected component.
To each $a\in E$ we get the winding number
$\mathfrak{w}_a(\gamma)$. If $b$  is another point in
$E$ which is sufficiently close to  $a$
it is clear that
\[
\bigl|\frac{1}{\gamma(t)-a}-\frac{1}{\gamma(t)-b}\bigr|<
\bigl|\frac{1}{\gamma(t)-a}\bigr |
\]


\noindent
Rouche's theorem from 1.4 implies that
$\mathfrak{w}_a(\gamma)=\mathfrak{w}_b(\gamma)$, i.e. 
for every closed curve $\gamma$ in $\Omega$,
the winding number
stays constant in each connected component of
${\bf{C}}\setminus\Omega$. 
This  enable us to construct single valued  Log-functions
in $\Omega$.
Namely, let $a\in E$ where $E$ is a connected
componen in the complement of
$\Omega$.
Consider 
$f=\text{Log}\,(z-a)$ and choose a single valued branch
$f_*$ at some point $z_0\in\Omega$. 
If $\gamma\subset\Omega $ is a closed curve with initial point at
$z_0$ the analytic continuation along $\gamma$ of the Log-function
gives:
\[
T_\gamma(f_*)= f_*+2\pi i\cdot \mathfrak{w}_a(\gamma)\tag {1}
\]


\noindent
Next, if $b$ is another point in $E$ we consider
$g_*=\text{Log}(z-b)$ and obtain
\[
T_\gamma(g_*)= g_*+2\pi i\cdot \mathfrak{w}_b(\gamma)\tag {2}
\]



\noindent
Since
$\mathfrak{w}_b(\gamma)=\mathfrak{w}_b(\gamma)$ it follows that
\[
T_\gamma(f_*)-T_\gamma(g_*)=f_*-g_*\tag{3}
\]
Hence the difference $\text{Log}(z-a)-\text{Log}(z-b)$
is a \emph{single valued} analytic function in $\Omega$. Taking  
is exponential we
find $\Psi(z)\in\mathcal O(\Omega)$ such that
\[
e^{\Psi(z)}=
\frac{z-a}{z-b}\tag{4}
\]


\noindent
Since $a\neq b$
we see that
$\Psi(z)\neq 0$
for all $z\in\Omega$. Next, we get the
harmonic function defined in 
$\Omega$
by
\[
p(z)=\mathfrak{Re}\bigl(\frac{1}{\Psi(z))}\bigr)=
\frac{\mathfrak{Re}(\Psi(z)}{|\Psi(z)|^2}\tag{*}
\]


\noindent
Notice that
$\mathfrak{Re}(\Psi(z))=\text{Log}\, |z-a|-\text{Log}|z-b|$
and since
$\text{Log}\, |z-a|\to-\infty$ as $z\to a$
we see from (*) that
\[
\lim_{z\to a}\, p(z)=0\tag{**}
\]


\noindent
Notice also that
$\Psi(z)$ exends to a continuous function on
$\bar\Omega\setminus (a,b)$ and we can
choose a small $\delta>0$ such that
\[ 
\text{Log}|z-a|-\text{Log}\,|z-b|<-1\quad\colon\quad
|z-a|\leq\delta\tag{ii}
\]

\noindent Then (i) and (ii) give

\medskip

\noindent {\bf 7.1 Theorem.}
\emph{Let $a\in\partial\Omega$ be such that
the connected component of ${\bf{C}}\setminus \Omega$
which contains $a$ is not reduced to the single point $a$.
Then there exists a harmonic function
$p^*(z)$
in $\Omega$
for which}
\[
\lim_{z\to a}\, p^*(z)=0
\]
\emph{and there exists $\delta>0$ such that}
\[
\max_{\{|z-a|=r\}\,\cap\Omega}\, p^*(z)<0\quad\colon\quad
z\in D_a(r)\,\cap\,\Omega
\]



\newpage

\centerline{\bf 8. Extensions by reflection}
\bigskip

\noindent
{\bf Introduction.} 
\emph{Das Spiegelungsprinzip} is due to H. Schwartz.
First we describe the standard case.
Let $f(z)$ be  analytic  in the upper half plane
$U_+=\{\mathfrak{Im}\,z>0\}$.
Let $J(a,b)=\{a<x<b\}$ be an interval, on  the real axis.
Suppose that $f$ extends to a continuous function to this open interval
and takes real values. In the lower half-plane $U_-$
we get the analytic function
\[  
f_*(z)=\bar f(\bar z)\tag{i}
\]

\medskip
\noindent By the result in XX the two functions are analytic
continuations of each other over
$(a,b)$. This means that $f$  has an analytic extension
to the open set $\Omega={\bf{C}}\setminus J$, where
$J_*=(-\infty,a]\cup\, [b,+\infty)$
is the closed complement of $(a,b)$ on the $x$-axis.
Next, suppose that
$e^{i\theta}f(z)$ extends to a real-valued function on
$(a,b)$ for some $\theta$. 
After multiplication
with $e^{-i\theta}$ we get an extension of $f$. That is, one has only to require
that the argument of $f$ is constant to obtain an
analytic continuation.
Suppose now that the argument of $f$ is constant  over a family of
pairwise disjoint  intervals $\{J(a_\nu,b_\nu)\}$. Then we get analytic
continuations across each interval. In particular one has: 
\bigskip

\noindent
{\bf 8.1 Theorem.}
\emph{Let $a_1<\ldots<a_N$ be a finite set of real numbers
and assume that $f$ extends to a continuous function on each of the intervals}
\[ 
J_0=(-\infty,a_1)\quad\colon \,J_\nu=(a_\nu,a_{\nu+1})\,\colon
2\leq\nu\leq N_1\quad \colon\, J_N=(a_N,+\infty)
\]
\emph{and on every such interval the argument of $f$ is some constant.
By successive reflections over there  intervals
we obtain a in general multi\vvv valued  analytic
of $f_*^\nu$ defined in 
${\bf{C}}\setminus (a_1,\ldots,a_N)$.}
\medskip

\noindent
{\bf Example.} In the upper half-plane $U_+$
we consider the analytic function
\[ 
f(z)=\sqrt{z}\cdot\sqrt{1-z}
\]
The single-valued branches of the root functions are chosen so that
\[ 
\sqrt{z}=\sqrt{r}\cdot e^{i\theta/2}\quad \colon\quad
\sqrt{z-1})=\sqrt{1+r^2-2r\cdot\text{cos}\,\theta}\cdot e^{i\phi}
\quad\colon z= re^{i\theta}
\] 
where 
$0<\theta<\pi$ and $\phi$ is the outer angle of the triangle in figure xx.
So here $0<\phi<\pi$ holds.
As we approach a point $0<x<1$
we get the boundary value
\[ 
f(x)=\sqrt{x}\cdot i\cdot\sqrt{1-x}
\]
Now get the analytic continuation $f_*^1$
across the interval $J_1=(0,1)$ which becomes an analytic function
defined in the lower half-plane $U_-$ by
\[
f_*^1(z)=-\bar f(\bar z)
\]
Notice that the minus-sign appears in order that
$f(x)=f_*^1(x)$ holds for $0<x<1$.
Suppose now that $x>1$. Then we get
\[ 
\lim_{y\to 0}\,f_*^1(x-iy)=\lim_{y\to 0}\,-\bar f(1(x+iy)=-\sqrt{x}\cdot\sqrt{x-1}
\]
So $f$ and $f_*^1$ do not agree on
the real interval $(1,+\infty)$. At the same time
$f_*^1$ can be continued analytically across
$(1,+\infty$ and gives 
an analytic function
$f_+{**}$ defined in $U_+$ where we obtain
\[ 
f_+^{**}(z)= -f(z)\quad\colon\quad z\in U_+
\]
Another analytic continuation of
$f_*^1$ takes place across $(-\infty,0)$.
When $x<0$ we have
\[
\lim_{y\to 0}\,f_*^1(x-iy)=
\lim_{y\to 0}\,-f(1(x+iy)=
\]
After $f$ has been extended to the lower half-plane
where we get an analytic function denoted by
$f_*$ we notice that
$f_*$ by the construction also
has boundary values with a constant argument as
we approach points on the real axis form below.
So $f_*$ also extends to the upper half-plane where we encounter
a new analytic function $f^*(z)$.
Next, we can continue $f^*$ to the lower half-plane and so on.
The result is that
$f$  extends to a multi-valued function in
${\bf{C}}\setminus (0,1)$.

\medskip

\noindent
{\bf 8.2 The use of conformal maps.}
For local extensions there exists  a  general result.
Let $D$ be an  open disc and $\gamma$ a Jordan arc
which joins two points on $\partial D$ and separates
$D\setminus\gamma$ into a pair
of disjoint Jordan domains.
Let $f$ is analytic in one of the Jordan domains, say
$D^*$. Assume also that $f$ extends to a continuous and real-valued
function on $\gamma$.
Now there exists a conformal map from $D^*$ to
the upper half-plane and using this it follows that
$f$ extends analytically across $\gamma$. of course, the
extension $f_*$ will in general only exist in a small
domain close to $\gamma$, i.e. this is governed via the
conformal mapping. But here exists at least
a locally defined  analytic continuation across $\gamma$.
\medskip

\noindent
{\bf 8.3 Boundary values on circles}
Let $f(z)$ be as above and suppose it extends continuously to
$\gamma$ where the absolute value is constant, say 1.
Using a conformal map from $D_*$ to the unit disc we may
assume that $\gamma$ is an interval of the unit disc $D$
and
$f$ is analytic in a small region $U\subset D$ where
$\gamma$ appears as a relatively open subset of
$\partial U$.
By hypothesis  $f(\gamma)$
is a subset of another unit circle and using a conformal map
from the disc bordered by this unit circle we
get the situation in 7.2. and conclude that
$f$ continues analytically across $\gamma$.
More generally, using a locally defined conformal map
there exists an analytic
extension of $f$ across $\gamma$ if we only assume that
the continuous boundary values of $f$ on
$\gamma$ are contained in some locally defined 
\emph{real-analytic curve}.
Finally, by a two-fold application of conformal mappings
we get the following
quite general result:
\medskip

\noindent
{\bf {8.4 Theorem.}}
\emph{Let $f(z)$ be analytic in a Jordan domain
$\Omega$ and suppose that
$\gamma$is an open arc of $\partial\Omega$ such that
$f$ extends continuosly from
$\Omega$ to $\Omega\cup\gamma$
and the restriction $f|\gamma$has a range
$f(\gamma)$  contained in a 
simple real-analytic curve $\gamma^*$.
Then $f$ extends analytically across $\gamma$, i.e. there exists
an open and connected neighborhood $U$ of $\gamma$ such that
the original $f$-function extends to the connected domain
$\Omega\cup U$.}
\medskip

\noindent
{\bf{Remark.}}
Theorem 8.4 follows from the fact that if
$\Omega_2$ and $\Omega_2$
are two Jordan domains whose boundaries
both are \emph{real analytic}
closed Jordan curves, then a conformal map from
$\Omega_1$ to $\Omega_2$ extends to a conformal map 
from an open neighborhood of
$\bar\Omega_1$ to an open neighborhood
of $\bar\Omega_2$.




\newpage
\centerline{\bf 9. The elliptic modular function}

\bigskip

\noindent
{\bf Introduction.}
We  shall construct
an analytic function
$\phi(z)$  in
the upper half-plane
$U_+=\mathfrak{Im}\, z>0$ whose
complex derivative of $\phi$ 
is everywhere $\neq 0$ and the image  $\phi(U_+)$ is equal to
the connected open set
$\Omega={\bf{C}}\setminus \{0,1\}$, i.e. the two points 0 and 1 are removed from
the complex plane.
So $\phi$ is locally conformal but not 1\vvv 1.
Moreover the function
is invariant under a group of M�bius maps which preserve
$U\uuu +$. Consider first the map
\[
z\mapsto \frac {z}{2z+1}=w\tag{1}
\]
Then
\[
 \mathfrak{Im}(w)=\frac{\mathfrak{)m}(z)}{|2z+1|^2}>0
 \]
 so (i) is a map from $U_+$ into itself
 asnd is conformal because we have the inverse map
 \[
 z=\frac{w}{1-2w}
 \]
Another conformal mapping on $U_+$ is $z\mapsto z+1$.
Together with (i) it generates a group 
$\mathcal F$ of conformal mappings on $U_+$.
We are going to construct an analytic function $\phi$ in
$U_+$ which is $\mathcal F$-invariant, i.e. 
 \[
\phi(z)=\phi(z+1)= \phi(\frac {z}{2z+1})\quad\colon\,
z\in U\uuu +\tag{2}
\]
and the range 
\[
 \phi(U_+)= {\bf{C}}\setminus \{0,1\}\tag{3}
\]
When (ii) holds one says that
$\phi$ is an $\mathcal F$\vvv automorphic function and
the subsequent construction of $\phi$ will show that
for each point $w\in U\uuu +$ one has the equality
\[
\mathcal F(w)= \{ z\in U_+\,\colon\, \phi(z)=w\}\tag{4}
\]
where the left hand side is the $\mathcal F$-orbit of $\phi$.
\medskip

\noindent
{\bf{The construction of $\phi$}}.
Consider the simply connected domain:
\[
V_0=U_+\,\cap {\ |z-1/2|>1/2}\cap\,
\{0<\mathfrak{Re}\, z<1\}
\]


\noindent
Here $\partial V_0$ consists of three pieces:
The vertical half\vvv lines $\ell_0=\{x=0\,\colon\,y>0\}$,
and
$\ell_0=\{x=1\,\colon\,y>0\}$, together with
 half\vvv circle
 \[
T^+_0=\{ 1/2+1/2\cdot e^{i\theta}\}\quad\colon\quad 0<\theta<\pi
\]


\noindent
Riemann's mapping theorem 
gives a  conformal mapping $\phi_0$ from
$V_0$ onto $U_+$ such that
$\phi\uuu 0$ yields  bijective maps 
from
$\ell_0$ onto $(-\infty,0)$ and 
$T_0$ onto $(0,1)$, and finally $\ell_1$ onto
$(1,+\infty)$. In particular $\phi_0(0)=0$
and $\phi_0(1)=1$ hold and as $z\to \infty$ in $V\uuu 0$ then
$\phi\uuu 0(z)\to \infty$ in $U\uuu +$.
See figure 1 for an illustration of this conformal mapping.
Following Schwarz we shall perform  reflections
to obtain an analytic function $\phi\uuu *$ defined in
the domain
\[
V\uuu *=\{0< \mathfrak{Re}\, z<1\}\cap\, \mathfrak{Im}\, z>0\}
\]
whose image is equal to ${\bf{C}}\setminus \{0,1\}$.

\medskip

\noindent
{\bf{The first reflection.}}
Consider the open set $W\uuu 0$ defined by
\[
W\uuu 0=\{ z\quad\colon \frac{\bar z}{2\bar z-1}\in V\uuu 0\}\tag{1}
\]
\medskip


\noindent
{\bf{Exercise.}}
Show that $W_0$ is a simply connected domain
bordered by three circular
arcs, i.e. 
\[
W_0=\{|z-1/4|>1/4\}\,\cap 
\{|z-3/4|>1/4\}\cap \{|z-1/2|<1/2\}
\]
A hint is that if $z=iy\in\ell_0$ thrn
\[
w=\frac{-iy}{-2iy-1}=\frac{iy}{2iy-1}\implies |w-1/4|= 1/4\tag{i}
\]
Similarly, if $z=1+iy\in \ell_1$ we find
\[
w=\frac{1-iy}{2-2iy-1}=\frac{1-iy}{1-2iy}\implies |w-3/4|= 1/4\tag{ii}
\]
Finally we have the half-circle $T_0^+=\{|z-1/2|= 1/2\,\cap U_+\}$
and with $z=1/2+1/2e^{i\theta}$ we get
\[
w=\frac{1/2+1/2 e^{-i\theta}} {1+e^{-i\theta}-1}=
\frac{e^{i\theta}+1}{2}\implies |w-1/2|= 1/2\tag{iii}
\]
\medskip


\noindent
Next, in  $W\uuu 0$ a two\vvv fold complex conjugation gives
the analytic function
$g\uuu 0(z)$ defined by
\[
g\uuu 0(z)= \bar \phi\uuu 0( \frac{\bar z}{2\bar z\vvv 1})       
\]
At the same time we notice that
$\phi_0$�is real\vvv valued on 
$T\uuu 0^+$ and then  (iii) above gives
\[
g\uuu 0(z)=\phi\uuu 0(z)\,\colon\, z\in T\uuu 0^+\tag{iv}
\]
By Schwarz' reflection principle the pair $\phi_0,g_0)$
yields  an analytic function 
$\phi_1$ defined in the simply connected set
\[
V_1=V\uuu 0\cup T\uuu 0^+\cup W\uuu 0
\]

\medskip

\noindent
{\bf{The second step.}}
Above we  constructed an analytic function
$\phi\uuu 1$ in $V\uuu 1$.
Since $\phi\uuu 0(iy)$ and $\phi\uuu 0(1+iy$
takes real values for all�$y>0$, we see
that
$\phi\uuu 1$ takes real values on the half\vvv circles
\[
T^+\uuu{10}=\{|z-1/4|= 1/4\,\cap U_+\}
\quad\colon
T^+\uuu{11}=\{|z-3/4|= 1/4\,\cap U_+\}
\]
Again we can apply  Schwarz reflection principle to
get an analytic extensions of $\phi\uuu 1$
across each of these half\vvv circles. More precisely, put
\[
W\uuu{10}=\{ \bar z\quad\colon\quad \frac{\bar z}{4\bar z\vvv 1}\in
V\uuu 1\}
\]
If $z=1/4+e^{i\theta}/4$ belongs to $T\uuu {10}^+$
we obtain
\[
\frac{1/4+e^{\vvv i\theta}/4}{e^{\vvv i\theta}}=1/4+e^{i\theta}/4
\]
Reflection gives  an analytic function $g\uuu{10}$
in $W\uuu{10}$ defined by
\[
g\uuu{10}(z)= 
\bar \phi\uuu 1(\frac {\bar z}{4\bar z\vvv 1})
\]
The reader may also verify that
\[
W_{10}=\{|z-1/8|>1/8\} \cap \{|z-3/8|>1/8\} \cap
\{|z-1/4|<1/4\}
\]
To get an extension across $T^+\uuu {11}$ we put
\[
W\uuu{11}=\{ \bar z\quad\colon\quad \frac{\bar z}{4\bar z\vvv 3}\in
V\uuu 1\}
\]
If $z=3/4+e^{i\theta}/4$
we obtain
\[
\frac{3/4+e^{\vvv i\theta}}{3+e^{\vvv i\theta}\vvv 3}
=3/4+e^{i\theta}/4
\]
This gives  an analytic extension where $g\uuu {11}(z)$
is defined in $W\uuu{11}$ by
\[
 g\uuu {11}(z)=
 \phi\uuu 1(\frac{\bar z}{4\bar z\vvv 3})
 \]
As a result we get an analytic function
$\phi$
defined in the simply connected domain bordered by
$\ell_0$ and $\ell_1$ and the union of four
half-circles of radius 1/8 centered at the points
$1/8, 3/8, 5/8,7/8$.
See figure � xx for an illustration.


\medskip

\noindent
At this stage
 it is clear how one proceeds to construct larger and larger domains
 $\{W\uuu k\}$ and analytic
functions $\phi\uuu k$
 via reflections over suitable half\vvv circles. See
 figure XX.
The result is an analytic function $\phi\uuu *(z)$ defined in
the half\vvv strip 
\[
\square\uuu +=0<\mathfrak{Re}\, z<1\}\cap 
\{\mathfrak{Im}\, z<0\}
\]



\noindent
The construction shows that the derivative of  the
analytic function $\phi\uuu *$ in $\square\uuu +$
is everywhere $\neq 0$ and  the range
is ${\bf{C}}\setminus \{0,1\}$
Next, since $\phi\uuu *$ is real\vvv valued on $\ell\uuu 0$ and
$\ell\uuu 1$ it extends by reflection over these two vertical lies and
the reader may verify that we obtain an analytic function
$\phi$ defined in the whole upper half\vvv plane which is 1\vvv periodic, i.e.
\[
\phi(z+1)=\phi(z)
\]

\medskip

\noindent
{\bf{Exericse.}}
Verify that the constructions imply thaty
\[ 
\phi(z)= \phi(\frac{z}{2z+1})=\phi(z+1)\quad\colon z\in U\uuu +
\]
and that the orbit equation (4) holds. Finally, show that
the construction entails that
the range of $\phi$ is ${\bf{C}}\setminus\{0,1\}$.

\medskip

\noindent
{\bf{9.1  The multi-valued inverse}}.
Since 
$\phi $ is locally conformal there exists 
a multi-valued inverse function  denoted by
$\mathfrak{m}$.
Namely, set $\Omega={\bf{C}}\setminus \{0,1\}$ and consider the point
$\zeta_0=i$.
We first find   the  unique point $z_0\in V_0$ such that $\phi(z_0)=i$.
At $\zeta_0$ we get a unique germ $\mathfrak{m}_0(\zeta)
\in\mathcal O(\zeta_0)$
such that

\[ 
\mathfrak{m}_0(\phi(z))= z
\] 
hold for $z$ close to $i$.
Next, let $\gamma$ be a curve in $\Omega$ which starts at $i$ and has some end-point
$\zeta_1$.
Since $\lambda$ is locally conformal there exists a unique curve
$\gamma^*$ in $U_+$ such that
\[
\phi(\gamma^*(t))= \lambda(t)\quad\colon\, 0\leq t\leq 1
\]
As explained in
xx 
there exists 
an analytic extension of
$\mathfrak{m}_0$ along $\gamma$ which locally produces inverses of
the $\phi$-function.
The resulting multi-valued $\mathfrak{m}$-function
gives  the set of values  $W(\mathfrak{m},\zeta)$ for every 
$\zeta\in\Omega$ which is in a 1-1 correspondence
with the inverse fiber
$\lambda^{-1}(\zeta)$ and hence equal to an orbit under the
group 
$\mathcal F$.



 


\newpage


\centerline
{\bf \large{10. Poincare's theory of Fuchsian groups}}
\bigskip

\noindent
The  theory of Fuchsian groups was
created by Poincar�.
His   articles
\emph{Th�orie des groupes fuchsiens} and
\emph{Memoire sur les fonctions
fuchsiennes} were  published 1882
in the first
first volume of Acta Mathematica and
the article \emph{Memoire sur les groupes klein�ens}
appeared in  volume III.
The last  article  is  more advanced  and we shall not discuss
Kleinan groups here. Nor do we discuss
the article \emph{Memoire sur les fonctions
z�tafuchsiennes}. The connection to arithmetic was
presented in a later article \emph{Les fonctions
fuchsiennes et l'Arithm�tique} from 1887. One should also
mention the
article
\emph{Les fonctions
fuchsiennes et l'�quation $\Delta(u)=e^u$}
where Poincar� proved that this second order differential equation has a 
subharmonic solution with prescribed
singularities on every closed Riemann surface attached to an algebraic equation.
The last work started  potential theoretic analysis
on complex manifolds.
Here we  only discuss   material
from the first two cited articles.
\medskip

\noindent 
Poincar� was  inspired by
earlier work, foremost by Bernhard Riemann, Hermann
Schwarz and  Karl Weierstrass.
For example, he  used the construction  of multi-valued analytic
extensions by Weierstrass which leads to the
\emph{Analytische Gebilde} of a multi-valued function
$f$ defined in some connected
open subset  $\Omega$ of 
${\bf{C}}$. This \emph{Analytische Gebilde} is a connected
complex manifold $X$ on which $f$
becomes a single valued analytic function $f^*$. More precisely, there
exists a locally biholomorphic map
\[
\pi\colon\, X\mapsto\Omega
\]
When $U\subset\Omega$ is simply connected
the inverse image $\pi^{-1}(U)$ is a union of pairwise
disjoint open sets
$U^*_\gamma$ where
the single-calued analytic function $f^*$
is determined by a branch $T_\gamma$ of $f$, i.e. one has
\[
T_\gamma(f)(\pi(x)=f^*(x)\quad\colon
x\in U^*_\gamma
\]

\medskip

\noindent
Major contributions are also due to Schwarz.
In 1869 he used the reflection principle
and calculus of variation to settle the Dirichlet problem
and used this to prove  the uniformisation  theorem for
connected 
domains bordered by $p$ many real analytic and closed Jordan curves where
$p$ in general is $\geq 2$.
Of special interest is the multi-valued $\mathfrak{m}$-function
from � 8 defined in ${\bf{C}}\setminus\{0,1\}$.
It is related to the elliptic integral of the first kind
and hence to Jacobi's $\mathfrak{sn}$-function
which appears in the equation of motion
when a rigid body rotates around a fixed point.
\medskip

\noindent
{\bf A comment.}
The study   of Fuchsian groups  was not restricted
to analytic function theory.
Poincar�'s main concern  was to develop the
theory of  differential systems, both linear and non-linear.
His research was
also directed towards to the general theory about abelian functions and their
integrals, inspired by Abel's pioneering work.
Hundreds of text-books have appeared after
Poincar�. Personally I find that  his own and often quite
personal
presentation 
superseeds most   text-books which individually only
treat
some  fraction from the great visions by Poincar�.
His  original work
offers therefore a good introduction for the student
who enters studies
about linear and non-linear
differential systems in an algebraic context, together with
function theory which leads to Fuchsian as well as
Kleinian
groups and there associated functions. See in particular
the book \emph{Analyse de ses travaux scientifiques}
which contains a survey of his
the scientific work. In several chapters Poincar�
describes in his own words
various research areas from the period between
1880 until 1907, which has the merit that
it 
not only contains a summary of results
but also explanations of the
the main
ideas and methods which led to the theories.
\medskip

\noindent
Of course  there exists more recent advancement   in function theory.
Here one should foremost
mention work by Lars Ahlfors. So in addition
to the cited reference above I recommend text-books by
Ahlfors, especially his book  \emph{Conformal Invariants} which 
contains material about the 
theory of extremal length which was 
created by Arne Beurling in the years 1942\vvv 1946.
From a complex analytic   point of view the
discoveries by Ahlfors and Beurling
have a wider scope and has led to
many still unsolved problems in complex analysis.
including the study of  quasi-conformal mappings.
In addition to this we refer to the excellent material in the
text\vvv book [A-S]
by Ahlfors and Sario about Riemann surfaces.



\newpage


\end{document}











