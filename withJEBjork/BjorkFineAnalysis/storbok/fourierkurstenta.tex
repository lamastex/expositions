



\documentclass{amsart}
\usepackage[applemac]{inputenc}
\addtolength{\hoffset}{-12mm}
\addtolength{\textwidth}{22mm}
\addtolength{\voffset}{-10mm}
\addtolength{\textheight}{20mm}

\def\uuu{_}

\def\vvv{-}

\begin{document}




\noindent
{\bf{0.4 Fekete's inequality.}}
The interplay between Fourier series and
analytic functions in the unit disc $D=\{|z|<1\}$
leads to many interesting results.
Consider as in (0.2) the sine series
\[ 
\phi(\theta)=\sum_{n=1}^\infty \,\frac{\sin n\theta}{n}
\]
In $D$ we have the analytic function
\[ 
f(z)=
\sum_{n=1}^\infty\,\frac{z^n}{n}=-\log (1-z)
\]
Notice  that
\[ 
f(x)=-\log (1-x)
\] 
tends to $+|infty$ as $x\to 1$ along the real axis.
The Taylor polynomials
\[ 
S_N(z)=\sum_{n=1}^{n=N}\,\frac{z^n}{n}
\] 
attain their  maximum norms on the closed unit disc
when $z=1$ and here
\[ 
S_N(1)=\sum_{n=1}^{n=N}\,\frac{1}{n}\simeq \log N
\]
At the same time we have seen in (0.2) that
$\mathfrak{Im}(f(z))$ is a bounded function in $D$.
Fekete proved that the example above is 
extremal in the following
sense:

\medskip

\noindent
{\bf{Theorem.}}
\emph{There exists an absolute constant $C$ such that
if
$g(z)=\sum\, c_nz^n$ is an analytic function in
$D$ for which the maximum norm of
$\mathfrak{Im}(g(z))$ is $\leq 1$, then}
\[
\max_{\theta}\, \bigl|\mathfrak{Re}\,\sum_{n=0}^{n=N}
\, c_n\cdot e^{in\theta}\bigr|\leq C\cdot \log N
\quad\colon N\geq 2
\]
Fekete's result will
be proved in � XX during a closer study about
analytic functions in the unit disc and their associated Fourier series.
\medskip





\noindent
{\bf{A  result by Carleman}}.
Let $f(\theta)$
be a $2\pi$-periodic and continuous function.
If $\epsilon>0$ and $N\geq 1$ we 
denote by $\rho(N;\epsilon)$
the 
number of integers $0\leq n\leq N$ for which the maximum norm
\[
\max_\theta\, |S_n(\theta)-f(\theta)|\geq \epsilon
\]
With these notations we prove in � x that
\[
\lim_{N\to\infty}\, \frac{\rho(N;\epsilon)}{N+1}=0\tag{i}
\]
hold for every $\epsilon>0$.
It means that Gibb's phenomenon form a statistical point of view
is exceptional, i.e. "failure of convergence" occurs
only for a sparse subsequence of Fourier's partial sums.
Actually (i) is a consequence of a more precise result
whicgh goes as follows: The continuous function $f$ is uniformly continuos and we set
\[
\omega_f(\delta)=\max\, |f(\theta_1)-f(\theta_2)|\quad\colon
|\theta_1-\theta_2|\leq \delta
\]
\medskip

\noindent
{\bf{0.3 Theorem.}}
\emph{There exists an absolute constant $K$ such that the following hold for
every $2\pi$-periodic continuous function $f$ whose maximum norm is $\leq 1$}
\[
 \frac{\rho(N;\epsilon)}{N}\leq \frac{K}{\epsilon^2}\cdot
 (\frac{1}{N}+\omega_f(\frac{1}{N})^2)\tag{*}
\]
\medskip

\noindent
{\bf{Remark.}}
Since
$\frac{1}{N}+\omega_f(\frac{1}{N})^2$ tends to zero as $N\to+\infty$
 and (*) holds for each $\epsilon>0$ we
 get (i).
 




\noindent
{\bf{Fejer's inequality.}}
Several remarkable inequalities for trigonometric polynomials were  established by
Fejer in [Fejer] where
a central issue is to construct  trigonometric polynomials
expressed by a sine series
which are $\geq 0$ on the  interval $[0,\pi]$.
Consider as
an example is the sine\vvv series
\[
S\uuu n(\theta)=\sum\uuu{k=1}^{k=n}\, \frac{\sin k\theta}{k}
\]
Here $\{S_n(\theta)\}$
are trigonometric polynomials which are odd
functions of $\theta$. They are realted to the analytic function in
the unit disc given by the series
\[
f(z)=\sum\uuu{k=1}^\infty\,\frac{z^k}{k}
\]
Notice that this series represents the  analytic function in $D$ given by 
$\log \, (1-z)$.
This complex log-function extends
analytically across the unit circle $T$ outside $\{z=1\}$.
If $0<\theta<\pi$ we notice that
\[
\mathfrak{Im}(f(e^{i\theta})=-\arg(1-e^{i\theta})=\frac{\pi-\theta}{2}
\]
At the same time
$e^{ik\theta}= \sin\, k\theta$ and therefore
\[
\sum_{k=1}^\infty\,  \frac{\sin k\theta}{k}=\frac{\pi-\theta}{2}\quad\colon 0<\theta<\pi
\]
Moreover, there exist pointwise limits
\[
\lim_{n\to\infty}\, S_n(\theta)=\frac{\pi-\theta}{2}\quad\colon 0<\theta<\pi
\]
At the same time
$S_n(0)=0$ for every $n$
so one cannot expect that 
the pointwise convergence for small positive $\theta$
holds uniformly.
In this connection  Fejer proved the following:
\medskip






\noindent
{\bf{0.2 Theorem.}}
\emph{For every $n\geq 1$ one has the inequality}
\[
0<S\uuu n(\theta)\leq 1+\frac{\pi}{2}\quad\colon\quad
0<\theta<\pi
\]
The upper bound was proved by  in [Fej] and Fejer  conjectured that
$S\uuu n(\theta)$ stays positive on $(0,\pi)$.
This was later confirmed  by 
Jackson in [xx] and
Cronwall in [xx]. 
Here is an occasion to use a computer and plot
graphs of the functions $\{S_n(\theta)\}$ to 
analyze the 
rate of convergence when
$\theta\simeq 0$ and also confirm
the inequality in Fjher's theorem numerically ,i.e. plot
graphs of $S-N(\theta)$ for large values of $n$
and check the validity of the inequalities above
numerically.











\newpage


\centerline{\bf\large{Fourier series}}
\bigskip


\noindent

\centerline{\emph{Contents}}
\bigskip


\noindent
\emph{A: The  kernels of Dini, Fejer and Jackson}

\bigskip


\noindent
\emph{B: Legendre polynomials}



\bigskip


\noindent
\emph{C. The space $\mathcal T_n$}


\bigskip


\noindent
\emph{D. Tchebysheff  polynomials}
\bigskip


\noindent
\emph{E. Fejer series and Gibbs phenomenon}

\bigskip


\noindent
\emph{F. Partial Fourier sums and convergence in the mean}
\bigskip

\noindent
\emph{G. Best approximation by trigonometric polynomials.}



\bigskip


\centerline {\bf{Introduction.}}
\bigskip

\noindent
Fourier series 
were invented by
Fourier   to solve  heat and the wave equations.
We expose results in the 1-dimensional case
and  remark only that
one also constructs Fourier series in several variables of
functions $f(x_1,\ldots,x_n)$ which are $2\pi$-periodic with respect to
each variable in ${\bf{R}}^n$.

\bigskip

\centerline {\bf{Outline of contents.}}
\medskip

\noindent
Section A contains  basic material about  Fourier series
where the kernels of Dini and Fejer are introduced.
At the end of � A we  construct   the Jackson kernel which
give     approximations of a given periodic function $f$ by 
trigonometric polynomials 
where the rate of approximation is
controlled by the modulos of continuity of $f$.
Sections B\vvv C are devoted to
results about  extremal polynomials.
A complex version appears  in � D where
Theorem D.4 relates the transfinite diameter of compact subsets
of ${\bf{C}}$ with Tchebysheff  polynomials.
� F is devoted to  a result by Carleman about
convergence in the mean of partial Fourier sums.
From a statistical point of view this result 
confirms the convergence of Fourier's partial sums where
Theorem F.2 gives an absolute constant $K$
such that for every $2\pi$\vvv periodic and continuous function $f$ whose
maximum norm is $\leq 1$, the following inequality holds for
every positive integer $n$ and each
$0<\delta<\pi$
where $\{s_\nu\}$
are Fourier's partial sums of $f$:
\[
\sqrt{ \frac{1}{n+1}\cdot \sum\uuu{\nu=0}^{\nu=n}\,
||s\uuu \nu\vvv f||^2}\leq 
\frac{1}{\sqrt{n+1}}\cdot
[n^{1+1/2}\cdot \delta\cdot \omega_f(\delta)+2K\delta^{-1/2}+K]
\]
where $\{||s\uuu\nu\vvv f||\}$ denote  maximum norms over
$[0,2\pi]$ and $\omega_f$ the modulos of continuity.
If 
$\epsilon>0$ we 
take
$\delta=\frac{1}{\epsilon n}$
for large $n$,  
the left hand side is  majorised by
\[
\frac{\omega_f(1/\epsilon n)}{\epsilon}+2K\sqrt{\epsilon}+
\frac{1}{\sqrt{n+1}}\cdot K
\]
Keeping $\epsilon$ fixed while $n$ increases this tends to zero
which entials that  "with high probability"
the
maximum norms
of $|||s_\nu-f||$ are small as $\nu$ varies over large
integer intervals.




\medskip

\noindent
Section � H treats   results due to 
de Vall� Poussin about  best approximations by
trigonometric polynomials of prescribed degree where one
starts with some real-valued and continuous $2\pi$-periodic
function $f$. If $n\geq 1$ we denote by $\mathcal T_n$
the $2n+1)$-dimensional real vector space
of trigonometric polynomials of degree $\leq n$, i.e. functions of the form
\[
P(x)=\frac{a_0}{2}+ \sum_{\nu=1}^{\nu=n}\, a_\nu\cdot \cos \nu x+
\sum_{\nu=1}^{\nu=n}\, b_\nu\cdot \sin \nu x
\]
The best approximation of degree $n$ is defined by:
\[
\rho_f(n)=\min_{P\in\mathcal T_n}\, ||f-P||\tag{*}
\]


\noindent
Among the results in � H we mention the following  lower
bound inequality expressed by the Fourier coefficients of $f$ defined
by
\[
\widehat f(n)= \frac{1}{2\pi}\int_0^{2\pi}\, e^{-inx}\cdot f(x)\,dx
\]


\noindent
{\bf{Theorem.}}
\emph{For each $n\geq 1$ one has the inequality}
\[ 
\rho_f(n)\geq 
|\widehat f(n+1)|-\sum_{j=1}^\infty |\widehat f((n+1)(2j+1))|
\]
\medskip

\noindent
We remark that this lower bound
of the $\rho$-numbers are of special interest when
the Fourier coefficients of $f$ have many gaps.


\bigskip


\centerline{\bf{Carleson's theorem.}}

\bigskip

\noindent
The theory about Fourier series containes
quite adavanced results.
Among these is Carelaon's theorem sabout the almost everywhere
convergence of Fourier's partial sums.
Let $f(\theta)$
be a complex-valued 
and continuous function defined on the interval
$\{0\leq \theta\leq 2\pi\}$ which satisfies
$f(0)=f(2\pi)$. For every integer $n$ we set
\[ 
\widehat f(n)=\frac{1}{2\pi}\cdot \int_0^{2\pi}\
e^{-in\phi}f(\phi)\cdot d\phi
\]
One refers to $\{\widehat f(n)\}$ as the Fourier coefficients of $f$ and 
Fourier's partial sum function of degree $N$ is defined by
\[
S_N(\theta)=\sum_{n=-N}^{n=N}\, \hat f(n)\cdot e^{in\theta}\tag{0.0}
\]
The question arises if
\[
\lim_{N\to\infty}\, \max_\theta\, |S_N(\theta)-f(\theta)|=0\tag{0.1}
\]
So (0.1) means that Fourier's partial sums convege uniformyly to $f$.
Examples where (0.1) fails 
were discovered at an early stage
and lead to  Gibb's phenomenon. More precisely, there exists contionuous functions
$f$ where the uniform not only fails, but
for certain $\theta$-values the sequence
$\{S_N(\theta)\}$  even fails to converge.
A relaxed condition of (0.1) is to ask if the pointwise limit
\[
\lim_{N\to\infty}\, S_NN(\theta)=f(\theta)\tag{0.2}
\] 
exists  for all $\theta$ outside a null set in the sense of Lebesgue, i.e.
is it true that Fourier's  partial sums converge almost everywhere to
$f$.
This question was open for more than a half century until
the affirmative answer 
was established  by Carleson in 1965.
This result constitutes
one of the greatest achievements ever in analysis, and the  proof goes beyond the
level of these notes.
The reader may consult Carleson's article [xxx]
which includes  a
remarkable inequality which goes as follows: Let
$\ell^2$ be the Hilbert space of sequences of complex numbers
$c_0,c_1,\ldots$ such that $\sum\,|c_n|^2<\infty$.
To every such a sequence we introduce trigonometric polynomials
\[ 
S_N(\theta)= \sum_{n=0}^{n=N}\, c_ne^{in\theta}
\]
Define the maximal function by
\[ 
S^*(\theta)= \max_{N\geq 1}\, |S_N(e^{i\theta})|
\]

\noindent
{\bf{Carleson's inequality.}} \emph{There exists a constant $C$ such that
the following hold when $\{c_n\}\in\ell^2$:}
\[
\int_0^{2\pi}\, S^*(\theta)^2\, d\theta\leq C\cdot\sum_{n=1}^\infty\,|c_n|^2
\] 
\bigskip




\centerline{\bf{Bernstein's example.}}
\medskip


\noindent
A remarkable construction was given by S. Bernstein 
in the article [Comptes
Rendus 1914].
Let $p$ be a prime number of the form $4\mu+1$ where $\mu$ is a positive
integer.
For each integer $n\geq 1$ we have the Legendre symbol
$L(n;p)$ which is +1 is $ k$ has a quadratic remainder modolu $p$ and
otherwise $L(n;p)=-1$.
Define the trigonometric polynomial
\[
\mathcal B_p(\theta)=\frac{2}{p^{\frac{3}{2}}}\cdot 
\sum_{n=1}^{n=p-1}\, (p-n)\cdot L(n;p)\cdot
\cos n\theta
\]
Then Bernstein  proved that
\[
\max_\theta\, |\mathcal B_p(\theta)|\leq 1\tag{i}
\]
At the same time we notice that
\[
\frac{2}{p^{\frac{3}{2}}}\cdot\sum_{n=1}^{p-1}\, |(p-n)\cdot L(n;p)|=\frac{p-1}{\sqrt{p}}\simeq \sqrt{p}\tag{ii}
\]
Bernstein's trigonometric polynomials have extremal properties.
For 
consider an arbitrary cosine series
\[
u(\theta)=\sum_{n=1}^{n=N}\, a_n\cdot \cos\theta
\]
Now the $L^2$-integral
\[
\frac{1}{\pi} \int_0^{2\pi}\, u^2(\theta)\, 
d\theta=\sum_{n=1}^{n=N} \, a^2_n
\]
If the maximum norm of $u$ is one 
the $L^2$ integral is majorized by 2 and the Cauchy-Schwarz inequality gives
\[
\sum\,|a_n|\leq \sqrt{2\cdot N}
\]
Bernstein's example shows that this inequality is essentially sharp.
Another notable phenomenon in
Bernstein's example
 is the following: We have
\[
\int_0^{2\pi}\mathcal B^2_p(\theta)\,d\theta=
 \frac{4}{p^3}\cdot \pi\cdot \sum_{n=1}^{n=p-1}\, n^2 
 \]
The right hand side is bounded by an absolute constant $C$.
Hence 
the maximum norm and the $L^2$-norm of $\mathcal B_p$
are comparable , i.e.
there is a fixed constant $0<c<1$ such that
\[
\frac{c}{\pi}\leq \frac{||\mathcal B_p||_\infty}{||\mathcal B_p||_2}
\leq\frac{1}{\pi c}
\]

\medskip

\noindent
{\bf{Remark.}}
Bernstein's construction was based upon arithmetic.
Later Salem proved that
the Bernstein's example
is generic in the sense that
by random choice of signs in a given sequence
$\{a_k\}$ with prescribed $L^2$-norm  equal to one,  the corresponding maximum
norms of the partial sums are
not so large
with "high probabilities".
To give an example: Let $N\geq 2$ and consider the family 
$\mathcal F_N$ of cosine series
\[
f(\theta)=\frac{1}{\sqrt{N}}\cdot
\sum_{n=1}^{n=N}\, 
\epsilon_n\cdot \cos n\theta
\]
Here$\{\epsilon_n\}$ is random sequence where each
$\epsilon_n$ is +1 or -1.
Notice that the $L^2$-integrals
\[
\int_0^\pi\, f(\theta)^2\, d\theta=\frac{\pi}{2}
\]
Above one has a sample space 
where each choice of a siugn-sequence
$\{\epsilon_n\}$ produces a function in
$\mathcal F_N$. To be precise, we
get $2^N$ many functions in this family.
The evaluation at $\theta=0$ corerssponds to
a Bernoulli trial, i.e,. tossing a coin $N$ times
and measure the difference of heads and tails, divided by
$\sqrt{N}$.
Here the centeral limit theorem applies, i.e by de Moivre's
discovery from 1733, the
random  outcome of
the numbers
$\{f(0)\,\colon F\in \mathcal F\}$
is expressed by a discrete random variable whose densities
converge to the normal distribution as $N\to\infty$.
\medskip

\noindent
A more involved study arises when one regards
vakues of the $f$-functions over the whole
interval $[0,\pi]$. Fir example, one can consider the
random variabke on the samp�le space
above defined by
\[
f\mapsto \max_{0\leq\theta\leq\pi}\, |f(\theta)|
\]
Results about the asymptotic behaviour  of
the distributions of these random variables 
as $N$ increases  have been obtained by Salem
and inspired much later work. The reader may consult [Salem] and 
[Kahane] for an account about Fourier series with
random coefficients.
It goes without saying that this leads to
a quite involved theory.






































\newpage

\centerline
{\bf{A: The kernels of Dini, Fejer and Jackson}} 
\bigskip

\noindent
Denote by $C_\text{per}^0[0,2\pi]$
the family of complex-valued continuous
functions
$f(\theta)$
on 
$[0,2\pi]$ which satisfy
$f(0)=f(2\pi)$. 
The Fourier coefficients of such a function $f$
are defined by:
\[ 
\widehat f(n)=\frac{1}{2\pi}\cdot \int_0^{2\pi}\
e^{-in\phi}f(\phi)\cdot d\phi
\]
where $n$ are integers.
Fourier's partial sum  of degree $N$ is defined by
\[
S^f_N(\theta)=\sum_{n=-N}^{n=N}\, \hat f(n)\cdot e^{in\theta}\tag{A.0}
\]

\noindent
{\bf{Exercise.}} 
Show that if 
$f$ is real-valued and $N$ is a positive integer, then 
the Fourier series  takes the form
\[
S_N^f(\theta)= \frac{a\uuu 0}{2}+
\sum\uuu {k=1}^N \,a_k\cdot \cos k\theta+
\sum\uuu {k=1}^N\,b_k\cdot \sin k\theta
\]
where  $a\uuu 0=\frac{1}{\pi}\int\uuu 0^{2\pi}\, f(x)\cdot dx$
and when $k\geq 1$:
\[
a\uuu k=\frac{1}{\pi}\int\uuu 0^{2\pi}\, f(x)\cdot \cos kx\cdot dx
\quad\colon\quad
b\uuu k=\frac{1}{\pi}\int\uuu 0^{2\pi}\, f(x)\cdot \sin kx\cdot dx
\] 

\bigskip



\noindent
{\bf A.1.The Dini kernel.}
If $N\geq 0$ we set
\[ 
D_N(\theta)=
\frac{1}{2\pi}\sum_{n=-N}^{n=N}\, e^{in\theta}
\]

\medskip


\noindent
{\bf A.2 Proposition.} \emph{One has the formula}
\[
D_N(\theta)=\frac{1}{2\pi}\cdot \frac{\text{sin}((N+\frac{1}{2})\theta)}{\text{sin}\,\frac{\theta}{2}}
\tag{*}
\]
\medskip
\noindent
\emph{Proof.}
We have
\[
\sum_{n=-N}^{n=N}\, e^{in\theta}
=e^{-iN\theta}
\cdot \sum_{n=0}^{n=2N}\, e^{in\theta}
=e^{-iN\theta}\cdot
\frac{
e^{i(2N+1)\theta}-1}{e^{i\theta}-1}=
\]
\[
e^{-iN\theta-i\theta/2}\cdot\frac{
e^{i(2N+1)\theta}-1}{2i\cdot \sin \theta/2}=
\frac{2i\cdot\text{sin}((N+1/2)\theta)}{2i\cdot \sin \theta/2}
\]
and (*) follows after division with $2i$.
\medskip


\noindent
{\bf A.3 Exercise.}
Show that the folloeing hold for each $N\geq 0$:
\[ 
S^f_N(\theta)=\int_0^{2\pi}\, D_N(\theta-\phi)\cdot f(\phi)\cdot d\phi=
\int_0^{2\pi}\, D_N(\phi)\cdot f(\theta+\phi)\cdot d\phi
\]
\bigskip




\noindent
{\bf A.4 The Fejer kernel.}
For each $N\geq0$ we set
\[
\mathcal F_N(\theta)=\frac{D_0(\theta)+\ldots+D_N(\theta)}{2\pi(N+1)}
\]

\noindent
{\bf A.5 Proposition} \emph{One has the formula} 
\[
 \mathcal F_N(\theta)=\frac{1}{2\pi(N+1)}\cdot
 \frac{1-\text{cos}((N+1)\theta)}{2\cdot \text{sin}^2(\frac{\theta}{2})}\tag{**}
 \]

 
 \noindent
 \emph{Proof.}
To each $\nu\geq 0$ we have
$\text{sin}((\nu+1/2)\theta)=
\mathfrak{Im} \bigl [e^{i(\nu+1/2)\theta)}\bigr]$.
Hence
$F_N(\theta)$ is the imaginary part of
\[
\frac{1}{2\pi(N+1)}\cdot \frac{e^{i\theta/2}}{\text{sin}(\theta/2)}\cdot
\sum_{\nu=0}^{\nu=N}\, e^{i\nu\theta}
\]
Next, we have
\[
e^{i\theta/2}\cdot\sum_{\nu=0}^{\nu=N}\, e^{i\nu\theta}=
e^{i\theta/2}\cdot\frac{e^{i(N+1)\theta}-1}{e^{i\theta-1}}
=
\frac{e^{i(N+1)\theta}-1}{2i\cdot \text{sin}(\theta/2)}
\]
Since $i^2=-1$ we see that the imaginary part of the last
term is equal to
\[
 \frac{1-\text{cos}((N+1)\theta)}{2\cdot \text{sin}(\frac{\theta}{2})}
\]
and then (**) follows.


 \bigskip
 
 \noindent
 {\bf{A.6 Fejer sums.}}
 For each $f$ and every $N\geq 0$ we set
\[
F^f_N(\theta)=\int_0^{2\pi}\,\mathcal F_N(\phi)\cdot f(\theta+\phi)\cdot d\phi
\]

 \noindent
{\bf A.7 An inequality.}
If $a>0$ and $a\leq\theta\leq 2\pi-a$ we have  the inequality
\[
\text{sin}^2(\theta/2)\geq \text{sin}^2(a/2)\tag{i}
\]
Let $f$ be given and
denote by $M(f)$
the maximum norm of
$|f(\theta)|$ over $[0,2\pi]$. Then (i) gives
\[
 \int_a^{2\pi-a}\, \mathcal F_N(\phi)\cdot f(\theta+\phi)\cdot d\phi
\leq
\] 
\[
\frac{M}{2\pi(N+1)\cdot \text{sin}^2(a/2)}
 \int_a^{2\pi-a}\, (1-\text{cos}(N\phi))\cdot d\phi
\leq\frac{2M}{(N+1)\cdot \text{sin}^2(a/2)}\tag{A.7.1}
\]
\medskip

\noindent
{\bf A.8 Exercise.}
Given some $\theta_0$ and $0<a<\pi$ we set
\[
\omega_f(a)=\max_{|\theta-\theta_0|\leq a}\, |f(\theta)-f(\theta_0)|
\]
Use (A.7.1) to prove that
\[
|F^f_N(\theta_0)-f(\theta_0)|
\leq\frac{2M}{(N+1)\cdot \text{sin}^2(a/2)}+\omega_f(a)
\]
Conclude that  the \emph{uniform continuity}
of the function $f$ on $[0,2\pi]$ implies that
the sequence $\{F^f_N\}$ converges uniformly to $f$ over
the interval $[0,2\pi]$.
\medskip


\noindent
{\bf{A.9 Exercise.}}
Use (A.7-8) to show that  there exists an absolute constant $C$ such that
\[
||f\vvv \mathcal F\uuu n(f)||\leq C\cdot \omega\uuu f(\frac{1}{n})
\cdot \bigl(1+\log^+\,\frac{1}
{\omega\uuu f(\frac{1}{n})}\bigr)\tag{A.9.1}
\]
hold for all continuous $2\pi$\vvv periodic functions $f$.
\medskip


\bigskip


\centerline{\bf{A. 10 The Jackson kernel}}.
\medskip

\noindent
One may
ask if (A.9.1) can be improved in the sense that
there exists a constant $C$ which is independent of both $f$ and of $n$
such that
\[
\max_\theta\, ||f(\theta)-F^f\uuu n(\theta)|\leq C\cdot \omega\uuu f(\frac{1}{n})\tag{*}
\]


\noindent 
Examples show that no such uniform bound expressed
by a constant $C$ exists.
To obtain an inequality such as (*),
D. Jackson introduced a new kernel
in his
thesis \emph{�ber die Genauigkeit der Ann�hrerung stegiger funktionen
durch ganze rationala funktionen} from G�ttingen in 1911.
To each $2\pi$\vvv periodic and continuous function
$f(x)$ on the real line and every $n\geq 1$ we set
\[
\mathcal J^f\uuu n(x)= \frac{3}{2\pi}\cdot \int\uuu{\vvv \infty}^\infty
\, f(x+\frac{2t}{n})\cdot \bigl(\frac{\sin t}{ t}\bigr )^4\cdot dt
\]
\medskip

\noindent
{\bf{A.11 Theorem.}} \emph{The function
$\mathcal J^f\uuu n(x)$ is a trigonometric polynomial of
degree $2n\vvv 1$ at most and  one has the inequality}
\[
\max\uuu x\, |f(x)\vvv\mathcal J^f\uuu n(x)|\leq 
(1+\frac{6}{\pi})\cdot \omega\uuu f(\frac{1}{n})
\]


\noindent
\emph{Proof.}
The variable  substitution $t\to nt$ gives
\[ 
\mathcal J^f\uuu n(x)=
\frac{3}{2\pi n^3}\cdot \int\uuu{\vvv \infty}^\infty
\, f(x+2t)\cdot \bigl(\frac{\sin nt}{ t}\bigr )^4\cdot dt\tag{1}
\]
Since 
$t\mapsto f(x+2t)\cdot sin^4\, nt$
is  $\pi$\vvv periodic it follows that (1) is equal to
\[
\frac{3}{2\pi n^3}\cdot 
\int\uuu 0^\pi\, f(x+2t)\cdot \sum\uuu {k=\vvv \infty}^\infty
\,\frac{\sin^4 (nt)}{(k\pi +t)^4}\cdot dt\tag{2}
\]
Next,  recall from � XX that
\[
\frac{1}{\sin^2 z}=
 \sum\uuu {k=\vvv \infty}^\infty
\frac{1}{(z+k\pi)^2}
\] 
Taking a second derivative when  $z=t$ is real it follows that
\[
\partial\uuu t^2(\frac{1}{\sin^2 t})=
\frac{1}{6}\cdot \sum\uuu {k=\vvv \infty}^\infty
\frac{1}{(t+k\pi)^4}\tag{3}
\]
Hence we obtain
\[
\mathcal J^f\uuu n(x)=\frac{1}{4\pi n^3}\cdot
\int\uuu 0^\pi\, f(x+2t)\cdot
\sin^4 (nt)
\cdot\partial\uuu t^2(\frac{1}{\sin^2 t})\,dt\tag{*}
\]
\medskip

\noindent
Next, the function
\[
\sin^4 (nz)
\cdot\partial\uuu z^2(\frac{1}{\sin^2 z})
\] 
is entire and even and the reader may verify that it is
 a finite sum of  entire cosine\vvv functions which implies
that the Jackson kernel is expressed by
a finite sum of integrals:
\[
\mathcal J\uuu f^n(x)=\sum\uuu{k=0}^{2n\vvv 1}\, c\uuu k\int\uuu 0^{2\pi}\,
f(u)\cdot \cos\,k(x\vvv  u))\, du\tag{4}
\]
In particular 
$\mathcal J\uuu f^n(x)$ is a trigonometric polynomial of degree
$2n\vvv 1$ a most.
Integration by parts give the equality
\[
\int\uuu {\vvv\infty}^\infty\, 
(\frac{\sin nt}{ t}\bigr )^4\,dt=
\frac{1}{6}\int\uuu 0^\pi\, 
\sin^4 t\cdot \partial\uuu t^2(\frac{1}{\sin^2 t})\, dt=
\frac{4}{3}\int\uuu 0^\pi\, \cos^2 t\, dt=
\frac{2\pi}{3}\tag{5}
\]
Next, we leave it to the reader to verify
the inequality
\[
\frac{3}{2\pi}\int\uuu{\vvv \infty}^\infty\,
(1+2|t|)\cdot \bigl(\frac{\sin t}{t}\bigr )^4\cdot dt\leq 1+\frac{6}{\pi}\tag{6}
\]
\medskip




\noindent
From the above where we 
use (1) and (*) 
it follows that
\[
\mathcal J\uuu n^f(x)\vvv f(x)=
\frac{3}{2\pi}\cdot \int\uuu{\vvv \infty}^\infty
\, [f(x+\frac{2t}{n})\vvv f(x)]\cdot \bigl(\frac{\sin t}{t}\bigr )^4\cdot dt\tag{7}
\]
Now
\[
|f(x+\frac{2t}{n})\vvv f(x)|\leq \omega\uuu f(\frac{2t}{n})\leq
(2|t|+1)\cdot \omega\uuu f(\frac{1}{n})
\] 
where the last equality follows from Lemma XX.
Hence (7)
gives
\[
\max\uuu x\, 
|\mathcal J\uuu n^f(x)\vvv f(x)|\leq \omega\uuu f(\frac{1}{n})\cdot
 \frac{3}{2\pi}\cdot \int\uuu{\vvv \infty}^\infty
(2|t|+1)\cdot \bigl(\frac{\sin t}{t}\bigr )^4\cdot dt
\]
Finally, by (6) 
the last factor is majorized by $1+\frac{6}{\pi}$
and Jackson's inequality follows.



\bigskip


delaydelay 

\centerline {\bf{A.12 A lower bound for polynomial approximation.}}
\bigskip

\noindent
Denote by $\mathcal T\uuu n$ the linear space of trigonometric polynomials of
degree $\leq n$.
For a  $2\pi$-periodic and continuous function $f$ we put
\[ 
\rho\uuu f(n)=
\min\uuu{T\in\mathcal T\uuu n}\, ||f\vvv T||
\] 
where $||\cdot ||$
denotes the maximum norm over $[0,2\pi]$.
We shall establish a lower bound
for the $\rho$\vvv numbers when certain sign\vvv conditions hold for
Fourier coefficients.
In general, let $f$ be a periodic function and for each positive integer
$n$ we find $T\in \mathcal T\uuu n$ such that 
$||f\vvv T||= \rho\uuu f(n)$.
Since Fejer kernels do not  increase maximum norms one has
\[
||F^f\uuu k\vvv F^T\uuu k||\leq\rho_f(n)\tag{i}
\]
for every positive integer $k$. 
Apply this with $k=n$ and $k=n+p$
where $p$ is another  positive integer.
If $T\in\mathcal T_n$
the equation from Exercise XX gives
\[
T=\frac{(n+p)\cdot \mathcal F\uuu {n+p}(T)\vvv n\cdot \mathcal F\uuu n(T)}{p}\tag{ii}
\]
Since (i) hold for $n$, $n+p$ and 
$||f\vvv T||\leq \rho_f(n)$, the triangle inequality gives
\[
||f\vvv \frac{(n+p)\cdot \mathcal F\uuu {n+p}(f)\vvv n\cdot \mathcal F\uuu n(f)}{p}||
\leq 2\cdot \frac{n+p}{p}\cdot\rho\uuu f(n)\tag{iii}
\]
Next, by the formula (� xx) it follows that (iii) gives


\[
||f\vvv \frac{S\uuu n(f)+\cdots S\uuu {n+p\vvv 1}(f)}{p}||
\leq 2\cdot \frac{n+p}{p}\cdot \rho\uuu f(n)
\]
In particular we take $p=n$ and get the inequality
\[
||f\vvv \frac{S\uuu n(f)+\cdots S\uuu {2n\vvv 1}(f)}{n}||
\leq \frac{4}{n}\cdot \rho\uuu f(n)\tag{*}
\]


\noindent
{\bf{A.12 A special case.}}
Assume that $f(x)$ is an even function on
$[\vvv \pi,\pi]$ which gives a Fourier series:
\[
f(x)=\frac{a\uuu 0}{2}+ \sum\uuu {k=1}^\infty\, a\uuu k\cdot
\cos\,kx
\]
\medskip

\noindent
{\bf{A.12 Proposition}} \emph{Let $f$ be even as above and assume that
$a\uuu k\leq 0$ for every $k\geq 1$. Then the following inequality holds for
every $n\geq 1$:}
\[
f(0)\vvv \frac{S\uuu n(f)(0)+\cdots S\uuu {2n\vvv 1}(f)(0)}{n}
\leq \vvv \sum\uuu {k=2n}^\infty\, a\uuu k
\]



\noindent
The easy verification is left to the reader.
Taking the maximum norm over $[\vvv \pi,\pi]$
it follows from (*) that 
\[
\rho\uuu f(n)\geq
\frac{n}{4}\cdot \sum\uuu {k=2n}^\infty\, |a\uuu k|\tag{**}
\]
holds when  the sign conditions on
the Fourier coefficients above are satisfied.
Notice that (**) means that
one has a lower bound for polynomial approximations
of $f$.

\medskip

\noindent
{\bf{A.13 The function $f(x)=\sin |x|$}}
It is obvious that
\[
\omega\uuu f(\frac{1}{n})=\frac{1}{n}
\]

\noindent
Next, the periodic function $f(x($ is even
and hence we only get a cosine\vvv series. For each positive integer $m$
we have:
\[
a\uuu k=\frac{2}{\pi}\int\uuu 0^\pi\, \sin x\cdot \cos kx \cdot dx
\]
To evaluate these integrals we use the trigonometric formula
\[
sin \,(k+1)x\vvv \sin(k\vvv 1)x =2\sin x\cdot \cos kx
\]
Now the reader can verify that $a\uuu\nu=0$ when $\nu$ is odd while
\[
a\uuu{2k}=\vvv \frac{4}{\pi}\cdot \frac{1}{2k^2\vvv 1}
\]
Hence the requested
sign conditions hold
and
(**) entails that

\[
\rho\uuu f(n)\geq \frac{n}{\pi}\cdot  \sum\uuu {k=n}^\infty
\frac{1}{2k^2\vvv 1}
\]
Here the right hand side is $\geq \frac{C}{n}$ for a
constant $C$ which is independent of $n$.
So this  example shows that the inequality (*) in � A.11
is sharp up to a multiple with a fixed constant.



\newpage



















\centerline{\bf{B. Legendre polynomials.}}
\bigskip


\noindent
If $n\geq 1$ we denote by $\mathcal P_n$ the linear space of real-valued
polynomials of degree $\leq n$. An inner product is defined by
is defined by
\[
\langle q,p\rangle=
\int_{-1}^1\, q(x)p(x)\cdot dx
\]
Since 
$1,x,\ldots,x^{n-1}$ generate a subspace of  co-dimension  one
in $\mathcal P\uuu n$ we get:

\medskip

\noindent
{\bf {B.1 Proposition.}}
\emph{There exists a unique 
$Q_n(x)= x^n+q_{n-1}x^{n-1}+\ldots+q_0$
such that}
\[
\int_{-1}^1\, x^\nu\cdot Q_n(x)\cdot dx=0\leq  \nu\leq n-1
\]
\medskip

\noindent
To find $Q_n(x)$.
we consider the polynomial $(1-x^2)^n$ which vanishes up to order
$n$ at the  end-points 1 and -1. 
Its the derivative of order $n$ gives a polynomial of  degree $n$
and partial integrations show that
\[
\int_{-1}^1\, x^\nu\cdot \partial^n((x^2-1)^n))\cdot dx=0\leq  \nu\leq n-1
\]
The leading coefficient of $x^n$ 
in $\partial^n((x^2-1)^n))$
becomes
\[
c_n=2n(2n-1)\cdots (n+1)
\]
Hence we have
\[ Q_n(x)=\frac{1}{c_n}\cdot \partial ^n((x^2-1)^n)
\]
\medskip

\noindent
{\bf B.2 Definition.}
\emph{The Legendre polynomial of degree
$n$ is  given by}
\[
P_n(x)=k_n\cdot\partial ^n((x^2-1)^n)
\]
\emph{where the constant $k\uuu n$ is determined so that
$P\uuu n(1)=1$.}

\medskip

\noindent
Since $P_n$ is equal to $Q_n$ up to a constant we still have
\[
\int_{-1}^1\, x^\nu\cdot P_n(x)\cdot dx=0\leq  \nu\leq n-1
\]
From this we conclude that 
\[
\int_{-1}^1\, P_n(x)\cdot P_mx)dx=0\quad  n\neq m
\]
Thus, $\{P_n\}$ is an orthogonal family
with respect to the inner product defined above.
\medskip

\noindent
{\bf B.3 A generating function.}
Let $w$ be a new variable and set
\[
\phi(x,w)=1-2xw+w^2
\]
Notice that $\phi\neq 0$ when
$-1\leq x\leq 1$ and $|w|<1$.
Keeping $-1\leq x\leq 1$ fixed
we have the function
\[ 
w\mapsto
\frac{1}{\sqrt{1-2xw+w^2}}
\]
Next, as   $|\zeta|<1$ one has the Newton series
\[
\frac{1}{\sqrt{1-\zeta}}=\sum\, g_n\cdot \zeta^n\quad
\text{where}\quad g_n=\frac{3\cdot5\cdots(2n-1)}{2^n}
\]
It follows that
\[
\frac{1}{\sqrt{1-2xw+w^2}}=\sum\, g_n(2xw-w^2)^2
\]
With $x$ kept fixed the series is expanded into $w$-powers, i.e.  set
\[
\frac{1}{\sqrt{1-2xw+w^2}}=\sum\, \rho_n(x)\cdot w^n
\]
It is easily seen that as $x$ varies then
$\rho_n(x)$ is a polynomial of degree
$n$. Moreover, we notice that the coefficient of
$x^n$ in $\rho_n(x)$ is equal to
\[
g_n\cdot 2^n
\]
Next, if $x=1$ we have
\[
\frac{1}{\sqrt{1-2w+w^2}}= \frac{1}{1-w}= \sum\, w^n
\]
From this we conclude that
\[ 
\rho_n(1)=1\quad\text{for all}\quad n\geq 0
\]
\medskip

\noindent
{\bf B.4 Theorem.} \emph{One has the equality $\rho_n(x)=P_n(x)$ for each $n$, i.e.} 
\[
\frac{1}{\sqrt{1-2xw+w^2}}=\sum\, P_n(x)\cdot w^n
\quad 
\text{holds when}\quad  \vvv1\leq x\leq 1\,\colon\, |w|<1
\]



\bigskip

\noindent
{\bf{B.5 Exercise.}} Prove this result.
\bigskip

\noindent {{\bf{B.6 The series for $P_n(\text{cos}\,\theta)$}}.
With $x$ replaed by $\cos\,\theta$ we notice that
\[
1-2\cos\,\theta\cdot w+w^2=
(1-e^{i\theta}w)(1-e^{-i\theta}w)
\]
It follows that
\[
\frac{1}{\sqrt{1-2\text{cos}(\theta)w+w^2}}=
\frac{1}{\sqrt{1-1-e^{i\theta}w)}}\cdot 
\frac{1}{\sqrt{1-e^{-i\theta}w)}}
\]
The last product becomes
\[ \sum\sum\, g_m e^{im\theta}w^m\cdot g_\nu e^{-i\nu\theta}w^\nu
\]
Collecting $w$ powers the double sum becomes
\[
\sum\, \gamma_n(\theta)\cdot w^n
\quad\gamma_n(\theta)=\sum_{m+\nu=n}\, g_mg_\nu e^{i(m-\nu)\theta}
\]

\noindent
By Theorem B.4 the last sum represents $P_n(\text{cos}(\theta))$.
One has for example
\[ 
P_3(\text{cos}(\theta)=
2g_3\cdot\text{cos}(3\theta)+
2g_2g_1\cdot \text{cos}(\theta)
\]
where we used that $g_0=1$.
\medskip

\noindent
{\bf{B.7 An inequality  for $|P(x)|$.}}
Since the $g$-numbers are  $\geq 0$
we obtain
\[
|P_n(\text{cos}(\theta)|\leq
g_ng_0+
g_{n-1}g_1+\ldots+
g_1g_{n-1}+
g_0g_n=P_n(1)\quad\colon\,\,0\leq\theta\leq 2\pi
\]

\noindent
Hence we have proved
\medskip

\noindent
{\bf B.8 Theorem.}\emph{ For each $n$ one has}
\[ 
|P_n(x)|\leq 1\quad\colon\,\, -1\leq x\leq 1
\]


\noindent
Next, we study the values when $x>1$. Here one has
\medskip

\noindent
{\bf B.9 Theorem.} \emph{For each $x>1$ one has}
\[
1<P_1(x)<P_2(x)<\ldots
\]


\noindent
\emph{Proof.}
Let us put
\[
\psi(x.w)= 1+ \sum_{n=1}^\infty\, [P_n(x)-P_{n-1}(x)]\cdot w^n
\]
By Theorem B.4 this is equal to
\[
\frac{1-w}{\sqrt{1-2xw+w^2}}\tag{*}
\]
With $x>1$ we set $x=1+\xi$ and notice that
\[
1-2xw+w^2=(1-w)^2-2\xi w
\]
Hence (*) becomes
\[
\frac{1}{\sqrt{1-\frac{2\xi w}{1-w^2}}}=\sum\, g_n\cdot \frac
{(2\xi w)^n} {(1-w^2)^n}=\sum\, g_n\cdot (2\xi)^n\cdot
\frac{w^n}{(1-w^2)^n}\tag{**}
\]
Next, for each $n\geq 1$ we notice that the power series
of 
$\frac{w^n}{(1-w^2)^n}$ has positive coefficients. Since
$g_n(2\xi)^n>0$ also hold we conclude that
(**) is of the form
\[
1+ \sum_{n=1}^\infty\, q_n(\xi)\cdot w^n\quad\text{where}\quad q_n(\xi)>0
\]
Finally, Theorem B.9 follows since
\[ 
P_n(1+\xi)-P_{n-1}(1+\xi)=q_n(\xi)
\]

\bigskip

\centerline {\bf {B.10 An $L^2$-inequality.}}
\medskip

\noindent
Let $n\geq 1$ and denote by $\mathcal P_n[1]$ the space of
real-valued polynomials $Q(x)$ of degree $\leq n$ for which
$\int_{-1}^1\, Q(x)^2\cdot dx=1$ and set
\[ 
\rho(n)=\max_{Q\in\mathcal P-n[1]}\,
|Q|_\infty
\] 
where $|Q|_\infty$ is the maximum norm over
$[-1,1]$.
To find  $\rho(n)$ we use the orthonormal basis $\{P_k^*\}$
and write
\[
Q(x)=t_0\cdot P_0^*(x)+\ldots+t_n\cdot P_n^*(x)
\]
Since $Q\in\mathcal P_n[1]$ we have
$t_0^2+\ldots+t_n^2=1$. Recall also that
\[
P_\nu^*(x)=\sqrt{\frac{2\nu+1}{2}}\cdot P_\nu(x)
\]
Given $-1\leq x_0\leq 1$ the Cauchy-Schwarz inequality gives
\[
Q(x_0)^2\leq \sum_{\nu=0}^{\nu=n}\,
\frac{2\nu+1}{2}\cdot |P_\nu(x_0)|\leq 
\sum_{\nu=0}^{\nu=n}\,\frac{2\nu+1}{2}
\]
where the last inequality follows since the maximum norm of
each $P_\nu$ is $\leq 1$.
Finally, we notice that
\[
\sum_{\nu=0}^{\nu=n}\,\frac{2\nu+1}{2}
=\frac{(1-n)^2}{2}
\]
We conclude that
\[
|Q(x_0)|\leq \frac{n+1}{\sqrt{2}}
\]
\medskip

\noindent
{\bf B.11 The case of equality.}
To have equality above we take  $x_0=1$ and 
\[
 t_\nu=\alpha\cdot P^*_\nu(1)\quad\colon\quad \nu\geq 0
\]


\newpage

\centerline{\bf {C. The space $\mathcal T_n$}}
\bigskip

\noindent
Let $n\geq 1$ be a positive integer.
A real-valued trigonometric polynomial of degree
$\leq n$ is given by
\[
g(\theta)=a_0+
a_1\text{cos}\,\theta+\ldots
+
a_n\text{cos}\,n\theta+
b_1\text{sin}\,\theta+\ldots
b_n\text{sin}\,n\theta
\]
Here $a_0,\ldots,a_n,b_1,\ldots,b_n$ are real numbers.
The space of such functions is denoted by
$\mathcal T_n$ which  is a vector space over
${\bf{R}}$ of dimension $2n+1$.
We can write
\[
\text{cos}\,kx=\frac{1}{2}[e^{ikx}+e^{-ikx}]\quad\text{and}\quad
\text{sin}\,kx=\frac{1}{2i}[e^{ikx}-e^{-ikx}]\quad\colon\, k\geq 1
\]
It follows that there exist complex numbers
$c_0,\ldots,c_{2n}$ such that
\[
g(\theta)=e^{-in\theta}\cdot[c_0+c_1e^{i\theta}+\ldots+c_{2n}e^{i2n\theta}]
\]


\noindent
{\bf Exercise.}
Show that
\[
c_\nu+c_{2n-\nu}=2a_\nu\quad \text{and}\quad c_{\nu}-c_{2n-\nu}=2 b_\nu\implies
\]
\[
c_{2n-\nu}=\bar c_\nu\quad 0\leq\nu\leq n
\]


\noindent
{\bf C.1 The polynomial $G(z)$.}
Given $g(\theta)$ as above we set
\[ 
G(z)= c_0+c_1z+\ldots+c_{2n}z^{2n}\implies
e^{-in\theta}\cdot G(e^{i\theta})= g(\theta)
\]


\noindent
{\bf C.2 Exercise.}  Set
\[ 
\bar G(z)= \bar c_0+c\uuu 1z+\ldots+\bar c_{2n}z^{2n}
\]
and show that
\[ z^{2n}G(1/z)= \bar G(z)\tag{*}
\]
Use this to show that if
$0\neq z_0$ is a zero of $G(z)$ then
$\frac{1}{\bar z_0}$ is also a zero of $G(z)$.
\medskip

\noindent
{\bf{C.3 The case when $g\geq 0$}}. Assume that the $g$-function is non-negative.
Let
\[ 
0\leq \theta_1<\ldots<\theta_\mu<2\pi
\]
be the zeros on the half-open interval $[0,2\pi)$.
Since $g\geq 0$ every such zero has a multiplicity given by
an \emph{even} integer.
Consider also the polynomial $G(z)$. Exercise C.2    shows that
$\{e^{i\theta_\nu}\}$ are complex zeros of $G(z)$
whose multiplicities are even integers.
Next, if $\zeta$ is a zero where
$\zeta\neq 0$ and $|\zeta|\neq 1$, then (*) in C.2
implies that
$\frac{1}{\bar\zeta}$ also is  a zero and hence
$G(z)$ has a factorisation
\[
G(z)= c_{2n}
\cdot \prod_{\nu=1}^{\nu=\mu}\,
(z-e^{i\theta_\nu})^{2k_\nu}\cdot \prod_{j=1}^{j=m}\,
(z-\zeta_j)(z-\frac{1}{\bar\zeta_j})\cdot z^{2r}
\quad\text{where}\quad 2\mu+2m+2r=2n
\]
Here $0<|\zeta_j|<1$ hold for each $j$ and it may occur that
multiple zeros appear,  i.e. the
$\zeta$-roots need not be distinct and  the integer $r$ may be zero
or positive.
\medskip

\noindent
{\bf {C.4 The $h$-polynomial}}. Let
$\delta=\sqrt{|\zeta_1|\cdots|\zeta_m|}$ and put
\[
h(z)=c_{2n}\dot \delta\cdot \prod_{\nu=1}^{\nu=\mu}\,
(z-e^{i\theta_\nu})^{k_\nu}\cdot \prod_{j=1}^{j=m}\,
(z-\zeta_j)\cdot z^{r}
\]


\noindent
{\bf C.5 Proposition.} \emph{One has the equality}
\[
|h(e^{i\theta})|^2=g(\theta)
\]


\noindent
\emph{Proof.}
With $z=e^{i\theta}$ and $0<|\zeta|<1$ one has
\[
(e^{i\theta}-\zeta)
(e^{i\theta}-\frac{1}{\bar \zeta})=
(e^{i\theta}-\zeta)\cdot (\bar\zeta -e^{-i\theta})\cdot
e^{i\theta}\cdot\frac{1}{\bar \zeta}
\]
Passing to absolute values it follows that
\[
\bigl|(e^{i\theta}-\zeta)
(e^{i\theta}-\frac{1}{\bar \zeta})\bigr|=
\bigl|e^{i\theta}-\zeta\bigr|^2\cdot 
\frac{1}{|\zeta|}
\]
Apply this to every root $\zeta_\nu$ and take the 
product which
gives 
Proposition C.5.


\newpage

\noindent
{\bf C.6 Application.}
Let $g\geq 0$ be as above 
and assume that the constant coefficient $a_0=1$.
This means that
\[
1=\frac{1}{2\pi}\cdot 
\int_0^{2\pi}\, g(\theta)\cdot d\theta
\]
With
$h(z)=d_0+d_1z+\ldots+d_nz^n$
we get
\[
1 =\frac{1}{2\pi}\cdot 
\int_0^{2\pi}\,h(e^{i\theta})|^2\cdot d\theta
=|d_0|^2+\ldots+|d_n|^2
\]
Notice that
\[ |d_n|^2=|c_{2n}|\cdot\delta\quad\text{and}\quad
|d_0|^2= |c_{2n}\cdot \delta|\cdot\prod\,|\zeta_j|^2=|c_{2n}|\cdot \frac{1}{\delta}\tag{i}
\]
From this we see that
\[
|c_{2n}|\cdot(\delta+\frac{1}{\delta})= |d_0|^2+d_n|^2\leq 1\tag{iii}
\]
Here $0<\delta<1$ and therefore
$\delta+\frac{1}{\delta}\geq 2$ which together with
(iii) gives
\[
|c_{2n}|\leq \frac{1}{2}
\]
At the same time we recall  that
\[
c_{2n}=\frac{a_n+ib_n}{2}\implies |a_n+ib_n|\leq 1\tag{*}
\]


\noindent
\emph{Summing up} we have proved the following:
\medskip

\noindent
{\bf {C.7 Theorem.}}
\emph{Let $g(\theta)$
be non-negative in $\mathcal T_n$ with constant term
$a_0=1$. Then}
\[
|a_n+ib_n|\leq 1
\]


\noindent
{\bf {C.8 An application}}.
Let $n\geq 1$ and consider the space of all monic polynomials 
\[
P(x)=x^n+c_{n-1}x^{n-1}+\ldots+c_0
\]
where $\{c_\nu\}$ are real-
To each such polynomial we can consider the maximum norm
over the interval $[-1,1]$.
Then one has
\medskip

\noindent
{\bf {C.9 Theorem.}}
\emph{For each $P\in\mathcal P_n^*$ one has the inequality}
\[
\max_{-1\leq x\leq 1}\, |P(x)|\geq 2^{-n+1}
 \]


\noindent
\emph{Proof}. 
Consider some $P\in\mathcal P_n^*$ and
define the trigonometric polynomial
\[ 
g(\theta)= (\text{cos}\,\theta)^n
+c_{n-1}(\text{cos}\,\theta)^{n-1}+\ldots+c_0
\]
So here
$P(\text{cos}\,\theta)= g(\theta)$ and 
Theorem C.9 follows if we have proved that
\[
2^{-n+1}\geq 
\max_{0\leq \theta\leq 2\pi}\, |g(\theta))|\tag{1}
\]
To prove this we set
$M=\max_{0\leq \theta\leq 2\pi}\, |g(\theta))|$.
Next, we can write
\[ 
g(\theta)= a_0+a_1\text{cos}\,\theta\ldots+
a_n\text{cos}\,n\theta
\]
Moreover, since
\[
(\text{cos}\,\theta)^n=2^{-n}\cdot[e^{i\theta}+e^{-\theta}]^n
\]
we get
\[ 
a_n=2^n
\]
Now we shall apply Theorem C.8. For this purpose we construct
non-negative trigonometric polynomials. First we define
\[
g^*(\theta)= \frac{M-g(\theta)}{M-a_0}
\]
Then $g^*\geq 0$ and its constant term is 1.
We have also
\[ 
g^*(\theta)= 1-\frac{1}{M-a_0}\cdot \sum_{\nu=1}^{\nu=n}a_\nu\text{cos}\,\nu\theta
\]
Hence Theorem C.7 gives
\[
\frac{1}{|M-a_0|}\cdot |a_n|\leq 1\implies
|M-a_0|\geq 2^{-n+1}\tag{1}
\]
Next, we have also the function
\[
g_*(\theta)= \frac{M+g(\theta)}{M+a_0}
\]
In the same way as above we obtain:
\[ 
|M+a_0|\geq 2^{-n+1}\tag{2}
\]
Finally, (1) and (2) give
\[
M\geq 2^{-n+1}
\]
which proves Theorem C.9


\newpage

\centerline{\bf{D. Tchebysheff polynomials.}}
\medskip

\noindent
The inequality in Theorem C.9 is sharp. To see this
we shall construct a special polynomial $T_n(x)$ of degree $n$.
Namely, with $n\geq 1$ we can write
\[
\text{cos}\,n\theta=
2^{n-1}\cdot 
(\text{cos}\,\theta)^n+
c_{n-1}\cdot 
(\text{cos}\,\theta)^{n-1}+
\ldots+c_0
\]
Set
\[ T_n(x) =2^{n-1}x^n+
c_{n-1}\cdot x^{n-1}+\ldots+c_0
\]
Hence
\[ 
T_n(\text{cos}\,\theta)= \text{cos}\, n\theta
\]
We conclude that the polynomial
\[ 
p_n(x)=2^{-n+1}\cdot T_n(x)
\]
belongs to $\mathcal P_n^*$ and its maximum norm
is $2^{-n+1}$ which  proves that the inequality in Theorem C.9 is sharp.
\medskip

\noindent
{\bf D.1 Zeros of $T_n$}.
Set
\[ 
\theta_\nu=\frac{\nu\pi}{n}+\frac{\pi}{2n}
\]
It is clear that
$\theta_1,\dots,\theta_n$ are zeros of
$\text{cos}\, n\theta$.
Hence the  zeros of $T_n(x)$ are:
\[
x_\nu= \text{cos}\, \theta_\nu
\]
Notice that
\[
-1<x_n<\ldots<x_1<1
\]
Since $T_n(x)$ is a polynomial of degree
$n$ it follows that $\{x_\nu\}$ give all zeros and we have
\[
T_n(x)=2^{n-1}\cdot \prod\,(x-x_\nu)
\]


\noindent
{\bf{D.2 Exercise}}.
Show that
\[ 
T'_n(x_\nu)\cdot\sqrt{1-x_\nu^2}=n
\] 
hold for every zero of $T_n(x)$.

\medskip

\noindent
{\bf{D.3 An interpolation formula.}}
Since $x_1,\ldots,x_n$ are distinct it follows 
that if $p(x)\in\mathcal P_{n-1}$ is a polynomial of degree
$\leq n-1$ then
\[
 p(x)=
\sum_{\nu01}^{\nu=n}\, p(x_\nu)\cdot
\frac{1}{T'(x_\nu)}\cdot \frac{T(x)}{x-x_\nu}
\]
By the exercise above we get
\[
 p(x)=
\frac{1}{n}\cdot \sum_{\nu=1}^{\nu=n}\, (-1)^{\nu-1}p(x_\nu)\cdot
\sqrt{1-x_\nu^2}\cdot \frac{T(x)}{x-x_\nu}
\]
\medskip
\noindent
We shall use the interpolation formula above to prove
\medskip

\noindent
{\bf {D.4 Theorem}}
\emph{Let $p(x)\in\mathcal P_{n-1}$ satisfy}
\[
\max_{-1\leq x\leq 1}\,
\sqrt{1-x^2}\cdot |p(x)|\leq 1\tag{1}
\]
\emph{Then it follows that}
\[
\max_{-1\leq x\leq 1}\,
|p(x)|\leq n\tag{2}
\]
\medskip

\noindent
\emph{Proof.}
First, consider the case when 
\[
-\text{cos}\frac{\pi}{2n}\leq
x\leq \text{cos}\,\frac{\pi}{2n}\tag{*}
\]
Then we have
\[
\sqrt{1-x^2}\geq 
\sqrt{1-\text{cos}^2\frac{\pi}{2n}}=
\text{sin}\,\frac{\pi}{2n}
\]
Next,  recall the inequality
$\text{sin}\, x\geq \frac{2}{\pi}\cdot x$.
It follows that
\[
\sqrt{1-x^2}\geq \frac{1}{n}
\]
So when (1) holds in the theorem we have
\[ 
|p(x)|=\frac{1}{\sqrt{1-x^2}}\cdot
\sqrt{1-x^2}\cdot |p(x)|\leq
\frac{1}{\sqrt{1-x^2}}\leq n
\]
Hence the required inequality in Theorem D.4  holds when
$x$ satisfies (*) above.
Next, suppose that
\[
x_1\leq x\leq 1\tag{**}
\]
On this interval $T_n(x)\geq 0$ and from the interpolation formula
xx and the triangle  inequality we have
\[
|p(x)\leq\frac{1}{n}
\sum_{\nu=1}^{\nu=n}\,
\sqrt{1-x_\nu^2}\cdot |p(x_\nu)|\cdot
\frac{T(x)}{x-x_\nu}\leq\frac{1}{n}
\sum_{\nu=1}^{nu=n}\,
\frac{T(x)}{x-x_\nu}
\]
Next, the sum
\[
\frac{T(x)}{x-x_\nu}=T'_n(x)=n\cdot U_{n-1}(x)
\]
So when (**) holds we have
\[
|p(x)|\leq |U_{n-1}(x)|\tag{***}
\]
By xx the maximum norm of $U_{n-1}$ over $[-1,1]$ is $n$ and hence
(***) gives
\[
 |p(x)|\leq n
\]
In the same way one proves htat
\[
-1\leq x\leq x_n\implies 
 |p(x)|\leq n
\] 
Together with the upper bound in the case (xx) we get Theorem D.4.

\bigskip

\centerline{\bf{D.5 Bernstein's inequality.}}
\medskip

\noindent 
Let $g(\theta)\in\mathcal T_n$.
The derivative $g'(\theta)$ is another trigonometric polynomial and we have
\medskip

\noindent
{\bf Theorem.} \emph{For each $g\in\mathcal T_n$ one has}
\[
\max_{0\leq \theta\leq 2\pi}\,
|g'(\theta)|\leq n\cdot 
\max_{0\leq \theta\leq 2\pi}\,|g(\theta)|
\]
\medskip
\noindent
Before we prove this result
we establish an inequality for certain trigonometric polynomials.
Namely, consider a real-valued sine-polynomial
\[
S(\theta)= 
c_1\text{sin}(\theta)+ \ldots+
c_n\text{sin}(n\theta)
\]
Now $\theta\mapsto \frac{ S(\theta)}{\text{sin}\,\theta}$
is an even function of $\theta$ and therefore
one has
\[
\frac{ S(\theta)}{\text{sin}\,\theta}= 
a_0+a_1\text{cos}\,\theta+
\ldots+a_{n-1}(\text{cos}\,\theta)^{-n-1}
\]
Consider the polynomial
\[
p(x)= a_0+a_1x+\ldots+a_{n-1}x^{n-1}
\]
Then e see that:
\[
|p(\text{cos}\,\theta)|=
\frac{|S(\theta)|}{\sqrt{1-\text{cos}^2\,\theta}}
\]
Using this we apply Theorem D.4 to the polynomial $p(x)$ and conclude
\medskip

\noindent
{\bf{D.6 Theorem.}}
\emph{Let $S(\theta)= 
c_1\text{sin}(\theta)+ 
c_n\text{sin}(n\theta)$ be a sine-polynomial as above.
Then}
\[
\max_{0\leq\theta\leq 2\pi}\,
\frac{|S(\theta)|}{\text{sin}\,\theta}\leq n\cdot
\max_{0\leq\theta\leq 2\pi}\,
|S(\theta)|
\]
\bigskip

\noindent
{\bf D.7 Proof of Bernstein's theorem.}
Fix an arbitrary $0\leq\theta-0<2\pi$.
Set
\[
S(\theta)=g(\theta_0+\theta)-g(\theta_0-\theta)
\]
We notice that $S(\theta$is an odd polynomial of
$\theta$ and
$S(0)=0$, It follows that
$S(\theta)$ is a sine-polynomial as above of degree
$\leq n$. Notice also that
\[
\max_{0\leq\theta\leq 2\pi}\,|S(\theta)|\leq 2\cdot
\max_{0\leq\theta\leq 2\pi}\,|g(\theta)|
\max_{0\leq\theta\leq 2\pi}\,|g(\theta)|
\]
Theorem D.6 applied to $S(\theta)$ gives
\[
\bigl|\frac{g(\theta_0+\theta)-g(\theta_0-\theta)}{\text{sin}\,\theta}
\bigr|\leq 2n\cdot 
\max_{0\leq\theta\leq 2\pi}\,|g(\theta)|\tag{i}
\]
Next, in the left hand side we can take the limit as $\theta|\to 0$ and notice that
\[
2\cdot  g'(\theta_0)=
\lim_{\theta\to 0}\, 
\frac{g(\theta_0+\theta)-g(\theta_0-\theta)}{\text{sin}\,\theta}
\]
Hence (i) gives
\[
|g'(\theta_0)|\leq n\cdot 
\max_{0\leq\theta\leq 2\pi}\,|g(\theta)|
\]
Finally, since $\theta_0$ was arbitrary we get Bernstein's theorem.



\bigskip

\centerline{\bf{E. Fejers sine series and Gibb's phenomenon.}}
\bigskip

\noindent
Several remarkable inequalities for trigonometric polynomials were  established by
Fejer in [Fejer] where
a central issue is to find trigonometric polynomials
expressed by a sine series
which are $\geq 0$ on the  interval $[0,\pi]$.
Consider as
an example is the sine\vvv series
\[
S\uuu n(\theta)=\sum\uuu{k=1}^{k=n}\, \frac{\sin k\theta}{k}
\]



\noindent
{\bf{E.1 Theorem.}}
\emph{For every $n\geq 1$ one has the inequality}
\[
0<S\uuu n(\theta)\leq 1+\frac{\pi}{2}\quad\colon\quad
0<\theta<\pi
\]
The upper bound was proved by  in [Fej] and Fejer  conjectured that
$S\uuu n(\theta)$ stays positive on $(0,\pi)$.
This was confirmed in articles by 
Jackson in [xx] and
Cronwall in [xx]. The series (*) has a connection with
Gibb's phenomenon and Theorem E.1 can be illustrated by drawing graphs
of the $S$\vvv polynomials where the
situation when $\theta=\pi\vvv \delta$ for small positive
$\delta$ has special interest. Since $\cos\,\pi=\vvv 1$
the positivity entails that
\[
\sum\uuu {k=2}^n\,(\vvv 1)^k\cdot  \frac{\sin k\delta}{k}\geq \sin\delta \quad
\text{hold for every}\quad n\geq 2\quad  \text{and small}\quad
\delta>0\tag{*}
\]

\medskip


\noindent
{\bf{ Exercise.}}
Prove Theorem E.1 or consult the literature.
It is also  instructive
to confirm (*) by numerical experiments with a computer.
\medskip

\noindent
{\bf{E.2 Mehler's integral formula.}}
In XX we introduced the Legendre polynomials. It turns out that
\[
\mathcal P\uuu n(x)= \sum\uuu{\nu=1}^{\nu=n}\, P\uuu \nu(x)>0
\quad\colon\quad -1<x<1 \tag{*}
\]
is strictly positive for each $\vvv 1<x<1$.

\noindent
{\bf{Exercise.}} Prove (*) using Theorem E.1 and
Mehler's  integral formula
\[ 
\mathcal P\uuu n(\cos \theta)=
\frac{2}{\pi}\cdot \int\uuu 0^\pi\,
\frac{\sin(n+\frac{1}{2})\phi \cdot d\phi}
{\sqrt{2\cos \theta\vvv 2\cos \phi}}\tag{*}
\]


\newpage

\centerline{\bf{F. Convergence of arithmetical means}}
\bigskip

\noindent
Let $f(x)$ be a real\vvv valued and square integrable function
on $(\vvv \pi,\pi)$, i.e.
\[
\int\uuu{\vvv\pi}^\pi\, |f(x)|^2\, dx<\infty
\]
We say that
$f$ has a determined value $A=f(0)$ at $x=0$ if the following two conditions hold:
\[
\lim\uuu{\delta\to 0}
\frac{1}{\delta}\cdot \,\int\uuu 0^\delta\,
|f(x)+f(\vvv x)\vvv 2A|\, dx=0\tag{i}
\]
\[
\int\uuu 0^\delta\,
|f(x)+f(\vvv x)\vvv 2A|^2\, dx\leq C\cdot \delta
\quad
\text{holds for some constant} \quad C\tag{ii}
\]


\noindent
{\bf{Remark.}}
In the same way we can impose this  condition at  every point
$\vvv \pi<x\uuu 0<\pi$. To simplify the subsequent notations
we take $x=0$. 
If  $x=0$ is a Lebesgue point for $f$
and $A$  the  Lebesgue value we have (i).
Hence Lebesgue's Theorem entails  that (i)
holds almost everywhere when 
$x=0$ is replaced by other points $x\uuu 0$.
We leave it to the reader to show
that the second condition also is valid almost everywhere
when $f$ is square integrable but
in general there appears a  null set $\mathcal N$
where (ii) fails to hold while $\mathcal N$ 
contains some Lebegue points.
Next,  expand $f$ in a Fourier series
\[ 
f(x)=\frac{a\uuu 0}{2}+ \sum\, a\uuu k\cdot \cos kx+\sum\, b\uuu k\cdot \sin kx
\]
and with $x=0$ we consider the partial sums
\[ 
s\uuu n(0)= \frac{a\uuu 0}{2}+ a\uuu 1 +\ldots+a\uuu n+
b\uuu 1+\ldots+b\uuu n
\]
The result below is proved 
in [Carleman] and shows  that
$\{s\uuu n\}$ are  close to the determined value  for many $n$\vvv values.


\medskip

\noindent
{\bf{F.1 Theorem.}}
\emph{Assume that $f$ has a determined value $A$ at $x=0$.
Then the following hold for every positive integer $k$}
\[
\lim\uuu{n\to\infty}\, 
\frac{1}{n+1}\cdot
\sum\uuu{\nu=0}^{\nu=n}\, |s\uuu\nu\vvv A|^k=0\tag{*}
\]

\bigskip

\noindent
{\bf{Remark.}}
Carleson's  theorem 
asserts that $\{s\uuu n(x)\}$ converge to
$f(x)$ almost everywhere when $f\in L^2$.
When   a pointwise convergence holds 
the limit formula (*)  is  obvious.
However, it is in general not true that
the  pointwise convergence exists  at \emph{every point}
where  $f$ has a determined value.
So  "ugly points"  may appear in a null\vvv set
where pointwise convergence fails and here
Carleman's result offers a substitute.

\medskip

\noindent
{\bf{The case when $f\in \text{BMO}(T)$}}.
If $f$ has bounded mean oscillation the results from
� XX in  Special Topics show that
the  conditions (i\vvv ii) hold at every Lebesgue point of 
$f$.
So here one has  a      control for  averaged Fourier series of
$f$ expressed via its set of Lebesgue points.

\medskip


\noindent
{\bf{The case when $f$ is continuous.}}
Here (i-ii) hold everywhere so the averaged limit formulas hold
at every point.
We can say more
since $f$ is uniformly continuous.
Let $\omega_f(\delta)$ be the modulos of continuity
function and for each $n\geq 1$,
$||s_n-f||$ is the maximum norm of $s_n-f$
over
$[0,2\pi]$. Set

\[
\mathcal D_n(f)= \sqrt{\frac{1}{n+1}\cdot
\sum\uuu{\nu=0}^{\nu=n}\, ||s\uuu\nu\vvv f||^2}
\]


\medskip




\noindent
{\bf{F.2 Theorem}}. \emph{There exists an absolute constant $K$
such that the following hold for every
continuous function $f$ with maximum norm
$\leq 1$:}


\[ 
\mathcal D_n(f)\leq K\cdot \bigl[ \frac{1}{\sqrt{n}}+\omega_f(\frac{1}{n})\bigr]
\]





\newpage

\noindent
\centerline {\emph{Proof of Theorem F.1}}
\medskip


\noindent
Set $A=f(0)$ and $s_n=s_n(0)$. Introduce the function:
\[ 
\phi(x)= f(x)+f(\vvv x)\vvv 2A
\]
Applying Dini's kernel we have
\[
s\uuu n\vvv A=
\int\uuu 0^\pi\, \frac{\sin (n+1/2)x}{\sin\, x/2}\cdot \phi(x)\cdot dx
\]
By
trigonometric formulas   the  integral is  expressed by 
three terms for each
$0<\delta<\pi$:
\[
\alpha\uuu n=\frac{1}{\pi}\cdot \int\uuu 0^\delta\, 
\sin nx\cdot\cot x/2\cdot
\phi(x)\cdot dx
\]
\[ 
\beta\uuu n= 
\frac{1}{\pi}\cdot\int\uuu \delta ^\pi\, \sin nx\cdot\cot x/2\cdot
\phi(x)\cdot dx
\]
\[
\gamma\uuu n=\frac{1}{\pi}\cdot\int\uuu 0^\pi\, \cos nx\cdot \phi(x)\cdot dx
\]
\medskip

\noindent
By H�lder's inequality it suffices to show Theorem F.1 if
$k=2p$ is an even integer. Minkowski's inequality gives

\[
\bigl[\,\sum\uuu{\nu=0}^{\nu=n} |s\uuu\nu\vvv A|^{2p}\,\bigr]^{1/2p}\leq
\bigl[\,\sum\uuu{\nu=0}^{\nu=n}|\alpha\uuu \nu|^{2p}\,\bigr]^{1/2p}+
\bigl[\,\sum\uuu{\nu=0}^{\nu=n}|\beta\uuu \nu|^{2p}\,\bigr]^{1/2p}+
\bigl[\,\sum\uuu{\nu=0}^{\nu=n}|\gamma\uuu \nu|^{2p}\,\bigr]^{1/2p}\tag{1}
\]
\medskip

\noindent
Denote by $o(\delta)$ small ordo and $O(\delta)$ is big ordo.
When $\delta\to 0$
we shall establish the following:
\[
\bigl[\,\sum\uuu{\nu=0}^{\nu=n}|\alpha\uuu \nu|^{2p}\,\bigr]^{1/2p}=
n^{1+1/2p)}\cdot o(\delta)\tag{i}
\]
\[
\bigl[\,\sum\uuu{\nu=0}^{\nu=n}|\beta\uuu \nu|^{2p}\,\bigr]^{1/2p}\leq K\cdot p\cdot
\delta^{\vvv 1/2p}\tag{ii}
\]
\[
\bigl[\,\sum\uuu{\nu=0}^{\nu=n}|\gamma\uuu \nu|^{2p}\,\bigr]^{1/2p}\leq K
\tag{iii}
\]
\medskip

\noindent
In (ii-iii) $K$ is an absolute constant which is independent of
$p,n$ and $\delta$.
Let us first see why (i\vvv iii) give Theorem F.1.
Write $o(\delta)= \epsilon(\delta)\cdot \delta$
where $\epsilon(\delta)\to 0$.
With these notations (1) gives:
\[
\bigl[\,\sum\uuu{\nu=0}^{\nu=n} |s\uuu\nu\vvv A|^{2p}\,\bigr]^{1/2p}\leq
n^{1+1/2p} \cdot \delta\cdot \epsilon(\delta)
+Kp\cdot \delta^{\vvv 1/2p}+K\tag{*}
\]


\noindent
Next, let  $\rho >0$ and choose $b$ so large that
\[
pKb^{\vvv 1/2p}<\rho/3
\]
Take $\delta=b/n$ and with $n$ large it follows that
$\epsilon(\delta)$ is so small that
\[ 
b\cdot \epsilon(b/n)<\rho/3
\]
Then right hand side in (*) is majorized by
\[
\frac{2\rho}{3}\cdot n^{1/2p}+ K
\]
When $n$ is large we also have
\[
K\leq \frac{\rho}{3}\cdot n^{1/2p}
\]



\noindent 
Hence the left hand side in (*) is majorized by
$\rho\cdot n^{1/2p}$ for all sufficiently large $n$.
Since $\rho>0$ was arbitrary we get Theorem F.1 when  the power 
is raised by $2p$.


\bigskip

\centerline{\emph{Proof of (i\vvv iii)}}
\bigskip

\noindent
To obtain (i) we use the triangle inequality
which gives the following for every integer $\nu\geq 1$:
\[ 
|a_\nu|\leq \frac{2}{\pi}\cdot \max_{0\leq x\leq\delta} 
|\sin \nu x\cdot\cot x/2|\cdot\int_0^\delta\, |\phi(x)|\, dx
=\nu\cdot o(\delta)\tag{1}
\]
where the small ordo  $\delta$-term comes from the hypothesis expressed by (*)
in the introduction. Hence
the left hand side in (i) is majorized by
\[
\bigl[\sum_{\nu=1}^{\nu=n}\, \nu^{2p}\,\bigr]^{\frac{1}{2p}}
 \cdot o(\delta)=
n^{1+1/2p}\cdot o(\delta)
\]
which was requested to get  (i).
To prove (iii) we  notice  that
\[
\gamma_0^2+2\cdot \sum _{\nu=1}^\infty
\gamma_\nu^2=\frac{1}{\pi}\int_0^\pi\, |\phi(x)|^2\, dx
\]
Next, we have
\[
\sum_{\nu=1}^\infty \, |\gamma_\nu|^{2p}\leq
\bigl[\, \sum_{\nu=1}^\infty \, |\gamma_\nu|^2\, \bigr ]^{1/2p}\leq K
\]
where $K$
exists since
$\phi$ is square-intergable on $[0,\pi]$.
\medskip


\noindent
\emph{Proof of (ii)}.
Here several steps are required.
For each $0<s<\pi$
we define the  function $\phi\uuu s(x)$
by
\[
\phi\uuu s(x)=\phi(x)\quad\colon\quad 0<x<s
\] 
and extend it to an odd function, i.e. $\phi\uuu s(\vvv x)=\vvv \phi\uuu s(x)$
while  $\phi\uuu s(x)=0$ when $|x|>s$.
This odd function has a sine series
\[
\phi\uuu s(x)=\sum\uuu{\nu=1}^\infty
c\uuu\nu(s)\cdot \sin x\tag{1}
\]
Let us also introduce the functions
\[
\rho(s)=\int\uuu 0^s\, |\phi(x)|\cdot dx\quad\text{and}\quad
\Theta (s)=\int\uuu 0^s\, |\phi(x)|^2\cdot dx\tag{2}
\]


\noindent
The crucial step   towards the proof of (ii)
is the following:
\medskip

\noindent
{\bf{Lemma.}} \emph{One has the inequality}
\[
\sum\uuu{|nu=1}^\infty\, |c\uuu\nu(s)|^{2p}\leq
(\frac{2}{\pi})^{2p\vvv 1}\cdot \Theta(s)\cdot \rho(s)^{2p\vvv 2}
\]


\noindent
\emph{Proof.}
We  employ convolutions and define inductively a sequence
of functions $\{\phi\uuu{n,s}(x)\}$
where
$\phi\uuu{1,s}(x)= \phi\uuu s(x)$ and 
\[ 
\phi\uuu{n+1,s}(x)=\frac{1}{\pi}
\int\uuu {\vvv \pi}^\pi\, 
\phi\uuu{n,s}(y)\phi\uuu s(x+y)\cdot dy
\]
Since convolution yield products of the Fourier coefficients
and $2p$�is an even integer we  have the
standard formula:
\[
\sum\uuu{\nu=1}^\infty \, c\uuu n(s)^{2p}
=\phi\uuu{2p,s}(0)\tag{1}
\]
Next, using the Cauchy\vvv Schwarz inequality the reader may verify that

\[
|\phi\uuu{2,s}(x)|\leq \frac{2}{\pi}\cdot \Theta(x)
\]
This entails that
\[
\phi\uuu{3,s}(x)\leq
\frac{1}{\pi}
\int\uuu{\vvv \pi}^\pi\,
|\phi\uuu{2,s}(y)|\cdot |\phi\uuu s(x+y)|\cdot dy
\leq 
\frac{2}{\pi^2}\cdot \Theta(s)\cdot 
\int\uuu{\vvv \pi}^\pi\,
|\phi\uuu s(x+y)|\cdot dy=(\frac{2}{\pi})^2\cdot
 \Theta(s)\cdot \rho(s)
\]
Proceeding in this way it follows by an induction that
\[
\phi\uuu{2p,s}(x)\leq (\frac{2}{\pi})^{2p\vvv 1}
 \cdot \Theta(s)\cdot (\rho(s))^{2p\vvv 2}
\]
This holds in particular when $x=0$ and then (1) above gives Lemma 1.
\bigskip

\noindent
{\bf{A formula for the $\beta$\vvv numbers.}}
We have by definition

\[
\beta\uuu\nu=\frac{2}{\pi}\int\uuu{\delta}^\pi\,\sin\,\nu x\cdot
\frac{1}{2}\cot (\frac{x}{2})\cdot \phi(x)\cdot dx
\]
An  integration by parts and the construction of the Fourier coefficients
$\{c\uuu\nu(s)\}$ which applies with $s=\delta$
give:
\[
\beta\uuu\nu=
\vvv \frac{1}{2}
\cdot \,\cot \delta/2\cdot c\uuu\nu(\delta)+
+\frac{1}{4}
\int \uuu\delta^\pi\, 
c\uuu\nu(x)\cdot
\text{cosec}^2 (\frac{x}{2})\cdot dx\tag{*}
\]
Now we  profit upon Minkowski's inequality.
Let $q$ be the conjugate of $2p$, i.e $\frac{1}{q}+\frac{1}{2p}=1$
and choose  $\{\xi\uuu\nu\}$ to be the   sequence in $\ell^q$
of unit norm such that
\[
|\sum \xi\uuu\nu\cdot \beta\uuu\nu|=||\beta\uuu\bullet||\uuu{2p}
\]
where the last term is the left hand side in (ii).
At the same time (*) above and the triangle inequality give
\[
||\beta\uuu\bullet||\uuu{2p}\leq 
\vvv \frac{1}{2}
\cdot \,\cot (\delta/2)\cdot \sum\, |c\uuu\nu(\delta)|\cdot |\xi\uuu\nu|+
\frac{1}{4}\int \uuu\delta^\pi\, 
\text{cosec}^2 (\frac{x}{2})\cdot \sum\,|
c\uuu\nu(x)\cdot\xi\uuu\nu|\cdot dx\leq
\]
\[
 \frac{1}{2}
\cdot \,\cot (\delta/2)\cdot ||c\uuu\bullet(\delta)||\uuu{2p}
+\frac{1}{4}\int \uuu\delta^\pi\, 
\text{cosec}^2 (\frac{x}{2})\cdot 
||c\uuu\bullet(x)||\uuu{2p}\cdot dx\tag{**}
\]
\medskip

\noindent
At this stage we apply Lemma 1 and the assumption which 
give a constant $K$ such that
\[
\Theta(s)\leq K\quad\text{and}\quad \rho(s)\leq K\cdot s
\]

\medskip

\noindent
The last estimate actually is weaker than the hypothesis
but it will be  sufficient to get the requested
estimate of the $\ell^{2p}$\vvv norm  in (ii).
Lemma 1 gives a constant $K\uuu 1$ such that
\[
||c\uuu\bullet(\delta)||\uuu{2p}\leq K\uuu 1\cdot \delta^{1\vvv 1/p}
\]
At the same time we have a constant
$K\uuu 2$ such that
\[
\cot (\delta/2)\leq \frac{K\uuu 2}{\delta}
\]
The product in the first term from (**) is therefore majorized by
$K\uuu 1K\uuu 2\cdot \delta^{\vvv 1/2p}$ as requested in (ii).
For the second term we use Lemma 1 which first gives
\[
||c\uuu\bullet(x)||\uuu p\leq K\cdot x^{\vvv 1/2p}
\]
At this stage we leave it to the reader to verify that
we get a constant $K$ so that
\[
\int \uuu\delta^\pi\, 
x^{\vvv 1/2p}\cdot 
\text{cosec}^2 (\frac{x}{2})\cdot dx\leq K\cdot \delta^{\vvv 1/2p}
\] 
which finishes the proof of (ii).
\bigskip

\centerline{\bf{The case when
$f$ is continuous.}}

\medskip

\noindent
Under the normalisation that
the $L^2$-integral of $f$ is $\leq 1$
the inequalities (ii-iii) hold for an absolute constant $K$.
In (i) we notice that
the construction of
$\phi$ and the definition of $\omega_f$
give the estimates
\[ 
|a_\nu|\leq \nu\cdot \delta\cdot \omega_f(\delta)
\]
With $p=2$ this entails 
that (i) from the proof of Theorem F.1 is majorised by
\[
n^{1+1/2}\cdot \delta\cdot \omega_f(\delta)
\]
This holds for every $0\leq x\leq 2\pi$ and from the previous proof we conclude that
the following hold for each
$n\geq 2$ and every $0<\delta<\pi$:
\[ 
\mathcal D_n(f)\leq\frac{1}{\sqrt{n+1}}\cdot
[n^{1+1/2}\cdot \delta\cdot \omega_f(\delta)+2K\delta^{-1/2}+K]\tag{i}
\]
With $n\geq 2$ we take $\delta=n^{-1}$ and see that
(i) gives a requested constnt in Theorem F.2.








\newpage


\centerline{\bf{G. Best approximation by trigonometric polynomials.}}
\bigskip

\noindent
The results below are due  to de Vall� Poussin
and we follow  Chapter VIII in his text-book [V-P].
Consider
the $2n+2$-tuple 
\[ 
x_j=\frac{2\pi j}{(2n+2)}\quad\colon\quad 1\leq j\leq 2n+2
\]
Let $P(x)$ be a   trigonometric polynomial
in $\mathcal T_n$:
\[ 
P(x)=\sum_{\nu=-n}^{\nu=n}\, 
a_\nu\cdot e^{i\nu x}
\]
Let $f$ be a  $2\pi$-periodic and continuous function and put
\[
\rho_P(f)=\max_j\, |P(x_j)-f(x_j)|
\]

\noindent
Assume that $\rho_P(f)>0$ which gives
a unique $(2n+2)$-tuple of complex numbers
$\{u_j\}$ where every $u_j$ has absolute value $\leq 1$
and
\[ 
f(x_j)= \rho_P(f)\cdot u_j+ 
\sum_{\nu=-n}^{\nu=n}\, 
a_\nu\cdot e^{i\nu x_j}\quad\colon\quad 1\leq j\leq 2n+2\tag{1}
\]


\noindent
{\bf{G.1 Proposition.}} \emph{One has the equality}
\[
\rho_P(f)=
\bigl|\frac{f(x_1)-f(x_2)+\ldots+f(x_{2n+1}-
f(x_{2n})}
{u_1+u_2+ \ldots+u_{2n+1}+
u_{2n+2}}\,\bigr|\tag{*}
\]
\medskip

\noindent
\emph{Proof.} 
Consider  the $(2n+2)\times(2n+1)$-matrix
\[
\begin{pmatrix} 
e^{-inx_1}&\ldots& e^{inx_1}\\
e^{-inx_2}&\ldots& e^{inx_2}\\
\ldots&\ldots&\ldots \\
\ldots&\ldots&\ldots \\
e^{-inx_{2n+2}}&\ldots& e^{inx_{2n+2}}\\
\end{pmatrix}\tag{i}
\]
To each $1\leq k\leq 2n+2$
we denote by $\mathcal A_k$ the
$(2n+1)\times(2n+1)$-matrix which arises when the $k$:th row
is deleted. Using van der Monde formulas 
the reader can verify that
\[ 
A_k=\det(\mathcal A_k)=\prod^{(k)}_{1\leq i\leq j\leq 2n+2}\,
\sin\, \frac{x_j-x_i}{2}\tag{ii}
\]
where $(k)$ above the product sign indicates that
$i$�and $j$ both are $\neq 0$ in the product. We  leave it to the reader to show that
there exists a positive constant $A_*$ such that
\[ 
\det A_k=A_*\quad\colon\quad 1\leq k\leq 2n+2\tag{iii}
\]
Now 
(1) is  a system of linear equations where
$\rho_*(P),a_{-n},\ldots,a_n$ are the indeterminate variables. By
Cram�r's rule 
we can solve out $\rho_*(P)$ via 
the $2n+2$-matrix and (iii) gives
\
\[
\rho_P(f)=
\frac{A_1f(x_1)-A_2f(x_2)+\ldots+A_{2n+1}f(x_{2n+1}-
A_{2n}f(x_{2n})}
{A_1 u_1+A_2 u_2+ \ldots+A_{2n+1}u_{2n+1}+
A_{2n}u_{2n}}\tag{iv}
\]
Together with (iii) the requested equation (*) in Proposition  G.1 follows.


\medskip

\noindent
{\bf{G.2 Conclusion.}}
To find
\[ 
\rho_n(f)= \min_{P\in\mathcal T_n}\, \rho_P(f)
\]
we should  choose $P$
so that all the $u$-numbers are +1 or -1. This determines the $a$-numbers in the system (1), i.e. we find 
a unique polynomial $P_*\in\mathcal T_n$
for which
\[
\rho_n(f)= \max_{1\leq j\leq 2n+2}\, |P_*(x_j)-f(x_j]|
\]
Moreover,
the deviation
numbers
$|P_*(x_j)-f(x_j)|$ are all equal to 
\[
\frac{\bigl|f(x_1)-f(x_2)+\ldots+f(x_{2n+1}-
f(x_{2n})\,\bigr|}{2n+2}
\tag{**}
\]

\medskip


\noindent
{\bf{G.3 Exercise.}}
Let $f$ be given with a   Fourier series expansion:
\[ 
f(x)=\frac{1}{2}\cdot a_0+
\cdot \sum_{k=1}^\infty\,( \alpha_k\cdot \cos\,kx+\beta_k\cdot \sin\, kx)
\]
Use the formula (ii) from the proof above to
show that
\[
\frac{1}{2n+2}\cdot \sum_{\nu=1}^{\nu=2n+2}\, (-1)^\nu\cdot f(x_\nu)=
\sum_{j=0}^\infty\, a_{(2j+1)(n+1)}\tag{***}
\]
\medskip

\noindent
Together (**) and  (***)  give:

\medskip

\noindent
{\bf{G.4 Theorem.}}
\emph{For each integer $n$ and every $2\pi$-periodic 
function $f(\theta)$ one has the equality}
\[
\rho_n(f)= \bigl|a_{n+1}+ a_{3(n+1)}+ a_{5(n+1)}+\ldots\,\bigr|
\]


\noindent
{\bf{G.5 Remark.}}
Since the maximum norm taken over the whole interval $[0,2\pi]$
majorizes the maximum norm over
the $2n+2$-tuple above, we get
the  inequality:

\[
\min_{P\in\mathcal T_n}\, ||f-P||\geq\,\bigl|a_{n+1}+ a_{3(n+1)}+ a_{5(n+1)}+\ldots\,\bigr|
\]
where $||f-P||$ is the maximum norm over
$[0,2\pi]$. This gives the result announced in 
� 0.X from the introduction.






\newpage





\end{document}














