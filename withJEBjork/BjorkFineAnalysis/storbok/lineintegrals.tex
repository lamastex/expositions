

\documentclass{amsart}

\usepackage[applemac]{inputenc}


\addtolength{\hoffset}{-12mm}
\addtolength{\textwidth}{22mm}
\addtolength{\voffset}{-10mm}
\addtolength{\textheight}{20mm}


\def\uuu{_}

\def\vvv{-}

\begin{document}

\centerline{\bf\large Chapter II: Stokes Theorem}

\bigskip


\medskip


\noindent
0. Introduction
\medskip

\noindent
0: A. Calculus in ${\bf{R}}^2$.

\medskip


\noindent
1. Some physical explanations

\medskip


\noindent
2. Stokes Theorem in ${\bf{R^2}}$
\medskip


\noindent
3. Line integrals via differentials
\medskip


\noindent
4. Green's formula and the Dirichlet problem
\medskip


\noindent
5. Exact versus closed 1-forms

\medskip


\noindent
6. An integral formula for the Laplace operator






\bigskip

\noindent
\centerline {\bf 0. Introduction.}
\bigskip

\noindent
The results in � 0:A    cover  the basic
integral formulas which appear
in analytic function theory.
However, these results 
require certain regularity assumptions and 
we  give therefore
another proof of Stokes Theorem under relaxed
regularity conditions in � 2.
Stokes theorem in the plane follows from
Fermat's descriptions of normals 
and the arc-length of a  plane curve
and the fundamental theorem of calculus.
More precisely, let $f(x)$ be a positive and continuously differentiable function 
defined on some closed interval $[a,b]$.
Let $\Omega$ be the domain bordered
by
$\{y=0\}$ and the graph of $f$, i.e.
\[ 
\Omega=\{x,y)\colon 0<y<f(x)\colon b< x<b\}
\]
Let $g(x,y)$ be a $C^1$-function in $\Omega$ which
extends to a continuous function on its closure. Set
\[
J(x)=\int_0^{f(x)}\, g(x,y)\, dy\quad\colon a\leq x\leq b
\]
Rules for derivation give
\[
J'(x)=\int_0^{f(x)}\, g_x(x,y)\, dy+ g(x,f(x))\cdot f'(x)\tag{i}
\]
where $g_x$ is the partial derivative with respect to $x$.
The fundamental
theorem of calculus gives
\[
J(b)-J(a)= \int_a^b\, J'(x)\, dx\tag{ii}
\]
The right hand side in  (ii) is the sum of two integrals. The first is 
the double integral
\[
\iint_\Omega\, g_x(x,y)\,dxdy\tag{iii}
\]
The second becomes
\[
\int_a^b\, g(x,f(x))\cdot f'(x)\,dx\tag{iv}
\]
The crucial fact is Fermat's  formula:
\[
-\frac{f'(x)}{\sqrt{1+f'(x)^2}}={\bf{n}}_x\tag{v}
\]
where ${\bf{n}}_x$ is the 
 $x$-component of the outer normal
to the plane curve
$\gamma=\{y=f(x)\}$.
The reader should check the minus sign in (v) with the aid of a figure.
Moreover, Pythagoras' theorem
entails that  the arc-length along the curve is 
$\sqrt{1+f'(x)^2}\cdot dx$ and
the equations above give
\[
\iint_\Omega\, g_x(x,y)\,dxdy=J(b)-J(a)+
\int_\gamma\, g\cdot {\bf{n}}_x\, ds\tag{vi}
\]
Finally $J(b)$ and $J(a)$ can be expressed as line integrals along
the pair of vertical lines in the boundary of
$\Omega$. Along $\{x=b\}$
the outer normal ${\bf{n}}=e_x$
and $ds=dy$ and along
$\{x=a\}$ 
${\bf{n}}=-e_x$ and $ds=dy$.
 The right hand side in (vi) can therefore be written as
 \[
 \int_{\partial\Omega}\, g\cdot {\bf{n}}_x\, ds\tag{*}
\]
The equality between the line integral (*) and the area integral of
$g_x$ is a special case of
Stokes theorem.
In � xx we establish   the general case   using partitions of
the unity
which reduce the proof to the case described above.
\medskip


\noindent
{\bf{The use of differential forms.}}
Line integrals are often  expressed via  differential 1-forms
which lead 
to alternative 
expressions of  Stokes Theorem
in � 3 and � 5.
An important result appears in
Theorem 10 from   � 2 where a pair of functions appear 
in integral formulas.
This will be  applied
in  � 4 where
integral formulas related to the Laplace operator occurs.
Here is an  example  which will
be used later to study harmonic  and more generally subharmonic functions.
Let $u(x,y)$ be a $C^2$-function defined in some
open disc of radius $R$ centered at the origin in
${\bf{R}}^2$. Then the following hold for all pairs
$0<s<R$:
\[ 
u(0)=\frac{1}{2\pi}\int_0^{2\pi}\, u(s,\theta)\cdot d\theta+
\iint_{D(s)}\, \text{log}(\frac{s}{\sqrt{x^2+y^2}})\cdot \Delta u(x,y)\cdot dxdy\tag{1}
\]
where $D(s)$ is the disc of radius $s$ centered at the origin.
After another integration while (1) is applied as  $0<s<r<R$ we obtain:
\[
u(0)= \frac{1}{\pi r^2}\cdot \iint_{D(r)}\, u(x,y)\cdot dxdy-
\int_0^r\,K(r,s)\cdot \bigl [\int_0^{2\pi}  \Delta(u)(s,\theta)\cdot d\theta\bigr]\cdot ds\tag{2}
\]
where $K_r(s)$ is a  kernel function defined 
for pairs $0<s\leq r$ by:
\[ 
K(r,s)=s^3\cdot \int_1^{\frac{r}{s}}\, u\cdot \text{Log}(u)\cdot du\tag{3}
\]


\noindent
In (2) the first double integral in the right hand side is the mean-value of
$u$ over the disc $D(r)$. The second double integral
describes the
difference between this mean-value and the value of $u$ at the center
of the disc.
If the function $\Delta(u)$
is non-negative in the whole disc $D(R)$
we get
the mean-value inequality
\[
u(0)\leq  \frac{1}{\pi r^2}\cdot \iint_{D(r)}\, u(x,y)\cdot dxdy
\] 
for every $0<r<R$.
In Chapter V:B we shall learn that this gives  the starting point for
a study of subharmonic functions.
\bigskip

\noindent
{\bf{Polar coordinates.}} Set
\[ 
x=r\cdot \cos\theta\quad\text{and}\quad
y=r\cdot \sin\theta
\]
Looking at the first order partial derivatives outside the origin we have
\[
\partial\uuu r=\cos\theta\cdot \partial\uuu x+\sin\theta\cdot \partial\uuu y
\quad\text{and}\quad
\frac{1}{r}\cdot \partial\uuu \theta
=\vvv \sin \theta\cdot \partial\uuu x+\cos\theta\cdot \partial\uuu y
\]
It follows that
\[
\partial\uuu x=\cos\theta\cdot \partial\uuu r\vvv \frac{1}{r}\cdot
\sin \theta\quad\text{and}\quad
\partial\uuu y=\sin\theta\cdot \partial\uuu r+ \frac{1}{r}\cdot\cos\theta
\]
\medskip

\noindent
We have the Laplace operator $\Delta=\partial^2\uuu x+\partial^2\uuu y$
and leave it to the reader to verify that
\[
\Delta=\partial\uuu r^2+\frac{1}{r}\cdot\partial\uuu r+
\frac{1}{r^2}\cdot \partial\uuu\theta^2\tag{*}
\]


\noindent
Next, consider a $C^2$\vvv function $F(x,y)$
defined in the unit disc $D=\{x^2+y^2<1\}$ with $F(0,0)=0$.

\medskip

\noindent
{\bf{0.1 Theorem.}}
\emph{The 
following equality holds for every $0<r<1$:}
\[ 
\int\uuu 0^{2\pi}\,\bigl[\, r^2\cdot (\partial\uuu r(F))^2\vvv
(\partial\uuu\theta(F))^2\bigr]\cdot d\theta=
2\cdot \int\uuu 0^r\,\bigl[\, \int\uuu 0^{2\pi}\, s^2\partial\uuu s(F)\cdot\Delta(F)
\cdot d\theta \,\bigr]\dot ds\tag{*}
\]
\medskip

\noindent
To prove this via Stokes or Green's theorem would be cumbersome.
Instead on employs a Fourier series expansion. Namely, 
we can write
\[
F(r,\theta)= 
\sum\uuu{k=1}^\infty
a\uuu k(r)\cdot \cos\, k\theta+
b\uuu k(r)\cdot \sin\, k\theta
\] 
where $\{a\uuu k(r)\}$ and $\{b\uuu k(r)\}$ are two sequences of
functions which only depend on $r$.
\medskip

\noindent
{\bf{0.2 Exercise.}} Use the familiar vanishing results for integrals of
sine\vvv and cosine\vvv functions
to conclude that it suffices to prove
(*) in the special case when
$F= a\uuu k(r)\cdot \cos\, k\theta$ or $F= b\uuu k(r)\cdot \sin\, k\theta$
for a single integer $k$.
Next, consider the case when
\[ 
F=a(r)\cdot \cos\, k\theta
\] 
for some positive integer $k$.
Using the polar formula for $\Delta$  the right hand side
in (*) becomes

\[
\int\uuu 0^r\,\bigl[\, \int\uuu 0^{2\pi}
\, s^2\cdot a'(s)\cdot(a''(s)+\frac{1}{s}\cdot a'(s)\vvv
\frac{k^2}{s^2}a(s)\bigr]\cdot \cos^2 k\theta\cdot d\theta
\,\bigr]\cdot ds
\]
Notice that partial integration gives

\[
\int\uuu 0^r\, \, s^2\cdot a'(s)\cdot(a''(s)+\frac{1}{s}\cdot a'(s)\vvv
\frac{k^2}{s^2}a(s)\bigr]\cdot ds=\frac{r^2a'(r)^2\vvv k^2a(r)^2}{2}
\]
It follows that the right hand side in (*) is equal to
\[
\bigl[r^2a'(r)^2\vvv k^2a(r)^2\,\bigr]\cdot \int\uuu 0^{2\pi}\, \cos^2\,k\theta\cdot d\theta
\tag{1} 
\]
At this stage the reader can confirm that (1) is equal to the right hand side in
(*).
\medskip


\noindent
{\bf{0.3 Remark.}}
The integral formula in Theorem 0.1 will be
used to study subharmonic functions, i.e functions
$f$ for which $\Delta(F)\geq 0$. For example, if 
$\partial\uuu r(F)\geq 0$ holds in the disc, then
the right hand side in (*) is $\geq 0$ which gives the inequality
\[
\int\uuu 0^{2\pi}\,\partial\uuu\theta(F)^2\cdot d\theta\leq
r^2\cdot \int\uuu 0^{2\pi}\,\, \partial\uuu r(F)^2\cdot d\theta
\quad\text{for every}\quad 0<r<1\tag{**}
\]



















\newpage

\centerline{\bf{0:A. Calculus in ${\bf{R}}^2$}}
\medskip

\noindent
A parametrised curve $\gamma$ 
in
${\bf{R}}^2$ is defined by
a vector-valued function
\[ 
\gamma\colon\quad t\mapsto(x(t),y(t))\quad\colon\quad 0\leq t\leq T\tag{*}
\]
The  curve is of class $C^2$ if 
$x(t)$ and $y(t)$ are both of class $C^2$. 
We do not assume that (*)
is 1-1 so  the parametrized curve can have self\vvv intersections.
Line integrals are constructed as follows:
Let $u(x,y)$ and $v(x,y)$ be a pair of continuous functions
and
set
\[
\int_\gamma\, u\cdot dx=\int_0^T\, u(x(t),y(t))\cdot  x'(t)\, dt\tag{1}
\]
\[
\int_\gamma\, v\cdot dy=\int_0^T\, v(x(t),y(t))\cdot y'(t)\, dt\tag{2}
\]
We refer to (1) as a line integral in the 
$x$-direction and  (2) as a 
line integral in the $y$-direction.
The notation $\int\uuu\gamma$
is consolidated by the fact that if
$s\mapsto t(s)$ is some strictly increasing $C^1$-function from
an interval $0\leq s\leq S$ onto
$[0,T]$ then the  $s$-parametrisation does not change
the integrals since
\[
\int_0^T\, u(x(t),y(t)\cdot x'(t)\, dt=\int_0^S\, u(x(t(s)),y(t(s))
\cdot \frac{d}{ds}(x(t(s))\, ds
\] 
where we used the  differential equality
\[
\frac{d}{ds}(x(t(s))=x'(t(s))\cdot t'(s)
\]
However, the orientation is essential.
One moves from the initial point $\gamma(0)=P$ to the end\vvv point
$\gamma(T)=Q$  where
an arrow along the  curve is used to indicate the
direction during the integration. Thus, 
up to a sign the  line integrals depend upon the chosen orientation.
\medskip

\noindent
{\bf{A.1 Homotopy.}}
Consider a family of  $C^2$-curves
$\{\gamma_s\,\colon 0\leq s\leq 1\}$
where each single $\gamma_s$
is parametrised over an interval $[0,T]$ as above and
the curves have common end-points
$P=(x_*,y_*)$ and $Q=(x^*,y^*)$.
So when $t\mapsto (x(t,s),y(t,s))$ is a 
parametrisation of $\gamma_s$, then
\[
x(0,s)=x_*\quad\text{and}\quad x(T,s)= x^*
\quad\text{for all}\quad 0\leq s\leq 1\tag{1}
\]
and similarly for the $y$-function.
No further assumption is imposed, i.e. the
$\gamma$ curves in the family need not be simple and
it may occur that $P=Q$ or that $P\neq Q$.
Concerning the  functions $x(t,s)$ and $y(t,s)$ we impose the
condition that both are of class $C^2$. 
In particular the mixed second order derivatives  are equal.
\[
 x''_{ts}=x''_{st}\quad\text{and}\quad y''_{ts}=y''_{st}\tag{2}
\]


\noindent
{\bf{A.2 Theorem.}} \emph{Let $u(x,y)$ and $v(x,y)$ be  $C^1$-functions.
Then one has
the two equalities}
\[
\int_{\gamma_1}\, u\cdot dx-
\int_{\gamma_0}\, u\cdot dx
=\iint_\square\, u'_y\cdot (y'_s\cdot x'_t-y'_t\cdot x'_s)\, dsdt\tag{*}
\]
\[
\int_{\gamma_1}\, v\cdot dy-
\int_{\gamma_0}\, v\cdot dy
=\iint_\square\, v'_x\cdot (x'_s\cdot y'_t-x'_t\cdot y'_s)\, dsdt\tag{**}
\]


\noindent
\emph{Proof.}
We prove (*) while (**)  is left to the reader since
the proof is the same when $x$ and $y$ are interchanged.
The fundamental theorem of calculus  gives
\[
\int_{\gamma_1}\, u\, dx-
\int_{\gamma_0}\, u\, dx=\int_0^1\,\frac{d}{ds}\bigl(\int_0^T\, u\cdot x'_t\, dt
\bigr)
\,ds=
\]
\[
\iint_\square\, \bigl[ u'_x\cdot x'_sx'_t+u'_y\cdot y'_sx'_t+ u\cdot x''_{st}\bigr]
\,dsdt\tag{i}
\]
Next, consider for each \emph{fixed} $0\leq s\leq 1$
the integral
\[
\int_0^T\frac{d}{dt}(u(x(t,s),y(t,s))\cdot x'_s(t,s))\cdot dt\tag{ii}
\]
By (1) 
the functions
$s\mapsto x(0,s)$ and $s\mapsto x(T,s)$ are constant
and therefore
\[ 
x'_s(0,s)=x'_s(T,s)=0\quad\text{for each}\quad 0\leq s\leq 1\tag{iii}
\]
Hence the integral  (ii) is zero
for every $s$.
At the same time  we can differentiate the
function under the integral sign  and conclude that
\[
\int_0^T\,
(u'_x\cdot x'_t\cdot x'_s+u'_y\cdot y'_t\cdot x'_s+
u\cdot x''_{ts})\cdot dt=0\tag{iv}
\]
Since (iv) holds for every $s$  we get a vanishing double integral
\[
\iint_\square\,
(u'_x\cdot x'_t\cdot x'_s+u'_y\cdot y'_t\cdot x'_s+
u\cdot x''_{ts})\cdot ds dt=0\tag{v}
\]


\noindent
Finally, since $x(t,s)$ is assumed to be a $C^2$-function
the mixed derivatives
$x''_{ts}$ and  $x''_{st}$ are equal.
Subtracting the zero integral in (v) from (i)
we conclude that  (i) is equal to the double integral

\[
\iint_\square\, \bigl[ u'_y\cdot (y'_s\cdot x'_t-x'_t\cdot y'_s)\cdot dsdt
\]
which gives  (*) in Theorem A.2.
\bigskip


\noindent
{\bf{A.3 Application.}}
If the
partial derivatives $u'_y$ and $v'_x$ are equal,
then the reversed signs for  $y'_sx'_t-y'_tx'_s$
which appear in (*) and (**) give the
equality

\[
\int_{\gamma_1}\, (u\cdot dx+v\cdot dy)=
\int_{\gamma_0}\, (u\cdot dx+v\cdot dy)\tag{***}
\]

\noindent
{\bf{A.4 Remark.}}
Given the pair $(u,v)$ one refers to
$u\cdot dx+v\cdot dy=0$ as a differential 1-form.
By definition it is closed if and only if
\[
u'_y=v'_x
\]
So Theorem A.2  shows that the line integral of a closed
1-form
is not changed under a homotopy deformation where
the end-points are kept fixed.
\medskip

\noindent
{\bf{The case of closed curves.}}
The same proof as in  Theorem A.2 shows that if $\{\gamma\uuu s\}$
is a homotopic family of closed curves
and $u'\uuu y=v'\uuu x$ then we also have the equality
(***) above.

\bigskip

\noindent
{\bf{A.5 Stokes Theorem.}}
Consider  a map $\Phi$ from
a rectangle 
\[
\square=\{0\leq t\leq T\}\times
\{0\leq s\leq 1\}
\]
into ${\bf{R}}^2$. 
We  write
$\Phi(t,s)=(x(t,s),y(t,s))$
and  assume that $x(t,s)$ and $y(t,s)$ are $C^2$.-functions.
The  \emph{Jacobian} of $\Phi$ becomes

\[ 
\mathcal J_\Phi= y'_t\cdot x'_s-x'_t\cdot y'_s
\]
So if $u(x,y)$ is some $C^1$-function  then the double integral
\[
\iint_
{ \Phi(\square)}\, u'_y\cdot dxdy=
\iint_\square\, u'_y(x(t,s),y(t,s)\cdot 
[ y'_t\cdot x'_s-x'_t\cdot y'_s]\dot dsdt\tag{1}
\]
By Theorem A.2 the  double integral
is a difference of two line integrals
taken over $s=0$ and $s=1$ respectively.
Suppose now that $\Phi$ is 1-1 so that $\square$ is mapped to a Jordan domain
$\Omega$ whose boundary $\partial\Omega$
is a simple Jordan curve $\Gamma$. 
Here it may occur that $\Gamma$ has corner points at
the images of the four corner points of $\square$. But
in any case the situation is sufficiently regular in order that
we get a well defined line integral
\[
\int_\Gamma\, u\cdot dx
\]
Inspecting the sign of $y'_sx'_t-y'_tx'_s$ which appears in
the Jacobian $\mathcal J_\Phi$ respectively in (*) from Theorem A.2
we conclude that one has a minus sign in the equation below:

\[
\iint_\Omega
\, u'_y\cdot dxdy=-\int_{\partial\Omega}\, u\cdot dx\tag{2}
\]
Here the simple closed Jordan curve
$\partial\Omega$ is oriented in the positive direction, i.e. one moves
counter-clockwise
as this curve encloses $\Omega$. The reader should illustrate this by
drawing a figure.
In  similar fashion one derives the formula
\[
\iint_\Omega
\, v'_x\cdot dxdy=\int_{\partial\Omega}\, v\cdot dy\tag{3}
\]
\bigskip


\noindent
{\bf{A.5 Remark.}}
The results above are in principle all we need to move directly to Chapter 3 and 
study analytic functions where the Cauchy-Riemann equations are
used to ensure
that we are in "favourable situations" such as  (***)
above.
But for the reader's convenience we shall
repeat certain arguments and
give
another proof  in � 2  which has the merit that
regularity conditions can be relaxed.

\medskip


\noindent
{\bf{A result by Schwarz}}.
Above we used the equality of second order mixed derivatives.
A sufficient condition   for its validity is due to Schwarz
where the notable point is that one only requires that one 
of the mixed derivatives exists as a continuous function.
Consider a function $f(x,y)$ of two real variables whose
first order partial derivatives
$f\uuu x$ and $f\uuu y$ exist as continuous functions. Then one has:

\medskip

\noindent
{\bf{A.6 Theorem.}}
\emph{Assume that the mixed second order derivative
$f\uuu{xy}$ exists as a continuos function. Then the
mixed derivative $f\uuu{yx}$ exists and is equal to $f\uuu{yx}$.}
\medskip

\noindent
\emph{Proof.}
With $y$ kept fixed and
some chosen constant $k$  we consider the function
\[ 
\phi(x)= f(x,y+k)\vvv f(x,y)
\]
Taylor expansion up to order two with
the usual mean\vvv value taken as second term gives
for every $h$ a pair $0<\theta,\theta'<1$ such that

\[
\phi(x+h)\vvv \phi(x)= h\bigl(f\uuu x(x+\theta h,y+k)\vvv
f\uuu x(x+\theta h,y)\,\bigr)=
hkf\uuu{xy}(x+\theta h,y+\theta' k)\tag{1}
\]
The continuity of $f\uuu {xy}$ entails that
\[
f\uuu{xy}(x+\theta h,y+\theta' k)=f\uuu{x,y}(x,y)+\epsilon(h,k)
\] 
where the $\epsilon$\vvv function tends to zero as
$(h,k)\to (0,0)$.
Next, division with $k$ in (1) and a passage to the limit as $k\to 0$
gives
\[ f\uuu y(x+h,y)\vvv
f\uuu y(x,y)=
h(f\uuu{xy}(x+\theta h,y)+\epsilon(h,0))\tag{2}
\]
Since $\epsilon(h,0)\to 0$ when $h\to 0$
it follows from (2) that 
\[ 
\lim\uuu{h\to 0}\,  
\frac{f\uuu y(x+h,y)\vvv f\uuu(x,y)}{h}= f\uuu{xy}(x,y)
\]
By definition the left hand side yields the mixed derivative
$f\uuu{yx}$ and Schwarz' theorem is proved.























\newpage



\centerline{\bf{1. Some physical explanations.}}
\bigskip

\noindent
{\bf{Introduction.}}
The material in this section
is not necessary in the sequel. It is included to give a perspective upon
Stokes Theorem and we also discuss results  in dimension 3.
The first lessons about line integrals,
area integrals and volume integrals go back to Archimedes.
The beginner should first of all
understand how Archimedes computed the volume
of a pyramid $\Delta$.
Start in the $(x,y)$-plane from
a bounded and convex domain $U$ bordered by a piecewise
linear boundary $\partial U$ with some finite set
of corner points $\{p_k=(x_k,y_k)\}$.
They are arranged so that a line $\ell_k$ joints $p_k$ to $p_{k+1}$
when $1\leq k\leq N-1$ and  a line 
$\ell_N$ joins $p_N$ with $p_1$.
Next, consider a point $p^*=(x_0,y_0,z_0)$ where
$z_0>0$.
We obtain the pyramid $\Omega$ by joining $p^*$ to each
corner point. It is clear  that
 $\partial\Omega$ consists of 
has $N$ many planar pieces plus
lines when a pair intersect and a number of corner points.
The reader may illustrate this by a figure
when  $U$ is  a square.
With the notations above  
Arkimedes' proved the formula:
\[ 
\text{Vol}\, \Omega=\frac{z_0}{3}\cdot \text{Area}(U)\tag{*}
\]


\noindent
{\bf{Remark.}}
The proof relies upon the fundamental
fact that under dilation  expressed by some $s>0$ areas change with
the scale factor $s^2$ and volumes by $s^3$.
Then (*) follows when we regard portions of $\Omega$
where the $z$-coordinate is restricted to
small intervals $\{\frac{k}{N}\leq z\leq \frac{k+1}{N}\}$.
Taking a limit as $N\to\infty$
the scale principles give
\[
\text{Vol}\, \Omega=
\text{Area}(U)\cdot\int_0^{z_0}\,
(z_0-z)^2\cdot dz=\frac{z_0}{3}\cdot \text{Area}(U)
\]
A more involved case  compared with a pyramid
arises when we consider a bounded open set
$\Omega$ where $\partial\Omega$
consists of $N$ many planar sets $U_1,\ldots,U_N$ and each
intersection between two such planar sets is a line segment. 
Several $U$-sets may also intersect at corner points
on the boundary.
Now we have the \emph{area measure}
$dA_\nu$ on  every $U_\nu$ and the
\emph{outer normal} vector
${\bf{n}}_\nu$ to $U_\nu$, i.e. the unit vector which is $\perp$
to $U_\nu$ and points out from $\Omega$.
If ${\bf{n}}_\nu(x)$ denotes the $x$-component of this unit vector
one has the formula
\[
\text{Vol}\, (\Omega)=\sum_{\nu=1}^N\,
\iint_{U_\nu}\, x\cdot
{\bf{n}}_\nu(x)\cdot dA_\nu\tag{***}
\]
Above we can replace $x$ by $y$ or $z$ using
the components
${\bf{n}}_\nu(y)$ or ${\bf{n}}_\nu(z)$ when we compute   area 
integrals over $\{U\uuu\nu\}$.
This  volume formula  goes back to Archimedes via 
physical considerations.
\medskip

\noindent
{\bf{The principle of Arkimedes.}}
An experience  which every child less than five years
has already discovered,  is that a body which is gently placed in water
eventually comes to
a position at rest when   there are no
waves or streams.
The shape of the body can be highly irregular. Imagine
a piece of a broken tree with
several branches where some of them may
be above the waterline in the  floating position.
The intersection with
the free water line $z=0$ and the boundary of the tree need not even be connected.
In � 1.3  we explain
how the principle of Arkimedes
confirms the validity of (***) even for domains
$\Omega$ with a highly irregular boundary.
The equation (***)
is therefore a Law of Nature rather than a "theorem"
derived in mathematics.
\medskip

\noindent
{\bf{1.1 Stokes Theorem.}}
Let us  describe
how Newton  and his contemporaries   Boyle and Hooke 
were aware of a \emph{conceptual  proof}
of Stokes formula in three dimensions.  Newton
attributes the basic ideas 
below  to  D�scartes in his work 
\emph{Principia} from 1687 
when  he argues as follows to confirm Stokes theorem:
\medskip

\noindent
A bounded connected domain $\Omega$ is given in
${\bf{R}}^3$. The boundary may consist of several pairwise disjoint closed surfaces
of class $C^1$ at least,
i.e. sufficiently regular in order  that we can refer to surface
area measure and the outer normal along every component of
$\partial\Omega$. Inside $\Omega$ a large number of small particles - think of small balls - are moving. They have equal mass and
when they impinge with each other the impact is elastic.
At a point $p=(x,y,z)$ the "mean neighbor velocity" of balls close to 
$p$ is a vector valued function $v(p)=(f(p),g(p),h(p))$, i.e.
$f(p)$ is the velocity in the $x$-direction and so on.
The interior of $\Omega$ is divided into small pairwise disjoint cubes 
$\{\square_\alpha\}$ with sides parallell to the coordinate axes.
The effect of all impacts from  balls inside  one cube $\square_\alpha$
which hit the boundary of
$\square_\alpha$
during a small time interval is a force vector which approximately will
be
\[
F_\alpha=\rho\cdot (f_x,g_y,h_z)\cdot
\text{vol}(\square_\alpha)
\]
Here $f_x,g_y,h_z$�are  the partial 
derivatives inside $\square_\alpha$, $\rho$ a constant density of mass
and
$\text{vol}(\square)_\alpha$ the volume of the square.
The boundary of each   cube is   some hard material so that
via the force vectors $F_\alpha$, each cube 
"pushes" - or alternatively gets a push -
from some of its
six  many neighbor cubes with which it has a common side.
For example, if $f_x>0$ in a given cube
$\square_\alpha$ then $\square_\alpha$
tends to
push on the cube next to the right.
Along the boundary the impact is
expressed by the area integral
\[ 
\int_{\partial\Omega}
\rho\cdot (fn_x+gn_y+hn_z)dA\quad\colon\quad dA=\,\text{area measure}\tag{*}
\] 
where  $(n_x,n_y,n_z)$ the outer normal.
Since the balls cannot escape the  container $\Omega$ 
the \emph{principle of reaction forces} implies that we must have the equality
\[
\sum\,F_\alpha=
\int_{\partial\Omega}
\rho\cdot (fn_x+gn_y+hn_z)dA\tag{**}
\]
The sum to the left approximates the volume integral
of the function $\rho\cdot(f_x+g_z+h_z)$. Dividing out
$\rho$ we get
the equality
\[ 
\int\int\int\,
(f_x+g_y+h_z)dxdydz=
\int_{\partial\Omega}
(fn_x+gn_y+hn_z)dA\tag{***}
\]
\medskip

\noindent {\bf Remark.}
In (***) we  encounter an arbitrary  triple of functions and hence one has three
equations:


\[
\iiint f_x\cdot dxdydz=
\iint_{\partial\Omega}
fn_x\cdot dA
\] 
\[
\iiint g_y\cdot dxdydz=
\iint_{\partial\Omega}
gn_y\cdot dA
\] 
\[
\iiint h_z\cdot dxdydz=
\iint_{\partial\Omega}
hn_z\cdot dA\,,
\]
\medskip

\noindent
Personally I  find Newton's proof  convincing.
Advancements in  mathematics rely upon ideas as above.



\bigskip


\centerline{\bf 1.2. The principle of Archimedes}

\bigskip


\noindent
Consider a 3-dimensional  body $K$ 
placed in
${\bf{R}}^3$
where $(x,y,z)$ are the coordinates and $z$ is vertical so that
the force of gravity is $-g\cdot e_z$.
The body has some   distribution of mass which need not have
a constant density. Imagine a ship where
the density is large in the machine room and considerably lighter
in the lounge bar. In any case, 
the body has a specific weight $0<s<1$, i.e. if 
$V$
is the volume then
the mass of $K$ is $s\cdot V$. 
Now $K$ is gently put into the water
by a five year old child
in the middle of a  reasonably large lake
at a time when there are no winds or waves.
A  child predicts correctly that $K$  will float  and 
after a short time  even come to rest.
The problem is to determine the floating position 
using  the
force of gravity and the law of momentum in statics.
This was solved by
the the genius Archimedes. His studies about floating bodies
were reconsidered by Stevin  around 1600 after original work by Archimedes'
had been rediscovered. Like
Galilei in Pisa, Stevin  also
performed experiments in Amsterdam
to show that  velocities of falling bodes
are independent of their specific weight or total mass and he is
regarded as the creator of  modern statics.
Let
$\mathfrak{o}$ be the center of mass in 
$K$. 
Notice that $\mathfrak{o}$
need not be contained in $K$. A typical example is an oil-platform.
Let $K_*$ be the portion of
$K$ below the free waterline determined by the equation $z=0$.
In general $K_*$ can be
a disconnected set.
But connected components of $K_*$
are bounded by a surface below the free
waterline
and some
area domain in the plane $z=0$ which serves as a  "roof" for this component.
The reader should  illustrate  this by a figure.
Now there
exists the point $\mathfrak{0}^*$
which corresponds to the center of mass which is determined when
$K_*$ has a uniform density of mass.
With these notations Archimedes stated that when the body
has a resting floating position, then
the mass $M$ of the body is equal to the mass of water
which would fill the portion of $K$ below the free
waterline. So with $0<s<1$ it means that
\[
\text{vol}(K_*)=s\cdot M
\]
Moreover, the vector $\mathfrak{o}-\mathfrak{o}^*$
is $\perp$ to the horizontal water line, i.e. parallell to the direction where
the force of gravity acts. Finally,
these two conditions are both necessary and sufficient for
a floating  position at rest.
That the two conditions are \emph{necessary} seem likely while the
\emph{sufficiency} is more subtle. The reason is that there may exist
\emph{several floating positions} where
the two conditions above hold.
For example, let
$K$ be a solid cube with $\mathfrak{o}$ placed in the center and 
constant density of mass which gives some 
specific
 weight
$0<s<1$.
The reader should discover that there exist several positions 
where Arhimedes' conditions hold. Take for example
$s=0.4$ and draw figures to find different solutions where
$\mathfrak{o}-\mathfrak{o}^*$ is vertical.
This leads to the question of \emph{stability}.
Stability conditions were found by
Christian Huyghens.
But his 
subtle  analysis  goes beyond
the scope of these notes. Let us only mention that when
$\Omega$ is a square the floating position where
the free waterline is parallel to a pair of   sides of
$\Omega$, then Huyghens proved that this floating position is stable
if and only 
$0<s<2-\sqrt{3}$ or  $\sqrt{3}-1<s<1$.
So we have an unstable 
equilibrium when $2-\sqrt{3}<s<\sqrt{3}-1$. To obtain
stable floating positions
$\Omega$ must be tilted.
For example when $s=\frac{1}{2}$
one gets 
a stable position where the sides of the cube  have the angle $\pi/4$ to the free waterline. 
\bigskip

\noindent {\bf 1.23 Proof of Archimedes' theorem}.
To begin with we must understand why the body can float at rest. This amounts
to determine the \emph{forces of lifting}
on the part of $K$ below the waterline.
Following Stevin - and the later refinement by Huyghens -
the force of lifting is found as follows. 
From the inside close to a point
$p\in\partial K_*$ placed at some distance
$h$ below the waterline
one makes a small circular hole 
of radius $\epsilon$. A cylinder of equal radius is 
pressed a small  bit $\ell$
outside the surface of the submarine.
The effect is that a volume of water equal to
$\pi\ell\epsilon^2$ is lifted to the free waterline. This requires 
a work equal to $gh\cdot\pi\ell\epsilon^2$.
Then, if $P$ is the force of \emph{pressure} on the submarine close to $p$
the work to push the cylinder  is equal to
$P\cdot \ell$. At the same time the area removed from the surface
of the submarine is $\pi\epsilon^2$. The result is that the 
infinitesmal lifting force becomes
\[
F=gh
\]
Next, the force of pressure at $p$  from the outside water on the surface of the submarine is parallell to the normal $\mathfrak{n}$
of $\partial K_*$
where the reader by the aid of a figure realizes that
one uses the \emph{inward normal}.
Thus, if $\mathfrak{n}$ denotes the \emph{outer normal} to
$\partial K_*$ the  discussion gives:
\medskip

\noindent {\bf 1.3 Proposition.} \emph{The total lifting force
on the floating body is}
\[ 
-g\cdot \int_
{\partial K_*}\,  h(p)\cdot \mathfrak{n}(p) d\sigma
\]
\emph{where $d\sigma$ is the area measure on
$\partial K_*$.}
\medskip

\noindent
Next, Archimedes' first principle 
asserts that
the vertical component of the vector valued integral
above is equal to
$g$ times the volume
of $K_*$.
Since
$z=0$ on the free waterline, this  gives
the equality
\[ 
\iint_{K_*}\, dxdydz=
\int_{\partial K_*}\, z\cdot\mathfrak{n}_z\cdot d\sigma
\]
where $h=-z$ above since we stay  below the free waterline.
As we shall see later on this equation corresponds to
Stokes applied
to the $z$-component of
the normal vector $\mathfrak{n}$. 





\medskip





\noindent {\bf 1.4 A vanishing result.} Since $K$ comes to rest it cannot behave like a
moving fish, i.e. the two horisontal
components of the total lifting force must be zero. This means that
\[
\int_{K_*}\,  z\cdot \mathfrak{n}_x\cdot  d\sigma=
\int_{K_*}\,  z\cdot \mathfrak{n}_y \cdot d\sigma=0\tag{1}
\]



\noindent Again we shall learn that these 
two area integrals are zero by Stokes formula.
Hence we have consolidated the first principle of Archimedes, and
conversely    this principle already
predicted  general  integration formulas  since
the shape of $K_*$ can be   arbitrary. 

\medskip


\noindent 
\emph{1.5 Proof that
$\mathfrak{o}-\mathfrak{o}^*$ is vertical.}
To show this we analyze the force of momentum.
We may assume that coordinates are chosen so that
$\mathfrak{0}=(0,b)$ for some $b$ on the $y$-axis.
The \emph{Law of Momentum}
gives
\[ \mathcal M=g\cdot
\int_{\partial K_*}\, (x,y-b)\times
(-y\mathfrak{n}(x,y))\cdot ds\tag{2}
\]
Here the minus sign for $y$ appears since the force of pressure  was
found via the distance $h$ from a point below the  water line up to $y=0$.
In (2) we
decompose
the vector
${\bf{n}}$ and expanding the
vector product it follows that:
\[
\mathcal M=g\cdot
\int_{\partial K_*}\, -xy\cdot \mathfrak{n}_y(x,y))\cdot ds-
\int_{\partial K_*}\, y(y-b)\cdot
\mathfrak{n}_x(x,y)\cdot ds
\]
\medskip

\noindent
Now Stokes formula entails that
\[
\cdot \int_{\partial K_*}\, y\cdot
\mathfrak{n}_x(x,y)\cdot ds=
\int_{\partial K_*}\, y^2\cdot
\mathfrak{n}_x(x,y)\cdot ds=0\implies
\]
\[ \mathcal M=
-g\cdot \int_{\partial K_*}\, xy\cdot \mathfrak{n}_y(x,y))\cdot ds\tag{3}
\]
By Stokes formula (3)  is equal to
the area integral 
\[ 
-g\int_{K_*}\,x dxdy\tag{*}
\]
This integral must be zero when the body is at rest. This means precisely that the
$x$-component of $\mathfrak{o}^*$ is zero and 
Archimedes' second assertion follows.

\bigskip












\centerline{\bf 1.5 Curvature and arc-length.}
\bigskip

\noindent
Arc length measure of curves in the $(x,y)$-plane and
the curvature occur frequently in complex analysis.
Following Huyghens we explain  how to
determine  curvature by  dynamical considerations.
Let a plane curve be defined by an equation $y=y(x)$ where
$y''(x)>0$ for $x>0$ and $y(0)=y'(0)=0$.
Up to translation and rotation this is a general
situation. So we have a convex curve  which we follow as 
$x$ increases.
To  express the curvature Huyghen's 
considered a particle of unit mass which
can slide on the curve, say on the side just above
the curve.
No gravity occurs, i.e. imagine that 
a vertical wall is placed along the curve which prevents the particle to leave
the curve. At time zero it has  velocity $v$.
No friction forces are present which
means that the force acting on the particle at every  moment is
directed along the normal to the plane curve.
Let $t$ be the time variable which gives   a time
dependent function $t\mapsto (x(t),y(x(t))$.
At a  moment $t$ we denote by $\rho(t)$  the 
reaction force on the particle, i.e. the force which 
keeps the particle to move on along the curve.
Our assumptions imply that  $\rho(t)$ is normal to the curve
and directed upwards in the
$y$-direction. Regarding a figure the
reader discovers that the components are given by:
\[ 
\rho_x(t)=\rho(t)\cdot\frac{-y'(x(t))}{\sqrt{1+y'(x(t))^2}}
\quad\colon\,
\rho_y(t)=\rho(t)\cdot\frac{1}{\sqrt{1+y'(x(t))^2}}
\]
Newton's Law that
"force=mass times acceleration" gives:
\[
\ddot x=\rho(t)\frac{-y'(x(t))}{\sqrt{1+y'(x(t))^2}}
\quad\text{and}\quad
\ddot y=\rho(t)\cdot\frac{1}{\sqrt{1+y'(x(t))^2}}
\]
Let us now notice that $\dot y=y'(x(t))\cdot\dot x$.
Hence
\[
\dot x\cdot\ddot x=
\rho(t)\frac{-\dot y}{\sqrt{1+y'(x(t))^2}}=-\dot y\dot\ddot y
\]
It follows that  $\dot x\ddot x+\dot y\ddot y=0$ which means that
$v^2=\dot x^2+\dot y^2$ is constant. This proves  the
preservation of kinetic energy which is  valid  since
no other forces than the normal pressure acts on the particle. 
\medskip

\noindent 
{\bf 1.6 Determination of $\rho$.}
To find $\rho$ we start with
$\dot y=y'(x(t))\cdot\dot x$ and  taking the time derivative once more
it follows that
\[
\ddot y=y''(x(t))\cdot\dot x^2+y'(x(t))\cdot \ddot x
\]
Inserting the two formulas above for $\ddot x$ and $\ddot y$
it follows that
\[
{\sqrt{1+y'(x(t))^2}}\cdot\rho(t)=
y''(x(t))\cdot \dot x^2
\]
Now we also have $v^2=\dot x^2+\dot y^2=
\dot x^2+(1+y'(x(t))^2\cdot \dot x^2$. We conclude that


\[
\rho(t)=\frac{y''(x(t))}{[1+y'(x(t))^2]^{\frac{3}{2}}}\cdot v^2
\]
\bigskip

\noindent
This is gives the formula for the \emph{centrifugal force}.
The term
\[ 
\mathfrak{c}(x)=
\frac{y''(x)}{[1+y'(x)^2]^{\frac{3}{2}}}
\] 
is the \emph{geometric curvature} of the plane curve.
Huyghen's conclusion was that the centrifugal force 
is the quotient of $v^2$ with the curvature expressed as above.
Having attained this one may give a geometric description of
$\mathfrak{c}(x)$. Namely, 
$\frac{1}{\mathfrak{c}(x)}$
is the radius of a circle placed along the normal to the curve
passing through $x$ which has \emph{best contact} with the curve at the point 
$(x,y(x))$. This geometric description of the curvature could of course
have been given from the start. But the dynamical consideration gives
a better insight and is  extremely important in mechanics.
Moreover, Huyghens clarified  why the geometric 
description must be valid by computing  the centrifugal force 
when a particle is constrained to move
along a circular wall  of some radius $R$. Namely, 
in this case
the centrifugal force is constant during the 
motion and given by
\[
\mathcal C= \frac{v^2}{R}\tag{*}
\]
\medskip

\noindent
{\bf 1.7 Huyghen's proof in the circular case.}
His  proof to obtain th centrifugal force
under a circular motion is extremely elegant. To begin with 
Huyhens  regards a particle which
moves along a \emph{regular polygon} with $N$ corners inscribed in the circle of radius $R$
with constant velocity $v$ and hits the circle $N$ times
during one full turn.
The \emph{impact force} each time the particle of unit mass
hits the circle is given by:
\[ 
2\cdot \text{sin}\frac{\pi}{N}\cdot v\tag{**}
\]
The reader should draw a figure and use that
the corners of the polygon give rise to $N$ many triangles
with angle $2\pi/N$ at the center and 
discover that the sudden direction of the velocity vector is changed
by
$2\pi/N$ at every impact. Then (**)  follows by decomposing
the reaction force and  the definition of the sine-function.
Next, the total length of the polygon is
$N\cdot 2\cdot\text{sin}\frac{\pi}{N}$.
Hence the time $T$ to perform a full circular turn is
\[ 
T=\frac{N\cdot 2\cdot\text{sin}\frac{\pi}{N}}{v}
\]
Finally, we have $N$ many instants when impact takes place. So
after one circular turn we get
\[ 
F_{\text{imp}}=N\cdot 
2\cdot \text{sin}\frac{\pi}{N}\cdot v\implies
\frac{F_{\text{imp}}}{T}=
\frac{v^2}{R}
\]

\noindent
The left hand side expresses
the effect of impact while the particle impinges the circle at
the corner points and the effect of force per unit time does 
not depend on $N$. When $N\to\infty$ we 
get the "continuous formula" expressed by
(*) above.













\newpage










\centerline{\bf 2. Stokes Theorem in $\bf R^2$}
\bigskip

\noindent
{\bf Introduction}
\emph{Stokes Theorem} 
is often proved in an "intuitive fashion"
where figures  illustrate how one divides a  domain into simpler so that
repeated double integrals can be used.
We     give a proof without such artificial constructions. 
In the long run this is
essential since the cutting  of domains becomes messy
when the number of its boundary components increases.
This section may appear to be  "overkilling"
for the beginner. But my opinion is that the subsequent material  belongs to  the
foundation for complex analysis and learning a
proof 
of Stokes Theorem  is both important
and instructive. In � X from  the appendix about measure theory
we derive Stokes theorem in every dimension $n\geq 2$. Here we present the results when $n=2$.










\bigskip

\centerline{\bf\large 2.A. The case of graphic domains}

\bigskip

\noindent
The FCT = fundamental theorem of calculus -
asserts that if $g(x)$ is a function 
whose derivative exists as a continuous function, then
\[ 
g(x)=g(a)+\int_a^x\, g'(t)dt
\]


\noindent
Next, consider ${\bf{ R}}^2$ where $(x,y)$ are the coordinates.
A real valued function $f(x,y)$ is of class $C^1$ when the two partial derivatives
$f_x$ and $f_y$ both exist as continuous functions.
Consider a $C^1$-function $\phi(x)$ which depends on $x$ only
and is defined on a closed interval $0\leq x\leq A$. Assume that
$\phi(x)>0$ which gives the open set
\[
\Omega=\{(x,y)\colon\,0<x<A\quad\, 0<y<\phi(x)\}
\]
Now we  have the  double integrals
\[
\iint_\Omega\,f_x\cdot dxdy\quad\colon\quad
\iint_\Omega\,f_y\cdot dxdy\tag{*}
\]
We shall express these by certain line integrals. 
The second double integral is easy to handle
since  it is 
a repeated integral:
\[
\iint_\Omega\,f_y\cdot dxdy=\int_0^A\,[\int_0^{\phi(x)}\,
f_y(x,y)dy] \,dx=\int_0^A\,[f(x,\phi(x)-f(x,0)]dx\tag{1}
\]
\medskip

\noindent
Later  we  explain the intrinsic nature of this formula.
Let us turn to the double integral in the left hand side of (*).
Here we cannot find a  repeated integral when 
horizontal lines $\{x=a\}$  cut the domain $\Omega$
so that the sets $\Omega\cap\{x<a\}$ and $\Omega\cap\{x>a\}$ 
have several connected components. The reader should illustrate this by drawing some figures using an "ugly" $\phi$-function.
However, we can express 
the double integral as a sum of line integrals !
\medskip

\noindent
{\bf 1. A clever construction of  line integrals.} Put
\medskip

\noindent
\[
J(f)=\iint_\Omega\, f_x\cdot dxdy
\]
Consider the following function
$\psi$ of a single variable
\[
\psi(x)=\int_0^{\phi(x)}\, f(x,y)dy\implies
\]
\[ 
\psi'(x)=f(x,\phi(x))\phi'(x)+
\int_0^{\phi(x)}\, f_x(x,y)dy
\]
The FTC gives
\[ 
\psi(A)-\psi(0)=
\int_0^A\, \psi'(x)dx=
\int_0^A\, f(x,\phi(x)\phi'(x)+
\int_0^A[\int_0^{\phi(x)}\, f_x(x,y)dy\,]\cdot dy
\]
The last term is $J(f)$ and hence we get
\[
J(f)=\int_0^{\phi(A)}\, f(A,y)dy-\int_0^{\phi(0)}\, f(0,y)dy\
-\int_0^A\, f(x,\phi(x)\phi'(x)dx\tag{*}
\]
In (*) three line integrals appear. The first is taken along the vertical line 
$x=A$, the second along $x=0$  in the negative direction and
the last  along the curve $y=\phi(x)$.
Now we explain  their
geometric meaning.
\medskip

\noindent
\emph{First}, the vertical line $L_+=\{x=A\quad 0\leq y\leq \phi(A)\}$ is  part of 
the boundary
$\partial\Omega$. On this line the \emph{arc-length measure} is equal to $dy$
and 
 \emph{the outward normal} along $L_+$ is 
parallell to the $x$-axis so its component $n_x=1$. Hence we can write
\[
\int_0^{\phi(A)}\, f(A,y)dy=\int_{L_+}\, f n_x\cdot  ds\tag{1}
\]
\emph{Second}, consider $L_-=\{x=0\quad 0\leq y\leq \phi(0)\}$.
Again $dy$ is the arc-length measure while  the outer normal 
is directed in the negative $x$-direction, i.e.  $n_x=-1$.
Hence:
\[
-\int_0^{\phi(0)}\, f(0,y)dy=\int_{L_-}\, fn_x\cdot dy\tag{2}
\]
\emph{Third}.  On the curve $\Gamma=\{y=\phi(x)\}$ the arc-length is 
$ds=\sqrt{1+\phi'(x)^2}\cdot dx$ and
the $x$-component of 
the outer normal is 
\[
n_x=\frac{-\phi'(x)}{\sqrt{1+\phi'(x)^2}}
\]
Here the minus sign  becomes clear by inspecting a figure of the curve 
$y=\phi(x)$ where you discover that $n_x>0$ if
$\phi'<0$ and vice versa !
Hence we obtain
\[
-\int_0^A\, f(x,\phi(x)\phi'(x)dx=
\int_\Gamma\, f n_x ds\tag{3}
\]
where the first minus sign  reflects the sign-rule for $n_x$ along the boundary curve $\Gamma$.
Putting all this together we have proved the equality
\medskip
\[
J(f)=\int_\Gamma f n_x ds+
\int_{L_+} f n_x ds+\int_{L_-} f n_x ds\tag{**}
\]
No "ackward signs" occur in the line integrals above since
arc-length measure is always defined and  the
outer normal along the boundary of an open set is clarified by a picture.
There remains the portion of $\partial\Omega$ defined by
$I=\{ 0\leq x\leq A\quad y=0\}$.
Here  the outer normal is in the negative 
$y$-direction and  hence $n_x=0$. Therefore we only add a zero term by
$\int_I\, f n_x ds$ and
arrive at
\medskip

\noindent {\bf 2. Theorem. } \emph{One has
the equality}
\[
\iint_\Omega\, f_x\cdot  dxdy=\int_{\partial\Omega} f n_x\cdot  ds
\]
\medskip

\noindent
{\bf 3. An area formula.} 
Let $\phi(x)$ be  a piecewise linear function which is positive when
$0<x<A$  while $\phi(0)=\phi(A)=0$.
So we have corner points 
\[ p_\nu=(x_\nu,y_\nu)\,\quad 0=x_0<x_1\ldots <x_N=A
\]
while the $y_1,\ldots,y_{N-1}$ is any sequence of positive numbers.
Apply Theorem 2 with $f(x,y)=x$. The double integral is the area of 
$\Omega$, i.e. the domain bounded by the piecewise linear curve
$\Gamma$ and the interval $[0,A]$ on the $x$-axis. On the line segment
$\ell\uuu\nu$ which joins
$(x\uuu\nu,y\uuu\nu)$ with
$(x\uuu{\nu+1},y\uuu{\nu+1})$ the reader may verify by figure and
Pythagoras's theorem that along  $\ell\nu$ one has:

\[
n_x=\frac{y_\nu-y_{\nu+1}}
{\sqrt{(x_{\nu+1}-x_\nu)^2+
(y_{\nu+1}-y_\nu)^2}}
\quad\text{and}\quad
ds=\frac{\sqrt{(x_{\nu+1}-x_\nu)^2 +(y_{\nu+1}-y_\nu)^2}}{x\uuu{\nu+1}\vvv x\uuu\nu}\cdot dx
\]
It follows that

\[
\int\uuu{\partial\Omega}\, xn\uuu x\cdot ds=
\sum\, \frac{y\uuu\nu\vvv y\uuu{\nu+1}}{x\uuu{\nu+1}\vvv x\uuu\nu}\cdot
\int\uuu{x\uuu\nu}^{x\uuu{\nu+1}}\, x\,dx
=
\]
on each linear piece of $\Gamma$. 
After a summation  the line integral becomes
\[
\frac{1}{2}\cdot \sum_{\nu=0}^{\nu=N-1}\, 
(x_{\nu+1}^2-x_\nu^2)(y_\nu-y_{\nu+1})=\frac{1}{2}\cdot
\sum\, (y_\nu-y_{\nu+1})\cdot (x_{\nu+1}+x_\nu)\tag{*}
\]

\noindent
{\bf{Example.}}
Consider  the simplest case
where just one corner point appears at $(A/2,H)$.
So here $\Omega$ is a triangle whose  area is  $AH/2$. The formula gives on the other hand 
\[
\frac{1}{2}\cdot[\vvv H\cdot A/2+H\cdot (A+A/2))= AH/2
\] 
which confirms (*).
The reader should continue to test the area formula (*) in 
more complicated situations where the picewise linear curve has several negative and positive slopes.
\medskip
 
\noindent 
{\bf 4. Expression of the first double integral}
Consider the component $n_y$ of the outer normal to $\Omega$.
On the piece $I=\{0\leq x\leq A\colon\quad y=0\}$ we see that $n_y=-1$.
On the curve $y=\phi(x)$  a picture shows that
\[ 
n_y=\frac{1}{\sqrt{1+\phi'(x)^2}}
\]
\medskip

\noindent 
At the same time
$dx=\frac{1}{\sqrt{1+\phi'(x)^2}}ds$ and hence 
$n_yds=dx$ holds on the curve. Finally, the outer normal is in the $x$-direction 
on the two vertical lines of $\partial\Omega$. Hence we have proved:

\medskip

\noindent 
{\bf 5. Theorem.}
\emph{One has}
\[
\iint_\Omega\,f_y\cdot dxdy=\int_{\partial\Omega}\,
f n_y \cdot ds
\]
\medskip

\noindent
Thus, we have a similar formula as in Theorem 2 for the $y$-coordinate.
There remains to extend these two formulas to 
general domains which even may have several disjoint closed boundary curves.
But first we  resume our special case a bit further.
Consider an open cube $\square$ in ${\bf{ R}}^2$ where we after a translation may 
assume that it is centered at the origin.
Let $\psi(x,y)$ be a $C^1$-function whose gradient is $\neq 0$ at the origin and
to make a choice we assume that $\psi_y(0,0)<0$.
The \emph{implicit function theorem} gives
a $C^1$-function $\phi(x)$ and a positive function $h(x,y)$ such that
\[
\psi(x,y)=(\phi(x)-y)h(x,y)
\]
Shrinking $\square$ if necessary we may assume that $h>0$ in the whole of
$\square$. Put
\[
\Omega=\square\cap\,\{\psi>0\}=
\square\cap\,\{y<\phi(x)\}
\]
The previous results show that if $f$ is a $C^1$-function with a compact support in 
$\square$, then the two FCT-formulas hold for $\Omega$.
We refer to $\Omega$ as a \emph{graphic domain}. Next, 
the 
validity of the FCT-theorem
is  obviously invariant under a \emph{linear change of coordinates}. Hence
we   can start with a cube
whose sides are not parallell to the coordinate axis and use some direction of the gradient of $\psi$ when the implicit function theorem is used to obtain a graphic domain.
With  this kept in mind we  begin the proof in the general case.
\bigskip





\centerline{\bf {6. The case of domains in $\mathcal D(C^1)$.}}
\medskip




\noindent 
First we give

\medskip

\noindent
{\bf 7. Definition.} \emph{A bounded open and connected
subset $\Omega$ of ${\bf{ R}}^2$ has a 
$C^1$-boundary  if $\partial\Omega$ is the disjoint union of a finite family of 
simple and closed curves $\Gamma_1,\ldots,\Gamma_k$
each of which are of class $C^1$. 
The family  of   such domains is denoted by 
$\mathcal D(C^1)$.}
\bigskip

\noindent {\bf 8. Remark about the arc-length.}
A simple closed curve $\Gamma$ of class $C^1$ is  the image of a vector-valued function
\[
t\mapsto \gamma(t)= (x(t),y(t))\quad 0\leq t\leq T
\]
which is 1-1 except that $\gamma(0)=\gamma(T)$.
Moreover, the functions $x(t)$ and $y(t)$ are both of class $C^1$ and
$\gamma(t)$ is "moving" which means that
$\dot\gamma(t)\neq 0$ for all $t$, or equivalently 
$\dot x^2(t)+\dot y^2(t)>0$ for all $0\leq t\leq T$.
The curve $\Gamma$ can be parametrized in several ways. Among those one has the 
\emph{parametrization by arc-length}. In this case we use $s$ as parameter and then 
\[ 
(\frac{dx}{ds})^2+
(\frac{dy}{ds})^2=1\quad\colon\, 0\leq s\leq L
\]
where $L$  is the total arc-length of $\Gamma$.
The arc-length measure along $\Gamma$ is denoted by $ds$.
\medskip

\noindent 
{\bf  9. The normal to $\Gamma$.} Let $\Gamma$ be a $C^1$-curve.
To each point $p\in\Gamma$ we find a unit vector 
$n(p)$ which is normal to $\Gamma$. Given a parametrization by
arc-length we have:
\[
n=(n_x,n_y)\quad\colon\quad n_x=\frac{dy}{ds}\quad\text{and}\quad
n_y=-\frac{dx}{ds}
\]
\medskip
\noindent
However, the \emph{sign} of the normal depends on the chosen 
\emph{orientation} of $\Gamma$.
For example, let $\Gamma$ be a circle of radius $R$ centered at the origin.
Its arc-lengtth is $2\pi R$ and it has the parametrisation
\[ 
s\mapsto (R\cos(s),R\sin(s))\quad 0\leq s\leq 2\pi
\]
Given this  orientation we see that
\[ 
n_x=\cos(s)\quad n_y=\sin(s)
\] 

\bigskip

\noindent 
Now we can announce the fundamental result called the FTC in dimension 2.
\bigskip

\noindent
{\bf 10. Theorem.} \emph{Let $\Omega\in\mathcal D(C^1)$. For each 
function $f(x,y)$ of class $C^1$ one has }
\[ 
\iint_\Omega\, f_x\cdot dxdy=
\int_{\partial\Omega}\, f n_x\cdot  ds\quad\colon\quad
\iint_\Omega\, f_y\cdot dxdy=
\int_{\partial\Omega}\, f n_y\cdot  ds
\]

\noindent
\emph{where $n$ is the outward normal to $\Omega$ along each 
boundary curve.}
\bigskip

\noindent
{\bf Proof.} The theorem asserts that the formula
holds for every $C^1$-function. Following the
proof in the introduction by  D�scartes and Newton
we 
\emph{decompose} 
a given $C^1$-function $f$ and represent it by a sum of $C^1$-functions
where each function of the sum has a small support which enable us to apply the previous result for \emph{graphic domains}.
To achieve such a decomposition we first consider some boundary point 
$p\in\partial\Omega$. It belongs to some boundary curve $\Gamma$. 
Let $L$ be the tangent line to $\Gamma$ passing through $p$.
Then we construct a small square $\square(p)$ centered at $p$
and with two sides parallel to $L$ while the other are $\perp$ to $L$.
Here $\Gamma$ is locally defined by an equation $\phi(x,y)=0$ close to $p$
where $\phi$ is chosen so that 
\[ 
\Omega\cap\,\square(p)=\{\phi>0\}
\]
Regarding a picture - which the reader should draw - it is clear that
$\Omega\cap\,\square(p)$ is a graphic domain if the 
sides of $\square(p)$ are sufficiently
small.
By  the \emph{Heine-Borel Lemma} we can cover 
the compact boundary $\partial\Omega$
by a finite set of such squares, say $\square(p_1),\ldots,
\square(p_N)$. Notice that $p_1,\ldots,p_N$ are boundary points chosen from
different boundary curves but the cubes are so small that
\[ 
\square(p_\nu)\cap\,
\square(p_j)=\emptyset\quad\colon\quad \, p_\nu,p_j\,\,\, \text{belong to different boundary curves}
\]
\medskip  
\noindent
Next, consider the complementary set
\[
K=\Omega\setminus\,\cup\,\square(p_\nu)
\]
This becomes a compact subset of $\Omega$. By the \emph{Heine-Borel Lemma}
it can
be covered by a finite set of open cubes
$W_1,\ldots,W_m$ chosen so small that each closure $\bar W_i$ stays inside
$\Omega$.
\medskip

\noindent 
\emph{Partition of the unity.}
We have the finite family of cubes $\{\square(p_\nu),W_j\}$.
By the result in XXX there exist $C^1$-functions 
$g_1,\ldots,g_N,h_1,\ldots,h_m$ such that
\[ 
\sum g_\nu+\sum h_i=1\quad\,\colon\quad 
\text{Supp}(g_\nu)\subset\square(p_\nu)\quad
\text{Supp}(h_i)\subset W_i 
\]
where the first equality holds in some open neighborhood
$U$ of $\bar\Omega$.

\bigskip

\noindent 
\emph{Final part of the proof.}
We treat the first formula expressing the double integral of $f_x$ by line integral. Replacing $x$ by $y$ is  proved in the same way.
Given a $C^1$-function $f$
it is expressed by a sum:
\[ 
f=\sum\, fg_\nu+\sum\, fh_i
\]
Since the sum of partial derivatives $\partial_x(\sum g_\nu+\sum h_i)$ and
$\partial_y(\sum g_\nu+\sum h_i)$ both vanish  we get
\[
\iint_\Omega\, f_x\cdot dxdy=
\sum\iint_\Omega\, (fg_\nu)_x\cdot dxdy+
\sum\iint_\Omega\, (fh_i)_x\cdot dxdy
\]
and similarly 
\[ 
\int_{\partial \Omega}\, f n_x \cdot ds=
\sum \int_{\partial \Omega}\, fg_\nu n_x \cdot ds+
\sum \int_{\partial \Omega}\, fh_i n_x \cdot ds
\]
Hence it suffices to prove Theorem 10 for each term separately, i.e. for 
the functions $fg_\nu$ or $fh_i$. Here $fg_\nu$ has compact support in
$\square(p_\nu)$ and the graphic case applies, i.e. the required formula 
in Theorem 10 holds in this case. Next, consider $fh_i$. 
This function has compact support in
$\Omega$ so no boundary terms, i.e. no line integral appears. But at the same
time XX shows that
\[ 
\iint_\Omega\, (fh_i)_x\cdot dxdy=0
\] 
So these double integrals give no contribution and 
Theorem 10
is proved.



\bigskip



\centerline {\bf 11. Boundary with corner points}
\medskip


\noindent
Consider a  connected and bounded open set $\Omega$. 
Its boundary is a compact set. A point $p\in\partial\Omega$
is called regular if there exists a small disc $D$ centered at $p$
and a $C^1$-function $\phi$ in $D$ whose gradient vector is $\neq 0$ in $D$ such that
\[ 
\Omega\cap D=\{\phi<0\}\quad\colon\,\quad \phi(p)=0
\]
It is obvious that the set of regular points is an open 
subset of $\partial\Omega$ to be denoted by
$\text{reg}(\partial\Omega)$.
We assume that the set is non-empty,
i.e. we ignore to consider open sets with a very ugly boundary.
On the regular part of $\partial\Omega$ the arc-length measure $ds$ and the outer normal are defined.
Next, put 
\[
\sigma_\Omega=\partial\Omega\setminus
\text{reg}(\partial\Omega)
\]
So this is a compact set and we shall
impose a condition  on its size.
\medskip  

\noindent
{\bf 12. Federer's conditions.} First one requires that
the two projected images of $\sigma\uuu\Omega$
on the $x$-line respectively the $y$-line
are null sets
in the sense of Lebesgue. 
Thus, the condition is that to every $\epsilon>0$ there exists a finite family of disjoint open intervals
$J_\nu=(a_\nu,b_\nu)$ on the $x$-line such that
\[
\sum\,(b_\nu-a_\nu)<\epsilon\quad\colon\,\quad
(x,y)\in\sigma_\Omega\implies x\in\,\cup J_\nu
\]
and with a similar condition for the $y$-interval.
The second condition is  that the 
arc-length of the regular part is finite, i.e. that
\[ 
\int_{\text{reg}(\partial\Omega)} ds<\infty
\]
\medskip



\noindent
{\bf 13. Theorem} \emph{Assume that $\partial\Omega$ satisfies Federer's conditions.
Then}
\[ 
\iint_\Omega f_y \,dxdy=
\int_{\text{reg}(\partial\Omega)}\, f n_y \cdot ds
\] 
\emph{and similarly with $x$ replaced by $y$.}
\medskip

\noindent
\emph{ Proof.}
Let $\epsilon>0$ and choose intervals $J_\nu$ of total length $<\epsilon$ to satisfy Federer's condition for $x$-coordinates of points in $\sigma_\Omega$.
Choose a $C^1$-function $g(x)$ where $0\leq g\leq 1$
and
$g=1$ outside $\cup \,J_\nu$ while $g(x)=0$ in a neighbourhood
of the compact set 
\[
\{x\colon\,\exists\,y\colon \,(x,y)\in\sigma_\Omega\}
\]
Set
\[
h=gf
\]
By the choice of $g$
the support of $h$ avoids $\sigma_\Omega$.
Repeating the  proof of Theorem 10 for domains with
regular boundaries we obtain
\[
\iint_\Omega h_y dxdy=
\int_{\text{reg}(\partial\Omega)}\, h n_y ds\tag{1}
\]
Here $h_y=g(x)f_y$ and hence
\[
\iint_\Omega f_y dxdy-
\iint_\Omega h_y dxdy=
\iint_\Omega (1-g(x))f_y(x,y) dxdy\tag{2}
\]
The last double integral is estimated as follows.
By the choice of $g$ we have $1-g=0$ outside
a union of intervals of length $\leq\epsilon$.
So if $M$ is the maximum norm of $f_y$ taken over $\bar\Omega$ and 
$L$ is the maximum of two $y-$-coordinates for 
points in $\Omega$ with the same $x$-coordinate,
then we get
\[
|\iint_\Omega (1-g(x))f_y(x,y) dxdy|\leq ML\epsilon\tag{3}
\]
\

\noindent 
{\bf 14. Estimate of 
$\int_{\text{reg}(\partial\Omega)}\, f(1-g) n_y ds$.}
Here we  need a more delicate argument where the reader - as always when it comes to a more involved proof in analysis - should make suitable
pictures to discover the geometry. 
Let $\delta>0$ and consider the subset 
$W(\delta)$ of
$\text{reg}(\partial\Omega)$ where $|n_y|\geq\delta$.
On this set we notice that the arc-length is majorised by $|dx|$:
\[
|n_y|\geq\delta|\implies ds\leq\frac{1}{\delta}|dx|\tag{i}
\]
Moreover, the projection
$(x,y)\to x$ restricted to $W(\delta)$ has \emph{discrete fibers}, i.e. it is locally 1-1 as you see by drawing a figure with a small curve passing through any point in 
$W(\delta)$.
However, the set $W(\delta)$ need not be compact so we must 
perform another reduction. Namely, by hypothesis the total length of 
$\text{reg}(\partial\Omega)$ is finite. So with $\epsilon>0$ we can find a 
\emph{compact} set $K\subset\text{reg}(\partial\Omega)$
such that 
\[ 
\int_{\text{reg}(\partial\Omega)\setminus K}\, ds<\epsilon\tag{ii}
\]
Next, restrict the projection map $\pi$ defined by
$(x,y)\to x$ to $W(\delta)\cap K$.
Since this set 
is \emph{compact} and the projection is locally 1-1,
\emph{Heine-Borel Lemma} gives
an integer $M_\delta$ such that
the inverse fibers
\[
\pi^{-1}\cap W(\delta)\cap K\tag{iii}
\]
contain at most $M_\delta $ points for every $x$. 
Notice  that $M_\delta$   depends on $\delta$ but not upon $\epsilon$.
Using the above we obtain
\[
\int_{W(\delta)\cap K}\, (1-g)ds\leq 
\frac{1}{\delta}\int_{W(\delta)\cap K}\, (1-g)d|x|\leq 
\frac{M_\delta}{\delta}\cdot \int(1-g) dx\leq \frac{M_\delta}{\delta}\cdot\epsilon\tag{*}
\]
where the last inequality follows from the choice of $g$ in XX above.
We have also to other estimates. Let $|f[_K$
be the maximum norm of $f$ over $K$.
On
On $K\setminus W(\delta)$ we have $|n_y|\leq\delta$ and hence
\[
|\int_{K\setminus W(\delta)}\,f(1-g)n_y\cdot ds|\leq
\delta\cdot \int_{K\setminus W(\delta)}\,f(1-g)ds\leq \delta\cdot |f|_K\cdot
\int_{\partial\Omega_{\text{reg}}}ds
\tag{**}
\]  
Next, using (iii) above we have
\[
|\int_{[\text{reg}(\partial\Omega)\setminus K]}\, f(1-g)ds|\leq
|f|_{\partial\Omega}\cdot 
\int_{[\text{reg}(\partial\Omega)\setminus K]}\, ds\leq |f|_{\partial\Omega}\cdot 
\epsilon\tag{***}
\]




\noindent
Finally, notice that
\[ 
\text{reg}(\partial\Omega)=[ W(\delta)\cap K]\cup\,
[\text{reg}(\partial\Omega)\setminus K]\cup [K\setminus W(\delta)]\tag{v}
\]
Putting all this together we obtain the inequality

\[
|\int_{\text{reg}(\partial\Omega)}\, f(1-g) n_y ds|\leq
\frac{M_\delta\epsilon}{\delta}+A\cdot |f|_K\cdot \delta+|f|_{\partial\Omega}\cdot \epsilon\tag{vi}
\]
Here (vi) hold for all pairs $\delta,\epsilon)$.
To finish the proof
of Theorem 13 we choose an arbitrary small$\delta>0$ and
after $\epsilon$ is chosen so small that we first
have $\epsilon\leq\delta$ and also

\[
\frac{M_\delta\epsilon}{\delta}<\delta
\]
Then the left hand side is majorised by
\[ 
(1+A\cdot |f|_{\partial\Omega}+|f|_{\partial\Omega})\cdot\delta
\]
Since $\delta>0$ is arbitrary we get
Theorem 13.


\newpage

\centerline{\bf\large 3. Line integrals via differentials}

\bigskip

\noindent 
In  � 2 
arc-length and the outer normal were used to construct line integrals. 
One may also introduce the  differentials $dx$ and $dy$.
The construction of the outer normal shows that
\[
n_x\cdot ds=dy\quad\, n_y\cdot ds=-dx\tag{*}
\]
Hence one can express  Stokes Theorem  in the form
\[ 
\iint_\Omega\, f_x \cdot dxdy=\int_{\partial\Omega}\, fdy
\quad\colon\quad
\iint_\Omega\, f_y \cdot dxdy=-\int_{\partial\Omega}\, fdx
\]
\medskip


\noindent {\bf A Warning.}
When Stokes Theorem
is expressed in this way one must be careful with the orientation. 
The \emph{rule of thumbs} is used whenever $\partial\Omega$ borders an open set.
Personally I prefer to express line integrals by $n_x ds$ or $n_y ds$ 
since the geometric picture becomes  transparent.
However,  differentials have  an   advantage when
calculus is performed on \emph{manifolds} rather 
than the euclidian plane 
${\bf{ R}}^2$, since here arc-length and normal derivatives are not even 
defined until
the manifold has been equipped with a metric.
Thus, complex analysis on \emph{Riemann surfaces}
employs  differential forms.
\bigskip


\noindent
{\bf 3.0 Transformation laws.}
Let $(\xi,\eta)$ be the coordinate functions in another copy
of ${\bf{R^2}}$.
Let $\Omega$ be a domain in the $(x,y)$-plane and
consider a bijective map
\[
Q\colon\,(x,y)\to(\phi(x,y),\psi(x,y))\quad\colon\xi=\phi(x,y)\quad
\eta=\psi(x,y)\tag{1}
\]
defined in some neighborhood of $\bar\Omega$.
Here $\phi$ and $\psi$ are $C^1$-functions and we get the domain
$Q(\Omega)$ in the $(\xi,\eta)$-space.
Now we have
\[ 
d\xi=\phi_x\cdot dx+\phi_y\cdot dy\quad\colon\,
d\eta=\psi_x\cdot dx+\psi_y\cdot dy\quad\colon\tag{2}
\]
The \emph{Jacobian} of the $Q$-map is defined by the equation
\[
\mathcal J=\phi_x\cdot\psi_y-\phi_x\cdot\psi_y\tag{3}
\]
The $Q$-map preserves orientation when
$\mathcal J>0$ and from now on this is assumed.
The Jacobian changes area which is expressed by
\[ 
d\xi d\eta=J\cdot dxdy\tag{3}
\]


\noindent
Now we take some $C^1$-function $g(\xi,\eta)$ defined in the
$(\xi,\eta$-space.
In the $(x,y)$-space we get the function
\[ 
g_*(x,y)= g(\phi(x,y),\psi(x,y))\tag{4}
\]


\noindent
Let us  study the effect of the transformation when Stokes formula is
applied. Put $\Omega^*=Q(\Omega)$. We have the obvious equality:
\bigskip
\[
\int_{\partial \Omega^*}\, gd\xi=
\int_{\partial\Omega}\, g_\ast\cdot(\phi_xdx+\phi_y dy)\tag{5}
\]
Stokes formula applied to the right hand side gives
\[
\iint_\Omega\,
-(g_\ast\phi_x)_y+(g_\ast\phi_y)_x\cdot dxdy\tag {6}
\]
\medskip

\noindent
Here we get a cancellation since the mixed derivatives
$\phi_{xy}$ and $\phi_{yx}$ are equal. Hence (6) becomes:
\[\iint_\Omega\,
\,\bigl[-(g_\ast)_y\cdot\phi_x+(g_\ast)_x\cdot \phi_y\bigr ]\cdot dxdy\tag{7}
\]
\medskip

\noindent
If Stokes formula is applied to the left hand side in
(5) we get the area integral
\[
\iint_{\Omega^*}\, -g_\eta\cdot d\xi d\eta\tag{8}
\]
\medskip

\noindent
{\bf{3.1 Theorem.}} \emph{The integrals (7) and (8) are equal.}


\medskip

\noindent 
\emph{Proof.}
The equality follows using
transformation rules for
partial derivatives. Namely, from (4) we get
\[
(g_*)_x=\phi_x\cdot g_\xi+\psi_x\cdot g_\eta
\quad\colon\,
(g_*)_y=\phi_y\cdot g_\xi+\psi_y\cdot g_\eta\tag{9}
\]
Here we can solve out $g_\eta$ and find
\[
(\phi_x\psi_y-\phi_y\psi_x)\cdot g_\eta=
(g_*)_y\cdot\phi_x-
(g_*)_x\cdot\phi_y\tag{10}
\]


\noindent
In (10) we discover the Jacobian
as a factor for $g_\eta$.
Hence the  rule for area transformation in (3) and  the two minus signs
in (7) and (8) show that  
(7)=(8).
\bigskip

\noindent
{\bf 3.2 The pull-back of differential forms.}
The efficient way to analyze transforms  which   can be
extended to maps between manifolds goes as follows:
Let
\[ 
Q\colon\,(\xi,\eta)\mapsto (x,y)\quad\colon\, x=\phi(\xi,\eta)\quad
y=\psi(\xi,\eta)\tag{1}
\]


\noindent
The differential 1-forms $dx$ and $dy$
have inverse images  defined by
\[
(dx)^*=\phi_\xi\cdot d\xi+\phi_\eta\cdot d\eta\quad\colon\,\,\,
 (dy)^*=\psi_\xi\cdot d\xi+\psi_\eta\cdot d\eta\tag{2}
\]



\noindent
More generally,  to each pair of $C^1$-functions
$A(x,y),B(x,y)$ we get the
1-form
$\alpha=A(x,y)\dot dx+B(x,y)\cdot dy$. Its pull-back is
\[ 
\alpha^*=A^*(\xi,\eta)\cdot(dx)^*+
B^*(\xi,\eta)\cdot(dy)^*\tag{3}
\]
where $A^*(\xi,\eta)= A(\phi(x,y),\psi(x,y))$
and similarly for $B^*$.
\medskip

\noindent
{\bf 3.3 Exterior differentials.}
If $\alpha=A\cdot dx+B\cdot dy$ is a 1-form its exterior differential
is defined as
\[ 
d\alpha=-A_y\cdot dx\wedge dy+B_x\cdot \wedge dy\tag{4}
\]
\medskip

\noindent
The minus sign in front of $A_y$ is compatible with
Stokes formula is expressed in (3.0),i.e. we get
\[
\iint_\Omega\, d\alpha=\int_\Omega\,\alpha\tag{5}
\]
\medskip

\noindent
The formula in (5) summarizes the whole content of
Stokes formula.
It becomes especaily useful because of the following fundamental fact.
\bigskip

\noindent {\bf 3.4 Theorem.}
\emph{The pull-back of differential forms commutes
with exterior differential.}
\bigskip

\noindent
{\bf 3.5 Remark.} Theorem 3.4  asserts that
if we start from $\alpha$
and get $\alpha^*$ then
the 2-form $d(\alpha^*)$ in the $\xi,\eta)$-space is equal to
the pull-back of the 2-form $d\alpha$.
The reader should verify this or consult some text-book in calculus.
Passing to Stokes formula the result is
that if $\alpha$ is a 1-form in the $(x,y)$-space
then one has equality for  the area integrals
\[
\iint_\Omega\, d\alpha=\iint_{\Omega^*}\, d\alpha^*\tag{6}
\]
\bigskip

\noindent
{\bf 3.6 Currents.}
Above we recalled classic notions which  extend to
manifolds in dimension 2 and using some calculations with 
multi\vvv linear forms
one introduces differential forms of higher
degree  to manifolds in any dimension.
However, the classic approach has a drawback since
an equality like (6) assumes that one has a 1-1 map from  the
$(\xi,\eta)$-plane  into the $(x,y)$-plane.
The modern procedure
is to use distribution theory.
For example, consider a map
\[
Q\colon (\xi,\eta)\mapsto (\phi(x,y),\psi(x,y))
\]
where
$\phi$ and $\psi$ are $C^\infty$-functions.
Here the map $Q$  need not be 1-1.
Let $\gamma$ be a Jordan arc in the $(\xi,\eta)$-space. For example, an 
interval on some circle or a line segment.
The image set $Q(\gamma)$ can be a curve with self-intersections and so on.
A typical case is that for 
points $p\in Q(\gamma)$ the inverse set
$Q^{-1}(p)\cap\gamma$ is a finite set of points on
$\gamma$ but  the number may change as $p$
moves in $Q(\gamma)$.
So one should regard the image of $\gamma$ under the $Q$-map as a \emph{current}
acting as a linear form on 1-forms in the
$(x,y)$-space by the rule
\[
\alpha\mapsto \int_\gamma\,\alpha^*\tag {1}
\]
\medskip

\noindent The point is  that the pull-back $\alpha^*$ is defined even
if $Q$ is not 1-1, i.e. it is  given
by
\[ 
A^*(\xi,\eta)\cdot[\phi_x^*\cdot d\xi+\phi_x^*\cdot d\eta]+
B^*(\xi,\eta)\cdot[\psi_x^*\cdot d\xi+\psi_x^*\cdot d\eta]\tag{2}
\]
\medskip

\noindent
The current defined by (1) is denoted by $Q_*(\gamma)$ and called the
direct image of the integration current defined by
$\gamma$ in the $(\xi,\eta)$-space.
Here it is essential that we have given  an orientation on
$\gamma$ its direct image current is constructed.
The current $Q_*(\gamma)$ has distribution coefficients when
we specialize the 1-form $\alpha$. That is, there exists a map
\[ 
A(x,y)\in C^\infty({\bf{R^2}})\mapsto
\int_\gamma\, A^*(\xi,\eta)\cdot (dx)^*
\] 

\bigskip

\noindent
{\bf 3.7 Stokes formula in higher dimension}
For readers who  already are a bit familiar with 
differential geometry, the FCT in any dimension goes as follows: Let $X$ be an oriented manifold of some dimension $n\geq 2$ and of class $C^2$ at least.
Let $V$ be a locally closed and oriented submanifold 
of some dimension $1\leq k\leq n-1$.
Assume  that $\bar V$ is compact and that the boundary $\partial V$ 
satisfies Federer's condition, i.e. it contains an open part $\text{reg}(\partial V)$ which is an oriented $k-1$-dimensional manifold whose $(k-1)$-dimensional volume is finite,
and the $k-1$-dimensional Haussdorff measure of 
$\partial V\setminus \text{reg}(\partial V)$ is zero. Then the following hold for every 
differential $(k-1)$-form $\alpha$ of class $C^1$ defined in some open 
neighborhood of $\bar V$:
\[
\int_V\, d\alpha=\int_{\text{reg}(\partial V)}\,\alpha 
\]

\medskip

\noindent
{\bf An example} Let $n\geq 3$ and $1\leq k\leq n-1$. Suppose that
$P_1(x),\ldots,P_k(x)$ is a $k$-tuple of real valued polynomials 
of $n$ variables such that the set where the $k\times k$-matrix whose 
elements are 
\[
\partial P_i/\partial x_\nu(x)\quad\colon\,\quad 1\leq i,\nu\leq k
\] 
is invertible in some non-empty open set $U$ of $\bf R^n$, i.e. the
polynomial defined by the determinant of this matrix is not identically zero.
Then we obtain a locally closed submanifold of $\bf R^n$ defined by
\[ 
W=\{x\colon\, P_1(x)=\ldots=P_k(x)=0\}\,\cap U
\]
Next, let $Q_(x),\ldots,Q_m(x)$ be some $m$-tuple of polynomials
and put 
\[
\mathfrak{Q}=\{x\colon\, Q_\nu(x)<0\,\colon
1\leq\nu\leq m\}
\]
Assume also that the open set
$\mathfrak{Q}$ is bounded in $\bf R^n$, i.e. contained in
some open ball with sufficiently large radius centered at the origin.
Then the general Stokes Theorem holds for the locally closed
$k$-dimensional submanifold $V=W\cap\mathfrak{Q}$.
The proof that Federer's conditions hold  follows from a
result about \emph{semi-algebraic sets} due to Tarski and Seidenberg. The reader may consult the appendix in  H�rmander's text-book [H�:1] or his more
recent text-book series [H�] for an account about semi-algebraic sets which 
verify
Federer's conditions.
We remark that the resulting boundary integrals may become quite involved
since no extra conditions are imposed upon the $Q$-functions so the boundary
$\partial V$ may have "corners" which in the case $n\geq 3$ of course are hard to
vizualise.









\newpage

\centerline{\bf 4.  Green's formula and Dirichlet's problem}
\bigskip

\noindent
Let $\Omega\in\mathcal D(C^1)$
and $f$ is  a function of class $C^2$ which means that $f_x$ and $f_y$
are of class $C^1$. Stokes Theorem  applied to $f_x$ and $f_y$ give
\[
 \iint_\Omega\, f_{xx}dxdy=
\int_{\partial\Omega}\, f_x n_x ds\quad\colon\quad
\iint_\Omega\, f_{yy}dxdy=
\int_{\partial\Omega}\, f_y n_y ds
\]
\medskip  

\noindent
Adding the two equalities we obtain
\[
 \iint_\Omega\,( f_{xx}+f_{yy})dxdy=
\int_{\partial\Omega}\, (f_x n_x+f_y n_y) ds
\]
\medskip

\noindent
Here  $f_x n_x+f_y n_y$ is  the
\emph{directional derivative} of $f$ along the outer 
normal $n$  which we denote by $f_n$.
Next, $f_{xx}+f_{yy}$ is  the 
\emph{Laplacian} of $f$ and is denoted by
$\Delta(f)$. Hence we have proved
\medskip

\noindent {\bf 4.1 Theorem} \emph{Let $\Omega\in\mathcal D(C^1)$. For each $f$ of
class $C^2$ we have}
\[
 \iint_\Omega\,\Delta(f)\,dxdy=
\int_{\partial\Omega}\, f_n\, ds
\]
\medskip

\noindent {\bf 4.2 Remark.} Stokes Theorem
applied to the functions $f_x$ and $f_y$  gives the two formulas

\[
 \iint_\Omega\, f_{yx}dxdy=
\int_{\partial\Omega}\, f_x n_y ds\quad\colon\quad
\iint_\Omega\, f_{xy}dxdy=
\int_{\partial\Omega}\, f_y n_x ds
\]
When $f$ is of class $C^2$, the mixed
derivatives $f_{yx}$ and $f_{xy}$ are equal.
Hence we obtain the following equality for line integrals
\[
\int_{\partial\Omega}\, f_x n_y ds=
\int_{\partial\Omega}\, f_y n_x ds
\]
Notice that this equality is obvious using differentials to express the line integrals, i.e. 
since $dx=-n_y ds$ and $dy= n_x ds$ the equality above is expressed by
\[
\int_{\partial\Omega}\, f_xdx+f_y dy=0
\]
The vanishing of this line integral follows trivally since each
boundary curve of $\partial\Omega$ is closed.
However, the two equalities above have a non-trivial consequence. Namely, 
let $f$ be of class $C^2$ and consider its \emph{gradient vector}
\[
\nabla(f)=(f_x,f_y)= f_x\cdot e_x+f_y\cdot e_y
\]
where $e_x$ and $e_y$ are the euclidian basis vectors in
$\bf R^2$.
Subtracting the two equalities above we get 
\[
\int_{\partial\Omega}\, (f_x n_y-f_y n_x) ds=0
\]
Here  $f_x n_y-f_x n_x$ is equal to the \emph{vector product}
$\nabla\times n$. Hence we have proved
\medskip


\noindent
{\bf 4.3 Theorem} \emph{Let $\Omega\in\mathcal D(C^1)$. For each $f$ of
class $C^2$ we have}
\[
\int_{\partial\Omega}\, \nabla(f)\times n \cdot ds=0
\]
\medskip
\noindent
{\bf 4.4 Mean value integrals}
Consider the  case when $\Omega$ is an open disc. 
Without loss of generality we may assume its center is at the origin and 
let $R$ be the radius.
Denote the disc by $D_R$.
Here $\partial D_R$ is parametrized by
\[ 
\theta\mapsto R(\text{cos}(\theta),\text{sin}(\theta))\quad
\colon\quad 0\leq\theta\leq 2\pi
\]
Let $f$ be a $C^2$-function defined in some open neighbourhood of the closed
disc $\bar D_R$. Put
\[
M_f(R)=\frac{1}{2\pi}\int_0^{2\pi}\, f(R\cdot
\text{cos}\,\theta,R\cdot\text{sin}\,\theta)\,d\theta
\]
As $R$ varies we can take the derivative and rules of 
differentiation yield the following equality for each $0<r<R$:

\[
\frac{d}{dr}(M_f(r))=
\frac{1}{2\pi}\int_0^{2\pi}
\, [f_x(r
\text{cos}\,\theta,r\text{sin}\,\theta)\text{cos}(\theta)+
f_y(r
\text{cos}\,\theta,r\text{sin}\,\theta)\text{sin}(\theta)\,]
\,d\theta\quad
\]

\bigskip

\noindent
Since $rd\theta=ds$ and $n=(\text{cos}\,\theta,\text{sin}\,\theta)$
we can express the equation by
\[
\frac{d}{dr}(M_f(r))=\frac{1}{2\pi r}\int_{\partial D_r}\,
f_n ds
\]
Hence Theorem 4.1 gives:

\medskip


\noindent
{\bf 4.5 Theorem} \emph{For each $0<r<R$ one has}
\[
\int_{D_r}\Delta(f) dxdy=2\pi r\cdot \frac{d}{dr}(M_f(r))
\]
\medskip

\noindent
In particular, suppose that $f$ is a \emph{harmonic function}
which  means that it satisfies the Laplace equation, i.e. 
$\Delta(f)=0$. Then the left hand side is zero above and hence the mean-value
function $M_f(r)$ is constant. By continuity at the origin we see that
\[
\lim_{\epsilon\to 0}\, M_f(\epsilon)=f(0,0)
\]
Hence we get

\medskip

\noindent
{\bf 4.6 Theorem} \emph{Let $f$ be a harmonic function in disc $D_R$. Then}
\[ 
f(0,0)=M_f(r)\quad\colon\quad 0<r< R
\]
Staying with harmonic functions we also notice that Theorem 4.1 gives
\medskip

\noindent {\bf 4.7 Theorem}. \emph{Let $\Omega\in\mathcal D(C^1)$. For each 
$C^1$-function $h$ 
which is harmonic in  $\Omega$ one has}
\[ 
\int_{\partial\Omega}\, h_n ds=0
\]
\medskip

\noindent
{\bf 4.8 Formulas with two functions.} Let $\Omega\in\,\mathcal D(C^1)$ 
and $f,g$ is a pair of $C^2$-functions.
Applying Theorem 2.4 to $f_xg$ and $f_yg$ gives after a summation
\medskip

\[
 \iint_\Omega\, [\Delta(f)\cdot g+ f_xg_x+f_yg_y]\,
 dxdy=
\int_{\partial\Omega}\, f_n\cdot g \cdot ds
\]
\medskip

\noindent
Here $f_xg_x+f_yg_y$
is the \emph{inner product} of the gradient vectors of $f$ and $g$.
Since this term  is  symmetric  for the pair, we obtain the following  when the same formula a above is applied with $f$ and $g$
interchanged:
\bigskip

\noindent {\bf 4.9 Theorem} \emph{For each pair of $C^2$-functions $f,g$ one has:}
\[
 \iint_\Omega\, \bigl[\Delta(f)\cdot g-f\cdot \Delta(g)\bigr]\,
 dxdy=
\int_{\partial\Omega}\,\bigl[ f_n\cdot g-f g_n\bigr] \cdot ds
\]
\medskip

\noindent 
{\bf 4.10 Application.} Let $\Omega=D_R$ be a disc centered
at the origin. Let $f$ be harmonic in $D_R$. Given a point
$(a,b)\in D_R$ we define the $g$-function
\[
g(x,y)=\text{Log}(\sqrt{(x-a)^2+(y-b)^2})
\]
An easy  computation  which is left to
the reader shows that $g$ is a harmonic
function in $\bf R^2$ outsidee the point
$(a,b)$. With $\epsilon>0$ small we remove the open disc $D_\epsilon(a,b)$
and apply 
Green's formula to the domain
$\Omega_\epsilon=D_R\setminus \bar D_\epsilon(a,b)$.
Since both $f$ and $g$ are harmonic in $\Omega_\epsilon$ we obtain:
\[
\int_{\partial\Omega_\epsilon}\,f_n g\, ds=
\int_{\partial\Omega_\epsilon }\,f g_n \, ds
\]
This equality yields an interesting formula when 
$\epsilon\to 0$. First, $\partial \Omega_\epsilon=
\partial\Omega\cup\,\partial D_\epsilon$. For the line integrals over 
$D_\epsilon$
the following two
limit formulas hold:
\[
\lim_{\epsilon\to 0}\,
\int_{\partial D_\epsilon}\,f_n g\, ds=0\quad\colon\,
\lim_{\epsilon\to 0}\,\int_{\partial D_\epsilon}\,f g_n\, ds=
-2\pi f(a,b)\tag{*}
\]
where the last minus sign appears since $\partial D_\epsilon$
is a boundary component of $\Omega_\epsilon$ so that the
outward normal points into the disc $D_\epsilon$.
\medskip

\noindent {\bf 4.11 Exercise.} Prove the last limit formula in (*).
The hint is: We may assume that $(a,b)$ is the origin and 
since $\frac{d}{dr}\text{Log}\, r=\frac{1}{r}$
we find that the outer normal $g_n=-\frac{1}{\epsilon}$. At 
the same time $ds=\epsilon d\theta$ and now the reader 
verifies the second limit formula.
\medskip

\noindent
Using these two limit formulas we obtain
\[
2\pi f(a,b)=\int_{\partial D_R }\,f g_n \, ds-
\int_{\partial D_R }\,f_n g\,ds\tag{**}
\]
Hence the value of the harmonic function $f$  at $(a,b)$
can be
expressed by  sum of two line integrals on $\partial D_R$ where $f$ and $f_n$ appear.
Later on
we 
shall find \emph{Poisson's  formula} where the value of $f$ is expressed by a
line integral where only $f$ appears.

\bigskip






\centerline  {\bf 4.12 The Dirichlet problem}
\medskip

\noindent
The formula 4.8 with two functions
was the starting point when 
Dirichlet around 1840 posed the following 
problem:
\bigskip
\[
\text{Given}\,\,h\in C^0(\partial \Omega)\,\,\, \text{find}\,\,\,
f\in C^0(\bar\Omega)\quad\colon\,\, f|\partial\Omega=h\quad \Delta(f)|\Omega=0
\]
\bigskip

\noindent
To solve this Dirichlet considered the following \emph{variational problem}

\[
V(f)=\,\iint_\Omega\, (f_x^2+f_y^2) \cdot dxdy\quad\colon\quad f|\partial\Omega=h
\]
Now one seeks
\[
V_*=\min_f\, V(f)
\]
If $V(f)=V_*$ one has:
\[ 
\lim_{\epsilon\to 0}\, \frac{V(f+\epsilon g)-V(f)}{\epsilon}=0\quad\,\colon\quad 
\forall\, g\,\,\text{such that}\,\, g|\partial\Omega=0\tag{i}
\]
Now we notice that
\[ 
V(f+\epsilon g)-V(f)=2\epsilon\,\iint (f_xg_x+f_yg_y)dxdy
+\epsilon^2\iint (g^2_x+g^2_y)dxdy=
\]
\[
-2\epsilon\iint \Delta(f)g\,dxdy+\epsilon^2\iint (g^2_x+g^2_y)dxdy
\]
Passing to the limit when $\epsilon\to 0$
we conclude that if $f$ minimizes  $V$ then
\[
\iint \Delta(f)g\,dxdy=0\tag{ii}
\]
Since this hold for all $g$ vanishing on $\partial\Omega$
it follows that   $\Delta(f)=0$ in $\Omega$. 
Hence $f$ solves Dirichlet's problem.
\medskip

\noindent
{\bf 4.13 An obstacle.}
Dirichlet's solution is correct but
his proof became "shaky" when Weierstrass
discovered that there exist variational problems
of a similar nature as  above which \emph{fail} to have
an extremal
solution. In 1923  O. Perron gave a rigorous proof  using
subharmonic functions which 
answers  the question  when
Dirichlet's problem has a solution for every continuous boundary
function.

\bigskip

\noindent
{\bf 4.14 Theorem.} \emph{Let $\Omega$ be a bounded open  set such that for
every point $a\in\partial\Omega$
the connected component of the set
${\bf{C}}\setminus\Omega$ which contains $a$
is not reduced to the singleton set $\{a\}$. Then
each $h\in C^0(\partial\Omega)$
has a unique harmonic extension to
$\bar\Omega$.}
\bigskip

\noindent
{\bf Remark.}
We prove this in Chapter V.
Notice that 
Theorem 4.14 applies  when
$\Omega$ is of class $\mathcal D(C^1)$.

\medskip



\noindent
{\bf 4.15 Probabilistic solution.}
Theorem  4.14  can be  proved
by  
\emph{probabilistic considerations}. 
Let us describe this under the assumption that
$\partial\Omega$ is "nice".
Pick some 
arc $\gamma$ from the boundary, i.e. $\gamma$ is a simple 
closed curve contained in one of the closed boundary curves $\Gamma$.
Now one seeks a harmonic function $f$ in $\Omega$ such that its boundary value is equal to 1 at interior points of $\gamma$ and zero on 
$\partial\Omega\setminus\gamma$. So the boundary values of $f$ are determined 
except at the the two end-points of  
the closed $C^1$-curve $\gamma$.
\medskip

\noindent {\bf 4.16 The harmonic measure.} Let $p\in\Omega$.
Starting from $p$ we consider the \emph{Brownian motion}, i.e. 
perform a 2-dimensional random walk which may be approximated by regarding
small consequtive steps of length $\delta$ which moves the particle
with probability $1/4$ in each direction - i.e.  changing  $x$ or $y$
by + or \vvv $\delta$. With probability one the discrete random walk eventually crosses the boundary
$\partial\Omega$. Some of these cross $\gamma$ and this gives a number 
$0<\pi_\gamma(p)<1$ which is the probability for a random walk 
to cross $\gamma$. To be precise, one gets this number in the limit when
$\delta\to 0$.
Next, we use the fact that a function is harmonic if and only if
it satisfies a local \emph{mean-value condition}, i.e. a function $f$ is harmonic in 
$\Omega$ if and only if its value at point $q\in\Omega$ is equal to its mean-value over small discs centered at $q$. From this it follows easily that the function
\[ 
p\mapsto \pi_\gamma(p)
\] 
is harmonic in $\Omega$ and yields the solution to Dirichlet's
problem with boundary values as above.
Finally, since this is achieved for boundary arc  $\gamma$
one can deduce Theorem 4.14 by approximating a continuous function $h$ on 
$\partial\Omega$ with functions which are piecewise constant
\medskip

\noindent 
{\bf Remark} 
The probability expressed by  $\pi_\gamma(p)$
plays an important role later on since it is equal to the \emph{harmonic
measure} defined as the value at $p$ of the harmonic function in
$\Omega$ with boundary value zero on
$\partial\Omega\setminus\gamma$ and equal to one  on $\gamma$.
This probabilistic interpretation of the harmonic measure
gives also an intuitive
feeling for the harmonic measure.

\medskip

\noindent
A notable point is that  the probabilistic solution makes it possible to
use \emph{Monte Carlo simulations}  in order to  obtain good 
approximative solutions to Dirichlet's problem
which in general have
no "analytic solutions".
For example,  presence of several boundary components of
$\partial\Omega$ 
does not in principle cause any  problem when a  Monte Carlo simulation
is used.
A drawback is that  Monte Carlo simulations tend to be rather
time consuming since one must repeat random walks several times over small grids.
In an impressive \emph{Examensarbete by Oskar Sandberg} (2003)
at the Mathematics Department in Stockholm, a quite rapid Monte Carlo
simulation was developed . Here
one takes  larger random steps in each simulation 
using the \emph{mean-value property} for harmonic functions. The
interested reader may consult Sandberg's work  for  further details
where numerical solutions
to the Dirichlet problem
in dimension $\geq 3$ also are obtained by  Monte Carlo simulations.
See also the section about the material about
the  \emph{Brownian motion}  in
Chapter XX for further comments.

\bigskip

\centerline {\bf 4.18 The Dirichlet problem in a half-space.}
\medskip


\noindent
Consider the half space in ${\bf{ R}}^2$ defined by $U=\{(x,y)\colon \,y>0\}$.
So here $\partial U$ is the $x$-axis.
We construct harmonic a class of harmonic 
functions in $U$ by the following procedure:
\medskip

\noindent {\bf  4.19 Definition} \emph{To each pair of real numbers $a<b$ we let
$H_{a,b}(x,y)$ be the function in $U$ whose value at $(x,y)$ is the angle at 
this point in the triangle with corners at $(a,0),(b,0),(x,y)$}.
\medskip

\noindent
The reader should draw a figure which explains why 
$0<H_{a,b}(x,y)<\pi$ for all $(x,y)\in U$. Moreover one has the limit formulas
\medskip
\[
\lim_{y\to 0}\,H_{a,b}(x,y)=\pi\quad\,a<x<b
\quad\colon\quad
\lim_{y\to 0}\,H_{a,b}(x,y)=\pi\quad\,x<a\quad b<x
\]
Finally we also have
\[
\lim_{x^2+y^2\to\infty}\,H_{a,b}(x,y)=0
\]
Less obvious is that $H$ is  a \emph{harmonic} function in $U$.
So let us give:
\medskip


\noindent
\emph{Proof} Recall that the sum of the angles of a triangle is $\pi$.
Given the poins $a<b$ a figure shows that the angle
$H_{a,b}(x,y)$ is equal to
$\beta -\alpha$ where $\alpha$ is the angle between the vector from $a$ to $(x,y)$ and the positive $x$-axis, and similarly  $\beta$ is the angle
between the vector from $b$ to $(x,y)$  and the positive $x$-axis.
Since the sum of harmonic functions is again harmonic, i suffices to show that
the functions $\alpha(x,y)$ and $\beta(x,y)$ are harmonic.
Now it is clear - again   by a figure - that
\[
\text{tg}(\alpha)=\frac{y}{x-a}\implies
\alpha(x,y)=\text{arctg}\bigl(\frac{y}{x-a}\bigr)
\]
When $y>0$ and $x=a$ the $\alpha$-function is $\pi/2$. Outside this vertical line
we can take irs derivatives. Using the wellknown formula for the derivative of the 
arctg-function the reader may verify
that $\Delta(\alpha)=0$. 
\medskip  

\noindent {\bf 4.20 Remark}
See also XX where we give an alternative proof
that $H$
is harmonic using  complex valued  Log-functions. 
The  $H_{a,b}$-functions can be used to  solve the Dirichlet
problem since the two limit formulas after Definition 4.19
settle the case when the boundary function
is the characteristic function of an interval. To proceed further we need to 
study the $H$-functions when $b-a\to 0$.
Given $(x,y)\in U$ and some  $a$ on the $x$-axis we consider
for a small positive $\Delta$
the triangle with corners at $(x,y),(a,0),(a+\Delta,0)$. Let $\alpha$ 
be the angle at $(x,y)$
which therefore gets small with $\Delta$.
By wellknown results about the 
cosine- and the sine-function - especially that
$\frac{\text{sin}\,\alpha}{\alpha}\to 1$ as $\alpha\to 0$, the reader can easily verify 
that:
\[
\lim_{\Delta\to 0}\frac{\alpha}{\Delta}=\frac{y}{(x-a)^2+y^2}\tag{*}
\]
Armed this result 
we solve the
Dirichlet problem when $f(x)$ is a continuous function which
vanishes outside a bounded interval $[-A,A]$.
To keep variables distinct we use $\xi$ as the coordinate on the 
$x$-axis. Let $\{\xi_\nu\}$ be a finite and strictly increasing sequence
where the differences $\xi_{\nu+1}-\xi_\nu$ are small. Here
$\xi_0=-A$ and $\xi_N=A$ for some $A>0$.
Define the function
\[ 
G_N(x,y)=\sum\, f(\xi_\nu)\cdot H_{\xi_\nu,\xi_\nu+1}(x,y)\tag{i}
\]
Since $f$ is continuous the limit formulas for the $H$-functions above show that
\[
\lim_{y\to 0}\,G_N(a,y)\simeq \pi\cdot f(a)\tag{ii}
\] 
for every real $a$ where this approximative equality becomes more and
more accurate as the
maximum of the differences
$\xi_{\nu+1}-\xi_\nu$ tends to zero.
At the same time  the limit formulas for the $H$-functions imply
that the $G_N$-fuction is approximated by the \emph{Riemann sum}
\[
\sum\, f(\xi_\nu)\cdot (\xi_{\nu+1}-\xi_\nu)\cdot
\frac{y}{(x-\xi_\nu)^2+y^2}
\]
Next, by the construction of the Riemann integral of $f$ over $[-A,A]$
we also have
\[ 
G_N(x,y)\simeq\int_{-A}^A \frac{y}{(x-\xi)^2+y^2}f(\xi)d\xi\tag{iii}
\]
Passing to a limit where we take
refined $\xi$-partitions, the sequence $\{G_N(x,y)$ converges to a limit function
$G(x,y)$ which is harmonic in the upper half-plane and
\[
\lim_{y\to 0}\,G(a,y)= \pi\cdot f(a)\quad\colon\quad -A<a<A
\] 
If $a$ is outside the interval
$[-A,A]$ 
we see that
$\lim_{y\to 0}\,G(a,y)=0$.
Hence we have arrived at Poisson's solution:

\medskip


\noindent
{\bf 4.21 Theorem} \emph{Let $f(\xi)$ be a continuous function on the real 
$x$-axis which is zero outside a bounded interval.
Then the function}
\[ G(x,y)=\frac{1}{\pi}\int_{-\infty}^\infty \frac{y}{(x-\xi)^2+y^2}\dot f(\xi)d\xi
\]
\emph{is harmonic in the upper half plane and 
$G(\xi,0)=f(\xi)$ holds on the boundary.} 
\bigskip




\centerline{\bf{5. Exact versus closed 1-forms}}
\bigskip


\noindent
In this section a domain $\Omega$ in $\mathcal D(C^1)$ is given. It is connected 
and $\partial\Omega$ has $p$ many boundary curves for some $p\geq 1$.
The functions and other objects below are defined in
$\Omega$ and when integrals are taken over the boundary
it is assumed that the functions have
been extended to the closure of $\Omega$ in order that
boundary integrals are defined.
A differential 1-form is given by:
\[
W=f(x,y)\cdot dx+g(x,y)\cdot dy\tag{1}
\]
Here $f$ and $g$ are supposed to be of class $C^1$ at least.
The 1-form is closed if
\[ 
f'_y=g'_x\tag{2}
\]
Suppose there exists a $C^2$-function $U(x,y)$ such that
\[
U'_x=f\quad\text{and}\quad 
U'_y=g\tag{3}
\]
Since the mixed second order derivatives
$U''_{xy}$ and $U''_{yx}$ are equal we see that (3) gives (2).
When $U$ exists we say that the 1-form $W$ is exact and $U$ is called
the potential function of $W$.
\medskip

\noindent
{\bf{Remark.}}
In mechanics one refers to a 1-form $W$ as a field of forces, i.e.
to every point one assigns the force vector
$F=(f,g)$. If (3) holds we have the equality
$\nabla(U)=F$ and one says that $F$ is a potential field where
$U$ is its potential function. Notice that $U$ is determined up to a constant.
\medskip

\noindent
Let $\Gamma$ be  $C^1$-curve
with end-points $A=(x_0,y_0)$ and $B=(x_1,y_1)$.
For every 1-form $W$ we get the line integral
\[ 
\int_\Gamma\, W=\int_0^T\, 
\bigl[f(x(t),y(t))\cdot \dot x+g(x(t),y(t))\cdot \dot y
\bigr ]\cdot dt\tag{i}
\]
If $W$ is exact with a potential function $U$ we notice that
\[ 
\frac{d}{dt}(U(x(t),y(t))=
f(x(t),y(t))\cdot \dot x+g(x(t),y(t))\cdot \dot y
\]


\noindent
Hence (i) is equal to $U(B)-U(A)$. In other words, when
$W$ is exact then the line integral along a curve
$\Gamma$ only depends upon the two end-points and  it is
expressed by
the difference $U(B)-U(A)$.
\medskip


\noindent
{\bf{5.1 Exercise.}} Let $W$ be a closed 1-form and assume that
the line integral along every curve $\Gamma$
only depends on the end-points. Show that $W$ is exact.
\medskip

\noindent
{\bf{5.2 Non-exact 1-forms.}}
The standard example of a closed but non-exact 1-form occurs when
$\Omega$ is an annulus $r^2<x^2+y^2<R^2$
and we take

\[ 
W=\frac{ydx}{x^2+y^2}-\frac{xdy}{x^2+y2}
\]


\noindent
{\bf{5.3 Starshaped domains.}}
We say that $\Omega$ is star-shaped with respect to the origin
if the line from 0 to every point $p\in\Omega$ is contained in
$\Omega$.
In this case every closed 1-form is exact. To see this
we define the function $U(x,y)$ in $\Omega$ by
\[ 
U(x,y)=\int_0^1\,\bigl[ x\cdot f(tx,ty)+y\cdot g(tx,ty)]\cdot dt
\]
Now we get
\[ U'_x=\int_0^1\,f(tx,ty)+
\int_0^1\,\bigl[ tx\cdot f'_x(tx,ty)+ty\cdot g'_x(tx,ty)]\cdot dt\tag{1}
\]
Next, $g'_x=f'_y$ is assumed 
and we notice that
\[
\frac{d}{dt}(f(tx,ty))=x\cdot f'_x(tx,ty)+y\cdot f'_y(tx,ty)
\]
It follows that the last integral in (1) becomes

\[
\int_0^1\,t\cdot \frac{d}{dt}(f(tx,ty))\cdot dt=
f(x,y)-\int_0^1\,f(tx,ty))\cdot dt
\]
The last integral is cancelled via (1) and hence $U'_x=f$. In the same way one
shows that $U'_y=g$ and hence the 1-form is exact.

\medskip

\noindent
{\bf{5.4 The deformation theorem.}}
Let $A$ and $B$ be two given points in $\Omega$.
Consider a family of curves
$\{\Gamma_s\}$
where each $\Gamma_s$ has $A$ and $B$ as endpoints and
we are given a vector valued function
\[ 
\rho\colon\, (s,t)\mapsto (x(s,t),y(s,t)\quad
\colon 0\leq s\leq 1\quad
\text{and}\quad 0\leq t\leq T\tag{*}
\]
Here $t\mapsto (x(s,t),y(s,t))$�is the parametrization of
$\Gamma_s$.
With these notations we have

\medskip


\noindent
{\bf{5.5 Theorem.}} \emph{For every closed 1-form $W$ the function
below is a constant}
\[ 
s\mapsto \int_{\Gamma_s}\, W\tag{1}
\] 




\noindent
{\bf{Remark.}} We have essentially proved this in
the introduction, i.e. see Theorem A.X.
But we give another proof below which involves less calculations and 
at the same time illustrates the efficiency when the calculus of
differential forms is used.




\medskip

\noindent
\emph{Proof.} 
To begin with, we can restrict $s$ to some interval $[0,s_*]$ for every
$0<s_*\leq 1$ and therefore it suffices to prove that
(1) takes the same value of $\Gamma_1$ and $\Gamma_0$.
To achieve this we first consider an arbitrary
$C^1$-function $f$ and the 1-form $f\cdot dx$. We shall now find an
expression of the difference
\[ 
 \int_{\Gamma_1}\, f\cdot dx-
\int_{\Gamma_0}\, f\cdot dx\tag{3}
\]
For this purpose we consider the $\rho$-map in (*) and construct the inverse function
\[ 
f^*(s,t)= f(x(s,t),y(s,t))
\]
 We have also the inverse 1-form 
\[ 
\rho^*(dx)= x'_s\cdot ds+x'_t\cdot dt
 \]
Put $\square=\{ (s,t)\quad\colon \{0\leq s\leq 1\}\cap\{0\leq t\leq T\}\}$.
Since the end-points of the curves
$\{\Gamma_s\}$ are equal
it follows that (3) is equal to the boundary integral

\[ 
\int_{\partial\square}\, f^*\cdot \rho^*(dx)\tag{4}
\]
Stokes theorem applied to $\square$ implies that (4) is equal to
\[ 
\iint_\square\, df^*\cdot \rho^*(dx)=\iint_\square\, \rho^*(f\cdot dx)=
\iint_\square\, \rho^*(f'_y\cdot dy\wedge \cdot dx)
\tag{5}
\]
Let us remark that the last integral becomes
\[ 
\iint_\square\, (f'_y)^*\cdot (y'_s\cdot x'_t-y'_t\cdot x'_s)\cdot ds\wedge dt\tag{6}
\]
If we instead start with a 1-form $g\cdot dy$
then the same calculation gives
\[
 \int_{\Gamma_1}\, g\cdot dy-
\int_{\Gamma_0}\, g\cdot dy=
\iint_\square\, \rho^*(g'_x\cdot dx\wedge \cdot dy)
\tag{7}
\]
Finally, using the equality $dy\wedge dx=dx\wedge dy$ we obtain
the following equality which is valid for an \emph{arbitrary}
1-form $W=f\cdot dx+g\cdot dy$ which need not be closed:
\bigskip

\[
 \int_{\Gamma_1}\, W-
\int_{\Gamma_0}\, W=
\iint_\square\, \rho^*((g'_x-f'_y)\cdot dx\wedge \cdot dy)\tag{**}
\]
In the special case when $f'_y=g'_x$ the last integral is zero and
hence  
$\int_{\Gamma_1}\, W=\int_{\Gamma_0}\, W$ holds, i.e. Theorem 5.3 
is a special case of (**).
\bigskip

\noindent
{\bf{5.6 When is a 1-form exact.}}
If $p=1$, i.e. if $\Omega$ is a Jordan domain then topology
learns that every closed curve in $\Omega$ can be deformed to a point. 
Hence Theorem 5.3 implies that if $W$ is a closed 1-form
then its  line integral over every closed curve
is zero. It follows from Exercise 5.1 that $W$ is exact.
If $p>1$ topology learns that there exist $p-1$ many closed curves
$\Gamma_1,\ldots,\Gamma_{p-1}$ which give a basis for the homology of
the multiple connected domain.
If $W$ is a closed 1-form one assigns the period numbers
\[ 
\int_{\Gamma_k}\, W\quad\colon\,\, 1\leq k\leq p-1\tag{*}
\]
Now $W$ is exact if and only if all the period numbers are zero.
Suppose $W_1,\dots,W_{p-1}$
is some $(p-1)$-tuple of closed 1-forms
such that the $(p-1)\times(p-1)$-matrix with elements
\[ 
a_{ik}=\int_{\Gamma_k}\, W_i
\] 
is invertible.
Then, for every closed 1-form $W$ we can find $c_1,\ldots,c_{p-1}$ 
such that the periods
of $W-(c_1W_1+\ldots+c_{p-1}W_{p-1})$ vanish
and hence there exists a potential function $U$ such that
\[
W=c_1W_1+\ldots+c_{p-1}W_{p-1}+dU
\]



\newpage

\centerline{\bf{6. An integral formula for the Laplace operator.}}
\medskip


\noindent
Consider a $C^2$-function $f(x,y)$ defined in
the open disc $D(R)$ of radius $R$ centered at the origin.
Let $0<r<R$ and choose also a small $\epsilon>0$ and consider the annulus
$\Omega=\{\epsilon^2<x^2+y^2<r^2\}$.
In $\Omega$ we have  the function
\[ 
g_r(x,y)=\text{Log}(\frac{r}{\sqrt{x^2+y^2}})\tag{1}
\]
We have already seen that $g$ is harmonic in $\Omega$.
The boundary $\partial\Omega$ consists of the circle 
$x^2+y^2=\epsilon^2$ and the circe of radius $r$ and
we leave it to the reader to verify that Theorem 4.9 applied to $f$ and $g$
give the formula
\[
\iint_\Omega\, \Delta(f)(x,y)\cdot g_r(x,y)\cdot dxdy=
\int_0^{2\pi}\, f(r,\theta)\cdot d\theta -
\int_0^{2\pi}\,f(\epsilon,\theta)\cdot d\theta\tag{2}
\]
\medskip

\noindent
Keeping $r$ fixed while $\epsilon\to 0$
the continuity of $f$ at the origin gives:
\[
f(0)=\frac{1}{2\pi}\cdot \int_0^{2\pi}\, f(r,\theta)\cdot d\theta -
\frac{1}{2\pi}\cdot\iint_{D(r)}\, \Delta(f)(x,y)\cdot g(x,y)\cdot dxdy\tag{3}
\]


\noindent
With $0<r_*<R$ we can apply (3) to every $0<r<r^*$
and after an integration
using polar coordinates we obtain:
\[
f(0)=\frac{1}{\pi\cdot r_*^2}\cdot \iint_{D(r_*)}\,
f(x,y)\cdot dxdy-
\frac{1}{2\pi}\cdot\int_0^{r_*}\, [
\iint_{D(r)}\,  \Delta(f)(x,y)\cdot g_r(x,y)\cdot dxdy\bigr]\cdot dr
\]
\medskip

\noindent
The last double integral is evaluated via polar coordinates
and
the result is the formula from the introduction, i.e. the reader may verify that
the double integral becomes
\[
\int_0^{r_*}\,K(r^*,s)\cdot \bigl [\int_0^{2\pi}  \Delta(f)(s,\theta)\cdot d\theta\bigr]\cdot ds\tag{i}
\]
where  we for each pair $0<s\leq r^*$ have:
\[ 
K(r^*,s)=s^3\cdot \int_1^{\frac{r}{s}}\, u\cdot \text{Log}(u)\cdot du\tag{ii}
\]


\end{document}


\newpage


Let $\mathcal C$
be a simple closed curve of class $C^1$, i.e. a differentiable closed Jordan curve
which borders a Jordan domain $\Omega$.
As $z$ varies in $\Omega$ we obtain
the function
\[ 
J(z)= \int_\mathcal C\, \log\frac{1}{|z-q|}\cdot ds(q)
\] 
where $q$ denote points on $\mathcal C$ and $ds$ is the arc-length measure.
here $J(z)$ is a harmonic function in
$\Omega$ and it extends to a continuous function on
the closure. The reasoin is that
if $p\in\mathcal C$ then the functiion
\[
q\mapsto   \log\frac{1}{|z-q|}
\] 
is integrable with respect to the arc-length measure.
Less evident is the behaviour of normal derivatives of $J$
along the boundary curve.
This, when $p\in\mathcal C$ is given
we first have a defined integral value
\[
J(p)= \int_\mathcal C\, \log\frac{1}{|z-q|}\cdot ds(q)
\]
At $p$ we also have the inner normal ${\bf{n}}_i(p)$
which by definition is a vector of unit length
which is  $\perp$ to $\mathcal C$ at $p$ and
points into $\Omega$.
Then we seek
\[ 
\lim_{h\to  0}\,\frac{J(p-h{\bf{n}}_i(p))-J(p)}{h}
\]
It turns out that (x) has a limit
and we get a function on  $\mathcal C$ defined by
\[
\frac{\partial J}{\partial{\bf{n}}_i}(p)=
\lim_{h\to  0}\,\frac{J(p-h{\bf{n}}_i(p))-J(p)}{h}
\]
Another procedure when we evaluate normal derivatives with respect
to $q$ while $p\in\Omega$ is kept fixed.
Not so easy....

























\end{document}






