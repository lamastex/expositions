\documentclass{amsart}


\def\uuu{_}

\def\vvv{-}


\usepackage[applemac]{inputenc}

\addtolength{\hoffset}{-12mm}
\addtolength{\textwidth}{22mm}
\addtolength{\voffset}{-10mm}
\addtolength{\textheight}{20mm}

\begin{document}





\centerline{\bf\large Chapter 5.B. Subharmonic functions }

\bigskip

\noindent 
0. Introduction
\medskip


\noindent 
1. The subharmonic Log-function
\medskip


\noindent 
2. Basic facts about subharmonic functions
\medskip

\noindent 
3. Riesz representation formula
\medskip

\noindent 
4. Perron families

\medskip

\noindent
5. Piecewise harmonic subharmonic functions

\medskip

\noindent
6. On zero sets of subharmonic functions.
\medskip

\noindent
7. A subharmonic majorization


\medskip
\noindent
8. Subharmonic minorant of a given function


\bigskip





\centerline {\bf Introduction.}
\medskip




\noindent
The  theory of subharmonic functions is foremost due to
F. Riesz whose article
[Ri] from 1926 contains the essential facts about
subharmonic functions. 
See also the text-book
\emph{Subharmonic functions} by Rado from 1937
for an  account 
about subharmonic functions. 
We shall not discuss subharmonic functions in
${\bf{R}}^n$ when $n\geq 3$
which  for example are treated in
the text-book  [Hayman]. 
The basic results about subharmonic functions appear
in � 1-4 while the remaining sections
are devoted to  special topics.
For example, a  delicate result due to Beurling appears
in  Theorem 8.2 which 
gives
a sufficent  condition for the existence of a largest subharmonic minorant.
\medskip

\noindent
{\bf{0.0 Subharmonic functions in complex domains.}}
We identify ${\bf{R}}^2$ with ${\bf{C}}$
where real-valued functions
$u(x,y)$ become functions of the single complex
variable $z=x+iy$.
If $u(x,y)$ is a $C^2$-function defined in an open set $\Omega$
then 
the Laplacian $\Delta(u)$ is a non-negative  function if
and only if $u$ satisfies the \emph{local mean-value  inequality}.
In other words, for every point $p\in\Omega $ there exists
some
$\delta>0$ such that the disc $\{|z-p|<\delta\}\subset \Omega$
and 
\[ 
u(p)\leq
\frac{1}{\pi r^2}\cdot \iint_{D_r(p)}\,
u(x,y)\cdot dxdy\quad\text{for all}\quad 0<r<\delta \tag{*}
\] 
where $D_r(p)$ is the disc of radius $r$ centered at $p$.
These local mean-value inequalities  make sense for
non-differentiable functions.
Thus, let $u(x,y)$ be a real-valued and continuous function
defined in some open set $\Omega$.
We say that $u$ satisfies the
local mean-value  inequality if  there for every
$p\in\Omega$ exists some $\delta>0$ with
$\delta < \text{dist}(p,\partial\Omega)$
such that (*) above holds.
This family of continuous 
functions is denoted by 
$\text{SH}_c(\Omega)$ where the prefix $c$ indicates that we 
restrict the attention to 
continuous functions.

\medskip


\noindent
{\bf {0.1 The majorant principle.}}
The local mean value inequality for a  function
$u$ in $ \text{SH}_c(\Omega)$ gives the following  majorisation principle.
Consider a pair $(U,h)$ where
$U\subset\Omega$ is an open subset and
$h$  a continuous function on the compact closure $\bar U$
and harmonic in $U$.
Then 
\[ 
u|\partial U\leq h|\partial U\implies
u\leq h\quad\text{in the whole set}\,\, \,\bar U\tag{**}
\]
The  proof of (**) relies upon the mean-value equality for harmonic functions
and solving the Dirichlet problm for a disc one gets 
the converse result, i.e. a continuous function $u$ in
$\Omega$ for which the majorisation  (**) holds belongs to
$\text{SH}_c(\Omega)$.
Hence there is     an equivalent condition for a
continuous function to be subharmonic !
\medskip

\noindent
{\bf{0.2 Logarithmic potentials.}}
Let $\mu$
be a non-negative Riesz measures with compact support in
${\bf{C}}$.
Since  the function
$\log\,|z|$ is locally integrable
there 
exists the convolution
integral:
\[ 
L_\mu(z)=\frac{1}{2\pi} \int\, \log \,|z-\zeta|\cdot d\mu(\zeta)\tag{***}
\]
We refer to $L_\mu$ as the logarithmic potential of $\mu$.
It can be attained as the pointwise limit of a decreasing
sequence of $C^\infty$-functions. Namely, when
$\epsilon>0$ we set
\[ 
L^\epsilon_\mu(z)= 
\frac{1}{4\pi}\int\, \log\bigl(|z-\zeta|^2+\epsilon\bigr )\cdot d\mu(\zeta)
\]
The reader should verify that
these functions are of class $C^\infty$ and with
$z$ kept fixed we have

\[
\frac{dL_\mu^\epsilon(z)}{d\epsilon}=
\frac{1}{4\pi}\int\, \frac{1}{|z-\zeta|^2+\epsilon}
\cdot d\mu(\zeta)
\]
It follows that $\{L_\mu^\epsilon\}$
is a decreasing sequence of $C^\infty$-functions whose
pointwise
limit function is  $L_\mu$.
Hence the $L^1_{\text{loc}}$-function $L_\mu$
takes  values 
in $[-\infty,+\infty)$.
and is 
upper semi-continuous.
Put
\[
\text{Polar}(\mu)=\{ z\colon\, \,L_\mu(z)=-\infty\}
\]
This polar set is the intersection of
the open sets
$\{ L_\mu<-N\}$ taken over all positive integer $N$ and is
therefore
a so called $G_\delta$-set. Moreover, its
2-dimensional Lebesgue measure is
zero since
$L_\mu$ is locally integrable.
It turns out that the polar set of a given non-negative measure$\mu$  
belongs to a more restricted family of sets than
the family of  null sets in the sense of Lebesgue. See
� XX from Special Topics which is devoted to
complex potential theory.
\medskip

\noindent
{\bf{0.3 A mean-value function.}}
With $\mu$ fixed we define for a given $r>0$ 
\[ 
M_{\mu,r}(z)=\frac{\pi}{r^2}\, \int_0^r\int_0^{2\pi}\,
 L_\mu(z+se^{i\theta})\cdot s  ds\cdot d\theta\tag{1}
\]
It means that we take the mean value of $U_\mu$ over the disc
of radius $r$ centered at $z$. The function of $z$ in (1) is continuous
for each fixed $r>0$.
Set
\[
\Phi_r(z,\zeta)=
\frac{\pi}{r^2}\, \int_0^r\int_0^{2\pi}\,
\log |z+se^{i\theta}-\zeta|\cdot s  ds\cdot d\theta\tag{2}
\]
Fubini's theorem gives
\[ 
M_{\mu,r}(z)=
\int\,\Phi_r(z,\zeta)\cdot d\mu(\zeta)\tag{3}
\]
It is clear that the $\phi$-function only depends upon $|z-\zeta|$. 
If $a$ is real and positive we set
\[
\phi_r(a)= 
\frac{\pi}{r^2}\, \int_0^r\int_0^{2\pi}\,
\log \,|a+se^{i\theta}|\cdot s  ds\cdot d\theta=
\]
\[
\log\,|a|+
\frac{a^2\pi}{r^2}\, 
\int_0^{\frac{r}{a}}\int_0^{2\pi}\,
\log |1+te^{i\theta}|\cdot t  dt\cdot d\theta\tag{4}
\]
where the last equality follows after the variable substitution
$s\to at$. Denote the last function in
(4) by $\psi_r(a)$. Then we have
\[ 
M_{\mu,r}(z)-U_\mu(z)=
\int\,\psi_r(|z-\zeta|)\cdot d\mu(\zeta)\tag{5}
\]
\medskip


\noindent{\bf{Exercise.}}
Consider the function defined for $s>0$ by
\[
g(s)=\frac{1}{\pi}\cdot \int_0^s\int_0^{2\pi}\,
\log |1+te^{i\theta}|\cdot t  dt\cdot d\theta\tag{i}
\]
Show that $g(s)=$ when $s\leq 1$ and if $s>1$
the reader may verify that 
\[
\frac{dg}{ds}=2\cdot s\log s\implies
g(s)=s^2\cdot \log s-\frac{s^2}{2}+\frac{1}{2}\quad\colon s>1\tag{ii}
\]
Now 
\[
\psi_r(|z-\zeta|= \frac{|z-\zeta|^2}{r^2}\cdot g(\frac{r}{|z-\zeta|})\tag{iii}
\]
Together (5) and (ii-iii)  express the non-negative  difference
$M_{\mu,r}(z)-U_\mu(z)$.
The reader may  analyze the
the non-decreasing function
\[ 
r\mapsto M_{\mu,r}(z)
\]
while $z$ is fixed. For example  use (ii) to
find the derivative with respect to $r$.
 


\bigskip

\noindent
Now we enlarge the class of  sub-harmonic functions to include
$U_\mu$-functions which arise from  non-negative measures $\mu$.
\medskip

\noindent
{\bf{0.4 Definition.}}
\emph{Let $\Omega$ be an open set in
${\bf{C}}$.
The the family of functions
$u\in L^1_{\text{loc}}(\Omega)$ for which the distribution
$\Delta(u)$ is a non-negative Riesz measure is denoted by 
$\text{SH}(\Omega)$.}




\bigskip
\noindent
{\bf{Remark.}} The condition for a locally integrable function to
be subharmonic can be phrased in another way where
we do not use distribution derivatives.
Namely, let 
$u(x,y)$ be a function in $L^1_{\text{loc}}(\Omega)$.
Denote by $\mathfrak{Leb}(u)$
the set of its Lebesgue points.
If $p\in\mathfrak{Leb}(u)$
and $0<r<\text{dist}(p,\partial\Omega)$
we consider  the mean value
\[ 
M_r(p)=\frac{1}{\pi r^2}\cdot \iint_{D_r(p)}\,
u(x,y)\cdot dxdy\tag{1}
\]


\noindent
We say that $u$ satisfies the local mean value inequality
if there to every Lebesugue point $p$
exists some $\delta>0$ such that
\[
u(p)\leq M_r(p)\quad\colon\quad 0<r<\delta\tag{2}
\]



\noindent
In � xx we prove that this is equivalent to the condition that
$u$ is subharmonic in the sense of Definition 0.4.

\medskip

\noindent
{\bf{0.5 The Riesz representation formula.}}
It turns out that every subharmonic function in the sense of Definition 0.4
is
locally expressed by the logarithmic potential of 
a Riesz measure plus some harmonic 
function.
More precisely, let $u$ be subharmonic in an open
set $\Omega$ and put $\mu=\Delta(u)$.
Let $\Omega_0$ be a relatively compact subset of $\Omega$
and denote by $K$ its compact closure.
Extending $\mu$ to be zero outside $K$ we get the compactly supported
measure $\mu_K$ and its logarithmic potential $L_{\mu_K}$.
In XXX we prove that there exists a harmonic function
$H$ in $\Omega_0$ such that the equality below holds in $\Omega_0$:
\[
 u=L_{\mu_K}+ H\tag{*}
\] 

\bigskip

\noindent
{\bf{0.6 Rado's inequality.}}
Riesz' representation formula gives an
a priori inequality which goes as follows:
Let $D$ be the open unit disc and
denote by $\mathcal F$ the class of subharmonic functions
$u$ in $D$ such that
$u(0)=0$ and $u(z)<1$ for all $z\in D$. In � xx we prove that
there exists a constant $C$ such that
\[ 
\iint_{|z|\leq 1/2}\, e^{-u(z)}\, dxddy\leq C\tag{0.6.1}
\]
for every $u\in\mathcal F$.
This gives a conrol on negative values taken by $u$.
For example, set
\[ 
A_n=\{n |z|\leq 1/2\}\cap\, 
\{-(n+1)\leq u(z)\leq -n\}
\]
Then
\[ 
\sum_{n=1}^\infty\, |A_n|_2\cdot e^n<\infty\tag{0.6.2}
\]
\medskip


\noindent
{\bf{0.7 Examples of subharmonic functions.}}
Let $f(z)$ be analytic in
the open unit disc where $f(0)=0$ and   $|f(z)|<1$
for all $z\in D$. To each $0<r<1$ there exists the  function
$\mathcal N_r(w)$ in the disc $|w|<1 $, defined by
\[
\mathcal N_r(w)=\sum\,\log\,\frac{r}{|\zeta|}\quad\colon
\text{sum taken over all }\, \zeta\in D_r\quad\colon \, f(\zeta)=w\tag{0.7.1}
\] 
In the sum 
one repeats zeros of $f(z)-w$ with their multiplicity.
In XXX we show
that $\mathcal N_r(w)$ is a subharmonic function in
the unit disc of the complex $w$-plane.
This class of  subharmonic functions
was introduced  by  Nevanlinna when he
developed the  value distribution theory for meromorphic functions.
See  [Nev. page xx-xx] which describes  the usefulness of
this class of subharmonic functions. 
Extending the   construction of these $\mathcal N$-functions
to
universal covering spaces of the image domains $f(D_r)$
these  subharmonic functions
can be used in   value distribution theory on Riemann surfaces where the interested reader
can consult
the article [Lehto] by O. Lehto for details.








\medskip













\newpage

\centerline {\bf 1. The subharmonic  Log-function}
\bigskip

\noindent
{\bf 1.0 The function $L_\epsilon$}.
For each $\epsilon>0$
we set
\[ 
F_\epsilon(x,y)=
\log(x^2+y^2+\epsilon)
\]

\noindent
The partial derivative with respect to $x$ becomes:
\[
\partial_x(F_\epsilon)=\frac{2x}{x^2+y^2+\epsilon}\quad\colon\quad
\partial^2_x(F_\epsilon)=
\frac{2}{x^2+y^2+\epsilon}-\frac{4x^2}{(x^2+y^2+\epsilon)^2}
\]
and similarly for the partial $y$-derivative. A summation of
the second order partial derivatives gives:
\[ 
\Delta(F_\epsilon)=\frac{4\epsilon}{(x^2+y^2+\epsilon)^2}\tag{i}
\]
The double integral over
$\bf R^2$ becomes:
\[ \iint\, 
\frac{4\epsilon}{(x^2+y^2+\epsilon)^2}\, dxdy=
\int_0^\infty\int_0^{2\pi}\,
\frac{4\epsilon}{(r^2+\epsilon)^2}\cdot rd\theta=
\]
\[
4\pi \epsilon\cdot \int_0^\infty\, \frac{2r dr}{(r^2+\epsilon)^2}=
-4\pi\epsilon \cdot \frac{1}{r^2+\epsilon}\bigr|_0^\infty=4\pi\tag{ii}
\]
Put
\[
L_\epsilon(x,y)= 
\frac{1}{2\pi}\cdot
\text{Log}\,\sqrt{x^2+y^2+\epsilon}=
\frac{1}{4\pi}\cdot F_\epsilon(x,y)
\]
Then (ii) entails  that its double integral is equal to one and (i) gives
\[
\Delta(L_\epsilon)=\frac{1}{\pi}\frac{\epsilon}{(x^2+y^2+\epsilon)^2}\tag{iii}
\]
With $z=x+iy$ we can write
\[
L_\epsilon(z)= \frac{1}{2\pi}\cdot
\log\,\sqrt{|z|^2+\epsilon}
\]
Outside the origin we get the limit formula:
\[ \lim_{\epsilon\to 0}\,
L_\epsilon(z)=
\frac{1}{2\pi} \cdot \log\,|z|\tag{*}
\]



\noindent
{\bf 1.1 $L_\epsilon$ as distributions.}
Let $\phi\in C_0^\infty({\bf{R}}^2)$  be a test-function.
Green's formula gives
\[
\iint \Delta(L_\epsilon)\cdot \phi\, dxdy=
\iint L_\epsilon\cdot \Delta(\phi)\, dxdy\quad\colon\quad\epsilon>0\tag{1}
\]
The left hand side has a limit as $\epsilon\to 0$. Namely,
(iii) above gives:
\[ 
\lim_{\epsilon\to 0}\,
\frac{1}{2\pi}\iint
\frac{\epsilon}{(x^2+y^2+\epsilon)^2}\cdot\phi(x,y) \,dxdy=
\phi(0)\tag{ii}
\] 
The reader should confirm the limit formula (ii)
by computing the integral in
polar coordinates.
In distribution theory this is expressed as follows:
\bigskip

\noindent {\bf 1.2 Theorem.}
\emph{The distribution densities
$\Delta(L_\epsilon)$ converge to the unit point mass}
$\delta_0$. 
\medskip

\noindent
{\bf Remark.} Above we  constructed a \emph{regularisation}
of the Dirac measure.
Using (*) it means that 
the locally integrable function
 $\log\,|z|$ considered as a distribution is such that
 its Laplacian - taken in the distribution sense - is equal to
 $2\pi i\cdot \delta_0$. 
 One therefore says that
 $\frac{1}{2\pi}\cdot \log |z|$ is a fundamental solution to the
 $\Delta$-perator.
 \medskip
 
 
 \noindent
 Next,
 let $\mu$ be a Riesz measure in $\bf R^2$ with a compact support
which
defines a
distribution by
\[ 
\phi\mapsto \int \phi\cdot d\mu
\]
To each $\epsilon>0$ we construct the convolution
\[
L_\epsilon*\mu(x,y)=
\int\,
L_\epsilon(x-t,y-s)\cdot d\mu(t,s)
\]
Here $L_\epsilon*\mu$ are $C^\infty$-functions and
taking the Laplacian we get
\[
\Delta(L_\epsilon*\mu)(x,y)=
\frac{1}{2\pi}\int\,\frac{\epsilon}{(x-t)^2+(y-s)^2+\epsilon}
\cdot d\mu(t,s)
\]
Next, when $\phi\in C_0^\infty$ we perform integration with
respect to $(x,y)$ and obtain
\[
\iint\,
\Delta(L_\epsilon*\mu)(x,y)\cdot\phi(x,y)dxdy=
\frac{1}{2\pi}\int\,\bigl[
\iint\,\frac{\epsilon\cdot\phi(x,y)dxdy}{(x-t)^2+(y-s)^2+\epsilon}\,\bigr]
\cdot d\mu(t,s)
\]
\medskip

\noindent 
Here
the inner double intergal is a function of
$(s,t)$ which converges uniformly to $\,\phi$
as $\epsilon\to 0$. Hence a passage to the limit gives
\[
\lim_{\epsilon\to 0}\,
\iint\,
\Delta(L_\epsilon* \mu)(x,y)\cdot\phi(x,y)\,dxdy=
\int\,\phi\cdot d\mu
\]


\noindent
This is expressed by saying the the distribution
densities
$\Delta(L_\epsilon* \mu)$ converge to the distribution defined by
$\mu$.
Next, recall  from  distribution theory that convolution commutes
with
differentiation. Hence we get
\[
\lim_{\epsilon\to 0}\,
\Delta(L_\epsilon*\mu)=
\lim_{\epsilon\to 0}\,
\Delta(L_\epsilon)*\mu
=\mu\tag{*}
\]
where the last equality follows from Theorem 1.2.
\bigskip

\noindent
{\bf 1.3 The logarithmic potential.}
Recall  from measure theory that one can define the convolution of a compactly
supported 
Riesz measure $\mu$  with an $L^1$-functions. 
We apply this with the locally integrable function
$\log\sqrt{x^2+y^2}$ which in
complex notation  is written as $\log\,|z|$.
The convolution integral 
\[
U_\mu(z)=\frac{1}{2\pi}
\int\,\log\,|z-\zeta|\cdot d\mu(\zeta)
\]
is called the logarithmic potential of the Riesz measure $\mu$. Notice that
$U(z)$ belongs to 
$L^1_{\text{loc}}{\bf{C}})$. 
The  limit formulas from Theorem 1.2 and 
(*) above   yield
\medskip

\noindent
{\bf 1.4 Theorem.}
\emph{The Laplacian of $U_\mu$ taken in the distribution sense is
equal to $\mu$.}
\bigskip

\noindent
{\bf Remark.}
To confirm Theorem 1.4 we consider a test-function
$g$. Fubini's theorem gives the equality
\[
\frac{1}{2\pi}\cdot \int\,\Delta\, g(z)\cdot\bigr]
\,\int\,\log\,|z-\zeta|\cdot d\mu(\zeta)\,\bigr]\cdot dxdy=
\frac{1}{2\pi}\cdot \bigl[\int\,\Delta\, g(z)\cdot\log\,|z-\zeta|\cdot dxdy\,\bigr]
\cdot 
d\mu(\zeta)
\]
By Theorem 1.2 the last term becomes
\[ 
\int\, g(\zeta)\cdot d\mu(\zeta)
\]
Hence the definition of distribution derivatives gives
Theorem 1.4. 


\noindent
Next, recall that $\partial=\frac{1}{2}(\partial_x-i\partial_y)$.
A differentiation gives:
\medskip
\[
\partial(L_\epsilon)=
\frac{1}{4\pi}\cdot\frac{x-iy}{x^2+y^2+\epsilon}=
\frac{1}{4\pi}\cdot\frac{\bar z}{|z|^2+\epsilon}
\]
So outside the origin we get the limit formula
\[ 
\lim_{\epsilon\to 0}\,
\partial(L_\epsilon)(z)=
\frac{1}{4\pi z} \tag{**}
\]




\bigskip

\centerline {\bf 1.5 The Cauchy transform.}
\medskip

\noindent
The function $\frac{1}{z}$ is locally integrable
and can therefore be convolved with
a compactly supported Riesz measure.
Put
\[
\mathcal C_\mu(z)=
\int\,\frac{d\mu(\zeta)}{z-\zeta}
\]
Since this is a convolution of a measure 
with compact support and the locally integrable function
$\frac{1}{z}$
it belongs to 
$L^1_{\text{loc}}$. We refer to
$\mathcal C_\mu$ as the Cauchy transform of $\mu$.
Notice that $\mathcal C_\mu$
is an analytic function outside the
support of $\mu$. For example, its complex
derivative becomes
\[
\frac {d\mathcal C_\mu(z)}{dz}=
-\int\,\frac{d\mu(\zeta)}{(z-\zeta)^2}
\]



\centerline
{\bf 1.6 The equality
$\bar\partial(\mathcal C_\mu)=\pi\cdot \mu$.}
\medskip

\noindent
Recall from XX that the Laplacian $\Delta$
can be expressed as the product of the
first
order differential operators
\[
\partial=\frac{1}{2}(\partial_x-i\partial_y)\quad\colon\quad
\bar \partial=\frac{1}{2}(\partial_x+i\partial_y)
\]
More precisely
we have
\[
\Delta=4\cdot \bar \partial_z\cdot \partial_z
\]

\noindent
{\bf{Exercise.}}
Apply $\partial$ to the locally integrable function $U_\mu(z)$
and show the equality
\[
\partial(U_\mu)=\frac{1}{4\pi}\cdot\mathcal C_\mu
\]
By Theorem 1 .2 we also have
\[
\mu=\Delta(U_\mu)=4\cdot\bar\partial(\partial U))=
4\cdot \frac{1}{4\pi}\cdot \bar\partial(\mathcal C_\mu)
\]
\medskip

\noindent
From the above  the equality (1.6) follows.
Since it
is so important we state
\medskip

\noindent
{\bf 1.7 Theorem.}
\emph{One has the equality}
\[
\bar\partial(\mathcal C_\mu)=\pi\cdot\mu
\]


\noindent
{\bf Example.}
Let $\mu=\delta_0$ be the unit mass at the origin. In this case 
$\mathcal C_\mu(z)= \frac{1}{z}$ and   the definition of
distribution derivatives 
means that
\[
g(0)=
-\frac{1}{\pi}\int
\frac{\bar\partial(g)\, dxdy}{z}\quad\colon\quad g\in C_0^\infty({\bf{C}})\tag{i}
\]
\medskip

\noindent
Recall from XX that $dz\wedge d\bar z=-2i\cdot dxdy$. So 
the minus sign
is changed and the right and side
becomes
\[
\frac{1}{2\pi i}\int
\frac{\bar\partial(g) \cdot dz\wedge d\bar z}{z}\quad\colon\quad g\in C_0^\infty({\bf{C}})\tag{ii}
\]
Since $\frac{1}{z}$ is locally integrable this integral is equal to
\[
\lim_{\epsilon\to 0}
\frac{1}{2\pi i}\int_{|z|>\epsilon}
\frac{\bar\partial(g) \cdot dz\wedge d\bar z}{z}\quad\colon\quad g\in C_0^\infty({\bf{C}})\tag{iii}
\]
Now we regard the differential 1-form
$\alpha=\frac{g(z)\cdot dz}{z}$ and as explained in XX we have
\[ 
d\alpha=
\frac{\bar\partial(g)\cdot d\bar z\wedge dz}{z}=
-\frac{\bar\partial(g)\cdot dz\wedge d\bar z}{z}
\tag{iv}
\]
where we used the the exterior product of two 1-forms
is anti-commutative.
When Stokes formula is applied
the outer normal with respect to the exterior domain is minus the
usual outer normal with respect to the open disc
$|z|<\epsilon$. So when Stokes Theorem is applied to 
differential $1$-form $\alpha$ we see that (iii) above is equal to
\[
-\frac{1}{2\pi i}\cdot \lim_{\epsilon\to 0}
\int_{|z|>\epsilon}\,d\alpha=
\frac{1}{2\pi i}\cdot \lim_{\epsilon\to 0}
\int_{|z|=\epsilon}\,\frac{g(z)dz}{z}=g(0)\tag {v}
\]
This confirms the equality in Theorem 1.7 when
$\mu=\delta_0$.



\newpage


\centerline {\bf 1.8 Approximation theorems}
\medskip

\noindent
We shall apply Theorem 1.7 to deduce certain approximation theorems.
Let $K$ be a compact null set in ${\bf{C}}$, i.e. its 2-dimensional Lebesgue measure is zero.
We have the Banach space $C^0(K)$ of continuous and complex
valued functions on $K$. If $z\in {\bf{C}}\setminus K$ the rational function
$\frac{1}{z-\zeta}$ restricts to a continuous function on $K$.
Taking finite linear combinations we get functions
\[ 
\zeta\mapsto\,\sum\, c_\nu\frac{1}{z_\nu-\zeta}\quad
\colon z_1,\ldots,z_N\in {\bf{}}\setminus K\quad
\colon c_1,\ldots,c_N\in {\bf{C}}
\]
Denote by $R(K)$ the closure of this linear subspace of
$C^0(K)$. Thus, functions in $R(K)$ consist of continuous functions on
$K$ which can be uniformly approximated by rational functions with poles
outside $K$.
\medskip

\noindent {\bf 1.9 Theorem.}
\emph{For every compact null set $K$ one has
the equality $C^0(K)=R(K)$.}
\medskip

\noindent \emph{Proof.}
Suppose that $R(K)\neq C^0(K$. Riesz' representation formula
gives the existence of a non-zero  measure $\mu$ supported by
$K$ such that $\mu\perp R(K)$.
Consider the Cauchy transform

\[
\mathcal C_\mu(z)=
\int_K\,\frac{d\mu(\zeta)}{z-\zeta}
\]
Since $\mu\perp R(K)$ it is identically zero outside $K$.
Now $K$ is a null set so the $L^1_{\text{loc}}$-function
$\mathcal C_\mu$ is identically zero and so is its distribution
derivative $\bar\partial(\mathcal C_\mu)$. But then we get a contradiction from
Theorem 1.7  since this distribution derivative must recapture $\pi\cdot \mu$.
which by assumption is non-zero.

\bigskip


\noindent
{\bf Remark.}
The proof of Theorem 1.8 
relies upon the Hahn-Banach theorem which
gives the existence of a non-zero Riesz measure carried by
$K$ when  $R(K)\neq C^0(K)$.
The drawback of this proof 
is that it does not give any hint about how one actually approximates
a given continuous function on $K$ by rational 
functions having poles in the complement.
So instead of the easy proof above which
is based on an "argument by contradiction"
one would like to have a
\emph{constructive proof} of Theorem 1.7, i.e.
given a null-set $K$ one may ask for some sort of
algorithm to approximate every given
continuous function on $K$.
This point of view was put forward by E. Bishop
in his book [Bish]. His objection to
the proof above  is not just a
question of taste and philosophy. In fact, Erret Bishop
is one of the most prominent analysists in complex function theory.
It is therefore good to keep in mind
that various theoretical results do not give the whole story if one really
wants to apply them in more concrete situations.
\bigskip


\noindent {\bf 1.10 Megelyan's swiss-cheese.}
Following [Merg] we  construct a compact set
$K$ in $C$ with empty interior where
$R(K)\neq C^0(K)$.
Take the unit disc $D$ and remove
a finite number of discs $D_1,\ldots,D_N$ inside
$D$. They are chosen so that the closed discs $\{\bar D_\nu\}$
are pairwise disjoint and stay inside $D$.
For each $1\leq\nu\leq N$ we let $\mu_\nu$ be the measure
supported by
$\partial D_\nu$ and given by
$d\zeta$, i.e. if $D_\nu$ has a radius $r_\nu$
it is simply the measure whose Cauchy transform becomes
\[ 
\mathcal C_\nu(z)=\int_0^{2\pi}\,\frac {ir_\nu e^{i\theta}}
{z-(a_\nu+r_\nu)e^{i\theta}}\cdot d\theta
\]
Notice that $\mathcal C_\nu(z)=0$ outside $\bar D_\nu$. Indeed, this follows from the trivial observation from XXX.
Next, on $\partial D$
we get the measure $\mu^*$ defined by $d\zeta$ restricted to $|\zeta|=1$.
Let $\mathcal C^*(z)$ denote its Cauchy transform which now
is identically zero outside $D$. In addition to this it has
a compensating influence. For if $z_0\in D_\nu$ for some
$\nu$, we have by Cauchys formula
\[
\int_{\partial D_\nu}\,\frac{d\zeta}{z_0-\zeta}=
\int_{\partial D}\,\frac{d\zeta}{z_0-\zeta}
\]
Consider the measure
\[ 
\rho_N= \mu^*-(\mu_1+\ldots+\mu_N)
\] 
The previous observations  show that
$\mathcal C_{\rho_N}$ is zero 
in the set $\Omega_N= \cup\, D_\nu\,\cup \,|\zeta|>1$.
In other words, if 
\[
K_N=\bar D\setminus\,D_1\cup\ldots\cup\,  D_N
\] 
then this compact set is the support of  the
$L^1$-function $\mathcal C_{\rho_N}$.
\medskip

\noindent
At this stage we see how one should proceed to construct a
swiss-cheese. Namely, inside $D$ we construct a
denumerable sequence of discs $D_1,D_2,\ldots$ so that
the compact set

\[
K=\bar D\setminus\,\cup\,D_\nu
\]
has no interior points, i.e.
just make sure that the center points of the discs
$\{D_\nu\}$ appears as a dense subset of $D$. Moreover,
let $r_\nu$ be the radius of  $D_\nu$ and perform the construction so that
\[
\sum\, r_\nu<\infty
\]
The total variation of $\mu_\nu$ becomes $2\pi \cdot r_\nu$ and hence we get
a measure
\[ 
\rho_*=\mu^*-\sum_{\nu=1}^\infty\,\mu_\nu
\]
By the construction it is clear that $\rho_*\perp R(K)$
and hence $R(K)\neq C^0(K)$.
\medskip

\noindent
{\bf 1.11 Wermer's example.}
In [We ] appears  a Jordan arc $\Gamma$ 
whose 2-dimensional Lebesgue measure is positive whose
 existence 
goes back to work by Peano.
The restriction of polynomial $P(z)$
to  $\Gamma$ gives a ${\bf{C}}$-subalgebra of
$C^0(\Gamma)$. Let $P(\Gamma)$ be its uniform closure.
Then 
\[ 
P(\Gamma)\neq C^0(\Gamma)
\]
This inequality relies upon results
about analytic capacity.
The reader may consult the book [We] about uniform algebras for
a detailed account about this example. 
See also the text-book [Ga] by T. Gamelin which  is
devoted to uniform algebras. 







\newpage


\centerline{\bf \large 2. Subharmonic functions}
\bigskip

\noindent
If $\Omega$ is an open subset of $\bf C$ we denote by
$\text{SH}^2(\Omega)$ the class of 
$C^2$-functions $u$  in
$\Omega$ such that the continuous function
$\Delta(u)$ is non\vvv negative.
If $u\in \text{SH}^2(\Omega)$ and 
$\Omega_0$ is a domain in the class
$\mathcal D(C^1)$ which appears as a   relatively compact in $\Omega$, then
Green's formula  gives:

\[ \iint_{\Omega_0}\, \Delta(u)\,dxdx=
\int_{\partial\Omega_0}\, u_{{\bf{n}}}\cdot  ds
\]
Hence the subharmonicity entails that the integral of the outer normal
derivative  is $\geq 0$ for any such domain $\Omega_0$.
Consider  the  case when $\Omega=D_R$ is a disc 
centered at the origin and  $\Omega_0=D_r$ for some $r<R$.
In polar coordinates we get
\[
\int_{\partial D_r}\, u_{{\bf{n}}}\cdot  ds=
\int_0^{2\pi}\,
[\text{cos}\,\theta\cdot u_x+\text{sin}\,\theta\cdot u_y]\cdot rd\theta
\]
Next, define the mean-value function
\[ 
M_u(r)=\frac{1}{2\pi}\cdot
\int_0^{2\pi}\, u(r,\theta)\cdot d\theta
\]
Since $\frac{d}{dr}(u(r,\theta)=\text{cos}\,\theta\cdot u_x+\text{sin}\,\theta\cdot u_y$
we obtain
\[
\frac{d}{dr}(M_u(r))=
\frac{1}{\pi r}\int_{\partial D_r}\, u_n ds=
\frac{1}{2\pi r}\cdot
 \iint_{D_r}\, \Delta(u)dxdx
\]
\medskip

\noindent
Hence the function $r\mapsto M_u(r)$ is non-decreasing and
when $\Delta(u)>0$ it is even strictly increasing.
Since $u$ is continuos we have
\[
\lim_{r\to 0}\, M_u(r)=u(0,0)
\] It follows that $u$ satisfies the mean-value inequality
\[
u(0)\leq M_u(r)\quad\colon\quad 0<r<R\tag{*}
\]


\noindent {\bf 2.1 Harmonic majorization.}
Let $u\in\text{SH}^2(\Omega)$. If 
$\delta>0$ and $u_\delta(x,y)=u(x,y)+\delta(x^2+y^2)$
then $\Delta(u_\delta)=\Delta(u)+4\delta>0$.
Since $u_\delta\to u$ when $\delta\to 0$ we can always
approximate a subharmonic function by a 
decreasing sequence of strictly subharmonic functions.
Next, recall the following result from
from \emph{Calculus}. Let  $f(x,y)$ be a 
$C^2$-function defined in some open subset
of $\bf R^2$  which has a (not necessarily strict)
maximum at some point  $(x_0,y_0)$ , i.e. 
\[ 
f(x,y)\leq f(x_0,y_0)\quad\colon\quad (x-x_0)^2+(y-y_0)^2<\epsilon
\] 
hold for some small 
$\epsilon$. Then the \emph{Hessian} of $f$ at $(x_0,y_0)$ must
be negative semi-definite. In particular
the trace 
$f_{xx}+f_{yy}=\Delta(f)\leq 0$.
This elementary facts gives
\medskip

\noindent {\bf 2.2 Proposition.}
\emph{Let
$u$ be a strictly subharmonic function of class $C^2$
defined in  an open set
$\Omega$. Then $u$ cannot have any
local maximum in $\Omega$.
Thus, if $U$ is a relatively compact subset of
$\Omega$ then $u$ takes its maximum on the boundary of $U$, i.e.}
\[
\max_{\bar U}\, u=
\max_{\partial U}\,u
\]
\bigskip


\noindent
Next, Proposition 2.2.  together with
the mean-value property of harmonic functions give the following:
\bigskip

\noindent
{\bf 2.3 Theorem.}
\emph{Let
$u\in\text{SH}^2(\Omega)$
and let $h$ be a harmonic function in
$\Omega$. Then the following implication hold
for every relatively compact open subset $U$ of $\Omega$;}
\[
u\leq h\,\,\text{on}\,\, \partial U\,\,\implies
u\leq h\,\,\text{on}\,\, U
\]
\medskip

\noindent 
\emph{Proof.}
Follows from Proposition 2.2 since
$u-h$ is subharmonic.
We refer to this as the principle of
\emph{harmonic majorization}.


\newpage


\centerline {\bf 2.4 Subharmonic functions in $L^1_{\text{loc}}$.}
\medskip


\noindent
Now we  relax the $C^2$-hypothesis.
Let $u\in L^1_{\text{loc}}(\Omega)$
where  $u$ is real-valued.
We get the distribution
$\Delta(u)$ and
impose the condition that it is equal to a non-negative
Riesz measure $\mu$.
By the definition of distribution derivatives
this means that
\[
\iint _\Omega\, \Delta(\phi)\cdot udxdy=
\iint _\Omega\, \phi\cdot d\mu\quad\colon\quad \phi\in C_0^\infty(\Omega)\tag{*}
\]
\medskip

\noindent {\bf 2.5 Definition.}
\emph{A function $u$ in 
$L^1_{\text{loc}}(\Omega)$ is called subharmonic if
the distribution $\Delta(u)\geq 0$. The class of subharmonic 
functions in $\Omega$
is denoted by
$\text{SH}(\Omega)$.}
\bigskip

\centerline {\bf 2.6 Regularisations.} 
\medskip

\noindent
Definition 2.5  is a bit abstract since it is not
easy
to discover the distribution $\Delta(u)$ when $u$ is 
just assumed to be locally integrable.
So we shall find other conditions in order that
a function in $L^1_{\text{loc}}$ is subharmonic.
For this  we  use  regularisations.  In general, let
$u\in \text{SH}(\Omega)$ where $\Omega$ is a bounded
open set.
If $\delta>0$ we set
\[ 
\Omega[-\delta]=\{z\in\Omega\,\colon\,
 \text{dist}(z,\partial\Omega)\geq\delta\}
\]
Notice that
$\Omega[-\delta]$ is a compact subset of
$\Omega$.
For every test-function $\phi$ with compact support in
the  disc $D_\delta$ the convolution
$\phi*u$ exists in 
 $\Omega[-\delta]$. By the general formula from
 XXX we have
\[
 \Delta(\phi*u)= \phi*\Delta(u)=\phi*\mu
 \]



\noindent
where $\mu$ by assumption is a non-negative Riesz measure.
\medskip

\noindent
{\bf 2.7 The case when $\phi$ is radial.}
We shall use
test-functions which depend on
$x^2+y^2$ only. Let us recall the construction.
Start from a test-function $\phi_*(z)$
which is $>0$ in $|z|<1$ and has compact support in $|z|\leq 1$
and depends on $|z|$ only while
\[ 
\iint_D\, \phi_*(z)dxdy=1
\]
Then, to every $0<\delta<1$ we get the test-function
\[
\phi_\delta(z)=\frac{\phi_*(\frac{z}{\delta})}{\delta^2}
\]
which has support in  $\bar D_\delta$. Next, recall from XXX
that
for any
$L^1_{\text{loc}}$-function $f$
it follows that the convolution $\phi_\delta*f$ is a $C^\infty$.function.
We apply this with
$u$ above and conclude that 
$\phi_\delta*u$ is a $C^\infty$-function defined in some open
neighborhood of
 $\Omega[-\delta]$. By (*) in 2.6 we have
 \[ 
 \Delta(\phi_\delta*u)= \phi_\delta*\mu\tag{i}
 \]
Since both $\mu$ and $\phi_\delta$ are $\geq 0$, it follows that
the convolution is $\geq 0$. Hence
the Laplacian of the $C^\infty$-function
$\phi_\delta*u$ is $\geq 0$ so we can apply the results from
the $C^2$-case. In particular
$\phi_\delta*u$ satisfies the mean-value inequality
\[
\phi_\delta*u(p)\leq\,\frac{1}{\pi r^2}\cdot \int_{D_r(p)}
\phi_\delta(z-p)\cdot u(z)\cdot dx dy
\quad\colon\, p\in \Omega[-2\delta]\quad\colon\quad
0<r<\delta
\]


\noindent
If $p\in \Omega[-2\delta]$ is a Lebesgue point for
$u$ we can pass to the limit as $\delta\to 0$
and conclude that
\[
u(p)\leq\,\frac{1}{\pi r^2}\cdot \int_{D_r(p)}
u(z)\cdot dx dy
\quad\colon\, p\in \Omega[-2\delta]\quad\colon\quad
0<r<\delta\tag{*}
\]
\medskip

\noindent
This shows that $u$ satisfies the local mean-value inequality in
$\Omega[-\delta]$.
Since $\delta$ an be arbitrary small, it follows that
$u$ satisfies the local mean-value inequality in
the whole of $\Omega$, i.e. we have proved:
\medskip

\noindent
{\bf 2.8 Theorem.} \emph{Let $u\in\text{SH}(\Omega)$.
Then the following holds for
each Lebesgue point of $u$ in $\Omega$:}
\[
u(p)\leq \frac{1}{\pi r^2}\cdot \int_{D_r(p)}
u(z)\cdot dx dy
\quad\colon\, 0<r<\text{dist}(p,\partial\Omega)
\tag{*}
\]
 
\newpage

\noindent
{\bf Remark.} Above  we have recovered the  definition of subharmonic
functions from the introduction
via 
Definition 2.5.
The \emph{converse} also holds, i.e. if we 
from start assume that the $L^1_{\text{Loc}}$-function $u$
satisfies the local mean value inequality then
it is subharmonic in the sense of Theorem 2.8.
\medskip


\noindent{\bf 2.9 Exercise.}
Prove the converse. The hint is that
if
$\phi_\delta$ as above are radial test-functions and
(*) is assumed, then
the local mean value inequality hold for
$\phi_\delta*u$. Here we have $C^2$-functions and by
Green's formula one shows that $\delta(\phi_\delta*u)\geq 0$
follows. Finally one takes the limit as $\delta\to 0$
and the reader should now confirm that
$\delta(u)\geq 0$ holds in the distribution sense.
Show also that
this family of functions is monotone, i.e. verify the following:
\medskip

\noindent
{\bf 2.10 Proposition.} \emph{Let $u\in\text{SH}(\Omega)$. 
Then the sequence of functions $\{\phi_\delta*u\}$ decrease, i.e.}
\[ 
\phi_{\delta_1}*u(p)\leq
\phi_{\delta_2}*u(p)\quad\colon\quad \delta_1<\delta_2
\] 
\emph{Moreover, this decreasing sequence converges almost everywhere to
the measurable function $u$.}
\medskip

\noindent
Proposition 2.10 implies 
$u$ is almost everywhere equal to the pointwise limit if
a monotone sequence of continuous functions and therefore
we can always  take $u$ to be an upper semi-continuous function.
The set where it  becomes $-\infty$ is a null set.
Thus, every subharmonic function enjoys similar properties as
logarithmic potentials from � 1.



\newpage

\centerline{\bf 3. Riesz representation formula}

\bigskip


\noindent
Let $u\in\text{SH}(\Omega)$ where 
$\Omega$ is a bounded open set.
Now $\Delta(u)$ exists as a distribution
and we have by assumption 
\[
\iint\,\Delta(\phi)\cdot u\, dxdy\geq 0\quad\colon\,\phi\in C_0^\infty(\Omega)\tag{*}
\]


\noindent
We can take regularisations of $u$ as in � 2, i.e. construct
convolutions
$\phi_k*u$ where $\{\phi_k\}$ is a sequence of
non-negative test functions with smaller and
smaller compact support
in some disc $|x|\leq\delta$ while their integrals are one for every $k$.
Let $g\in C_0^\infty(\Omega)$  with compact support in
$\Omega[-\delta]$.
Green's formula gives:
\[
\iint\,\Delta(g)\cdot \phi_k*u\,dxdy=
\iint g\cdot \Delta(\phi_k*u)\, dxdy\tag{**}
\]
Since $u$ is subharmonic the functions
$\Delta(\phi_k*u)\geq 0$ in $\Omega[-\delta]$.
Hence they become  non-negative measures. Let us fix a compact subset 
$K$�in $\Omega[-\delta]$. 
For example, we can take
\[
K=\text{closure of}\,\,\Omega[-2\delta]
\]
Now we can regard the \emph{total mass}
\[ 
\rho_k=\iint_K\,
\Delta(\phi_k*u)\, dxdy
\]


\noindent
{\bf 3.1. An inequality.}
Since $u$ is subharmonic the mean-value inequality from � XX
gives the inequality:
\[ 
\phi_k*u(z)\leq u(z)\quad\colon\quad z\in \Omega[-2\delta]
\]
From this we conclude that if $g\in C_0^\infty(\Omega)$
is non-negative and identically one on
$\Omega[-2\delta]$ then (**) above gives 
\[
\rho_k\leq \int\,\Delta(g)\cdot u\, dxdy\quad\colon\quad k=1,2,\ldots\tag{i}
\]


\noindent
Together with a general result about
positive distributions in to be proved in � XX it follows that:

\medskip

\noindent
{\bf 3.2. Proposition.}
\emph{There exists 
a constant $C_K$ such that}
\[ 
\rho_k\leq C_K\quad\colon, k=1,2,\ldots
\]


\noindent 
The uniform bound in Proposition 3.2
gives the existence of 
a subsequence of the non-negative measures
$\{\Delta(\phi_k*u)\}$
which converges weakly to  non-negative 
Riesz measure $\mu$ in $K$.
At the same time we notice that
\medskip
\[ \lim_{k\to\infty}\,\phi_k*u\to u
\] 
where the
limit holds in $L^1$. Hence we have proved:
\bigskip

\noindent
{\bf 3.3. Proposition} \emph{For every
test-function $g\in C_0^\infty(\Omega)$ with
support contained in $K$ one has:}
\[ 
\iint \Delta(g)\cdot u\, dxdy=
\int\, g\,d\mu
\]
\medskip

\noindent 
{\bf 3.4. Constructing  Log-potentials.}
With $\delta>0$ we construct 
a test-function $\chi$ satisfying
\[
\chi=1\,\,\text{in}\,\,\,\Omega[-3\delta]\quad\colon\,
\chi\in C_0^\infty(\Omega[-\delta]
\]

\noindent
Next,
we use the $C^\infty$-functions
$L_\epsilon$ from � 2 in V:A and keeping  $\delta>0$ fixed we
define the functions:
\[ 
g_\epsilon=\chi\cdot L_\epsilon\quad\colon\,\quad 0<\epsilon<\delta\,\quad
\]

\noindent 
Passing to the
limit as $\epsilon\to 0$ while
$\chi$ is kept fixed, Proposition 3.3 and
Theorem 1.4 imply that  the function
\[ 
w(z)=u(z)-\int\,\log \,|\zeta-z|\,\cdot d\mu(\zeta)
\quad\colon\, z\in \Omega[-3\delta]
\]
has  a Laplacian in the \emph{distribution sense} which is
equal to zero in $\Omega[-3\delta]$.
\newpage

\noindent{\bf 3.5. Conclusions.}
So are we have not proved anything definitive.
But we have demonstrated that
one should regard two problems.
The first  is to explain why the
$\rho$-numbers stay bounded, i.e. to verify
Proposition 3.2.  The second   is 
to show that
if $\phi$ is some $L^1_{\text{loc}}$-function such that
$\Delta(\phi)=0$ holds in the distribution sense, then
$\phi$ is \emph{automatically} a nice function, i.e. at 
least $C^2$ and hence harmonic.
If this has been achieved  
the results  above show
that the  subharmonic function $u$
is represented as a logarithmic potential of a measure plus a
harmonic functions inside $\Omega[-3\delta]$. Since $\delta>0$
can be made arbitrary small this gives  a representation in any relatively
compact subset of $\Omega$. So there remains to establish two
general results from distribution theory.



\bigskip

\centerline {\bf 3.6. Positive distributions.}
\medskip

\noindent
Consider an open  square $\square=\{(x,y)\colon\,\, 0< x,y<1\}$.
Let $\mathcal L$ be a linear form on 
$C_0^\infty(\square)$ and assume that there exists
some 
integer $k\geq 0$ and a constant $C$ such that
\medskip
\[ 
|L(g)|\leq C\cdot ||g||_k\quad\colon\,g\in C_0^\infty(\square)\,\, \text{where}\,\,
||g||_k=\text{norm in}\,\, C^k(\square)
\]


\noindent We say that $L$ is positive if
\[ 
g\geq 0\implies L(g)\geq 0
\]


\noindent {\bf 3.7 Theorem.}
\emph{Let $L$ be defined and positive as above.
Then, for every $0<r<1$
there is a constant
$C_r$ such that}
\[
|L(g)|\leq C_r\cdot ||g||_0\quad\colon\text{Supp}(g)\subset\square_r
\]


\noindent 
\emph{Proof}
Given $r<1$ we construct
$\phi\in C_0^\infty(\square)$ where $\phi=1$ on
$\square_r$ and is non-negative.
Now, if $g$ ha support in $\square_r$
it follows that the function
\[ ||g||_0\cdot\phi-g\geq 0
\]
Since $L$ is positive we get
\[ L(g)\leq ||g||_0\cdot L(\phi)
\]
So we can take $C_r=L(\phi)$ and  Theorem 3.7 follows.




\bigskip

\centerline{\bf 3.8. The elliptic property of $\Delta$.}
\medskip

\noindent
Let $w\in L^1_{\text{loc}}(\Omega)$ for some bounded open set.
Assume that
\[
\iint\,\Delta(g)\cdot w\, dxdy=0\quad\colon\, g\in C_0^\infty(\Omega)
\]


\noindent {\bf 3.9 Theorem.} \emph{Under the assumption above 
$w$ is a harmonic function in $\Omega$.}


\bigskip

\noindent 
\emph{Proof.} We use similar regularisations as above.
With $\delta>0$
we choose the sequence $\{\phi_k\}$
and now 
$\phi_k*w\in C^\infty(\Omega[-\delta]$.
Since convolution commutes with
$\Delta$, it follows that
these functions are harmonic in
$\Omega[-\delta]$.
Moreover, since $w$ b assumption has a finite $L^1$-norm over
the relatively compact subset $\Omega[-\delta]$,
the $L^1$-norms of
$\{\phi_k*w\}$ are uniformly bounded in
$\Omega[-2\delta]$, i.e. we have a constant $C$ so that
\[ 
\iint_{\Omega(-2\delta)}
|\phi_k*w|\,dxdy\leq C
\]


\noindent
\emph{Poisson's formula} implies that we get a uniform bound for the
\emph{maximum norms}
in $\Omega[-3\delta]$, i.e. with another constant $C_1$ one has
\[
\max\,|\phi_k*w(z)|\leq C_1\quad\colon\, z\in\Omega[-3\delta]
\]
At this stage we apply Montel's results for normal families
of harmonic functions in
XX. Passing to a subsequence if necessary, it follows that
\[ 
\lim_{k\to\infty}\,\, \phi_k*w= G\,\,\text{holds uniformly
in compact subsets of}\,\, \Omega[-3\delta]
\]
where the limit function $G$ is harmonic. At the same time
$w\in L^1_{\text{loc}}$. From Lebesgue theory we know that
$\phi_k*w\to w$ and hence $w$ must be equal to  the "true" harmonic function
$G$ in $\Omega(-\delta)$. Since $\delta$ can be arbitrary small we conclude that
$w$ is a true harmonic function in the whole of $\Omega$.



\bigskip

\noindent{\bf 3.10 Remark.}
Above we gave a "pedestrian proof" which 
could have been given
in a quicker way if one admits further results
in distribution theory. Moreover,
the elliptic property of $\Delta$
holds for distributions, i.e. if $w$ is replaced by \emph{any} distribution
$\mu$ defined in some open set $\Omega$ where
$\Delta(\mu)=0$  holds in the sense of distributions, then
$\mu$ is a "true" harmonic function.
Thus can be shown by using regularisations as above.
Namely, exactly as above $\phi_k*\mu\in C^\infty(\Omega[-\delta]$.
Next, the distribution   $\mu$ restricted to the relatively compact set
$\Omega[-\delta]$ has a finite order $k$ say.
Using this one can proceed exactly as in the proof
of Theorem 3.9 except that
one has to be a bit more careful and take into the account growth
of the derivatives of the $\phi$-functions up to order $k$.
We leave the details to the reader who also may consult
text-books devoted to
distribution theory which show that 
Theorem 3.9 holds with $w$ replaced by a distribution.

\bigskip


\centerline{\bf {3.11 Analytic expansions of harmonic functions}}

\medskip

\noindent
The elliptic character of
$\Delta$ is made more precise
by a result due to L. Ehrenpreis 
which shows how to
express distributions 
via absolutely convergent integrals 
taken in ${\bf{C}}^2$ via the Fourier transform of $\mu$.
This result go beyond these notes
since the proofs rely upon several complex variables.
Ehrenpreis'
integral formulas appear in [Bj�rk: Chapter 8 ] and 
also in 
[H�-complex] as well as in [H�:2 Chapter PDE].
Let as explain the result for harmonic functions $w(x,y)$
in the unit disc.
\medskip

\noindent
{\bf 3.12 Integrals over harmonic exponentials}
Let $\zeta$ and $w$ be two complex numbers and set:
\[ 
\mathfrak{e}(x,y)= e^{i (x\zeta+y\eta)}
\]
We see that $\Delta(\mathfrak{e})=-(\zeta^2+w^2)\cdot \mathfrak{e}$.
Put
\[
 S=\{(\zeta,w)\in C^2\quad\colon\,\zeta^2+w^2=0\}
\]
Points on this algebraic hypersurface in ${\bf{C}}^2$ produce harmonic 
${\bf{e}}$-functions in the  $(x,y)$-plane.
It is therefore tempting to consider a complex-valued Riesz measure
$\mu$ in the 4-dimensional real $(\zeta,w)$-space with support in $S$ and
define the function 
\[ 
U(x,y)=\int_S\, e^{i(x\zeta+y w)}\cdot d\mu(\zeta,w)\tag{*}
\]


\noindent 
With $\zeta=\xi+i\eta$ and $w=u+iv$
we have 
\[
|e^{i(x\zeta+y w)}|= e^{-(x\eta+yv)}\tag{i}
\]
If $z=x+iy\in D$
so that $x^2+y^2<1$,  the Cauchy-Schwartz inequality gives
\[
|x\eta+yv|\leq\sqrt{\eta^2+v^2}\tag{ii}
\]
Assume that the mass distribution of $\mu$ 
satisfies
\[
\int_S\,e^{ \sqrt{\eta^2+v^2}}\cdot\,|d\mu(\zeta,w)|<\infty\tag{iii}
\]
Under this hypothesis we see from (i-iii) that
the integral defining $U(x,y)$ in (*)
converges for
every point $(x,y)\in D$ and gives  a harmonic
function.
\medskip

\noindent
{\bf 3.13 Extension to the complex Levi ball.}
From the real pair $(x,y)$ we can take pass to complex
numbers
$z_1,z_2$ with $\mathfrak{Re}(z_1)=x$ and
$\mathfrak{Re}(z_2)=y$.
Let us then try to evaluate the integral
\[ 
\mathcal U(z_1,z_2)=
\int_S\, e^{i(z_1\zeta+z_2 w)}\cdot d\mu(\zeta,w)
\]


\noindent
With $z_1=a_1+ib_1$ and $z_2=a_2+ib_2$
we get 


\[
| e^{i(z_1\zeta+z_2 w)}|=
e^{a_1\mathfrak{Re}(\zeta)-b_1\mathfrak{Im}(\zeta)+
a_2\mathfrak{Re}(w)-b_2\mathfrak{Im}(w)}
\]


\noindent
Let us put
\medskip
\[
\mathcal L=\{(z_1,z_2)\quad\colon\, |z_1|+z_2|<1\}
\]


\noindent
This open subset of ${\bf{C}}^2$ is called the Levi ball.
Notice that its intersection with the real subspace 
where $\mathfrak{Im}(z_1)=\mathfrak{Im}(z_2)$ is equal to the
unit disc $D$ in the $(x,y)$-plane.
The Cauchy-Schwartz inequality gives
\[
|a_1\mathfrak{Re}(\zeta)-b_1\mathfrak{Im}(\zeta)+
a_2\mathfrak{Re}(w)-b_2\mathfrak{Im}(w)|
\leq |z_1|\cdot\sqrt{\xi^2+\eta^2}+
|z_2|\cdot\sqrt{u^2+v^2}
\]


\noindent
At the same time we stay on $S$ so that
$\zeta^2+w^2=0$. Regarding the real part this gives
\[
\xi^2-\eta^2+u^2-v^2=0
\]
At this stage  the reader discovers the picture. More precisely we
obtain

\medskip



\noindent 
{\bf 3.14 Lemma.} \emph{When $(z_1,z_2)\in\mathcal L$
one has the inequality}
\medskip
\[
|z_1|\cdot\sqrt{\xi^2+\eta^2}+
|z_2|\cdot\sqrt{u^2+v^2}\leq\sqrt{\eta^2+v^2}
\quad\colon\,(\xi+i\eta,u+iv)\in S
\]


\noindent
The easy proof is left to the reader.
Using this inequality and assuming that the integral (iii) above is finite, it
follows that one has  \emph{absolutely convergent integrals}:
\[ 
\mathcal U(z_1,z_2)=
\int_S\, e^{i(z_1\zeta+z_2 w)}\cdot d\mu(\zeta,w)
\quad\colon\, (z_1,z_2)\in\mathcal L\tag{**}
\]



\noindent Moreover, when $|z_1|+|z_2|<1$, i.e. when we stay inside the 
open Levi ball
we can take complex derivatives with respect to $z_1$ and $z_2$ to conclude that
$\mathcal U$ is an analytic function in the open Levi ball.
Its restriction to the real subspace is the harmonic function $U(x,y)$
which by (**)
extends to an analytic function of two
complex variables in the Levi ball.
\medskip

\noindent
{\bf 3.15 Application.}
For each integer $N\geq 0$ we put

\[ 
\mathcal H_N=\{ u\in C^N(\bar D)\quad\colon\,
u\,\,\text{harmonic in}\,\, D\}
\]
This is a Banach space where the norm
$||u||_N$ can be taken as the sum of the maximum norm of
its derivatives up to order $N$ plus the maximum norm of $u$ itself.
With this notation the following result holds - i.e. a special case
from the $L^2$-estimates from [H�r]. See also
Chapter 4 in the text-book
[H�r:xx.]
\bigskip

\noindent
{\bf 3.16 Theorem.}
\emph{There exists a fixed integer $m^*$ and for   every $N\geq 0$
a constant $C_N$ such that every  $u\in\mathcal H_N$ 
can be represented inside $D$ by an absolutely convergent integral}
\[ 
u(x,y)=
\int_S\,
 e^{i(x\zeta+yw)}d\mu(\zeta,w)
\]
\emph{and the measure $\mu$ satisfies}
\[
\int_S\,
[1+\sqrt{\eta^2+v^2}\,]^{m^*-N}\cdot e^{i\sqrt{\eta^2+v^2}}\cdot
|d\mu(\zeta,w)|\leq
C_N\cdot ||u||_N
\]
\medskip

\noindent
{\bf Remark.}
H�rmander's proof shows that the constants $C_N$ have 
polynomial growth, i.e. there are constants $A,B$ such that
\[ C_N\leq A(1+N)^B\quad\colon\, N=1,2,\ldots
\]

\medskip

\noindent
{\bf{3.17 Question.}}
It would be interesting  to
determine the best possible $m^*$ for which constants
$C_N$ as above exist for all $N$. For this  
question one
need not insist that $m^*$ is an integer.
\medskip

\noindent 
{\bf 3.18 Expansions  by Hayman} 
The 
analytic extension of a harmonic function $u(x,y)$ in $D$ to the Levi ball can also be proved 
via
expansions of  $u(x,y)$ into harmonic polynomials.
This is done by Hayman in [xx] without any  use 
complex analysis in several variables.
Using Hayman's expansions it seems reasonable to get
a good upper bound for $m^*$ and perhaps even find the best possible choice.






\newpage

\centerline{\bf 4. Perron families.}
\bigskip


\noindent
Le $\Omega$ be a bounded open set in
${\bf{C}}$. No further assumptions are imposed, i.e.
$\Omega$ need not be connected and its boundary
can be "ugly". For example, it may  have
positive two-dimensional Lebesgue measure.
On $\partial\Omega$ we have a function $\phi(x)$
which takes values between 0 and some $M>0$. No
other conditions are imposed, i.e. $\phi$ need not even
be measurable.
Given $\phi$ we denote by $\mathcal P(\phi)$
the family of all $u\in\text{SH}^0(\Omega)$
for which

\[
\limsup_{z\to w}\, u(z)\leq \phi(w)\quad\colon\quad
w\in\partial\Omega
\]


\noindent 
In $\Omega$ we get the function
\[ 
H_\phi^*(z)=
\max_{u\in\mathcal P(\phi)}\, u(z)
\]
It is called Perron's maximal function of $\phi$.
With these notations one has
\medskip

\noindent
{\bf 4.1 Theorem.}
\emph{Perron's maximal function is harmonic in
$\Omega$.}
\bigskip

\noindent
\emph{Proof.}
The constant functions 0 belons to
$\mathcal P(\phi)$ and the maximum principle gives
$u(z)\leq M$ for every $u\in\mathcal P(\phi)$.
It suffices to show that $H^*_\phi(z)$ is harmonic in
a disc $D$ inside $\Omega$.
When $u\in\mathcal P(\phi)$
its restriction to $\partial D$ is a continuous function
where the solution to Dirichlet problem gives
the harmonic function $u^*$ in $D$.
The function defined as $u^*$ in $D$ and $u$ in
$\Omega\setminus D$ belongs to $\mathcal P(\phi)$.
At the same time
$u\leq u^*$ holds in $D$.
We conclude from this that the values of $H^*_\phi$ inside $D$
are obtained when we take $u$:s from the restricted
class of $\mathcal P(\phi)$ which are harmonic
and $\geq 0$ in $D$.
Denote this restricted class with
$\mathcal P_*(\phi)$.
So inside $D$
we have
\[
H_\phi^*(z)=
\max_{u\in\mathcal P_\ast(\phi)}\, u(z)\tag{i}
\]
The harmonic functions in $D$ which
come from 
$\mathcal P_\ast(\phi)$ take values between
0 and M and hence their restrictions to $D$
give a normal family of harmonic functions in $D$.
Let $a$ be the center of $D$.
The normal family property yields
a sequence $\{u_n\}$ in
$\mathcal P_*(\phi)$
such that
\[ 
H^*(a)=\lim_n\, u_n(a)\tag{ii}
\]
and $\{u_n\}$ converge uniformly to a harmonic function
$U(z)$ in $D$.
We claim that
\[ H^*(z)=U(z)\quad\colon\,z\in D\tag{iii}
\]


\noindent
For assume the contrary. 
To begin with, since $H^*(z)$
is the maximal function it is clear that
$U(z)\leq H^*(z)$ holds in $D$.
So if (iii) fails there exists
\[
z_0=a+re^{\theta_0}\in D\quad\colon\quad
U(z_0)<H_\phi^*(z_0)\tag {iv}
\]









\noindent
To see that (iv) cannot occur we  
regard the point $z_0$ and again  use again the normal family to
obtain a sequence $\{v_n\}$ in $\mathcal P_*(\phi)$
which converges uniformly to a harmonic function
$V(z)$ in $D$ where
\[ 
V(z_0)=H^*_\phi(z_0)\tag{v}
\]



\noindent
Now we can derive a contradiction if (iii) fails. 
Namely, in $\Omega$
we have the subharmonic function
$w_n=\text{max}\,(u_n,v_n)$
and taking its harmonic majorant inside $D$
we get a new sequence $\{w_n^*\}$
in
$\mathcal P^*_\phi$. Passing to a subsequence if necessary
$w_n^*$ converges to a limit function $W(z)$
which is harmonic in $D$ and by construction it is
$\geq$ both to $U$ and to $V$.
But from (ii) we have 
\[
U(a)=H^*_\phi(a)\implies U(a)=W(a)\tag {vi}
\]


\noindent
At the same time
$V\leq W$ and $V(z_0)= W(z_0)>U(z_0)$.
Hence we get a strict inequality
\[
\frac{1}{2\pi}\,\int_0^{2\pi}
U(a+re^{i\theta})\cdot d\theta< 
\frac{1}{2\pi}\,\int_0^{2\pi}
\,W(a+re^{i\theta})d\theta\tag {vii}
\]


\noindent
But this
contradicts the equality $U(a)=W(a)$ from
(vi) since both terms in
(vii) by the mean value equality express
$U(a)$ and $W(a)$.
Hence equality holds in (iii) and Theorem 4.1  is proved.




\bigskip


\centerline{\bf 5. Maximum of several harmonic functions}


\bigskip

\noindent
Let $H_1,\ldots,H_k$ be a finite family of harmonic functions
defined in some open set $\Omega$.
Put
\[
u(x,y)=\text{max}\,\{H_1(x,y),\ldots,H_k(x,y)\}
\]
Since harmonic functions satisfy the mean-value
condition it follows that the mean-value inequality holds for $u$ and hence
$u$ is subharmonic.
We are going to  describe the non-negative measure
$\Delta(u)$.
To attain this we introduce the set
\[
\Gamma=\,\bigcup_{i\neq\nu}\, \{H_i=H_\nu\}
\]


\noindent
Recall from XXX that the zero set of an arbitrary  harmonic function
consists of
a union of smooth real analytic curves
$\{\gamma_\alpha\}$ where each pair of these curves
may interest in a discrete set and when it occurs this
intersection is transveral.
Since $\Gamma$ is a finite union of zero sets of
harmonic functions it enjoys the same
description.
There appears also the discrete set $\sigma(\Gamma)$
where at least two curves intersect. Now
$\Gamma\setminus\sigma(\Gamma)$
is a disjoint union of smooth and connected real-analytic curves
$\{\gamma_k\}$  called
the
\emph{regular branches} of $\Gamma$.
If $\Omega_0$ is a relatively compact subset of
$\Omega$ only finitely many regular branches intersect
$\Omega_0$.
\medskip


\noindent
Next, the open complement $\Omega\setminus\Gamma$
has connected components denoted by
$\{\Omega_\alpha\}$. To every such component it is clear from the definition of $u$
that there exists
an integer $1\leq i(\alpha)\leq k$ such that
\[
u=H_{i(\alpha)}\quad\text{holds in}\quad  \Omega_\alpha\tag{1}
\]

\noindent
Let us now consider a regular branch $\gamma_k$ of $\Gamma$.
It borders two connected components, say
$\Omega_\alpha$ and $\Omega_\beta$.
From (1) we get the   pair $H_{i(\alpha)}$ and
$H_{i(\beta)}$ where
\[ 
H_{i(\alpha)}=H_{i(\beta)}\quad\text{holds on}\,\,\gamma_k
\]


\noindent
Since $\gamma_k$ is a regular branch it follows that the two gradient vectors
$\nabla(H_{i(\alpha)}$ and $\nabla)H_{i(\beta)}$
are not equal at any point on $\gamma_k$.
Let $\mathfrak{n}_\alpha$ be the normal to
$\gamma_k$ which is directed into $\Omega_\alpha$.
This means that
\[
H_{i(\alpha)}>H_{i(\beta)}\quad\text{holds in}\quad  \Omega_\alpha
\]
and with this choice of
$\mathfrak{n}_\alpha$ we have
\[
\partial_{\mathfrak{n}_\alpha}(H_{i(\alpha)})>
\partial_{\mathfrak{n}_\alpha}(H_{i(\beta)})\quad\text{on}\,\,\gamma_k
\] 


\noindent Thus, if $ds$ is the arc-length measure on $\gamma_k$
we get the positive measure
\[ 
\mu_{\gamma_k}=
[\partial_{\mathfrak{n}_\alpha}(H_{i(\alpha)})-
\partial_{\mathfrak{n}_\alpha}(H_{i(\beta)})]\cdot
ds\tag{*}
\]



\medskip

\noindent 
{\bf 5.1 Proposition.}
\emph{Along $\gamma_k$ the measure $\Delta(u)$ is given
by the positive density above.}
\medskip

\noindent
{\bf{Exercise.}}
Prove this result where the hint us to apply
Stokes formula.
\medskip

\noindent
Proposition 5.1 gives   the following
conclusive result:
\medskip

\noindent
{\bf{5.2 Theorem.}}
\emph{The non-negative Riesz measure
$\Delta(u)$ is equal to}
\[ 
\sum\, \mu_{\gamma_k}
\]
 \emph{where the sum is taken over all regular branches of $\Gamma$.}
\medskip

\noindent
\emph{Proof.}
By Propostion 5.1. there only remains to show that
$\Delta(u)$ cannot contain a discrete part from 
discrete point masses in $\sigma(\Gamma)$.
But this is clear for if 
$\Delta(u)$ contains a discrete measure
$c\cdot \delta_p$ with $c\neq 0$ and $p\in\sigma(\Gamma)$
then the logarithimc potential of this point mass yields a
discontinuous function while $u$ from the start
obviously is a continuos function.
\bigskip

\noindent {\bf 5.3 Subharmonic configurations.}
Above we have clarified that every 
finite set of harmonic functions  
$H_1,\ldots,H_k$ yields a 
subharmonic maximum function.
One may ask if this $k$-tuple can be used to construct other subharmonic functions
$u$ than the maximum function. Let us give
\medskip

\noindent
{\bf{5.4 Definition.}}
\emph{A subharmonic configuration of
$H_1,\ldots,H_k$
is a subharmonic function $u$ in
$\Omega$ whose Laplacian is supported by $\Gamma$
and for 
every connected component 
$\Omega_\alpha$ of of 
$\Omega\setminus\Gamma$ one has:}
\[
u=H_{i(\alpha)}\quad\text{for some}\quad
 1\leq i(\alpha)\leq k \tag{*}
\]
\medskip

\noindent
Thus, when $u$ is a subharmonic configuration
then
$\Omega\setminus\Gamma$ is covered by
by a $k$-tuple of pairwise
disjoint open
sets $W_1,\ldots,W_k$ such that
$u=H_i$ holds in $W_i$. Moreover, every
$W$-set is a union of connected components of
$\Omega\setminus\Gamma$.
\medskip


\noindent
{\bf{An example.}}
It turns out that there exist subharmonic configurations which
which are not given by the maximum function.
The following example is due to Borsea and B�gvad in [B-B]:
\bigskip

GIVE Example.
\bigskip

\noindent
{\bf{Local uniqueness. }}
The example above leads us  to find
conditions on the
harmonic functions in order
that they only admit the
obvious subharmonic configuration.
In the article [B-B-B] a local uniqueness result is proved which goes
as follows: Let $p\in\Omega$ be a point such that
$k$-tuple of gradient vectors
$\{\nabla(H_i)(p)\}$
all are extreme points in the convex hull they generate
in ${\bf{R}}^2$. Under this condition one has

\noindent

\medskip

\noindent
{\bf{5.5 Theorem.}}
\emph{Let $u$ be a subharmonic configuration of $H_1,\ldots,H_k$ 
defined in some open neighborhood of $p$ where 
all the $H$-functions are active, i.e. the closure of  the open set where
$u=H_i$ contains $\{p\}$ for each $1\leq i\leq k $.
Then}
\[ 
u=\max\,(H_1,\ldots,H_k)
\] 
\emph{holds in a neighborhood of $p$.}




\bigskip

\noindent






\noindent
{\bf {5.6 A study of Cauchy transforms}}
Let $\mu$ be a non-negative Riesz measure
whose support is a compact null set $K$.
Now we get the Cauchy transform
\[ 
\mathcal C_\mu(z)=\int_K\,\frac{d\mu(\zeta)}{z-\zeta}
\]
Let $\Omega$ be an open set which contains $K$ and $g_1,\ldots,g_k$ 
is some $k$-tuple of holomorphic
functions in $\Omega$.
We can impose the condition that
for every
connected component $\Omega_\alpha$ of
there exists $1\leq i(\alpha)\leq k$
such that
\[
\mathcal C_\mu|\Omega_\alpha=g_{i(\alpha)}\tag{*}
\]


\noindent
Let us also consider the logarithmic potential
\[ 
U_\mu(z)= \int_K\,\text{Log}\, (|z-\zeta|)\cdot d\mu(\zeta)
\] 
It turns out that the subharmonic function
$U_\mu$ is 
locally piecewise harmonic in $\Omega$. To prove this
it suffices to 
work 
locally inside $\Omega$ so without loss of generality
we may  assume that
$\Omega$ from the start is simply connected Then there exist
primitive analytic functions $G_1,\ldots,G_k$ of the 
$g$-functions.
The formulas from XXX show that (*) is equivalent   to the condition
that
for every $\Omega_\alpha$ there exists a constant $c_\alpha$ such that
\[
U_\mu(z)|\Omega_\alpha=\mathfrak{Re}(G_{i(\alpha)})+c_\alpha\tag{2}
\]
Here $\{H_i=\mathfrak{Re}(G_i)\}$ are harmonic functions in
$\Omega$.
If the number of the  constants $\{c_\alpha\}$ which appear above is finite 
it  follows that $U_\mu$ is piecewise harmonic with respect to
a finite set of harmonic functions, i.e. given by 
the family $\{H_i\}$
plus eventual constants via (2) above.
In this \emph{favourable}
case we get the similar  result  as in Theorem 5.2.
In particular
we conclude that the support of $\Delta(\mu)$
is a union of real-analytic $\gamma$-arcs.
In [BV-B-B] the following affirmative result is proved without any initial assumption of
the local finiteness of  the $c$-constants. 
\bigskip

\noindent
{\bf{5.7 Theorem.}}
\emph{When (*) holds above
it follows that the logarthmic potential
$U_\mu$ is locally piecewise harmonic and hence 
the support of $\Delta(\mu)$ consists of a locally finite union
of real-analytic curves.}
\medskip

\noindent
{\bf{Remark.}}
The proof Theorem 5.7  is quite involved and we  refer to
[B-B-B] for details.
The difficulty  is to show that (*)
implies that the number of $c$-constants from (**) is locally finite.



\bigskip



\noindent
{\bf 5.8 Cauchy transforms and algebraic functions.}
As above   $\mu$ is a non-negative measure 
supported by a compact null set $K$ in
${\bf{C}}$. The  Cauchy transform
$\mathcal C_\mu(z)$ is analytic in ${\bf{C}}\setminus K$.
Suppose  it satisfies an algebraic equation, i.e.
there exists some $m\geq 1$
and polynomials  $p_0(z),\ldots,p_m(z)$ such that
\[
p_m(z)\cdot \mathcal C^m_\mu(z)+\ldots
p_1(z)\cdot \mathcal C_\mu(z)+
p_0(z)=0\quad\colon\quad
z\in{\bf{C}}\setminus K\tag{*}
\]
\medskip

\noindent
Using Theorem  5.7 it is proved in [B-B-B] that
(*) implies that $K$ is a finite union of real-analytic curves
which  are related to
roots of the algebraic equation
\[
p_m(z)\cdot y^m +\ldots
p_1(z)\cdot y+
p_0(z)=0\quad\colon\quad
z\in{\bf{C}}\setminus K\tag{**}
\]
\medskip

\noindent
{\bf{5.9 Remark.}}
The  result in 5.8 is illustrated by
examples from the article [Bergquist-Rullg�rd).
Here asymptotic expansions for distributions of roots
of eigenpolynomials which appear for a 
class of ODE-equations which
extend the usual hypergeometric equation. 
The asymptotic distributions of roots are given by probability measures $\mu$ whose Cauchy transforms
satisfy an algebraic equation of the form
\[ 
\mathcal C_\mu^m(z)=\frac{1}{Q(z)}(*)
\] 
where $Q(z)$ is a monic polynomial of degree $m$ with simple zeros
$\alpha_1,\ldots,\alpha_k$.
It is proved in [B-R] that there exists a \emph{unique}  probability measure
$\mu$ with compact support  whose Cauchy transform satisfies (*).
Moreover, the support of $\mu$ is an analytic tree
$\Gamma$, i.e. a connected set given  by
a finite union of real-analytic  Jordan
arcs which meet at some corner points. Moreover.
${\bf{C}}\setminus\Gamma$ is connected.


\newpage

\centerline{\bf{6. On zero sets of subharmonic functions.}}

\medskip

\noindent
Let  $\Omega$ in ${\bf{C}}$ be a bounded open set and
denote by
$\text{SH}_0(\Omega)$
the set of  subharmonic functions in   $\Omega$
whose Laplacian is a Riesz measure supported by a 
compact null set. Every such function $v$ is locally a
logarithmic potential of $\Delta(V)$ plus a harmonic function
and 
can therefore be taken to be
upper
semi-continuous. Moreover,  the distribution
derivatives  $\partial V/\partial x$
and
$\partial V/\partial y$  belong to
$L^1_{\text{loc}}(\Omega)$.
Before we announce Theorem 6.1 
we introduce a geometric construction.
If $U$ is an open subset of $\Omega$
we  construct
its forward star\vvv domain
as follows:
To each $\zeta\in U$
we find the largest $s(\zeta)>0$
such that the line segment
\[
\ell\uuu\zeta(s(\zeta))=
\{ \zeta+x\,\colon \, 0\leq x<s(\zeta)\}\subset \Omega
\]
Now we put
\[ 
\mathfrak{s}(U)=\bigcup\uuu{\zeta\in U}\, 
\ell\uuu\zeta(s(\zeta))\tag{*}
\]
and refer to this open set as the
forward star domain of $U$.
\bigskip

\noindent{\bf{6.1 Theorem.}}
\emph{Let 
$V\in\text{SH}_0(\Omega)$ and put
$K=\text{Supp}(\Delta(V))$. Suppose that
$V=0$ in an open subset $U$ of $\Omega\setminus K$
and furthermore}
\[
\partial V/\partial x(z)<0,\quad\text{holds in}\quad 
\Omega
\setminus (K\cup U)\,.\tag{*}
\]
\emph{Then $V=0$ in $\mathfrak{s}(U)$.}
\medskip



\noindent
\emph{Proof.}
It is clear that it suffices to show the following:
Let $z_0\in U$ and consider a horisontal
line segment 
\[
\ell=\{z=z_0+s\,\colon\,
0\leq s\leq s_0\}
\]
which is contained in $\Omega$.
Then, if $0<\delta<\text{dist}(\ell,\partial\Omega)$
and the open disc $D_\delta(z_0)$ of radius $\delta$ centered at
$z_0$ is contained in $U$, it follows that $V$ vanishes in the open set
\[
\{z\,\colon\, \text{dist}(z,\ell)<\delta\}\tag{1}
\]



\noindent
Notice  that (1) is a relatively compact subset of $\Omega$.
Consider the complex derivative
\[
\partial V/\partial z=
\frac{1}{2}(\partial V/\partial x\vvv i
\partial V/\partial y)
\]
This yields a complex\vvv valued and locally integrable function
and since $V=0$ in $D\uuu\delta(z\uuu 0)$
it is clear that $V=0$ in the open set from (1) if we  prove that
$\partial V/\partial z=0$ holds almost everywhere in (1)
To prove this we take some  $\epsilon>0$ and put
\[
\Psi\uuu\epsilon (z)=\log \, (\partial V/\partial z\vvv\epsilon)
\tag{2}
\]
\medskip
where the single-valued branch of the complex Log-function
is chosen so that
\[
\pi/2<\mathfrak{Im}\,\Psi\uuu\epsilon <3\pi/2\tag{3}
\]
Hence we can write
\[
\Psi\uuu\epsilon (z)=\text{Log}|\epsilon-\partial V/\partial z|+i\tau(z)
\quad\,\colon\quad  \pi/2<\tau(z)<3\pi/2\tag{4}
\]


\noindent 
\emph{A regularisation.}
Choose a non-negative
test-function $\phi$ with compact support in
$|z|\leq \delta$ while  $\phi(z)>0$ if $|z|<\delta$
and $\iint \phi(z)dxdy=1$. We construct the
convolution $\sigma*\Psi\uuu\epsilon $ which is defined in the
subset of $\Omega$ whose points have distance 
$>\delta$ to $\partial\Omega$.
Rules for first order derivations of a convolution
give

\[
\bar\partial/\bar\partial\bar z\bigl(\phi \ast \Psi\uuu\epsilon \bigr )=
\frac{\phi\ast \bar\partial\partial (V)}{\partial V/\partial z)\vvv
\epsilon}=
\frac{1}{4}\cdot\frac{1}{\partial V/\partial z\vvv\epsilon}\cdot 
\phi\ast \Delta(V)
\]
Taking the real part we get 
\[
\mathfrak{Re}(\bar\partial/\bar\partial\bar z(\phi\ast \Psi\uuu\epsilon))=
\frac{\partial V/\partial x\vvv \epsilon}{4
|\epsilon-\partial V/\partial z|^2}\cdot\phi*\Delta(V)\tag{5}
\]


\medskip
\noindent
To simplify notations we set
\[
\sigma(z)=\text{Log}|\epsilon-\partial V/\partial z|\tag{6}
 \]


\medskip


\noindent
The definition
of the $\bar\partial$-derivative and the decomposition
$\Psi\uuu\epsilon =\sigma+i\cdot \tau$
together with the inequality (5) give
\[
\partial_x(\phi*\sigma)\leq\partial_y(\phi*\tau)\tag{7}
\]

\medskip

\noindent
In the right hand side we 
use the partial $y$\vvv derivative on $\phi$, i.e. we use the general formula:
\[
\partial_y(\phi*\tau)=
\partial_y(\phi)*\tau
\]
Since $\pi/2\leq \tau\leq 3\pi/2$ the absolute value of this function
is majorized by
\[
M=\frac{3\pi}{2}\cdot||\partial_y(\phi)||_1\tag{8}
\]
where
$||\partial_y(\phi)||_1$ denotes the $L^1$-norm.
Next,  
consider the function $s\mapsto \phi*\sigma(z\uuu 0+s)$
where $0\leq s\leq s_0$ whose  $s$-derivative becomes
$\partial_x(\phi*\sigma)(z+s)$. Hence (7\vvv 8) give:

\[
\frac{d}{ds}(\phi*\sigma(z+s))\leq M
\]
\[
\implies
\phi*\sigma(z\uuu 0+s\uuu 0)\leq\phi*\sigma(z\uuu 0)+M\cdot s_0\tag{9}
\]

\medskip 

\noindent 
From now on  $\epsilon<1$ so that $\log \epsilon<0$. 
Since $V=0$ in $D\uuu\delta(z\uuu 0)$
we also have $\partial V/\partial z=0$ in this disc
and conclude that
\[
\phi*\sigma(z\uuu 0)=\log\,\epsilon\tag{10}
\]
Next, we have
\[ 
\sigma=\log\,\epsilon+
\log\,|1\vvv \frac{\partial V/\partial z}{\epsilon}|
\]
Put
\[
f\uuu\epsilon=\log\,|1\vvv \frac{\partial V/\partial z}{\epsilon}|\tag{11}
\]
Then (9\vvv 10) give the inequality
\[
\phi\ast f\uuu\epsilon(z\uuu 0+s\uuu 0)\leq M\cdot s\uuu 0\tag{12}
\]
So from 
the construction of a convolution this means that
\[
\iint\uuu{|\zeta|\leq \delta }\,
f\uuu\epsilon (z\uuu 0+s\uuu 0+\zeta)\cdot \phi(\zeta)\cdot d\xi d\eta
\leq M\cdot s\uuu 0\tag{13}
\]





\medskip

\noindent
Next, we can write
\[
f\uuu\epsilon (z)= \frac{1}{2}\cdot \log\, \bigl(
(1\vvv \frac{\partial V/\partial x}{\epsilon})^2+
(\frac{\partial V/\partial y}{\epsilon})^2\bigr)\tag{14}
\]
Since $\partial V/\partial x\leq 0$ holds almost everywhere we have
$f\geq 0$ almost everywhere and 
at each point $z$ where $\partial V/\partial z\neq 0$
we have
\[ 
\lim\uuu{\epsilon\to 0}\, f\uuu\epsilon(z)=+\infty\tag{15}
\]
Since (13 holds for each $\epsilon>0$
and $\phi>0$ in the open disc $|\zeta|<\delta$
we conclude that
$\partial V/\partial z=0$ must hold almost everywhere
in the disc $D\uuu\delta(z\uuu 0+s\uuu 0)$.
Here we considered the largest $s$\vvv value 
along the horisontal line $\ell$.
Of course, we get a similar conclusion for each $0<s<s\uuu 0$
and hence $\partial V/\partial z=0$ holds almost everywhere in
the open set (1).
At the same time
$V=0$ in $D\uuu\delta(z\uuu 0)$ and we conclude that
$V=0$ in the open set from (1) as requested.
\bigskip

\noindent
{\bf{Remark.}}
The method used in the
proof above is due to Bergqvist\vvv Rullg�rd in [Be\vvv Ru]
where the Theorem  was proved under the assumption
that
the range of $V$ is a finite set. But the essential idea to regard
the complex log\vvv function from (2) in the proof and employ
regularisations already occur in [Be\vvv Ru].




\newpage

\centerline{\bf{7. A subharmonic majorization.}}
\bigskip

\noindent
Let $\{0<\beta\uuu 1<\beta\uuu 2<\ldots\}$ be a sequence of positive real numbers.
Given $a>0$ we construct harmonic functions in the right half\vvv plane
$\Omega=\{\mathfrak{Re}\, z> a\}$ as follows: Set $q\uuu\nu=a+i\beta\uuu\nu$ and
to each $z\in\Omega $  we get  the triangle
with corner points at the three points $z,eq_\nu,eq\uuu{\nu+1}$
where $e$ is Neper's constant.
Denote the angle at $z$ by $\theta\uuu\nu(z)$. Notice that 
$0<\theta_\nu(z)<\pi$.
As explained in
� XX  $\theta_\nu(z)$ is a harmonic function
with the boundary value $\pi$  on $\mathfrak{Re}\, z=a$
when  $\beta\uuu\nu<y<\beta\uuu{\nu+1}$ 
while the boundary value is zero outside the closed 
$y$\vvv interval $[\beta\nu,\beta\uuu{\nu+1}]$.
If we instead consider the points $\{q^*\uuu\nu=a\vvv ie\beta\uuu\nu\}$
we get similar harmonic angle functions
$\{\theta\uuu\nu^*\}$
when we regard the angle at $z$ formed by the triangle
with corner points at $z \vvv eq\uuu\nu,\vvv eq\uuu{\nu+1}$.

\medskip

\noindent
{\bf{Exercise.}} Show by euclidian geometry that
if $b<0$ is real and positive then
\[ 
\sin\,\theta\uuu\nu(a+b)= \frac{eb(\beta\uuu{\nu+1}\vvv\beta\uuu\nu)}{
\sqrt{ (\beta\uuu\nu^2+b^2)(\beta\uuu{\nu+1}^2+b^2)}}\tag{1}
\]
Use also that $\beta\uuu \nu<\beta\uuu{\nu+1}$ and show
that (1) gives

\[
\theta\uuu\nu (a+b)> \frac{e b\beta\uuu 1^2\cdot (\beta\uuu{\nu+1} \vvv\beta\uuu\nu)}
{\beta\uuu{\nu+1}\cdot \beta\uuu\nu\cdot(e^2 \beta\uuu 1^2+b^2)}=
C(b,\beta\uuu 1)\cdot (\frac{1}{\beta\uuu\nu}\vvv\frac{1}{\beta\uuu{\nu+1}})
\quad\colon\quad
C(b,\beta\uuu 1)=\frac{e b\beta\uuu 1^2}{e^2 \beta\uuu 1^2+b^2}
\]

\medskip

\noindent
In addition to these $\theta\uuu\nu$\vvv functions we get the angle functions
$\{\theta^*\uuu n\}$ where we for each $n\geq 2$ consider the
harmonic extension to the half\vvv plane whose boundary
values are $\pi$  on $y>\beta\uuu n$ and zero when
$y<\beta\uuu n$.
This harmonic function is denoted by $\theta^*\uuu n(z)$
and by a figure the reader can verify  that
\[
\sin\, \theta^*\uuu n(a+b)=\frac{b}{\beta\uuu n^2+b^2)}\implies
\theta^*\uuu n(a+b)>\frac{eb\beta\uuu 1^2}{e^2 \beta\uuu 1^2+b^2}
\cdot \frac{1}{\beta\uuu n}\tag{2}
 \]
 





\medskip

\noindent
{\bf{7.1 A class of harmonic functions.}}
Let us also consider a sequence
of positive real numbers
$\{\lambda\uuu\nu\}$. To each  $n\geq 2$ we get the harmonic function
$u\uuu n(x,y)$ in $\Omega$
defined by
\[
u\uuu n(x,y)=\frac{1}{\pi}\cdot \sum\uuu{\nu=1}^{\nu=n\vvv 1} \lambda\uuu\nu\cdot
(\theta\uuu\nu+\theta^*\uuu\nu)+
\lambda\uuu n\cdot (\theta\uuu n+\theta^*\uuu n)\tag{3}
\]
If $z=a+b$ is real with $b>0$ the inequalities in (1\vvv 2) give
\[
u\uuu n(a+b)\geq \frac{C(b,\beta\uuu 1)}{\pi}
\cdot
\bigl[\, \sum\uuu{\nu=1}^{\nu=n\vvv 1}\,
\lambda\uuu\nu(\frac{1}{\beta\uuu\nu}\vvv \frac{1}{\beta\uuu{\nu+1}})+
\frac{\lambda\uuu n}{\beta\uuu n}\,\bigr]\tag{4}
\]



\noindent
From the above we get:


\medskip

\noindent
{\bf{7.2 Proposition.}}
\emph{Let $\{\lambda\uuu\nu\}$ and $\{\beta\uuu\nu\}$ be such that}
\[ 
\lim\uuu{n\to\infty}\,
\bigl[\, \sum\uuu{\nu=1}^{\nu=n\vvv 1}\,
\lambda\uuu\nu(\frac{1}{\beta\uuu\nu}\vvv \frac{1}{\beta\uuu{\nu+1}})=+\infty\tag{*}
\]
\emph{Then the sequence $\{u\uuu n(a+b)\}$ increases to $+\infty$ for every $b>0$.}

\medskip

\noindent
{\bf{Remark.}} If  
the $\lambda$\vvv sequence  increases  a partial summation gives
\[
\sum\uuu{\nu=1}^{\nu=n}\, \frac{\lambda\uuu \nu\vvv\lambda\uuu{\nu\vvv 1}}
{\beta\uuu\nu}= \sum\uuu{\nu=1}^{\nu=n\vvv 1}\,
\lambda\uuu\nu(\frac{1}{\beta\uuu\nu}\vvv \frac{1}{\beta\uuu{\nu+1}})
\]
where we have put $\lambda\uuu 0=0$. Hence (*) is equivalent to the divergence of
the 
positive series 
\[
\sum\uuu{\nu=1}^\infty \, \frac{\lambda\uuu \nu\vvv\lambda\uuu{\nu\vvv 1}}
{\beta\uuu\nu}\tag{**}
\]
\medskip


\noindent
{\bf{7.3 An application.}}
Let $\{\beta\uuu\nu\}$ and $\{\lambda\uuu\nu\}$ 
be  two  strictly increasing sequences
of positive real numbers.
Consider an analytic function $\Phi(z)$ defined in
the half\vvv plane $\mathfrak{Re}\,z>a$ with continuous
boundary values on $\mathfrak{Re}\,z=a$ which satisfies the inequalities

\[
|\Phi(z)|\leq \bigl(\frac{\beta\uuu\nu}{|z|}\,\bigr )^{\lambda\uuu\nu}
\quad\colon\quad \nu=1,2,\ldots\tag{*}
\]
\medskip

\noindent
{\bf{7.4 Exercise.}}
Denote by $u\uuu *(z)$ the harmonic function in
the half\vvv plane whose boundary values are one on
$\vvv\beta\uuu 1<y<\beta\uuu 1$ and otherwise zero.
Show that   (*) 
gives the following inequality on
$\mathfrak{Re}\, z=a$ for every $n\geq 0$ and $-\infty<y<+\infty$
 \[
\log\,|\Phi(a+iy)|+u\uuu n(a+iy)\leq 
\log K\cdot u\uuu *(a+iy)\quad\colon\quad 
K=\max\uuu {\vvv\beta\uuu 1\leq  y\leq \beta\uuu 1}\, |\Phi(a+iy)|\tag{7.4.1}
\]

\medskip

\noindent
Since $u\uuu n$ and $u\uuu *$ are  harmonic functions
while $\log\,|\Phi\,|$ is subharmonic, 
the principle of harmonic majorization implies that  (7.4.1) holds
in $\Omega$.
In particular, for every
real 
$b>0$ we have
\[
\log\,|\Phi(a+ib)|+u\uuu n(a+ib)\leq \log K\cdot u\uuu *(a+ib)\tag{7.4.2}
\]


\noindent
When $\Phi$ is not identically zero  we can  fix some
$b>0$ where
$\Phi(a+ib)\neq 0$ and (7.4.2) entails that
the sequence $\{u\uuu n(a+ib)\}$ is bounded.
Hence Proposition 7.2  gives

\medskip

\noindent
{\bf{7.5 Theorem.}} \emph{Assume there exists a non\vvv zero analytic function
$\Phi(z)$ in the half\vvv plane
$\mathfrak{Re}\, z>a$ such that (*) holds in (7.3). Then}
\[
\sum\uuu{\nu=1}^\infty \, \frac{\lambda\uuu \nu\vvv\lambda\uuu{\nu\vvv 1}}
{\beta\uuu\nu}<\infty
\]
\medskip

\noindent
{\bf{Remark.}} We can rephrase the result above and get a 
vanishing principle. Namely, if the positive series in Theorem 7.5 is divergent
an analytic function $\Phi(z)$ in the half\vvv plane satisfying
(7.3) must be identically zero.
\medskip


\noindent
{\bf{7.6 Example.}}
Let $\{c\uuu n\}$ be a sequence of positive real numbers.
Suppose that $\Phi(z)$ is analytic in the half\vvv space
and satisfies
\[
|\Phi(z)|\leq \frac{c\uuu n}{|z|^n}\quad\colon\quad n=1,2,\ldots
\]
Then $\Phi$ must vanish identically if the series

\[
\sum\uuu{n=1}^\infty\, \frac{1}{c\uuu n^{\frac{1}{n}}}=+\infty
\]

\medskip


\noindent
\centerline {\bf{7.7 A determined moment problem.}}
\medskip


\noindent
In a moment problem one asks for the existence of
a non\vvv negative measure $\mu$ on the real $x$\vvv axis
such that
all higher order moments
\[ 
\int\uuu 0^\infty\, x^n\cdot d\mu(x)= c\uuu n\tag{*}
\] 
where the sequence $\{c\uuu n\}$ is given in advance.
In 1894 Stieltjes settled this problem. Namely,
a necessary and sufficient condition for the existence of $\mu$ is that
if
$\{\beta\uuu\nu\}$ are the  coefficients in the
continued fractions associated to
$\{c\uuu n\}$, then
the series
\[ 
\sum\, (\vvv 1)^{\nu}\cdot \frac{\beta\uuu\nu}{x^{\nu+1}}
\] 
is positive.
We shall now prove
a uniqueness in relation to the moment problem.
Namely, let $\{c\uuu n\}$ be given and suppose that
(*) has at least one solution $\mu$.
Following Stieltjes one says that the moment problem is 
determined if
two solutions $\mu$ and $\gamma$ only differ 
by a discrete measure.
A necessary and sufficient condition for the moment problem to be
determined was established by Hamburger in [XX]. The
condition is that the continued fractions of $\{c\uuu n\}$
is completely convergent. For this notion we refer to 
[ibid] and also to [Japanese] as well as the article
\emph{sur le probleme des moments} by M. Riesz in [Arkiv 1923].
\medskip

\noindent 
Since the  calculation of a continued fraction is quite involved
one may ask for   growth conditions on
$\{c\uuu n\}$ which at least are are sufficient for the moment problem to be determined.
Using the previous results
the following sufficiency result was proved in
[Carleman\vvv Chapter VIII]:
\medskip

\noindent
{\bf{7.8 Theorem.}}
\emph{Hamburger's  moment problem is determined if}
\[
\sum\,\frac{1}{c\uuu n^{\frac{1}{2n}}}=+\infty
\]
\medskip


\noindent
\emph{Proof.}
Recall from � xx that a signed measure $\nu$ without  atoms. 
is determined by the transform
\[ 
V(z)= 
\int\uuu{\vvv\infty}^\infty\, \frac{d\nu(x)}{z\vvv x}
\]
More precisely, if $V(z)$ vanishes identically in the half\vvv plane
$\mathfrak{Im}\, z>0$ then $\nu=0$.
from this we can deduce Theorem 7.8 as follows. Suppose that
$\mu\uuu 1$ and $\mu\uuu 2$ are two non\vvv negative measures
which solve the moment problem and that 
the difference $\nu=\mu\uuu 1\vvv\mu\uuu 2$ has no atoms.
Set
\[ 
F\uuu k(z)=
\int\uuu{\vvv\infty}^\infty\, \frac{d\mu\uuu k(x)}{z\vvv x}
\quad\colon\quad k=1,2
\]
So now $V=F\uuu 1\vvv F\uuu 2$. Next, or each $n\geq 1$
we can write
\[ 
\frac{1}{z\vvv x}=\vvv XXX
\]
page 80!!!

\medskip
It follows that

\[ 
F\uuu k(z)= \vvv \sum\, \frac{c\uuu \nu}{z^{\nu+1}}\vvv
XXX
\]
Taking the difference it follows that

\[ 
V(z)= F\uuu 1(z)\vvv F\uuu 2(z)= \frac{1}{z^{2n+1}}\,
\int\, \frac{x^{2n}\cdot d\nu(x)} {1\vvv \frac{x}{z}}
\]
Next, notice that if $z$ is in the upper half\vvv plane then
\[
|1\vvv \frac{x}{z}|\leq \frac{|z|}{\mathfrak{Im}\, z)}
\]
So if  $\mathfrak{Im}\,z\geq 1$ , the triangle inequality gives

\[  
|V(z)|\leq \frac{1}{|z|^{2n}}\cdot \int\, x^{2n}\cdot |d\nu(x)|
\leq \frac{2c\uuu{2n}}{|z|^{2n}}
\]
\medskip

\noindent
Now the assumed divergence in the theorem and (xx) imply that
$V$ is identically zero and Theorem XX is proved.

\bigskip
after a TRICK to get Stieltjes case. see page 81 in carleman.


\newpage


\centerline{\bf{8. Subharmonic minorant of a given function.}}
\bigskip


\noindent
{\bf{Introduction.}}
Let $\Omega$ be a bounded
open and connected subset of ${\bf{C}}$ and
$F$  a non-negative and upper semi-continuous function in
$\Omega$ which may take values $+\infty$ at some points. 
Denote by $\mathcal F_*$ the class of subharmonic functions $u$ in
$\Omega$ such that  $u(x)\leq F(x)$ hold for every
$x\in\Omega$.
Set
\[ 
\mathcal S_F(x)= \max_{u\in \mathcal F_*}\, u(x)
\]
We refer to $\mathcal S_F$ as Sj�berg's maximal function associated to
$F$.
Recall that every subharmonic function $u$ 
is upper semicontinuous which implies that
for every compact subset $K$ of $\Omega$ there is a constant $C$
such that $u(x)\leq C$ on $K$.
It turns out that this is the sole obstruction in order
that $\mathcal S_F$ itself
subharmonic.
More precisely the following result  was proved by
Sj�berg in [1938; Scand. Congress Helsinki]
\medskip

\noindent
{\bf{8.1. Theorem.}}
\emph{If  
$\mathcal S_F$ is bounded on  every compact subset of
$\Omega$ then it is subharmonic and gives therefore the largest subharmonic function
in the class
$\mathcal F_*$.}

\bigskip


\noindent
{\bf{Remark.}}
In addition to [Sj�berg[ we refer to Domar's article
\emph{Subharmonic minorants of a given function}
[Arkiv 1954] where Sj�berg's resuts is extended in a wider context
and a similar result is proved for subharmonic functions in
${\bf{R}}^k$ when $k\geq 3$.
One expects
that
if  $F(x)$
enjoys some finiteness condition then
Sj�berg's condition holds so that  $\mathcal S_F$  gives
the largest subharmonic minorant to $F$.
The following sufficency
result was proved by  Beurling in [Beurling xx]:

\medskip

\noindent
{\bf{8.2. Theorem.}} \emph{The function $\mathcal S_F$ is subharmonic
if  there exists $\epsilon>0$ such that}
\[
\iint_\Omega\, \bigl[\log^+F(x,y)\,\bigr]^{1+\epsilon}<\infty\tag{*}
\]
\medskip

\noindent
\emph{Proof.}
Assume (*) and with $\epsilon$ kept fixed the
integral is denoted by 
$J(F)$.
We shall prove that 
if $K$�is a compact subset of $\Omega$
then there exists an integer $n$
such that
\[
\max_{x\in K} u(x)\leq e^n\tag{1}
\]
hold for every $u\in\mathcal F_*$ and 
Sj�berg's result entails that
$\mathcal S_F$�is subharmonic.
To prove (1)
we  fix a positive integer $\lambda$ and
a positive constant $C$ such that
\[
\frac{e}{\pi C^2}+e^{-\lambda}\leq 1\tag{2}
\]
Let us  take some $u\in \mathcal F_*$ and to each integer $\nu$ we set
\[ 
U_\nu=\{ e^\nu\leq u<e^{\nu+1}\}\tag{3}
\]
Here  $\{U_\nu\}$ is a family of disjoint sets whose union is
$\Omega$ and to each $\nu$ we denote by $\ell_\nu$ the area of $U_\nu$.
Next, suppose that for some integer  $n>\lambda $ 
there exists  a point $z_n\in \Omega$ 
such that 
\[
u(x_n)\geq e^n\quad\text{and}\quad 
\text{dist}(x_,\partial\Omega)>C\cdot \sqrt{\ell_{n-\lambda}+\ldots +\ell_n}\tag{4}
\]



\noindent
Consider the disc $D_n^*=\{|z-z_n|\leq 
C\cdot \sqrt{\ell_{n-\lambda}+\ldots +\ell_n}\}$.
\medskip

\noindent
\emph{Sublemma.}\emph{The inequality (4) entails that}
\[
\max_{z\in D_n^*}\, u(z)\geq e^{n+1}
\tag{5}
\]


\noindent
\emph{Proof.}
We 
argue by a contradiction. Set
\[
\rho=C\cdot \sqrt{\ell_{n-\lambda}+\ldots +\ell_n}
\]



\noindent
If  (5) fails the upper semi-continuity of $u$ gives
some $\rho_*>\rho$
such that  the disc
$\Delta= \{|z-z_n|\leq\rho^*$ is  contained in $\Omega$
and
$u\leq e^{n+1}$ holds in $\Delta$.
The mean-value inequality gives
\[ 
e^n\leq u(z_n)\leq \frac{1}{\pi\rho_*^2}\,
\iint_\Delta\, u(x,y)\,dxdy\tag{6}
\]
Now $U_\nu\cap \Delta=\emptyset$ if $\nu>n$
and since the $U$-sets are disjoint 
the right
hand side in (6) is  majorised by
\[
\frac{1}{\pi\rho_*^2}\cdot 
e^{n+1}(\ell_n+\dots+\ell_{n-\lambda})+e^{n-\lambda}
=e^n\cdot \frac{\rho^2}{\rho_*^2}\cdot [
e\cdot\frac{\ell_n+\dots+\ell_{n-\lambda}}{\pi}+e^{-\lambda}]\tag{7}
\]
The last factor is $\frac{e}{\pi C^2}+ e^{-\lambda}$
which is $\leq 1$ by (2) above.
Hence  (6-7) would give
\[
e^n\leq  e^n\cdot \frac{\rho^2}{\rho_*^2}
\]
Thus is a contradiction since
$\rho_*>\rho$ and  
hence 
$(4)\implies(5)$ holds.

\medskip

\noindent
\emph{Proof continued.}
Next, given  $z_n\in \Omega$ where
$u(z_n)\geq e^n$ we set
\[ 
\xi(m)= C\cdot\sum_{\nu=n}^{\nu=m}\, 
\sqrt{\ell_{\nu-\lambda}+\ldots+\ell_\nu}\quad\colon\, \forall\, m>n
\]


\noindent
Repeated use of $(4)\implies (5)$
shows that when $m>n$ and 
the disc $\{|z-z_n|\leq\xi(m)\}$ stays in $\Omega$ then it contains a point $z_m$ 
where $u(z_m)\geq e^m$.
Since
$u$ is bounded over compact subsets of $\Omega$
we must have
\[
\text{dist}(z_n,\partial\Omega)\leq
C\cdot \sum_{\nu=n}^\infty\, 
\sqrt{\ell_{\nu-\lambda}+\ldots+\ell_\nu}\tag{*}
\]
\medskip

\noindent
Let us  majorize the right hand side in (*).
Since $\sqrt{a_1+\ldots+a_k}\leq \sqrt{a_1}+\ldots +\sqrt{a_k}$
hold for all tuples of positive numbers 
the right hand side in (*) is majorized
by
\[
C\cdot \sum _{\nu=n}^\infty\, 
[\sqrt{\lambda_{\nu-\lambda}+\ldots+\lambda_\nu}]\leq
C\cdot(\lambda+1)
\sum_{\nu=n-\lambda} ^\infty
\, \sqrt{\lambda_\nu}
\]
To estimate the sum of the square roots
we pick the positive $\epsilon$ in the Theorem and write
\[
\sum_{\nu=n-\lambda} ^\infty
\, \sqrt{\lambda_\nu}=
\sum_{\nu=n-\lambda} ^\infty
\,\nu^{-1/2-\epsilon/2}\cdot   \sqrt{\lambda_\nu}\cdot \nu^{1/2+\epsilon/2}\leq
\sqrt{\sum_{\nu=n-\lambda} ^\infty
\,\nu^{-1-\epsilon}}\cdot 
\sqrt{\sum_{\nu=n-\lambda}
\lambda_\nu\cdot \nu^{1+\epsilon}}
\]
where the Cauchy-Schwarz inequality was used.
Since $\epsilon>0$
the first factor 
is a function $n\mapsto \tau(n)$ given as a  square root of  tail sums
of a convergent series and hence $\tau(n)\to 0$ when
$n\to+\infty$.
Now (*) is majorised by
\[ 
C\cdot \tau(n)\cdot 
\sqrt{\sum_{\nu=n-\lambda}
\lambda_\nu\cdot \nu^{1+\epsilon}}\tag{**}
\]
Next, we have 
\[
\log^+\,u(z)\geq \nu\,\colon|, z\in U_\nu\}
\]
Since the sets $\{U_\nu\}$ are disjoint it follows that

\[
\sum_{\nu=n-\lambda}
\lambda_\nu\cdot \nu^{1+\epsilon}\leq
\iint_\Omega\, [\log^+\, u(x,y)|^{1+\epsilon}\, dxdy\leq J(F)
\]
where the last inequality follows since
$u\leq F$.
Hence (*) gives

\[ 
\text{dist}(z_n,\partial\Omega)\leq
C\cdot \sqrt{J(F)}\cdot \tau(n)
\]


\noindent
Finally, if $K$ is a compact subset of $\Omega$ its distance to
$\partial\Omega$ is a positive number $a_K$
and since
$\tau(n)\to 0$ we find a large integer $N$
such that
$C\cdot \sqrt{J(F)}\cdot \tau(n)\leq a_K$ and by the above 
$u\leq e^n$ holds on $K$. Since $u\in\mathcal F_*$ was arbitrary
we have proved that $\mathcal S_F$ is bounded on $K$ and Theorem 2 is proved.
















\end{document}