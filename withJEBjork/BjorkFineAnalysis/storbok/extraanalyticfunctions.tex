
\documentclass{amsart}

\usepackage[applemac]{inputenc}

\addtolength{\hoffset}{-12mm}
\addtolength{\textwidth}{22mm}
\addtolength{\voffset}{-10mm}
\addtolength{\textheight}{20mm}

\def\uuu{_}

\def\vvv{-}

\begin{document}




\centerline{\bf\large Chapter III. Complex analytic functions}

\bigskip


\centerline{\emph{ Contents}}

\bigskip

\noindent
0. Introduction
\medskip


\noindent
1. Complex line integrals
\medskip

\noindent
2. The Cauchy-Riemann equations
\medskip

\noindent
3. The $\bar\partial$-operator
\medskip


\noindent
4. The complex derivative
\medskip

\noindent
5. Theorems by Morera and Goursat
\medskip





\noindent
6. Cauchy's Formula
\medskip

\noindent
7. Complex differentials 
\medskip

\noindent
8. The Pompieu formula
\medskip


\noindent
9. Normal families
\medskip

\noindent
10. Laurent series
\medskip

\noindent
11. An area formula
\medskip



\noindent
12. A theorem by Jentsch
\medskip

\noindent
13. An inequality by Siegel 
\medskip

\noindent
14. Zeros of product series.
\medskip



\noindent
15. Hadamard products 
\medskip













\bigskip

\noindent {\bf Introduction.}
Expressing a complex number  by $z=x+iy$ we  identify
$\bf C$ with ${\bf{ R}}^2$.
Consider a complex valued function
$f(z)=u(x,y)+iv(x,y)$ where   
$\mathfrak{Re}(f)=u$ is  the real part and 
$\mathfrak{Im}(f)=v$  the imaginary part.
Let
$f$ be defined in a domain
$\Omega\in\mathcal D(C^1)$ which
extends continuously to the
closure $\bar\Omega$ where
$u$ and $v$ are $C^1$-functions in
$\Omega$, i.e. the four partial derivatives
$u_x,u_y,v_x,v_y$ exist as continuous functions in
$\Omega$.
Now there exists 
complex line integral
\[
\int_{\partial\Omega}\, f(z)\cdot dz=
\int_{\partial\Omega}\,udx \vvv v dy+
i\cdot \int_{\partial\Omega}\, udy+vdx
\tag{*}
\]
Theorem 2.4 in Chapter II shows that 
both integrals are  zero
if the pair $u,v)$ satisfies
the following two differential equations  in  $\Omega$:
\[
u_x=v_y\quad\colon\quad u_y=-v_x\tag {**}
\]


\noindent
We refer to (**) as  the \emph{Cauchy-Riemann equations} and 
$(u,v)$ is  called a $CR$-pair  when (**) holds and then we say that
$f=u+iv$ is an analytic function of the complex variable $z$.
In � 6 we 
establish  
\emph{Cauchy's formula}:

\[
f(z_0)=\frac{1}{2\pi i}\cdot
\int_{\partial\Omega}\,\frac{f(z)dz}{z-z_0}\,\quad\text{for all}\,\,
z_0\in\Omega\,.\tag{***}
\]
\medskip

\noindent 
We will use
(***) to prove
that
when $(u,v)$ is a CR-pair in some open set $\Omega$, then
they are not only of class $C^1$ but have continuous derivatives of
any order, i.e. both $u$ and $v$
are
$C^\infty$-functions.
\medskip


\noindent
{\bf{A notation.}}
Let $\Omega$ be an  open set in
${\bf{C}}$. The family of $C^1$-functions $f=u+iv$
for which $(u,v)$ is a CR-pair in $\Omega$ is denoted by
$\mathcal O(\Omega)$. We refer to this as the class of analytic
functions in $\Omega$ and remark that one also
refers to the class of \emph{holomorphic functions}, i.e. the notion of 
complex analytic functions and holomorphic functions are the same.
\newpage

\noindent Next follows a brief account of the material in this chapter.


\medskip

\noindent
{\bf{ 0.1 Complex derivatives}}.
Consider a  $C^1$-function $f(z)=u(x,y)+iv(x,y)$ 
defined in some open set 
$\Omega$. If $z_0\in\Omega$  there exist  complex difference quotients
\[ 
\frac{f(z_0+\Delta z)-f(z_0)}{\Delta z}\quad\colon \quad z_0\in\Omega
\]
We say that $f$ has a \emph{complex derivative} at $z_0$ if
these complex difference quotients have a limit as
$\Delta z=\Delta x+i\Delta y\to 0$. During
the
passage to
the limit
$\Delta x$ and $\Delta y$ are not constrained, i.e. it is only
required that both $\Delta$-numbers tend to zero. For example, we can
allow that $\Delta x$ tends much faster to zero than
$\Delta y$, or vice versa. 
In � 4 we show that the existence of a complex derivative
at a point $z_0=x_0+iy_0$ is \emph{equivalent} to the condition that
$u_x(x_0,y_0)=v_y(x_0,y_0)$ and
$u_y(x_0,y_0)=-v_x(x_0,y_0)$  hold. Hence the
$C^1$-function $f(z)$ in $\Omega$ has a complex derivative everywhere if and
only if $(u,v)$ is a $CR$-pair.
\medskip

\noindent
In  � 5 we  relax the regularity hypothesis that
$f$ from the start is of class $C^1$. More precisely, assume only that
$f(z)$ is continuous in $\Omega$ and that the
pointwise defined complex derivative exists for 
every $z_0\in\Omega$.
\emph{Goursat's theorem} shows that
$f$ is automatically of class $C^1$ and hence
$(u,v)$ is a CR-pair and they even become $C^\infty$-functions in
$\Omega$.
\medskip

\noindent
{\bf{0.2 Schwarz' reflection.}}
Let $f(z)$ be analytic in an open rectangle
\[ 
\square_+=\{ (x,y)\quad\colon\, 0<y<b\quad
\text{and}\quad -A<x<A\}
\]
A reflection in the real $x$-axis gives the rectangle
\[
\square_-=\{ (x,y)\quad\colon\, -b<y<0\quad
\text{and}\quad -A<x<A\}
\]
In $\square_-$ we define the complex valued function
\[ 
g(z)=\bar f(\bar z)=u(x,-y)-iv(x,-y)
\]
One verifies easily that
$g(z)$ becomes analytic in $\square_-$.
Suppose now that 
$f(z)$ extends continuously 
to the $x$-axis, i.e.  
\[ 
\lim_{\epsilon\to 0}\, f(x+i\epsilon)=f_*(x)
\] 
exists uniformly with respect to $x$.
If $f_*(x)$ is \emph{real-valued}
then $f$ and $g$ attain the same boundary
values on the real $x$-interval. In � 5 we prove that
under this assumption the pair $f,g$
not only match each other as continuous functions
on the $x$-interval but there exists an analytic function
$F(z)$ defined in the whole open square
\[
\square=\{ (x,y)\quad\colon\, -b<y<b\quad
\text{and}\quad -A<x<A\}
\] 
such that $F=f$ in $\square_+$ and $F=g$ in $\square_-$.
This result is due to Hermann Schwarz who
applied the reflection principle 
to establish many other results, such as
the existence of a conformal map from the unit disc
to a domain bordered by a piecewise linear Jordan curve.

\medskip

\noindent
{\bf{0.3 A more general reflection.}}
Let $\square_+$ and $\square_-$
be as in (0.2).
Consider a
pair $f\in\mathcal O(\square_+)$ and
$g\in\mathcal O(\square_-)$.
Then the following extension of Schwarz's result holds:


\medskip

\noindent
{\bf{0.4 Theorem.}}
\emph{Assume that}
\[ \lim_{\epsilon\to 0}
\int_{-A}^A\, \bigl|f(x+i\epsilon)-g(x-i\epsilon)\bigr|\cdot dx=0
\]
\emph{Then there exists $F\in\mathcal O(\square)$ which gives an analytic
extension of the pair $(f,g)$.}
\medskip

\noindent
This result  is due to Carleman in [Car]. 
The proof relies upon properties of
subharmonic functions and is given in
the Appendix about Distributions. 
Notice that no
growth conditions on $f$ and $g$ are imposed
from the start in the theorem above.



\medskip




\noindent
{\bf{0.5 An inequality by Schwarz}}.
Let $D$ be the unit disc $|z|<1$ and $f$  
an analytic function in
$D$ whose maximumn norm is at most one, i.e.
\[
|f(z)|\leq 1\quad\colon\, z\in D\tag{i}
\]
Consider a bounded and connected subset 
$\Omega$ of the upper half-disc $D_+$
whose boundary intersects
the real axis in some interval $(a,b)$. Now
$\partial\Omega$ is the union of $[a,b]$
and the remaining portion of the boundary is denoted by
$\Gamma$ which  is a subset of
$D_+$ and the points
$a$ and $b$ belong to
the closure of $\Gamma$. Put
\[
\Omega_*=\{z\,\,\colon\, \bar z\in \Omega\}
\]
and assume that
$\Omega\cup\Omega_*\cup(a,b)$  is an open subset of $D$.
Under this condition one has:
\[
\max_{a\leq x\leq b}\, |f(x)|\leq \sqrt{|f|_\Gamma}\tag{*}
\]
To prove this  Schwarz considered the function
\[
 F(z)=f(z)\cdot \bar f(\bar z)
\]
which to begin with is analytic
in $\Omega\cup\Omega_*$. The reflection principle entails that
$F$ extends to an analytic set in the open set
$U=\Omega\cup\Omega_*\cup(a,b)$. Here we notice that 
the boundary
$\partial U=\Gamma\cup\Gamma_*$. From (i) it is clear that
\[
|F|_{\partial U}\leq |f|_\Gamma
\]
In � xx we establish the maximum principle for analytic functions
which entails  that
when $a\leq x\leq b$ then
\[
|f(x)|^2= |F(x)|\leq  |f|_\Gamma
\]
and (*) follows when we take the square root.



\bigskip

\noindent
{\bf{0.6 The $\bar\partial$-operator}}.
The Cauchy-Riemann equations can
be expressed by a
a single first order differential equation. Namely,   introduce
the differential operator
\[
\bar\partial=\frac{1}{2}(\partial_x+i\partial_y)
\]
If $f=u+iv$ is a complex-valued function it is clear
that $(u,v)$ is a CR-pair if and only if
$f$ satisfies the homogeneous $\bar\partial$-equation:
\[
\bar\partial(f)=
\frac{1}{2}[f_x+if_y)=0
\]
An important result is the \emph{Pompieu formula}
which shows how to solve the
\emph{inhomogeneous}
$\bar\partial$-equation
\[ 
\bar\partial (f)=g
\]
when $g$ is a continuous function with compact suppoprt. See � 8 for details.
 \medskip

\noindent
{\bf{0.7 Normal families.}}
Cauchy's integral  formula  is used to show
that an analytic function $f$ defined in
an open disc
$D_R(z_0)=\{|z-z_0|<R\}$  is represented by
a power series
\[ 
\sum\, c_n(z-z_0)^n\tag{*}
\] 
whose radius of convergence is $\geq R$.
In this way $\mathcal O(D_R(z_0))$ is identified with
the set of power series
whose radius of convergence  is $\geq R$.
Moreover, the
coefficients $\{c_n\}$ are determined by the formula
\[ 
c_k=
\frac{1}{k!}\cdot f^{(k)}(z_0)\quad\colon\, k=0,1,2, \ldots
\]


\noindent
Using  power series we establish  results due to Montel 
concerned with the topology on
$\mathcal O(\Omega)$.
Of special importance is the following result: 

\medskip

\noindent
\emph{Let
$\{f_\nu\}$ be a sequence of
analytic functions in a domain
$\Omega$ whose maximum norms are
uniformly bounded, i.e. there is a constant $M$ such that
$|f_\nu(z)|\leq M$ hold for all $z\in\Omega$ and every
$\nu$.
Then the sequence contains at least one subsequence
$\{g_k=f_{\nu_k}\}$ which  converges uniformly to
an analytic  function $g_*$ in every relatively compact subset of
$\Omega$. Moreover, if there exists an integer $N$ such that
every $g_k$ has at most $N$ zeros counted with multiplicity in
$\Omega$, then the same hold for
the limit function $g_*$, unless it is
identically zero.}
\medskip

\noindent
This result will be used frequently later on. For example in 
the study of conformal mappings in Chapter VI.

\medskip

\noindent
{\bf{0.8 Laurent series.}}
In � 10 we study analytic functions defined
in
domains 
$\{r<|z|<R\}$.
Here the boundary consists of the inner circle $|z|=r$ and the
outer circle $|z|=R$.
Let  $f(z)$ be analytic in such an annulus which extends to a continuous
function on
the boundary. Then we can
apply Cauchy's formula from (***)  and obtain  
a series representation of $f$
where one part of the series is an expansion with
\emph{negative} powers of $z$.





\medskip

\noindent
{\bf{0.9 Conformal properties}}
Let $f(z)$ be  an analytic function with non-zero complex derivative 
defined in some domain $\Omega$.
We can regard  $f$  as 
a map from the complex $z$-plane into another
complex plane and put
\[
\zeta=f(z) 
\]
where $\zeta=\xi+i\eta$.
Consider some point $z_0\in\Omega$ and with
$f=u+iv$ we have
$\xi=u(x,y)$ and $\eta=v(x,y)$.
The complex-valued function $f$ is now identified with a
vector-valued function from
the real $(x,y)$-space to the real $(\xi,\eta)$-space whose 
\emph{Jacobian} 
is the $2\times 2$-matrix
\[ 
J=\bigl(\begin{matrix}u_x&v_x\\u_y&v_y
\end{matrix}\bigr)
\]
\medskip

\noindent
The Cauchy-Riemann equations imply that
the two column vectors are orthogonal, i.e. 
\[
u_xv_x+u_yv_y=0
\]
holds at every point in $\Omega$. We have also the determinant formula:
\[
 \text{det}(J)= u_xv_y-u_yv_x=u_x^2+u_y^2=|f'(z)|^2\tag{*}
 \]
By a wellknown result in Calculus this implies that
the vector valued map 
is  infinitesmally a  rotation times
a dilation  with the factor $|f'(z)|^2$
at every point $z\in\Omega$.
This implies that the map is locally conformal.
Namely, let $z_0\in\Omega$
and consider a pair of $C^1$-curves
$\gamma_1,\gamma_2$ which pass $z_0$ and let
$\alpha$ be the angle between them.
Then the angle between the
image curves $f(\gamma_1)$ and $f(\gamma_2)$
is  equal to $\alpha$.
This  means that infinitesmal angles are preserved and is 
expressed by saying that
$f$ yields a conformal map.
  
\medskip



\noindent
{\bf{Remark on quasi-conformal mappings.}}
In  the 
theory about quasi-conformal mappings
many "magical phenomena" occur.
See the text-books [Ahl] by Ahlfors and
[L-V] by Lehto and Virtanen for the
quasi-conformal  theory.
Let us  recall that a differentiable 
function $f=u+iv$
which locally is an orientation preserving
homeomorphism is     quasi-conformal of 
order $\leq K$ for some number $K\geq 1$ if
the first order derivative of $f$ satisfy:
\[
\bigl|\bar\partial(f)\bigr|\leq
\frac{K-1}{K+1}\cdot \bigl|\partial(f)\bigr|
\]
When $K=1$ this means that $\bar\partial(f)=0$, i.e. $f$
is complex analytic. 
Consider as an example
the linear function 
$f(z)=z+\frac{\bar z}{2}$. Here we require that
$2(K-1)\geq K+1$, i.e.  take $K=3$.
Notice that  the linear map: 
\[ 
(x,y)\mapsto  (3x/2,y/2)
\] 
sends  small circles centered at the origin to
small ellipses. So the geometry is more involved 
when quasi-confomal mappings are studied.
 
\medskip

\noindent{\bf{0.10 Area formulas.}}
Let $\Omega$ be a domain in $\mathcal D(C^1)$ where
$\partial\Omega$ consists
of simple and closed boundary
curves
$\gamma_1,\ldots,\gamma_p$.
Let $f(z)\in \mathcal O(\Omega)$ and suppose  it extends to
a continuous function on $\bar\Omega$. Assume also
that $f$ is bijective on $\bar\Omega$ and that the image domain
$f(\Omega)$ also belongs to $\mathcal D(C^1)$.
Then $f(\Omega)$
is  bordered by
a $p$-tuple of disjoint boundary curves
$f(\gamma_1),\ldots,f(\gamma_p)$.
Write  $f=u+iv$ and
to each $1\leq k\leq p$
we consider a parametrisation by arc-length along
$\gamma_k$ and  evaluate
the line integral
\[ 
J(k)=\int_0^{\ell(\gamma_k)}\,
u(z_k(s))\cdot \frac{dv(z_k(s))}{ds}\cdot ds\tag{*}
\]
where $\gamma_k\colon s\mapsto z_k(s)$ and 
$\ell(\gamma_k)$ is the arc-length of $\gamma_k$.
With these notations  Stokes Theorem from Chapter II
gives the following area formula:
\medskip

\noindent
{\bf{0.11 Theorem}} 
\emph{The area of $f(\Omega)$ is equal to $J(1)+\ldots+J(p)$.}


\medskip

\noindent{\bf{Remark.}} Since
the absolute value of $f'(z)$ changes the area measure 
we have also the equality:
\[
\text{Area}[f(\Omega)]=
\iint_\Omega\, |f'(z)|^2\cdot dxdy\tag{**}
\]


\noindent
In Section XX we establish a third area formula using
complex line integrals along the boundary curves of $\Omega$:
\[
\text{Area}[f(\Omega)]=\int_{\partial\Omega}\, 
\bar f(z)\cdot f'(z)\cdot dz\tag{***}
\]


\noindent
The fact that one disposes  these
three   area formulas is  
quite useful.

\bigskip


\noindent
{\bf{0.12 A local limit formula.}}
If $f(z)$ is analytic in a bounded domain
it suffices to know its values on a small portion of the boundary.
Following  the 
introduction in Carleman's book [Quasianalytic]
we shall illuminate this in a special case where formulas are 
quite explicit. Here is the set-up.
Let $0<\alpha<1/2$ be a real number and 
let $\ell_*$ be the ray
along
the non-negativel real axis
$\ell_*$ and the second ray $\ell^*=\{ re^{i\alpha}\,\colon\, r\geq 0\}$.
Let $\Gamma$ be a Jordan arc with end-points
$A\in\ell_*$ and $B\in \ell^*$ while the remaining part of
$\Gamma$ is in the interior of
the open sector bordered by the two rays. So here $A$ is a 
positive real number and $B=b\cdot e^{i\alpha}$ for some $b>0$.
Let $\Omega$ be the Jordan domain bordered by
$\Gamma$ and the straight lines
$0A$ and $0B$ which intersect at the origin.
Consider a point $\zeta=r\cdot e^{i\alpha/2}$
which belongs to $\Omega$.
The Jordan arc $\Gamma$ is assumed to be rectifiable so that  complex line integrals along
$\Gamma$ are defined.
Let us now consider an analytic function
$f$
in $\Omega$ which  extends to a continuous function on
the closure $\bar\Omega$.
\medskip

\noindent
{\bf{0.13 Theorem.}}
\emph{The value of $f$ at $\zeta$ is obtained by the limit formula}
\[ 
f(\zeta)= \lim_{\sigma\to\infty}\,
\frac{e^{-\sigma}}{2\pi i}\cdot \int_\Gamma
\frac{f(z)\cdot e^{\sigma\bigl(\frac{z}{\zeta}\bigr)^{\frac{1}{\alpha}}}}{z-\zeta}\cdot dz
\tag{*}
\]


\noindent
\emph{Proof.}
For each positive real number $\sigma$ we have the function
\[ 
F_\sigma(z)= f(z)\cdot e^{\sigma\bigl(\frac{z}{\zeta}\bigr)^{\frac{1}{\alpha}}}\tag{1}
\]
When  $0\leq s\leq A$
we have
\[
\bigl(\frac{s}{\zeta}\bigr)^{\frac{1}{\alpha}}
=\bigl(\frac{s}{r}\bigr)^{\frac{1}{\alpha}}\cdot e^{-\pi i/2}=
-i\cdot\bigl( \frac{s}{r}\bigr)^{\frac{1}{\alpha}}
\]
Since exponentials of purely imaginary numbers have absolute value  we get
\[
|F_\sigma(s)|= |f(s)|
\]
On the ray $OB$ we find
a similar formula.
Hence $F_\sigma(z)$ is bounded on $OA$ and $OB$. We also notice that
\[ 
F_\sigma(\zeta)=e^\sigma\cdot f(\zeta)
\]
Cauchy's integral formula applied to $F_\sigma$ gives
\[
 f(\zeta)= e^{-\sigma}\cdot F_\sigma(\zeta)=
e^{-\sigma}\cdot
\frac{1}{2\pi i}\cdot \int_{\partial\Omega}
\frac{F_\sigma(z)}{z-\zeta}\cdot dz
\]



\noindent
The last line integral is the sum over $\Gamma$ and the two line integrals along
$0A$ and $0B$.
Since $F_\sigma $ is bounded on the line segments and $e^{-\sigma}\to 0$
as $\sigma\to+\infty$
we conclude that
\[
f(z)=\lim_{\sigma\to\infty}\, 
\frac{e^{-\sigma}}{2\pi i}\cdot \int_\Gamma 
\frac{F_\sigma(z)}{z-\zeta}\cdot dz
\]
By the construction of the $F$-functions this entails
the limit formula (*).
\bigskip


\noindent
{\bf{Remark.}}
If $f$ from the start is defined in some domain $U$ which
is starshaped with respect to the origin then
we pick $0�\zeta\in U$
and
after a rotation we may assume that
$\zeta$ is real and positive. With a very small $\alpha$
we can consider the rays from the origin
where $\text{arg}(z)$ is $\alpha$ or $-\alpha$
and by a picture the
reader can see that $f(\zeta)$ via Theorem  0.13 is expressed by a limit where
the integral is taken over a small portion of
$\partial U$.

















\newpage




\centerline {\bf 1. Complex Line Integrals}

\bigskip

\noindent
Consider a complex valued $C^1$-function of a real $t$-variable:
\[ 
t\mapsto z(t)=x(t)+iy(t)\quad\colon\quad 0\leq t\leq T
\]
The $C^1$-condition means that both $x(t)$ and $y(t)$ are
continuously differentiable functions of $t$. The $t$-derivative becomes:
\[ 
\dot z(t)=\dot x(t)+i\cdot \dot y(t)
\]
If $f=u+iv$ is a complex valued continuous function we
get the line integral
\[
\int_0^T\, f(z(t))\dot z(t)dt=\int_0^T\,
[u(x(t),y(t))+iv(x(t),y(t))](\dot x(t)+i\cdot \dot y(t))\cdot dt
\]
or expressed in a more abbreviated form:
\[
\int_0^T\, f(z)\dot z\cdot dt=\int_0^T\, 
[u(x,y)+iv(x,y)](\dot x+i\dot y)\cdot dt\tag{0.1}
\]
The right hand side
is a  sum of line integrals with
respect to $x$ and $y$ respectively.
By the general result in XX,
(0.1) does not depend on the
chosen parametrization of the oriented image 
$\Gamma$.
So we can therefore write (0.1) as
\[
\int_\Gamma\,fdz=\int_\Gamma\,u+iv)(dx+idy)\tag{0.2}
\]
When the right hand side is decomposed into its real and  imaginary parts
we get:
\[
\int_\Gamma\,fdz=\int_\Gamma\,udx-vdy+
i\cdot \int_\Gamma\, u dy+iv dx\tag{0.3}
\]
We refer to (0.2) as
the complex line integral along 
$\Gamma$. Recall that the \emph{choice of orientation}
is essential, i.e. if the orientation on $\Gamma$ is 
opposite the line integral changes sign.

\bigskip
\noindent
{\bf 1.1 Complex line integrals as Riemann sums.}
Consider as above a
curve $\Gamma$ with end points $a$  and
$b$. Following the  orientation we choose a 
finite sequence of points $a=z_0,z_1,\ldots,z_N=b$
where each $z_\nu\in \Gamma$ and take the Riemann sum
\[ 
\sum_{\nu=0}^{N-1}\, f(z_\nu)(z_{\nu+1}-z_\nu)\tag{i}
\]
When $\text{max}\,|z_{\nu+1}-z_\nu|\to 0$ these sums converge to the line
integral of $f$ along $\Gamma$. To see this we use the hypothesis
that $\Gamma$ has a $C^1$-parametrisation,
say $t\mapsto z(t)$.
Now $z_\nu=z(t_\nu)$ with $0=t_0<t_1<\ldots <t_N=T$.
Since $f$ is continuous the function
$t\mapsto f(z(t))$ is continuous. From the previous definition of
the line integral we have
\[ 
\int_\Gamma\, f\cdot dz=
\int_0^T\,f(z(t))\cdot \dot z(t)dt  \tag{ii}
\]
Here (ii) is approximated just as in Calculus by a Riemann sum:
\[
\sum\, f(z(t_\nu))\cdot \dot z(t_\nu)\cdot (t_{\nu+1}-t_\nu)\tag{iii}
\]
The continuity of the $t$-derivative $t\mapsto \dot z(t)$
give accurate approximations
\[
\dot z(t_\nu)\cdot (t_{\nu+1}-t_\nu)\simeq z_{\nu+1}-z_\nu
\] 
for every $\nu$. Hence
the Riemann sums in (iii) converge to
(ii) when
\[
\max \{(t_{\nu+1}-t_\nu)_{\nu=1}^N\}\quad \text{tends to zero}
\]



\bigskip

\noindent {\bf 1.2 Integration on rectifiable curves.}
The approximative sums in (1)
appear in the construction of Riemann-Stieltjes integrals.
Since the complex integral is decomposed into a real and an
imaginary part we can therefore use the
result from XX where we proved
the existence of integrals
of the Riemann-Stieltjes type. More precisely, if $\Gamma$ has a
parametrisation $t\mapsto z(t)=x(t)+iy(t)$
where both $x(t)$ and $y(t)$ are continuous functions with
bounded variation, then we can define Stieltjes' line integral
\[
\int_\Gamma\, f(z)\cdot dz
\]
where we only have to assume that
$f(z)$ is a bounded Borel function.
See Measure Appendix for details about this construction of
general line integrals.

\bigskip

\noindent {\bf 1.3 The case when
$\Gamma$ is a circle.}
Let $R>0$ and  $\Gamma$ is the circle $|z|=R$ 
equipped with its usual positive orientation.
Since $z\neq 0$ on $\Gamma$ we can divide a function 
$f$ with $z$ and get the line integral
\[ 
\int_\Gamma\,\frac{f(z)dz}{z}\tag{i}
\]
Using the parametrisation $\theta\mapsto Re^{i\theta}$ this line
integral becomes
\[ 
\int_0^{2\pi}\,
\frac{f(Re^{i\theta})\cdot iRe^{i\theta}d\theta}{Re^{i\theta}}=
i\cdot \int_0^{2\pi}\,
f(Re^{i\theta})d\theta\tag{ii}
\]
This formula plays a crucial role later on when
we derive Cauchy's formula and develop residue calculus.
Under the sole assumption that $f$ is a continuous function
defined in some open disc centered at the origin, we use the
equality above when $R=\epsilon$ and $\epsilon\to 0$. Namely,
since
$f$ is continuous at the origin we get  the limit formula:

\[
f(0)=\frac{1}{2\pi}\cdot \lim_{\epsilon\to 0}\,
\int_0^{2\pi}\,
f(\epsilon e^{i\theta})d\theta=
\frac{1}{2\pi i}\cdot \lim_{\epsilon\to 0}\, \int_{|z|=\epsilon}\,
\frac{f(z)dz}{z}\tag{iii}
\]
The origin can be replaced by any other point
$z_0$.
Since the limit formula above
is so important  for the subsequent residue calculus
we state it separately:
\bigskip

\noindent {\bf 1.4 Theorem} \emph{Let $f(z)$ be a continuous function in some open set
$\Omega$. Then}
\[
f(z_0)=\frac{1}{2\pi i}\cdot \lim_{\epsilon\to 0}\, \int_{|z-z_0|=\epsilon}\,
\frac{f(z)dz}{z-z_0}\quad\colon\quad z_0\in\Omega
\]
\medskip

\noindent

















\bigskip

\centerline{\bf �  2. The Cauchy-Riemann equations}
\bigskip

\noindent
Consider a complex-valued function
$f(z)=f(x+iy)=u(x,y)+iv(x,y)$.
We assume that $f$ is of class $C^1$
and let
$\Omega\in\mathcal D(C^1)$. From � 1 we have
\[ 
\int_{\partial\Omega}\, fdz=
\int_{\partial\Omega}\, udx-vdy+i\dot \int_{\partial\Omega}\,udy+vdx
\]
\medskip
Apply Theorem 2.4 in Chapter I to each term in the right hand side.
This gives:
\[
\int_{\partial\Omega}\, fdz=\iint_\Omega\,(-(u_y+v_x)\cdot dxdy+
i\cdot \iint_\Omega\,(u_x-v_y)\cdot dxdy\tag{i}
\]
\medskip

\noindent
The right hand side is zero if the real and
the imaginary part vanish which obviously follows
if 
the following equations hold in the whole of $\Omega$:
\[
u_x(x,y)=v_y(x,y)\,\quad\colon\,\quad u_y(x,y)=-v_x(x,y)\quad\colon\quad
(x,y)\in\Omega\tag{*}
\]
Hence we have proved the following
\bigskip 

 \noindent {\bf 2.1 Theorem}\emph{ Let $f=u+iv$ be a complex-valued
$C^1$- function such that the pair $(u,v)$ satisfies (*). 
Then}
\[ \int_{\partial\Omega}\, fdz=0\quad\colon\quad\Omega\in\mathcal D(C^1)
\]


\noindent
Theorem 2.1 suggests the following
\medskip

\noindent
{\bf 2.2 Definition} \emph{A pair of real-valued $C^1$-functions
$(u,v)$ satisfying $u_x=v_y$ and $u_y=-v_x$ is called a 
Cauchy-Riemann pair and
in this case $f=u+iv$ is called an analytic function.}
\bigskip





 
\centerline {\bf � 3. The $\bar\partial$-operator}
\medskip

\noindent
The  Cauchy-Riemann equations can be
described by a single first order differential operator.
Namely,  set
\[
\bar\partial=\frac{1}{2}(\partial_x+i\partial_y)\tag{i}
\]
If $f=u+iv$ is a $C^1$-function we get:
\[
\bar\partial(f)=
\frac{1}{2}[\partial_x(f)+i\partial_y(f)]=
\frac{1}{2}[u_x+iv_x+iu_y-v_y]=\frac{1}{2}(u_x-v_y)+
\frac{i}{2}(u_y+v_x)
\]
We conclude that $\bar\partial(f)=0$ if and only if
$(u,v)$ is a $CR$-pair.
One refers to $\bar\partial$ as the \emph{Cauchy-Riemann operator}
since it determines when $(u,v)$ becomes a $CR$-pair.
\medskip

\noindent 
{\bf 3.1 Example} Let $f=\bar z=x-iy$ be the conjugate function.
Here 
$\bar\partial(\bar z)=1$ and 
hence the pair $u=x$ and $v=-y$ is not $CR$. 
On the other hand, let $m\geq 1$ and consider $f(z)=z^m$.
Since $\bar\partial$ is a first order differential 
operator, Leibniz' s rule from Calculus gives
\[
\bar\partial(z^m)=mz^{m-1}\bar\partial(z)
\]
Next, we have
\[ 
\bar\partial(x+iy)=\frac{1}{2}[1+i^2]=\frac{1}{2}[1-1]=0
\]
So with $z^m=u+iv$ it follows that $(u,v)$ is a $CR$-pair. 
Take as an example the
case $m=3$ where
$u=x^3-3x\cdot y^2$ and
$v=3x^2\cdot y-y^3$.
\medskip


\noindent
{\bf 3.2 Remark} If $f,g$ is a a pair of $C^1$-functions then
Leibniz's rule for the first order differential operator
gives
\[ 
\bar\partial(fg)= f
\bar\partial(g) +g
\bar\partial(f)
\]
It follows that
if both $f$ and $g$ are analytic so is $fg$. 
Hence the class of analytic functions is stable under products.
Next, let $f_1,\ldots,f_m$ a  finite set of analytic functions.
Put
\[ 
\phi=\bar z\cdot f_1(z)+\ldots+\bar z^m\cdot f_m(z) 
\]
Then $\phi$ cannot be analytic unless every $f_\nu$ is zero.
To see this one proceeds by induction over $m$. Namely, for each
$m\geq 2$ we get the $m$:th order differential operator
$\bar\partial^m$ which satisfies
\[
\bar\partial^m(\bar z^\nu)=0\quad\colon\,\nu<m\quad\colon\quad
\bar\partial^m(\bar z^m)=m\!
\]
Using this the reader may verify the assertion about the
$\phi$-function.
\medskip

\noindent {\bf 3.3 Analytic polynomials.}
Since 
$x=\frac{z+\bar z}{2}$ and
$y=\frac{z-\bar z}{2i}$, every
polynomial  in  the two variables $(x,y)$ can be expressed as 
a polynomial in 
$z$ and $\bar z$.
Let $m\geq 1$ and denote by $H_\text{Pol}(m)$
the space of homogeneous polynomial of degree $m$ with complex coefficients.
Such a polynomial can be uniquely written in the form
\[
P(z,\bar z)=\sum_{\nu=0}^{\nu=m}
c_\nu \cdot z^{m-\nu}\bar z^\nu\quad \colon\quad c_0,\ldots,c_m \,\, \text{complex constants}
\]
The observation in 3.2 shows that
$P$ is analytic if and only if the sole term is $z^m$.
Hence $H_\text{Pol}(m)$ is 1-dimensional complex vector space generated by the 
monomial $z^m$. This shows that the class of analytic functions is quite sparse.







\newpage

\centerline{\bf  � 4. The  Complex Derivative}
\bigskip

\noindent
Let $f(z)=u(x,y)+iv(x,y)$ be a complex-valued function
of class $C^1$ in a domain $\Omega$.
Given a point $z_0=x_0+iy_0$ and a small complex number
$\Delta z=\Delta x+i\Delta y$ we regard the difference
\[
f(z_0+\Delta z)-f(z_0)=
u(x_0+\Delta x,y_0+\Delta y)+
iv(x_0+\Delta x,y_0+\Delta y)
\]
Keeping $\Delta x$ and $\Delta y$ fixed for a while we have the function
\[
\phi(s)=
u(x_0+s\Delta x,y_0+s\Delta y)+
iv(x_0+s\Delta x,y_0+s\Delta y)\quad\colon\quad 0\leq s\leq 1
\]
\emph{Rolle's mean value theorem} gives a pair 
$0<\theta_1,\theta_2<1$ such that the sum


\[
\Delta x\cdot u'_x(x_0+\theta_1\Delta x,y_0+\theta_1\Delta y)+
\Delta y\cdot u'_y(x_0+\theta_1\Delta x,y_0+\theta_1\Delta y)+
\]
\[
i[\Delta x\cdot v'_x(x_0+\theta_2\Delta x,y_0+\theta_2\Delta y)+
\Delta y\cdot u'_y(x_0+\theta_2\Delta x,y_0+\theta_2\Delta y)]\tag{1}
\]


\noindent
is equal  to $\phi(1)-\phi(0)=f(z_0+\Delta z)-f(z_0)$.
In this sum the four first order partial derivatives are evaluated at points close to 
$(x_0,y_0)$ when 
\[
|\Delta z|=\sqrt{(\Delta x^2+(\Delta y)^2} \to 0
\]
The continuity of the partial derivatives and the 
imply that the sum from (1)  becomes
\[
\Delta x\cdot u'_x(x_0,y_0)+
\Delta y\cdot u'_y(x_0,y_0)+
i\Delta x\cdot v'_x(x_0,y_0)+
i\Delta y\cdot v'_y(x_0,y_0)+\text{small ordo}(|\Delta z|)\tag{2}
\]
\medskip

\noindent
The  remainder term $o(|\Delta z|)$  
comes  from the continuity of
first order derivatives.
If we  \emph{assume} that $u,v$ is a Cauchy-Riemann pair
we can replace $v'_y$ with $u'_x$ and $u'_y$ with $-v'_x$. Then (2)
becomes
\[
(\Delta x+i\Delta y)u'_x+i(\Delta x+i\Delta y)v'_x+\text{small ordo}(\delta)\tag{3}
\]



\noindent 
Here $\Delta z=\Delta x+i\Delta y$ appears as a common factor.
The small ordo-term gives the limit formula:
\[
\lim_{\Delta z\to 0}\,
\frac{f(z_0+\Delta z)-f(z_0)}{\Delta z}=u'_x(x_0,y_0)++iv'_x(x_0,y_0)\tag{4}
\]
\medskip

\noindent
So when $(u,v)$ is a $CR$-pair the complex difference quotients
$\frac{f(z_0+\Delta z)-f(z_0)}{\Delta z}$ have a limit as $\Delta z\to 0$. The limit is called the \emph{complex derivative} of $f$ at the point $z_0$. Put
\[
f'(z_0)=
\lim_{\Delta z\to 0}\,
\frac{f(z_0+\Delta z)-f(z_0)}{\Delta z}\tag{*}
\]
\medskip

\noindent
{\bf   A converse.} Assume that $f=u+iv$ has a complex derivative at
$z_0=x_0+iy_0$. We can approach this point in two ways -  along the $x$ axis 
or along the $y$-axis. With $\Delta z=\Delta x$ we get
from the definition of partial derivatives 
\[
f'(z_0)=
\lim_{\Delta x\to 0}\,
\frac{f(z_0+\Delta z)-f(z_0)}{\Delta z}=u_x(x_0,y_0)+iv_x(x_0,y_o)
\]
If we instead take $\Delta z=i\Delta y$ we have
\[
f'(z_0)=
\lim_{i\Delta y\to 0}\,
\frac{f(z_0+\Delta z)-f(z_0)}{\Delta z}=\frac{1}{i}u_y(x_0,y_0)+v_y(x_0,y_o)
\]
Identifying the real and the imaginary parts of the two expressions for the complex derivative $f'(z_0)$ we recover the Cauchy-Riemann equations.
Hence we have proved the following
\medskip

\noindent {\bf 4.1 Theorem.} \emph{ A $C^1$-function $f=u+iv$ defined in a domain
$\Omega$ has a complex derivative at every point
if and only if $(u,v)$ is a $CR$-pair.}

\medskip

\noindent {\bf 4.2 The space $\mathcal O(\Omega)$.}
Let $\Omega$ be an open subset of $\bf C$. The class of analytic functions in
$\Omega$ is denoted by $\mathcal O(\Omega)$. From the result in 2.7 this gives a 
subalgebra of all complex valued $C^1$-functions
in $\Omega$.


\newpage

\centerline {\bf � 5. Morera's and Goursat's theorems.}
\medskip

\noindent
Let $f(z)$ be a continuous 
and complex-valued function defined in an open square
$\square=\{-A<x,y<A\}$.
To every  point $p=(a,b)\in \square$ we get the rectangle
$\Gamma$ with corners at the origin, $(a,0),(a,b),(0,b)$.
Suppose that $\int_\Gamma\,f(z)dz=0$ for
every such rectangle.
Define a function $F(z)$ by
\[ 
F(x+iy)=\int_0^x\, f(t,0)dt+\int_0^y\, f(a,s)ids
\]
It is obvious that $F'_y=if$. Next, the hypothesis on $f$ implies that we also have
\[
F(x+iy)=-\int_0^y\, f(0,s)ids-\int_0^x\, f(t,y)dt
\] 
From this we see that $F'_x=-f$. Hence
$F'_x=F_y$ and since $F$ also is a $C^1$-function this implies that
$F(z)$ is analytic  by the result in 2.5.
Next, if we \emph{knew} that $F$ is of class $C^2$
we can take the mixed second order derivatives and obtain
\[
-f'_y=F''_{yx}=F''_{xy}=if'_x
\]
This gives $f'_x=if'_y$ and 2.5 proves that $f$ is analytic.
in the next section we show that
$F$ actually is of class $C^2$ and hence 
it follows that $f$ is analytic. Of course, allowing more
rectangles we can conclude:
\medskip

\noindent
{\bf 5.1 Theorem.}
\emph {Let $f(z)$ be continuous in an open set
$\Omega$. Assume that $\int_\Gamma\,f(z)dz=0$
for every rectangle inside
$\Omega$ with sides parallell to the coordinate
axes. Then $f$ is analytic in
$\Omega$.}
\bigskip

\noindent 
{\bf 5.2 Some estimates}. We begin with some preliminary results which
will be used to prove Theorem 5.3 below.
A
rectangle $\Gamma$
gives four smaller rectangles
$\gamma_1,\gamma_2,\gamma_3,\gamma_4$
which arise when we  join the parallell sides with lines from
their opposed mid-points.
Now $\gamma_1,\gamma_2$ has one side in common and so on.
By drawing a figure and keeping in mind how
the orientation is chosen along each small rectangle we see that:
\[
\int_\Gamma\, f(z)dz=
\sum_{\nu=1}^{\nu=4}\, f(z)dz
\]
This process can be continued.
Suppose now that $f(z)$ is a bounded function which is
no essential restriction since we otherwise may replace
$\Omega$ by a smaller set. if $|f(z)|\leq M$ in $\Omega$
the construction of a complex line integral gives
\[ 
|\int_\Gamma\, f(z)dz\,|\leq \ell(\Gamma)\cdot M
\quad\colon\,\ell(\Gamma)=\text{sum of lengths of the sides}
\]
Suppose now  that we have a rectangle $\Gamma$ where
$\int_\Gamma\, f(z)\neq 0$.
Let us put
\[
|\int_\Gamma\, f(z)|=A\cdot|\ell(\Gamma)
\] 
where $A$ now is $>0$.
Dividing $\Gamma$ in four
smaller portions we see that the equality
above gives
\[ 
\max_\nu\,|
\int_{\gamma_\nu}\, f(z)dz|\geq
\frac{1}{4}\cdot |\int_\Gamma\, f(z)dz| =\frac{A}{4}\cdot\ell(\Gamma)
\]
At the same time 
$\ell(\gamma_\nu)=\frac{1}{2}\cdot \ell(\Gamma)$.
Hence we find at least one small $\gamma$-rectangle where
\[
\int_{\gamma_\nu}\, f(z)dz|\geq
\frac{A}{2}\cdot\ell(\gamma_\nu)
\]
Starting from one such $\gamma$-rectangle it is decomposed in four pieces and so
on. In this way we obtain a nested descreasing sequence of
rectangles $\{\gamma_\nu\}\,\colon\,\nu=1,2,\ldots\}$
such that
\[
\int_{\gamma_\nu}\, f(z)dz|\geq
\frac{A}{2^\nu}\cdot\ell(\gamma_\nu)\quad\colon\nu\geq 1\quad\colon\,
\ell(\gamma_\nu)= 2^{-\nu}\ell(\Gamma)
\]
At this stage we are prepared to prove:

\medskip

\noindent
{\bf 5.3 Goursat's theorem.}
\emph{Let $f(z)$ be a continuous function in
$\Omega$ and assume it has a complex derivative at
every point. Then $f$ is analytic.}

\medskip

\noindent 
\emph{Proof.}
It suffices to work locally and we may take
$\Omega=\square$ as above.
If $f$ fails to be analytic we find $\Gamma$ and $A>0$ where
\[
|\int_\Gamma\, f(z)dz\,|=A\cdot  \ell(\Gamma)\cdot M
\]
Then there exists a nested sequence $\{\gamma_\nu\}$
and since there sides tend to zero, it follows by
Bolzano's theorem that
there is a limit point $z_0=\cap\,\gamma_\nu$.
By assumption $f$ has a complex derivative
at $z_0$. It means that for every $\epsilon>0$ there
exists $\delta>0$ such that
\[
|f(z)-f(z_0)-f'(z_0)(z-z_0)|\leq\epsilon |z-z_0|
\quad\colon\quad |z-z_0|<\delta
\]
If $\nu$ is sufficiently large then $\gamma_\nu$
is contained in the disc $D_\delta(z_0)$.
Next, we notice that
\[ 
\int_{\gamma_\nu}\,[f(z_0)+(z-z_0)f'(z_0)]dz=0
\]
Next, when $z\in\gamma_\nu$ we notice that
$|z-z_0|\leq \ell(\gamma_\nu)/2$.
Hence by the vanishing in XX and
the inequality XX above, the triangle inequality gives
\[
|\int_{\gamma_\nu}\,f(z)dz\,|
\leq\frac{1}{2}\epsilon\,\ell(\gamma_\nu)^2=
\frac{1}{2}\cdot \epsilon\cdot 4^{-\nu}\ell(\Gamma)
\]


\noindent
At the same time, during the construction of the nested sequence
we have 
\[
\int_{\gamma_\nu}\, f(z)dz|\geq
\frac{A}{2^\nu}\cdot\ell(\gamma_\nu)= A\cdot 4^{-\nu}\cdot
\ell(\Gamma)
\]
Now we get a contradiction since we take $\epsilon$ arbitrary small , i.e. 
it would even be sufficient
to take $\epsilon<2A$ and then the contradiction will follow
as soon as  $\nu$
is so large that $\gamma_\nu\subset D_\delta(z_0)$.


\bigskip

\noindent{\bf 5.4 A result by Carleman.}
Analytic functions
are also characterized by a local mean value condition.
\medskip


\noindent {\bf 5.5 Theorem} \emph{Let $f$ be a continuous function in $\Omega$ such that}
$\int_{\partial D}\, f(z) dz=0$ for ever disc $D\subset\Omega$. 
\emph{Then
$f\in\mathcal O(\Omega)$.}

\medskip

\noindent 
\emph{Proof.} Let $r>0$ be small and put 
\[
\Omega_r=\{z\in\Omega\,\,\colon\,\text{dist}(z,\partial\Omega)>r\}
\]
To each $z\in \Omega_r$ we define the mean value
\[ 
F_r(z)=\iint_{|\zeta-z|\leq r}\, f(z)dxdy=
\int_0^r\int_0^{2\pi}\, f(z+se^{i\theta}) s ds d\theta
\]
Now we prove that $F_r\in\mathcal(\Omega_r)$.
To see this we consider its partial derivatives with respect to $x$ and $y$.
With $z=x+iy$ and $\Delta x$ small, $F(x+\Delta x+iy)$
is the area integral over a disc centered at $(x+\delta x,y)$. Drawing a figure for
computing area integrals the reader should discover that we obtain
\[ 
F_x(z)=\int_0^{2\pi}\,\text{cos}(\theta)f(z+re^{i\theta}d\theta
\quad\text{and}\quad  
F_y(z)=\int_0^{2\pi}\,\text{sin}(\theta)f(z+re^{i\theta}d\theta
\]
It follows that
\[ 
F_x+iF_y=\int_0^{2\pi}\,e^{i\theta}f(z+re^{i\theta}d\theta=0
\] 
where the last equality follows from the mean value assumption.
Hence $F$ satisfies the $\bar\partial$-quation from XX and is therefore
analytic. Now this holds for any $r>0$ and  the continuity of $f$ gives:
\[ 
f(z)=\lim_{r\to 0}\frac{1}{\pi r^2}\cdot  F_r(z)
\]
Hence $f$ is the limit of a sequence of analytic functions and therefore analytic by the result to be proved in XXX.

\bigskip















\centerline{\bf � 6. Cauchy's integral formula}
\bigskip

\noindent
{\bf 6.1 The local residue.} Let us repeat the result which led to Theorem 1.2
once more since it plays such a fundamental role.
Let $z_0\in\bf C$
and let $g(z)$ be a continuous function defined
in some open disc of radious $r$ centered at $z_0$.
To each $0<\epsilon<r$ we set
\[
R_\epsilon(g)=\frac{1}{2\pi i}\int_{|z-z_0|=\epsilon}\,\frac{g(z)dz}{z-z_0}
\]
Using polar coordinates we get
\[ R_\epsilon(g)=\int_0^{2\pi i}\, \frac{g(z_0+\epsilon e^{i\theta}i d\theta}
{e^{i\theta}}=\int_0^{2\pi}\,\int_0^{2\pi}
g(z_0+\epsilon e^{i\theta}d\theta
\]
By continuity of $g$ at $z_0$ it follows that
\[ 
\lim_{\epsilon\to 0}\, R_\epsilon(g)=g(z_0)\tag{*}
\]
This local limit formula will now be applied
below   when $g(z)$ is an analytic function.
\bigskip

\noindent
{\bf 6.2 Cauchy's formula}
Let $f(z)\in \mathcal O(\bar\Omega)$ where $\Omega\in\mathcal D(C^1)$.
Let $z_0\in\Omega$ and with $\epsilon>0$ is 
small we remove the open disc of radius 
$\epsilon$ centered at $z_0$. Put
\[
\Omega_\epsilon=\Omega\setminus\{ |z-z_0|\leq\epsilon\}
\]
Now $\partial\Omega_\epsilon=\partial\Omega\cup\,\partial D_\epsilon$
where $D_\epsilon=\{z-z_0|<\epsilon\}$.
We get the function
\[
g(z)=\frac{f(z)}{z-z_0}\in\mathcal O(\bar\Omega_\epsilon)
\]
Applying Theorem 2.1  we get 
\medskip
\[
\int_{\partial\Omega}\,\frac{f(z)dz}{z-z_0}=
\int_{\partial D_\epsilon}\,\frac{f(z)dz}{z-z_0}
\]
Here $\epsilon$ can be arbitrarily small. The limit formula in (*) from
4.1 
shows that
the last term is equal to $2\pi if(z_0)$. Hence we have proved
\medskip

\noindent {\bf 6.3 Theorem.} \emph{Let $ f\in\mathcal O(\bar\Omega)$. Then}
\[ 
f(z_0)=
\frac{1}{2\pi i}\int
\int_{\partial\Omega}\,\frac{f(z)dz}{z-z_0}\quad\, \colon\quad z_0\in \Omega
\]
\bigskip

\noindent
{\bf 6.4 Expressions for derivatives}
Cauchys formula represents $f(z)$ inside $\Omega$
in the same way as in 3.3. Hence we can take derivative of any order, i.e. 
for each $m\geq 1$ we get
\[
f^{(m)}(z)=\frac{m\,!}{2\pi i}\int
\int_{\partial\Omega}\,\frac{f(z)dz}{(z-z_0)^{m+1}}\quad\, \colon\quad z_0\in \Omega
\]
This proves in particular that when $f$ is analytic, then it has 
complex derivatives. Moreover,
$f^{(m)}(z)$ yield now analytic functions in
$\Omega$.
In
XXX  we study  power series representations
and obtain even more information about regularity of
$f$ as well as of its real and imaginary parts.
Notice  that the conclusion above applies to the
function $F(z)$ from section 5 and hence Theorem 6.3 above
finishes the proof of \emph{Morera's Theorem.}




\bigskip




\noindent
{\bf 6.5 The case when $\Omega$ is a disc}
Let $\Omega=D_R$ be the disc of radius $R$ centered at the origin
and  $f(z)$ is an analytic function defined in a neighborhood of the closed disc
$\bar D_R$.
Using the parametrisation $\theta\mapsto Re^{i\theta}$
which gives  $dz=iRe^{i\theta}d\theta$.  Cauchy's formula gives:
\[
f(z)=\frac{1}{2\pi }\cdot
\int_0^{2\pi}\,
\frac{f(Re^{i\theta})\cdot Re^{i\theta}d\theta}{
Re^{i\theta}-z}\quad\, \colon\quad z\in D_R\tag{*}
\]
\medskip

\noindent
{\bf Remark.} Now (*) yields a series
representation of $f$.
First 
division with $Re^{i\theta}$ gives
\[
f(z)=
\frac{1}{2\pi }\cdot
\int_0^{2\pi}\,
\frac{f(Re^{i\theta})\cdot d\theta}{1-\frac{z}{R}\cdot  e^{-i \theta}}\tag{i}
\]
Since $|z|<R$ we can expand the denominator in a convergent
geometric series:
\[
\frac{1}{1-\frac{z}{R}\cdot  e^{-i \theta}}=
\sum_{\nu=0}^\infty\,
R^{-\nu}e^{-i\nu\theta}\cdot z^\nu\tag{ii}
\]
\medskip

\noindent
Put
\[
c_\nu=\frac{1}{2\pi }\cdot R^{-\nu}
\int_0^{2\pi}\,f(Re^{i\theta})\cdot e^{-i\nu\theta}d\theta\quad\colon\quad
\nu=0,1,\ldots\tag{iii}
\]
Then we obtain the series representation
\[ 
f(z)=\sum_{\nu=0}^\infty\, c_\nu\cdot z^\nu\tag{iv}
\]
\medskip

\noindent
{\bf Remark.} The convergence of this series when $|z|<R$ is clear.  For if
$M$ is the maximum of $|f(Re^{i\theta})|$ as $0\leq\theta\leq 2\pi$ we
see that
\[ 
|c_\nu|\leq M\cdot R^{-\nu}\tag{*}
\]
The coefficients $c_\nu$ 
correspond to \emph{Fourier series coefficients} of the periodic function 
of $\theta$ defined by
\[ 
\theta\mapsto 
f(Re^{i\theta})
\]
The  interplay between Fourier series and
analytic functions will  be discussed in   XXX.
\bigskip

\noindent
{\bf{6.6 Exercise.}}
Consider a power series $f(z)=\sum\, c_\nu \cdot z^\nu$ as above where
(*) holds for a pair $M$ and $R$. Now we also get a convergent power series
\[ 
g(z)=\sum_{\nu=1}^\infty \nu \cdot c_\nu\cdot z^{\nu-1}
\]
It turns out that $g$ is the derivative of $f$.
To show this we consider a
point $z_0$ in the disc $D_R$.
With a small $\Delta z$ we have the difference quotient:
\[ 
\frac{f(z_0+\Delta z)-f(z_0)}{\Delta z}
\]
Now the reader should verify the inequality:
\medskip

\[
\bigl|\frac{f(z_0+\Delta z)-f(z_0)}{\Delta z}-g(z_0)\bigr|
\leq |\Delta z|\cdot 
\sum_{\nu=2}^\infty\, |c_\nu|\cdot
\]



\newpage





\centerline{\bf  � 7. Complex differentials}

\bigskip

\noindent
{\bf 1. Calculus with differential forms.}
With $z=x+iy$ we get the differential 1-form
\[ dz=dx+idy
\]
Similarly, $\bar z=x-iy$ gives $d\bar z= dx-idy$. Hence we obtain

\[ dx=
\frac{1}{2}(dz+d\bar z)\quad\colon
dy=\frac{1}{2i}(dz-d\bar z)
\]
So
every differential 1-form $Adx+Bdy$ where $A,B$ say are complex-valued
$C^1$-functions can  be expressed by $dz$ and $d\bar z$:
\[ 
Adx+Bdy=
\frac{1}{2}(A-iB)dz+\frac{1}{2}(iA+B)d\bar  z
\]
\medskip

\noindent 
{\bf 2. Exterior derivatives.}
If $Adx+Bdy$ is a 1-form its exterior derivative is the 2-form
\[ 
d(Adx+Bdy)=(A_y-B_x)\cdot dx\wedge dy
\]
With these notations Stokes formula for a domain
$\Omega\in\mathcal D(C^1)$ can be expressed by
\[
\iint_\Omega\,
d(Adx+Bdy)=\int_{\partial\Omega}\, Adx+Bdy
\]
In the left hand side one integrates a function
times the 2-form $dx\wedge dy$ over $\Omega$
which is the same thing as taking
the usual area integral. Thus, the left hand side in
Stokes formula becomes
\[
\iint_\Omega\, (-A_y+B_x)dxdy
\]
Regarding $A$ and $B$ separately we recover the formula in XXX.
\medskip

\noindent
{\bf 3. Complex expressions.}
Consider now a 1-form expressed as $fdz$ where $f$ is a complex-valued
$C^1$-function. We get
\[ 
d(fdz)=df\wedge dz=\bar\partial(f) d\bar z\wedge dz
\quad\colon\quad
\bar\partial(f)=\frac{1}{2}(f_x+if_y)\tag{*}
\]
It is  both illumminating and essential to 
confirm (*) using differentials expressed by  $dx$ and $dy$. To begin with
we have
\[ 
dx\wedge dx=dy\wedge d y=0\quad\colon\quad
dy\wedge dx=-dx\wedge dx\tag{i}
\]
From this we get
\[ d\bar z\wedge dz=2i dx\wedge dy\quad\colon
\quad 
dz\wedge d\bar z=-2i dx\wedge dy\tag{ii}
\]
Hence (*) is equivalent to
\[
d(fdx+ifdy)=
(-f_y+if_x)dx\wedge dy=\frac{1}{2}(f_x+if_y)\cdot
2i dx\wedge dy\tag{**}
\]
The reader may discover that (**) holds.
Next, let us start with a 1-form $gd\bar z$. Then we get

\[ d(g d\bar z)= \partial(g)dz\wedge d\bar z\quad\colon\quad
\partial(g)=\bar\partial(f)=\frac{1}{2}(g_x-if_y)\tag{***}
\]
\bigskip


\noindent
{\bf 4. Stokes formula in complex form.}
Using (*) and (***) together with (ii) above the real version of
Stokes formula from II:XX gives the following two formulas:
\[ 
\iint
d(fdz)=
\iint
\bar\partial(f)d\bar z\wedge dz=\int_{\partial\Omega}\, fdz\tag{*}
\]
\[
\iint
d(gd\bar z)=
\iint
\partial(g)dz\wedge d\bar z=\int_{\partial\Omega}\, g d\bar z\tag{**}
\]
\bigskip

\noindent
{\bf 5. Example.}
Let $\Omega\in\mathcal D(C^1)$  and let $f(z)$ is an
analytic function in $\Omega$ which extends to a $C^1$-function on its closure.
With $\epsilon>0$ we get the $C^1$-function
\[ 
u_\epsilon=\text{Log}\,|f|^2+\epsilon
\]
We see that
\[ \partial(u_\epsilon)=
\frac{\bar f\cdot f'}{|f|^2+\epsilon}\tag{i}
\]
where $f'$ is the complex derivative of the analytic function $f$.
Stokes formula applied to the 1-form $\partial(u_\epsilon)dz$
gives:

\[ 
\iint_\Omega\,
\bar\partial(
\frac{\bar f\cdot f'}{|f|^2+\epsilon})\cdot d\bar z\wedge dz=
\int_{\partial\Omega}\,
\frac{\bar f\cdot f'}{|f|^2+\epsilon}\cdot dz\tag{ii}
\]
\medskip

\noindent
A computation left to the reader shows that the left hand side becomes
\[
2i\cdot \iint_\Omega\, \frac{\epsilon\cdot |f'|^2}{(|f|^2+\epsilon)^2}
dxdy
\]
Assume that $f$ has no zeros on
$\partial\Omega$. Then we get the limit formula
\[
\lim_{\epsilon\to 0}\,
\int_{\partial\Omega}\,
\frac{\bar f\cdot f'}{|f|^2+\epsilon}\cdot dz=
\int_{\partial\Omega}
\frac{ f'dz}{f}\tag{iii}
\]
\medskip

\noindent 
Hence the left hand side in (i) has a limit, i.e. there exists
\[
\lim_{\epsilon\to 0}
2i\cdot \iint_\Omega\, \frac{\epsilon\cdot |f'|^2}{(|f|^2+\epsilon)^2}
dxdy\tag{iv}
\]
Since
$\epsilon$ appears  in the numerator this area integral
tends to zero outside zeros of $f$. But when
$f(\alpha)=$ for some $\alpha\in\Omega$ we get a contribution.
Suppose for example that $\alpha=0$ and close to the
origin $f(z)=az^m(1+b_1z+\ldots+)$ where $a\neq 0$ and
$m\geq 1$.
Taking an area integral over a small disc $D_\delta$
centered at the origin we  see that

\[
\lim_{\epsilon\to 0}\,
\iint_{D_\delta}\,
\frac{\epsilon|f'|^2}{
(|f|^2+\epsilon)^2})\cdot dxdy=
\lim_{\epsilon\to 0}\,
\iint_{D_\delta}\,
\frac{\epsilon \cdot m^2 |a|^2 |z|^{2m-2}}{
(|a|^2|z|^{2m}+\epsilon)^2})\cdot dxdy\tag{iv}
\]
Integrating in polar coordinates the last  term becomes
\[
\lim_{\epsilon\to 0}\,
2\pi\cdot\int_0^\delta\,
\frac{\epsilon \cdot m^2 |a|^2 r^{2m-2}}{
(|a|^2 r|^{2m}+\epsilon)^2})\cdot rdr\tag{v}
\]
A easy computation which is left to the reader
shows that this limit is $\pi\cdot m$.
Performing this limit in small discs around each zero of $f$ in $\Omega$
we conclude the following:
\bigskip

\noindent
{\bf 5.1 Theorem.}
\emph{Let $\mathcal {N}_f(\Omega)$ denote the
integer equal to the sum of zeros of $f$ in
$\Omega$ counted with their multiplicities, while
$f\neq 0$ on $\partial\Omega$. Then we have the equality:}
\[ 
\mathcal{N}_\Omega(t)
= \frac{1}{2\pi i}
\int_{\partial\Omega}
\frac{ f'(z)}{f(z)}\cdot dz
\]
\bigskip





\noindent
{\bf Another example.} Let $f$ be as above and let $g$
be some $C^1$-function in $\Omega$ which vanishes
the zeros of  $f$ which we assume are all simple.
With $u_\epsilon$ defined as in the previous example
we  apply Stokes formula to the
1-form $g\partial(u_\epsilon)dz$. Computing $d(g\partial(u_\epsilon)dz$ gives
a sum of two area integrals
\[
2i\cdot \iint_\Omega\, \frac{\epsilon\cdot g\cdot |f'|^2}{(|f|^2+\epsilon)^2}
dxdy+2i\cdot \iint_\Omega\, 
\frac{\bar\partial(g)\bar f\cdot f'}{|f|^2+\epsilon}
dxdy\tag{i}
\]
The last double integral has a limit. The reason is that
\[
\lim_{\epsilon\to 0}\,
\frac{\bar f\cdot f'}{|f|^2+\epsilon}=\frac{f'}{f}\tag{1}
\]
exists in the space of integrable functions, i.e. we use that
$\frac{1}{f}$ is locally integrable when $f$ has simple zeros.
So the limit hen $\epsilon\to 0$ of the second double integral becomes
\[
2i\cdot \iint_\Omega\, 
\frac{\bar\partial(g)\cdot f'}{f}dxdy\tag{2}
\]
\medskip

\noindent
In the first integral $\epsilon$ is in the denominator so
the area integral tends to zero outside zeros of $f$.
To analyze the situation at a zero taken as the origin where
we may take $f(z)=z$ since $f$ by assumoption has simple zeros
there remains to regard
\[
\lim_{\epsilon\to 0}\, \cdot \iint_{D_\delta}\, 
\frac{\epsilon\cdot g}{(|z|^2+\epsilon)^2}
dxdy\quad\colon\quad g(0)=0\tag{3}
\]
We leave it as an exercise to show that this limit is zero
for every $g$ vanish at the origin. The hint is to integrate in polar coordinates.
Hence we have established
\medskip


\noindent
{\bf 5.2 Theorem.}
\emph{Assume that $f$ has simple zeros in
$\Omega$ and $g$
is a $C^1$-function which vanishes at these zeros.
Then}


\[
2i\cdot \iint_\Omega\, 
\frac{\bar\partial(g)\cdot f'}{f}dxdy=
\int_{\partial \Omega}\, 
\frac{g\cdot f'}{f}dz
\]
\medskip

\noindent {\bf{6.  An area formula.}}
Recall that $dz=dx+idy$. It follows that
\[ 
\bar z\cdot dz= 
i(xdy-ydx)+ xdx+ydy
\]
If $\Omega$ is a domain in $\mathcal D(C^1)$ we proved in
Chapter XX the two area formulas:

\[
\int_{\partial\Omega}\, x\cdot dy=
-\int_{\partial\Omega}\, y\cdot dx=\text{Area}(\Omega)
\]
At the same time the line integrals of $xdx$ and $ydy$ are zero.
We conclude that


\[
2i\cdot \text{Area}(\Omega)=\int_{\partial\Omega}\, \bar z\cdot dz\tag{*}
\]


\noindent
This area formula is often used.
Consider for example an analytic function $f(z)$
in $\Omega$ which extends to a $C^1$-function to the closure and
assume that the map $f\colon\bar\Omega\to \bar U$
is a homeomorphism where $U$ is another domain in
$\mathcal D(C^1)$.
So here $f$ yields a con formal map from
$\Omega$ onto $U$ and
it  sends each of the $p$ many boundary curves to
$\Omega$ onto boundary curves of $U$.
With $w=f(z)$ as a new complex variable
the area formula applied to the image domain gives:
\medskip

\[
2i\cdot \text{Area}(f(\Omega))=
\int_{\partial U}\, \bar w\cdot dw
\]
Here the last line integral is equal to

\[
\int_{\partial \Omega}\, \bar f(z)\cdot f'(z)\cdot dz
\]

\noindent
Hence we have found  an elegant formula to express the area of the image domain 
$f(\Omega)$ via
a line integral along $\partial\Omega$.



\newpage





\centerline {\bf{� 8. The equation $\bar\partial(f)=a\cdot f+b\cdot \bar f$.}}
\medskip


\noindent
Let $a(x,y)$ and $b(x,y)$ be a pair of continuous and complex-valued function
defined in some open and connected domain
$\Omega$ and 
suppose that $f(z)$ is a solution to the differential equation
in (8.2).
The Pompeiu formula  entails that if
$z_0\in\Omega$ and $R>0$ is chosen so that the disc
$D_R$ centered at $z_0$ is contained in
$\Omega$, then the following holds 
when $|z-z_0|<R$:
\[ 
f(z)=
\frac{1}{2\pi i}\cdot \int_{\partial D_R}\, 
\frac{f(\zeta)d\zeta}{\zeta-z}-
\frac{1}{\pi}\iint_{D_R}\,
[af+b\bar f]\cdot \frac{d\xi d\eta}{\zeta-z}\tag{*}
\]

\noindent
These locally defined integral equations
lead to a  uniqueness theorem.
Let us say that a continuous function $g(z)$ in $\Omega$
is flat at a point $z_0$ if
\[
\lim_{z\to z_0}\, \frac{g(z)}{[z-z_0)^n}=0\quad\text{hold for every}\quad
n\geq 1
\]

\medskip

\noindent
{\bf{7.1 Theorem.}}
\emph{Let  $f$ be a continuous solution to the differential equation
$\bar\partial(f)= af+b\bar f$  in $\Omega$. Then
$f$ cannot be flat at any point in $\Omega$ unless $f$ is identically zero.}
\bigskip

\noindent
\emph{Proof.}
Suppose that $f$ is flat at some
$z_0\in\Omega$. After a translation we may assume that
$z_0$ is the origin.
For each integer $n\geq 1$ we set:
\[ 
f_n(z)= \frac{f(z)}{z^n}
\]
Then we see that $f_n$ satisfies the equation
\[
\bar\partial(f_n)= af_n+ b\cdot \bigl(\frac{\bar z}{z}\bigr)^n\cdot f_n
\]


\noindent
The Pompeiu formula shows
that $f_n$ satisfies the integral equation
\[
f_n(z)=
\frac{1}{2\pi}\cdot \int_{\partial D_R}\,\frac{f_n(\zeta)\cdot d\zeta}{\zeta-z}+
\frac{1}{\pi}\iint_{D_R}\,
[af_n+b_nf_n]\cdot\frac{ d\xi d\eta}{\zeta-z}\tag{*}
\]
The triangle inequality gives:
\[
|f_n(z)|\leq 
\frac{1}{2\pi}\cdot \int_{\partial D_R}\,\frac{|f_n(\zeta)|\cdot |d\zeta|}{|\zeta-z|}+
\frac{1}{\pi}\iint_{D_R}\,
|af_n+b_n\bar f_n|\cdot \frac{d\xi d\eta}{|\zeta-z|}\tag{**}
\]

\noindent
Next, for each $z\in D_R$ we notice
the inequality
\[
\frac{1}{2\pi}\cdot \iint_{D_R}\frac{dxdy}{|\zeta-z|}\leq
2R\tag{1}
\]

\noindent
Using (**) and (1) an integration with respect to $z$ over
the disc $|z-z_0|<R$ gives:
\[
\iint_{D_R}\, |f(z)|dxdy\leq 
2R\cdot \int_{\partial D_R}\, 
|f_n(\zeta)|\cdot |d\zeta|+
4R\cdot 
\iint_{D_R}
|af_n+b_n\bar f_n|
\cdot d\xi d\eta\tag{2}
\]
\medskip


\noindent
Let $M$ be the maximum norm of $|a|+|b|$ over $D_R$.
Changing the integration variables $\zeta=\xi+i\eta$ in the last integral above
it follows that

\[
\iint_{D_R}\, |f_n(z)|\cdot dxdy\leq 
2R\cdot 
\int_{\partial D_R}\, 
|f_n(\zeta)|\cdot |d\zeta|+
4MR\cdot 
\iint_{D_R}\,|f_n(z)|
\cdot dx dy\tag{3}
\]
\medskip

\noindent
Above (3) holds for every $R$ such that the disc of radius $R$ centered at 
$z_0$ stays in $\Omega$.
We can choose $R$ so small that
$4MR<1$ and replacing $f_n$ by $\frac{f}{z^n}$ we get


\[
\iint_{D_R}\, \frac{|f(z)|}{|z|^n}\cdot dxdy\leq 
\frac{2R}{1-4MR}\cdot R^{-n}\cdot  \int_{\partial D_R}\, 
|f(\zeta)|\cdot |d\zeta|\tag{4}
\]
This inequality holds for every $n\geq 1$.
if $f$ is not identically zero in the disc $D_R$ we find some
$z_*\in D_R$ and a small $\delta>0$ such that
$|z_*|+\delta=\rho<R$ and $f(z)\neq 0$ in the closed
disc $\{|z-z_*|\leq\delta\}$.
If $m_*$ is the minimum of $|f(z)|$ in this disc
we get from (4):

\[
\rho^{-n}m_*\cdot \pi\delta^2\leq 
\frac{2R}{1-4MR}\cdot R^{-n}\cdot  \int_{\partial D_R}\, 
|f(\zeta)|\cdot |d\zeta|\tag{5}
\]
Since $\rho<R$ this inequality cannot hold for all $n$. Hence we have established a 
contradiction
and conclude that $f$ must be identically zero in the disc $D_R$.
Finally, since the domain is connected we can continue from new flat points
in neighborhoods where $f$ is identically zero and conclude that
$f=0$ holds in the whole domain $\Omega$.
\bigskip


\noindent
{\bf{Exercise.}}
Use similar methods as in the proof above to show that if $f$ is a solution to 
the differential equation in 8.2 which is not identically zero, then the set of zeros must be a
\emph{discrete} subset of $\Omega$, i.e. there cannot exist
$z_*\in\Omega$ and a sequence of distinct
points $\{z_\nu\}$ such that $z_\nu\to z_*$ and $f(z_\nu)=0$ for all $\nu$.






















\bigskip



\centerline{\bf {� 9. Normal families.}}

\bigskip


\noindent{\bf Introduction.}
Cauchys integral formula for a disc $D$
gives the existence of power series expansions for 
analytic functions $f(z)$. Moreover, we obtain
bounds for higher order derivatives.
For example, suppose that $f(z)\in\mathcal O(D_r)$ where 
$D_r$ is centered at the origin and has radius
$r$. Assume that $f$ extends to a continuous function on the closed disc
$\bar D_r$.
If $|z|<r$ we have seen that:
\[ 
f'(z)=\frac{1}{2\pi i}\int_{|z|=r}\,\frac{f(\zeta)d\zeta}{(\zeta-z)^2}\tag{*}
\]
Let  $|z|\leq r/2$ and with $\zeta=re^{i\theta}$ during the integration, the
triangle inequality gives

\[ |f'(z)|\leq
\frac{1}{2\pi}\cdot\int_0^{2\pi}\,
\frac{|f(re^{i\theta}|\cdot r \cdot d\theta}
{(r-|z|)^2}
\]
So if  $M=\max\,|f(re^{i\theta})|$
and $|z|\leq r/2$ we get
\[ 
\max_{|z|\leq r/2}\, |f'(z)| \leq \frac{4M}{r}
\]


\noindent
Thus, we obtain a uniform bound for the derivative via 
the maximum norm of $f$ over a larger disc.
From this one gets useful convergence principle.
\bigskip

\noindent {\bf 9.1. Theorem.}
\emph{Let $\Omega$ be an open set and
$\{f_\nu\in\mathcal O(\Omega)\}$ a sequence with uniform bounded maximum
norms, i.e. $|f_\nu|_\Omega\leq M$ hold for all $\nu$
and some constant $M$.
Then there exists at least one subsequence
$\{g_j=f_{\nu_j}\}$ such that $g_j\to g_*$ holds uniformly over compact subsets of
$\Omega$ where the  limit function $g_*$  is analytic i $\Omega$.}
\bigskip

\noindent The proof uses 
\emph{Arzela's Theorem}. Namely, by the local estimate for first
order derivatives above, the uniform bound
for maximum norms implies that
$\{f_\nu\}$ yields an \emph{equi-continuous} family of continuous functions over every compact subset of $\Omega$.  Next, one exhausts $\Omega$
by some increasing sequence of compact subsets to get the theorem above.
We leave the details  as an exercise. Bu we  shall 
give a more detailed account  in
the case when $\Omega$ is a disc where one discovers how
to obtain convergent subsequences.
In addition to Theorem 9.1 the following  result is often used: 
\medskip



\noindent {\bf {9.2. Theorem.}}
\emph{Let $\Omega$ be an open set and
$\{f_\nu\in\mathcal O(\Omega)\}$ a sequence
which converges uniformly to a limit function
$g_*$ over each compact subset of
$\Omega$. Assume also that there exists
a relatively compact subset $\Omega_0$ of
$\Omega$ and an integer $k\geq 1$,
such that every $f_\nu$ has exactly $k$ zeros 
in $\Omega_0$ - as usual counted with multiplicity, while no zeros occur outside 
$\Omega_0$. Then the limit function 
$g_*$ is either identically zero or it has exactly
$k$ zeros in $\Omega$ counted with multiplicity
and they all  belong to the closure
$\bar\Omega_0$.}


\bigskip
\noindent
{\bf Remark.} The proof uses
\emph{Rouche's
theorem} which is proved  in Chapter IV.
Once we have a convergence 
$f_\nu\to g_*$, it follows that
the sequence of complex derivatives also converges, i.e. following
implication holds:
\[ 
f_\nu\to g_*\implies f'_\nu\to g'_*\tag{*}
\]
This will be used in the proof of \emph{Riemann's Mapping Theorem} in
XX where one regards a convergent sequence $f_\nu\to g_*$
and assume that the derivatives $f'_\nu$ have no zeros. For in this situation
the discussion above 
shows that $g_*$ is either identically a constant or else
$g'_*$ never has zeros.



\medskip

\noindent{\bf {9.3  Convergence in discs}}
Let $\delta_0>0$ and denote by $D^*$ 
the open disc centered at the origin
of radius $1+\delta_0$. If $f\in\mathcal O(D^*)$ we obtain the
the series representation from 4.5:
\[ 
f(z)=\sum c_\nu z^\nu
\]
Assume that  $f$ is bounded, i.e. there is a constant $M$ such that
$|f(z)|\leq M$ for all $z\in D^*$.
For any $0<\delta<\delta_0$ we then have
\[ c_\nu=
\frac{1}{2\pi}\cdot (1+\delta)^{-\nu}\int_0^{2\pi}\, f((1+\delta)e^{i\theta})
e^{-i\nu\theta}d\theta
\]
The triangle inequality gives
\[ 
|c_\nu|\leq M\cdot (1+\delta)^{-\nu}\quad\colon\, \nu=0,1,2,\ldots
\]
These upper bounds for the coefficients of $f$ depend on
$M$ only. The inequality holds for every 
$\delta<\delta_0$ and with  $\delta=\delta_0$ we  still
have the estimates of $\{c_\nu\}$.
Suppose now that $N$ is a positive integer and
$\epsilon>0$ some positive number such that

\[ |c_\nu|\leq\epsilon\quad\colon\,0\leq\nu\leq N
\]
Then, if $0<r<1$ we have

\[ f(re^{i\theta})=
 \sum_{\nu=0}^{\nu=N}\, 
c_\nu r^\nu\cdot e^{i\nu\theta}+
 \sum_{\nu=N+1}^{\infty}\, 
c_\nu r^\nu\cdot e^{i\nu\theta}
\]
By the triangle inequality we can estimate both sums. The result is
\[
f(re^{i\theta})|\leq \epsilon\cdot 
 \sum_{\nu=0}^{\nu=N}\, 
r^\nu+
M\cdot  \sum_{\nu=N+1}^{\infty}\, 
\frac{r^\nu}{(1+\delta_0)^\nu}
\]
Taking the whole sum over the two geometric series above we get
\medskip

\noindent 
{\bf 9.4 Proposition.} \emph{One has}
\[
\max_\theta\,
|f(re^{i\theta})|\leq \epsilon\cdot\frac{1}{1-r}+
M\frac{ r^{N+1}}{(1+\delta_0)^{N+1}}
\cdot\frac{1}{1-\frac{r}{1+\delta_0}}
\]
\medskip

\noindent {\bf
9.5 Application.}
Let $\{f_k\}$ be a sequence in
$\mathcal O(D^*)$ where the maximum norm of every
$f_k$ is $\leq M$.
Each $f_k$ has a series expansion $\sum\, c_\nu(k)z^\nu$.
For each fixed $\nu$ we have a bounded  sequence
$\{c_\nu(1),c_\nu(2),\ldots\}$
of complex numbers. By the diagonal procedure we can find 
a subsequence $k_1<k_2<\ldots$ such that there exists
\[
\lim_{j\to \infty} c_\nu(k_j)=c_\nu^*\quad\colon\, \nu=0,1,\ldots
\]
The uniform estimates for the $c_\nu(k)$-coefficents give
\[ 
|c_\nu^*|\leq M\cdot(1+\delta_0)^{-\nu}
\]
Hence there exists the analytic function in $D^*$:
\[
g(z)=\sum c_\nu^*\cdot z^\nu
\]
Renumber the $f$-functions by setting $g_j=f_{k_j}$.
The convergence entails  that
for each  pair 
$\epsilon>0$ and  $N\geq 1$, there exists
some integer $N^*$ such that
\[
|c_\nu(k_j)-c_\nu^*|\leq\epsilon\quad\colon\, 0\leq\nu\leq N\quad\colon\, j\geq N^*
\]
\medskip

\noindent
Now we can apply Proposition 9.4 to estimate the maximum norm
of $|g_j-g^*|$ over discs of radius $r<1$.
In particular we obtain
\bigskip

\noindent {\bf 9.6 Proposition.}
\emph{The sequence $\{g_j=f_{k_j}\}$ converges uniformly to the
limit function
$g^*$ over each disc $|z|\leq r$ with $r<1$.}
\bigskip

\noindent
{\bf 9.7 Convergene in the whole of $D^*$.}
The rate of convergence for maximum norms over
discs $|z|\leq r<1$
in Proposition 9.6 are well controlled. By relaxing the
estimates a bit it is still true that $g_j\to g^*$ holds
uniformly in discs $|z|\leq r$ for \emph{any} $r<1+\delta_0$.
To see this one  applies the "$(\epsilon,\delta)$ - yoga".
So let $\epsilon>0$ and $r<1+\delta_0$ be given.
We first find a large $N$ so that
\[
M\frac{ r^{N+1}}{(1+\delta_0)^{N+1}}
\cdot\frac{1}{1-\frac{r}{1+\delta_0}}\leq\epsilon
\]
Then, if $|z|\leq r$,Proposition 9.4 and the
triangle inequality yield:
\[ 
|g_j(z)-g^*(z)|\leq\
\sum_{\nu=0}^{\nu=N}\,
|c_\nu(k_j)-c_\nu^*|\cdot r^\nu+2\epsilon\quad\colon\, j\geq N^*
\]
Since $\lim_{j\to\infty}\,c_\nu(k_j)\to c_\nu^*$
hold for each $0\leq\nu\leq N$  we find some large $N^{**}>N^*$
such that
\[
\sum_{\nu=0}^{\nu=N}\,
|c_\nu(k_j)-c_\nu^*|\cdot r^\nu\leq \epsilon\cdot\frac{1}{(1+\delta_0)^N}
\]
Then we see that maximum norms of $g_j-g^*$ are $\leq 3\epsilon$
if $j\geq N^{**}$. This proves that uniform convergence holds
over all discs inside $D^*$.
\bigskip

\noindent{\bf 9.8  Remark.}
The conclusion above is
quite striking since we from the start only assume that
the coefficients begin to converge and after conclude that
one gets uniform convergence to the limit function
$g^*$ over every compact disc inside $D^*$.

\newpage  

\centerline{\bf {� 10. Laurent series.}}

\bigskip

\noindent
Let $R^*>1$ and consider the open domain
\[ 
\Omega=\{z\quad\colon\, 1<|z|<R^*\}
\]
We refer to $\Omega$ as an open annulus where
$|z|=1$ is the inner circle and
$|z|=R^*$ is the outer circle.
Let $f(z)$ be analytic in
$\Omega$. When $1<|z|<R$ is fixed we can choose
a pair $r,R$ such that
\[
1<r<|z|<R<R^*
\]
Cauchy's formula applies to the domain
$r<|z|<R$ and gives the equality:

\[ 
f(z)=\frac{1}{2\pi i}\cdot
\int_{|\zeta|=R}\,
\frac{f(\zeta)\cdot d\zeta}{\zeta-z}
-\frac{1}{2\pi i}\cdot
\int_{|\zeta|=r}\,
\frac{f(\zeta)\cdot d\zeta}{\zeta-z}\tag{*}
\]
\medskip

\noindent
The idea is now to expand $\zeta-z$ in a geometric series.
When $|\zeta|=R>|z|$ we have
\[
\frac{1}{\zeta-z}=\frac{1}{\zeta}\cdot\sum_{\nu=0}^\infty
\frac{z^\nu}{\zeta^\nu}
\]
Let us put
\[ 
a_\nu=
\frac{1}{2\pi i}\cdot \int_{|\zeta|=R}\,
\frac{f(\zeta)\cdot d\zeta}{\zeta^{\nu+1}}\quad\colon\quad \nu=0,1,2,\ldots
\]

\noindent
Then the first integral in (*) becomes
\[ 
f^*(z)= 
\sum\, a_\nu\cdot z^\nu\tag{1}
\]


\noindent
Next, when $|\zeta|= r<|z|$ we have the series expansion
\[
\frac{1}{\zeta-z}=-\frac{1}{z}\cdot\sum\,\frac{\zeta^\nu}{z^\nu}
\]
Put
\[
b_\nu=\frac{1}{2\pi i}\cdot \int_{|\zeta|=R}\,
f(\zeta)\cdot \zeta^{\nu-1}\cdot
d\quad\colon\quad \nu=1,2,\ldots
\]
Taking the negative sign into the account in (*) we see that
the second integral becomes
\[ 
f_*(z)=\sum_{\nu=1}^\infty\,\frac{b_\nu}{z^\nu}\tag{2}
\]


\noindent
{\bf {1. Definition.}}
\emph{The function $f^*(z)$ is called the positive part of $f$ and
$f_*(z)$ the lower part of $f$.}
\medskip

\noindent
{\bf Remark.}
In the construction above 
$R$ can be chosen arbitrarily close to $R^*$.
With $R=R^*-\epsilon$ we get the finite maximum norm
\[ 
||f||_{R_*-\epsilon}=\max_{\zeta|= R_*-\epsilon}\, |f(\zeta)|
\]
The triangle inequality gives
\[ 
|a_\nu|\leq\frac
{||f||_{R^*-\epsilon}}{(R^*-\epsilon)^\nu}
\]
Since $\epsilon$ can be made arbitrary small we conclude that
the radiius of convergence for the series (1) is $\geq 1$. Hence we have
proved:
\medskip

\noindent
{\bf 2. Proposition.}
\emph{The positive part $f^*(z)$ is analytic in the disc
$|z|<R^*$.}
\medskip

\noindent
In a similar way the reader may verify the following:
\medskip

\noindent
{\bf {3. Proposition.}}
\emph{The inner part $f_*(z)$ is analytic in the exterior disc
$|z|>1$.}
\bigskip

\noindent
{\bf 4. The Laurent series.} The analytic function $f(z)$ in the annulus can be written
as a sum
$f^*(z)+f_*(z)$. Here
$f^*\in\mathcal O(D_{R^*})$ while  $f_*$ is  analytic in the exterior disc
$|z|>1$. We can take the two series together and hence $f(z)$ is represented by

\[ 
f(z)=
\sum_{\nu=0}^\infty\,a_\nu\cdot z^\nu
+\sum_{\nu=1}^\infty\,\frac{b_\nu}{z^\nu}
\]
This is called the Laurent series of $f(z)$.
\medskip

\noindent
{\bf 5. The residue coefficient $b_1$.}
From the construction above we have
\[ 
b_1=\frac{1}{2\pi i}\int_{|\zeta|=R}\, f(\zeta)\cdot d\zeta
\]
Notice that we can perform the integral over any
circle, i.e. above we can take any
$1<R<R*$.
The special role of $b_1$ is that $b_1=0$  holds if and only if
$f(z)$ has a primitive analytic function in the annulus .
\medskip

\noindent
{\bf 6. Exercise.}
Prove the assertion above, i.e. that
there exists an analytic function
$F(z)$ in the annulus such that
$F'(z)=f(z)$ if and only if $b_1=0$.






\medskip

\noindent
{\bf 7. Example.}
Consider the analytic function in the annulus
$1<|z|<R$ defined by
\[ 
\phi(z)=\frac{1}{z+1}-\frac{1}{z-1}
\]
Here $\phi(z)$ has the Laurent series expansion
\[ 
\phi(z)=-2\cdot(\frac{1}{z^2}+\frac{1}{z^4}+\ldots)
\]
Hence there exists the primitive function
\medskip

\noindent
\[
\Phi(z)= 2\cdot\bigl(\frac{1}{z}+\frac{1}{3z^3} +\frac{1}{5z^5}+\ldots\bigr)
\]
Next, we notice that
$\Phi(z)$ can be expressed by  the complex Log-function


\[ 
\log\,\bigl(\frac{z+1}{z-1}\bigr)\tag{*}
\]
Here some care must be taken. First it is clear that
(*) is  defined when
$z=x$ is real with
$x>1$.
 So on the real interval $(1,+\infty)$ we have the equality
 \[ 
G(x)=\log\,\bigl(\frac{x+1}{x-1}\bigr)
\]


\noindent
{\bf 8. Exercise.}
Let $R>1$ and consider an analytic function
\[
g(z)=\sum\, c_\nu\cdot z^\nu
\]
in the open  disc
$D_R$.
It can be restricted to the real line
$[-1,1]$. Set
\[ 
K_g(z)=\int_{-1}^1\, \frac{g(z)-g(s)}{z-s}\cdot ds
\]
The algebraic identity $z^\nu-s^\nu=(z-s)(z^{\nu-1}+\ldots+s^{\nu-1})$
gives
\[
K_g(z)=2c_1+
\sum_{\nu=2}^\infty\, c_\nu\cdot \int_{-1}^1\,\bigl[ z^{\nu-1}+
z^{\nu-2}s+\ldots+s^{\nu-1}\bigr]\cdot ds
\]
Since we integrate over $[-1,1]$ the integrals taken over
odd $s$-powers are all zero and a computation gives:
\[
K_g(z)= 2c_1+
2\cdot \sum_{\nu=2}^\infty\, c_\nu\cdot \bigl[
z^{\nu-1}+\frac{z^{\nu-3}}{3}+
\frac{z^{\nu-5}}{5}+\ldots\bigr]\tag{1}
\]


\noindent
Notice
that  the sum in each bracket is finite.
Show that $K_g(z)$ is analytic in the disc $|z|>R$ and using
the $\Phi$-function from Example 7 the reader should verify that
$K_g(z)$ is equal to the positive part of the Laurent series defined by
$g(z)\cdot \Phi(z)$ in $1<|z|<R$.

\newpage








\centerline {\bf � 11. An area formula.}
\medskip

\noindent
Consider a Laurent series
\[ 
f(z)=\sum_{-\infty}^\infty\, c_\nu\cdot z^\nu
\]
which represents the analytic function
defined in $\{R_*<|z|<R^*\}$.
If $R_*<r<R^*$
we get the parametrised  curve

\[
\theta\mapsto \theta\mapsto f(re^{i\theta})\tag{*}
\]

\noindent
Consider the situation where (*) is bijective so that
the image is a closed
Jordan curve $J_r$ in the complex $\zeta$-plane
which borders a bounded Jordan domain $\Omega_r$.
We shall express the area of $\Omega_r$
with the the coefficients in Laurent series of $f$.
The formula depends upon the orientation in the map (*) which may be
either positive or negative.



\bigskip

\noindent
{\bf 1. Theorem.} \emph{One has has the equality}
\[ 
\text{area}(\Omega_r)=\text{sign}(*)\cdot \pi\cdot
\sum_{-\infty}^\infty\, n\cdot |c_n|^2\cdot r^{2n}
\]
\emph{where $\text{sign}(*)$ is +1 or -1 depending upon the 
orientation in (*).}


\medskip

\noindent
{\bf{2. Example.}}
If $f(z)=z$ the orientation is positive and here
$\Omega_r$ is the disc of radius $r$ and the formula is okay since
its area is $\pi\cdot r^2$. On the other hand, if
$f(z)=\frac{1}{z}$ the orientation is negative and  the sign-rule
in Theorem 1 gives a correct formula.
\bigskip


\noindent
emph{Proof of Theorem 1.}
Set $w=f(z)$ and suppose  that
(*) is positively oriented. The
area formula from XX gives:

\[
\text{area}(\Omega_r)=\int_{J_r}\,
\bar w\cdot dw=\int_{|z|=r}\, \bar f(z)\cdot f'(z)\cdot dz
\]
\medskip

\noindent 
The right hand side
is a double sum
\[
\sum\sum\, \bar c_k\cdot m\cdot c_m\cdot \int_{|z|=r}\, \bar z^k\cdot z^{m-1}\cdot dz
\]
extended over all pairs of integers.
Since $\bar z^k=\frac{r^{2k}}{z^k}$
holds on $|z|=r$ we have
\medskip

\[
\int_{|z|=r}\, \bar z^k\cdot z^{m-1}\cdot dz=
r^{2k}\cdot \int_{|z|=r}\, z^{-k}\cdot z^{m-1}\cdot dz
\]
By Cauchy's residue formula the last integrals are zero for when $k\neq m$
and become $r^{2k}\cdot 2\pi i$ if $k=m$. Now we can read off the formula
in
Theorem 1. The proof  when (*) has a negative orientation 
is the same after signs have been  reversed.
\bigskip

\noindent
{\bf{3. A special case}}. Let
\[ 
f(z)=\frac{1}{z}+\sum_{n=0}^\infty\, c_n\cdot z^n
\]
where the positive series represents an analytic function
in some disc $|z|<R$.Suppose that (*) holds for some
$0<r<R$ where the bijective map from (*) has a negative
orientation which gives:
\[ 
\text{area}(\Omega_r)=\pi\cdot \bigl[\frac{1}{r}-\sum_{n=1}^\infty\, 
n\cdot |c_n|^2\cdot r^{2n}\bigr]
\]
Since the area is a positive number we get 
the inequality 
\[
\sum_{n=1}^\infty\,
n\cdot |c_n|^2\cdot r^{2n}\leq \frac{1}{r}\tag{1}
\]
\medskip

\noindent
{\bf{4. Koebe's inequality.}}
Let $\phi$
be an analytic function  in 
the unit disc where $\phi(0)=0$ and $\phi'(0)=1$ 
and $\phi$ is a conformal map from 
$D$ onto some simply connected
domain in ${\bf{C}}$. Put
\[
f(z)=\frac{1}{\phi(z)}
\]
Now $f$ is analytic in the punctured disc $0<|z|<1$
and since $\phi$ is a conformal map
it follows that $f$ maps circles $|z|=r$ onto closed Jordan curves
where the orientation now is negative for each $0<r<1$.
Moreover $f$ has a  Laurent series
\[
\frac {1}{z}+\sum_{n=0}^\infty\, c_n\cdot z^n
\]
and since  the inequality (1) holds for every
$r<1$ we have
\[ 
\sum_{n=1}^\infty\,
n\cdot |c_n|^2\cdot r^{2n}\leq 1\tag{i}
\]
At the same time the given $\phi$-function has a
series expansion
\[ 
\phi(z)= z+d_2z^2+d_3z^3+\ldots
\]
Let us assume that $\phi$ is an \emph{odd} function
which to begin with gives $d_2=0$ and 
\[
\frac{1}{\phi}= \frac{1}{z}\cdot \frac{1}{1+d_3z^2+d_4z^3+\ldots}
\]
From this we conclude that
\[ 
c_1=-d_3
\]
and the inequality (i) gives
\[ 
|d_3|\leq 1\tag{ii}
\]
\medskip

\noindent
{\bf{Exercise.}}
Use (i) above to show
that equality  holds in (ii) if and only if
\[ 
\phi(z)=\frac{z}{1-\lambda z^2}
\] 
for some $\lambda$ whose absolute value is one.


\medskip

\noindent
Next,
consider a conformal map $\psi$ from $D$ to a simply connected
domain whose
Taylor series at $z=0$ is of the form
\[ 
\psi(z)= z+a_2z^2+\ldots
\]
\medskip

\noindent
{\bf{5. Theorem.}}
\emph{One has the inequality}
$ |a_2|\leq  2$.


\medskip

\noindent
\emph{Proof.} 
To begin with
$\psi(z^2)$ is analytic in $D$ and starts with $z^2$ which gives
the analytic function
\[
\frac{\psi(z^2)}{z^2}=1+ a_2\cdot z^2+\sum_{m\geq 2}\, b_m\cdot z^{2m}\tag{i}
\]
Since $\psi(z)\neq 0$
when $z\neq 0$
the analytic function in (i) is also
$\neq 0$ in $D$. Now
the unit disc is simply connected and hence
there exists an analytic  square root function:
\[
\phi(z)=z\cdot\sqrt{\frac{\psi(z^2)}{z^2}}\tag{ii}
\]
\medskip

\noindent
\emph{Sublemma} The function
$\phi$ is  odd function and yields a conformal mapping, i.e. 
$\phi$ is 1-1 in $D$.
\medskip


\noindent
The proof of this Sublemma is left as an exercise.
In particular we have a  series expansion:
\[ 
\phi(z)= z+d_3\cdot z^3+\ldots\tag{iii}
\]
Notice that the square root in (i) has a series expansion
\[
1+\frac{a_2}{2}\cdot z^2+\text{higher order terms}\tag{iv}
\] 
It follows that 
\[
d_3=\frac{a_2}{2}\tag{v}
\]
Then (ii) from  (4)
above gives $|a_2\leq 2$ as required.
\medskip

\noindent
{\bf{6. Exercise.}} Show
that the equality 
$|a_2|=2$ holds in Theorem 5 if and only if
$\psi$ is of the form

\[ 
\psi(z)=\frac{z}{(1-\lambda z)^2}\quad\text{for some}\quad |\lambda|=1
\]
Show also that every such function indeed yields a conformal map from
$D$ onto a simply connected domain which 
becomes a  \emph{radial slit domain}.
For example, take $\lambda=1$ and show that
the $\psi$-image of $D$  becomes
\[ 
{\bf{C})}\setminus\bigl[-1/4,+\infty\bigl)
\]


\bigskip


\centerline {\bf � 12. A theorem by Jentzsch}
\bigskip

\noindent
{\bf Introduction.} Let $\sum\, c_nz^n$ b a convergent series
in the unit disc $D$ whose
radius of convergence is one.
To each $n\geq 1$ we get the Taylor polynomials
\[ 
s_n(z)=c_0+c_1z+\ldots+c_nz^n
\]
Denote by
$\mathcal N(s_n)$ the zero set  of  $s_n$.
\medskip

\noindent
{\bf 12.1 Theorem.}
\emph{For each  point $e^{i\theta}$ on  the unit circle there
exists a strictly increasing sequence
$1\leq n_1<n_2<\ldots$ and points $z_k\in\mathcal N(s_{n_k})$
such that $\lim_{k\to\infty}\, z_k= e^{i\theta}$.}
\bigskip

\noindent
{\bf Remark.} This result was proved by   X. Jentzsch in the article [Jen: 
Acta mathematica 1918].
To illustrate the theorem we consider the analytic function
$f=\frac{1}{1-z}$.
It has no zeros in $D$ but the zeros of
\[
s_n(z)= 1+\ldots+z^n
\]
are roots of unity which cluster on the whole unit circle as
$n\to\infty$.
The fact that a similar clustering of zeros occur
for partial sums of a function of the form
\[ 
f(z)=\sum\, z^{\nu_k}
\] 
where $1\leq \nu_1<\nu_2<\ldots$ is an arbitrary increasing sequence of
integers is less evident but has an affirmative answer by
the theorem above.
\bigskip

\noindent
\centerline {\emph{Proof of Theorem  12.1}}

\medskip

\noindent
We argue  by contradiction. Suppose that some point on
$T$ is not a cluster point for zeros of the
$s$-functions. After a rotation we may assume that this point is
$z=1$. So now  there exists  $0<\delta<1$ 
and a positive integer $k_*$ such that
\[ 
s_k(z)\neq 0\quad\colon\, |z-1|<\delta
\quad\colon k\geq k^*\tag{1}
\]
There remains to derive a contradiction.
Since $s_k\to f$ holds uniformly inside the unit disc
and $f$ is not reduced to a constant, it follows from
XX and (1)  that
$f(z)\neq 0$ in the domain
\[
\Omega=D\cap\{ |z-1|<\delta\}
\]
Since $\Omega$ is simply connected there exists
$h\in\mathcal O(\Omega)$
where
\[ 
f(z)= e^{h(z)}
\]
We also find a sequence 
$\{g_k\in\mathcal O(D_\delta(1))\}$
such that
\[ 
s_k(z)= e^{g_k(z)}\quad\colon\quad z\in D_\delta(1)
\]
Cnosider the fixed point  $\xi=1-\delta/8$.
Now $s_k(\xi)\to f(\xi)$ and we can theefore choose  
branches of $\{g_k\}$ so that
\[ 
\lim_{k\to\infty}\ g_k(\xi)= h(\xi)\tag{2}
\]


\noindent
Since $s_k(z)\to f(z)$ holds uniformly on
compact subsets of  $\Omega$
it follows that $g_k(z)\to h(z)$  holds with uniform
convergence  over compact subsets of
$\Omega$.
Next, consider  the functions
\[
\phi_k(z)=e^{\frac{g_k(z)}{k}}-1\tag{3}
\]


\noindent
The convergence $g_k\to h$ in $\Omega$
entails that 
\[
\phi_k(z)\to 0\tag{4}
\]
where this convergence is uniform over compact subsets of
$\Omega$. Next we have

\medskip




\noindent {\bf Lemma.} \emph{The $\phi$-functions are uniformly bounded in
$D_\delta(1)$.}
\medskip

\noindent
\emph{Proof.}
First it is clear that there 
exists a constant $A$  such that
\[ 
|c_n|\leq A\cdot 2^n
\]
With $\delta<1$ and $|z|\leq 1+\delta$ 
the
triangle inequality gives
\[ 
|s_k(z)|\leq  A\cdot 2^{k+1}\cdot (1+\delta)^k
\quad\colon\quad k=1,2\ldots\tag{i}
\]
Since
$e^{\mathfrak{Re}(g_k)(z)}=|s_k(z)|$
we see that (i) gives:
\[
\mathfrak{Re}(g_k)(z)\leq (k+1)\cdot\log 2+
k(1+\delta)+\log^+ A\tag{ii}
\]
Dividing by $k$
we see that
$\{\frac{\mathfrak{Re}(g_k)(z)}{k}\}$ are uniformly bounded
above when
$z\in D_\delta(1)$ and Lemma 1 follows since
\[ 
|\phi_k(z)|\leq e^{\frac{\mathfrak{Re}\, g_k(z)}{k}}+1
\]
\bigskip

\noindent
\emph{Final part of the proof}: 
The uniform bound in the Lemma and the convergence from (4) imply that
$\{\phi_k(z)\}$ converges uniformly to zero in
compact subsets of $D_\delta(1)$. In particular we can find some
$0<\epsilon<\delta$ such that 
\[ 
|\phi_k(1+\epsilon)|<\epsilon/2
\]
hold for all large $k$. It follows that
\[
 |s_k(1+\epsilon)|=|1+\phi_k(1+\epsilon)|^k\leq(1+\epsilon/2)^k\tag{*}
\]
holds when $k$ large and then 
\[ 
|c_k(1+\epsilon)^k|=|s_k(1+\epsilon)-s_{k-1}(1+\epsilon)|
\leq |s_k(1+\epsilon)|+|s_{k-1}(1+\epsilon)|\leq
2\cdot (1+\epsilon/2)^k\implies
\]
\[ 
\limsup\uuu{k\to\infty}\,
|c_k|^{\frac{1}{k}}\leq \lim 2^{\frac{1}{k}} \cdot \frac{1+\epsilon/2}{1+\epsilon}
=
\frac{1+\epsilon/2}{1+\epsilon}
\]
The last term is $<1$ which
contradicts
that the  radius of convergence of $\sum\, c_kz^k$ is one and hence 
Theorem 12.1 is proved.



\newpage



\centerline{\bf � 13. An inequality by Siegel}
\bigskip

\noindent 
The  result below was proved by C. Siegel in
Math. Zeitschrift Bd. 10 page 175 (1921)
Let $n\geq 2$ and $p(z)$ is a monic polynomial
\[
p(z)=z^n+a_1z^{n-1}+\ldots+a_n
\]
Let $\alpha_1,\ldots\alpha_n$ be the roots where
eventual multiple roots are repeated.
\medskip

\noindent {\bf Theorem.}
\[ 
\prod_{\nu=1}^{\nu=n}\, (1+|\alpha_\nu|)
\leq 2^n\cdot \sqrt{1+|a_1|^2+\ldots+|a_n|^2}
\]




\noindent \emph{Proof.}
For every complex number the formula from XXX gives

\[ 
1+|z|\leq 2\cdot\text{max}(1,|z|)=
2\cdot 
\text{exp}\bigl[\,\frac{1}{2\pi}\cdot \,\int_0^{2\pi}\,\text{Log}\bigl|e^{i\theta}-z|d\theta\,\bigr]\tag{*}
\]
 

\noindent
Apply this to each root of $p$ and take the product. This gives
\[
\prod_{\nu=1}^{\nu=n}\, (1+|\alpha_\nu|)
\leq 2^n\cdot
\text{exp}\bigl [\,\frac{1}{2\pi}\cdot \,\int_0^{2\pi}\,
\,\sum_{\nu=1}^{\nu=n}\,
\,\text{Log}\bigl|e^{i\theta}-\alpha_\nu\, \bigr|\cdot d\theta\,\bigr]=
\]
\[
2^n\cdot
\text{exp}\bigl[\,\frac{1}{2\pi}\cdot \,
\int_0^{2\pi}\,
\text{Log}|p(e^{i\theta})|\,d\theta\,\bigr]
\]


\noindent To estimate the last term we employ 
Blascke's factorization of $p(z)$ in the unit disc $|z|<1$:
\[
p(e^{i\theta})=B(e^{i\theta})\cdot e^{\phi(e^{i\theta})}\quad\colon
|B(e^{i\theta}|=1\quad\colon\,
\phi(z)\in \mathcal O(D)\implies
\]

\[
\frac{1}{2\pi}\cdot\int_0^{2\pi}\,
\text{Log}|p(e^{i\theta})|\,d\theta=
\frac{1}{2\pi}\cdot\int_0^{2\pi}\,\mathfrak{Re}(\phi(e^{i\theta}))d\theta=
\mathfrak{Re}(\phi)(0)
\]
\medskip

\noindent
where the last equality holds since $\mathfrak{Re}(\phi)$ is harmonic.
Hence there only remains to show the inequality
\[
e^{\mathfrak{Re}(\phi)(0)}\leq
\sqrt{1+|a_1|^2+\ldots+|a_n|^2}\tag{i}
\]


\noindent
To show this we use the Plancherel formula which gives
\[
\int_0^{2\pi}\,|p(e^{i\theta})|^2d\theta=2\pi\cdot(1+|a_1|^2+\ldots+|a_n|^2)\tag{ii}
\]
In addition, we have
\[
|p(e^{i\theta})|^2=
e^{2\cdot \mathfrak{Re}(\phi(e^{i\theta}))}\tag{iii}
\]
Now we can finish the proof as follows:
Since $e^{\phi(z)}$ is analytic  the mean value formula gives:
\[
e^{\phi(0)}=\frac{1}{2\pi}\cdot
\int_0^{2\pi}\,
e^{\phi(e^{i\theta})}\cdot d\theta\tag{iv}
\]
Since the absolute value $|e^{\phi(z)}|= e^{\mathfrak{Re}(\phi)(z)}$
hold for all $z$, the triangle inequality gives
\[
e^{\mathfrak{Re}(\phi)(0)}\leq\frac{1}{2\pi}\cdot
\int_0^{2\pi}\,
e^{\mathfrak{Re}(\phi(e^{i\theta}))}\cdot d\theta\tag{v}
\]
The Cauchy-Schwarz inequality applied to (v) gives:
\[
e^{\mathfrak{Re}(\phi)(0)}\leq\frac{1}{\sqrt{2\pi}}
\cdot
[\int_0^{2\pi}\,
e^{2\mathfrak{Re}(\phi(e^{i\theta}))}\cdot d\theta\,[^{\frac{1}{2}}\tag{vi}
\]
Now (ii) and (iii) give the requested inequality (i).

\newpage



\centerline {\bf{� 14. Zeros of product series.}}
\bigskip

\noindent
Let $\{\lambda\uuu k\}$ be a non\vvv decreasing sequence of
positive real number
where  $\lambda\uuu 1\geq 2$ and
\[
 \sum\,\lambda\uuu k^{\vvv (1+\epsilon)}<\infty 
\quad \text{hold for every}\quad \epsilon>0
\tag{1}
\]
This gives  the analytic function
$\phi(z)$ defined in $\mathfrak{Re}\, z>1$
by
\[ 
\phi(z)=\prod\, (1\vvv \lambda\uuu k^{\vvv z})\tag{*}
\]
The logarithmic  derivative becomes
\[
\frac{\phi'(z)}{\phi(z)}=
\vvv \sum\, \log \lambda\uuu k\cdot
\frac{\lambda\uuu k^{\vvv z}}{1\vvv\lambda\uuu k^{\vvv z}}
\]
Expanding
the denominator $1\vvv\lambda\uuu k^{\vvv z}$ into a geometric series
we obtain:

\[
\frac{\phi'(z)}{\phi(z)}=
\vvv \sum \uuu{n=1}^\infty
\sum\uuu{k=1}^\infty\, \log \lambda\uuu k\cdot\lambda\uuu k^{\vvv nz}
\]
\medskip

\noindent
If $z=1+\epsilon+iy$
for some $\epsilon>0$ and a real $y$ we set
\[
\rho\uuu\epsilon(n;y)=
\sum\uuu{k=1}^\infty\, \log \lambda\uuu k\cdot\lambda\uuu k^{\vvv n(1+\epsilon)}
\cdot \lambda\uuu k^{\vvv iny}
\]
\medskip

\noindent
{\bf{14.1 A remarkable positive sum.}}
Let $y\uuu 0\neq 0$
and  consider the sum
\[
S\uuu\epsilon(n)=
3\cdot \rho\uuu\epsilon(n;0)+4\cdot \rho\uuu\epsilon(n;y\uuu 0)+
\rho\uuu\epsilon(n;2y\uuu 0)
\]
Taking  real parts we obtain
\[
\mathfrak{Re}\, S\uuu\epsilon(n)=
\sum\uuu{k=1}^\infty\,
\sum\uuu{n=1}^\infty
 \log \lambda\uuu k\cdot\lambda\uuu k^{\vvv n(1+\epsilon)}
\cdot\,\bigl[
3+4\cdot \mathfrak{Re}\, \lambda\uuu k^{\vvv iny\uuu 0}+
\mathfrak{Re}\, \lambda\uuu k^{\vvv iny\uuu 0}\,\bigr]\tag{i}
\]
In this double sum the terms inside the brackets become
\[
3+
4\cdot \cos\, n\cdot \lambda\uuu k\cdot y\uuu 0+
\cos\, 2n\cdot \lambda\uuu k\cdot y\uuu 0
= 2\bigl( 1+\cos\, n\cdot \lambda\uuu k\cdot y\uuu 0)^2
\] 
where the last equality follows from the trigonometric formula
$\cos 2a= 2\cos^2\vvv 1$.
Hence we have the formula
\[
\mathfrak{Re}\bigl[\, 
3\cdot \frac{\phi'(1+\epsilon)}{\phi(1+\epsilon)}+
4\cdot \frac{\phi'(1+\epsilon+iy\uuu 0)}{\phi(1+\epsilon+iy\uuu 0)}+
\frac{\phi'(1+\epsilon+2i y\uuu 0)}{\phi(1+\epsilon+2iy\uuu 0)}\,\bigr]=
\]
\[
\vvv \sum\uuu{n=1}^\infty \sum\uuu {k=1}^\infty
\log \lambda\uuu k\cdot\lambda\uuu k^{\vvv n(1+\epsilon)}
\cdot  2\bigl( 1+\cos\, n\cdot \lambda\uuu k+\cdot y\uuu 0)^2\tag{*}
\]
\medskip

\noindent
{\bf{14.2 Conclusion.}}
Keeping $y\uuu 0$ fixed we get the following for each $\epsilon>0$:
\[
\mathfrak{Re}\bigl[\, 
3\cdot \frac{\phi'(1+\epsilon)}{\phi(1+\epsilon)}+
4\cdot \frac{\phi'(1+\epsilon+iy\uuu 0)}{\phi(1+\epsilon+iy\uuu 0)}+
\frac{\phi'(1+\epsilon+2i y\uuu 0)}{\phi(1+\epsilon+2iy\uuu 0)}\bigr]<0
\] 

\bigskip

\noindent
{\bf{14.3 Absence of asymptotic zeros.}}
Let $y\uuu 0>0$  
and assume that one has three limit expansions:
\[
\phi(1+\epsilon)=\epsilon^\alpha\cdot \gamma\uuu 0(\epsilon)
\]
\[
\phi(1+\epsilon+iy\uuu 0))=\epsilon^{\beta}
\cdot \gamma\uuu 1(\epsilon)
\]
\[
\phi(1+\epsilon+2iy\uuu 0))=\epsilon^\kappa\cdot \gamma\uuu 2(\epsilon)
\]
where $\alpha,\beta,\kappa$ are real constants
and each $\gamma$\vvv function is 
$\neq 0$ for small positive $\epsilon$ and together with their  derivatives  satisfy
\[
\lim\uuu{\epsilon\to 0}\, \epsilon\cdot 
\frac{\gamma\uuu j'(\epsilon)}
{\gamma\uuu j(\epsilon)}=0\quad\colon\quad 0\leq j\leq 2
\]

\medskip

\noindent
Under the conditions above we see
that the function
\[ 
\epsilon\mapsto
\mathfrak{Re}\bigl[\, 
3\cdot \frac{\phi'(1+\epsilon)}{\phi(1+\epsilon)}+
4\cdot \frac{\phi'(1+\epsilon+iy\uuu 0)}{\phi(1+\epsilon+iy\uuu 0)}+
\frac{\phi'(1+\epsilon+2i y\uuu 0)}{\phi(1+\epsilon+2iy\uuu 0)}\,\bigr]
\]
has an asymptotic expansion of the form
\[
\frac{3\cdot \mathfrak{Re}\,\alpha+
4\cdot \mathfrak{Re}\,\beta+\mathfrak{Re}\,\kappa}{\epsilon}
+\gamma^*(\epsilon)
\]
where $\epsilon\cdot \gamma^*(\epsilon)\to 0$.
\bigskip

\noindent
At the same time we have the inequality from 14.2
and
checking signs we have the following result:

\medskip

\noindent
{\bf{14.4 Theorem.}}
\emph{If $\phi(z)$ has the three  asymptotic expansions above
for some $y\uuu 0\neq 0$ then the following inequality must hold:}
\[
3\cdot \mathfrak{Re}\,\alpha+4\cdot \mathfrak{Re}\,\beta +
\mathfrak{Re}\,\kappa<0
\]
\medskip

\noindent
{\bf{14.5 The case when
$\phi$ has a meromorphic extension.}}
When
$\phi(z)$ has a meromorphic extension
across $\mathfrak{Re}(z)=1$ the
asymptotic expansions in (14.3) exist for every $y_0>0$
and 
$\alpha,\beta,\kappa$ are integers.
Suppose also that $\phi$ has a simple pole at
$z=1$ which gives
$\alpha=- 1$. If
$\phi(1+iy_0)=$ for some
$y_0>0$ the corresponding
$\beta$-integer is $\geq 1$
and
Theorem 14.4  implies that $\kappa\leq -1$ which means that
$\phi$ must have a pole
at $1+2iy\uuu 0$. In particular we 
have the following:

\medskip

\noindent
{\bf{14.6 Theorem}}\emph{ If
$\phi$ has a simple pole at $z=1$ and
extends to a meromorphic function to an  open domain
$\Omega$ which contains the closed  half-space
$\mathfrak{Re}\, z\geq 1$ with no further poles on
$\mathfrak{Re}\, z=1$ then}
\[ 
\phi(1+iy)\neq 0\quad\text{for all}\quad y> 0
\]


\medskip

\noindent
{\bf{Remark.}}
The idea   to employ positive cosine\vvv functions
to investigate zeros and poles of Dirichlet series was introduced by de Valle Poussin
in his proof of the Prime Number Theorem.
Above we   used the positivity
of the function $a\mapsto 3+4\cos \,a+\cos 2a$. Other
cosine\vvv sums which  produce functions which are everywhere
$\geq 0$ can be used to
obt<ain
more involved relations about  poles and zeros of Dirichlet series on
the  critical line of convergence.

\newpage


\centerline{\bf{� 15. Hadamard products.}}
\medskip

\noindent
We expose material from the article \emph{Les series enti�res} 
(Acta Math. 1899) by Hadamard and start with some geometric considerations.
Let $U$ and $V$ be two Jordan domains
which both contain the origin. Put $\sigma=\partial U$ and
$\gamma=\partial V$.
The inversion  map $z\mapsto z^{\vvv 1}$
maps every closed Jordan curve which does not pass the origin to a similar closed Jordan curve. In particular, each $z\neq 0$
yields a closed Jordan curve
$\gamma(z)$ whose points are $\frac{z}{w},\colon\, w\in \gamma\}$.
Following Hadamard we give

\medskip


\noindent
{\bf{15.1 Definition.}}
\emph{Denote by $\mathcal H(U,V)$ the set of 
$z$ such that the closed Jordan curve $\gamma(z)$ is contained in $U$.}
\medskip

\noindent
It is clear that $\mathcal H(U,V)$ is an open set which contains
a neighborhood of the origin since this by hypothesis is  a common interior point of
$U$ and $V$.
\medskip


\noindent
{\bf{15.2 Exercise.}}
Show that the open set
$\mathcal H(U,V)$ is connected and  the equality
\[
\mathcal H(U,V)=\mathcal H(V,U)
\] where the right hand side is constructed when we instead
consider
the closed Jordan curves
$\sigma(z)$ and require that they are contained in $V$.
\medskip

\noindent 
Next, let $f\in\mathcal O(U)$ and $g\in \mathcal O(V)$ 
which
extend continuously up to the boundaries
of $U$ and $V$ respectively.
If $z\in \mathcal H(U,V)$
we can take a complex line integral along $\gamma$ and define
\[ 
\psi(z)=\frac{1}{2\pi i}\int\uuu\gamma\,
f(\frac{z}{\zeta})\cdot g(\zeta)\, d\zeta
\]
\medskip

\noindent
{\bf{15.3 Exercise.}} Show that $\psi(z)$ is analytic in $\mathcal H(U,V)$.
\medskip

\noindent
We refer to $\psi$ as the Hadamard convolution of the pair $f,g$ and write
$\psi=f*g$. The equality in Exercise 15.2 gives
$f*g=g*f$ which me ns that we also have
\[ 
\psi(z)=\frac{1}{2\pi i}\int\uuu\sigma\,
g(\frac{z}{\zeta})\cdot f(\zeta)\, d\zeta
\] 
\medskip

\noindent
{\bf{Example.}}
Let $U$ and $V$ be discs of radius $R$ and $r$ centered at the origin
where $R\geq r$.
In this case $\mathcal H(U,V)$ is the disc of radius $rR$
and the reader can verify that
the Taylor series of $\psi$ becomes
\[
\psi(z)=\sum\, a\uuu n\cdot b\uuu n\cdot z^n
\]
where $f=\sum\, a\uuu nz^n$ and
$g=\sum\, b\uuu nz^n$.
\bigskip

\noindent
{\bf{A study of poles.}}
Suppose that $f$ and $g$ both are rational functions
and let $\alpha\uuu 1,\ldots,\alpha\uuu m$ be the poles of $f$ and
$\beta\uuu 1,\ldots,\beta\uuu k$ the poles of $g$.
We assume that both $f$ and $g$ are analytic in a neighborhood of 
the origin and taking 
their Taylor series we construct $\psi(z)$ as above.
Then the following result was proved by Hadamard in ibid]:

\medskip

\noindent
{\bf{15.4 Theorem.}}
\emph{The function $\psi(z)$ extends to a meromorphic function in
the complex plane with  poles contained in the set
$\{\alpha\uuu\nu\cdot \beta\uuu j\}$.}
\medskip


\noindent
{\bf{Exercise.}} Prove this theorem. The hint is to use
Newton's frational decomposition of the rational functions $f$ and $g$.










 
















\end{document}







