

\documentclass{amsart}
\usepackage[applemac]{inputenc}
\addtolength{\hoffset}{-12mm}
\addtolength{\textwidth}{22mm}
\addtolength{\voffset}{-10mm}
\addtolength{\textheight}{20mm}

\def\uuu{_}

\def\vvv{-}

\begin{document}














\newpage


\centerline{\bf\large Series}
\bigskip



\centerline{\emph{Contents}}
\medskip
\noindent
0. Introduction
\medskip

\noindent
1. Additive series
\medskip

\noindent
1:B Counting functions

\medskip

\noindent
2. Power series


\medskip



\noindent
2:B Radial limits
\medskip

\noindent
2:C A theorem by Landau


\medskip

\noindent
3. Product series
\medskip

\noindent
4. Blascke products
\medskip

\noindent
5. Estimates using the counting function
\medskip

\noindent
6. Convergence on the boundary
\medskip

\noindent
7. An example by Hardy
\medskip

\noindent
8. Convergence under substitution
\medskip

\noindent
9. The series $\sum\, (a_1\cdots a_\nu)^{\frac{1}{\nu}}$
\medskip

\noindent
10. Thorin's convexity theorem.


\medskip

\noindent
11. Cesaro and H�lder limits
\medskip

\noindent
12. Power series and arithmetic means
\medskip

\noindent
13. Taylor series and quasi\vvv analytic functions







\bigskip












\noindent
\centerline {\bf Introduction.} 



\bigskip

\noindent
The study of infinite series is fundamental. 
The concepts are initially easy to grap but many
results require quite involved proofs.
In � 1-3 we study general facts about   additive series,
power series, and finally  product series.
Then we enter more advanced material.
An example is the Hardy-Littlewood theorem in � 6, and
� 8,9 and 12  expose resiults due to Carelman 
whose proofs also are quite technical.
The discussion in � 13
goes beyond the present scope where we
only describe som facts from 
Carleman's  deep studies about quasi-analytic  functions.
We shall  work with complex series. The reason is that
this is useful  when one begins to study  analytic functions  
of one complex variable.
\emph{Blaschke products}   play a central   role
in analytic function theory
and 
are studied  in � 4.
Another example where analytic function theory is used
occurs in the proof of Thorin's convexity theorem in � xx.
\medskip

\noindent
The literature about series is  extensive and treated
in many text-books where the reader may find
examples and applications.
Apart from the results due to Carleman  which
were established after 1920, the
material to be presented is inspired by classic texts-books  by Landau
and de Vall�-Poussin where the Hardy-Littlewood Theorem from 1913
is of the most recent origin.
\medskip

\noindent
{\bf{A remark.}}
Some students prefer to enter abstract, or general theories.
here we are confronted with material whose underlying concpets
are almost trivial. But I would like to point out that
the results are fundamental and used  in many areas of mathematics.
Personally  I  think that the best way to advance studies in mathematics is to
learn details of  proofs 
which do not require sophistaceted
definitions. It does not means that the proofs are simple.
To digest details in a section like  � 8  is for example
quite  demanding.







\newpage


\centerline
{\bf I. Additive series}

\medskip

\noindent
{\bf 1. Partial sums and convergent series.} To each  sequence $\{a_\nu\}$
of complex numbers
indexed by non-negative integers 
one associates  the partial sums:
\[
S_N=\sum_{\nu=0}^{\nu=N}\, a_\nu\quad\colon\quad N=1,2,\ldots
\]
If  $\{S_N\}$
converges converge to a limit $S_*$ one 
says
that $\{a_\nu\}$ yields a  convergent series
and writes
\[
S_*=\sum_{\nu=0}^{\infty}\, a_\nu\tag{1}
\]

\medskip


\noindent {\bf 2. Absolute convergence.}
The series is
absolutely convergent  if
\[ 
\sum_{\nu=0}^\infty\,|a_\nu|<\infty
\]

\medskip

\noindent
{\bf 3. A majorant principle.}
Let $\{b_\nu\}$ be a bounded sequence and
$\{a_\nu\}$ a sequence such that 
$\sum\,|a_\nu|<\infty$. Then it is obvious that
\[
\sum\,|a_\nu\cdot b_\nu|<\infty
\]
{\bf{Warning.}}
Absolute convergence implies that  $\sum\,a_\nu$ converges. 
For if $\{S_N\}$ are the partial sums 
the triangle inequality gives
\[
|S_M-S_N|=
|\sum_{\nu=N+1}^M\,a_\nu\,|\leq \sum_{\nu=N+1}^M\,|a_\nu|
\quad\colon\quad M>N\geq 0
\]
The absolute convergence therefore implies that the
sequence of partial sums is a \emph{Cauchy sequence} of complex numbers and 
hence has a limit.
The converse is false.
The series defined by the sequence
$\{a_\nu=\frac{(-1)^\nu}{\nu}\colon\,\,\nu\geq 1\}$
is not
absolutely convergent since 
\[
\sum_{\nu=1}^\infty\,\frac{1}{\nu}=\infty
\]

\medskip


\medskip


\noindent
{\bf 4. Alternating series.}
Let $a_0,a_1,\ldots$ be a sequence of positive real numbers which
is strictly decreasing, i.e. $a_0>a_1>\ldots$.
Assume also that $\lim a_n=0$.
Then one gets a convergent series by taking alternating signs, i.e.
the series below  converges:
\[ 
\sum_{\nu=0}^\infty\,(-1)^\nu\cdot a_\nu\tag{*}
\]


\noindent 
\emph{Proof.} 
Even partial sums are expressed by a positive series:
\[
S_{2N}=(a_0-a_1)+\ldots+(a_{2N}-a_{2N-1})>0\quad\colon\,N\geq 1
\]
At the same time 
\[ 
S_{2N}= a_0-[(a_1-a_2)+\ldots+a_{2N-1}-a_{2N}]
\]
where the last term is the negative of a positive series. Hence
$\{S_{2N}\}$
is a
non-increasing sequence of positive real numbers which    implies that
there exists a limit:
\[ 
\lim_{N\to\infty}\, S_{2N}= S_*
\]
Finally, since $a_n\to 0$ we get the same limit using odd indices
and hence  (*) is convergent.




\medskip


\noindent
{\bf 5. The partial sum formula.}
Consider two sequences $\{a_\nu\}$ and
 $\{b_\nu\}$.
Set
\[
S_N=\sum_{\nu=0}^{\nu=N}\, a_\nu\quad\colon\quad
T_N=
\sum_{\nu=0}^{\nu=N}\, a_\nu\cdot b_\nu
\]
\medskip 
\noindent
Since $a_\nu=S_\nu-S_{\nu-1}$ it follows that
\[
T_N=\sum_{\nu=0}^{\nu=N}\, (S_\nu-S_{\nu\vvv 1})\cdot b_\nu
=
a_0b_0+
\sum_{\nu=1}^{\nu=N\vvv 1}\, S_\nu\cdot(b_\nu-b_{\nu+1})+
S_N\cdot b_N\tag{*}
\]
This formula which resembles partial integration of functions is quite useful.

\medskip

\noindent {\bf {6. Exercise}}
Let $b_1\geq b_2\geq\ldots$
be a non-increasing sequence of positive real numbers.
Show that for
every finite $n$-tuple $a_1,\ldots,a_n$ of complex numbers one has
the inequality
\[
\bigl| b_1a_1+\ldots+b_na_n\bigr|\leq
b_1\cdot M\quad\text{where}\quad M=\max_{1\leq k\leq n}\, |a_1+\ldots+a_k|
\]




\medskip

\noindent {\bf{7. Theorem by Abel.}}
\emph{Assume that the partial sums $\{S_N\}$ 
of $\{a_\nu\}$
is a bounded sequence and that 
the positive series $\sum\,|b_\nu-b_{\nu+1}|<\infty$ where  $b_\nu\to 0$
as $\nu\to\infty$. Then the series below is  convergent}
\[
\sum_{\nu=0}^{\infty}\, a_\nu\cdot b_\nu
\] 


\noindent
\emph{Proof.} By (3)
the series
$\sum_{\nu=0}^\infty\, S_\nu\cdot(b_\nu-b_{\nu+1})$
is absolutely convergent and by the hypothesis we also have
$S_N\cdot b_N\to 0$ as $N\to\infty$.
Hence (*) in (5) shows that $\{T_N\}$
has a limit  $T_*$ expressed by the absolutely convergent series
\[ 
T_*=a_0b_0+
\sum_{\nu=1}^\infty\, S_\nu\cdot(b_\nu-b_{\nu+1})
\]



\noindent
Next,
let $\{b_\nu(p)\}$ be a doubly indexed sequence of
non-negative numbers which satisfies
\[
\nu\mapsto b_\nu(p)
\quad\text{is non-increacing for each}\quad p=0,1,\ldots
\]
\[
\lim_{p\to\infty}\, b_\nu(p)=1
\quad\text{for each}\quad \nu=0,1,\ldots
\]
\medskip

\noindent
{\bf{8. Theorem}}.\emph{ For each
convergent series $\sum\, a_\nu$ it follows that}
\[
\sum_{\nu=0}^\infty a_\nu\cdot b_\nu(p)
\] 
\emph{converges and one has the  limit formula}
\[
\lim_{p\to\infty}\, 
\sum_{\nu=0}^\infty a_\nu\cdot b_\nu(p)
=\sum_{\nu=0}^\infty a_\nu
\]
{\bf{9. Exercise.}} Prove Theorem 8.. The hint is to employ the partial sum
formula.
\medskip

\noindent
We finish with another useful result.

\medskip
\noindent
{\bf{10. Theorem}}
\emph{Let $\{a_\nu\}$
be a non-decreasing sequence of real numbers such that}
\[ 
\sum_{k=1}^\infty\, 2^{-k}(a_{k+1}-a_k)<\infty\tag{*}
\]
\noindent
\emph{Then}
\[
\lim_{k\to\infty}\, 2^{-k}a_k\to 0\tag{**}
\]


\noindent
\emph{Proof.}
Let $S_N$ denote partial sums of (*). If $M>N$ the partial summation formula gives:
\[
S_M-S_N=2^{-M}a_{M+1}-2^{-N}a_N+
\sum_{\nu=N+1}^{M}\, 2^{-\nu}a_\nu\tag{i}
\]
By the assumption
$a_N\leq a_\nu$ when $\nu>N$  which gives:
\[ S_M-S_N\geq
2^{-M}a_{M+1}-2^{-N}a_N+
a_N\cdot \sum_{\nu=N+1}^{\nu=M}\, 2^{-\nu}=
\]
\[
2^{-M}a_{M+1}-2^{-N}a_N+
a_N\cdot 2^{-N}(1- 2^{-M+N})=2^{-M}a_{M+1}-
a_N\cdot ^{-M}
\]


\medskip

\noindent
Now we can argue as follows. Since (*) holds the partial sums
is a Cauchy sequence. So if $\epsilon>0$
there exists $N_*$ such that
$S_M<S_{N_*}+\epsilon$ for every $M>N_*$. With $N=N_*$ above we therefore get
\[
2^{-M}a_{M+1}\leq \epsilon+a_{N_*}\cdot 2^{-M}
\quad\colon\quad M>N_*\tag{iii}
\] 
With
$N_*$ fixed we find a large $M$ such that (iii) entails
\[
2^{-M-1}a_{M+1}\leq 2\cdot \epsilon
\]
Since $\epsilon$ is arbitrary we get the limit (**).







\bigskip

\centerline {\bf B. Counting functions.}
\medskip

\noindent
A counting function $N(s)$
is an integer-valued function with unit 
jumps at some strictly increasing sequence
$0<s_1<s_2<\ldots$. Suppose that
$N(s)$ is defined for $0<s<1$ and assume that:
\[ 
\int_0^1\, (1-s)\cdot dN(s)<\infty\tag{*}
\]


\noindent
{\bf 1.B Theorem.} When (*) holds it follows that
\[
\lim_{s\to 1}\, (1-s)N(s)=0\tag{**}
\]


\noindent
\emph{Proof.}
Put $a_k=N(1-2^{-k})$.
The interval $[0,1]$ can be divided into
the intervals $[1-2^{-k},1-2^{-k-1})$.
It is easily seen that (*) implies that
\[ 
\sum\, 2^{-k}(a_{k+1}-a_k)<\infty
\]
Hence Theorem 7 gives
\[ 
\lim_{k\to\infty}\, 2^{-k}N(1-2^{-k})=0\tag{i}
\]
Next,
if 
$s<1$ we choose $k$ such that
$1-2^{-k}\leq s<1-2^{-k-1}$ and then
\[
(1-s)N(s)\leq 2^{-k}\cdot N(1-2^{-k-1})=2\cdot2^{-k-1}\cdot N(1-2^{-k-1})
\tag{ii}
\] 
Hence (i) implies that (**) tends to zero as required.


\bigskip

\centerline {\bf 2.B A study of $\sum\, (1-a_k)$}.
\medskip

\noindent
Let $0<a_k<1$ and suppose that the series
\[
 \sum\, (1-a_k)<\infty\tag{1}
\]
Let $N(s)$ be the counting function with unit jumps at $\{a_k\}$.
So (1) means that
\[ 
\int_0^1\, (1-s)dN(s)<\infty\tag{2}
\]

\noindent
{\bf{Exercise.}}
With $s=1-\xi$ close to 1, i.e. when
$\xi$ is small the Taylor expansion of the Log-function at
$s=1$ gives
\[
\log\frac{1}{s}=
\log\frac{1}{1-\xi}=\xi+\text{higher order terms in}\,\,\xi
\]
Use this to prove that  (2) holds if and only if
\[
\int_0^1\,\log(\frac{1}{s})\cdot dN(s)<\infty\tag{3}
\]
\medskip

\noindent
Next, assume that (2) holds and for each $0<r<1$ we set

\[
S(r)=\int_0^r\,\log\, \frac{1}{s}\cdot dN(s)\quad\text{and}\quad
T(r)=\int_0^r\,\log\,\frac{r}{s}\cdot dN(s)\tag{4}
\]
\medskip


\noindent
Since $\log(\frac{1}{s})-\log(\frac{r}{s})=\log\,\frac{1}{r}$
it follows that
\[ 
S(r)-T(r)=\log\,\frac{1}{r}\int_0^r\,dN(s)=
\log\,\frac{1}{r}\cdot N(r)\tag{5}
\]


\noindent
Now
$\log\,\frac{1}{r}\simeq 1-r$ as $r\to 1$ and since (2)  is assumed, it
follows from Theorem 1.B that:
\[
\lim_{r\to 1}\log\,\frac{1}{r}\cdot N(r)=0
\]
Hence (5) gives
\[
\lim_{r\to 1}\, S(r)-T(r)=0\tag{6}
\]


\noindent With the notations above we have therefore proved
\medskip

\noindent {\bf 3.B Theorem.} \emph{Assume that (2) holds. Then
the following limit formula holds}


\[
\lim_{r\to 1}\int_0^r\,\log\,(\frac{r}{s})\cdot dN(s)=
\int_0^1\,\log(\frac{1}{s})\cdot dN(s)
\]
\medskip

\noindent
{\bf 4.B Remark.}
By the multiplicative property of the Log-function
the last term becomes
\[
\sum\,\log\,\frac{1}{a_k}
=\log\,\prod\, \frac{1}{a_k}\tag {i}
\]
In the right hand side there appears a \emph{product series}
defined by the $a$-sequence. In  section III we study
product series in more detail but already here we have seen an example of
the interplay between additive series and product series.
Notice also that via the equivalence of (1) and (2)
above, Theorem 3.B gives the following:
\medskip

\noindent
{\bf 5.B Theorem.} \emph{Let $\{a_k\}$ be a sequence with each
$0<a_k<1$. Then the additive series
$\sum\, (1-a_k)$ is convergent if and only if
the product series}
\[ 
\prod_{k=1}^\infty\, \frac{1}{a_k}<\infty\tag {i}
\]
\emph{Moreover, when (i)  holds one has the limit formula}
\[
\lim_{r\to 1}\,
\prod_r\, \frac{r}{a_k}=
\prod_{\nu=1}^\infty\, \frac{1}{a_k}<\infty\tag {ii}
\]
\emph{where $\prod_r$ is extended over those $k$ for which $a_k\leq r$.}


\medskip

\centerline {\bf{6.B Asymptotic formulas.}}
\medskip


\noindent
Here we announce a deeper result without proof.
Let $\{\lambda\uuu \nu\}$ be a strictly increasing sequence of positive numbers
where $\lambda_\nu=+\infty$ as $\nu\to \infty$.
Consider another sequence of positive numbers
$\{a\uuu\nu\}$ and assume that the series
\[
\sum\uuu{\nu=1}^\infty\, \frac{a\uuu\nu}{\lambda\uuu\nu}<\infty
\]
For each $s>0$ we denote by $\Lambda(s)$ be the largest integer $\nu$
such that $\lambda\uuu\nu\leq s$.
So the $\Lambda$\vvv function is a non\vvv decreasing integer\vvv valued 
function with jumps at every $\lambda\uuu\nu$.
Next, define the following pair of
functions when $0<x<\infty$:
\[
\mathcal A(x)=\sum\uuu{\nu<\Lambda(x)}\, a\uuu\nu
\quad\text{and}\quad
f(x)= \sum\uuu{\nu=1}^\infty\, \frac{a\uuu\nu}{\lambda\uuu\nu+x}
\]
Notice that $\mathcal A(x)$ is non\vvv decreasing while $f(x)$ is
decreasing.
With these notations the following hold:

\medskip

\noindent
{\bf{7.B Theorem.}}
\emph{Assume that there exists some $0<\alpha<1$ and a positive constant
$C$ such that}
\[
\lim\uuu{x\to \infty}\, x^\alpha\cdot f(x)=C
\]
\emph{Then there also exists the limit}
\[
\lim \uuu{x\to \infty}\, x^{\alpha\vvv 1}\cdot \mathcal A(x)=
\frac{C}{\pi}\cdot \frac{\sin \pi\alpha}{1\vvv\alpha}
\]
\medskip

\noindent
{\bf{Remark.}} 
Theorem 7.B  is due to Carleman and was established in  his lectures 
at Institute Mittag\vvv Leffler in 1935. The proof requires
analytic methods based upon  Fourier transforms
and is given in � XX from Special Topics.
In the special case when $\lambda\uuu\nu=\nu$ we see that
$\mathcal A(n)$ is equal to the partial sum
$S\uuu n$ of $\{a\uuu\nu\}$ for positive integers $n$.
So in this  special case Theorem 7.B 
asserts  that if the positive series
\[
\sum\, \frac{a\uuu\nu}{\nu}<\infty
\]
and if there exists 
\[ 
\lim\uuu{x\to\infty}\,x^\alpha\cdot \sum\, \frac{a\uuu\nu}{\nu+x}=C
\] 
then 
\[
\lim\uuu{n\to\infty}\frac{n^\alpha\cdot S\uuu n}{n}=
\frac{C}{\pi}\cdot \frac{\sin \pi\alpha}{1\vvv\alpha}
\]










\newpage





\centerline{\bf\large II. Power series.}
\medskip

\noindent
 Starting with a sequence
$\{a_\nu\}$ and a complex number $z\neq 0$ 
we get the  sequence $\{a_\nu\cdot z^\nu\}$.
If this sequence yields a convergent additive series
the sum is denoted by 
$S(z)$. 
\medskip

\noindent 
{\bf 1. Definition} \emph{The set of all $z\in\bf C$
for which the series}
\[ 
\sum_{\nu=0}^\infty\, a_\nu\cdot z^\nu
\] 
\emph{converges is denoted by $\mathfrak {conv}(\{a_\nu\})$ and called the 
set of convergence for the $a$-sequence.}
\medskip

\noindent
{\bf Remark.} It may occur that 
$\mathfrak{conv}(\{a_\nu\})$ just contains $z=0$. 
An example is when $a_\nu=\nu\,!$.
But if the absolute values $|a_\nu|$ do not 
increase too fast, then $\mathfrak {conv}(\{a_\nu\})$
contains non-zero complex numbers.
Since the terms of a
convergent sequence  is 
 bounded, each
 $z_0\in \mathfrak{conv}(\{a_\nu\})$
 gives a constant $M$ such that
\[
|a_\nu|\cdot |z_0|^\nu\leq M\quad\colon\quad\nu=0,1,\ldots
\tag{1}
\]
If $|z|<|z_0|$ it follows that the series defined by 
$\{a_\nu\cdot z^\nu\}$ is absolutely convergent.
Indeed, 
we have
\[ 
|a_\nu\cdot z^\nu|\leq M\cdot \frac{|z|^\nu}{|z_0|^\nu}
\]
Here $r=\frac{|z|}{|z_0|}<1$ and the geometric series
$\sum\, r^\nu$ is convergent. Hence
the \emph{Majorant principle }  from I.3  yields the
the absolute convergence of
$\{a_\nu\cdot z^\nu\}$.

\medskip

\noindent
{\bf 2. The radius of convergence.}
Above we  saw that if  $z_0\in\mathfrak{conv}(\{a_\nu\})$ then
the domain of convergence contains the open disc of radius $|z_0|$.
Put
\[
\mathfrak{r}=\max\, |z|\quad\colon\quad
z\in \mathfrak{conv}(\{a_\nu\})\tag{*}
\]
Assume that
$\mathfrak{conv} (\{a_\nu\})$ is not reduced to $z=0$. Then $\mathfrak r$
is a positive number
or $+\infty$. It is called the radius of convergence for
$\{a_\nu\})$. The case $\mathfrak{r}=+\infty$
means that the series
\[ 
\sum\, a_\nu\cdot z^\nu
\]
converges for all $z\in\bf C$.
\bigskip

\noindent 
{\bf 3. Hadamard's formula for $\mathfrak{r}$.}
Given a sequence $\{a_\nu\}$ its radius of convergence
is
found by taking a limes superior. More precisely
\[ 
\frac{1}{\mathfrak{r}}=
\limsup_{\nu\to\infty}\,\,
|a_\nu|^{\frac{1}{n}}\tag{*}
\]


\noindent 
{\bf{3.1 Exercise.}}
Prove Hadamard's formula.
\medskip

\noindent
{\bf{3.2 Remark.}} A sufficient condition in order that 
$\mathfrak{r}\geq 1$ for a given sequence $\{a_\nu\}$
can be  checked as follows.
Suppose that
\[ 
|a_\nu|\leq e^{\rho(\nu)}
\]
for some sequence $\{\rho(\nu)\}$. With  $r<1$ we can write
$r=e^{-\delta}$ for some $\delta>0$
and obtain
\[
|a_n|\cdot r^n\leq \text{exp}(\rho(n)-\delta\cdot n)\quad\colon\, n=1,2,\ldots
\]
From this we  conclude that $\mathfrak{r}\geq 1$ holds  if
\[
\lim_{n\to\infty}\, \rho(n)-\delta\cdot n=-\infty
\quad\text{for each}\quad \delta>0\tag{**}
\]

\medskip


\noindent
{\bf{4. Application.}}
Let $\sum\, a_n\cdot z^n$
be a power series whose radius of convergence is one.
Let $\{b_n\}$ be some other sequence of complex numbers.
We seek for conditions in order that
the series $\sum\, b_n a_n\cdot z^n$
also converges when $|z|<1$.
The result below gives a sufficient condition for this to hold.
\newpage

\noindent
{\bf{5. Theorem.}}
\emph{Let $\{\gamma_n\}$
be a sequence of positive numbers such that}
\[ 
\lim_{n\to\infty}\,
\frac{\gamma_n}{n}\cdot \text{log}(n)=0\tag{i}
\]
\emph{Then the $\mathfrak r$-number of
$\{b_\nu\cdot a_\nu\}$ is $\geq 1$ for every
$b$-sequence such that}
\[
|b_n|\leq n^{\gamma_n}\quad\colon\quad n=1,2,\ldots
\]

\noindent
{\bf{6. Exercise.}} Prove this theorem.
 It applies in particular when 
$\gamma_n=k$ for some positive integer $k$ and hence the radius of convergence of
$\{\nu^k\cdot a_\nu\}$ is at least one. Of course, this can be seen directly from
the formula in (3) above since
\[
\lim\uuu{n\to \infty}\, n^{\frac{k}{n}}=1
\] 
hold for every positive integer $k$.

\medskip










\noindent 
{\bf{7. Hadamard's Lemma.}}
Let $\{c_n\}$ be a sequence if numbers such that the following two conditions hold:

\[ 
\limsup_{n\to \infty}\, |c_n|^{\frac{1}{n}}=1\tag{i}
\]
There exists some $0<\alpha<1$ such that
\[
\bigl|c_{n+1}^2-c_{n+2}\cdot c_n\bigr|\leq \alpha^n
\quad\colon\quad n=1,2,\ldots\tag{ii}
\]
Show that (i-ii) imply that
one has an unrestricted limit:
\[ 
\lim_{n\to \infty}\, |c_n|^{\frac{1}{n}}=1\tag{*}
\]






\medskip










\noindent 
{\bf{8. Exercise.}}
Let $a_0,a_1,\ldots$ be a sequence of positive real numbers.
Suppose there exists an integer $m$ and a constant
 $C$ such that
\[ 
a_k\leq \frac{a_{k-1}+\ldots+a_{k-m}}{k}
\quad\text{for all}\quad k\geq m
\]
Show that no matter how $a_0,\ldots,a_{m-1}$ are determined initially
it follows that 
the power series
\[ 
\sum\, a_\nu\cdot z^\nu
\] 
has an infinite radius of convergence, i.e. for every
$R>0$ the positive series
$\sum\, a_\nu\cdot R^\nu<\infty$.





\newpage



\noindent 
\centerline {\bf  II.B Convergence at the boundary}

\bigskip



\noindent 
Let $\{a_\nu\} $ be a sequence with  $\mathfrak{r}=1$.
Given some $0\leq\theta\leq 2\pi$ we have the complex number 
$e^{i\theta}$  whose absolute value is one. It
is not always true that the series
\[
\sum_{\nu=}^\infty a_\nu\cdot e^{i\nu\theta}\tag{1}
\] 
converges. So we have a possibly empty subset of $[0,2\pi]$ defined by
\medskip
\[ 
\mathcal F=\{0\leq\theta\leq 2\pi\}\quad\colon\quad
\text{The series (1) converges for}\,\,\theta\tag{2}
\]
\medskip

\noindent {\bf  1. Example}
Let $\{a_\nu=\frac{1}{\nu}\}$.
Here $\mathfrak{r}=1$ and
the series
$\sum\,\frac{1}{\nu}$ is divergent.
On the other hand 
\[ 
\sum\,\frac{e^{i\nu\theta}}{\nu}
\]
converges for each  $0<\theta<2\pi$. In other words
\[
\mathcal F=(0,2\pi)\tag{i}
\]

\noindent
To see this we notice that if
$b_\nu= e^{i\nu\theta}$ with $0<\theta<2\pi$ then the partial sums are:
\[ 
S_N=\frac{1-e^{i(N+1)\theta}}{e^{i\theta}-1}
\]
This sequence is bounded
and since the positive series
$\sum\,(\frac{1}{\nu}-\frac{1}{\nu+1})$
converges, the reader can deduce the inclusion (i) from
Abel's theorem in A.7

\bigskip


\noindent{\bf 2. Radial limits}
Let  $\{a_\nu\}$ be a sequence whose radius of convergence is 1.
If $0<r<1$ and $0\leq \theta\leq 2\pi$ we get the convergent series
\[
S(r,\theta)=\sum_{\nu=0}^\infty\, 
a_\nu \cdot r^\nu e^{i\nu\theta}
\]
Keeping $\theta$ fixed we say that one has a radial limit if
there exists
\[ 
\lim_{r\to 1}\, S(r,\theta)=S_*(\theta)\tag{*}
\]

\noindent 
Denote by Let $\mathfrak{rad}(\{a_\nu\})$  the set of $\theta$ for which the limit above exists.
The question arises if $\theta\in\mathfrak{rad}(\{a_\nu\}) $
implies that the series 
$\sum\, a_\nu e^{i\nu\theta}$ converges.
This is not true in general. The simplest example is 
to take $a_\nu=(-1)^\nu$
and $\theta=1$. Here 
$S(r,0)=\frac{1}{1+r}$ whose limit is $\frac{1}{2}$
while  $\sum\,a_\nu$  diverges  since the $a$-sequence
does not tend to zero. 
But the converse is true, i.e. one has:


\bigskip
\noindent{\bf {3. Theorem}}
\emph{Let $\{a_\nu\}$ give a convergent additive series with sum $S\uuu *$.
Then there exists the limit}

\[ 
\lim\uuu{x\to 1}\, \sum\, a\uuu n\cdot x^n
\]
\emph{Moreover, the radial limit is equal to the series sum $S\uuu *$ of the additive series.}
\medskip

\noindent
\emph{Proof.}
We can always modify $a_0$ and assume that $S_*=0$.
Set
\[ 
\rho_N=\max_{\nu\geq N}\, |S_\nu|\tag{i}
\]
So the hypothesis is now that $\rho_N\to 0$ as $N\to\infty$.
For each  $0<x<1$ we set:
\[ 
S_N(x)=\sum_{nu=0}^{\nu=N}\, a_\nu\cdot x^\nu
\] 
When $0<x<1$ is fixed  the infinite power series
\[ 
S_*(x)=\sum_{\nu=0}^{\infty}\, a_\nu\cdot x^\nu\tag{ii}
\] 
converges.
Next,
when $0<x<1$ then
the sequence $\{b_\nu=x^\nu-x^{\nu+1}\}$
is non-increasing. 
Hence Exercise 6 from [Additive Series] implies   that
that for each pair $M>N$ 
and every $0<x<1$ one has 
\[
|S_M(x)-S_N(x)|\leq\rho_N
\]
Since this holds for every $M>N$ and the series
(ii) converges we obtain
\[
[S_*(x)-S_N(x)|\leq\rho_N\tag{iii}
\]
\noindent
Next, the triangle inequality
gives:
\[
|S_*(x)|\leq |S_*(x)-S_N(x)|+
|S_N(x)-S_N|+|S_N|\leq 
\]
\[
2\cdot \rho_N+
|S_N(x)-S_N|
\]
Finally,  if $\epsilon>0$ we first choose $N$ so that
$2\cdot \rho_N<\epsilon/2$ and with $N$ fixed we have
\[ 
\lim_{x\to 1}\, S_N(x)=S_N
\]
This proves the requested limit formula
\[
\lim_{x\to 1}\, S_*(x)=0
\]

\bigskip

\centerline{\bf 4.  A theorem of Landau.}
\medskip

\noindent
One can also study
limits on
sparse sets which converge to a boundary point.
Results of this nature appear in the 
article
\emph{�ber die Konvergenz einiger Klassen von unendlichen
Reihen am Rande des Konvergenzgebietes} by Landau from 1907.
Here we announce and prove one of these  results.
Consider a sequence of complex numbers
$\{z_k\}$ in the open unit disc $D$ 
which converge to 1.
We say that the sequence is of   Landau type  if there exists a constant $
{\bf{L}}$
such that
\[
\frac{|1-z_k|}{1-|z_k|}\leq {\bf{L}}\quad\colon\quad
\frac{1}{{\bf{L}}}\leq k\cdot |1-z_k|\leq {\bf{L}}\quad\colon\, k=0,1,2,\ldots
\tag{i}
\]
\medskip
\noindent
{\bf Remark.}
The first inequality means that
$z_k$ come close to the real axis as $|z_k|\to 1$.
The second condition means that the sequence of absolute values
$1- |z_k|$ decreases  in a 
regular fashion.
\bigskip


\noindent
{\bf 4.1 Theorem}.
\emph{Let 
$\{a_\nu\}$ be a sequence such that
$\nu\cdot a_\nu\to 0$ as $\nu\to+\infty$ and suppose there exists a sequence
$\{z_k\}$  of the Landau type such that  there exists a limit}
\[ 
\lim_{k\to\infty}\sum a_\nu \cdot z_k^\nu=A
\]



\noindent
\emph{Then  the series
$\sum\, a_\nu$ is convergent and the series sum is equal to $A$.}

\medskip



\noindent
\emph{Proof.}
Since
$\nu\cdot a_\nu\to 0$ it follows that
\[
\lim_{k\to\infty}\,
\frac{1}{k}\cdot \sum_{\nu=1}^k\, a_\nu=0\tag{i}
\]
Next, set
\[
f(k)=\sum_{\nu=1}^{\nu=k}\, a_\nu z_k^\nu\quad\text{and}\quad 
S_k=\sum_{\nu=1}^{\nu=k}\, a_\nu\tag{ii}
\]
The triangle inequality gives
\[
|S_k-f(k)|\leq 
\bigl|\sum_{\nu=1}^{\nu=k}a_\nu(1-z_k^\nu)-
\sum_{\nu>k}\, a_\nu z_k^\nu\bigr |\leq
\]
\[
\sum_{\nu=1}^{\nu=k}|a_\nu|(1-z_k|\cdot \nu +
\sum_{\nu>k}\, |a_\nu|\cdot | z_k|^\nu=W(k)_*+W(k)^*\tag{iii}
\]
Put 
\[
\epsilon(k)=\max\,\{\nu\cdot |a_\nu|\colon\,\, \nu\geq k+1\}\implies
|a_\nu|\leq \frac{\epsilon(k)}{k}\quad\colon\,\nu\geq k+1\tag{iv}
\]
Since we also have
$|z_k|^{k+1}\leq 1$
it follows from (iv) that
\[ 
W^*(k)\leq \frac{\epsilon(k)}{k}\cdot\frac{1}{1-|z_k|}
\leq\frac{{\bf{L}}\cdot \epsilon(k)}{k\cdot |1-z_k|}\leq
{\bf{L}}^2\cdot\epsilon(k)\tag{v}
\]
At the same time we have
\[
W_*(k)\leq k\cdot |1-z_k|\cdot\frac{\,\sum_{\nu=1}^{\nu=k}\nu\cdot |a_\nu|}{k}
\leq{\bf{L}}\cdot\frac{\,\sum_{\nu=1}^{\nu=k}\nu\cdot |a_\nu|}{k}\tag{vi}
\]
Now we are done,  i.e. $W_*(k)\to 0$
by
the observation in (*) and $W^*(k)\to 0$  since
the hypothesis on $\{a_\nu\}$
gives
$\epsilon(k)\to 0$.










\newpage

\centerline{\bf\large III. Product series}
\bigskip

\noindent
Consider a sequence of positive real numbers
$\{q_\nu\}$. To each $N\geq 1$ we define the partial product
\[
\Pi_N=\prod_{\nu=1}^{\nu=N}\,q_\nu
\]
If $\lim_{N\to\infty}\,\Pi_N$ exists we say that the infinite product
converges and put
\[
 \Pi_*=
\prod_{\nu=1}^\infty\,q_\nu
\]
It is clear  that if the product converges then
$\lim_{\nu\to\infty}\, q_\nu=1$.
A very useful result goes as follows:

\medskip

\noindent 
{\bf 1. Theorem.} \emph{Let $\{q_\nu\}$ be a sequence where
$0<q_\nu<1$ hold for all $\nu$.
Then the following three conditions are equivalent:}
\[ 
\sum\,(1-q_\nu)<\infty\quad\colon\,
\sum\,\text{Log}\,\frac{1}{q_\nu}<\infty\quad\colon
\prod_{\nu=1}^\infty\,q_\nu>0
\]


\noindent
{\bf{Exercise}}
Prove this theorem. A hint is that
the function $\log r$ 
has the Taylor expansion close to $r=1$
given by
\[
\log r=(r-1)+(r-1)^2/2+\ldots
\]


\medskip

\noindent {\bf 2. Proposition}
\emph{One has the inequality}
\[
|\text{Log}(1+z)-z|\leq  |z|^2\quad\colon\,|z|\leq 1/2
\]

\noindent
{\bf{Exercise.}}
Prove this inequality.
A hint is to use the series expansion of rhe complex log-function:
 \[
\log (1+z)=
z-z^2/2+z^3/3+\ldots
\]



\medskip

\noindent 
Next, consider a complex sequence $a(\cdot)$ where
$|a_\nu|\leq\frac{1}{2}$ hold for all $\nu$ and put:
\[
\Pi_N=\prod_{\nu=1}^{\nu=N}\,(1-a_\nu)\implies
\log\, (\Pi_N)=
\sum_{\nu=0}^{\nu=N}\,
\log (1-a_\nu)
\]
Proposition 2 gives  the inequality
\[
|\log (1-a_\nu)+a_\nu|\leq |a_\nu|^2\tag{*}
\]

\noindent
This enable us to investigate the convergence of the product series with
the aid of the additive series for $\{a_\nu\}$. We get for example
\[
|\log\, (\Pi_N)+\sum_{\nu=1}^{\nu=N}\, a_\nu\,|\le
\sum_{\nu=1}^{\nu=N}\, |a_\nu|^2\tag{**}
\]

\noindent
From (**) we can conclude:
\medskip

\noindent
{\bf 3. Theorem.}
\emph{Let $\{a_\nu\})$ be a sequence where each $|a_\nu|\leq\frac{1}{2}$ and
$\sum\, |a_\nu|^2<\infty$. Then  $\sum\,a_\nu$ converges if and only if
the product series $\Pi\,(1-a_\nu)$ converges.
Moreover, when convergence holds one has the equality}
\[ 
\log\,(\Pi_*)=\sum_{\nu=1}^\infty\,
\log(1-a_\nu)
\]


\newpage

\centerline {\bf IV. Blaschke products.}
\bigskip


\noindent 
Let $\{a_\nu\}$ be a sequene in the open unit disc $D$ which
are  enumerated 
so that their absolute values are non-decreasing.
But repetitions may occur, i.e. several $a$-numbers can be equal.
We always assume that $|a_\nu|\to 1$ as $\nu\to+\infty$. Hence
$\{a_\nu\}$ is a discrete subset of $D$.
To each $\nu$ we set
\[ 
\beta_\nu(\theta)=
\frac{e^{i\theta}-a_\nu}{ 1-
e^{i\theta}\cdot\bar a_\nu}\cdot\frac{\bar a_\nu}{|a_\nu|}\quad\colon\quad
0\leq\theta\leq 2\pi\tag{1}
\]


\noindent
The \emph{Blaschke product of order $N$} is the partial product
\[ 
B_N(\theta)=\prod_{\nu=1}^{\nu=N}\,
\beta_\nu(\theta)\tag{2}
\]


\noindent 
The question arises when the product series converges and gives a limit
\[
B_*(\theta)=\prod_{\nu=1}^\infty\,
\beta_\nu(\theta)\tag{3}
\]


\noindent To analyze this we use polar coordinates and put
\[
a_\nu= r_\nu e^{i\theta\nu}
\]
Each $\beta$\vvv number has absolute value one
and if $\theta\neq \theta\uuu\nu$ for every
$\nu$ we have
\[ 
\beta_\nu(\theta)=e^{i\cdot\gamma(r_\nu,\theta-\theta_\nu)}
\quad\colon\quad 0<\gamma(r_\nu,\theta-\theta_\nu)<2\pi
\tag{4}
\]


\noindent
{\bf{Exercise.}}
Show that when $\vvv \pi/2<\theta \vvv \theta\uuu\nu<\pi/2$
then the construction of the $\arctan$\vvv function gives
\[
\gamma(r,\theta\vvv \theta\uuu\nu)=
\text{arctg}\,[\frac{(1-r^2)\cdot \text{sin}(\theta\uuu\nu\vvv\theta)}
{1+r^2-2r\text{cos}(\theta\uuu \nu\vvv \theta)}\,]\tag{4}
\]


\medskip

\noindent {\bf 4.2. Blashke's condition}
We impose the condition that the positive series
\[ 
\sum\, (1-r\uuu \nu)<\infty\tag{*}
\]
\noindent
{\bf{4.3 Exercise.}}
If  $x$ is a real number we set
\[
\{x\}=\min\uuu {k\in{\bf{Z}}}\, [x\vvv 2\pi k|
\]
Assume that (*) holds. Show that the Blaschke product has a radial limit at
$\theta=0$ if and only if there exists the limit
\[
\lim\uuu{N\to\infty}\,\bigl\{
\,\sum\uuu{\nu=1}^{\nu=N}\,
\frac{(1-r_\nu)\cdot \theta_\nu}
{(1-r_\nu)^2+\theta_\nu^2}\,\bigr\} \tag{4.3.1}
\]
In other words, (4.3.1) holds if and only if
the infinite product $B_*(0)$ exists and
\[
\lim_{r\to 1}\, B(r)= B_*(0)
\]
{\bf{4.4.Remark.}}
Notice  that $\theta_\nu$ may be $<0$ or $>0$
and it is not necessary that all of them become close to
$0$. 
To determine all sequence of
pairs $(r\uuu\nu,\theta\uuu\nu)$
where 4.3.1 holds and $\theta\uuu\nu\to 0$
appears to be a very difficult problem.
\bigskip


\noindent
In  complex analysis
one considers the analytic function
defined in the open unit disc $|z|<1$ by
\[ 
B(z)=\prod_{\nu=0}^\infty\, \frac{z-a_\nu}{1-\bar a_\nu\cdot z}
\cdot e^{-i\text{arg}(a_\nu)}\tag{i}
\]
which exists under the sole condition that Blascke's condition (*) is valid.
A major resultdue tio Blaschke
asserts that the
\[ 
\lim_{r\to 1}\,
 B(re^{i\theta})= B_*(\theta)
\]
exists for almost every
$\theta$, and
the absolute value 
of the limit valkue $B_*(\theta)$ is equal to one almost everywhere,
taken in the sens of Lebesgue.
But the determination of the set of all
$0\leq \theta\leq 2\pi$ for which the radial limit exists
is not clear when
no special assumptions are imposed on
the $\{\theta_\nu\}$-sequence.
For example, divergence may appear when many 
$\theta_\nu$:s  are close to $\theta$ even if 
$\{r\uuu\nu\}$ tend rapidly to 1.
\medskip


\newpage

\centerline{\bf \large V. Estimates using the counting function.}
\bigskip

\noindent
Let $\{\alpha_\nu\}$ be a complex sequence where
$0<|\alpha_1|\leq |\alpha_2|\leq\ldots\}$, amnd assume that
the absolute values tend to $+\infty$. We get
the counting function $N(R)$ which for every $R>0$ is the number of
$\alpha_\nu$ with absolute value $\leq R$.
Consider the situation when
there exists a constant $C$ such that
\[ 
N(R)\leq C\cdot R\quad\text{for all}\quad R>0\tag {*}
\]

\medskip

\noindent
{\bf 5.1. The first estimate.} To each $R>0$ we set
\[
S(R)=\prod\, (1+\frac{R}{|\alpha_\nu|})\quad\colon
\text{product taken over all}\,\,|\alpha_\nu|\leq 2R\tag {2}
\]
Then we have
\[ 
S(R)\leq e^{KR}\quad\text{where}\quad
K=2C(1+\text{Log}\,\frac{3}{2})\tag{*}
\]


\noindent
To prove this we consider $\log\,S(R)$. A partial integration gives:
\[ 
\log\,S(R)=\int_0^{2R}\,\log\,(1+\frac{R}{t})\cdot dN(t)=
\log\, (1+\frac{1}{2})\cdot N(2R)+
\int_0^{2R}\, \frac{R\cdot N(t)}{t(t+R)}\cdot dt
\]


\noindent Since $\frac{R}{t+R}\leq 1$ for all $t$, the last integral is
estimated by $2R\cdot C$ and (*) follows.


\medskip


\noindent
{\bf 5.2. The second estimate.}
For each $R>0$ we consider
infinite tail products:
\[
S^*(R)=\prod\,(1+\frac{R}{\alpha_\nu})\cdot e^{-\frac{R}{\alpha_\nu}}
\quad\colon\text{product taken over all}\,\,|\alpha_\nu|\geq 2R\tag {i}
\]


\noindent
To estimate (i) we notice that 
the analytic function $(1+\zeta)e^{-\zeta}-1$ has a double zero
at the origin. This gives  a constant $A$ such that
\[
|(1+\zeta)e^{-\zeta}-1|\leq A\cdot |\zeta|^2\quad\colon\, |\zeta|\leq\frac{1}{2}\tag{ii}
\]


\noindent
Since $|\alpha_ \nu|\geq 2R$ for every $\nu$ we obtain:
\[ \log^+|(1+ \frac{R}{\alpha_\nu})\cdot e^{-\frac{R}{\alpha_\nu}}|
\leq\log\,[1+A\frac{R^2}{|\alpha_\nu|^2})\leq
A\cdot \frac{R^2}{|\alpha_\nu|^2}\tag{iii}
\]


\noindent
From (6) we get
\[
\log^+(S^*(R))\leq
AR^2\int_{2R}^\infty\, \frac{dN(t)}{t^2}= A\cdot N(2R)+
2AR^2\cdot \int_{2R}^\infty\, \frac{N(t)}{t^3}\tag{iv}
\]


\noindent
The last term is estimated by 
\[
2AR^2\cdot C\cdot\int_{2R}^\infty\,\frac{dt}{t^2}=AC\cdot R\tag {8}
\]


\noindent
Adding up the result we get
\medskip

\noindent 
{\bf 5.3 Theorem.}
\emph{One has the inequality}
\[ 
S^*(R)\leq \frac{5A}{4}\cdot C\cdot R
\]


\newpage






\centerline{\bf \large{VI. Theorems by Abel, Tauber, Hardy and Littlewood}}
\bigskip


\noindent
{\bf Introduction.}
Consider a power series
$f(z)=\sum\, a_nz^n$ whose radius of convergence is one.
If $r<1$ and $0\leq\theta\leq 2\pi$
the series
\[ 
f(re^{i\theta})= 
\sum\, a_nr^ne^{in\theta}
\]
Passing to $r=1$ it is in general not true that the series
$\sum\, a_ne^{in\theta}$ is convergent. An example arises if we consider
the geometric series
\[ \frac{1}{1-z}= 1+z+z^2+\ldots
\] 
So here $a_n=1$ for all $n$ and hence the absolute values
$|a_nre^{in\theta}|=1$ for sll $n$ so the series
$\sum\, a_ne^{in\theta}$ diverges.
At the same time we notice that when
$0<\theta<2\pi$  there exists the limit
\[
\lim_{r\to 1}\, \sum\, r^ne^{in\theta}=\frac{1}{1-re^{i\theta}}
\]
This leads to the following  problem where we without loss of
generality can take 
$\theta=0$.
Consider as above a
convergent power series and assume that there exists the limit
\[
\lim_{r\to 1}\, \sum\, a_nr^n\tag{*}
\]
When can we conclude that the series
$\sum\, a_n$ also is convergent and that
one has the equality
\[
 \sum\, a_n=
\lim_{r\to 1}\, \sum\, a_n r^n\tag{**}
\]
The first result in this direction was established by Abel in a work from 1823:
\medskip

\noindent
{\bf{A. Theorem}} \emph{Let $\{a_n\}$ be a sequence such that
$\frac{a_n}{n}\to 0$
as $n\to \infty$
and there exists}
\[
A=\lim_{r\to 1}\, \sum\, a_nr^n
\] 
\emph{Then  
$\sum\, a_n$ is convergent and the  sum is $A$.}

\medskip

\noindent
An extension of Abel's result was
established by  Tauber in 
1897. 
\medskip

\noindent
{\bf B. Theorem.}
\emph{Let $\{a_n\}$ be a sequence of real numbers such that}
\[ 
\lim_{n\to\infty}\, \frac{a_1+2a_2+\ldots+na_n}{n}=0
\]
\emph{Then, if there exits the radial limit}
 \[
A=\lim_{r\to 1}\, \sum\, a_nr^n
\]
\emph{it follows that 
the series $\sum\, a_n$ is convergent and the sum is $A$.}

\noindent

\bigskip


\centerline {\bf C. Results by 
Hardy and Littlewood.}.
\medskip

\noindent
The following extension of Abel's  result was proved by Hardy and Littlewood
in 1913:

\medskip

\noindent
{\bf {C Theorem.}}
\emph{Let $\{a_n\}$ be a sequence of real numbers such that there 
exists a constant $C$ so that
$\frac{a_n}{n}\leq C$ for all $n\geq 1$. Assume also that the
power series $\sum\, a_nz^n$ converges when
$|z|<1$. Then the same conclusion as in Abel's theorem holds.}
\medskip

\noindent
{\bf {Remark.}}
The proof of Theorem C
requires several steps where an essential ingredient is a result 
about
positive series from the cited article which has independent interest.

\newpage

\noindent
{\bf {D. Theorem.}}
\emph{Assume that
each $a_n\geq 0$ and that there exists the limit:}
\[ 
A=
\lim_{r\to 1}\, (1-r)\cdot \sum\,a_nr^n\tag{*}
\]
\emph{Then  there exists the limit}
\[
A=\lim_{N\to \infty}\, \frac{a_1+\ldots+a_N}{N}\tag{**}
\]
\medskip

\noindent
Notice that we do not impose any growth condition on
$\{a_n\}$ above, i.e. the sole assumption
is the existence of the   limit (*). 
\medskip

\noindent
{\bf{Remark.}} The proofs of Abel's and Tauber's results are rather  easy
while C and D require more effort.




\bigskip


\centerline{\bf{1. Proof of Abel's theorem.}}

\medskip

\noindent
Without loss of generality we can assume that
$a_0=0$ and set  $S_N=a_1+\ldots+a_N$.
Given $0<r<1$ we let $f(r)=\sum\, a_nr^n$. For every positive integer $N$
the triangle inequality gives:
\[
\bigl| S_N-f(r)\bigr|\leq
\sum_{n=1}^{n=N}\, |a_n|(1-r^n)+
\sum_{n\geq N+1}\, |a_n|r^n
\]
Set $\delta(N)=\max_{n\geq N}\,\frac{|a_n|}{n}$.
Since  $1-r^n)=(1-r)(1+\ldots+r^{n-1}\leq (1-r)n$
the last sum is majorised by
\[
(1-r)\cdot \sum_{n=1}^{n=N}\, n\cdot |a_n|
+
\delta(N+1)\cdot
 \sum_{n\geq N+1}\, \frac{r^n}{n}
\]
Next, the obvious inequality 
$\sum_{n\geq N+1}\, \frac{r^n}{n}\leq\frac{1}{N+1}\cdot \frac{1}{1-r}$
gives the new majorisation
\[
(1-r)\cdot \sum_{n=1}^{n=N}\, \frac{|a_n|}{n}
+\frac{\delta(N+1)}{N+1}\cdot  \frac{1}{1-r}\tag{1}
\]
This hold for all pairs $N$ and $r$.
To each $N\geq 2$ we take $r=1-\frac{1}{N}$
and hence the right hand side in (1)
is majorised by

\[
\frac{1}{N}\cdot \sum_{n=1}^{n=N}\, \frac{|a_n|}{n}
+\delta(N+1)\cdot \frac{N}{N+1}
\]
Here both terms tend to zero as $N\to\infty$. Indeed, Abel's condition 
$\frac{a_n}{n}\to 0$   implies that
$\frac{1}{N}\cdot \sum_{n=1}^{n=N}\, \frac{|a_n|}{n}$ 
tends to zero as $N\to \infty$. Hence
we have proved the limit formula:
\[ 
\lim_{N\to\infty}\,\bigl |s_N-f(1-\frac{1}{N})\bigr|=0\tag{*}
\]
Finally it is clear that (*)  gives Abel's result.

\bigskip

\centerline{\bf{3. Proof of Tauber's theorem.}}
\medskip

\noindent
We may assume that $a_0=0$. Notice that
\[
a_n=\frac{\omega_n-\omega_{n-1}}{n}\quad\colon\, n\geq 1
\]
It follows that
\[ 
f(r)=\sum\, \frac{\omega_n-\omega_{n-1}}{n}\cdot r^n
=\sum\,\omega_n\bigl(\frac{r^n}{n}-\frac{r^{n+1}}{n+1}\bigr )
\]
Using  the equality
$\frac{1}{n}=
\frac{1}{n+1}=
\frac{1}{n(n+1)}$ we can  rewrite the right hand side as follows:
\[
\sum\,\omega_n\bigl(\frac{r^n-r^{n+1}}{n+1}+\frac{r^n}{n(n+1)}\bigr )
\]
Set
\[ g_1(r)=\sum\,\omega_n\cdot\frac{r^n-r^{n+1}}{n+1}
=(1-r)\cdot \sum\, \frac{\omega_n}{n+1}\cdot r^n
\]
By the hypothesis 
$\lim_{n\to\infty}\,  \frac{\omega_n}{n+1}=0$ and then it is
clear that we get
\[
\lim_{r\to 1}\, g_1(r)=0
\]
Since we also have $f(r)\to 0$ as $r\to 1$ we conclude that
\[
\lim_{r\to 1} \sum\, \frac{\omega_n}{n(n+1)}\cdot r^n=0\tag{1}
\]
Next, with
$b_n= \frac{\omega_n}{n(n+1)}$ we have
$nb_n= \frac{\omega_n}{n+1}\to 0$.
Hence Abel's theorem applies so (1) gives  convergent series
\[
\sum\, \frac{\omega_n}{n(n+1)}=0\tag{2}
\]
If $N\geq 1$ we have the partial sum
\[
S_N=\sum_{n=1}^{n=N} 
\, \frac{\omega_n}{n(n+1)}=
\sum_{n=1}^{n=N}, \omega_n\cdot\bigl(\frac{1}{n}-\frac{1}{n+1}\bigr)
\]
The last term becomes
\[
\sum_{n=1}^{n=N}\,\frac{1}{n}(\omega_n-\omega_{n-1})-
\frac{\omega_N}{N+1}=
\sum_{n=1}^{n=N}\,a_n-\frac{\omega_N}{N+1}
\]
Again, since
$\frac{\omega_N}{N+1}\to 0$ as $N\to\infty$ we conclude that the convergent series
from (2) implies that the series
$\sum\, a_n$ also is converges and has sum equal to zero.
This finishes the proof of Tauber's result.
\bigskip


we need some results from calculus in one variable.
So before we enter the proofs of the  theorems above insert
some  preliminaries.
\medskip

\centerline{\bf{3. Results from calculus}}

\medskip

\noindent
To prove the theorems by Hardy sand Littlewood
we need some results from calculus in one variable.
So before we enter the proofs of Theorem C and D
we insert
some  preliminaries.
Below $g(x)$ is a real-valued function defined on $(0,1)$ and
of class $C^2$ at least.
\medskip


\noindent
{\bf 3.1 Lemma } \emph{Assume that there exists a constant $C>0$ such that}
\[
g''(x)\leq C(1-x)^{-2}\quad\colon\, 0<x<1\quad\text{and}\quad
\lim_{x\to 1}\, g(x)=0
\] 
\emph{Then one  has the limit formula}:
\[
\lim_{x\to 1}\, (1-x)\cdot g'(x)=0
\]

\medskip

\noindent
{\bf 3.2 Lemma } \emph{Assume that the second order derivative
$g''(x)>0$.
Then the following implication holds for each $\alpha>0$:}

\[
\lim_{x\to 1}\, (1-x)^\alpha\cdot g(x)=1\implies
\lim_{x\to 1}\, (1-x)^{\alpha+1}\cdot g'(x)=\alpha
\]




\noindent
{\bf{Remark.}}
If $g(x)$ has higher order derivatives
which all are
$>0$ on $(0,1)$
we can iterate the conclusion in Lemma 1.2 where 
we take $\alpha$ to be positive integers.
More precisely, by an induction over
$\nu$ the reader may verify that if
\[
\lim_{x\to 1}\, (1-x)\cdot g(x)=1
\]
exists and  if 
$\{g^{(\nu)}(x)>0\}$ for all 
every $\nu\geq 2$ then 
\[
\lim_{x\to 1}\, (1-x)^{\nu+1}\cdot g^{(\nu)}(x)=\nu\,!
\quad\colon\, \nu\geq 2\tag{*}
\]



\bigskip

\noindent
Next, to each integer $\nu\geq 1$ we denote by $[\nu-\nu^{2/3}]$
the largest integer $\leq\,(\nu-\nu^{2/3}).$ Set
\[
J_*(\nu)=\sum_{n\leq [\nu-\nu^{2/3}]}\,
n^\nu e^{-\nu}
\quad\colon\quad J^*(\nu)=\sum_{n\geq [\nu+\nu^{2/3}]}\,
n^\nu e^{-\nu}
\]

\medskip

\noindent
{\bf 3.3 Lemma }
\emph{There exists a constant $C$
such that}
\[
\frac{J^*(\nu)+J_*(\nu)}{\nu\,!}\leq \delta(\nu)\quad\colon\quad
\delta(\nu)=C\cdot \text{exp}\, \bigl(-\frac{1}{2}\cdot \nu^{\frac{1}{{3}}}\bigr )
\quad\colon\,\nu=1,2,\ldots
\]

\medskip

\centerline{\emph{Proofs}}

\bigskip

\noindent
We prove only  Lemma 1.1 which is a bit tricky 
while the proofs of Lemma 1.2 and 1.3 are
left as  exercises to the reader.
Fix $0<\theta<1$. Let
$0<x<1$ and set
\[
x_1=x+(1-x)\theta
\]
The mean-value theorem in calculus gives
\[ 
g(x_1)-g(x)=\theta(1-x)g'(x)+\frac{\theta^2}{2}(1-x)^2\cdot g''(\xi)\quad
\text{for some}\quad \, x<\xi<x_1\tag{i}
\]
By the hypothesis
\[
g''(\xi)\leq C(1-\xi)^{-2}\leq C)1-x_1)^{-2}
\]
Hence (i) gives
\[
(1-x)g'(x)\geq
\frac{1}{\theta}(g(x_1)-g(x))-
C\cdot \frac{\theta}{2}\frac{(1-x)^2}{1-x_1)^2}=
\]
\[
\frac{1}{\theta}(g(x_1)-g(x))-
\frac{C\cdot \theta}{2(1-\theta)^2}
\]
Keeping $\theta$ fixed we have by assumption
\[
\lim_{x\to 1}\, g(x)=0
\]
Notice also that $x\to 1\implies x_1\to 1$. It follows that



\[
\liminf_{x\to 1}\,\,
(1-x)g'(x)\geq -\frac{C\cdot \theta}{2(1-\theta)^2}
\]
Above  $0<\theta<1$ is arbitrary, i.e. we can choose 
small $\theta>0$ and hence we have proved that
\[
\liminf_{x\to 1}\,
(1-x)g'(x)\geq 0\tag{*}
\]
\medskip

\noindent
Next we  prove the opposed inequality

\[
\limsup_{x\to 1}\,
(1-x)g'(x)\leq 0\tag{**}
\]
To get (**) we apply the mean value theorem in the form
\[
g(x_1)-g(x)=\theta(1-x)g'(x_1)-\frac{\theta^2}{2}(1-x)^2\cdot g''(\eta)\quad
\colon\, x<\eta<x_1\tag{ii}
\]
Since $(1-x_1)=\theta(1-x)(1-\theta)$ we get
\[ 
(1-x_1)g'(x_1)=\frac{1-\theta}{\theta}\cdot (g(x_1)-g(x))+
\frac{(1-\theta)\theta}{2}\cdot(1-x)^2g''(\eta)\tag{iii}
\]
Now $g''(\eta)\leq C(1-\eta)^{-2}\leq C(1-x_1)^{-2}$ so
the right hand side in (iii) is majorized by
\[
\frac{1-\theta}{\theta}\cdot (g(x_1)-g(x))+
C\cdot \frac{(1-\theta)\theta}{2}\cdot(1-x)^2(1-x_1)^2=
\]
 \[
\frac{1-\theta}{\theta}\cdot (g(x_1)-g(x))+
C\cdot \frac{\theta}{2(1-\theta}\tag{iv}
\]
Keeping $\theta$ fixed while $x\to 1$ we obtain:
\[
\liminf_{x\to 1}\, (1-x)g'(x)\leq 
C\cdot \frac{\theta}{2(1-\theta}
\]
Again we can choose arbitrary small $\theta$ and hence (**) 
holds which finishes the proof of 
Lemma 3.1.



\bigskip








\centerline{\bf{4. Proof of Theorem D. }}
\medskip

\noindent
Set $g(x)=\sum\, s_nx^n$ which is defined when $0<x<1$.
Notice that 
\[
(1-x)g(x)=\sum\, a_nx^n
\]
Since $s_n\geq 0$ for all $n$  all the higher order derivatives
\[
g^{(p)}(x)= \sum_{n=p}^\infty\, n(n-1)\cdots (n-p+1)s_nx^{n-p}>0
\]
when $0<x<1$.
The hypothesis (*)
and the inductive result in the
remark after Lemma 1.2 give:
\[
\lim_{x\to 1}\, (1-x)^{\nu+2}\cdot
\sum\, s_n\cdot n^\nu x^n=(\nu+1)!\quad\colon\,\nu\geq 1\tag{1}
\]
We shall use the substitution $e^{-t}=x$ where $t>0$.
Since $t\simeq 1-x$ when
$x\to 1$ we see that
(1) gives
\[
\lim_{t\to 0}\, t^{\nu+2}\cdot 
\sum\, s_n\cdot n^\nu e^{-nt}=(\nu+1)!\quad\colon\,\nu\geq 1\tag{2}
\]
Put
\[ J_*(\nu,t)=\frac{t^{\nu+2}}{(\nu+1)!}\cdot 
\sum_{n=1}^\infty\, s_n\cdot n^\nu e^{-nt}
\]
So (2) gives for each fixed $\nu$
\[
\lim_{t\to 0}\, J_*(\nu,t)=1\tag{3}
\]

\medskip

\noindent
Next, for each pair $\nu\geq 2$ and $0<t<1$ we define the integer
\[ 
N(\nu,t)=\bigl[\frac{\nu-\nu^{2/3}}{t}\bigr]\tag{*}
\]
Since the sequence $\{s_n\}$ is non-decreasing we get
\[
s_{N(\nu,t)}\cdot\sum_{n\geq N(\nu,t)}\, n^\nu e^{-nt}\leq
\sum_{n\geq N(\nu,t)}\, s_n\cdot n^\nu e^{-nt}\leq\frac{(\nu+1)!\cdot J_*(\nu,t)}{t^{\nu+2}}
\tag{i}
\]
Next,  the construction of $N(\nu,t)$ and Lemma 1.3 give:

\[
\sum_{n\geq N(\nu,t)}\, n^\nu e^{-nt}\geq \frac{\nu !}{t^{\nu+1}}\cdot (1-\delta(\nu))\tag{ii}
\]
where the $\delta$ function is independent of $t$ and
tends to zero as $\nu\to\infty$. Hence (i-ii) give
\[
s_{N(\nu,t)}\leq \frac{(\nu+1)}{t}\cdot \frac{1}{1-\delta(\nu)}\cdot
J_*(\nu,t)\tag{iii}
\]
Next, by the construction of $N$ one has
\[ 
N(\nu,t)+1\geq \frac{\nu-\nu^{2/3}}{t}=\frac{\nu}{t}\cdot(1-\nu^{-1/3})
\]
It follows that (iii) gives
\[
\frac{s_{N(\nu,t)}}{N(\nu,t)+1}\leq 
\frac{\nu+1}{\nu}
\cdot\frac{1}{1-\nu^{-1/3}}\cdot\frac{1}{1-\delta(\nu)}\cdot 
J_*(\nu,t)\tag{iv}
\]
Since $\delta(\nu)\to 0$
it follows that for any $\epsilon>0$ there exists some
$\nu_*$ such that
\[
\frac{\nu_*+1}{\nu_*}
\cdot\frac{1}{1-\nu_*^{-1/3}}\cdot\frac{1}{1-\delta(\nu_*)}<1+\epsilon \tag{v}
\]
\medskip

\noindent
Increasing $\nu_*$�if necesssty we also notice that
the construction of $N(\nu_*,t)$ gives
\[
\bigl|\, N(\nu_*,t)-\frac{\nu_*+1}{t}\bigr |<\epsilon
\]
CVhpoisng a fixed  $\nu_*$  as above,we find for each large postive integer
$N$ some $t_N$�such that $N=N(\nu_*,t_N)$, and  notice that
\[
 N\to+\infty\implies t_N\to 0\tag{vi}
\]

\medskip
\noindent
Next, (iv) and (v) yield:
\[
\frac{s_N}{N+1}<(1+\epsilon)\cdot J_*(\nu_*,t_N)\tag{vii}
\]
Now (vi) and the limit in (3) which applies with
$\nu_*$ is kept fixed while while $t_N\to 0$
entail that
\[
\lim_{N\to \infty}\, J(\nu_*,t_N)=1\tag{viii}
\]
At the same time 
$\frac{N}{N+1}\to 1$ and since $\epsilon>0$ was arbitrary
we conclude  from  (vii) that:
\[ 
\limsup_{N\to\infty}\, 
\frac{s_N}{N}\leq 1\tag{4}
\]
So Theorem 2 follows if we also prove that
\[
\liminf_{N\to\infty}\, 
\frac{s_N}{N}\geq 1\tag{5}
\]
The proof of (II) is similar where we now define  the integers
\[ 
N(\nu,t)=\bigl[\frac{\nu+\nu^{2/3}}{t}\bigr]
\]
Then 
\[ 
S_{N(\nu,t)}\cdot\sum_{n\leq N(\nu,t)}\, n^\nu e^{-nt}\geq
\frac{(\nu+1)!\cdot J_*(\nu,t)}{t^{\nu+2}}-
\sum_{n> N(\nu,t)}\,s_n\cdot n^\nu e^{-nt}
\]
Here the last term can be estimated  since the Lim.sup
inequality (4) gives a constant $C$ such that
$s_n\leq Cn$ for all $n$ and then
\[
\sum_{n> N(\nu,t)}\,s_n\cdot n^\nu e^{-nt}
\leq C\cdot \sum_{n> N(\nu,t)}\,n^{\nu+1} e^{-nt}
\leq C\cdot \delta(\nu)\cdot 
\frac{(\nu+1)!}{t^{\nu+2}}
\]
where Lemma 1.3  entails that
$\delta(\nu)\to 0$ as $\nu$ increases. At the same time Lemma 1.3 also gives
\[
\sum_{n\leq N(\nu,t)}\, n^\nu \cdot e^{-nt}=
\frac{\nu !}{t^{\nu+1}}\cdot (1-\delta_*(\nu)
\] 
where $\delta_*(\nu)\to 0$ as $\nu\to +\infty$.
Given $\epsilon>0$ we choose $\nu_*$ large so that
$C\cdot \delta(\nu_*)<\epsilon$�and $\delta_*(\nu_*)<\epsilon$.
Increasing $\nu_*$�if necesssty we also notice that
the construction of $N(\nu_*,t)$ gives
\[
\bigl|\, N(\nu_*,t)-\frac{\nu_*+1}{t}\bigr |<\epsilon
\]
Choosing a fixed  $\nu_*$  as above,we find for each large postive integer
$N$ some $t_N$�such that $N=N(\nu_*,t_N)$, and  notice that
Keeping $\nu_*$ fixed we conclude that
\[
S_{(N(\nu_*,t)}\cdot \frac{\nu_* !}{t^{\nu_*+1}} \cdot (1-\epsilon)
\geq \frac{(\nu_*+1)!}{t^{\nu_*+2}}\cdot [
J_*(\nu_*,t)-\epsilon]\implies
\]
\[
S_{(N(\nu_*,t)} \cdot (1-\epsilon)
\geq \frac{(\nu_*+1)}{t}\cdot [
J_*(\nu_*,t)-\epsilon]
\]
For large integers $N$ we find $t_N$ so that
$N(\nu_*,t_N)=N$ and we notice that
the construction of $N(\nu_*,t)$

\newpage



\centerline{\bf{4. Proof of Theorem D. }}
\medskip

\noindent
Set $g(x)=\sum\, s_nx^n$ which is defined when $0<x<1$.
Notice that 
\[
(1-x)g(x)=\sum\, a_nx^n
\]
Since $s_n\geq 0$ for all $n$  all the higher order derivatives
\[
g^{(p)}(x)= \sum_{n=p}^\infty\, n(n-1)\cdots (n-p+1)s_nx^{n-p}>0
\]
when $0<x<1$.
The hypothesis (*)
and the inductive result in the
remark after Lemma 1.2 give:
\[
\lim_{x\to 1}\, (1-x)^{\nu+2}\cdot
\sum\, s_n\cdot n^\nu x^n=(\nu+1)!\quad\colon\,\nu\geq 1\tag{1}
\]
We shall use the substitution $e^{-t}=x$ where $t>0$.
Since $t\simeq 1-x$ when
$x\to 1$ we see that
(1) gives
\[
\lim_{t\to 0}\, t^{\nu+2}\cdot 
\sum\, s_n\cdot n^\nu e^{-nt}=(\nu+1)!\quad\colon\,\nu\geq 1\tag{2}
\]
Put
\[ J_*(\nu,t)=\frac{t^{\nu+2}}{(\nu+1)!}\cdot 
\sum_{n=1}^\infty\, s_n\cdot n^\nu e^{-nt}
\]
So (2) gives for each fixed $\nu$
\[
\lim_{t\to 0}\, J_*(\nu,t)=1\tag{3}
\]

\medskip

\noindent
Next, for each pair $\nu\geq 2$ and $0<t<1$ we define the integer
\[ 
N(\nu,t)=\bigl[\frac{\nu-\nu^{2/3}}{t}\bigr]\tag{*}
\]
Since the sequence $\{s_n\}$ is non-decreasing we get
\[
s_{N(\nu,t)}\cdot\sum_{n\geq N(\nu,t)}\, n^\nu e^{-nt}\leq
\sum_{n\geq N(\nu,t)}\, s_n\cdot n^\nu e^{-nt}\leq\frac{(\nu+1)!\cdot J_*(\nu,t)}{t^{\nu+2}}
\tag{i}
\]
Next,  the construction of $N(\nu,t)$ and Lemma 1.3 give:

\[
\sum_{n\geq N(\nu,t)}\, n^\nu e^{-nt}\geq \frac{\nu !}{t^{\nu+1}}\cdot (1-\delta(\nu))\tag{ii}
\]
where the $\delta$ function is independent of $t$ and
tends to zero as $\nu\to\infty$. Hence (i-ii) give
\[
s_{N(\nu,t)}\leq \frac{(\nu+1)}{t}\cdot \frac{1}{1-\delta(\nu)}\cdot
J_*(\nu,t)\tag{iii}
\]
Next, by the construction of $N$ one has
\[ 
N(\nu,t)+1\geq \frac{\nu-\nu^{2/3}}{t}=\frac{\nu}{t}\cdot(1-\nu^{-1/3})
\]
It follows that (iii) gives
\[
\frac{s_{N(\nu,t)}}{N(\nu,t)+1}\leq 
\frac{\nu+1}{\nu}
\cdot\frac{1}{1-\nu^{-1/3}}\cdot\frac{1}{1-\delta(\nu)}\cdot 
J_*(\nu,t)\tag{iv}
\]
Since $\delta(\nu)\to 0$
it follows that for any $\epsilon>0$ there exists some
$\nu_*$ such that
\[
\frac{\nu_*+1}{\nu_*}
\cdot\frac{1}{1-\nu_*^{-1/3}}\cdot\frac{1}{1-\delta(\nu_*)}<1+\epsilon \tag{v}
\]
\medskip

\noindent
Increasing $\nu_*$�if necesssty we also notice that
the construction of $N(\nu_*,t)$ gives
\[
\bigl|\, N(\nu_*,t)-\frac{\nu_*+1}{t}\bigr |<\epsilon
\]
CVhpoisng a fixed  $\nu_*$  as above,we find for each large postive integer
$N$ some $t_N$�such that $N=N(\nu_*,t_N)$, and  notice that
\[
 N\to+\infty\implies t_N\to 0\tag{vi}
\]

\medskip
\noindent
Next, (iv) and (v) yield:
\[
\frac{s_N}{N+1}<(1+\epsilon)\cdot J_*(\nu_*,t_N)\tag{vii}
\]
Now (vi) and the limit in (3) which applies with
$\nu_*$ is kept fixed while while $t_N\to 0$
entail that
\[
\lim_{N\to \infty}\, J(\nu_*,t_N)=1\tag{viii}
\]
At the same time 
$\frac{N}{N+1}\to 1$ and since $\epsilon>0$ was arbitrary
we conclude  from  (vii) that:
\[ 
\limsup_{N\to\infty}\, 
\frac{s_N}{N}\leq 1\tag{4}
\]
So Theorem 2 follows if we also prove that
\[
\liminf_{N\to\infty}\, 
\frac{s_N}{N}\geq 1\tag{5}
\]
The proof of (II) is similar where we now define  the integers
\[ 
N(\nu,t)=\bigl[\frac{\nu+\nu^{2/3}}{t}\bigr]
\]
Then 
\[ 
S_{N(\nu,t)}\cdot\sum_{n\leq N(\nu,t)}\, n^\nu e^{-nt}\geq
\frac{(\nu+1)!\cdot J_*(\nu,t)}{t^{\nu+2}}-
\sum_{n> N(\nu,t)}\,s_n\cdot n^\nu e^{-nt}
\]
Here the last term can be estimated  since the Lim.sup
inequality (4) gives a constant $C$ such that
$s_n\leq Cn$ for all $n$ and then
\[
\sum_{n> N(\nu,t)}\,s_n\cdot n^\nu e^{-nt}
\leq C\cdot \sum_{n> N(\nu,t)}\,n^{\nu+1} e^{-nt}
\leq C\cdot \delta(\nu)\cdot 
\frac{(\nu+1)!}{t^{\nu+2}}
\]
where Lemma 1.3  entails that
$\delta(\nu)\to 0$ as $\nu$ increases. At the same time Lemma 1.3 also gives
\[
\sum_{n\leq N(\nu,t)}\, n^\nu \cdot e^{-nt}=
\frac{\nu !}{t^{\nu+1}}\cdot (1-\delta_*(\nu)
\] 
where $\delta_*(\nu)\to 0$ as $\nu\to +\infty$.
Given $\epsilon>0$ we choose $\nu_*$ large so that
$C\cdot \delta(\nu_*)<\epsilon$�and $\delta_*(\nu_*)<\epsilon$.
Increasing $\nu_*$�if necesssty we also notice that
the construction of $N(\nu_*,t)$ gives
\[
\bigl|\, N(\nu_*,t)-\frac{\nu_*+1}{t}\bigr |<\epsilon
\]
Choosing a fixed  $\nu_*$  as above,we find for each large postive integer
$N$ some $t_N$�such that $N=N(\nu_*,t_N)$, and  notice that
Keeping $\nu_*$ fixed we conclude that
\[
S_{(N(\nu_*,t)}\cdot \frac{\nu_* !}{t^{\nu_*+1}} \cdot (1-\epsilon)
\geq \frac{(\nu_*+1)!}{t^{\nu_*+2}}\cdot [
J_*(\nu_*,t)-\epsilon]\implies
\]
\[
S_{(N(\nu_*,t)} \cdot (1-\epsilon)
\geq \frac{(\nu_*+1)}{t}\cdot [
J_*(\nu_*,t)-\epsilon]
\]
For large integers $N$ we find $t_N$ so that
$N(\nu_*,t_N)=N$ and we notice that
the construction of $N(\nu_*,t)$





\newpage




\bigskip
\centerline{\bf{5. Proof of Theorem C}}
\medskip

\noindent
Set $f(x)=\sum\, a_nx^n$. Notice that it suffices to prove
Theorem C when
the limit value
\[
\lim_{x\to 1}\, \sum\,a_nx^n=0
\]
Next,
the assumption that $a_n\leq\frac{c}{n}$ for a constant $c$ gives
\[
f''(x)=\sum\, n(n-1)a_nx^{n-2}\leq c\sum\, (n-1)x^{n-2}=\frac{c}{1-x)^2}
\]
The hypothesis $\lim_{x\to 1}\, f(x)=0$ and Lemma xx therefore gives
\[
\lim_{x\to 1}\, (1-x)f'(x)=0\tag{i}
\]
Next, notice the equality
\[
\sum_{n=1}^\infty\, \frac{na_n}{c} x^n=
\frac{x}{c}\cdot f'(x)\tag{ii}
\]
At the same time
$\sum_{n=1}^\infty\,x^n=\frac{x}{1-x}$
and hence (i-ii)  together give:

\[
\lim_{x\to 1}\, (1-x)\cdot\sum\, (1-\frac{na_n}{c} )\cdot x^n=1
\]
Here $1-\frac{na-n}{c}\geq 0$ so Theorem 2 gives
\[
\lim_{N\to \infty}\,
\frac{1}{N} \sum_{n=1}^{n=N}\, (1-\frac{na_n}{c})=1
\]
It follows that
\[
\lim_{N\to \infty}\,\frac{1}{N} \cdot \sum_{n=1}^{n=N}\, na_n=0
\]
This means precisely that the condition in Tauber's Theorem holds and hence 
$\sum\, a_n$ converges and has series sum
equal to 0 which finishes the proof of Theorem C.









\newpage




\medskip

\centerline{\bf VII. An example by Hardy}

\medskip

\noindent
Consider the  series expansion of
\[
(1-z)^{\alpha}=\sum\, b_nz^n
\]
where $\alpha$ in general is a complex number.
Newton's binomial formula gives:
\[ 
b_n=\frac{\alpha(\alpha+1)\ldots (\alpha+n-1)}{ n\,!}\tag{*}
\]
Apply this with
$\alpha=i$. If $n\geq 1$ which entails that
\[
|n\cdot b_n|=\frac{|(i+1)|\ldots |i+n-1|}{ n\,!}=
\sqrt{\bigl(1+\frac{1}{1^2})\cdot (1+\frac{1}{2^2})\cdots
(1+\frac{1}{(n-1)^2}}
\]
It follows that
\[
\lim_{n\to\infty}\, 
|n\cdot b_n|=
\sqrt{\,\prod_{\nu=1}^\infty\, \bigl(1+\frac{1}{\nu^2})}
\]
\medskip

\noindent
In particular $|b_n|\simeq\frac{1}{n}$ when $n$ is large and therefore
the series
\[ 
\sum_{n=2}^\infty\,\frac{|b_n|}{\text{Log}\,n}=+\infty\tag{i}
\]


\noindent
In spite of the divergence above one has:

\medskip

\noindent
{\bf Theorem 7.1.} \emph{The  series}
\[ 
\sum_{n=2}^\infty\, \frac{b_n}{\text{Log}\,n}\cdot e^{in\phi}
\]
\emph{converges uniformly when
$0\leq\phi\leq 2\pi$.}
\medskip

\noindent
Before we give the proof of Theorem 7.1 we
need a result if independent interest.
\medskip

\noindent
{\bf{7.2. Theorem.}}
\emph{Let $\{a_n\}$ be a sequence of complex numbers
such that $|a_n|\leq \frac{C}{n}$ hold for all $n$ and some constant $C$
and the analytic function
$f(z)=\sum\, a_nz^n$ is bounded in $D$, i.e. }
\[
 |f(z)|\leq M\quad \colon z\in D
\]
\emph{Then, if $B_n(z)=a_0+ a_1+\ldots+a_nz^n$
are the partial sums one has the inequality}
\[
\max_\theta\,  |B_m(e^{i\theta})|\leq M+2C
\quad\text{for all}\quad m=1,2,\ldots\]


\noindent
\emph{Proof.}
When $0<r<1$ and $\theta$ are given we have
\[
\bigl|B_m(\theta)-f(re^{i\theta})\bigr|=
\sum_{n=0}^{n=m}\, a_ne^{in\theta}(1-r^n)-
\sum_{n=m+1}^\infty \, a_ne^{in\theta}\cdot r^n\leq
\]
\[
(1-r)\sum_{n=0}^{n=m}\, n\cdot|a_n|+
\sum_{n=m+1}^\infty \, |a_n|\cdot r^n
\]
\medskip

\noindent
With $m$ given we apply this when $r=1-1/m$.
Then the last sum above is estimated above by
\[
\frac{1}{m}\cdot \sum_{n=1}^{n=m}\, n\cdot\,|a_n|+
\frac{c}{m}\cdot \sum_{n=m+1}^\infty \,  r^n\leq
C+\frac{c}{m}\cdot \sum_{n=0}^\infty \,  r^n=2C\tag{*}
\]
Finally, since the maximum norm of $f$ is $\leq M$
the triangle inequality gives
\[
 |B_m(e^{i\theta})|\leq 2C+M
\] 
Here $m$ and $\theta$ are arbitrary so Theorem 7.2 follows.









\newpage



\noindent
\emph{Proof of Theorem 7.1}
To
each $m\geq 2$ we consider the partial sum series
\[
S_m(\phi)=
\sum_{n=2}^{n=m}\, \frac{b_n}{\text{Log}\,n}\cdot e^{in\phi}
\]
Theorem 7.1 follows if there  to every $\epsilon>0$  exists
an integer $M$ such that
\[ 
\max_{0\leq\phi\leq2\pi}\,
|S_m(\phi)-S_M(\phi)|<\epsilon\quad\colon\quad
\forall\,\, m>M\tag{1}
\]


\noindent
To prove (1) we employ the partial sums
\[ 
B_n(\phi) =\sum_{\nu=1}^{\nu=n}\, b_\nu\cdot e^{i\nu\phi}\tag{2}
\]


\noindent For each pair $m>M\geq 2$, the  partial 
summation formula in gives
\[
S_m(\phi)-S_M(\phi)=
\]
\[
\sum_{n=M}^{n=m}\, B_n(\phi)\cdot\bigl[\,\frac{1}{\text{Log}\, n}-
\frac{1}{\text{Log}\, (n+1)}\bigr]-
\frac{B_{M-1}(\phi)}{\text{Log}\,M}+
\frac{B_m(\phi)}{\text{Log}\,(m+1)}\tag{3}
\]
\medskip

\noindent
Now we can apply Theorem 7.2 and find a constant $K$ such that
\[
|B_n(\phi)|\leq K\quad>\colon\quad n\geq 2\,\quad\text{and}\quad
0\leq\phi\leq 2\pi\tag{4}
\]


\noindent
Notice  that
if (4) holds then (3) gives the inequality
\[
\bigl|S_m(\phi)-S_M(\phi)\bigr|\leq
K\cdot 
\sum_{n=M}^{n=m}\,\bigl[\frac{1}{\text{Log}\, n}-
\frac{1}{\text{Log}\, (n+1)}\bigr]+
\frac{1}{\text{Log}\,M}+
\frac{1}{\text{Log}\,(m+1)}=\frac{2K}{\text{Log}\, M}
\]
\medskip

\noindent
Hence  Theorem 7.1 is proved if we
establish the inequality (4).
To prove this we  study
the  analytic function defined in the open
unit disc $D$  by
the convergent power series:
\[
(1-z)^i=\sum\, c_n\cdot z^n\tag{*}
\]
Since $\mathfrak{Re}(1-z)>0$ when
$|z|<1$ there exists
a single valued branch of $\log(1-z)$ and
the function above can be written
as
\[
g(z)=e^{i\cdot\log\,(1-z)}
\]
Now the argument of
$\log(1-z)$ stays in $(-\pi/2,\pi/2$
and we conclude that
\[ 
|g(z)|\leq e^{\pi/2}\quad\colon z\in D
\]
Hence the $g$-function is bounded in $D$.
Now
\[ g(z)=\sum\, b_nz^n
\] 
and  Newton's binomial formula gives:

\[ 
|b_n|\leq \frac{C}{n}\quad \colon n\geq 1
\]
Then it is clear that Theorem 7.2 applied  to $g$ gives (4) above.
\


\newpage




\medskip

\centerline{\bf {8. Convergence under substitution.}}
\bigskip

\noindent
{\bf{Introduction.}}
Let $\{a_k\}$ be a sequence of complex numbers where
$\sum\, a_k$ is convergent. This gives an analytic function
$f(z)$ defined in the open disc by
\[
f(z)=\sum\, a\uuu n\cdot z^n\tag{1}
\]
If $0<s <1$
we can expand $f$ around $s$ and obtain another series
\[
f(s+z)=\sum\, c\uuu n\cdot z^n\tag{2}
\]
From the   convergence of $\sum\, a\uuu k$
one  expects 
that the series
\[
\sum\, c\uuu n\cdot (1\vvv s)^n\tag{3}
\]
also is convergent. This is indeed true and was proved by Hardy and Littlewood
in (H\vvv L]. A more general result was established in [Carleman]
and we are going to expose results from Carleman's article.
Let $f(z)$ be give as in (1)
and consider another
power series
\[ 
\phi(z)=\sum\, b_\nu\cdot z^\nu\tag{1}
\]
which 
represents an analytic function
$D$ where
$|\phi(z)|<1$ hold when $|z|<1$. Then there exists  the
composed analytic function 
\[ 
f(\phi(z))= \sum_{k=0}^\infty\ c_k\cdot z^k\tag{*}
\]
We seek conditions on $\phi$ in order that  the convergence of
$\{a\uuu k\}$ entails that
the series
\[
 \sum\, c\uuu k\quad\text{also converges}\tag{**}
\] 

\medskip

\noindent
First we consider the special case when
the $b$\vvv coefficients are real and non\vvv negative.
\medskip


\noindent
{\bf{8.1. Theorem.}} \emph{Assume that $\{b\uuu \nu\geq 0\}$
and that $\sum\, b\uuu \nu=1$. Then  (**) converges and the sum is equal to 
$\sum\, a\uuu k$.}



\bigskip


\noindent
\emph{Proof.}
Since  $\{b\uuu\nu\}$ are real and non\vvv negative
the Taylor series for $\phi^k$ also has non\vvv negative
real coefficients for every  $k\geq 2$. Put
\[ 
\phi^k(z)=\sum \, B\uuu{k\nu}\cdot z^ \nu
\]
and for  each pair of integers $k,p$ we set
\[
\Omega\uuu{k,p}=\sum\uuu{\nu=0}^{\nu=p} \, B\uuu{k\nu}\
\]
By  assumption on $\{b\uuu\nu\}$ we have $\phi81)=1$
which entials that
$\phi^k(1)=01$�for every $k\geq 2$ and hence
\[
\sum\uuu {\nu=0}^\infty\, B\uuu{k\nu}=1
\quad\colon\,\, k\geq 1\tag{i}
\]
\medskip

\noindent
{\bf{Sublemma.}} \emph{The following hold for every fixed $p\geq 0$}

\[ 
\lim_{N\to\infty}\, \Omega_{N,p}=0\quad\text{and}\quad
k\mapsto \Omega_{k,p}
\quad\text{decreases}\tag{ii}
\]
The proofs of (ii-iii) are left as excercises to the reader.
\medskip

\noindent
Next, the Taylor series of the composed analytic function 
$f(\phi(z))$ becomes
\[
\sum a\uuu k\cdot \phi^k(z)
=\sum\uuu{\nu=0}^\infty\,\bigl[\sum\uuu {k=0}^\infty\, a\uuu k\cdot B\uuu{k\nu}\bigr ]\cdot z^\nu
\]

\noindent
For  each positive integer $n^*$ we set
\[ 
\sigma_p[n^*]=
\sum_{\nu=0}^{\nu=p}\, \,\bigl[\,\sum_{k=0}^{k=n^*}\, a_k\cdot B_{k,\nu}\bigr]\tag{1}
\]
\[
\sigma_p(n^*)= \sum_{\nu=0}^{\nu=p}\, \,\bigl[\,\sum_{k=n^*+1}^\infty\, a_k\cdot B_{k,\nu}\,\bigr]=\sum\uuu{k=n^*+1}^\infty\,a\uuu k \cdot \Omega\uuu{k,p}
\tag{2}
\]
Notice that 
\[
\sigma_p[n^*]+
\sigma_p(n^*)=\sum\uuu {k=0}^{k=p}c\uuu k\quad\text{hold for each}\quad p
\]
Our aim is to show that the last partial sums have a 
limit. To obtain  this we
study the $\sigma$\vvv terms   separately. Introduce the partial sums
\[
s\uuu n=\sum\uuu{k=0}^{k=n}\, a\uuu k
\]
By assumption  there exists a limit $s\uuu n\to S$ which  entails that
the sequence $\{s\uuu k\}$ is bounded and so is
the sequence $\{a\uuu k=s\uuu k\vvv s\uuu{k\vvv 1}\}$.
The first limit formula in the Sublemma entails
that
the last term in (2) tends to zero when $n^*$ increases. So if $\epsilon>0$
we find
$n^*$ such that
\[
n\geq n^*\implies |\sigma\uuu p(n)|
\leq \epsilon\tag{3}
\]



\noindent

\noindent
\emph{A study of $\sigma_p[n^*]$}.
Keeping $n^*$ and $\epsilon$ fixed we apply (i) for each $0\leq k\leq n^*$
and find an integer $p^*$ such that
\[
1- \sum_{\nu=0}^{\nu=p} B_{k,\nu}\leq \frac{\epsilon}{n^*+1}
\quad\text {for all pairs }\quad p\geq p^*\,\colon\, 0\leq k\leq n^*\tag{4}
\]

\noindent
The   triangle inequality gives
\[
|\sigma_p(n^*)-s_{n^*}|\leq \frac{\epsilon}{n^*+1}\cdot \sum\uuu{k=0}^{k=n^*}\, |a\uuu k|
\quad\text {for  all  }\quad p\geq p^*\tag{5}
\]

\noindent
Since  $\sum\, a_k$ converges the sequence
$\{a\uuu k\}$ is bounded, i.e. we have
a constant $M$ such that
$|a\uuu k|\leq M$ for all $k$.
Hence (4) gives
\[
|\sigma_p[n^*]-S|\leq |s_{n^*}-S|+\epsilon\cdot M
\quad\colon\quad p\geq p^*\tag{6}
\]
Together with (5) this entails that
\[
n\geq n^*\implies |\sum\uuu{k=0}^{n^*}\,  |c_k-S|\leq 
\epsilon+|s_{n^*}-S|+ M\cdot \epsilon
\]
Above  $\epsilon$ is arbitrary small and since $n^*$ csn be chosen large
while $s_{n^*}\to S$,  
we conclude that 
$\sum c\uuu k$ converges and the limit is equal to $S$. This 
finishes the proof of Theorem 8.1.



\bigskip

\centerline{\bf{Another result.}}




\bigskip

\noindent
Now we  relax the condition that $\{b\uuu\nu\}$ are real and nonnegative
but  impose extra conditions on $\phi$.
First we assume that $\phi(z)$ extends to a continuous function on
the closed disc, i.e. $\phi$ belongs to the disc\vvv algebra.
Moreover, 
$\phi(1)=1$ while $|\phi(z)|<1$ 
for all $z\in\bar D\setminus \{1\}$ which means that
$z=1$ is a peak point for $\phi$.
Consider   also  the function
$\theta\mapsto \phi(e^{i\theta})$ where $\theta$ is close to zero.
The final condition on $\phi$ is
that there exists some positive real number
$\beta$ and a constant $C$ such that
\[ 
|\phi(e^{i\theta})\vvv 1\vvv i\beta|\leq C\cdot \theta^2\tag{1}
\]
holds in some interval $\vvv \ell\leq \theta\geq\ell$.
This gives for every integer $n\geq 2$
another constant $C\uuu n$ so that
\[ 
|\phi^n(e^{i\theta})\vvv 1\vvv in\beta|\leq C\uuu n\cdot \theta^2\tag{2}
\]
Hence the following integrals exist for all pairs
of integers $p\geq 0$ and $n\geq 1$:
\[
J(n.p)= \int\uuu{\vvv\ell}^\ell\,
\frac{\phi(e^{i\theta})^n\cdot (1\vvv \phi(e^{i\theta})}{
e^{ip\theta}\cdot (1\vvv e^{i\theta})}\cdot d\theta\tag{3}
\]
With these notations one has
\medskip

\noindent{\bf{8.2. Theorem.}}
\emph{Let $\phi$ satisfy the conditions above.
Then, if 
there exists a constant $C$ such that}
\[ 
\sum\uuu {k=0}^\infty\, |J(k,p)|\leq C\quad\text{for all}\quad p\geq 0\tag{*}
\] 
\emph{it follows that
the series (**) from the introduction
converges  and the   sum is equal to $\sum\, a\uuu k$.}
\bigskip







\noindent
{\emph {Proof}
With similar   notations as in  the  proof  of Theorem 8.1
we introduce the $\Omega$-numbers by:
\[
\Omega\uuu{k,p}=\sum\uuu{\nu=0}^{\nu=k}\, B\uuu{k\nu}
\]
Repeating the proof of Theorem 8.1 the reader may
verify that
the series $\sum c\uuu k$ converges and has the limit $S$
if the following two conditions hold:
\[
\lim\uuu{N\to\infty}\, \Omega\uuu{N,p}=0
\quad \text{holds for every} \quad p\tag{i}
\]  
\[
\sum \uuu{k=0}^{\infty}\, 
\bigl|\Omega\uuu{k+1,p}\vvv \Omega\uuu{k,p}\bigr|\leq C
\quad\text{for  a constant}\quad C
\tag{ii}
\] 
where $C$ is
is independent of $p$.
Here (i) is clear since 
$\{g\uuu N(z)= \phi^N(z)\}$ converge uniformly to zero in
compact subsets of the unit disc and therefore
their Taylor coefficients  tend to zero with $N$.
\medskip

\noindent
\emph{Proof of (ii)}. Residue calculus  gives:
\[
\Omega_{k+1,p}-\Omega_{k,p}=
\frac{1}{2\pi i}
\int_{|z|=1}\frac{\phi^k(z)}{z^{p+1}}\cdot \frac{1-\phi(z)}{1-z}\cdot dz\tag{iii}
\]
\medskip

\noindent
Let $\ell$ be a small positive number
and    $T_\ell$ denotes  the portion of the
unit circle where
$\ell \leq \theta\leq 2\pi\vvv \ell$. 
Since 1 is a peak \vvv point for $\phi$ there exists
some $\mu<1$ such that
\[ 
\max\uuu{z\in T\uuu\ell}\,|\phi(z)|\leq \mu
\]
This gives
\[
\frac{1}{2\pi}\cdot \bigl|\int_{z\in T\uuu\ell }\frac{\phi^k(z)}{z^{p+1}}\cdot \frac{1-\phi(z)}{1-z}\cdot dz
\bigl|\leq \mu^k\cdot \frac{2}{|e^{i\ell}\vvv 1}\bigr |\tag{iv}
\]
Since the geometric series $\sum\,\mu^k$ converges
it follows from (iii) and the construction of the $J\uuu\ell$\vvv functions in
Theorem 8.2 that (ii) is valid  precisely when 
\[ 
\sum\uuu{k=0}^\infty\, |J\uuu \ell (k,p]|\leq C
\]
hold for a constant which is independent of $p$.
But this holds by the hypothesis  (*)  and finishes
the proof of Theorem 8.2.

\newpage

\noindent
{\bf{8.3. An oscillatory integrals.}}
Condition  (*) Theorem 8.2 is   implicit. 
A sufficient condition  that
the $J$\vvv integrals satisfy (*) can be expressed by
conditions on the 
$\phi$\vvv function close to
$z=1$. To begin with the condition   (1) above Theorem 8.2
entails that 
\[ 
\phi(e^{i\theta})= e^{i\beta \theta+\rho(\theta)}\tag{i}
\]
holds in a neighborhood of $\theta=0$ 
where the $\rho$\vvv function behaves like big ordo of $\theta^2$ when $\theta\to 0$.
The next result gives  the requested
convergence of the composed series
expressed  by an additional   condition on the $\rho$\vvv function 
in (i) above.

\bigskip

\noindent{\bf{8.4. Theorem.}}
\emph{Assume that $\rho(\theta)$ is a $C^2$-function on some interval 
$\vvv\ell <\theta<\ell$ and that the second derivative
$\rho''(0)$ is real and negative.
Then (*) in Theorem 8.2 holds.}
\bigskip


\noindent
The proof is left as  an exercise to the reader. If necessary, consult
Carleman's article [Car] which contains  a detailed proof.






\newpage

\centerline{\bf IX. The series $\sum\, \bigl [a_1\cdots a_\nu\bigr ]^{\frac{1}{\nu}}$}
\bigskip

\noindent
We shall prove a result from [Carleman:xx. Note V page 112-115].
Let $\{a_\nu\}$ be a sequence of positive real numbers
such that $\sum\, a_\nu<\infty$ and 
$e$ denotes Neper's constant. 

\medskip

\noindent
{\bf 9.1 Theorem.} \emph{Assume that the series $\sum\, a_\nu$  is 
convergent and let $S$ be the sum.
Then one has the strict inequality}
\[
\sum_{\nu=1}^\infty\, 
\bigl [a_1\cdots a_\nu\bigr ]^{\frac{1}{\nu}}<e\cdot S\tag{*}
\]



\noindent
{\bf Remark.}
The result is sharp in the sense that $e$ cannot be replaced by a smaller constant.
To see this we
consider a large positive integer $N$ and take the finite series
$\{a_\nu=\frac{1}{\nu}\quad\colon 1\leq \nu\leq N\}$.
Stirling's limit formula gives:
\[
\bigl [a_1\cdots a_\nu\bigr ]^{\frac{1}{\nu}}\simeq\frac {e}{\nu}\quad\colon\,\nu>>1
\] 
Since the harmonic series
$\sum\,\frac{1}{\nu}$ is divergent we conclude that
for every $\epsilon>0$ there exists some large integer $N$ such that
$\{a_\nu=\frac{1}{\nu}\}$ gives
\[
\sum_{\nu=1}^{\nu=N}\, 
\bigl [a_1\cdots a_\nu\bigr ]^{\frac{1}{\nu}} >
(e-\epsilon)\cdot \sum_{\nu=1}^{\nu=N}\, \frac{1}{\nu}
\]



\noindent
There remains to prove the strict upper bound (*)
when $\sum\, a_\nu$ is an arbitrary  convergent positive series.
To attain this we  first establish  inequalities for finite series.
Given a positive integer $m$ we consider the variational problem
\[
\max_{a_1,\ldots,a_m}\, 
\sum_{\nu=1}^{\nu=m}\, 
\bigl [a_1\cdots a_\nu\bigr ]^{\frac{1}{\nu}} \quad
\text{when}\quad a_1+\ldots+a_m=1\tag{1}
\]
\medskip

\noindent
Let $a^*_1,\ldots,a^*_m$ give a maximum and set
$a_\nu^*=e^{-x_\nu}$.
The Lagrange multiplier theorem gives a number
$\lambda^*(m)$ such that if
\[ 
y_\nu=\frac{x_\nu+\ldots+x_m}{\nu}
\]
then
\[
\lambda^*(m)\cdot e^{-x_\nu}=\frac{1}{\nu}\cdot e^{-y_\nu}+\ldots+
\frac{1}{m}\cdot e^{-y_m}\quad \colon\quad 1\leq\nu\leq m\tag{2}
\]
A summation over all $\nu$  gives
\[ 
\lambda^*(m)=e^{-y_1}+\ldots+e^{-y_m}=
\sum_{\nu=1}^{\nu=m}\, 
\bigl [a^*_1\cdots a^*_\nu\bigr ]^{\frac{1}{\nu}}
\]
Hence $\lambda^*(m)$ gives
the maximum for the variational problem which is
no surprise since $\lambda^*(m)$ is 
Lagrange's multiplier.
Now we shall
prove  the strict inequality
\[ 
\lambda^*(m)<e\tag{3}
\]
We prove (3) by  contradiction.
To begin with, subtracting the successive equalities in (2) we get
the following equations:
\[
\lambda^*(m)\cdot [e^{-x_\nu}-e^{-x_{\nu+1}}]=
\frac{1}{\nu} \cdot e^{-y_\nu}\quad
\colon\quad 1\leq\nu\leq m-1\tag{4}
\]
\[
m\cdot \lambda^*(m)=e^{x_m-y_m}\tag{5}
\]
Next, set
\[ 
\omega_\nu=\nu(1-\frac{a_{\nu+1}}{a_\nu})\colon\quad 1\leq\nu\leq m-1\tag{6}
\]
With these notations it is clear that (4) gives
\[
\lambda^*(m)\cdot\omega_\nu=
e^{x_\nu-y_\nu}\quad \colon\quad 1\leq\nu\leq m-1\tag{7}
\]
It is clear that (7) gives:
\[
\bigl(\lambda^*(m)\cdot\omega_\nu\bigr)^\nu=
e^{\nu(x_\nu-y_\nu)}=
\frac{a_1\cdots a_{\nu-1}}{a_\nu^{\nu-1}}\tag{8}
\]
By an induction over $\nu$ which is left to the reader it follows
the $\omega$-sequence satisfies the recurrence equations:
\medskip
\[
\omega_\nu^\nu=\frac{1}{\lambda^*(m)}\cdot
\bigl(
\frac{\omega_{\nu-1}}{1-\frac{\omega_{\nu-1}}{\nu-1}}\bigr)^{\nu-1}
\quad \colon\quad 1\leq\nu\leq m-1\tag{9}
\]
Notice that we also have
\[
\omega_1=\frac{1}{\lambda^*(m)}\tag{10}
\]


\noindent
{\bf{A special construction.}}
With $\lambda$ as a parameter
we define a sequence $\{\mu_\nu(\lambda)\}$
by the recursion formula:
\[ 
\mu_1(\lambda)=\frac{1}{\lambda}\quad\text{and}\quad
[\mu_\nu(\lambda)]^\nu=\frac{1}{\lambda}\cdot
\bigl[
\frac{\mu_{\nu-1}(\lambda)}{1-\frac{\mu_{\nu-1}(\lambda)}{\nu-1}}\bigr]^{\nu-1}
\quad \colon\quad \nu\geq 2\tag{**}
\]


\noindent
From (5) and (9) 
it is clear that 
$\lambda=\lambda^*(m)$ gives
the equality
\[
\mu_m(\lambda^*(m))=m\tag{***}
\]
Now we come to the key point during the whole proof.

\medskip

\noindent
{\bf{Lemma}}
\emph{If $\lambda\geq e$ then
the $\mu(\lambda)$ -sequence satisfies}
\[
\mu_\nu(\lambda)<\frac{\nu}{\nu+1}\quad \colon\quad \nu=1,2,\ldots
\]


\noindent
\emph{Proof.}
We use an induction over $\nu$. With $\lambda\geq e$ we have
$\frac{1}{\lambda}<\frac{1}{2}$ so the case $\nu=1$ is okay.
If $\nu\geq 1$ and the lemma  holds for $\nu-1$, then
the recursion formula (**) and the hypothesis $\lambda\geq e$ give:
\[
[\mu_\nu(\lambda)]^\nu=\frac{1}{\lambda}\cdot
\bigl[
\frac{\mu_{\nu-1}(\lambda)}{1-\frac{\mu_{\nu-1}(\lambda)}{\nu-1}}\bigr]^{\nu-1}<
\frac{1}{e}\cdot
\bigl[
\frac{\frac{\nu-1}{\nu}}{1-\frac{\nu-1}{\nu(\nu-1)}}\bigr]^{\nu-1}
\]
Notice that the last factor  is 1 and hence:
\[
[\mu_\nu(\lambda)]^\nu<e<\bigl(1+\frac{1}{\nu})^{-\nu}
\]
where the last inequality follows from the wellknown
limit  of Neper's constant.
Taking the $\nu$:th root we get
$\mu_\nu(\lambda)<\frac{\nu}{\nu+1}$ which finishes the induction.
\medskip

\noindent
{\bf{Conclusion.}}
If $\lambda^*(m)\geq e$ then  the lemma  above and the
equality (***)  would entail that
\[ 
m=\mu(\lambda^*(m))<\frac{m}{m+1}
\] 
This is impossible when $m$ is a positive integer and hence 
we must have proved the  strict inequality
$\lambda^*(m)<e$.


\medskip


\noindent
{\bf{The strict inequality for an infinite series.}}
It remains to prove that the strict inequality holds for
a convergent series with an infinite number of terms.
So assume that we have an equality
\[
\sum_{\nu=1}^\infty\, 
\bigl [a_1\cdots a_\nu\bigr ]^{\frac{1}{\nu}}=e\cdot \sum_{\nu=1}^\infty\, a_\nu\tag{i}
\]
Put as as above
\[ 
\omega_n=n(1-\frac{a_{n+1}}{a_n})\tag{ii}
\]
Since we already know that the left hand side  is at least equal to  the right hand side
one can apply  Lagrange multipliers and we leave it to the reader to verify that
the  equality (i) gives  the recursion formulas
\[
\omega^n_n=\frac{1}{e} \cdot \bigl[
\frac{\omega_{n-1}}{1-\frac{\omega_{n-1}}{n-1}}\bigr]^{n-1}\tag{iii}
\]
Repeating the proof of the Lemma  above it follows that
\[
 \omega_n<\frac{n}{n+1}\implies
\frac{a_{n+1}}{a_n}>\frac{n}{n+1}\tag{iv}
\]
where (ii) gives the implication.
So with
$N\geq 2$ one has:
\[ 
\frac{a_{N+1}}{a_1}>
\frac{1\cdots N}{1\cdots N(N+1)}=\frac{1}{N+1}
\]
Now $a_1>0$ and since the harmonic series $\sum\,\frac{1}{N}$ is 
divergent it would follows that
$\sum\, a_n$ is divergent.
This contradiction
shows that a strict inequality must hold
in Theorem 9.1.

\newpage

\centerline{\bf{10. Thorin's convexity theorem.}}

\bigskip


\noindent
{\bf{Introduction.}}
In the article [Thorin]
a  convexity theorem was established which goes as follows:
Let $N\geq 2$ be a positive integer and $\mathcal A=\{A_{\nu k}\}$ a
complex $N\times N$-matrix.
To each pair of real numbers $a,b$ in the square
$\square=\{0<a,b<1\}$ we set
\[ 
M(a,b)=
\max_{x,y}\, \bigl|
\sum\sum\, A_{\nu k}\cdot x_k\cdot y_\nu|
\quad\colon\, \sum\, |x_\nu|^{1/a}= \sum\, |y_k|^{1/b}=1
\]

\medskip

\noindent{\bf{10.1 Theorem}}
\emph{The function
$(a,b)\mapsto \log M(a,b)$
is convex in $ \square$}.


\medskip

\noindent
The proof relies upon Hadamard's inequality for maximum norms of
bounded analytic functions in strip domains.
More precisely, let $f(w)$ be an entire function which is
bounded in the infinite strip domain
\[
\Omega=\{ \sigma+is\quad\colon 
0\leq \sigma\leq 1\colon -\infty<s<\infty\}
\]
Set
\[
 M_f(\sigma)=
\max_s\, f(\sigma+is)|\quad\colon\quad 0\leq \sigma\leq 1
\]
Then the following is proved in � XX: 
\[ 
M_f(\sigma)\leq M_f(0)^{1-\sigma}\cdot M_f(1)^\sigma\tag{*}
\]


\noindent
\emph{Proof of Theorem 10.1.}
With $0<a,b<1$ fixed we
consider $N$-tuples $x_\bullet$ and $y_\bullet$ 
in ${\bf{C}}^N$
and write
\[ 
x_\nu=c_\nu^a\cdot e^{i\theta_\nu}\quad
\text{ and}\quad
y_k=d_ke^{i\phi_k}
\]
where the $c$-and the $d$-numbers are real and positive whenever
they are $\neq 0$.
It is is clear that
\[ 
M(a,b)=\max_{c,d,\theta,\phi}\,
\bigl|\sum\sum\, A_{\nu k}\cdot c_\nu^a\cdot d_k^b\cdot
e^{i\theta_\nu}e^{i\phi_k}\bigr|\tag{1}
\]
where the maximum is taken over  $N$-tuples
$\{c_\bullet\}$ and $\{d_\bullet\}$ of non-negative real numbers such that
\[ 
\sum c_\nu=\sum d_k=1\tag{2}
\] 
and   $\{\theta_\nu\}$ and $\{\phi_k\}$
are arbitrary $N$-tuples from the periodic interval $[0,2\pi]$.
Consider a pair $(a_1,b_1)$ and $(a_2,b_2)$ in $\square$
and let $(\bar a,\bar b)$ be the middle point.
Then we find $c^*,d^*,\theta^*,\phi^*$ so that
\[ 
M(\bar a,\bar b)=
\bigl|\sum\sum\, A_{\nu k}\cdot (c^*_\nu)^a\cdot (d^*_k)^b\cdot
e^{i\theta^*_\nu+i\phi^*_k}\bigr|\tag{3}
\]



\noindent
Let $w=\sigma+is$ be a complex variable 
and define the analytic function $f$ by

\[
f(w)=
\sum\sum\, A_{\nu k}\cdot (c^*_\nu)^{a_1+w(a_2-a_1)}\cdot 
(d^*_k)^{b_1+w(b_2-b_1)}\cdot
e^{i\theta^*_\nu+i\phi^*_k}\tag{4}
\]
It is clear that $f(w)$ is an entire analytic function and
$|f(1/2)|= M(\bar a,\bar b)$.
Next, with $w=is$ purely imaginary we have

\[ 
f(is)= \sum\sum\, A_{\nu k}\cdot (c^*_\nu)^{a_1}\cdot 
(d^*_k)^{b_1}\cdot
e^{is(a_2-a_1)\log c_\nu^*+is(b_2-b_1)\log d_k^*}\cdot 
e^{i\theta^*_\nu+i\phi^*_k}\tag{5}
\]
For each pair $\nu,k$
the exponential product
\[
e^{is(a_2-a_1)\log c_\nu^*+is(b_2-b_1)\log d_k^*}\cdot 
e^{i\theta^*_\nu+i\phi^*_k}=
e^{i(\theta_\nu(s)+\phi_k(s))}
\]
for some pair $\theta_\nu(s),\phi_k(s)$.
From (1) we see that
\[
\max_s\, |f(is)|\leq M(a_1,b_1)\tag{6}
\]
In the same way the reader can verify that
\[ 
\max_s\, |f(1+is)|\leq M(a_2,b_2)\tag{7}
\]
Now Hadamard's inequality (*) entails that
\[
\log M(\bar a,\bar b)\leq \frac{1}{2}\cdot 
[
\log M(a_1, b_1)+\log M( a_2, b_2)
\]
This proves the required convexity.

\newpage

\centerline{\bf{11. Cesaro and H�lder limits}}
\bigskip

\noindent
{\bf{Introduction.}}
In 1880 Cesaro introduced a certain summation procedure which
which is a substitute for divergent series and leads to the notion of 
Cesaro summability to be  defined below.
Another summability was introduced by H�lder
and later Knopp and Schnee 
proved that
the conditions for Cesaro\vvv respectively H�lder
are equivalent. 
In Theorem 11.8 we present  
the elegant proof due to Schur taken from 
[Landau; Chapter 2].
For  a given  sequence of complex numbers $a_0,a_1,a_2,\ldots$
we put:
\[ 
S_n=a_0+\ldots+a_n
\]
If $k\geq 0$ we define inductively
\[ 
S_n^{(k+1)}=
S_0^{(k)}+\ldots+
S_n^{(k)}\quad\text{where}\quad S_n^{0)}= S_n
\]




\medskip

\noindent
{\bf{11.1  Definition.}}
\emph{For a given integer $k\geq 0$
we say that
the sequence $\{a_n\}$ is Cesaro summable of order $k$
if there exists a limit}
\[ 
s\uuu *(k)=\lim_{n\to\infty}\,
\frac{k !}{n^k}\cdot S_n^{(k)}\tag{*}
\]

\medskip

\noindent
{\bf{11.2 Exercise.}}
Assume that $\{a\uuu n\}$ is Cesaro summable of some order
$k$. Show that 
\[ 
a\uuu n=O(n^k)
\]


\noindent
{\bf{11.3 The power series
$f(x)= \sum\, a\uuu nx^n$}}.
Assume that $\{a\uuu n\}$ is Cesaro summable of some order $k$. 
Exercise 11.2 shows  that
the power series $f(x)$ has a radius of convergence which
is at least one and 
for every integer $k\geq 0$ the reader can  verify the equality
\[ 
f(x)=\sum a_nx^n
=(1-x)^{k+1}\cdot \sum S_n^{(k)}\cdot x^n\tag{*}
\]

\noindent
{\bf{11.4 Exercise.}}
Deduce from the above  that if $\{a\uuu n\}$
is Cesaro summable of some order $k\uuu *$ with limit value $s\uuu *(k\uuu *)$ then
one has the limit formula:
\[
s\uuu *(k\uuu *)=\lim\uuu{x\to 1}\, f(x)
\]

\noindent
Now we prove that Cesaro summability of some order implies the
summability for every higher order.
\medskip





\noindent
{\bf{11.5 Proposition.}}
\emph{If (*) holds for some $k\uuu *$ then the Cesaro 
limit exists for every $k\geq k\uuu *$ and one has the equality
$s\uuu *(k)=s\uuu *(k\uuu *)$.}

\medskip

\noindent
\emph{Proof.}
Cesaro summability of some order $k$
with a limit $s\uuu *(k)$ means that
\[
S_n^{(k)}= \frac{n^k}{k!}\cdot s\uuu *(k)+o(n^k)\tag{i}
\]
where the last term is small ordo.
If (i) holds we get
\[
S_n^{(k+1)}=\frac{s\uuu *(k)}{k !}
\sum\uuu{\nu=0}^n\, n^\nu+
o\bigl(\sum\uuu{\nu=0}^n\, n^\nu\bigr)
=\frac{s\uuu *(k)}{k !}\cdot [\frac{n^{k+1}}{k+1}\vvv 1]+
o(n^{k+1})
\]
From this the reader discovers the requested induction step and
Proposition 11.5 follows.
\medskip

\medskip

\noindent
{\bf{11.6 H�lder's summation.}}
To each sequence of complex numbers $a_0,a_1,a_2,\ldots$
we put
\[ 
H_n^{(0)}=a_0+\ldots+a_n
\]
and if  $k\geq 0$ we define inductively
\[ 
H_n^{(k+1)}=\frac{
H_0^{(k)}+\ldots+
H_n^{(k)}}{n+1}
\]

\noindent
{\bf{11.7 Definition.}}
\emph{The sequence $\{a_n\}$ is H�lder summable of order $k$
if there exists a limit}
\[ 
\lim_{n\to\infty}\,
H_n^{(k)}\tag{**}
\]



\medskip

\noindent
{\bf{11.8 Theorem}}
\emph{A sequence $\{a_n\}$ is Cesaro summable of
of some order $k$ if and only if it is H�lder summable of the same order
and there respectively limits are the same.}
\bigskip

\noindent
The proof of Theorem 11.8 requires several steps.
First we introduce  arithmetic mean value sequences attached to every 
sequence $\{x\uuu 0,x\uuu 1,\ldots\}$:

\[ 
M(\{x\uuu \nu\})[n]= \frac{x\uuu 0+\ldots+x\uuu n}{n+1}
\]
Next, to each $k\geq 1$ we 
construct the sequence $T\uuu k(\{x\uuu\nu\})$
by
\[ 
T\uuu k(\{x\uuu\nu\})[n]=\frac{k\vvv 1}{k}\cdot M(\{x\uuu \nu\})[n]+
\frac{x\uuu n}{k}
\]
\medskip

\noindent
So above $M$ and $\{T\uuu k\}$
are linear operators which send a complex sequence to another complex sequence.
The reader may verify that these operators commute, i.e.
\[
 T\uuu k\circ M= M\circ T\uuu k
\] 
hold for every $k$ and similarly the $T$\vvv operators commute.
For a given $k$ we can also regard the passage to 
the Cesaro sequence $\{S\uuu n^{[k)}\}$
as a linear operator which we denote by
$\mathcal C^{(k)}$. Similarly we get the H�lder operators
$\mathcal H^{(k)}$ for every $k\geq 1$.
\medskip

\noindent
{\bf{11.9 Proposition.}}
\emph{The following identities hold}
\[
T\uuu k\circ \mathcal C^{(k\vvv 1)}= M\circ \mathcal C^{(k)}
\quad\colon\quad k\geq 1\tag{i}
\]
\[
\mathcal H^{(k)}=
T\uuu 2\circ\ldots\circ T\uuu k\circ \mathcal C^{(k)}
\quad\colon\quad k\geq 2\tag{ii}
\]



\noindent
{\bf{11.10 Exercise.}} Prove (i) and (ii) above.
\medskip

\noindent
As a last preparation towards the proof of Theorem 11.8
we need certain  limit formulas  which show that the
$T$\vvv operators have robust properties. First we have:
\medskip

\noindent
{\bf{11.11 Lemma}}
\emph{Let $\{x_1,x_2,\ldots \}$ be a sequence of complex numbers
and $q$ a positive integer such that}
\[
\lim_{n\to\infty}\, 
q\cdot \frac{x_1+\ldots+x_n}{n}+x_n=0
\]
\emph{Then  it follows that}
\[
\lim_{n\to\infty}\, x_n=0
\]


\noindent
\emph{Proof}.
Set $y_n=q(x_1+\ldots+x_n)+n x_n$.
By an induction over $n$ one verifies that
\[
\sum_{\nu=1}^{\nu=n}\, (\nu+1)\cdots (\nu+q-1)\cdot y_\nu=
(n+1)\cdots(n+q)\cdot \sum_{\nu=1}^{\nu=n}\,x_\nu\tag{1}
\] 
hold for every $n\geq 1$.
By the hypothesis $y_n=o(n)$ where $o(n)$ is small ordo of $n$.
It follows that the left hand side in (1) is $o(n^{q+1})$
and since the product 
$(n+1)\cdots(n+q)\simeq n^q$ we conclude that
\[
\sum_{\nu=1}^{\nu=n}\,x_\nu=o(n)\tag{2}
 \]
Finally, we have
\[ 
nx_n=y_n-q\cdot  \sum_{\nu=1}^{\nu=n}\,x_\nu
\]
and by (2) and the hypothesis the right hand side is $o(n)$ which
after division with $n$ gives
$x_n=o(1)$ as required.

\medskip

\noindent
{\bf{11.12 Proposition. }}
\emph{Let $\{x\uuu \nu\}$�be a sequence and $k\geq 1$ an integer
such that there exists}
\[
\lim\uuu{n\to \infty}\, T\uuu k(\{x\uuu\nu\}][n]= s
\]
\emph{Then it follows that $\{x\uuu n\}$ converges to $s$.}
\medskip

\noindent
{\bf{11.13 Exercise.}} Deduce Proposition 11.12 from Lemma 11.11.

\bigskip

\noindent
\centerline{\emph{11.14 Proof of Theorem 11.8.}}
\medskip

\noindent
The easy case $k=1$ is left to the reader and we proceed to 
prove the theorem when $k\geq 2$.
Assume first that $\{a\uuu n\}$
is Cesaro summable of some order $k\geq 2$
with a limit $s$.
Exercise 11.12 implies that $T\uuu k\circ \mathcal C^{(k)}$ sends
$\{a\uuu n\}$ to a convergent sequence with limit $s$.
If $k\geq 3$ we apply the exercise to $T\uuu{k\vvv 1}$ and continue until
the composed operator
\[
 T\uuu 2\circ\cdots\circ T\uuu k\circ \mathcal C^{(k)}
\] 
sends the $a$\vvv sequence to a convergent sequence with limit $s$.
By (ii) in Proposition 11.8 this entails that
$\{a\uuu k\}$ is H�lder summable of order $k$ with limit $k$.
Conversely, assume that $\{a\uuu n\}$
is H�lder summable of some order $k\geq 2$.
The equality (ii) from  Proposition 11.9  gives
\[ 
\mathcal H^{(2)}= T\uuu 2\circ \mathcal C^{(2)}
\]
Hence Proposition 11.13 applied to $T\uuu 2$ shows that H�lder summability
of order 2 entails Cesaro summability of the same order.
Next, if $k\geq 3$ we  again use (ii) in 11.9 and conclude that
the sequence
\[
T\uuu 3\circ\ldots T\uuu k\circ \mathcal C^{(k)}(\{a\uuu n\})
\] 
is convergent.
By repeated application of (ii) in 11.18 applied to $T\uuu 3,\ldots,T\uuu k$
we conclude that the $a$\vvv sequence is Cesaro summable of order $k$
and has the same limit as the H�lder sum.









\newpage


\centerline{\bf{12.  Power series and arithmetic  means.}}

\bigskip

\noindent
Consider a power series
\[ 
f(x)=\sum\, a\uuu n\cdot x^n
\]
which converges when $|x|<1$ and  assume also that
\[
\lim\uuu{x\to 1}\,\sum\, a\uuu n\cdot x^n=0\tag{*}
\]
For each $k\geq 1$ we get the sequence $\{S\uuu n^{(k)}\}$
from the previous section and we prove the following:

\medskip

\noindent
{\bf{12.1 Theorem.}}\emph{ Assume (*) and that there exists some integer 
$k\geq 1$ such that}
\[
\lim\uuu{n\to\infty}\, S\uuu n^{(k)}=0
\]
\emph{Then the series $\sum\, a \uuu n$ converges.}
\medskip


\noindent
{\bf{Example.}}
Consider the case $r=1$ where
\[
S\uuu n^{(1)}=\frac{na\uuu 0+(n\vvv 1)a\uuu 1+\ldots+a\uuu n}{n}
\]
The sole assumption that
$S\uuu n^{(1)}\to 0$ does not imply
$\sum\, a\uuu n$ converges. But in addition (*) is assumed in Theorem 12.1 
which will give the convergence.
The proof of Theorem 12.1
is based upon the following convergence criterion where (*) above is tacitly assumed.


\medskip

\noindent
{\bf{12.2 Proposition.}} \emph{The series $\sum\, a\uuu n$ converges if there to every
$\epsilon>0$ exists a pair $(p\uuu 0,n\uuu 0)$ such that}
\[
p\geq p\uuu 0\implies J(n\uuu 0,p)=\bigl|\, 
\int\uuu 0^1\, \frac{\sin 2p\pi(x\vvv 1)}{x\vvv 1}\cdot 
\sum\uuu{n=n\uuu 0}^\infty\, a\uuu n x^n
\cdot dx
\,\bigr|<\epsilon
\]
\medskip

\noindent
{\bf{Exercise.}} Prove this classic result which already
was wellknown to Abel.
\bigskip

\noindent
\emph{Proof of theorem 12.1}.
To profit upon  Proposition 12.2 we need the two inequalities
below which are valid for all pairs of positive integers $p$ and $n$:

\[
\bigl|\, \int\uuu 0^1\, \sin (2p\pi x)\cdot x^k(1\vvv x)^n
\cdot dx\,\bigr|\leq 2\pi(k+2)\,!\cdot\frac{p}{n^{k+2}}\tag{i}
\]

\[
\bigl|\, \int\uuu 0^1\, \sin 2p\pi x\cdot x^k(1\vvv x)^n
\cdot dx\,\bigr|\leq \frac{C(k)}{p\cdot n^k}\tag{ii}
\]
where the constant $C(k)$ in (ii) as indicated only depends upon $k$.
The verification of (i\vvv ii) is left to the reader.
Next, recall from (*) in  � 11.4
that:
\[
f(x)=\frac{(1\vvv x)^{k+1}}{(k+1)!}\cdot \sum\, S\uuu n^{(k)} n^k\cdot x^n\tag{iii}
\]
Let $\epsilon>0$ and choose $n\uuu 0$ such that
\[ 
n\geq n\uuu 0\implies |S\uuu n^{(k)}|<\epsilon\tag{iv}
\]
which is possible from the assumption in  Theorem 12.1
Notice that (iii) gives the equality
\[
\sum\uuu{n=n\uuu 0}^\infty\, a\uuu n x^n=
\frac{(1\vvv x)^{k+1}}{(k+1)!}\cdot \sum\uuu{n=n\uuu 0}^\infty
\, S\uuu n^{(k)} n^k\cdot x^n\tag{iii}
\]

\noindent
Hence  (iv) and the triangle inequality shows that with 
$n\uuu 0$ kept fixed, the absolute value of the integral in Proposition 12.2  is 
majorized as follows for every $p$:
\[ 
J(n\uuu 0,p)\leq \epsilon\cdot 
\sum\uuu{n=n\uuu 0}^\infty
\frac{n^k}{(k+1)!}\cdot
\bigl|\,
\int\uuu 0^1\, \sin (2p\pi x)\cdot x^k(x\vvv 1)^n
\cdot dx\,\bigr|
\]
In (iv) we have chosen  $n\uuu 0$ and for an arbitrary $p\geq p\uuu 0=
n\uuu 0+1$
we decompose the sum from $n\uuu 0$ up to $p$ and
after we take a sum with $n\geq p+1$ which means that
$J(n\uuu 0,p)$ is majorized by $\epsilon$ times
the sum of
the following two expressions:

\[
\sum\uuu{n=n\uuu 0}^{n=p}
\frac{n^k}{(k+1)!}\cdot
\bigl|\,
\int\uuu 0^1\, \sin (2p\pi x)\cdot x^k(x\vvv 1)^n
\cdot dx\,\bigr|\tag{1}
\]
\[
\sum\uuu{n=p+1}^\infty
\frac{n^k}{(k+1)!}\cdot
\bigl|\,
\int\uuu 0^1\, \sin (2p\pi x)\cdot x^k(x\vvv 1)^n
\cdot dx\,\bigr|\tag{2}
\]

\noindent
Using (i) above it follows that (1) is estimated by

\[
2\pi\cdot (k+2) !\cdot \frac{C(k)}{p}\cdot (p\vvv n\uuu 0)
\leq 2\pi\cdot (k+2) !\cdot C(k)=K\uuu 1
\]
Next, using (ii) it follows that (2) is estimated by

\[
\pi\cdot \frac{k+2}{k+1}\cdot p\cdot \sum\uuu{n=p+1}^\infty n^{\vvv 2}
\leq\pi\cdot \frac{k+2}{k+1}=K\uuu 2
\]
So with $K=K\uuu 1+K\uuu 2$ we have
\[ 
J(n\uuu 0,p)\leq 2K\cdot \epsilon
\] 
for every $p\geq n\uuu 0+1$
and since $\epsilon>0$ was arbitrary  the proof of Theorem 12.2
is finished.

\newpage

\centerline{\bf{13. Taylor series and quasi\vvv analytic functions.}}
\bigskip

\noindent
{\bf{Introduction.}}
Let $f(x)$  an infinitely differentiable function
 defined on the interval $[0,1]$.
At $x=0$ we can take the derivatives and set
 \[
C\uuu\nu= f^{(\nu)}(0)
\]
In general the sequence
$\{C\uuu \nu\}$ does not determine $f(x)$. The standard example is the 
$C^\infty$\vvv function defined for $x>0$ by $e^{\vvv 1/x}$ and
zero on $x\leq 0$. Here $\{C\uuu\nu\}$ is the null sequence and yet
the function is no identically zero.
But if we impose sufficiently strong growth conditions on the derivatives of $f$ over
the whole interval $(\vvv 1,1)$
then $\{C\uuu\nu\}$
determines $f$. In general, 
let $\mathcal A=\{\alpha\uuu\nu\}$ be an increasing sequence of positive real numbers
and denote by $\mathcal C\uuu\mathcal A$ the class of $C^\infty$\vvv functions
on $[0,1]$ where the maximum norms of the derivatives satisfy
\[
\max\uuu x\, |f^{(\nu)}(x)|\leq k^\nu\cdot \alpha\uuu\nu^\nu
\quad\colon\quad \nu=0,1,\ldots\tag{*}
\]
for some $k>0$ which may depend upon $f$.
One says that $\mathcal C\uuu\mathcal A$ is a quasi\vvv analytic class
if every $f\in C\uuu \mathcal A$ whose Taylor series is identically
zero at $x=0$ vanishes identically on $[0,1)$.
In the article [Denjoy 1921),  Denjoy proved that
$C\uuu\mathcal A$
is quasi\vvv analytic if the series
\[
 \sum\, \frac{1}{\alpha\uuu\nu}=+\infty\tag{**}
\] 
The conclusive result which gives a necessary and sufficient condition on 
the sequence $\{\alpha_\nu\}$ in order  that $C\uuu \mathcal A$
is quasi\vvv analytic is proved in Carleman's book [1923].
The criterion is as follows:

\medskip

\noindent
{\bf{Theorem.}}
\emph{Set $A\uuu \nu=\alpha\uuu\nu^\nu$ for each $\nu\geq 1$.
Then $C\uuu\mathcal A$ is quasi\vvv analytic if and only if}
\[
 \int\uuu 1^\infty\, \log\,\bigl[\,\sum\uuu{\nu=1}^\infty\,
\frac{r^{2\nu}}{A\uuu \nu^2} \,\bigr] \cdot \frac{dr}{r^2}=+\infty
\]

\medskip

\noindent
For the proof of this result we refer to � XX in Special Topics.
\medskip

\noindent
{\bf{The reconstruction theorem.}}
Since quasi\vvv analytic functions by definition
are determined by their Tayor series at a single point
there remains the question how to determine $f(x)$ in a given  
quasi\vvv analytic class $C\uuu \mathcal A$
when the sequence of its Taylor coefficients at $x=0$ are given.
Such a reconstruction   was announced   by
Carleman in [CR\vvv 1923] and the  detailed proof appears in
[Carleman\vvv book].
Carleman
considered a class of  
variational problems
to  attain  the reconstruction.
Let $n\geq 1$ and for a given sequence
of real numbers $\{C\uuu 0,\ldots,C\uuu{n\vvv  1}\}$
we consider the class of $n$\vvv times differentiable functions $f$
on $[0,1]$ for which
\[
f^{(\nu)}(0)= C\uuu\nu\quad\colon\quad \nu=0,\ldots ,n\vvv 1\tag{i}
\]
Next, let $\{\gamma\uuu 0,\gamma\uuu 1,\ldots,\gamma\uuu n\}$
be some $n+1$\vvv tuple of positive numbers and consider
the variational problem
\[
\min\uuu f J\uuu n(f)=
\sum\uuu {\nu=0}^{\nu=n}\,\gamma\uuu\nu^{\vvv 2\nu}
\cdot \int\uuu 0^1\, [f^{(\nu)}(x)\,]^2\cdot dx\tag{ii}
\]
where the competing family consist
of $n$\vvv times differentiable functions on 
$[0,1]$ satisfying (i) above. 
The strict convexity of $L^2$\vvv norms
entail  that the variational problem has a unique minimizing function 
$f\uuu n$
which depends linearly upon $C\uuu 0,\ldots,C\uuu n$.
In other words, there exists a unique doubly indexed sequence
of functions $\{\phi\uuu{p,n}\}$ defined for pairs $0\leq p\leq n$
such that 
\[
f\uuu n(x)= \sum\uuu{\nu=0}^{\nu=n\vvv 1} \, C\uuu p\cdot \phi\uuu{p,n\vvv 1}(x)
\]
where the functions
$\{\phi\uuu {0,n\vvv 1},\ldots \phi\uuu {n\vvv 1,n\vvv 1}\}$
only depend upon $\gamma  \uuu 0,\ldots,\gamma  \uuu n$.
\newpage

\noindent
{\bf{A specific choice of the $\gamma$\vvv sequence.}}
Let $\mathcal A=\{\alpha\uuu \nu\}$ be a Denjoy sequence, i.e. 
(**) above diverges. Set $\gamma\uuu 0=1$ and 
\[ 
\gamma\uuu\nu= \frac{1}{\alpha\uuu\nu}\cdot
\sum\uuu{p=1}^{p=\nu}\,
\alpha\uuu p\quad\colon\quad \nu\geq 1
\]
Given some $F(x)\in C^\infty[0,1]$
we get the sequence $\{C\uuu\nu= F^{(\nu)}(0)\}$
and to each $n\geq 1$ we consider the variational problem above
using the $n$\vvv tuple 
$\gamma\uuu 0,\ldots,\gamma\uuu{n\vvv 1}$
which yields the extremal function $f\uuu n(x)$.
With these notations Carelan  proved the following:
\medskip

\noindent
{\bf{ 13.1 Theorem.}}
\emph{If  $F(x)$ belongs to  the class $C\uuu\mathcal A$
it follows that }
\[
\lim\uuu{n\to\infty}\, f\uuu n(x)=F(x)
\] 
where the convergence holds uniformly on interval $[0,a]$ for every $a<1$.

\medskip

\noindent{\bf{13.2 A series expansion.}}
Using Theorem 13.1  Carleman also  proved that
when  the series (**) diverges, then there exists a 
doubly indexed sequence $\{a\uuu{\nu,n}\}$ defined for pairs $0\leq \nu\leq n$
which only depends on the sequence $\{\alpha\uuu\nu\}$
such that if   $F(x)$ belongs to $\mathcal C\uuu\mathcal A$
then it is given by a limit of series:
\[
F(x)=\lim\uuu{n\to\infty}\, \sum\uuu{\nu=0}^{\nu=n}
\, a\uuu{\nu,n}\cdot \frac{F^{(\nu)}(0)}{\nu\,!}\cdot x^\nu\quad\colon\quad 
0\leq x<1
\]
\medskip


\noindent
{\bf{ Remark.}}
Above we exposed the reconstruction for  quasi\vvv analytic classes of the Denjoy type.
For a general quasi\vvv analytic class
a similar result is proved in
[Carleman]. Here the
proof and the result is of a more technical nature 
so we refrain to present  the details.
Concerning the doubly indexed $a$\vvv sequence it is found in a rather implicit
manner via solutions to the variational problems and an extra complication is that
these $a$\vvv numbers depend upon the given
$\alpha$\vvv sequence.
it appears that several open problems remain concerning
effective formulas and the reader may also consult [Carleman: page xxx]
for some open questions related to the reconstruction above.


. 
\medskip


\noindent
{\bf{13.4 Quasi\vvv analytic boundary values.}}
Another  problem 
is concerned with  boundary values of analytic functions
where
the set of non\vvv zero Taylor\vvv coefficients is sparse.
In general, consider a power series $\sum\, a\uuu nz^n$
whose radius of convergence equal to one. Assume that there exists an interval $\ell$
on the unit circle such that
the analytic function $f(z)$ defined by the series extends to
a continuous function in the closed sector where
$\text{arg}(z)\in\ell$. So on $\ell$ we get a continuous boundary
value function $f^*(\theta)$
and suppose that $f^*$ belongs to some quasi\vvv analytic class on
this interval.
Let $f$ be given by the series
\[
f=\sum\, a\uuu n\cdot z^n
\]
Suppose that gaps occur
and write  the sequence of non\vvv zero coefficients
as $\{ a\uuu{n\uuu 1}, a\uuu{n\uuu 2}\ldots\}$
where $k\mapsto n\uuu k$ is a  strictly increasing sequence.
With these notations the following result is due to Hadamard:
\medskip

\noindent
{\bf{13.5 Theorem.}}\emph{
Let $f(z)$ be analytic in the open unit disc
and assume it has a continuous extension to some open  interval on the 
unit circle where 
the boundary function $f^*(\theta)$ is real\vvv analytic. Then there exists an integer 
$M$ such that}
\[ 
n\uuu{k+1}\vvv n\uuu k\leq M
\]
\emph{for all $k$. In other words,   the sequence of non\vvv zero coefficients 
cannot be too  sparse.}
\bigskip

\noindent
Hadamard's result was extended to the quasi\vvv analytic case in
[Carleman] where it is proved that if $f^*$ belongs to some quasi\vvv analytic class
determined by a sequence $\{\alpha\uuu\nu\}$
then the gaps cannot be too sparse, i.e. after a rather involved analysis
one finds that $f$ must be identically zero if
the integer function $k\mapsto n\uuu k$ increases too fast.
The rate of increase depends upon $\{\alpha\uuu\nu\}$
and it appears that no precise descriptions of
the growth of $k\mapsto n\uuu k$ which would ensure unicity
is known for a general quasi\vvv analytic class, i.e. even in
the situation considered by
Denjoy.
So there remains many interesting open questions concerned with quasi-analyticity.



 


\end{document}











 












\newpage
