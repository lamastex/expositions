\documentclass{amsart}
\usepackage[applemac]{inputenc}

\addtolength{\hoffset}{-12mm}
\addtolength{\textwidth}{22mm}
\addtolength{\voffset}{-10mm}
\addtolength{\textheight}{20mm}



\def\uuu{_}

\def\vvv{-}

\begin{document}







\centerline{\bf\large {Chapter 7. Residue calculus.}}


\bigskip

\noindent
\centerline{\emph{Contents}}
\bigskip
















\noindent
\emph{0: Introduction}

\medskip


\noindent
\emph{0.A Four examples of residue calculus }

\medskip


\noindent
\emph{0:B Summation formulas}
\medskip


\noindent
\emph{0:C Asymptotic expansions}
\medskip


\noindent
\emph{0:D Ugly examples}



\medskip


\noindent
\emph{0:E Fractional decomposition}


\medskip


\noindent
\emph{0:F Computing local residues}



\medskip


\noindent
\emph{0.G Line integrals of multi-valued functions}
\medskip


\noindent
\emph{0:H Solving a differential equation}


\medskip


\noindent
\emph{0:I More involved integrals}
\medskip


\noindent
\emph{0:J  Abel integrals and equations}

\medskip


\noindent
\emph{0:K  Location of zeros of polynomials}




\bigskip













\centerline{\emph{Special integrals.}}
\bigskip

\noindent
\emph{A: The integral $e^{-x^2}$}


\medskip

\noindent
\emph{B: Integrals of rational functions}


\medskip

\noindent
\emph{C: The integral $\int_0^\infty\,\frac{dx}{P(x)}$}

\medskip

\noindent
\emph{D: The integrals $\int_{-\infty}^\infty\,\frac{e^{iax}\cdot dx}{P(x)}$}

\medskip

\noindent
\emph{E: Principal value integrals}
\medskip

\noindent
\emph{F: The integral $\int_0^\infty\,\frac{T(\sin x)}{P(x)}\cdot dx$}

\medskip

\noindent
\emph{G: Adding complex zeros to $P$}


\medskip

\noindent
\emph{H: The integral $\int_0^\infty\, \frac{x^a}{1+x^2}\cdot dx$}





\medskip

\noindent
\emph{I: The use of the log-function}



\medskip
\noindent
\emph{J: Trigonometric integrals}


\medskip

\noindent
\emph{K: Summation formulas.}

\medskip

\noindent
\emph{L:  Fourier integral}
\medskip

\noindent
\emph{M: Multi-valued Laplace integrals}

\bigskip

\centerline{\bf{Introduction}}

\bigskip



\noindent
The
examples  in 0.A - 0.J describe  general  methods which often appear
in  residue calculus. A more extensive discussion about 
zeros of polynomials
appears in section 0.K.
The second part, listed by  sections A-M
and are devoted to  specific examples.
We shall often  study   integrals
where multi\vvv valued functions  appear as integrands and refer to
elementary
text\vvv books
for more standard examples. Let us remark that
the first extensive use of multi-valued integrands 
appeared in Abel's pioneering work \emph{xxx} from 1827.
The interested reader may consult [Abel legacy] 
where several articles give for an account about Abel's discoveries and
the usefulness of complex line integrals in other areas such as
algebraic geometry. 
\medskip

\noindent
{\bf{Solutions to differential equations.}}
One  application of residue calculus is the construction
of solutions to ordinary differential equations.
Prior to Abel's work  specific multi-valued integrands were
used by Cauchy, Legendre and Laplace
to obtain solutions of differential equations.
An example due to Laplace
goes as follows: Consider a differential operator
of the form
\[
P(x,\partial)=x\cdot \partial^m+\sum_{\nu=0}^{\nu=m-1}\, 
(a_\nu+b_\nu x)\partial^\nu
\]
where $\partial=\frac{d}{dx}$
and $m\geq 2$, while $\{a_\nu\}$ and $\{b_\nu\}$ are complex numbers.
To find solutions we consider the polynomials
\[
Q(t)=t^m+\sum_{\nu=0}^{\nu=m-1}\,b_\nu t^\nu\quad\colon\quad
P(t)=\sum_{\nu=0}^{\nu=m-1}\,a_\nu t^\nu
\]
Assume  that $Q$ has simple zeros $\beta_1,\ldots,\beta_m$ which 
in general are complex numbers.
Newton's fractional decomposition gives
\[
\frac{P}{Q}=
\sum\, \frac{c_\nu}{t-\beta_\nu}\quad\colon\,
c_\nu= \frac{P(\beta_\nu)}{Q'(\beta_\nu)}
\]
Set
\[
Z(t)= \frac{1}{Q(t)}\cdot e^{\int\, \frac{P}{Q}}=
\frac{1}{Q(t)}\cdot\prod\, (t-\beta_\nu)^{c_\nu}\tag{i}
\]
Let $\gamma$ be a simple and rectifiable
curve in the complex $t$-plane which avoids the zeros of $Q$.
Along $\gamma$ there exists a single-valued branch of $Z$ and
we try to define the function
\[ 
y(x)= \int_\gamma\, Z(t)\cdot e^{xt}\, dt
\]
Let $t_*$ and $t^*$ be the end-points of the oriented
curve $\gamma$.
Now
\[
P(x,\partial)(y)=\int_\gamma\, (xQ(t)+P(t))Z(t)\cdot e^{xt}\,dt\tag{ii}
\]
Since the $t$-derivative of  $e^{xt}$ is  $xe^{xt}$, a  partial integration identifies
(ii) with
\[
Q(t)Z(t) \cdot e^{xt} \bigr|\, _{t_*}^{t^*}+
\int_\gamma\, (-\frac{d}{dt}(Q(t)Z(t))+P(t))Z(t))\cdot e^{xt}\,dt
\]
From (i) it is clear that $\frac{d}{dt}(Q(t)Z(t))=P(t))Z(t)$ and hence
(ii) is zero if
\[
Q(t_*)Z(t_*) \cdot e^{xt_*}=Q(t^*)Z(t^*) \cdot e^{xt^*}\tag{iii}
\]
This gives a method to find solutions to the equation (*), i.e. one seeks
$\gamma$-curves such that (iii) hold.
Here one can take unbounded and non-closed $\gamma$-curves.
For example, if $x<0$ we notice that if $t_*=t^*$
and this common number has a large postive real parti, then
the absolute value of $e^{xt_*}$ is small compared to
$Q(t_*)Z(t_*))+P(t_*))Z(t_*))$. In particular (iii) holds if
$\gamma$ is an infinite simple curve which starts at $+\infty$ and moves for a while in
the negative direction on the positive real $t$-line to some
number
$t_0$,and then continues via a simple closed curve
$\gamma_0$ and the final piece of  $\gamma$ 
is $[t_0,+\infty)$.
Thus, every such $\gamma$-curve gives a solution $y(x)$ defined for
$x<0$. 


\medskip

\noindent
If we instead seek solutions defined when $x>0$ one picks 
$\gamma$-curves which 
start at $-\infty$ and move to $-t_0$ and continues along a closed curve
$\gamma_0$, and return to $-\infty$ in the negative diretion on
the interval $t\leq -t_0$.
In the first consstruction we notice that
the solutions $y(x)$ actually become analytic functions of the 
complex variable $z=x+iy$ which are defined in the half-plane $\mathfrak{Re}(z)<0$.
Since $P$ has order $m$ it follows from general facts that
the number of ${\bf{C}}$-linearly independent solutions 
which are analytic in this half-space is an $m$-dimensional 
complex vector space.
So in spite  of the  general constructons via a choice of
$\gamma$-curves above, the resuting $y$-functions can only generate
an $m$-dimensional vector space.
The merit  of the Laplace construction is that there actually exists
an $m$tuple of $\gamma$ -curves as above  whose 
the resulting $y$-functions give a basis for the 
solution space in
the half-plane
$\mathfrak{Re}(z)<0$. A similar result  holds for 
solutions in
the left half-plane.
Of course, after this has been achieved there still remains  to
find global solutions to (*).
It turns out that they exist but
since $x$ appears as a factor for the highest differential
$\partial^m$, the solutions are not always entire functions, i.e one should
also allow  distribution solutions.
We shall discuss this in more detail in � xx and mention only that
the space of distribution solutions to (*) is a complex vecotr space of dimension $m+1$
and there exist distribution solutions  supported by
the half-line $x\leq 0$.
So the Laplace constructions give a start,  but not 
the whole story about solutions to 
(*). A more favourable case occurs when
$P(x,\partial)=\partial^m+P_*(x,\partial)$ where
$P_*$ has order $\leq m-1$. In this case the solutions 
extend to entire functions and
form an $m$-dimensional complex vector space.
See � xx for some specifix exampes where we shall employ the Laplace construction.

\bigskip


\noindent
{\bf{A  calculation by Riemann.}}
In the complex $z$\vvv plane we consider the line
\[
\Gamma=\{ z=a+se^{3\pi i/4}\}
\] 
where $0<a<1$ and $\vvv \infty <s<\infty$.
Along $\Gamma$ we have
\[ 
z^2=a^2+2as \cdot e^{3\pi i/4}\vvv is^2
\]
We get an entire function
$\Phi(w)$ of a new complex variable defined by
\[
\Phi(w)= 
\int\uuu\Gamma\, \frac{e^{\vvv \pi iz^2+2\pi i w s}}
{e^{\pi iz}\vvv e^{\vvv \pi iz}}\,
dz\tag{*}
\]
In fact, this holds since 
the integrand contains 
$e^{- s^2}$ as a factor which  ensures the convergence for all complex $w$.
In � xx we dexribe how  Riemann obtained   the equation:
\[
\Phi(w)=\frac{e^{\pi iw}\vvv e^{\pi iw^2}}{e^{\pi iw}\vvv
e^{\vvv \pi i w}}\tag{**}
\]
More generally, if $\tau$ is another complex variable one has the analytic function
\[
\Theta(w,\tau)=
\int\uuu\Gamma\, \frac{e^{\pi i\tau z^2+2\pi i w s}}
{e^{\pi iz}\vvv e^{\vvv \pi iz}}\,
dz\tag{***}
\]
which is analytic when
$\mathfrak{Re}\,\tau<0$ and
$w\in {\bf{C}}$.
Residue calculus is used  while the integration contour is changed to
discover certain functional equations and attain
formulas for the $\Theta$\vvv function.
Let
us  remark that formulas such as (**) led Riemann to
his famous hypothesis about zeros of
the zeta\vvv function. See � x in Chapter VI for further comments.
Studies of more involved integrals 
were later  carried out by Weierstrass. 


\medskip








\noindent
{\bf{Multi\vvv valued integrands.}}
The complex log\vvv function and fractional powers $z^\alpha$ where
$\alpha$ is not an integer appear often as integrands
and during
integrations. 
A typical case is when $\sqrt{1\vvv x^2}$ appears in the integrand
and  
integration is  over the real interval $[\vvv 1,1]$.
Here one  starts from the
single\vvv valued analytic function
$g(z)$ defined  in ${\bf{C}}\setminus [\vvv 1,1]$
by
\[ 
g(z)= z\cdot\sqrt{1\vvv z^{\vvv 2}}
\]
To be precise,  consider the extended complex plane
where the point at infinity is added and then
$\Omega={\bf{C}}\cup\infty\setminus [\vvv 1,1]$
is simply connected which implies that there exists the
single\vvv valued branch of the square root of 
$1\vvv z^{\vvv 2}$, i.e.
\[ 
g(z)=z\cdot \sqrt{1\vvv z^{\vvv 2}}
\] 
is defined in the whole of $\Omega$ where $g$
has a simple pole at $\infty$.
If $z=iy$ is purely imaginary
then
\[
g(iy)= iy\cdot \sqrt{1+y^{\vvv 2}}
\]
This implies that
\[ 
g(iy)= i\cdot \sqrt{1+y^2}\quad\text{when}\quad y>0
\] 
while $g(iy)=- i\sqrt{1+y^2}$ when $y<0$.
The change of sign is crucial when
reside calculus is employed.
For example, consider
the  integral 
\[ 
J=\int\uuu{\vvv 1}^1\, \frac{dx}{\sqrt{1\vvv x^2}}\tag{*}
\]
Take  the $g$\vvv function above and
construct  the complex line integral over
the closed curve $\gamma$ which consists of the two line
segments from $\vvv 1+i\epsilon$ to $1+i\epsilon$ respectively
$\vvv i\epsilon$ to $1\vvv i\epsilon$ together with
two small half\vvv circles. See figure XX.
Using the change of sign and passing to the limit as $\epsilon\to 0$
it follows that
\[ 
2J=\vvv i\cdot \int\uuu {|z|=R}\, \frac{dz}{g(z)}
\] 
where we have taken a line integral over circles of radius $R>1$.
Passing to the limit as $R\to+\infty$ one easily verifies that
\[
J=\pi\tag{**}
\]
Let us remark that the last equality can be proved directly since
\[
2J=\int \uuu 0^1\, \frac{dx}{\sqrt{1\vvv x^2}}\tag{***}
\]
and here (***)  is computed via the variable substitution
$x\to\sin\,\theta$ which gives (**).
This example has   served to illustrate a method. In more involved cases
residue calculus is needed  to attain exact formulas.
\medskip

\noindent
{\bf{Exercise.}}
Let $0<a<1$ and consider the integral
\[ 
J=\int\uuu 0^1\, \frac{dx}{(1\vvv x)^a\cdot x^{1\vvv a}}
\]
In the domain ${\bf{C}}\setminus [0,1]$ there exists
the analytic function
\[ g(z)=z(1\vvv\frac{1}{z})^a
\]
Use that
$(\vvv 1)^a= e^{\pi i a}$ to show that
\[
J=\frac{\pi}{\sin (\pi  a)}
\]
Next, let $m\geq 3$ be an integer and put
\[ 
J\uuu m=\int\uuu 0^1\, \frac{dx}{1\vvv x^m)^{1/m}}
\] 
The substitution $x\mapsto t^{1/m}$
gives
\[
J\uuu m=\frac{1}{m}\int\uuu 0^1\, \frac{dt}{t^{1/m\vvv 1}\cdot (1\vvv t)^{1/m}}=
\frac{\pi}{m\cdot \sin(\pi/m)}
\]
Notice that the last term converges to 1 as $m\to+\infty$. The reader should
discover that  this limit is a consequence of Neper's limit formula for $e$.

\bigskip

\noindent
{\bf{Complex line integrals and homotopy.}}
Remove the three  points $0,1,\infty$ from 
the Riemann sphere. The  open complement $\Omega$
is not simply connected and  has a fundamental group
generated by a pair of closed and simple curves
$\Gamma\uuu 1,\Gamma\uuu 0$
where $\Gamma\uuu 1=\{|z\vvv 1|= 1/2$ and
$\Gamma\uuu 2=\{ |z|= 1/2\}$.
In algebraic topology one learns that
the fundamental group $\pi\uuu 1(\Omega$ is a free group generated
by the homotopy classes of these two curves.
It means for example that the composed curve
\[ 
\Gamma\uuu 1^{\vvv 1}\circ \Gamma\uuu 0^{\vvv 1}
\circ \Gamma\uuu 1\circ \Gamma\uuu 0\tag{*}
\]
is not homotopic to a trivial curve.
Even though this assertion is intuitively clear the formal proof
is not so easy. While algebraic topology was developed
by Poincar� and his contemporaries, a convincing method to prove
that (*) is not homotopic to a trivial curve 
is to perform a complex line integral of a suitable mult\vvv valued
analytic function and pursue its analytic continuation
by the Weierstrass' procedure.
In � xx we expose how this is done by an
explicit calculation of
a complex line integral along (*) using a special
multi\vvv valued function as integrand.

\bigskip


\noindent
{\bf{Example from physics.}}
Historically many residue formulas were established via equations 
derived by physical laws.
So here  one  encounters important  situations  where residue calculus
is needed. 
For example, Gauss used residue calculus  to establish 
equations in 
electro\vvv magnetic fields  governed by
the Biot\vvv Savart Law.
Another area where residue calculus appears naturally is hydromechanics.
The interested reader should consult the excellent text\vvv book 
[XXX]
by Horace Lamb which contains 
many instructive examples with  physical  background. For 
more advanced material related to quantum mechnaics
the reader will find applications of residue calculus in
the text\vvv book series by   L.D Landau and 
E.M. Lifschitz.
\medskip

\noindent
{\bf{The simple pendelum.}}
A fundamental   function  
appears  during the motion of
a simple pendelum.
Consider a particle $p$ of unit mass  attached at the end\vvv point of a 
rigid bar of some length $\ell$ whose other end is suspended at a fixed point while the bar and
$p$ oscillates in a vertical plane where gravity is the sole external force.
The system has one degree of freedom expressed by  the angle
$\theta$ between the bar and the vertical line which is directed downwards.
The kinetic energy of the one\vvv point system becomes
\[ 
T=\frac{\ell^2}{2}\cdot \dot\theta^2
\]
The equation of motion becomes
\[ 
\ell^2\ddot\theta=\vvv g\ell\cdot \sin\theta
\]
With  initial conditions $\theta(0)=0$ and $\dot\theta(0)= v>0$
the time\vvv dependent $\theta$\vvv function satisfies
the differential equation
\[
\dot\theta^2=
\frac{2g}{\ell}
\cdot (\cos \theta\vvv 1) +v^2
\]
We assume that $v$ us not to large, i.e.
\[ 
v^2<\frac{2g}{\ell}
\]
Then there exists $0<\theta^*<\pi/2$ such that
\[
\cos \theta^*=1\vvv \frac{\ell v^2}{2g}
\]
Now $t\mapsto \theta(t)$ oscillates between $ \vvv \theta^*$ and $\theta^*$.
The time for a quarter of the whole period, i.e. the time to reach
$\theta^*$ becomes
\[ 
T=\int\uuu 0^{\theta^*}\, \frac{d\theta}{\sqrt{\frac{2g}{\ell}
\cdot (\cos \theta\vvv 1) +v^2}}
\]
This formula shows that the determination of exact $T$\vvv values 
with varying intial velocity $v$ boils down to study the function
\[ \theta^*\mapsto
\int\uuu 0^{\theta^*}\, \frac{d\theta}{\sqrt{\cos\theta\vvv \cos\theta^*}}
\]

\noindent
Admitting the inverse $\arccos$\vvv function as "elementary" and using the 
substitution $\cos\theta\to x$ we are led to consider integrals of the form
\[ 
J(a)=\int\uuu a^1\, \frac{dx}{\sqrt{(1\vvv x^2)\cdot (x\vvv a)}}
\] 
where  $0<a<1$.
This $J$\vvv function belongs to  a 
class of functions   which 
were investigated
by   Legendre and Jacobi.
Let us also remark that not only the numerical value of $J(a)$ as 
$0<a<1$ is of interest here.
It turns out that this real\vvv analytic function defined on $(0,1)$ extends to
a multi\vvv valued analytic function in ${\bf{C}}\setminus \{0,1\}$
where it satisfies a  Fuchsian  differential equation
which  gives a further motivation for including $J(a)$ in a class of
"elementary functions".
\medskip

\noindent
{\bf{Example by Huyghens}}.
Let $p$ be a particle of unit mass which  moves on
the horizontal $(x,y)$\vvv plane where no friction is present and
gravity does not affect
the motion.
The particle slides on an infinite bar suspended at the origin which
can rotate and the bar has no mass. So
we have a particle systems with two degrees of freedom
where  the position of the mass point 
is given in polar coordinates
$(r,\theta)$. 
Here $\theta$ is the angle between the bar and the positive
$x$\vvv axis. At time $t=0$ we suppose the initial conditions are
\[ 
\theta(0)=0\quad\colon \dot \theta(0)=\omega\quad\colon
r(0)=A\quad\colon \dot r(0)=0
\] 
where $\omega$ and $A$ are positive.
in this situation we have Kepler's identity
\[
r^2\cdot \dot \theta=A^2\omega\tag{1}
\]
We also get the differential equation
\[
\dot r^2+\frac{A^4\omega^2}{r^2}=A^2\omega^2\tag{2}
\]
\medskip

\noindent
{\bf{Exercise.}}
Express $\theta$ as a function of $r$ and use this to prove that
the increasing time dependent function $\theta(t)$ has a limit
as $t\to+\infty$. More precisely a calculation gives the formula

\[ 
\lim\uuu {t\to\infty}\, \theta(t)=\int\uuu 1^\infty\,\frac{ds}{s\cdot \sqrt{s\vvv 1}}\tag{3}
\]
Huyghens, Newton and Wallis performed  used
series expansions to show that (3) is equal to $\pi$.
The reader is invited to prove this equality using residue calculus.
Thus, as time increases the bar moves from the position along the $x$\vvv axis to
positions which come closer to the positive $y$axis.
Notice that the limit formula is independent of the pair $\omega$ 
and $A$. The reader should contemplate upon this by reflecting over
daily life experience of  the centrifugal force.


 











\medskip

\noindent
{\bf{Fourier transforms.}}
On
the real $x$ line we have the  function
$\frac{1}{x\vvv  i}$. It is not integrable but
defines a tempered distribution $\mu$ whose  Fourier transform
is defined when $\xi\neq 0$ by the integral
\[ 
\widehat \mu(\xi)=
\int\uuu {\vvv \infty}^\infty\, \frac{e^{\vvv i\xi x}}{x\vvv i}\cdot dx\tag{1}
\]
If $\xi<0$ and $z=x+iy$ with $y>0$ the absolute value 
$|e^{\vvv i\xi z}|= e^{\xi y}$ decreases when
$y\to+\infty$. Using this fact the reader can verify that
\[
\widehat\mu(\xi)=2\pi i\cdot e^\xi
\] 
holds when $\xi<0$ and reversing the sign  verify that
$\widehat\mu (\xi)=0$ when $\xi>0$.
On the reader $\xi$\vvv line we have the tempered distribution defined by
$e^\xi$ when $\xi<0$ and zero if $\xi\geq 0$. Its inverse Fourier transform
becomes
\[
\frac{1}{2\pi}\cdot 2\pi i\int\uuu {\vvv \infty}^0\, e^{ix\xi}\cdot e^{\xi}\cdot d\xi=
\frac{i}{ix+1}=\frac{1}{x\vvv i}
\]
This confirms the calculations using
residues via Fourier's inversion formula.
\medskip

\noindent
{\bf{Fourier transforms on ${\bf{R}}^+$.}}
Let $0<a<1$ and $\beta$ is a complex number outside the
non\vvv negative real line.
When $\zeta=\alpha+is$ is complex
we set
\[ 
J(\zeta)= \int \uuu 0^\infty\, x^\zeta\cdot
\frac{x^a}{x\vvv \beta}\cdot \frac{dx}{x}\tag{*}
\]
Sincc $|x^{\alpha+is}|= x^\alpha$ when $x$ is real and positive
we see that (*) converges when
\[
a<\alpha<1\vvv a\tag{1}
\]
After the reader has become familiar with this calculus
it is an easy exercise to verify the equation:
\[ 
J\cdot \bigl[1\vvv e ^{2\pi i(\zeta+a\vvv 1)}\bigr]=
2\pi i\cdot \beta^{\zeta+a\vvv 1}
\]

\noindent{\bf{Specific examples.}}
Let $\beta=b$ where $b>0$ which gives
\[
\beta^{\zeta+a\vvv 1}= b^{\zeta+a\vvv 1}\cdot e^{\pi i(\zeta+a\vvv 1)}
\] 
where we used that $\vvv 1= e^{\pi i}$.
Suppose also that $\zeta=\alpha$ is real  which gives
\[
J\cdot \bigl[1\vvv e ^{2\pi i(\alpha+a\vvv 1)}\bigr]=
2\pi i\cdot b^{\alpha+a\vvv 1}\cdot e^{\pi i(\zeta+a\vvv 1)}
\]
Using the formula for the complex sine\vvv function the reader may verify that
\[
J=\pi\cdot \frac{b^{\alpha+a\vvv 1}}{\sin\, \pi(1\vvv a\vvv \alpha)}
\]
Since $1\vvv a\vvv\alpha>0$ is assumed the formula shows that
$J$ is real and positive which it should  be since the choice gives
\[
J=\int\uuu 0^\infty\,\frac{x^{\alpha+a}}{x+b}\cdot \frac{dx}{x}
\]
where the integrand is real and positive.
Next,  consider the case where $\zeta=is$ is purely imaginary
and $\beta=b$ with $b>0$ while $a=1/2$.
The general formula (xx) gives

\[ 
J\cdot [1\vvv e^{\vvv 2\pi s}\cdot e^{\vvv 2\pi i/2}]
=2\pi i\cdot b^{is\vvv 1/2}\cdot e^{\pi s}\cdot e^{\vvv \pi i/2}
\]
Since $e^{\vvv \pi i}=1$ and $e^{\vvv \pi/2}=i$ while $i^2=\vvv 1$
it follows that
\[
J(1+e^{\vvv 2\pi s})=
e^{\vvv \pi s}\cdot 2\pi\cdot b^{is\vvv 1/2}
\]
Introducing the complex cosine\vvv function we get the formula
\[
\int\uuu 0^\infty\, x^{is}\cdot \frac{\sqrt x}{x+b}\cdot \frac{dx}{x}=
\pi\cdot b^{is\vvv 1/2}\frac{1}{\cos(\pi i s)}
\]
\medskip

\noindent
{\bf{Remark.}}
The last equation yields a formula for the Fourier transform
of the $L^1$\vvv function  $\frac{\sqrt{x}}{x+b}$
on the multiplicative line ${\bf{R}}^+$ where $\frac{dx}{x}$ is the Haar measure.
Replace $is$  by the complex variable $\zeta$
and set
\[ 
J(\zeta)= \int\uuu 0^\infty\, x^\zeta\cdot \frac{\sqrt x}{x+b}\cdot \frac{dx}{x}
\]
Then the computations above show that
\[ 
J(\zeta)= \frac{\pi\cdot b^{\zeta\vvv 1/2}}{\cos (\pi \zeta )}
\]
Notice that the right hand side is an analytic function in the strip domain
$1/2<\mathfrak{Re} \zeta <1/2$ while we encounter
poles when
$\zeta=1/2$ or $\vvv 1/2$ whose appearance is clear from the integral which defines
$J(\zeta)$ because we get divergent integrals in these two cases.
At the same time (xx) gives a meaning to the integral (xx) for al complex
$\zeta$, i.e. the result is a globally defined meromorphic functions
with simple poles at the zeros of the complex cosine\vvv function.
This illustrates the usefulness of residue calculus since it was 
needed to get the precise formula (xx) above.























\medskip

\noindent{\bf{Principal values}}.
Consider the  
integral 
\[ 
J(a)=\int_0^1\frac{dx}{x-a}\tag{*}
\] 
where $0<a<1$.
The principal value 
is defined by:
\[ \lim_{\epsilon\to 0}\,\bigl[
\int_0^{a-\epsilon}\frac{dx}{x-a}
+
\int_{a+\epsilon}^1\frac{dx}{x-a}\,\bigr]\tag{**}
\]
When $0<\epsilon<a$ and $a+\epsilon<1$
we can evaluate both integrals and get:
\[
\int_0^{a-\epsilon}\frac{dx}{x-a}= -\log \epsilon+\log a
\quad\colon \quad
\int_{a+\epsilon}^1\frac{dx}{x-a}=
\log \epsilon-\log (1-a)
\]
where it is not even necessary
to perform a limit since
(**) takes the same value for all 
$0<\epsilon<a$.
In particular we get
the formula
\[ 
J(a)=\log \frac{1-a}{a}\tag{***}
\]
The construction  (**)
can be understood by complex integrals.
Namely,  for any real number $a>0$
there exists the complex log-function
\[ 
\log(z-a)
\] 
with a single valued branch
in the upper half-plane $\mathfrak{Im}(z)>0$
whose complex derivative is $\frac{1}{z-a}$.
For $\epsilon>0$
we can take the line integral on the horizontal line
from $i\epsilon$ to $1+i\epsilon$ which gives:
\[
\int_0^1\,\frac{dx}{x-a+i\epsilon}=\log{1-a+i\epsilon}-
\log{(-a+i\epsilon)}\tag{1}
\] 
Passing to the limit as $\epsilon\to 0$ the right hand side
becomes
\[ 
\log(1-a)-\log a-\pi i=\log\,\frac{1-a}{a}-\pi i\tag{2}
\]
To clarify this limit formula we
rewrite the left hand side in (1) which amounts to compute
\[
\int_0^1\,\frac{(x-a-i\epsilon)\cdot dx}{(x-a)^2+\epsilon^2}\tag{3}
\]
Separating real and imaginary parts it follows from (1) that
one has the two limit formulas:
\[
\lim_{\epsilon\to 0}
\int_0^1\,\frac{(x-a)\cdot dx}{(x-a)^2+\epsilon^2}=\log\,\frac{1-a}{a}
\quad\text{and}\quad
\lim_{\epsilon\to 0}
\int_0^1\,\frac{\epsilon\cdot dx}{(x-a)^2+\epsilon^2}=\pi\tag{4}
\]
\medskip

\noindent
{\bf{Exercise.}}
Clarify  why
the first formula in (4) agrees with the previously defined
principal value integral.
Prove also the second formula using the arctan-function.

\medskip


\noindent
{\bf{Another example.}}
Consider the  integral:
\[
J=\int_0^\infty\frac{1}{x}\cdot \log\bigl(\frac {|x+1|}{|x-1|}\bigr)\cdot dx\tag{1}
\]
The reader may  verify that the integrand in (1)
is a continuous function whose
value at $x=0$ is equal to $2$ and when
$|x|\to\infty$  the integrand decays as
$x^{-2}$.  So   we have an absolutely convergent integral.
Residue calculus  is used to compute the integral.
The idea is to consider the function
\[
 g(z)=\frac{1}{z}\cdot \log\,\frac{z+1}{z-1}
\] 
which is analytic in the upper half-plane.
\medskip

\noindent{\bf{Exercise.}}
For a large $R$ and a small $\epsilon>0$
we take the complex line integral of $g$
along the closed curve $\Gamma$
which consists of the real interval $[\epsilon,R]$, the
quarter circle $\{z=Re^{i\theta}$ where
$0\leq\theta\leq \pi/2$ and the imaginary interval $[i\epsilon,iR]$
and finally the small quarter circle of radius $\epsilon$.
Since $g(z)$ is analytic the
complex line integral over $\Gamma$ is
zero.
The reader may verify that
\[ 
\lim_{R\to\infty}\, \int_0^{\pi/2} g(Re^{i\theta})\cdot iRe^{i\theta}\cdot d\theta=0\tag{2}
\]
Next, the integral along the imaginary  line $\epsilon\leq y\leq R$
where the line integral taken in the opposite direction becomes
\[
-\int_\epsilon^R\, \log(\frac{iy+1}{iy-1})\cdot \frac{dy}{y}
\]
Since $|iy+1|=|iy-1|$ this integral is purely imaginary.
Regarding the real part and using (2) above the reader should verify that:
\[ 
J=\lim_{R,\epsilon}\, \int_\epsilon^R\,\frac{1}{x}\cdot \log\bigl(\frac {|x+1|}{|x-1|}\cdot dx
=\lim_{\epsilon\to 0}
\mathfrak{Re}\,\int_0^{\pi/2}\, 
g(\epsilon e^{i\theta})\cdot i\epsilon e^{i\theta}\cdot d\theta
\]
Notice that

\[
i\epsilon e^{i\theta}\cdot
g(\epsilon e^{i\theta})=
i\cdot \log\,\frac{\epsilon e^{i\theta}+1}{\epsilon e^{i\theta}-1}=
i\cdot \log\,1+\epsilon e^{i\theta})-i\cdot \log(-1+\epsilon e^{i\theta})=
\]
Now
\[ 
\lim_{\epsilon\to 0}\,\log\,(-1+e^{i\theta})=\pi\cdot i
\quad\text{and}\quad
\lim_{\epsilon\to 0}\,\log\,{\epsilon (1+e^{i\theta}-1})=0
\]
It follows that
\[
\lim_{\epsilon\to 0}\,
\int_0^{\pi/2}\, g(\epsilon e^{i\theta})\cdot i\epsilon e^{i\theta}
=\pi^2/2
\]
Hence the integral from (1) has the value:
\[
J=\frac{\pi^2}{2}
\]



\newpage


\centerline
{\bf{0:A Four  examples of residue calculcus.}}



\medskip

\noindent {\bf{0.1 Example.}}
Let $P(z)$ and $Q$ be two polynomials where
$\deg(P)\geq \deg(Q)+1$ and $-1<a<0$ is a real number.
Assume that $P$ has no zeros on the 
non-negative real axis and set:
\[
 J=\int_0^\infty\, \frac{x^a\cdot Q(x)}{P(x)}\cdot dx
\]
To find $J$  we consider the function $g(z)=\frac{z^a\cdot Q(z)}{P(z)}$
which is multi-valued outside the origin. The trick is to integrate
$g$ over a contour starting from
$x=\epsilon>0$ until $x=R$ is reached, followed by an integral taken over
the circle $|z|=R$ and after one returns from $R$ to $\epsilon$ on the $x$-axis and finish by 
an integral over the circle $|z|=\epsilon$ which  is performed clock-wise, i.e. in the negative direction. Passing to the limit as $R\to+\infty$ and
$\epsilon\to 0$ one uses the multi-valued behaviour of $z^\alpha$
and get
\[
(1-e^{2\pi i a})\cdot J=2\pi i\cdot
\sum\,\mathfrak{res}(g:\alpha_\nu)\tag{*}
\] 
where the sum is taken over the zeros of $P$.

\medskip

\noindent
{\bf{0.1.1 Exercise.}} Take $Q=1$ and $P=z-i$ above.
Then we get
\[
(1-e^{2\pi i a})\cdot J=2\pi i\cdot i^a= 2\pi i\cdot e^{\pi ia/2}
\]


\noindent
Consider   the case $a=-\epsilon$ with a  positive $\epsilon$.
Since $\frac{1}{z-i}= \frac{z+i}{z^2+1}$ 
we get
\[
J=\int_0^\infty\,\frac{x^{-\epsilon}(x+i)}{1+x^2}\cdot dx
\]
It is instructive to
check the equation (*) via (xx).
Separating the real and imaginary part the reader may verify the formula
\[
\int_0^\infty\,\frac{x^{1-\epsilon}}{1+x^2}\cdot dx=
\pi \cdot \frac{1+\cos (\pi\epsilon)}{(2+2\cos(\pi \epsilon)\cdot 
\sin(\pi \epsilon/2)}
\]
To check this formula we consider a limit as $\epsilon\to 1$.
Since $\cos \pi=0$ the reader may verify that the limit in the right hand side
becomes $\frac{\pi}{2}$ which is okay since we know from calculus that
\[
 \lim\uuu{\epsilon\to 0}\, 
 \int_0^\infty\,\frac{x^{1-\epsilon}}{1+x^2}\cdot dx=
 \int_0^\infty\,\frac{1}{1+x^2}\cdot dx=\frac{\pi}{2}
\]


\medskip

\noindent {\bf{0.2 Example.}}
Let $P,Q$ be a pair of polynomials where
$\deg(P)\geq \deg(Q)+2$
and consider the integral
\[
 J=\int_0^\infty\, \frac{Q(x)}{P(x)}\cdot dx
\]
To overcome the lack of a multi-valued integrand we use the complex 
log-function $\log z$ and define 
\[ 
g(z)=\frac{Q(z)}{P(z)}\cdot\log z
\]
Perform an integral of $g$ over the same contour as in the previous example.
After one turn around
$|z|=R$  $\log z$ has changed its branch with the constant
$2\pi i$ and since the "home run" is the integral from $R$ back to the origin
we get:
\[ 
J=-\sum\,\mathfrak{res}(g:\alpha_\nu)\tag{*}
\]
where the sum is taken over the zeros of $P$.
For example, take $Q=1$ and $P(z)=(z+a)(z+b)$ where $a> b>0$ are positive real numbers.
Then the  right hand side in (*) becomes
\[
-\bigl[\frac{\log a+\pi i}{b-a}+\frac{\log b+\pi i}{a-b}\bigr ]=
\frac{\log (a/b)}{a-b}
\]
It is always good to confirm a general formula. Take $b=1$ while $a=1+\epsilon$
and pass to the limit as $\epsilon\to 0$ which gives: 
\[ 
\int_0^\infty\frac{dx}{(x+1)^2}=
\lim_{\epsilon\to 0}\, \frac{\log (1+\epsilon)}{\epsilon}=1
\]
Next, consider the integral
\[
J=\int_0^\infty\frac{dx}{1+x^3}
\]
The polynomial $1+z^3$ has simple roots
$j\uuu 1=e^{\pi/3}, j\uuu 2=1,j\uuu 3= e^{5\pi/3}$.
The formula (*) above gives

\[
J=\vvv \frac{1}{3}\cdot \sum \frac{\log j\uuu \nu}{j\uuu \nu^2}
= \frac{1}{3}\cdot \sum j\uuu \nu\cdot \log j\uuu \nu
\]
where the last equality holds since $j\uuu \nu^3=1$ for each $\nu$.
The reader should check that the last sum becomes
\[
\frac{i}{3}(j\uuu 1\cdot \pi/3+j\uuu 2\cdot \pi+
j\uuu 3\cdot 5\pi/3)=\frac{i}{3}\cdot (i\sqrt{3}/2\cdot \pi/3\vvv 
i\sqrt{3}/2\cdot 5\pi/3)=\frac{2\pi}{3\cdot\sqrt{3}}
\]







\medskip

\noindent
{\bf{0.3 Example.}}
Let $a>0$  be real and put
\[
J=\int_{-1}^1\,\frac{dx}{(1+ax^2)\sqrt{1-x^2}}\tag{0.1}
\]
To compute this integral we use the analytic function
defined in the complement of the real interval $-1\leq x\leq 1$
by
\[ 
g(z)= z\cdot \sqrt{1-z^{-2}}
\]
The reader should verify the two limit formulas:
\[ 
\lim\uuu{\epsilon\to 0}\, g(x+i\epsilon)=
\cdot \sqrt{1\vvv x^2}\quad\colon\quad 1<x<1\tag{2}
\]
\[
\lim\uuu{\epsilon\to 0}\, g(x\vvv i\epsilon)=
\vvv i\cdot \sqrt{1\vvv x^2}\quad\colon\quad 1<x<1\tag{3}
\]
Now we consider the function
\[
f(z)= \frac{1}{(1+az^2)\cdot g(z)}
\]
It has to simple poles when $1+az^2=0$, i.e. we find purely imaginary
poles at plus and minus $i\cdot a^{\vvv 1/2}$.
Notice also that 
\[ 
\lim\uuu{R\to\infty}\, \int\uuu {|z|=R}\,
f(z)\cdot dz=0\tag{4}
\]
We use (4) and apply residue calculus while we integrate
over closed curves around $[\vvv 1,1]$ as illustrated by figure XX.
Using (2\vvv 3) it follows  that

\[
\frac{2}{i}\cdot J= 2\pi\cdot i\cdot \frac{2}{2a\alpha\cdot g(\alpha)}\tag{5}
\]
where $\alpha=i\cdot a^{\vvv 1/2}$ and we used that $g$ is odd when the two residues are
added. At this stage the reader can verify that
\[ 
J=\frac{\pi}{\sqrt{1+a}}
\]






\medskip


\noindent
{\bf{0.4 Example.}}
Consider the integral
\[ 
J=\int_{-\infty}^\infty\,\frac{e^{ax}}{e^x-iA}\cdot dx\tag{0.4}
\]
where $0<a<1$ and $A>0$ is real.
To find $J$ we use  the meromorphic function
$g(z)=\frac{e^{az}}{e^z-iA}$
which has simple poles when
$e^z=iA$. 
Consider the complex line integral taken over the boundary of 
a rectangle 
\[
\square= \{-R\leq x\leq R\}\times \{0\leq y\leq 2\pi\}
\]
The reader should verify that
$e^z\vvv iA$ has a simple zero at the point
$\log A+i\pi/2$
which therefore gives a simple pole of $g(z)$.
We have also $g(x+2\pi i)= e^{2\pi ia}\cdot g(x)$ when $x$ is real.
When $R\to+\infty$ we get a limit where  residue calculus  gives:
\[ 
(1-e^{2\pi i a})\cdot J=
2\pi i\cdot \mathfrak{res}\bigl(g(z):\log A+i\pi/2\bigr)=
2\pi i\cdot \frac{A^a\cdot i^a}{iA}=2\pi A^{a-1}\cdot e^{a\pi i/2}\tag{*}
\]






\bigskip









\centerline {\bf{0:B  Summation formulas.}}

\medskip

\noindent
Various sums are often computed  using 
the meromorphic function:
\[ 
g(z)=\frac{\cos\pi z}{\sin \pi z}\tag{*}
\]
It has simple poles at all integers with
the common residue  $\frac{1}{\pi}$.
Consider a pair of polynomials $P,Q$ 
where $\deg{P}\geq 2+\deg(Q)$ and $P(n)\neq 0$
hold at all integers.
Residue calculus gives the summation formula
\[
\frac{1}{\pi}\cdot \sum_{-\infty}^\infty\,\frac{Q(k)}{P(k)}=
\sum\,\mathfrak{res}(g(z)\cdot\frac{Q(z)}{P(z)}:\alpha_\nu)\tag{**}
\] 
where the right hand is the sum of residues over all zeros of $P$.
As an illustration we  take $P(z)=z^2+1$ which has simple zeros at
$i$ and $-i$. Since
\[ 
g(z)= i\cdot \frac{e^{iz}+e^{-iz}}{e^{iz}-e^{-iz}}
\]
a  computation shows that the right hand side in (**) becomes
$\frac{e^\pi+e^{-\pi}}{e^\pi-e^{-\pi}}$. Hence
\[ 
\sum_{-\infty}^\infty\, \frac{1}{1+k^2}=\pi\cdot 
\frac{e^\pi+e^{-\pi}}{e^\pi-e^{-\pi}}
\]
{\bf{0.B.1 Exercise.}}
Let $\alpha$ be a complex number which is not an integer.
Show that
\[\sum_{k=-\infty}^{\infty}\, \frac{1}{k+\alpha)^2}=
\frac{\pi^2}{\sin^2\pi\alpha}
\]
{\bf{0.B.2 Exercise.}}
Certain summation formulas can  be established directly without residue calculus.
Consider the meromorphic function 
\[ 
g_*(z)=\frac{\pi}{\sin\pi z}
\]
It has simple poles at all integers and we can write out an
infinite sum of rational functions which will match these poles.
Namely, consider
\[
g(z)=\frac{1}{z}+
\sum_{n=1}^\infty\,(\frac{1}{z-2n}+\frac{1}{z+2n})-
\sum_{n=0}^\infty\,(\frac{1}{z-2n-1}+\frac{1}{z+2n+1})
\]
It is easily seen that $g_*-g$ has no poles
and the reader should verify that this entire
is bounded and hence a constant and
finally that this constant is zero.
We can
express $g$ via a series where the convergence for $z$-values outside the set of integers
is expressed more directly, i.e. one has
\[
\frac{\pi}{\sin\pi z}
=\frac{1}{z}+
\sum_{n=1}^\infty\,\frac{2z}{z^2-4n^2}-
\sum_{n=0}^\infty\,\frac{2z}{z^2-(2n+1)^2}\tag{***}
\]
For example, with $z=1/4$ one gets
\[
\sqrt{2}\cdot\pi=4-\sum_{n=1}^\infty\,\frac{8}{48n^2-1}+
 \sum_{n=0}^\infty\,\frac{8}{48(2n+1)^2-1}
\]
The right hand side is an infinite sum of
rational numbers while the trancendental number $\pi$ appears in the left
hand side. So the formula  is quite remarkable.
\medskip

\noindent
{\bf{0.B.3 Wallis limit formula.}}
There exists a meromorphic function with simple poles at all integers
defined by the series
\[ 
g_*(z)=\frac{1}{z}+\sum_{n=1}^\infty\, \frac{2z}{z^2-j^2}
\]
At the same time we have the function
from (*) which also has simple poles at the integers
with residues $\frac{1}{\pi}$.

\medskip

\noindent
{\bf{0.B.4 Exercise.}} Show that
\[
\frac{\cos \pi z}{\sin\pi z}=\frac{1}{\pi}\cdot g_*(z)\tag{i}
\]
Next, use that the derivative of $\sin \pi z$ is 
equal to $\cos\pi z$ and deduce the product formula
\[
\sin\pi z=\pi z\cdot \prod_{n=1}^\infty\, \bigl(1-\frac{z^2}{n^2}\bigr )\tag{ii}
\]
Next, take $z=1/2$ in (ii).
If $N\geq 2$ we consider partial products in the right hand side
which gives the limit formula
\[ 
\lim_{N\to\infty}\, 
 \prod_{n=1}^{n=N}\, \bigl(1-\frac{1}{4\cdot n^2}\bigr)=
 \frac{2}{\pi}\tag{iii}
 \]
{\bf{0.B.5 Exercise.}} Rewrite the product and show that (iii) entails Wallis' limit formula:

\[
\sqrt{\frac{\pi}{2}}= \lim_{N\to\infty}\, \frac{2\cdot 4\cdots 2N}{1\cdot 3\cdot 5\cdots(2N-1)}
 \cdot \frac{1}{\sqrt{2N+1}}
 \]
 

 



\medskip




\centerline{\bf{0.C Asymptotic expansions.}}
\medskip

\noindent
Residue calculus is often 
used to find  asymptotic formulas. 
We describe a result of this nature.
Let $\{\lambda_\nu\}$ 
be a strictly increasing sequence of positive real numbers
and $\{a_\nu\}$ some sequence of positive real numbers.
Assume that there exists some positive number $r_*$ such that 
\[ 
f(x)=\sum_{\nu=1}^\infty\,\frac{a_\nu}{\lambda_\nu+x}
\] 
is convergent for all $x>r_*$. 
To each $x>0$ we denote by $\omega(x)$
the largest integer $\nu$ such that $\lambda_\nu<x$.
\medskip

\noindent
{\bf{0.C.1 Theorem.}}
\emph{Suppose  that
the following limit formula holds for some
$0<\alpha<1$ and a constant $A$:}
\[ 
\lim_{x\to+\infty}\, x^{-\alpha}\cdot f(x)=A\tag{1}
\]
\emph{Then it follows that}
\[ 
\lim_{x\to+\infty}\, x^{1-\alpha}\cdot \sum_{\nu=1}^{\nu=\omega(x)}\,a_\nu=
\frac{A}{\pi}\cdot \frac{\sin \pi\alpha}{1-\alpha}\tag {2}
\]

\noindent
The proof  requires 
Fourier analysis and  Wiener's
general Tauberian theorem.
So here more advanced methods are needed but
residue calculus is used to compute
the  value of this limit.







\bigskip


\centerline {\bf{0.D Ugly examples.}}
\medskip

\noindent
There are  situations where
an integral cannot be expressed in an elementary fashion even
if it is defined by elementary
functions. For example,  consider
the integral
\[
\int_0^1\, \frac{e^x}{1+x}\cdot dx\tag{1}
\]
With $z=x+iy$
the function $g(z)= \frac{e^z}{z+1}$
is analytic in the half space
$\mathfrak{Re} z >-1$.
The line integral of $g$ along rectangles
$\{0\leq x\leq 1\}\times0\leq y\leq R\}$ is zero and
after a passage to the limit when $R\to+\infty$ we see that (1) is equal to
\[
\int_0^\infty\,\frac{e^{is}}{1+is}\cdot ids-
\int_0^\infty\,\frac{e^{1+ is}}{2+is}\cdot ids\tag{2}
\]
The conclusion is that (1) can be calculated by an "exact formula"
 if we can handle the integrals:
\[ 
 J(a)=
 \int_0^\infty\,\frac{e^{is}}{s-ia}\cdot ds\quad\colon a>0\tag{3}
\]
Here one encounters an annoying
fact. If we instead consider the integral
\[ 
J^*(a)=  \int_{-\infty}^\infty\,\frac{e^{is}}{s-ia}\cdot ds\quad\colon a>0\tag{4}
\] 
then there is no problem to compute it. In fact, we shall learn that 
the value of (4)
is found by ordinary residue calculus and  becomes
$2\pi i\cdot e^{-a}$,
obtained from a residue at $z=ia$ when we consider the
function $g(z)=\frac{e^{iz}}{z-ia}$ in the upper half-plane where it has a simple pole
at $z=ia$.
But (3) cannot be found in this simple fashion.
After  the substitution $s\mapsto a\xi$ we see that (3) is equal to
\[
J(a)=\int_ 0^\infty\frac{e^{ia\xi}}{\xi-i}\cdot d\xi\tag{5}
\] 
Apart from the factor $\frac{1}{2\pi}$
this is an inverse Fourier transform of the tempered
distribution on the real $\xi$-line which is supported by $\{\xi\geq 0\}$
given by the density
$\frac{1}{\xi-i}$ on $\xi>0$.
This illustrates a close interplay between
Fourier transforms
and the 
calculations of various integrals.
Let $a$ be replaced by $x$ to
indicate that $J(x)$ is a function of $x$ 
which  is defined on $x>0$ but becomes
a tempered distribution on the real $x$-line
via Fourier's inversion formula.
It follows for example that the distribution $J$ satisfies
the differential equation
\[ 
\partial_x(J)+J=2\pi i\cdot H^*\tag{6}
\] 
where $H^*$ 
is the inverse Fourier transform of the Heaviside distribtution on the $\xi$-line
which is 1 if $\xi\geq 0$ and zero on $\{\xi<0\}$.
In XX we  return to a study of
the $J$-distribution and get a certain formula for
the evaluation of the integral in (4).

\medskip


\noindent
Let us now turn to  "nice" situations and begin with
some general formulas which are
used  in residue calculus.


 


\bigskip

\centerline {\bf{0:E Fractional decompostion}}
\medskip

\noindent
The vanishing
below  holds for every pair of polynomials
$p,q$
if
$\deg(p)\geq\deg(q)+2$:
\[ 
\lim_{R\to\infty}\, \int_{|z|=R}\, \frac{q(z)\cdot dz}{p(z)}=0\tag{*}
\]


\noindent
A second useful formula 
is  the fractional decomposition:
\[
\frac{1}{p(z)}=\sum_{k=1}^{k=n}\, \frac{c_k}{z-\alpha_k}
\quad\text{where}\quad
c_k=\frac{1}{p'(\alpha_k)}
\]
where
\[ 
p(z)=\prod_{k=1}^{k=n} \,(z-\alpha_k)\quad\text{has simple zeros}
\] 


\medskip

\noindent
{\bf{0.E.1 Exercise.}}
Show that (*) applied with $q(z)=1$ gives:
\[ 
c_1+\ldots+c_n=0 \tag{1}
\]
Next, let $1\leq \nu\leq n-1$ and show that one has the fractional decomposition
\[
\frac{z^\nu}{p(z)}=\sum_{k=1}^{k=n}\, \frac{\alpha_k^\nu}{p'(\alpha_k)}
\cdot\frac{1}{z-\alpha_k}\tag{2}
\]

\bigskip








\centerline {\bf{0:F Computing local residues.}}

\medskip

\noindent
When multiple zeros occur  local calculations are needed to find  residues.
The typical case is as follows: We have an analytic function $f(z)$
defined in disc centered at $\{z=0\}$
and with a zero of order $k\geq 2$ at the origin.
Now
\[ 
\frac{1}{f(z)}= c_kz^{-k}+\ldots+c_1z^{-1}+d_0+d_1z+\ldots
\]
Here $c_1$ is the residue coefficient.
In practice an expansion
\[
f=bz^k(1-(b_1z+b_2z^2+\ldots))
\]
is known from the start. To find $c_1$ therefore amounts to find
the coefficient of $z^{k-1}$ in the power series 
\[
\frac{1}{1-(b_1z+b_2z^2+\ldots)}= 1+w_1z+w_2z^2+\ldots\tag{*}
\]
In (*) we can take $\{b_\nu\}$ to be arbitrary
and seek for algebraic expressions of the $w$-numbers.
This leads to every integer $k\geq 1$ to a certain polynomial 
$R_k$ of the $b$-variables.
We see for example that
\[
b_1=w_1\quad \colon\quad b_1^2+b_2=w_2
\quad \colon \quad b_1^3+b_1b_2+b_3=w_3\tag{1}
\]


\noindent
{\bf{0.F.1 Exercise.}} 
Show that for every $k\geq 1$ there exists a
polynomial of the form
\[ 
R_k(b_\bullet)= \sum\, \rho_{i_1\ldots i_m}b_1^{i_1}\cdot b_m^{i_m}\tag{**}
\]
where $1\leq m\leq k$ holds in each term and
\[ 
i_1+2i_2+\ldots+ki_k=k
\] 
hold for every $k$-tuple of the non-negative
$i$-numbers.
Use also  (**)  to continue the computations in (1) above
for higher $k$-values. One has for example
\[
b_1^4+3b^2_1b_2+b_1b_3+b_2^2+b_4=w_4
\]


\noindent
Employ a computer to 
extend the result to get exact formulas for
a  set of positive integers, say up to $k=50$. Notice that
all the $\rho$-coefficients in (**) are positive integers.
\medskip

\noindent
{\bf{0.F.2 Exercise.}}
Let $g(z)$ be a meromorphic function with a pole of
order $k$ at $z=0$. Then $z^k\cdot g(z)$ is holomorphic
at the origin. Show that the residue of $g$ given by the coefficient
$c_1$ of $z^{-1}$ in the Laurent expansion is given by
\[ 
\frac{1}{(k-1)!}\cdot \partial^{k-1}(z^k\cdot g)(0)\tag{1}
\]

\noindent
Take for example $g(z)=\frac{1}{\sin^3 z}$ which has a triple pole a
$z=0$. We write

\[ 
\sin z=z(1-z^2/3!+z^4/5 !-\ldots)) z\cdot \rho(z)
\]
By (1) the residue becomes

\[
\frac{1}{2}\partial^2(\frac{1}{\rho(z)}=
-\frac{1}{2}\cdot\partial (\frac{\rho'(z)}{\rho^2(z)})
\] 
Now the reader can verify that the residue becomes $\frac{1}{6}$.




\newpage

\centerline
{\bf{0.G Line integrals of multi-valued functions.}}
\bigskip

\noindent
Let $\Omega$ be a connected domain in
${\bf{C}}$
and $f_*$ is a germ of a multi-valued analytic 
function at some point $z_*\in\Omega$. Let
$\gamma$ be a curve which starts at $z_*$ and stays in
$\Omega$. The end-point $z^*$ of $\gamma$
can be equal, to $z_*$, i.e. we do not exclude the case when
$\gamma$ is closed.
Now $f_*$ can be extended 
in the sense of Weierstrass along 
 $\gamma$ and
using  its analytic continuation along $\gamma$ the line integral below is defined:
\[ 
\int_\gamma\ f\cdot dz\tag{1}
\] 
\medskip

\noindent
{\bf{Exercise.}}
Use the monodromy theorem to show that
if $\gamma$ and $\gamma^*$ are 
curves
starting at $z_*$ with the same end-point $z^*$ and   homotopic in 
a  family of curves which stay in $\Omega$ and have
end-points at $z_*$ and $z^*$,
then the integral (1) taken 
over $\gamma$ or $\gamma^*$ are equal.
\medskip

\noindent
{\bf{0.G.1 An application.}}
Remove  the to points $-1$ and $+1$ from
${\bf{C}}$  which gives the domain
$\Omega={\bf{C}}\setminus\{-1,1\}$ and
consider the multi-valued function
\[ 
f(z)= \sqrt{1+z}\cdot (1-z)^a\quad\text{where}\quad 0<a<1
\]
whose   local branch $f_*$ at the origin is chosen so that $f_*(0)=1$.
Let $\gamma_1$ be the closed curve
at the origin which follows the circle $\{z-1|=1\}$
and is oriented in the counter-clockwise direction.
Similarly, $\gamma_2$ is the closed curve which now follows the
circle $\{|z+1|=1\}$ in the counter clock-wise direction.
We have also the closed curves $\gamma_1^{-1}$ and $\gamma_2^{-1}$
with reversed orientation.
Now we get the composed closed curve
\[ 
\gamma= \gamma_2^{-1}\circ\gamma_1^{-1}\circ \gamma_2\circ\gamma_1\tag{1}
\]
\noindent
Along $\gamma_1$ we have $z=1+e^{i\theta}$
and the line integral  becomes
\[
J_1=\int_0^{2\pi}\, \sqrt{2+e^{i\theta}}\cdot e^{ia\theta}\cdot ie^{i\theta}\cdot d\theta\tag{i}
\]
Next, when the integral over $\gamma_2$ is computed we have
performed an analytic continuation of $f$ along $\gamma_1$
which means that we have a new local branch of $f$ at $z=0$ which takes the value 
$e^{2\pi ia}$.
So along $\gamma_2$ we get
the contribution
\[
J_2=
e^{2\pi ia} \cdot \int_0^{2\pi}\, e^{ia\theta}\cdot
\sqrt{-2-e^{i\theta}}\cdot ie^{i\theta}\cdot d\theta=
\]
\[
i\cdot e^{2\pi ia} \cdot \int_0^{2\pi}\, e^{ia\theta}\cdot
\sqrt{2+e^{i\theta}}\cdot ie^{i\theta}\cdot d\theta=i\cdot e^{2\pi ia} \cdot J_1
\tag{ii}
\]
For the integral along $\gamma_1^{-1}$
we  start with a local branch where $f(0)=-e^{2\pi i a}$
and since this line integral  performed in the
clockwise direction the contribution becomes
\[
J_3=e^{2\pi i a}\cdot J_1
\]
Finally, the local branch of $f$ at $z=0$ when we start integration along $\gamma_2^{-2}$
is minus one
and we see  the contribution of the last line integral becomes
\[
J_4=e^{-2\pi i a}J_2=i \cdot J_1
\]
From this we conclude that
the line integral of $f$ taken over $\gamma$ is equal to
\[
(1+i)(1+e^{2\pi i\alpha})\cdot J_1\tag{*}
\]
The reader may verify that
$J_1\neq 0$ and hence the line integral of $f$ over the closed curve
$\gamma$ is non-zero.
The exercise above therefore shows that $\gamma$ cannot be homotopic to
the trivial curve which stays at the origin in $\Omega$.
This means
that the image
$\{\gamma\}$ in the fundamental group
$\pi_1(\Omega)$ is non-zero, i.e.
the homotopy classes $\{\gamma_1\}$ and $\{\gamma_2\}$ do not commute in  this group.
\medskip

\noindent
{\bf{Remark.}}
The example above  shows how  to
a classic result in topology using
complex line integrals. The topological result  is that
the fundamental group $\pi_1(\Omega)$ is a free group generated
by
the homotopy classes of $\gamma_1$ and $\gamma_2$.
That this indeed holds can be proved integrating
 multi-valued functions of the form along composed closed $\gamma$-curves.
 \[ f(z)=(z-1)^a\cdot (z+1)^b
\] 
where $a,b$ can be arbitrary pairs of complex numbers.
\medskip

\noindent{\bf{0.G.2 Exercise.}}
Let $0<a<1$ be a real number. Consider the multi-valued function
$f(z)= z^a\cdot\log z$ defined outside the two points 0 and 1.
Let $R>1$ and at $z=R$ we choose the local branch $f_*$ where
$f_*(R)= R^a\cdot \log R$ is real and positive.
Calculate the line integral
\[
\int_\gamma\, f\cdot dz
\] 
where $\gamma$ is the circle $\{|z|=R\}$
oriented in the counter-clock wise sense.

\medskip


\noindent
{\bf{0.G.3 Example.}} Let $a>0$ be real and consider the integral:
\[
J=\int_0^1\,\frac{dx}{(1+ax^2)\sqrt{1-x^2}}\tag{*}
\]
To evaluate this integral we use the fact that
there exists a \emph{single-valued} analytic function
$\sqrt{1-z^2}$ in ${\bf{C}}\setminus [0,1]$.
Choose a closed contour formed by a 
the line segment where $y=\epsilon$ and $0\leq x\leq 1$
plus a small half circle around $1$ and  return along the line
$y=-\epsilon$ while $x$ moves from 1 to zero and finish with a small half-cirlce from
$-i\epsilon$ to $i\epsilon$. Notice
that the line integral over
large circles $|z|=R$ of
$\frac{1}{(1+az^2)\sqrt{1-z^2}}$ tend to zero. Now Cauchy's
residue formula  gives
\[
(1-i)\cdot J=
2\pi i\cdot\mathfrak{res}((1+az^2)\sqrt{1-z^2}:\frac{i}{a})+
2\pi i\cdot \mathfrak{res}((1+az^2)\sqrt{1-z^2}:\frac{-i}{a})
\]
Here a computation gives the equality:
\[
J=\frac{\pi}{2\sqrt{1+a}}
\]
\medskip


\centerline {\bf{0.H Solving a differential equation.}}
\medskip

\noindent
Line integrals of multi-valued functions
are also used in other situations. 
Here follows an  example
from chapter VII in the text-book [Cartan]
where we remark that the original constructions are due to Laplace.
Let $n\geq 2$ and consider the differential equation
\[
(a_nz+b_n)\cdot y^{(n)}(z)+\ldots+ (a_1z+b_1)\cdot y'(z)+ (a_0z+b_0)\cdot y(z)=0\tag{*}
\]

\noindent
Here $\{a_k\}$ and $\{b_k\}$ are complex constants with $a_n\neq 0$.
Define the two polynomials
\[ 
A(z)= \sum\, a_kz^k\quad\text{and}\quad B(z)=\sum\, b_kz^k\tag{1}
\]

\noindent
Assume that the zeros of $A$ are simple and denote them by
$c_1,\ldots,c_n$.
Under this assumption 
the set of entire functions which solve (*) is a complex vector space of dimension
$(n-1)$.
To find these solutions we use the fractional decomposition and write
\[
\frac{B(z)}{A(z)}= \alpha+\frac{\alpha_1}{z-c_1}+\ldots+
\frac{\alpha_n}{z-c_n}\tag{2}
\]
Next, define the function
\[ 
U(z)= e^{\alpha z}\cdot \prod\, (z-c_k)^{\alpha_k}\tag{3}
\]
Since $\{\alpha_k\}$ in general are not integers this $U$-function is multi-valued.
Outside the zeros of $A$ we notice that
one has the equality
\[
\frac{U'(z)}{U(z)}= \frac{B(z)}{A(z)}\tag{4}
\]
This will be used to construct solutions to (*).
Namely,  fix a point $z_0\in {\bf{C}}\setminus \{c_k\}$
and
in a small disc centered at $z_0$ we choose a local branch of $U$.
Next, let $\gamma$ be a closed curve which stays
in 
${\bf{C}}\setminus \{c_k\}$ and has $z_0$ as a common start and end-point.
For each complex number $z$
we can evaluate the
line integral and get a function
\[
f(z)=\int_\gamma\, e^{z\zeta}\cdot \, \frac{U(\zeta)}{A(\zeta)}\cdot d\zeta\tag{5}
\]
It is clear that $f$ is an entire function of $z$
and each complex derivative is given by:
\[
f^{(k)}(z)=\int_\gamma\, e^{z\zeta}\cdot \, \zeta^k\cdot \frac{U(\zeta)}{A(\zeta)}\cdot d\zeta
\]
So the  construction of the polynomials $A$ and $B$
show that $f$ is a solution to the differential equation (*) if
\[
\int_\gamma\, e^{z\zeta}\cdot [z\cdot A(\zeta)+B(\zeta)]
\cdot \frac{U}{A}(\zeta)\cdot d\zeta=0\tag{**}
\] 
where the equality holds for all $z$.
\medskip

\noindent
{\bf{A partial integration.}}
Since $\partial_\zeta(e^{z\zeta})= z\cdot e^{z\zeta}$
it follows that
\[
\int_\gamma\, e^{z\zeta}\cdot z
\cdot U(\zeta)\cdot d\zeta=
e^{z\zeta}\cdot U(\zeta)|
_{\gamma_*}^{\gamma^*}-
\int_\gamma\, e^{z\zeta}\cdot  U'(\zeta)\cdot d\zeta\tag{6}
\]


\noindent
At the same time
(4) above gives the equality $U'=\frac{BU}{A}$ and we conclude that
(**) holds if and only if
\[
e^{z\zeta}\cdot U(\zeta)|
_{\gamma_*}^{\gamma^*}=0\tag{7}
\]
By the construction of the line integral  where the multi-valued function
$U$ appears, (7) means  that
the after analytic continuation along $\gamma$ one has the equality
\[
T_\gamma(U)(z_0)=U(z_0)\tag{***}
\]

\noindent
If we want that the $f$-function in (5) is not identically zero
we must choose closed curves $\gamma$ which are not trivial, i.e.
homotopic to the constant curve at $z_0$, and at the same time
(***) should hold.
To achieve this we
consider for each $1\leq k\leq n$
a simple closed curve
$\gamma_k$ at $z_0$ whose winding number with respect to
$c_k$ is equal to one, while the winding number with respect to the remaining
$c$-rots are zero.
This means that the homotopy classes of $\gamma_1,\ldots,\gamma_n$ 
generate the free group $\pi_1({\bf{C}}\setminus\{c_k\})$.
Notice also that
\[ 
T_{\gamma_k}(U)(z_0)= e^{2\pi i\cdot \alpha_k}\cdot U(z_0)\tag{8}
\] 
hold for every $k$.
To satisfy (***) we introduce the following $(n-1)$-tuple of closed curves:
\[ 
\gamma^*_k=\gamma_k^{-1}\circ\gamma_1^{-1}\circ
\gamma_k\circ\gamma_1\quad\colon\, 1\leq k\leq n\tag{9}
\]
With this choice we have
\[
T_{\gamma^*_k}(U)(z_0)= e^{-2\pi i\cdot \alpha_k}\cdot
e^{-2\pi i\cdot \alpha_1}\cdot e^{2\pi i\cdot \alpha_k}\cdot e^{2\pi i\cdot \alpha_1}\cdot
U(z_0)=U(c_0)\tag{10}
\]
Hence the differential equation is solved by the functions
\[
f_k(z)=\int_{\gamma^*_k}\, e^{z\zeta}\cdot \, \frac{U(\zeta)}{A(\zeta)}\cdot d\zeta
\quad\colon 2\leq k\leq n\tag{11}
\]
\medskip

\noindent
There remains to prove that
the functions above are ${\bf{C}}$-linearly independent
and give a basis
for the entire solutions to (*). The fact that the 
complex vector space of entire 
solutions to (*) has dimension $n-1$ at most
can be proved in several ways. One is to apply results from
$\mathcal D$-module theory. See (xx).
Less obvious that $f_2,\ldots,f_n$  are ${\bf{C}}$-linearly independent.
To see this we suppose that one has a relation
$q_2f_2(z)+\ldots+q_nf_n(z)=0$ where $\{q_k\}$ are complex constants.
This is an identity for all $z$ and by expanding $e^{z\zeta}$
it would follow that:
\[
\sum_{k=2}^{k=n}\, q_k\cdot \int_{\gamma^*_k}\, 
\zeta^m\cdot \frac{U(\zeta)}{A(\zeta)}\cdot d\zeta=0
\quad\text{for all}\quad m=0,1,\ldots\tag{12}
\]

\medskip


\noindent
Now the homotopy classes of the $\gamma^*$-curves are
different in the fundamental group and using this
one can show that (12) implies that all the $q$-numbers are zero which gives
the requested
${\bf{C}}$-linear independence of the $f$-functions.
See also � xx where we go further and construct distribution solutions to (*) which
in partocular calrifies the calim that
the space of entire solutions is $(m-1)$-dimensional.

 



\bigskip











\centerline {\bf{0:I. More involved integrals.}}

\bigskip

\noindent
Even though standard residue calculus settles 
a quite extensive family of integrals there remain integrals where the
evaluation is more cumbersome and eventually force us to employ numerical calculations.
Consider for example the integral
\[
J(a)=\int_0^1\frac{dx}{x^a\cdot (1+x)^a}\tag{*}
\] 
where $0<a<1$.
We find a series for the solution using the expansion
\[ 
(1+x)^{-a}=\sum_{n=0}^\infty c_n(a)\cdot x^n\quad\colon 0<x<1\implies
\]
\[ 
J(a)=\sum_{n=0}^\infty c_n(a)\cdot\frac{1}{n+1-a}\tag{**}
\]
Recall also that
\[
c_n(a)= (-1)^n\cdot \frac{a(a+1)\cdots(a+n-1)}{n !}\quad\colon
n=1,2,\ldots
\]
So above we have an explicit series and it is a matter of taste if one includes the $J$-function
which evaluates the integral among the "elementary functions".
The $J$-integral above is related to  integrals of the form
\[
I(a)= \int_{-\infty}^\infty\, \frac{dx}{(1+x^2)^a}\tag{***}
\]
which exist when $a$ is real and $>1/2$.
To solve (***) it is tempting to consider the multi-valued analytic function
$g(z)=(1+z^2)^{-a}$.
If $R>1$ we get the simply connected domain
$\Omega_R$ which is the upper disc $\{|z|<R\}$
where $\mathfrak{Im}(z)>0$ and the imaginary interval 
$[0,i]$ is removed.
In this domain there exists a single-valued branch of the
$g$-function which admits a factorisation
\[ 
g(z)= (z+i)^{-a}\cdot (z-i)^{-a}
\]
The whole line integral
\[
\int_{\partial\Omega_R}\, g(z)\cdot dz=0
\]
On the portion of $\partial\Omega_R$ given by the half-circle of radius $R$
we get a vanishing integral as $R\to+\infty$.
On the portion on the real $x$-axis we must
take into the account that the restriction of $g$ to the negative real axis has changed.
More precisely we have performed an analytic continuation of
$(z-i)^{-a}= e^{-a\log(z-i)}$
and the effect is that 
\[
 g(x)= e^{-\pi a i}\cdot\frac{1}{1+x^2}\quad\colon x<0
 \]




\noindent
{\bf{0.I.1 Exercise.}}
Conclude from the above that
\[
 (1+e^{-\pi a i})\cdot \frac{I(a)}{2}=\int_\Gamma\, g(z)\cdot dz\tag{1}
 \] 

\noindent
where $\Gamma$ is the contour give by two copies of the imaginary
 interval $[0,i]$.
As illustrated by a picture
the portion of the complex line integral of $g(z)\cdot dz$
on the "positive side"  is taken in the negative direction
and therefore contributes with the term:
\[ 
 -\int_0^1\, \frac{idy}{(1-y^2)^a}\tag{2}
 \]
 On the "negative side" the $g$-function has a new branch
and taking the negative direction into the account
 the contribution to  the complex line integral $\int_\Gamma\, g(z)\cdot dz$
becomes
 \[
e^{-\pi a i}\cdot \int_0^1\,\frac{idy}{(1-y^2)^a}\tag{3}
\]


\medskip

\noindent
{\bf{Conclusion.}}
From (1-3) above we obtain
 \[
(1+e^{-\pi a i})\cdot \frac{I(a)}{2}=(1-e^{-\pi a i})\cdot i\cdot
\int_0^1\,\frac{dy}{(1-y^2)^a}
 \]
Multiply both sides with $e^{\pi i a/2}$. Then the reader can verify that
\[ 
\cos(\pi a/2)\cdot I(a)= 2\cdot\sin(\pi a/2)\cdot \int_0^1\frac{dy}{(1-y^2)^a}\tag{*}
\]

\medskip


\noindent
{\bf{0.I.2 Exercise.}}
To each $0<a<1$
we have the integral
\[
L(a)=\int_0^1\,\frac{dy}{(1-y^2)^a}
\]
By the substitution $y\mapsto\cos\theta$ the integral becomes
\[
\int_0^1\,(\cos \theta)^{(1-a}\cdot d\theta
\]
From a numerical point of view this is a robust integral since
the integrand is a continuous function and
the reader should check numerical values with a computer
as $0<a<1$ varies.



\bigskip



\centerline{\bf{0.J Abel integrals and equations}}
\medskip

\noindent
Residue calculus can be used to
solve   integral equations.
Given $0<a<1$ we consider the  equation
\[ 
\int_0^1\, \frac{\phi(y)}{|x-y|^a}\cdot dy=f(x)\tag{*}
\]


\noindent
Here $f(x)$ is a given function which at least is continuous
and one seeks $\phi$.
A complete solution was 
given by Carleman in an article entitled
\emph{�ber die Abelsche Integralgleichung mit konstanter
Integrationsgrenzen} from 1922. 
To begin with $f$ yields the function
\[ 
F(t)=-\frac{1}{\pi}\cos\,\frac{a\pi}{2}\cdot \int_0^1\,
\frac{1}{s-t}\cdot f(s)\bigl[(s(1-s)\bigr]^{\frac{a-1}{a}}\cdot ds\tag{1}
\]


\noindent
The unique solution $\phi(x)$ is found given via
a complex contour integral where $\Gamma_x$
for each $0<x<1$ is a simple closed curve 
whose intersection with the positive real axis is the singleton set
$\{x\}$ and one has;
\[
\phi(x)= \frac{1}{2\pi i}\int_{\Gamma_x}\, \frac{1}{(t-x)^{1-a}}
\cdot \bigl[t(t-1)\bigr]^{\frac{1-a}{2}}\cdot F(t) dt\tag{2}
\]



\medskip

\noindent
Another  integral equation is
\[ 
\int_0^1\, \log\,|x-y|\cdot \phi(y)\cdot dy=f(x)\tag{**}
\]
In Carleman's article is is proved the $\phi$-solution is
unique and determined by the  formula:
\[
\frac{1}{\pi^2}\cdot \frac{1}{\sqrt{x(1-x)}}\cdot
\int_0^1\,\frac{f'(s)\sqrt{s(1-s)}}{s-x}\cdot ds-
\frac{1}{2\pi^2\log 2\cdot \sqrt{x(1-x)}}\cdot
\int_0^1\,\frac{f(s)}{\sqrt{s(1-s)}}\cdot ds
\]
\medskip

\noindent
In Chapter XX we shall prove the inversion formula above
using  results about  distributions and boundary
values of analytic functions.


\newpage


\centerline{\bf{0.K Location of zeros of polynomials.}}

\medskip

\noindent
In the article [xx] from 1876, Eduard Routh
applied Cauchy's residue calculus to analyze positions of roots to polynomials.
For applications to dynamical systems the main concern is to
determine zeros in a half\vvv plane such as
$\mathfrak{Re}(z)>0$.
We shall discuss some examples.
Consider a polynomial 

\[ 
P(z)= z^{2m}+c\uuu{2m\vvv 1}z^{2m\vvv 1}+\ldots + c\uuu 1 z+c\uuu 0
\]
whose coefficients are real numbers.
We assume also that $P$ has no zeros in the imaginay axis and  the zeros of
\[ 
y\mapsto \mathfrak{Re}\, P(iy)
\] 
are simple.
When $|y|$ is large we have
\[
P(iy)\simeq (\vvv 1)^m\cdot y^{2m}
\]
\medskip

\noindent
{\bf{Exercise.}} 
Assume that  $m$ is even
and $\mathfrak{Re}(P(iy))$ has at least one zero. Show
that there exists a positive  integer $k$
and a strictly increasing sequence 
\[ 
\alpha\uuu 1<\beta\uuu 1<\ldots<\alpha\uuu k<\beta\uuu k
\]
where $\{i\alpha\uuu\nu\}$ and $\{i\beta\uuu\nu\}$ are the zeros of
$\mathfrak{Re}(P)$ on the imaginary axis.
By assumption $\mathfrak{Im}\,P\neq 0$ hold at these simple zeros of the real part..
With these notations   Routh's formula gives:
\medskip

\noindent
{\bf{K.1 Theorem.}}
\emph{The number of zeros of $P$ counted with multiplicity in the right
half\vvv plane is equal to}
\[
m\vvv\frac{1}{2}\sum\uuu {\nu=1}^{\nu=k}\,\bigl[
\text{sign}(\mathfrak{Im}\,P(i\alpha\uuu \nu)\vvv
\text{sign}(\mathfrak{Im}\, P(i\beta\uuu \nu)\bigr]
\]
\medskip

\noindent
{\emph{Proof.}}
We  use  the argument principle and
study the  the function
\[
y\mapsto \text{arg}\,P(iy)
\] 
as $y$ decreases from $+\infty$ to $\vvv\infty$ on the real $y$\vvv line.
An induction over $k$ gives:
\[
\lim\uuu {R\to\infty}\, [\text{arg}\,P(iR)\vvv\text{arg}\,P(\vvv iR)]
=\vvv \pi\cdot \sum\uuu {\nu=0=}^{\nu=k}
\,[\text{sign}\,\mathfrak{Im}\,P(i\alpha\uuu \nu)\vvv
\text{sign}\,\mathfrak{Im}\,P(i\beta\uuu \nu)\bigr]
\]
At the same time
the argument of $P$ along a 
half\vvv circle $z=Re^{i\theta}$ with $\vvv \pi/2<\theta<\pi/2$
increases by the term $\simeq 2m\pi$ when $R$ is large.
Now Routh's theorem follows from the argument principle in
Chapter IV.
\medskip

\noindent
{\bf{K.2 Example.}}
Consider the case $m=2$ and a polynomial of the form
\[ 
P(z)= z^4+ 2Az^2+Bz+Cz^3\vvv 1
\]
where $A,B,C$ are real.
We get
\[ 
\mathfrak{Re} P(iy)= y^4\vvv 2Ay^2\vvv 1=(y^2\vvv A)^2\vvv A^2\vvv 1
\]
Here two 
real roots appear via the equation
\[ 
y^2=A+ \sqrt{A^2+1}
\]
More precisely, we get two roots $\vvv\rho$ and $\rho$ where

\[
\rho=\sqrt{\sqrt{A^2+1}+A}
\]
At the same time
\[
\mathfrak{Im}\,P(iy)= By\vvv Cy^3
\]
With the notations in  Theorem K.1
we have $\alpha\uuu 1=\vvv \rho$ and $\beta\uuu 1=\rho$.
Here the   difference
\[
\text{sign}(B\cdot \vvv\rho + C\rho^3\vvv
\text{sign}(B\cdot \rho \vvv C\rho^3=
2\cdot \text{sign}(C\rho^3\vvv B\rho)
\]
Taking the minus sign into the account in Routh's formula
we conclude that $P$ has one zero in
the right half\vvv plane if
\[
C\rho^2> B\tag{i}
\]
while it has 3 zeros in this half\vvv plane when
\[
C\rho^2< B\tag{ii}
\]
where $\rho=A+\sqrt{A^2+1}$.
Notice that when $P$ is restricted to the real axis
then it is $<0$ when $|x|$ is small and $>0$ when
$|x|$ is large so $P$ has always at least one real zero on
$x>0$ and one on $x<0$ while (i\vvv ii) determine the real part 
of the two remaining zeros
which are conjugate since the coefficients of $P$ are real.



\medskip

\noindent
{\bf{The case when $P$ has odd degree.}}
Consider the case of a cubic polynomial
\[ 
P(z)= z^3+a\uuu 2z^2+a\uuu 1z+a\uuu 0
\] 
where 
each $a$\vvv number is real and positive.
Now
\[ 
P(iy)=\vvv iy^3\vvv a\uuu 2y^2+ia\uuu 1y+a\uuu 0
\]
We assume in addition that $P$ has no zeros on the imaginary axis.
The real part has two zeros $\rho>0$ and $\vvv\rho$ where
\[
\rho=\sqrt{\frac{a\uuu 0}{a\uuu 2}}
\]
Let us  pursue the variation of $\text{arg}\,P(iy)$ while 
$y$ descreases from $+\infty$ to $\vvv \infty$.
On the positive real axis $P(x)$ is real and positive
and it follows that
\[
\lim\uuu {R\to\infty}\, \text{arg}\,P(iR)=3\pi/2
\]
Next, the real part of $P(iy)$ is $<0$ as long as $y>\rho$
and when $y=\rho$ the imaginary part becomes
\[
\rho(a\uuu 1\vvv \rho^2)
\]
Suppose that this  term is $>0$. Then a figure shows that
the argument has decreased from
$3\pi/2$ to $\pi/2$.
Next, while $\vvv\rho<y<\rho$ the real part is $>0$
so $P(iy)$ moves in the right  half\vvv plane and
when $y=\vvv \rho$
the imaginary part gets a reversed sign
to (1) which means that the argument now has decreased from
$\pi/2$ to $\vvv \pi/2$. So up to $y=\vvv\rho$ the 
decrease of the  argument is $\vvv 2\pi$.
Finally, when $y<\vvv\rho$ then
the real part is again $<0$ while $\mathfrak{Im}\, P(\vvv i\rho)<0$
and after we see that
the imaginary part
increases  when $y\to\vvv \infty$
which means that the argument of $P$ continues to decrease and the total effect is that
\[
\lim\uuu {R\to\infty}\, [\text{arg}\,P(iR)\vvv\text{arg}\,P(\vvv iR)]=
\vvv 3\pi
\]
At the same time
$P(Re^{i\theta}\simeq R^3e^8{i\theta}$ when $R>>1$ so the argument increases by
$3\pi$ along the half\vvv circle of radius $R$ in the right half\vvv plane.
From this we conclude that $P$ has no zeros in half\vvv discs
$\{|z|<R\cap \mathfrak{Re}\, z>0\}$
which mens that the zero of the cubic polynomial are confined to the 
left half\vvv plane.
hence we have proved

\medskip

\noindent {\bf{Theorem.}}
\emph{Let $P(z)$ be a cubic polynomial
where $\{a\uuu k\}$ are real and positive and $P$ has no zeros on the imaginary axis.
Then all roots belong to the left half\vvv plane if}
\[
a\uuu 1a\uuu 2>a\uuu 0
\]
\bigskip


\noindent
{\bf{Exercise.}}
Show that if 
$a\uuu 1a\uuu 2<a\uuu 0$ then $P$ has exactly one root in
the right half\vvv plane.
\bigskip


\noindent
{\bf{Some other examples.}}
Consider a polynomial of degree 3:
\[ 
P(z)=z^3+Az\vvv 1
\]
where $A>0$. Hence  the derivative 
$P'(x)= 3x^2+A>0$ on the real $x$\vvv axis so 
$P(x)$ has at most one real zero and since $P(0)=\vvv 1$ this zero $\rho$
must be $>0$.
The two remaining roots appear in a conjugate pair $\xi,\bar\xi$
and since $z^2$ is missing in $P$ we have
\[
\xi+\bar\xi+\rho=0
\]
Since $\rho>0$ we conclude that
$\mathfrak{Re}\,\xi<0$, i.e, $P$ has one root in the right
half\vvv plane. It is instructive
to check this via
the argument  principle.
On the imaginary axis we notice that
$\mathfrak{Re} \,P(iy)=\vvv1$ is constant so
$y\mapsto P(iy)$ moves in the left half\vvv plane and when
$R>>0 $ 
we see that
\[ 
\text{arg} P(iR)\simeq \vvv \pi/2
\]
A figure shows that
the argument of $P(iy)$  decreases from $\vvv\pi/2$ to $\vvv 3\pi/2$.
At the same time the argument increases by the factor $3\pi$ as we move on a circle from 
$\vvv iR$ to $iR$.
The total  variation of the argument along a large half\vvv circle
becomes $3\pi\vvv\pi=2\pi$ which
reflects the fact that $P$ has one root in the right half\vvv plane.
\medskip


\noindent
{\bf{Another case.}}
Consider a polynomial of the form
\[ 
P(z)= z^3+Az+1
\] 
where $A$ is  real.
Here $y\to P(iy)$ moves in the right half-plane
and we have
\[ 
\arg P(-iR)\simeq\pi/2\quad\text{and}\quad
\arg P(iR)\simeq-\pi/2
\]
So  the variation of the argument
as $y$ moves from $R$ to $\vvv R$ 
increases by the factor $\pi$. From this we can conclude that $P$ has two zeros in
the right half\vvv plane.
Notice that $P(0)=B>0$ which implies that $P(x)$ has a real zero 
$\rho$ on the negative axis.
The two remaining roots appear in a conjugate pair $\xi,\bar\xi$
and since $z^2$ is missing we have
\[ 
2\xi+\rho=0
\] 
which implies that $\mathfrak{Re}\,\rho>0$
in accordance with the previous verification via
the argument principle.

\medskip

\noindent
{\bf{The case $P(z)= z^5+z+1$.}}
Here we get
\[ 
P(iy)= i(y^5+y)+1
\]
The variation of $\arg(P)(iy)$
as $y$ moves from $R$ to $\vvv R$
is now $\pi$
while that along the half-circle is $5\pi$. The conclusion is that
$P(z)$ has two roots in the right half-plane.
Notice that in this example
\[ 
P'(x)= 5x^4+1>0
\] 
So $x\mapsto P(x)$ is strictly increasing on the real $x$-line 
where it has a simple zero $x_*<0$
since $P(0)=1>0$.
We have seen that two complex roots $\alpha$ and $\bar\alpha$ appear
with a common real part $>0$ while the two other complex roots
$\beta$ and $\bar\beta$ have a negative real part.
The reader should find numerical values for these complex roots
and confirm the assertion that two roots appear in the right
half-plane.
Let us remark that the polynomial above appears in Abel's article [Abel] where 
he demonstrated that his specific
algebraic equation of degree five cannot be solved by
roots and radicals.
\bigskip

\noindent
{\bf{K.3 The Hurwitz-Routh theorem.}}
Consider a polynomial
\[ 
P(z)= z^4+a_1z^3+a_2z^2+a_3z+a_4
\]
whose coefficients are real numbers.
We assign the $4\times 4$-matrix

\[
A=\begin{pmatrix} a_1&a_3&0&0\\
1&a_2&a_4&0\\
0&a_1&a_3&0\\
0&1&a_2&a_4\end{pmatrix}
\]
The matrix has four principal minors. The first is just $a_1$, the second
$a_1a_2-a_3$ and the third
\[
a_4(a_1(a_2a_3-a_1a_4)-a_3^2)
\]
The last minor is $\det(A)$.
With these notations the Hurwitz-Routh theorem for polynomials of degree
4 asserts that the roots of $P(z)$ all belong to
the left half-plane if and only if the four minors above are all $>0$.
A similar criterion holds for polynomials of arbitrary high degree. More precisely,
consider a polynomial

\[ 
P(z)= z^n+a_1z^{n-1}+\ldots a_{n-1}z+a_n
\] 
with real coefficients.
The necessary and sufficient condition in order that all roots belong to the 
half-plane $\mathfrak{Re}(z)<0$
is expressed by minors of an $n\times n$-matrix $A$
with elements
\[
\alpha_{ik}=a _{2k-i}\tag{*}
\]
where $a_0=01$ while $a_\nu=0$ when $\nu<0$ or $\nu>n$.
For example, with $k=n$ we get a non-zero element in the $n$:th column
if and only if $2n-i\leq n$ which means that only $i=n$
is possible, i.e. $\alpha_{nn}=1$ while $\alpha_{in}=0$ when
$1\leq i\leq n-1$.
The Hurwitz-Routh theorem asserts that the roots of $P$ belong to the right half-plane
if and only if all principal minors of the $A$-matrix are positive.
For a proof we refer to Chapter 11 in [xx] which in addition to 
this criterion expressed by signs of minors contains w ealth of other results and
also an extensive historic account where one 
major contributions in addition to those of Routh and Hurwitz are due to
Sturm.
Let us also mention that instead of using the argument principle 
which involves complex computations, one can
consider the so called Cauchy index which
arises when one pursues the real\vvv valued function
\[
\frac{\mathfrak {Re}\, P(iy)}{\mathfrak {Im}\, P(iy}
\]
where jumps at zeros of the imaginary part
leads to sign\vvv chains and makes it possible to
apply the rule of Descartes for zeros of real\vvv valued functions.
here a very efficient algorithm was discovered and developed by Sturm
which can be used to determine the number of zeros of a polynomial with
real
coefficients in the right and the left
half\vvv plane. The interested reader should consult chapter 10 in
the excellent text\vvv book [XX] for further details and
which in addition gives a very complete account of the extensive 
theory dealing with positions of zeros of
polynomials which in general can have complex coefficients.




































\newpage

\centerline{\bf\large {Special Integrals.}}


\bigskip



\centerline {\bf{A. The integral of  $e^{-x^2}$}}.

\medskip

\noindent
Recall that one can use a trick in calculus to evaluate:
\[ 
J=\int_0^\infty\, e^{-x^2}\cdot dx\tag{1}
\]
Namely,  use polar coordinates
in the first quadrant which gives
\[ 
J^2=\int_0^{\pi/2}\,[\int_0^\infty\, r\cdot e^{-r^2}dr] d\theta=\pi/4
\]
Hence we get $J=\sqrt{\pi}/2$.
If we instead take the
substitution $x^2\to t$ one gets:
\[
 J=\frac{1}{2}\cdot \int_0^\infty  t^{-1/2}\cdot e^{-t}\cdot dt
\]

 
 \noindent
In general we consider an integral of the form
\[ 
J_a=\int_0^\infty t^{-a}\cdot e^{-t}\quad\colon\quad 0<a<1\tag{1}
\]


\noindent
In (1) we recognize the $\Gamma$-function and conclude that:
\[ 
J_a=\Gamma(1-a)\tag{*}
\]

\medskip


\noindent
This is admitted as an "analytic formula". More precisely
one should include the
$\Gamma$-functions in the family of 
"elementary functions". Of course a computer is needed for numerical values
as $a$ varies. 



\bigskip

\centerline {\bf{B. Integrals of rational functions.}}

\medskip

\noindent
Let $P(z)$ be a polynomial of degree $n\geq 2$ and 
assume that it has no real zeros which implies that the integral below exists:
\[ 
J=\int_{-\infty}^\infty\, \frac{dx}{P(x)}\tag{1}
\]

\noindent
Consider the case when the roots of $P(z)$ are simple. Newton's formula gives:
\[ 
\frac{1}{P(z)}= \sum_{k=1}^{k=n}\, \frac{1}{P'(\alpha_k)}
\cdot\frac{1}{z-\alpha_k}\tag{2}
\] 
where the sum extends over the roots $\alpha_1,\ldots,\alpha_n$.
The absolute convergence of (1) implies that we have a limit
as we integrate over 
$-R\leq x\leq R$. 

\medskip

\noindent
{\bf{Exercise.}} Show that
\[ 
J=2\pi i\cdot \sum^*\, \frac{1}{P'(\alpha_k)}=
-2\pi i\cdot \sum_*\, \frac{1}{P'(\alpha_k)}
\] 
where the sum is taken over the zero of $P$ in the upper, respectively the lower
half-plane.
Conclude that
one always has
\[
\sum_{k=1}^{k=n}\, \frac{1}{P'(\alpha_k)}=0\tag{*}
\]
 
\medskip

\noindent
\emph{Hint.}
Let $\alpha=a+ib$ be a complex number with $b\neq 0$.
If $b>0$
there exists a single-valued branch of
$\log(z-\alpha)$ along the real axis where
\[
-\pi<\arg(x-\alpha)<0\tag{1}
\]
while $-\alpha$ moves in the lower half-plane.
It follows that
\[
\int_{-R}^R\, \frac{dx}{x-\alpha}=
\log(R-\alpha)-\log(-R-\alpha)=
\log\,\frac{|R-\alpha|}{(R+\alpha|}+
i\cdot (\arg(R-\alpha)-\arg(-R-\alpha))
\]
From (1) the reader may verify that
\[ 
\lim_{R\to\infty}\, \arg(R-\alpha)=0\quad\text{and}\quad
\lim_{R\to\infty}\, \arg(-R-\alpha)=-\pi
\]
At the same time we notice that
\[
\lim_{R\to\infty}\,\log\,\frac{|R-\alpha|}{(R+\alpha|}=0
\]
It follows that
\[
\lim_{R\to\infty}\, \int_{-R}^R\, \frac{dx}{x-\alpha}= \pi i
\]
In the case $\alpha=a+ib$ where $b<0$ the reader can verify that
the limit integral instead takes the value $-\pi i$.











\bigskip

\centerline{\bf{C. The integral $\int_0^\infty\, \frac{dx}{P(x)}$}}
\bigskip

\noindent
Assume as above that the zeros of $P$ are simple and non-real.
Removing the half-line $0\leq x<\infty$
from the complex plane we obtain  a single value branch of $\log z$
whose imaginary part stays in $(0,2\pi)$.
Now we apply residue calculus to the function
\[ 
g(z)=\log z\cdot \frac{1}{P(z)}
\]
\medskip

\noindent
{\bf{Exercise.}}
Explain how to choose suitable contours and use that
$\log z$ after one positive turn around the origin changes
the branch by adding $2\pi i$ to get the formula:
\[
-2\pi i\cdot \int_0^\infty\, \frac{dx}{P(x)}=2\pi i\cdot
\sum\,\mathfrak{res}_{z=\alpha_k}(\log z\cdot \frac{1}{P(z)})\tag{*}
\]
where the residue sum is taken over all zeros of $P$.

\medskip






\noindent
{\bf{C.1 Example.}} Consider  the special case $P(z)= z^2+1$. Now
\[ 
\frac{1}{xz^2+1}=\frac{1}{2i}\cdot [\frac{1}{z-i}-\frac{1}{z+i}]
\]
It follows tha the residue sum becomes
\[ 
\frac{1}{2i}\cdot 
(\log i-\log (-i))=\frac{1}{2i}\cdot (\pi i/2-3\pi i/2)=
-\pi/2
\]
Taking the minus sign into the account in (*) we conclude that
\[ \int_0^\infty\,\frac{dx}{1+x^2}=\pi/2
\] 
which confirms  a wellknown formula in calculus.




\bigskip

\centerline{\bf{D. The integrals $\int_{-\infty}^\infty\, \frac{e^{ia x}\cdot dx}{P(x)}$}}
\bigskip

\noindent
Assume that $P$ has no real zeros and of degree $\geq 2$.
When $a$ is a real number it is clear that  the integral above exists.
Let us consider  the case $a>0$.
The entire  analytic function
$e^{iaz}$ is small in the upper half-plane since
\[
|e^{ia(x+iy)}|= e^{-ay}
\]
Assume that the complex roots
of $P$ are simple
and consider its fractional decomposition.
Then  the integral in (D) becomes
\[ 
\lim_{R\to\infty}\, \sum\, \frac{1}{P'(\alpha_k)}
\cdot \int_{-R}^R\, \frac{e^{iax}\cdot dx}{x-\alpha_k}
\tag{1}
\]
\medskip

\noindent
{\bf{D.1 Exercise.}}
Show that (1) is equal to
\[ 
2\pi i\cdot \sum^*\, \frac{e^{ia\cdot \alpha_k}}{P'(\alpha_k)}
\tag{2}
\]
where $\sum^*$ extends over those $k$ for which
$\mathfrak{Im}(\alpha_k)>0$.
\medskip

\noindent
{\bf{D.2 Example.}}
Consider the case $P(x)=x^2+1$.
Then $\alpha_1=i$ is the sole simple root in the upper half-plane and we get
\[ 
\int_{-\infty}^\infty\, \frac{e^{iax}\cdot dx}{1+x^2}=
2\pi i\cdot c_1\cdot e^{-a}=\pi\cdot e^{-a}
\] 


\medskip


\noindent
{\bf{D.4 Exercise.}} Let $a<0$ and assume that  
the zeros of $P(z)$ which belong to
the lower half-plane are all simple.
Show that in this case the integral (1) becomes
\[
-2\pi i\cdot \sum_*
\, \frac{e^{ia\cdot \alpha_k}}{P'(\alpha_k)}
\]
where the sum extends over zeros with
$\mathfrak{Im}(\alpha_k)<0$.
The reader should explain why a minus sign  occurs
by the aid of a figure and the orientation of the
continuous which is used when the residue formula is applied.


\medskip






\centerline{\bf{E. Principal value integrals.}}
\bigskip

\noindent
Outside $x=0$ we have the odd function $\frac{1}{x}$
which entails that
\[
\int_{-1}^{-\epsilon}\,\frac{dx}{x}+
\int_\epsilon ^1\frac{dx}{x}=0
\]
From this it is clear that the limit below exists for
every $C^1$-function
$g(x)$ on $[0,1]$:
\[ 
\lim_{\epsilon\to 0}\, \int_{\epsilon<|x|<1}\, \frac{g(x)\cdot dx}{x}
\]


\noindent
In general, let $P(x)$ be a polynomial of some degree
$k\geq $ whose zeros are all real and simple.
Using principal values we can define the integral
\[ 
\text{PV}\, \int_{-\infty}^\infty\, \frac{dx}{P(x)}\tag{1}
\]
To compute (1) we consider the
function $g(z)=\frac{1}{P(z)}$ which is analytic in the upper half-plane
and integrated along
a large half-circle  of radius $R$,
and along the real axis where small intervals around each real root
are replaced by half-circles of radius $\epsilon$.
\medskip

\noindent
{\bf{E.1 Exercise.}}
Draw a figure to illustrate the contour integral of $g(z)$ which is used above.
Next, the complex line integral of $g$ along this contour is zero.
The definition of the principal value integral shows that
(1) is equal to the limit
\[
\sum\,\int_0^\pi\, \epsilon\cdot \frac{e^{i\theta}\cdot i d\theta}
{P(a_k+\epsilon\cdot e^{i\theta})}=\pi\cdot \sum\, \frac{1}{P'(a_k)}\tag{*}
\]









\bigskip



\centerline {\bf{F. The integral $J=\int_0^\infty\,\frac{T(\sin x)}{P(x)}\cdot dx$}}

\bigskip

\noindent
Let $P(x)$ be a polynomial of  degree $N\geq 2$
whose zeros are all real and simple. Set
\[ 
T(x)=\sum_{\nu=1}^{\nu=m}\, c_\nu\cdot \sin(\nu x)
\]
where $\{c_\nu\}$ are real and 
assume that $T(a_k)=0$ holds for each
real zero of $P$.
To compute the integral in (F)
we consider the analytic function
\[
g(z)=\sum\, c_\nu\cdot e^{i\nu z}\cdot \frac{1}{P(z)}\tag{1}
\]
Take a complex line integral over a contour
which consists of a large half-circle $\{|z|=R\}$
in the upper half-plane  and on $-R\leq x\leq R$ we replace small 
intervals around the zeros $\{a_k\}$ by small half-circles.
If $\Gamma_{R,\epsilon}$ denotes the contour we get
\[
0=\int_{\Gamma_{R,\epsilon}}\, g(z)\cdot dz\tag{2}
\]
Next, in the upper half plane
the imaginary part of $g$  is small which  implies that
\[ 
\lim_{R\to+\infty}\, \int_0^\pi g(Re^{i\theta})\cdot iRe^{i\theta}\cdot d\theta=0\tag{3}
\]


\noindent
{\bf{F.1 Exercise.}}
Conclude from the above that one has
the formula
\[
\int_{-\infty}^\infty\, 
\frac{T(\sin x)}{P(x)}\cdot dx=
\lim_{\epsilon\to 0}\mathfrak{Im}\,[
\sum_{k=1}^{k=N}\, \int_0^\pi\, g(a_k+\epsilon\cdot e^{i\theta})\cdot
i\epsilon\cdot e^{i\theta}\cdot d\theta\,]\tag{1}
\]


\noindent
Show also that
for each real zero $a_K$ one has
the formula
\[
\lim_{\epsilon\to 0}\mathfrak{Im}\,[
\int_0^\pi\, g(a_k+\epsilon\cdot e^{i\theta})\cdot
i\epsilon\cdot e^{i\theta}\cdot d\theta\,]=
\mathfrak{Im}\bigl [\,\sum\, c_\nu e^{i\nu a_k}\cdot\frac{1}{P'(a_k)}\cdot \pi i\bigr ]\tag{2}
\]


\noindent
Finally, since $\{c_\nu\}$ are real and $P'(a_k)$ also is real
we see that the last term becomes
\[
\frac{\pi}{P'(a_k)}\cdot \sum\, c_\nu\cdot \cos(\nu a_k)\tag{3}
\]


\noindent
Hence we have established the formula
\[
\int_{-\infty}^\infty\,\frac{T(\sin x)}{P(x)}\cdot dx=
\sum_{k=1}^{k=N}\, \frac{\pi}{P'(a_k)}\cdot T(\cos a_k)\tag{*}
\]
\medskip

\noindent
{\bf{F.2 Exercise.}}
Above we assumed that the degree of $P$ is $\geq 2$. Show that the same formula
as in (*)  holds if
$P$ is linear, i.e is of the from
$x-a_1$ for some real $a_1$.
In particular consider the case   $a_1=0$ and the sine-function $\sin x$. 
Then the general formula (*)  gives
\[
\int_{-\infty}^\infty\,\frac{\sin x}{x}\cdot dx=\pi
\]


\bigskip

\centerline{\bf{G. Adding complex zeros to $P$}}

\bigskip

\noindent
Above the zeros of $P$ were  real and simple.
Suppose now that $P$  has some  non-real roots occur 
which 
appear in conjugate pairs.
In this case the calculation are the similar to those   in (F),  except that we
also get residues from zeros of $P$ in the upper half-plane.
More precisely, the line integrals of $g$ over
$\Gamma_{R,\epsilon}$ from (F:1) are no longer zero.
With $R$ large and $\epsilon$ small the contour
$\Gamma_{R,\epsilon}$ contains all
zeros of $P(z)$ i the open upper half-plane.
Let us assume that  these zeros are simple and denote them
by
$\beta_1,\ldots,\beta_M$. So here
$\mathfrak{Im}(\beta_\nu)>0$ for each $\nu$.
Each $\beta$-root gives a residue
\[ 
2\pi i\cdot 
\frac{1}{P'(\beta_j)}\cdot
\sum\, c_\nu e^{i\nu\cdot\beta_j}\tag{1}
\]
Here we shall take the imaginary part
to get a contribution.
The result is that in the
formula (*) from (F)  one adds a term in the right hand side which becomes
\[
2\pi\sum_{j=1}^{j=M}\, 
\mathfrak{Re}[\frac{1}{P'(\beta_j)}\cdot
\sum\, c_\nu e^{i\nu\cdot\beta_j}]\tag{**}
\]

\medskip

\noindent
{\bf{G.1 Example.}}
Consider the case $P(x)=x(x^2+1)$ and $T(x)=\sin x$.
We get the root $\beta_1=i$ and  notice that
$\frac{1}{P'(\beta_1)}\cdot
e^{i\cdot\beta_1}=\frac{1}{2i^2}\cdot e^{-1}=-\frac{1}{2e}$ which gives
\[ 
\int_{-\infty}^\infty\, \frac{\sin x\cdot dx}{x(1+x^2)}=
\pi-\frac{\pi}{e}
\]




\bigskip


\centerline{\bf{H. The integral $\int_0^\infty\, \frac{x^a}{1+x^2}\cdot dx$}}.


\noindent
Let $0<a<1$. To find the integral above
use the
function $z^a$ 
which under analytic continuation in the upper half-plane reaches the negative real
axis where we have
\[
 (-x)^a=x\cdot e^{\pi i a}\quad\colon\quad x>0\tag{1}
 \]
At the same time $\frac{1}{1+z^2}$ has a simple pole at $z=i$.
So if $J$ is the value of the integral in (H:1) we obtain:
\[
 J(1-e^{\pi ia})=2\pi i\cdot \mathfrak{res}(\frac{z^a}{1+z^2}:i)
\]
At $z=i$ we use that $i= e^{\pi i/2}$ so that $z^a=e^{a\pi i/2}$
and find that the right hand side above becomes:
\[ 
 2\pi i\cdot \frac{e^{a\pi i/2}}{2i}=\pi\cdot e^{a\pi i/2}\implies
 \]
\[ 
J=\pi\cdot\frac{e^{a\pi i/2}}{1-e^{\pi ia})}=
2\cdot \pi\cdot\sin\, (\pi a/2)\tag{*}
\]



\noindent
{\bf{H.1 Exercise.}}
Use the methods from XX and explain a formula for the integral
\[
\int_0^\infty\,\frac{x^a\cdot dx}{P(x)}
\]
when $P$ is a polynomial of degree $\geq 2$ and without
real zeros.
\medskip

\noindent
{\bf{H.2 The case when $P$ has negative real zeros.}}
If this occurs one uses another method. The strategy
is to take a complex integral of
$g(z)=\frac{z^a}{P(z)}$ which starts with the interval $[\epsilon,R]$ 
where $\epsilon$ is small and $R$ is large.
Then one
takes the circle $|z|=R$ and after one turn
one integrates back from $x=R$ to $x=\epsilon$.
Finally a small integral is taken over $|z|=\epsilon$ and  the reader should illustrate
the whole construction by a figure.
\medskip

\noindent{\bf{H.3 Exercise.}}
If $\Gamma_{\epsilon,R}$ is the contour described above then
it borders a simply connected domain where a single-valued branch of $z^a$ exists.
Then we can apply the residue formula  and the reader should verify that
\[
(1-e^{2\pi a i})\cdot J= 2\pi i\sum\,\mathfrak{res}(g(z):\alpha_k)
\] 
where the sum is taken over all zeros of $P$.


\noindent{\bf{H.4 Exercise }}
Consider the case $P(x)=(x+1)^2$ which  has a double zero at $x=-1$.
Use the formula above to show that

\[ 
\int_0^\infty\, \frac{x^a \cdot dx}{(x+1)^2}=\frac{\pi a}{\sin(\pi a)}
\]





\bigskip

\centerline {\bf{I. Use of the Log-function.}}


\medskip

\noindent
Consider the integral
\[
J=\int_0^\infty\,\frac{dx}{1+x+x^5}
\]
To compute it  we consider the multi-valued analytic function
\[ 
g(z)= \frac{\log z}{1+z+z^5}
\]
Start the integration on the real $x$-axis from
0 to $R$ and continue the complex line integral over
the large circle $|z|=R$ and after one returns from $R$ to $x=0$
in the negative direction. While this is done we have a new branch of the log-function, i.e.
it is now $\log x+2\pi\cdot i$.
Taking the negative direction into the account during the
last integration along the non-negative $x$-axis it follows that
\[ 
2\pi i\cdot J=-2\pi i\cdot \sum\,\mathfrak{res}\bigl( \frac{\log z}{1+z+z^5}\bigr)\tag{*}
\]


\noindent
Notice the minus sign above !
\medskip

\noindent
{\bf{I.1 A simpler example.}} Suppose above that 
we instead take the polynomial $1+z^2$. It has simple roots at i and $-i$.
Now

\[ 
\log i=\pi\cdot i/2\quad\text{and}\quad \log -i=3\pi\cdot i/2
\]
The sum of residues therefore becomes

\[
\frac{\pi\cdot i/2}{2i}+ \frac{3\pi\cdot i/2}{-2i}=\frac{\pi}{2}
\]
Thanks to the minus sign in (*) above we conclude that

\[
\int_0^\infty\,\frac{dx}{1+x^2}= \frac{\pi}{2}
\]
This reflects a wellknown formula which can be established directly, i.e
use that $\frac{1}{1+x^2}$ is the derivative of the arctg-function.
But it is 
illuminating to see that the general procedure using the multi-valued log-function works.

\bigskip

\centerline{\bf{J. Trigonometric integrals.}}

\medskip

\noindent
A  trigonometric  polynomial is
of  the form
$P(\theta)= \sum\, c_k\cdot e^{ik\theta}$
where the coefficients $\{c\uuu k\}$ in general are complex numbers
and the
sum extends over a finite set of integers which may be both positive and
negative.
Consider a quotient of two such trigonometric polynomials
\[ 
R(\theta)=\frac{P(\theta)}{Q(\theta)}
\] 


\noindent
Assume that $Q(\theta)\neq 0$ for all $0\leq\theta\leq 2\pi$
and put:
\[ 
J_R=\int_0^{2\pi} R(\theta)\cdot d\theta\tag{*}
\]
To find (*) we use the substitutions
$e^{ik\theta}\mapsto z^k$ and obtain:
\[ 
J_P=\int_{|z|=1}\, \frac{P(z)}{Q(z)}\cdot \frac{dz}{iz}\tag{**}
\]

\noindent One must not forget $\frac{1}{iz}$ which appears since
\[
ie^{i\theta}\cdot d\theta=dz\implies d\theta=\frac{dz}{iz}
\]

\noindent
If $M$ is a sufficiently large integer then
\[
Q(z)=z^{-M}\cdot Q_*(z)\quad\text{and}\quad P(z)=z^{-M}\cdot P_*(z)\tag{1}
\] 
where $P_*$ and $Q_*$ are polynomials in $z$.
Using (1) there remains to evaluate
\[ 
\int_{|z|=1}\, \frac{P_*(z)}{Q_*(z)}\cdot \frac{dz}{iz}\tag{2}
\]

\noindent
Usually one picks residues in the open unit disc $D$. However, there are cases when
$Q_*$ has many zeros in $D$ and then one can 
use residue calculus in the exterior disc.
More precisely, choose a large positive number $r$
so that
$\{|z|<r\}$ contains all zeros of $P_*$ and $Q_*$.
Let $m(P_*)$ and $m(Q_*)$ be the degrees of the polynomials.
If $m(P\uuu *)<m(Q\uuu *)$
the 
the integral (*) becomes
\[
2\pi\cdot \sum\,\mathfrak{res}(R(\alpha\uuu k)
\]
with the sum taken over zeros of $Q*$ in the exterior disc $|z|>1$.


\newpage

\noindent
{\bf{J.1 Experiment with a computer.}}
It is instructive to perform a calculation via residues and 
compare the result by  a  computer which provides a
numerical  answer within a fraction of
a
second.
Consider for example the integral
\[
J=\int_0^{2\pi}\,\frac{d\theta}{1+a\cdot\cos\theta}
\] 
where $a$ is a complex number with absolute value $<1$. Then
\[ 
J=\frac{1}{i}\cdot \int_{|z|=1}\,\frac{2z dz}{2z+az^2+a}
\]
The quadratic polynomial has a simple zero $\alpha\in D$ and
residue calculus gives the formula below which after can be solved numerically for
different values of $0<a<1$.
\[ 
J=2\pi \cdot \frac{\alpha }{1+a\alpha}
\]


\medskip


\centerline {\bf{K. Summation formulas.}}
\medskip

\noindent
Consider the meromorphic function
\[ 
g(z)=\frac{\cos\pi z}{\sin\pi z}\tag{1}
\]
It has simple poles at all integers.
Notice that $g$ can be written as
\[
i\cdot\frac{e^{i\pi z}+e^{-i\pi z}}{e^{i\pi z}-e^{-i\pi z}}\tag{2}
\]


\noindent
{\bf{K.1 Exercise.}}
Show that there exists a constant $A_1$ such that
the following  holds for all integers $N$ and every
real number $s$:
\[ 
\bigl|g((N+1/2)i+is)\bigr|\leq A_1\tag{*}
\]
\noindent
Show also that there exists a positive constant $A_2$
such that the following hold when
$z=x+iy$ and  $|y|\geq 1$:
\[ 
|g(x+iy)\leq A\quad\tag{**}
\]
Thus, the $g$-function is bounded when we stay away a bit from the real axis.
Using (*) and (**) 
we can establish various summation formulas.
In general, let $p$ and $q$ be two polynomials where we assume
that $\deg(p)\geq \deg(q)+1$ and that $p$ has no real zeros.
If $N$ is a positive integer
and $R>\geq 1$ we consider the rectangle
\[
\square_{R,A}=
\{-N-1/2<x< N+1/2\}\times \{-R<y<R\}
\]
Here $N$ and $R$ are chosen so that this rectangle contains all  zeros
$\{\alpha_\nu\}$
of
$p$.
Then Cauchy's residue formula is applied when we integrate
$\frac{q}{p}\cdot g$ over the boundary of this rectangle.
The result is
\[
\frac{1}{2\pi i}\int_{\partial\square_{R,A}}\,\frac{q(z)}{p(z)}\cdot g(z)\cdot dz=
\pi\cdot \sum_{k=-N}^{k=N}\,  \frac{q(k)}{p(k)}
+\sum \mathfrak{res}(\frac{q}{p}\cdot g: \alpha_\nu)\tag{***}
\]
\medskip


\noindent
{\bf{K.2 Exercise.}}
Assume that $\deg(p)\geq\det(q)+1$.
Show that the line integrals over $\partial\square_{A,R}$ tend to zero when
$A>>R>>1$ and conclude that one has the general summation formula:
\[
\pi\sum_{k=-\infty}^{k=\infty}\,  \frac{q(k)}{p(k)}=
-\mathfrak{res}(\frac{q}{p}\cdot g: \alpha_\nu)\tag{****}
\]


\noindent
{\bf{Remark.}}
Consider as an example the case $p(z)=z-i$.
Then the left hand side becomes

\[ 
\pi[-\frac{1}{i}+\sum_{k=1}^\infty\, (\frac{1}{k-i}-\frac{1}{k+i})]=
-\cot(i)=i\cdot\frac{e^\pi+e^{-\pi}}{e^{\pi}-e^{-\pi}}\implies
\]
\[
\pi\cdot [1+\sum_{k=1}^\infty\frac{2}{1+k^2}]=\frac{e^\pi+e^{-\pi}}{e^{\pi}-e^{-\pi}}\
\]







\bigskip


\centerline {\bf{L. A Fourier integral.}}

\medskip

\noindent
We shall calculate an integral which is used in
certain Tauberian  theorems. 
The  formula in (**) below  
is used to calculate certain Fourier transforms 
and gives rise to highly non-trivial limit formulas in
Wiener's study of
Tauberian theorems.
The computations below illustrate
that one is sometimes confronted with extra difficulties
in order to handle singular log-functions.
Our aim is to find a formula for the  integral:
\[
J(s)=
\int_0^\infty \frac{\log\,|1-x^2|}{x^2}\cdot x^{is}dt\quad\colon\quad s>0\tag{*}
\]
Since the absolute value $|x^{is}|=1$ and
$\log(1-x^2)\simeq -x^2$ when $x$ is small we see that the integral converges.
But it is not clear how to compute
it via residue calculus. However, we shall see that this can be done after a number of steps.
In the upper half-plane  there exists an analytic function
defined by:
\[ 
g(z)=
\frac{\log\,(1-z^2)}{z^2}
\cdot z^{is}\tag{1}
\]
Here the single valued branch of $\log\,(1-z^2)$
is chosen so that its argument belongs to $(-\pi,0)$.
We have also
$z^{is}= e^{is\log z}$ 
where the single valued branch of
$\log z$ has an argument in $(0,\pi)$ as usual.
On the positive imaginary axis we get 
\[
g(iy)=
\frac{\log\,(1+y^2)}{y^2}
\cdot (iy)^{is}=\frac{\log\,(1+y^2)}{y^2}\cdot y^{is}\cdot e^{-\pi s/2}\tag{2}
\]

\noindent
After these preparations
we
consider the complex line integral of the $g$-function over the closed curve
given by the real interval $0\leq x\leq R$, the
quarter-circle $\{z=Re^{i\theta}\colon\, 0\leq\theta\leq \pi/2\}$
and the imaginary interval from $iR$ to $0$.
Along the real axis the argument of the log-function changes.
More precisely we notice that
the imaginary part of $\log(1-x^2)$ is zero when $0<x<1$ and is $-\pi$
if $x>1$. From this we obtain
\[ 
\lim_{R\to\infty}\, \int_0^R\, g(x)\cdot dx=
\int_0^\infty \frac{\log\,|1-x^2|}{x^2}
\cdot x^{is}\cdot dx-
i\pi\cdot \int_1^\infty\frac{x^{is}\cdot dx}{x^2}\tag{3}
\]


\noindent
Next,  with $s$ real and positive 
the absolute value
$|z^{is}|$ is bounded in the upper half-plane and
the reader can verify that the line integral of $g$ along the quarter circle tends to zero
when $R\to+\infty$.
There remains to consider the line integral along the imaginary
axis which on the closed contour above is taken in the negative direction.
Taking this sign into the account together with  (2)  and the vanishing of
the complex line integral over the whole closed contour we see that (3) is equal to
\[
\int_0^\infty\,\frac{\log(1+y^2)}{(iy)^2}
\cdot y^{is}\cdot idy\tag{4}
\]
After a partial integration (4) becomes
\[
\frac{1}{is-1}\cdot i\cdot 
\int_0^\infty \frac{2y}{1+y^2}\cdot y^{is-1}\cdot dy
=\frac{2}{s+i}\cdot\int_0^\infty \frac{y^{is}\cdot dy}{1+y^2}\tag{5}
\]
To calculate the last integral we use that $(-1)^{is}=e^{\pi i\cdot is}=e^{-\pi s}$
and conclude that
\[
(1+e^{-\pi s})\cdot \int_0^\infty \frac{y^{is}\cdot dy}{1+y^2}=
\int_{-\infty}^\infty \frac{y^{is}\cdot dy}{1+y^2}\tag{6}
\]
The last integral is found by residue calculus and 
as a consequence the reader may verify that (4) becomes
\[
\frac{2}{s+i}\cdot \frac{1}{1+e^{-\pi s}}\cdot
2\pi i\cdot \frac{i^{is}}{2i}=\frac{2}{s+i}\cdot \frac{1}{1+e^{-\pi s}}\cdot
\pi\cdot e^{-\pi s/2}\tag{7}
\]
\medskip

\noindent
{\bf{Conclusion.}} One has the equality

\[
\int_0^\infty \frac{\log\,|1-x^2|}{x^2}
\cdot x^{is}\cdot dx
=\frac{i\pi}{1-is}+\frac{2\pi}{s+i}\cdot\frac{1}{
e^{\pi s}+e^{-\pi s}}\tag{**}
\]
Notice that the right hand side is zero when $s=0$
which  gives
\[
\int_0^\infty \frac{\log\,|1-x^2|}{x^2}
\cdot dx=0\tag{**}
\]

\bigskip


\centerline{\bf{M. Multi-valued integrands.}}

\medskip

\noindent
A more involved study arises 
when
the integrands are branches of multi-valued functions 
and one seeks values which depend upon parameters.
Let us give an example.
\[
J(z)=\int_0^1\, \frac{dt}{\sqrt{t(z-t)}}\tag{*}
\]
When $z$ is real and $>1$ we can evaluate the integral as in ordinary
calculus. In the half-plane $\mathfrak{Re}(z)>1$ we see that
$J(z)$ is an analytic function of $z$
whose complex derivative becomes
\[
J'(z)=-\frac{1}{2}\cdot \int_0^1\, \frac{dt}{\sqrt{t(z-t)^3}}\tag{**}
\]


\noindent
It turns out
that  $J(z)$ extends to an analytic function
where the sole branch points are 0 and 1.
To begin with we  can choose a single valued branch of $\sqrt{z-t}$ when
$\mathfrak{Im}(z)>0$ so that $J(z)$ is analytic in the upper half-plane.
Less obvious is that $J$ extends analytically across the open real interval $0<x<1$.
One can prove this using a deformation of the contour  which defines $J$, i.e. replace
$[0,1]$ by curves in the complex $t$-plane which joint 0 and 1.
Examples of  deformation   the
contour
during the analytic continuation
of the $J$-function
appear in the classic literature. It was for example used by Hermite and
appears in many text-books devoted to
algebraic functions. See for example the
excellent
material in
Paul Appel's books which  contains
a wealth of examples related to integrals on algebraic curves and
especially so called hyper-elliptic integrals.
\medskip

\noindent{\bf{Use of $\mathcal D$-module theory.}}
A method which avoids the rather involved deformation of contours
to achieve the analytic continuation
goes back to
Fuchs, and was  later
put forward in a much wider context
in lectures by Pierre Deligne at Harvard University in 1967,
inspired by  deep  studies by Nils Nilsson
from the article  [Nilsson-1965]
which
deals with  
integrals over algebraic chains in
higher dimensions and
leads to the notion of  Nilsson class functions.
Here we stay in dimension one and begin to  seek a 
differential operator $Q(z,\partial_z)$ with polynomial coefficients
which annihilates the $J$-function.
Set $\nabla=z\partial_z$ which gives

\[
-\nabla(J)=
\frac{1}{2}\cdot \int_0^1\, \frac{(z-t)dt}{\sqrt{t(z-t)^3}}+
\frac{1}{2}\cdot \int_0^1\, \frac{t\cdot dt}{\sqrt{t(z-t)^3}}=
\frac{J}{2}+\frac{1}{2}\int_0^1\, \frac{\sqrt{t}\cdot dt}{(z-t)^\frac{3}{2}}
\]
\medskip

\noindent
In the last integral  we perform a partial integration with respect to $t$
and obtain
\[
\sqrt{t}\cdot\frac{1}{\sqrt{z-t}}|_0^1
- \frac{1}{2}\int_0^1\, \frac{dt}{\sqrt{t\cdot (z-t)}}\implies
\nabla(J)=-\frac{1}{\sqrt{z-1}}\tag{*}
\]
It follows that $J$
extends to a multi-valued function outside 0 and 1. Since
\[
(z-1)\partial(z-1)^a=a(z-1)^a
\] 
for all $a$ it follows from (*) that
\[
[(z-1)\partial+1/2]\cdot \nabla(J)=0\implies
\]
\[
(z-1)z\cdot \partial^2(J)+\frac{3}{2}\cdot \nabla(J)-\partial(J)=0\tag{*}
\] 

\medskip

\noindent
{\bf{Exercise}}
Investigate the multi-valued
behavior of $J$ around 0 and 1.
More precisely, $J$ generates a Nilsson class function of rank
2 and as described in Chapter 4 this leads to
the local monodromy expressed by $2\times 2$-matrices
at each of these ramification points.

\bigskip

\centerline{\bf{L. Multi\vvv valued Laplace integrals.}}

\bigskip

\noindent
Let $k\geq 1$ and  $a\uuu 1,\ldots,a\uuu k$ is some
$k$\vvv tuple of distinct points in
${\bf{C}}$. Let us also consider another $k$\vvv tuple of non-zero complex numbers
$\lambda\uuu 1,\ldots,\lambda\uuu k$.
In the complex $w$-plane
we have the multi-valued analytic function
\[
g(w)=\prod\, (w\vvv a\uuu \nu)^{\lambda\uuu\nu}
\]
defined in the complement of the set of $a$-points.
If say $\lambda_1$ is not an integer then analytic continuation of
$g$ along a small circle centered at $a_1$
changes local branch....
Let us now consider a simple curve
$C$ in $\Omega$
which contains two infinite pieces
$w=t+ib^*$ and $w=t+ib_*$ where $-\infty<t\leq a$.
Along $C$ we choose some  single valued branch of the function in (*)
denoted by $g_C(w)$.
If $z$ is a new complex variabve whose real part is $<0$
we get an absolutely convergent integral
cand it is clear that $J(z)$ is analytic in
$\mathfrak{Re}(z)>0$.
Moreover, the complex derivative
\[
 J_C(z)=
 \int\uuu {C\uuu R}\, e^{zw}\cdot g_C(w)\, dw\tag{1}
\]
\[
 J_C'(z)=
 \int\uuu {C\uuu R}\, e^{zw}\cdot w\cdot g_C(w)\, dw\tag{2}
\]
We can also perform a partial
integration and the reader may check that
\[
-z\cdot J(z)= \int\uuu {C\uuu R}\, e^{zw}\cdot \partial_w(g_C)(w)\, dw\tag{3}
\]
We can also replace the unbounded curves above by
closed Jordan curves $\Gamma$ as long as they do not contain any
$a$-point. Here it is obvious that 
\[
 J_\Gamma(z)=
 \int\uuu {\Gamma}\, e^{zw}\cdot g_\Gamma(w)\, dw
\]
is an entire function of $z$.
However,
during the analytic continuation of
$g$ along $\Gamma$, it may occur that
it takes distinct values after one full turn.
In this case (3) does not hold.
 







Next, the multi-valued function in (*) is a solution to a first
order differential equation. More precisely, in the
Weyl algebra of differential operators in the
$w$-variable we set
\[
Q(w,\partial_w)=q(w)\partial_w+\sum\,\lambda_\nu\cdot f_\nu(\lambda)
\]
and then $Q(g_C)=0$ holds.
Int the Weyl algebra  of the $z$-variable we associate the
differential operator
\[
Q^*(z,\partial_z)=-q(\partial_z)\circ z+\sum\,\lambda_\nu\cdot f_\nu(\partial_z)
\]
Then (x-xx) entail that
$Q^*(J)=0$.
From the algebraic calculations with differential operators in
�xx we can write
\[
Q^*(z,\partial_z)=-zq(\partial_z)+q'(-\partial_z)+\sum\,\lambda_\nu\cdot f_\nu(\partial_z)
\]
\medskip

\noindent
{\bf{Example. }}
Let $\alpha$  and $\gamma$ 
be a pair of non\vvv zero complex numbers and set
\[
 Q(w,\partial\uuu w)=(w+w^2)\partial\uuu w+2w\vvv\gamma w+1
 \vvv\alpha
 \]
Then we obtain 
\[
Q^*(z,\partial\uuu z)=z\cdot \partial^2+(\gamma\vvv z)\partial \vvv \alpha
\]
Starting with the $z$-variable we seek null solutions to the second order
differential operator $Q^*$.
To get such solutions we
first introduce the $g$-function which is a null solution to $Q$.
We try
\[
g(w)= w^a(1+w)^b
\]
Then we obtain
\[
Q(g)=(a(1+w)+bw+2w\vvv\gamma w+1-\alpha)g
\]
and this is zero for a unique pair $a,b$.
With this choice it follows that
the second order differential operator
$Q^*$ annihilates $J(z)$.
From the general results in � xx it follows that
each choice of $C$ and a branch of (xx) along this 
curve yields a $J$- function which from the start is analytic in
the left half-plane and after has an analytic contionuation to
the punctured $z$-plane where it satisfies the
equation $Q^*(J)=0$.
So one is led to analyze the null solutions of $Q^*$.
 



In as complex $w$\vvv plane we consider the simple curve
$C\uuu R$ which consists of the half circle
$\{w= Re^{i\theta}\quad \vvv \pi/2\leq \theta\leq \pi/2\}$
and the two horizontal  lines 
\[
\ell*(R)= w=
\{t\vvv iR\quad \vvv \infty<t\leq 0\}\quad\text{and}\quad
\ell^*(R)= \{t+iR\quad \vvv \infty<t\leq 0\}
\]
Her $R$ is chosen so large that the absolute values
$|a\uuu \nu|<R$  for each $\nu$.
In a neighborhood of $w=R$ we get local branches of the complex powers
$w^{\lambda\uuu \nu}$, i.e. with $w=R+\zeta$ and $\zeta$ small one has
\[
(R+\zeta)^{\lambda\uuu\nu}= e^{\lambda\uuu\nu\cdot \log(R+\zeta)}
\]
and the branch is chosen so that the value when $\zeta=0$ is
the ordinary complex exponential
$e^{\lambda\uuu\nu\cdot \log(R}$ where $\log R$ is real.
Now each function $w^{\lambda\nu}$ extends analytically along $C\uuu R$
and if $z$ is another complex variable whose real part is $<0$
we see that there exists a convergent integral

\[
 J(z)=
 \int\uuu {C\uuu R}\, e^{zw}\cdot
 \prod\, (w\vvv a\uuu \nu)^{\lambda\uuu\nu}
 \cdot dw\tag{*}
\]
It is clear that this $J$\vvv function is analytic in the half\vvv plane
$\mathfrak{Re}\, z>0$. 
In general $J(z)$ does not extend to an entire function.
Consider as an example the case $k=1$ with $a\uuu 1=0$
and set $\lambda=\lambda\uuu 1$. So here
\[ 
J(z)=\int\uuu {C\uuu R}\, e^{zw}\cdot w^\lambda\cdot dw
\]
When $z=x$ is real and positive we can perform the
variable substitution $w\mapsto u/x$ and get
\[ 
J(x)=
x^{\vvv \lambda\vvv 1}
\int\uuu {C\uuu {Rx}} e^u\cdot u^\lambda\cdot du
\]
The integral above is expressed  via the
$\Gamma$\vvv function, i.e. 
\[
J(x)=
x^{\vvv \lambda\vvv 1}\cdot \frac{2\pi i}{\Gamma(\vvv \lambda)}
\]
Hence  $J(z)$ extends to the function
$z^{\vvv\lambda\vvv 1}$ times
the constant $\frac{2\pi i}{\Gamma(\vvv \lambda)}$.
So unless $\lambda$ is an integer we get a multi\vvv valued $J$\vvv function.
For the general case (*) one has

\medskip

\noindent
{\bf{L.1 Theorem.}} \emph{The $J$\vvv integral extends to
${\bf{C}}\setminus (a\uuu 1,\ldots,a\uuu k)$
as  an analytic function
which in general is multi\vvv valued with ramification points
or poles  at
$a\uuu 1,\ldots,a\uuu k$.}
\medskip

\noindent
To prove this result one finds a differential 
equation with polynomial
coefficients satisfied by $J$, i.e. the efficient procedure is to use
$\mathcal D$\vvv module technique.
More precisely, suppose we have found a differential operator
\[
Q(w,\partial\uuu w)=\sum\, q\uuu j(w)\cdot \partial^j\uuu w
\]
in the Weyl algebra $A\uuu 1$ with respect to the $w$\vvv variable such that
\[ 
Q\bigl(  \prod\, (w\vvv a\uuu \nu)^{\lambda\uuu\nu})=0
\]
Then we associate the differential operator $Q^*$ in the $z$\vvv variable given by
\[
Q^*(z,\partial\uuu z)= \sum\, q\uuu j(\vvv \partial \uuu z)\cdot z^j
\] 
\medskip

\noindent
{\bf{Exercise.}} Show that we get $Q^*(J(z))=0$.
Then hint is that a partial integration gives
\[
zJ(z)= \vvv
 \int\uuu {C\uuu R}\, e^{zw}\cdot
\partial\uuu w\bigl( \prod\, (w\vvv a\uuu \nu)^{\lambda\uuu\nu}\,\bigr)
 \cdot dw\tag{i}
\]
At the same time we notice that
\[
\partial\uuu z(J)= \int\uuu {C\uuu R}\, e^{zw}\cdot
w\cdot \prod\, (w\vvv a\uuu \nu)^{\lambda\uuu\nu}\,\bigr)
 \cdot dw\tag{ii} 
 \]
 
\noindent
A computation will show that the differential operator
$Q^*$ has order $k$ and that the polynomial coefficient in front of
$\partial\uuu z^k$
is given by $\prod (z\vvv a\uuu\nu)$ from which
Theorem L.1 follows.
\medskip

\noindent
{\bf{Remark.}}
The computations above can be reversed and lead to integral representations of
functions which are solution to a differential equation defined by
$Q^*(z,\partial\uuu z)$ for a suitable $Q^*$.
A classic case are the confluent hypergeometric functions
which  are solutions
to certain differential equations.
More precisely, let $\alpha$  and $\gamma$ 
be a pair of non\vvv zero complex numbers. One
seeks solutions $f(z)$ to the second order equation
\[ 
z\cdot \partial^2(f)+(\gamma\vvv z)\dot \partial(f)\vvv \alpha\cdot f=0
\]
So here
\[ 
Q^*= 
z\cdot \partial^2+(\gamma\vvv z)\partial \vvv \alpha
\]
which in the non\vvv commutative Weyl algebra  can be written in the form
\[
\partial^2\cdot z\vvv 2\partial+ \gamma\partial\vvv
\partial \cdot z+1\vvv \alpha
\]
Hence $Q^*$ is associated to the differential operator
\[
 Q(w,\partial\uuu w)=w^2\partial\uuu w+2w\vvv\gamma w+w\partial\uuu w+1
 \vvv\alpha
 \]



\newpage



 



\end{document}