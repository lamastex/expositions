



\documentclass{amsart}
\usepackage[applemac]{inputenc}
\addtolength{\hoffset}{-12mm}
\addtolength{\textwidth}{22mm}
\addtolength{\voffset}{-10mm}
\addtolength{\textheight}{20mm}

\def\uuu{_}

\def\vvv{-}

\begin{document}





\centerline{\bf{ 1.A. The field of complex numbers}}

\bigskip


\noindent
A complex number $z$ is expressed by $x+iy$ where $(x,y)$ is a point in 
${\bf {R}}^2$. This identifies the complex plane ${\bf{ C}}$  with
${\bf{ R}}^2$. When $z=x+iy$ we set
\[
\mathfrak{Re}(z)=x\quad\colon\,\mathfrak{Im}(z)=y
\]
and refer to $x$ as the real part and $y$ as the imaginary part of
$z$.
The sum of two complex numbers is defined by
\[ z_1+z_2=(x_1+x_2)+i(y_1+y_2)\tag{i}
\]
\emph{Complex multiplication} 
is defined by:
\[ 
(x_1+iy_1)(x_2+iy_2)=
(x_1x_1-y_1y_2)+
i(x_1y_2+ix_2y_1)\tag{ii}
\]
One verifies that the product satisfies the associative law. If 
$z=x+iy$ its multiplicative inverse becomes
\[ 
z^{-1}=\frac{x-iy}{x^2+y^2}\,.\tag{*}
\]


\noindent
{\bf 1.1 Conjugation and absolute value.} If $z=x+iy$ its complex conjugate is 
$x-iy$ and  denoted by $\bar z$. The absolute 
value of $z$ is defined as 
$\sqrt{x^2+y^2}$ and  denoted by $|z|$.
The map $z\mapsto \bar z$
corresponds to reflection of plane vectors
with respect to the $x$-axis and (*) gives
\[
z^{-1}=\frac{\bar z}{|z|^2}\tag{**}
\]



\noindent
{\bf 1.2. The complex argument.} In ${\bf{ R}}^2$ 
we have polar coordinates $(r,\phi)$. If $z$ is non-zero
we  write:
\[ 
z=
r\cdot \text{cos}\,\phi
+i\cdot r\cdot\text{sin}\,\phi\quad\colon\quad r=|z|\,.\tag{1.2}
\]
The angle  $\phi$ is denoted
by $\text{arg}(z)$ and  called the argument of
$z$. 
Since   trigonometric functions are periodic,
$\text{arg}(z)$  is   determined up to an integer multiple of $2\pi$.
Specific choices of $\text{arg}(z)$ appear
in different situations. As an example we consider the 
upper half-plane $\mathfrak{Im}(z)>0$ where
one usually takes $0<\phi<\pi$ for  $\text{arg}(z)$. In the 
right half plane $\mathfrak{Re}(z)>0$
one takes $-\pi/2<\phi<\pi/2$.
Another case occurs when
we consider the polar representation of complex numbers
$z$ outside the \emph{negative} real axis $(-\infty,0]$.
Then  every $z$ has a 
unique polar
representation in (1.2) with $-\pi<\phi<\pi$.

\medskip

\noindent
{\bf 1.3. The complex number $e^{i\phi}$.}
This is  complex number has  absolute value  one and  
argument $\phi$. Thus
\[ 
e^{i\phi}=\text{cos}\,\phi
+i\cdot r\text{sin}\,\phi\,,\tag{1.3}
\]
where $e$ as \emph{ Neper's constant} 
defined by
\[ 
e=\lim_{n\to\infty}\,(1+\frac{1}{n})^n
\]
The notation (1.3) 
comes from the Taylor series expansions of the sine- and the cosine
functions. Recall from \emph{Calculus} that
\[
\text{sin}\,\phi=
\sum_{\nu=0}^\infty\,(-1)^\nu\cdot
\frac{\phi^{2\nu+1}}{(2\nu +1)!}\quad\colon\quad
\text{cos}\,\phi=
\sum_{\nu=0}^\infty\,(-1)^\nu\cdot
\frac{\phi^{2\nu}}{(2\nu !)}\tag{i}
\]
Adding these series and using that $i^2=-1$ which  gives $i^4=1$ and so on, we
get:
\[ 
\text{cos}\,\phi+
i\cdot \text{sin}\,\phi=
\sum_{\nu=0}^\infty\,\frac{i\phi)^\nu}{\nu\,!}\tag{ii}
\]
The last series resembles  the series of the real exponential function
from \emph{Calculus}:
\[ 
e^x=\sum_{\nu=0}^\infty\,\frac{x^\nu}{\nu\,!}\quad\colon\quad x\in{\bf{R}}
\tag{iii}
\]

\medskip


\noindent
{\bf 1.4 Addition formula for $\text{arg}(z)$.}
From Euclidian geometry
one has addition formulas for the sine-and the cosine functions:
\[ 
\text{sin}(\phi_1+\phi_2)=
\text{sin}(\phi_1)\text{cos}(\phi_2)+
\text{sin}(\phi_2)\text{cos}(\phi_1)\tag{1}
\]
\[
\text{cos}(\phi_1+\phi_2)=
\text{cos}(\phi_1)\text{cos}(\phi_2)-
\text{sin}(\phi_1)\text{sin}(\phi_2)\tag{2}
\]
Since (1) and (2) are
very essential results in complex analysis we
recall the proof. Let $\Delta$ be a triangle
with angles $\alpha,\beta,\gamma$ where
and $A,B,C$ denote the opposed sides and 
consider the case when
both $\alpha$ and $\beta$ are $<\pi/2$.
Now $\sin\gamma= \sin(\pi-\alpha-\beta)=
\sin\,(\alpha+\beta)$ and 
the sine-theorem in  euclidian geometry gives:
\[
\frac{\sin\alpha}{A}=
\frac{\sin\beta}{B}=
\frac{\sin\gamma}{C}\tag{i}
\]
Draw the normal line from the corner where
the angle which hits
the opposed side at a point whose
distance to the
$\alpha$-corner is $x$. So then
$C-x$ is the distance to the $\beta$-corner.
Looking at a figure
the reader can recognize that
\[
\cos\alpha=\frac{x}{B}\quad
\cos\beta=\frac {C-x}{A}\implies
A\cdot \cos\beta+B\cdot \cos\alpha+
=C\tag{ii}
\]
Together with (i) we get
\[
\sin(\alpha+\beta)= \frac{C}{A}\sin\alpha=
\sin\alpha\cdot \cos\beta+\sin\alpha\cdot \frac{B}{A}\cos\alpha
\]
Finally, the first equality in (i)
gives
$\sin\alpha\cdot \frac{B}{A}\cos\alpha=\sin\beta\cdot \cos\alpha$
and the requested addition formula for the sine-function follows.
A similar proof gives the addition formula for the cosine-function.


\medskip

\noindent
Next, the construction of complex multiplication and
(1.3) yields
the equality
\[
r_1\cdot e^{\phi_1}\cdot
r_2\cdot e^{\phi_2}=
r_1r_2\cdot e^{\phi_1+\phi_2}
\]
for all pairs of positive numbers
$r_1,r_2$ and a pair of $\phi$-angles.
So when complex arguments 
are identified up to integer multiples of $2\pi$ we get:
\[ 
\text{arg}(z_1)+
\text{arg}(z_2)=\text{arg}(z_1z_2)\tag{3}
\]
for each pair of non-zero complex numbers.
By an induction over $k$
the following hold for every
$k$-tuple of complex numbers:
\[
\sum_{\nu=1}^{\nu=k}\, \text{arg}(z_\nu)=
\text{arg}(\prod_{\nu=1}^{\nu=k}\, z_\nu)\,.\tag{*}
\]


\noindent
We refer to (*) as the addition formula for the argument function.
It plays a fundamental role in complex analysis.


\medskip


\noindent
{\bf{1.5 Associated matrices.}}
Let $z=a+ib$ be a complex number.
Identifying ${\bf{C}}$ with ${\bf{R}}^2$
the complex multiplication with $z$
yields a linear operator represented by a matrix.
More precisely, the euclidian basis vectors $e_1,e_2$ correspond to
the complex numbers 1 and $i$.
Since $z\cdot i=ia-b$  the $2\times 2$-matrix $M_z$
associated to multiplication with $z$ becomes
\[ 
M_z=
\begin{pmatrix}  a&-b\\b&a
\end{pmatrix}
\]
Notice that the determinant of $M_z$
is $a^2+b^2$
and the inversion formula (*) from 1.1 corresponds to
the matrix identity
\[
M_z^{-1}= \frac{1}{a^2+b^2}\cdot 
\begin{pmatrix}  a&b\\-b&a
\end{pmatrix}
\]
\bigskip

\noindent
{\bf{1.6 A polynomial approximation.}}
With $0\leq \phi\leq 2\pi$
we consider the $\phi$-polynomials
\[
P_n(\phi)=\bigl(1+\frac{i\phi}{n}\bigr)^n\tag{i}
\]
Notice that
\[
\text{arg}(1+\frac{i\phi}{n})=\frac{\phi}{n}
\]
The addition formula (*) in 1.4 therefore gives
\[
\text{arg}(P_n(\phi))=\phi
\] 
for every $n$.
Next, regarding absolute values we have
\[
|1+\frac{i\phi}{n}|^2=1+\frac{\phi^2}{n^2}\implies
|(P_n(\phi)|^2=
\bigl(1+\frac{\phi^2}{n^2}\bigr)^n
\]
Recall from calculus that
$\log(1+t)\leq t$ for each real $t>0$.
With $0\leq\phi\leq 2\pi$ it follows that
\[
\log |P_n(\phi)|^2\leq \frac{\phi^2}{n}\leq \frac{4\pi^2}{n}
\]
It follows that
\[
\lim_{n\to\infty}\, |P_n(\phi)|=1
\]
holds uniformly on $[0,2\pi]$
and the reader may also notice that Neper's
limit formula for $e$ entails that
\[
\lim_{n\to\infty}\, P_n(\phi)= e^{i\phi}
\]


\noindent{\bf{Remark.}}
The function $\phi\mapsto e^{i\phi}$ is $2\pi$-periodic which entails that
\[ 
\lim_{n\to\infty}\, (1+\frac{2\pi i}{n})^n=1
\]
It is instructive to check this limit formula  numerically with a computer for
some relatively large values of $n$. Notice also that 
the periodicity  gives the following limit formula for every integer $k$:
\[ 
\lim_{n\to\infty}\, \bigl(1+\frac{(2k+1)\pi i}{n}\bigr)^n=-1
\]

\medskip


\noindent {\bf{1.7 Complex numbers and  geometry.}}
Many results in euclidian geometry can be proved
in a neat fashion by complex numbers.
Let us give an example.
Consider a triangle
$\Delta$  with sides of length $a,b,c$.
Let $\alpha$ be the angle at the corner $p$ point opposed to
the side of length $a$. Then
\[ 
\cos\alpha=\frac{b^2+c^2-a^2}{2bc}\tag{*}
\] 
To prove this we may without loss of generality assume
that
the corner point $p$ is the origin
and the two other corner points of
$\Delta$ are  represented by a pair of complex vectors
$z$ and $w$. Here
\[ 
a^2=|z-w|^2\quad\colon\, |z|^2=b^2\quad\colon |w|^2=c^2
\]
The formula  (*) is invariant under  dilation with
$z$ replaced by $rz$ and $w$ by $rw$ for some $r$, and also under a 
rotation. So without loss of generality we can take $w=1$ and $z=x+iy$ with
$x>0$.
In this case a figure - or rather the definition of the cosine-function gives
\[ 
\cos\alpha=\frac{x}{|z|}
\]
So (*) amounts to prove the equation
\[
\frac{x}{|z|}=\frac{|z|^2+1-|1-z|^2}{2|z|}\tag{i}
\]
Above $|z|$ is cancelled and
we have
\[
|z|^2+1-|1-z|^2=x^2+y^2+1-( y^2+(1-x)^2)=2x
\]
which gives (i) and  (*) follows.
This illustrates how complex numbers
provide an efficient tool
to establish  geometric formulas.
\medskip

\noindent
{\bf{Exercise.}}
Let $\Delta$ be a triangle with corner points at the origin, $(1,0)$ and
$z_0=x_0+iy_0$ where $|z|\leq \sqrt{2}$ and both $x_0$ and $y_0$ are positive.
The line $\ell$ passing $(1,0)$ which is $\perp$ to the vector $z_0$
consists of complex numbers of the form $1+{\bf{R}}\cdot iz_0$
where we use that the vectors $z_0$ and $i\cdot z_0$ are
$\perp$ to each other.
The normal from the corner point $z_0$
stays on the line $\{x=x_0\}$
and to get the intersection point  of $\ell$ and this normal we seek a real number
$a$
such that
\[
x_0= 1+ai(x_0+iy_0)\implies a= \frac{1-x_0}{y_0}\tag{i}
\]
Hence the intersection point becomes
$(x_0+iy_*$ where
(i) gives
\[
y_*=x_0\cdot \frac{1-x_0}{y_0}\tag{ii}
\]
Next, we draw the line from the origin passing
$(x_0,y_*)$ and it turns out that it is $\perp$ to
the vector $1-z_0$. This amounts to show that
there exists a \emph{real} number $b$ such that
\[
x_0+iy_*=bi(1-z_0)= by_0+ib(1-x_0)\tag{iii}
\]
But this is clear from (ii) which
shows that (iii) holds with $b=\frac{x_0}{y_0}$.
So these complex computations verify the wellknown 
fact that the three  normals  intersect at a point.


 




\newpage






\centerline {\bf B. The fundamental theorem of algebra.}
\medskip

\noindent
{\bf{Introduction.}}
The proof of Theorem B.2  below was given by
Cauchy in 1815 based upon
the analytic result that the absolute value of a complex-valued continuous function
on a compact disc  achieves its minimum
some   point.
In the article [Weierstrass]
from 1868
Weierstrass 
gave another proof.
Here follows a citation 
from the introduction in [ibid]:
\emph{Obgleich wir gegenw�rtig von dem in Rede stehenden 
Fundamentaltheoreme der Algebra eine Reihe strengen Beweise besitzen, so d�rfte
doch die Mitteilung der nachstehenden Begr�ndung desselben,
deren Eigenth�mlichkeit haupts�chlich darin besteht, dass sie ohne
Heranziehung von Hilfsmitteln und begriffen
die der Algebra
fremd sind, rein arithmetisch
durchgef�hrt wird, vielen Mathematikern nicht unwillkommen sein.}
So Weierstrass points out that in spite of the already known
existence proofs, a procedure
which is not too   remote from algebra 
derives 
the fundamental theorem of algebra  by  arithematical methods, a fact that
might be appreciated by many mathematicians.
The merit in [ibid] is that it  gives  a 
method to get 
numerical approximations of 
roots to a 
polynomial 
\[
P(z)=
z^n+c_{n-1}z^{n-1}+\ldots+c_1z+c_0\quad\colon\quad 
c_0,\ldots,c_{n-1}\,\,\text{are complex numbers}\tag{*}
\]
of some degree $n\geq 2$. To begin with
Weierstrass  made the observation 
that the fundamental theorem of algebra
amounts to show that for an arbitrary
$n$-tuple
$c_0,c_1,\ldots,c_{n-1}$ expressing $P$ by (*)
there exists a unique 
unordered $n$-tuple of complex numbers
$\alpha_1,\ldots,\alpha _n$ such that
\[
P(z)= \prod_{\nu=1}^{\nu=n}\, (z-\alpha_\nu)
\]
In other words
the mapping
of unordered $n$-tuples of $\alpha$-numbers to their associated
symmetric polynomials is injective and the range 
is equal to all ordered
complex $n$-tuples
$c_0,c_1,\ldots,c_{n-1}$. Or equivalently, for each
$n$-tuple of complex numbers
$w_1,\ldots,w_n$ there exists a unique unordered $n$-tuple $\{\alpha_\nu\}$
such that
\[ 
\sum_{\nu=1}^{\nu=n}\,\alpha_\nu^k=w_k\quad\colon
1\leq k\leq n
\]

\noindent
Above
the polynomial $P(z)$ in (*)
has simple zeros if and only if
the ideal generated by $P(z)$�and its derivative $P'(z)$
is equal to ${\bf{C}}[z]$, i.e. 
there exists a unique pair of polynomials $A,B$ such that
\[
1=A(z)P(z)+B(z)P'(z)\tag{i}
\]
where $\deg A\leq n-2$.
The existence of such a pair $A,B$ is equivalent
to the existence of a solution of a linear system of equations
in $2n-1$ many indeterminates corresponding to coefficients of $A$ and $B$.
Cramer's rule gives a criterion for the existence of a solution
which
expressed by
an algebraic equation
\[
\mathcal D_n(c_0,\ldots,c_{n-1})=0
\]
where $\mathcal D_n$ is a polynomial in $n$ variables with 
integer coefficients. We leave it as an exercise to find $\mathcal D_n$, if necessary a text-book
in algebra can be consulted.
Starting with an $n$-tuple $\{c_\nu\}$
where the (i) has a solution, Weierstrass demonstrates that
that for each $\epsilon>0$
there exists a finite number of arithmetical operations which give
an unordered $n$-tuple of complex numbers $a_1,\ldots,a_n$ 
determined by the given $c$-coefficients 
such that if
\[
Q(z)=\prod_{\nu=1}^{\nu=n}\, (z-a_\nu)=z^n+\sum\, c^*_\nu z^\nu
\]
then $|c_\nu-c^*_\nu|<\epsilon$ for each $0\leq \nu\leq n-1$.
Next, starting with a sufficiently small $\epsilon$ Weierstrass proved
that the 
roots of the $Q$-polynomial 
approximate  the true roots of $P$ by recursive formulas.
More precisely,
set
\[ 
a_\nu^{(1)}=a_\nu-\frac{P(a_\nu)}{\prod_{j\neq \nu}(a_\nu-a_j)}
\]
Inductively  we put:
\[
a_\nu^{(k+1)}=a_\nu-\frac{P(a^{(k)}_\nu)}{\prod_{j\neq \nu}(a^{(k)}_\nu-a^{k)}_j)}
\]
Then it is proved in [ibid] that  the true roots of $P$ are given by
\[
a_\nu^*=\lim_{k\to \infty}\,a_\nu^{(k)}
\]
Moreover,  the rate of convergence is
rapid in the sense that there is a constant 
$C$ which depends on $P$ and the choice of $\epsilon$ such that
\[
|a_\nu^*- a_\nu^{(k)}|\leq C\cdot  2^{-k}\quad\text{
for every}\quad  1\leq\nu\leq n
\]
Weierstrass' constructions 
can be implemented into  a computer which leads to
to approximations of   zeros   polynomials
with high accuracy.
So readers interested in numerical investigations should consult
the  rich   material in   [Weierstrass].


\medskip

\noindent
{\bf{Cauchy's proof}}.
Here we admit
the  existence  of
extremal values taken by continuous functions on compact sets.
Let $P(z)$ be given in  (*).
If $P$ has a zero $\alpha$ one gets a factorisation
\[ 
P(z)=(z-\alpha)(z^{n-1}+d_{n-2}z^{n-2}+\ldots+d_1z+d_0)
\]
where the $d$-coefficients are found by algebraic identities.
One has for example
\[ 
d_{n-2}=c_{n-1}-\alpha\quad\colon\quad d_{n-3}=
c_{n-2}-\alpha d_{n-2}
\] 
and so on.
If the factor polynomial of degree $n-1$ also has a complex root we
can continue and conclude
\medskip

\noindent
{\bf Proposition.} \emph{Assume that every polynomial $P(z)$ 
has at
least one complex root. Then it has a factorisation}
\[ 
P(z)=\prod_{\nu=1}^{\nu=k}\,(z-\alpha_\nu)
\]
\emph{Here $k$ is the degree of $P$ and
$\alpha_1,\ldots,\alpha_k$ is a $k$-tuple of complex numbers where repetitions
occur when
$P$ has multiple roots.}
\bigskip

\noindent
Hence the fundamental theorem of algebra follows 
if we have proved:

\bigskip

\noindent 
{\bf B.1  Theorem} \emph{Every polynomial $P(z)$ has at least one root.}
\bigskip

\noindent
{\bf{Remark.}}
The proof below  relies upon the fact that
absolute values of complex polynomials cannot achieve local minima.
Consider as an example some integer $k\geq 2$ and the function
\[
g(z)= |1+z^k|^2
\] 
Here $g(0)=1$ but
$z=0$ is not a minimum for with a small $\epsilon>0$
we can take $z=\epsilon\cdot e^{\pi i/k}$
which gives $z^k= \epsilon^k$ and hence
\[ 
g(\epsilon\cdot e^{\pi i/k})=(1\vvv \epsilon^k)^2<1
\]
Notice the contrast to arbitrary real polynomials where
a minimum can occur. For example, the polynomial 
$g(x,y)= 1+ x^4+x^2y^2+y^4$ has a minimum at the origin
and no zeros in the $(x,y)$\vvv plane.

\medskip


\noindent
 \emph{Proof of Theorem B.1:}
We are given $P(z)$ as in (*) above and 
put $M=|c_0|+\ldots+|c_{k-1}|$.
If $|z|\geq 1$
the triangle inequality  gives
\[
|P(z)|\geq |z|^k-M\cdot |z|^{k-1}\geq |z|-M\tag{i}
\]
With $R=M+2\cdot |c_0|$ it follows that
\[
|z|\geq R\implies |P(z)|\geq R-M\geq 2\cdot |c_0|=2\cdot |P(0)|\tag{ii}
\]
Next,
the restriction of $P(z)$ to the closed disc $|z|\leq R$
is a continuous function and therefore the absolute value
takes a minimum at some point $z_0$ which in particular gives
$|P(z\uuu 0)|\leq |P(0)|$.
Hence (ii) implies that we have a global minimum, i.e.
\[
|P(z\uuu 0)|\leq |P(z)|\tag{iii}
\] 
hold for all $z$.
To show that (iii) entails $P(z\uuu 0)=0$
we argue by contradiction, i.e suppose that
$P(z\uuu 0)\neq 0$ and with a new variable $\zeta$ we get
the polynomial 
\[ 
P(z_0+\zeta)=P(z_0)+d_m\zeta^m+d_{m+1}\zeta^{m+1}+\ldots+d_k\zeta^k\tag{iv}
\] 
where $1\leq m\leq k $ and $d_m\neq 0$. We  find real numbers
$\alpha,\beta$ such that
\[
\, P(z_0)= |P(z_0)|e^{i\alpha}\quad\text{and}\quad
d_m=|d_m|e^{i\beta}\tag{v}
\]

\medskip

\noindent
Next, with
$\epsilon>0$ we set
\[ 
\zeta=\epsilon\cdot e^{i\cdot \frac{\pi+\alpha-\beta}{m}}\tag{vi}
\]
Since $e^{i\pi}=-1$  this choice of $\zeta$ together with
(v) gives
\[
P(z_0)+d_m\zeta^m=(1-|d_m|\cdot \epsilon^m)P(z_0)\tag{vi}
\]
Put $M^*=|d_{m+1}|+|d_{m+2}|+\ldots |d_k|$.
When $\epsilon<1$ the triangle inequality gives
\[
|d_{m+1}\zeta^{m+1}+d_{m+2}\zeta^{m+2}+\ldots d\uuu k z^k|
\leq M\cdot \epsilon^{m+1}\tag{vii}
\]
Together with (vii)
another application of the triangle inequality gives:
\[
|P(z_0+\epsilon\cdot 
e^{i\cdot \frac{\pi+\alpha-\beta}{m}})|\leq
|P(z_0)|(1-|d_m|\epsilon^m|+M\cdot \epsilon^{m+1}=
\]
\[
|P(z_0)|-\epsilon^m\bigl(
|d_m|\cdot |P(z_0)|-M\cdot\epsilon\bigr)\tag{viii}
\]
Now we can take
\[
0<\epsilon<\frac{|d_m|\cdot |P(z_0)|}{M}
\] 
and then (viii) gives a strict inequality
\[
|P(z_0+\zeta)|
<|P(z_0)|
\]
This contradicts that $z_0$ gave a minimum for the absolute value of $P$
and the proof is
finished.



\medskip

\noindent
{\bf Proof by residue calculus.}
Later Cauchy gave other proofs using reside theory in his 
famous text-books devoted to analytic functions.
For if the polynomial $P(z)$ in (*) has no complex zeros
then
$P^{-1}(z)$ is an entire function and
taking the complex derivative $P*(z)$ it follows that
the complex line integrals
\[
\int_{|z|=R}\, \frac{P'(z)}{P(z)}\, dz=0
\]
for all $R$. When
the line integral is evaluated in polar coordinates it becomes
\[
\int_0^{2\pi}\, 
\frac{n+(n-1)c_{n-1}R^{-1}e^{-i\theta}+
\ldots c_1R^{-n-1}e^{-i(n-1)\theta}}
{1+c_{n-1}R^{-1}e^{-i\theta}+\ldots+c_0R^{-n}e^{-in\theta}}\, d\theta
\]
Passing to the limit as $R\to+\infty$
the last integral converges to $n$ which gives
the contradiction. Cauchy concluded that $P$ must have at least one zero
and actually residue calculus immediately entails that the  number of zeros
counting multiplicities is equal to the degree of the polynomial.

\medskip


\noindent {\bf{Remark.}}
If the degree of $P(z)$ is $\leq 4$
one can find the  roots by  
{\emph{Cardano's formula}. See � B.3 for an example.
But
as soon as the degree is $\geq 5$
it is in general not possible to find the zeros of a polynomial
by  roots and radicals even if the coefficients are integers.
This was proved by
Niels Henrik Abel in 1823 whose 
article [Ab:1]  published in the first volume of Crelle's Journal 
contains pioneering
results about
algebraic field extensions.
Abel 
used  these new  discoveries  to prove
that the general algebraic
equation of degree $\geq 5$ cannot be solved by roots and
radicals by investigating a  system of 120 linear equations expressed by
the
coefficients of a polynomial in degree $\geq 5$. An example 
from Abel's work where a Cardano solution fails is
the equation
\[ 
z^5+z+1=0
\]
For  an account
about Abel's contributions
the reader should consult
articles  from
\emph{The Abel Legacy}
published in 2004 on the occasion of the
first
Abel Prize. After
Abel's  decease in 1829, 
Everiste Galois  constructed  a  group to every
 polynomial which as an alternative to Abel's criterium
 also can be used
 to decide
when   zeros of a polynomial can be
found by roots and
radicals. This leads to \emph{Galois theory} which brings the theory
about field extensions together with group theory and has become
a central topic   in algebra.

\medskip


\noindent
{\bf{An algebraic problem.}}
Consider a pair of polynomials 
\[
p(z)= z^n+a\uuu 1z^{n\vvv 1}+\ldots+a\uuu{n\vvv 1}z+a\uuu n
\]
\[
q(z)= z^n+b\uuu 1z^{n\vvv 1}+\ldots+b\uuu{n\vvv 1}z+b\uuu n
\]
where $\{a\uuu\nu\}$ and $\{b\uuu\nu\}$ are rational numbers.
Both polynomials are assumed to be irreducible in the unique factorization domain
$Q[z]$ which entails that
the roots of $p$ and $q$ are simple.
By the fundamental theorem of algebra we can write

\[
p(z)= \prod\, (z\vvv\alpha\uuu j)\quad\text{and}\quad
q(z)= \prod\, (z\vvv\beta\uuu j)
\]
Each root $\alpha\uuu j$ of $p$ 
generates a field $K=Q[\alpha\uuu j]$
which as a vector space of $Q$ has dimension $n$
and a basis is given by
$1,\alpha\uuu j,\ldots,\alpha\uuu j^{n\vvv 1}$. In fact, this
holds since $p(z)$ was irreducible and we remark that
the field $K$ is isomorphic to the field
$\frac{Q[z]}{(p)}$
where $(p)$ denotes the pirnicipal ideal generated by $p$
in the polynomial ring
$Q[z]$.
Similar conclusions hold for the roots of $q$.
Now
one may ask when there exists a pair of roots $\alpha\uuu j,\beta\uuu \nu$
for $p$ and $q$ respectively, such that the  fields
$K[\alpha\uuu j]$�and $K[\beta\uuu \nu]$ are equal.
By elementary field theory
the necessary and sufficient condition for the
equality $K[\alpha\uuu j]=K[\beta\uuu \nu]$ is that 
\[ 
\beta\uuu \nu=q\uuu 0+q\uuu 1\cdot \alpha\uuu j+\ldots+
q\uuu{n\vvv 1}\alpha\uuu j^{n\vvv 1}\tag{i}
\] 
holds for some $n$\vvv tuple $\{q\uuu\nu\}$
of rational numbers. The problem is to find
equations satisfied by the pair of $n$\vvv tuples $\{a\uuu j\}$ and
$\{b\uuu k\}$ which appear as coefficients of the two polynomials in order
that (i) holds for some pair of roots.
This is a problem in algebraic elimination theory
and  solved as follows:
Let $\lambda,\xi\uuu 0,\ldots,\xi\uuu {n\vvv 1}$ be $n+1$ many new
variables and set
\[
S(\lambda,\xi\uuu 0,\ldots,\xi\uuu {n\vvv 1})=
\prod\uuu {j=1}^{j=n} \, \bigl (\lambda\vvv 
(\xi\uuu 0+\xi\uuu 1\alpha\uuu j+\ldots+\xi\uuu{n\vvv 1}\alpha\uuu j^{n\vvv 1})\,\bigr)
\]
This is a \emph{symmetric} expression in
the $n$\vvv tuple of roots of $p$ and text\vvv books in elementary algebra teaches
that every symmetric polynomial of the roots can be expressed as a polynomial of the coefficients with integer coefficients.
It follows that
\[
S(\lambda,\xi\uuu 0,\ldots,\xi\uuu {n\vvv 1})=
\sum\uuu{j=0}^{j=n\vvv 1}\, 
\phi\uuu j(a\uuu 1,\ldots,a\uuu n,\xi\uuu 0,\ldots,\xi\uuu {n\vvv 1})\cdot \lambda^j\tag{ii}
\]
where $\{\phi\uuu j\}$
are polynomials with integer coefficients 
of the two $n$\vvv tuples $\{a\uuu\nu\}$ and $\{\xi\uuu\nu\}$
expressed by explicit interpolatation formulas.
With these notations, (1) is satisfied for a pair of roots if and only if
the $\lambda$\vvv polynomial in (ii) has at least one root in common with
$q$. To check if this holds one employs
a determinant of a certain $2n\times 2n$\vvv matrix
whose elements are determined explicitly by
the coefficients $\{b\uuu\nu\}$ and the $n$\vvv tuple $\phi\uuu j(a,\xi)$.
See Exercise � xx from  � I:C for this.
The conclusion  is that
there exists a polynomial of the $\xi$\vvv variables
\[
\mathcal S(\xi\uuu 0,\ldots,\xi\uuu {n\vvv 1})
=\sum\, \rho\uuu\gamma (a\uuu\bullet,b\uuu\bullet)\cdot \xi^\gamma 
\] 
where $\{\rho\uuu \gamma\}$ are polynomials of
the $2n$\vvv tuple formed by the coefficients of the given polynomials and 
$\gamma =(\gamma \uuu 0,\ldots \gamma\uuu {n\vvv 1}) $
are multi\vvv indices expressing
the monomials
\[ 
\xi^\gamma=
\xi\uuu 0^{\gamma\uuu 0}\cdots
\xi\uuu{n\vvv 1}^{\gamma\uuu{n\vvv 1}}
\]
Now (i)  has a solution with rational numbers
$\{q\uuu \nu\}$ if and only if
$\mathcal S(q\uuu 0,\ldots,q\uuu{n\vvv 1})=0$.
In other words, the necessary and sufficient condition to obtain (i) for a 
pair of roots is that
the $\mathcal S$\vvv polynomial of $n$ variables has
at least one zero in the $n$\vvv dimensional $\xi$\vvv space given
by an $n$\vvv tuple of rational numbers.
Using terminology from  algebraic geometry it means that the algebraic hypersurface
$\{\mathcal S=0\}$ contains at least one rational point.
This example  gives a glimpse of elimination theory where
the problems consist in finding various algorithmic formulas.
Concerning the specific problem above we remark that
calculations which lead to equations in order
that 
(i)  holds were carried out  in work by
Delannay and
Tschebotar�w for polynomials of degree $\leq 4$.
The interested reader may consult the plenary talk by
Tschebotar�w from the IMU\vvv congress at Z�rich in 1932
which describes  the interplay between the
problem above and  Galois theory. It appears 
that a complete  investigation for arbitrary large $n$ remains  unsettled.
\bigskip


\noindent{\bf{B.2 Algebraic numbers.}}
Of special interest are complex numbers which are \emph{algebraic}
over the field $Q$ of rational numbers, i.e. complex numbers 
$\alpha$ which are roots to some polynomial whose coefficients are 
rational numbers.
The set of all such complex numbers is a subfield of
$\bf C$ denoted by $A$. 
Inside $A$ there occur  subfields generated by roots to a finite
family of polynomials. These  subfields are finite dimensional vector spaces over
$Q$ and  called finite algebraic fields.
Given such a field $K$ one then gets a subring $\mathcal D(K)$
which consists of all $\alpha\in K$ such that
$\alpha$ is a root of
a monic polynomial with integer coefficients, i.e. $\alpha$ satisfies an equation
\[ 
\alpha^m+c_{m-1}\alpha^{m-1}+\ldots+c_1\alpha+c_0\quad\colon\quad
c_0,\ldots,c_{m-1}\,\,\text{are integers}
\]
The ring $\mathcal D(K)$ is a Dedekind ring and enjoys  nice properties
which are exposed in text\vvv books
devoted to algebraic number fields.
Analytic function theory is used to
study approximations of algebraic numbers by
rational numbers.
Let $\xi$ be a positive real number which satisfies
an algebraic equation
\[
\xi^n+c_{n-1}\xi^{n-1}+\ldots+c_1\xi+c_0=0
\]
where $\{c_\nu\}$ are integers
and the polynnomial $P(z)= z^n+\sum\, c_\nu z^\nu$ is irreducible
in the unique factorisation
domain
$Q[z]$.
In 1908 Thue proved a remarkable result in the article \emph{Bemerkungen �ber gewisse N�herungsbr�che algebraishen Zahlen}.
Namely, for
every $\epsilon>0$ the set of positive rational numbers
$q=\frac{x}{y}$
such that
\[
\bigl|\xi-\frac{x}{y}\bigr|\leq \frac{1}{y^{\frac{n}{2}+1+\epsilon}}\tag{*}
\] 
is finite.
Thue's result means  that
there are lower bounds for
approximations of algebraic integers which are not rational
numbers.
In his thesis from 1921, Siegel proved
that if $\xi$ is as above
then the set of rational numbers
$\frac{x}{y}$ for which
\[
\bigl|\xi-\frac{x}{y}\bigr|\leq \frac{1}{y^{2\sqrt{n}}}\tag{**}
\] 
is finite. Notice that Sigel's result
improves (*) as soon as $n\geq 16$.
The proof of  (**)
is quite involved.
The interested reader may  consult
Siegel's article
\emph{�ber N�herungswerte algebraischen Zahlen}( Math. Zeitschrift 1921)
 for refined results about
approximations of algebraic numbers by
rationals.
But let us give one of
the minor steps from 
Siegel's impressive work.
\medskip

\noindent
{\bf{An inequality by Siegel.}}
Let $p(z)= z^n+c_{n-1}z^{n-1}+\ldots+ c_1z+c_0$
be a polynomial with integer coefficients.
Suppose that $p(z)$ has a factorisation
in the polynomial ring
$Q[z]$:
\[ 
p(z)= (k_0z^m+\ldots+ k_{m-1}z+k_m)\cdot q(z)
\] 
where 
$1\leq m< n$ and $k_0,\ldots,k_m$ are integers with no 
common divisor $\geq 2$ while $q(z)$ is a polynomial of degree
$n-m$ in $Q[z]$.
Set
\[
\rho^*=\max\{ |c_0|,\ldots,|c_n|\}\quad\text{and}\quad
\rho_*=\max\{ |k_0|,\ldots,|k_m|\}
\]
Then one has the inequality
\[
\frac{\rho_*}{\rho^*}\leq (m+1)\cdots n\tag{*}
\]



\noindent
\emph{Proof.}
Consider first a polynomial
$f(z)= a_0z^k+\ldots+a_k$ of some degree $k\geq 1$ with
arbitrary complex coefficients.
Let $\lambda\neq 0$ be another complex numbers
and set
\[ 
g(z)= (z-\lambda)f(z)= d_0z^{k+1}+\ldots+ d_kz+d_{k+1}
\]
Let $d^*=\max\{ |d_\nu|\}$ and $c^*=\max\{|c_\nu\}$.
Then one has the inequality
\[
\frac{a^*}{d^*}\leq k+1\tag{1}
\]
To prove (1) we notice that
\[
a_\nu=d_0\lambda^ \nu+d_1\lambda^{\nu-1}+\ldots+ d_\nu\quad\colon\, 
0\leq \nu\leq k
\]
If $|\lambda|\leq 1$
it follows that
\[ 
|a_\nu|\leq |d_0|+|d_1|+
+\ldots+ |d_\nu|\leq (\nu+1)\cdot d^*
\] 
Since this holds for every $\nu$ we get
$a^*\leq (k+1)d^*$ as requested.
Next, if $|\lambda|>1$
we rewrite
(1) so that
\[
(z-\frac{1}{\lambda})(a_kz^k+\dots+a_0)=
-\frac{1}{\lambda}\cdot (
d_0+\ldots+ d_{k+1}z^{k+1})
\]
Since $\frac{1}{\lambda}$ has absolute value $\leq 1$
the previous case entails that
\[
a^*\leq (k+1)\frac{d^*}{|\lambda|}\leq (k+1)d^*
\]
and hence
(1) also holds when  $|\lambda|\geq 1$.
To prove  (*) we consider
the factorisation
\[  
q(z)=
k_0^{-1}(z-\lambda_1)\cdots(z-\lambda_{n-m})
\]
Hence the polynomial $p(z)$
arises from 
$p_*(z)= k_0^{-1}(k_0z^m+\dots+ k_m)$
via an $(n-m)$-fold application of the case above
and from this the reader can deduce that
\[
\frac{\rho_*}{\rho^*}\leq (m+1)\cdots n
\]
which proves (*).




\end{document}






