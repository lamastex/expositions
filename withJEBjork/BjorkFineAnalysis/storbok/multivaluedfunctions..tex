
\documentclass{amsart}


\usepackage[applemac]{inputenc}
\addtolength{\hoffset}{-12mm}
\addtolength{\textwidth}{22mm}
\addtolength{\voffset}{-10mm}
\addtolength{\textheight}{20mm}

\def\uuu{_}

\begin{document}

\centerline{\bf\large  Chapter 4: Multi-valued analytic functions}

\bigskip

\noindent
0. Introduction
\medskip

\noindent
1. Angular variation and Winding numbers
\medskip

\noindent
2. The argument principle
\medskip

\noindent
3. Multi-valued functions

\medskip


\noindent
4. The monodromy theorem
\medskip

\noindent
5. Homotopy and Covering spaces
\medskip

\noindent
6. The uniformisation theorem
\medskip


\noindent
7. The $\mathfrak{p}^*$-function


\medskip

\noindent
8. Eistensein's theorem
\medskip

\noindent
9. Reflections across a boundary


\medskip

\noindent
10. The elliptic modular function
\medskip


\noindent 
11. Work by Henri Poincar�





\bigskip



\noindent
\centerline {\bf\large{ Introduction.}}
\medskip


\noindent
In � 1 we define winding numbers of curves
in ${\bf{R^2}}$
where no complex variables appear.
After his we study
complex line integrals where
Theorem 1.7 which is the first main result in this chapter.
The second main result is
Theorem 2.1, referred to as the
\emph{argument principle}. It gives  a gateway to
find zeros of an analytic function since the counting function of its
zeros is expressed by winding numbers arising from
the image under the given function of closed boundary curves to a domain where
we seek the number of zeros of $f$.
\medskip

\noindent � 3 starts with 
a construction due to Weierstrass
and leads to the total analytic continuation of   germs of
analytic functions.
A  picture arises when we  
consider the \emph{total sheaf space} 
$\widehat{\mathcal O}$
whose stalks are germs of analytic functions 
at points in ${\bf{C}}$.
This topological manifold  is locally homemorphic to
small discs in $\bf C$ which   express 
germs of multi-valued
functions $f$.
More precisely, 
if $\rho$ denotes the local homemorphism from 
$\mathcal {\widehat O}$ onto $\bf C$ , then
there is a 1-1 correspondence between connected open subsets of
$\rho^{-1}(\Omega)$ and the class of multi-valued analytic functions in
$\Omega$, where on does not exclude those functions $f$ which may
have analytic continuation to larger sets.
Using the Weierstrass procedure to
perform analytic continuation along a curve by regarding
local power series in small discs placed in succesion along the curve, one
can  prove
that various  classes of multi-valued analytic functions are
\emph{normal} in the sense of Montel. In other words, 
results for single-valued analytic functions from � X in Chapter III
can be extended to the multi-vaued case.

\medskip

\noindent
In � 4 we prove the 
\emph{Monodromy Theorem}  and describe
multi-valued functions in a punctured disc. After this we insert material
from topology in � 5 and announce
the uniformisation theorem in � 6.
In  �7 we construct the
${\bf{p}}^*$-function which will be used
to solve 
the Dirichlet problem in Chapter 5 and 
� 8 is devoted to a  result due to
Eisenstein. The
proof is instructive since it 
teaches how to manipulate with
multiple-valued algebraic root functions. In � 9 we recall the 
\emph{Spiegelungsprinzip}
by  Hermann Schwarz which is
applied
in � 10 to construct the
\emph{modular function} and after can be used
to establish
the uniformisation theorem for domains in
${\bf{C}}$.
A special local study about algebraic functions is made in � 11
where the results are used to construct Riemann surfaces of
algebraic function fields.
The final Section 12 contains  a brief exposittion 
about  contributions by Poincar�, foremost
his constructions of Fuchsian groups.




\newpage



\centerline{\bf 1. Angular variation and Winding numbers}



\bigskip

\noindent
Let $(x,y)$ be the coordinates
in ${\bf{R}}^2$ and
consider a vector-valued function of the real parameter $t$:
\[ 
t\mapsto (x(t),y(t))\quad\colon\quad 0\leq t\leq T
\]
We assume that the image does not contain the origin, i.e. 
$x^2(t)+y^2(t)>0$ and  $x(t)$ and $y(t)$  are both
continuously differentiable. Set
\[
\dot x(t)=\frac{dx}{dt}\quad\text{and}\quad
\dot y(t)=\frac{dy}{dt}\quad\text{and}\quad 
r(t)=\sqrt{x(t)^2+y(t)^2}
\]
\medskip

\noindent 
{\bf{1.1 Proposition.}}
\emph{There exists a unique continuous map $t\mapsto \phi(t)$ such that}
\[ 
x(t)=r(t)\cdot \text{cos}\,\phi(t)\quad\colon\quad
y(t)=r(t)\cdot \text{sin}\,\phi(t)\quad\colon 0\leq t\leq T\tag{*}
\]
\emph{where the $\phi$-function satisfies
the initial condition:}
\[
x(0)=r(0)\cdot \text{cos}\,\phi(0)\quad\text{and}\quad
y(0)=r(0)\cdot \text{sin}\,\phi(0
\]



\medskip
\noindent
\emph{Proof.}
We  can solve the first order  ODE-equation.
\[
\dot\phi=
\frac{x\dot y-y\dot x}{x^2+y^2}
\]
with initial condition $\phi(0)$ as above.
There remains to show that (*) holds for all $t$.
Let us for example verify the
$x(t)=r(t)\cdot \text{cos}\,\phi(t)$.
It suffices to prove that
\[ 
\frac{d}{dt}( r(t)\cdot \text{cos}\,\phi(t))=\dot x\tag{1}
\]
To prove  this we notice that the left hand side becomes
\[
\dot r\cdot\text{cos}(\phi)-
r\cdot\text{sin}(\phi)\dot\phi=
\frac{\dot r\cdot x}{r}-y\cdot\dot\phi
\tag{2}
\]
Next, since $r=\sqrt{x^2+y^2}$ we have
\[ 
r\dot r=x\dot x+y\dot y
\]
Hence (2) is equal to
\[
\frac{x^2\dot x+xy\dot y}{r^2}-
\frac{y(x\dot y-y\dot x)}{r^2}=\frac{(x^2+y^2)\dot x}{r^2}=\dot x
\]
This proves Proposition 1.1

\bigskip

\noindent
{\bf {1.2 The angular variation.}}
Since the sine  and the cosine functions are $2\pi$-periodic,
the $\phi$-function  above is only uniquely determined up to
integer multiples of $2\pi$. A specific choice
arises from  the   value $\phi(0)$.
However, we get an
\emph{intrinsic number}  by the difference
\[ 
\phi(T)-\phi(0)\tag{**}
\] 
This number is  called the  \emph{angular variation}
of the  function
$t\mapsto (x(t),y(t))$.
If we  choose another parametrization where $t=t(\tau)$ is 
non-decreasing and $0\leq \tau\leq T^*$, then 
we start with the vector valued function $\tau\mapsto x(t(\tau),y(t(\tau))$
and find $\phi(\tau)$. Calculus shows that
(**) is the same. Thus, the  angular variation
of an oriented  parametrized $C^1$-curve is intrinsically defined.


\bigskip

\noindent
{\bf A notation.} The angular  variation along
a parametrized curve
$\gamma$ is denoted by
$\mathfrak{a}(\gamma)$.
If $\gamma$ is a curve we can construct the
curve $\gamma^*$ with the opposite direction:
\[ \gamma^*(t)=\gamma(T-t)
\]
It is clear that
\[
\mathfrak{a}(\gamma^*)=-
\mathfrak{a}(\gamma)
\]
In other words,  up to a sign the angular variation is
determined by the orientation of the curve.


\newpage


\centerline {\bf 1.3 The case of closed curves.}
\medskip

\noindent
If $x(0)=x(T)$ and $y(0)=y(T)$
the   variation is an integer multiple of $2\pi$.
So if $\gamma$ is a closed  parametrized curve then
$\mathfrak{a}(\gamma)$ is an integer multiple of $2\pi$ and we set
\[
\mathfrak{w}(\gamma)=\frac{\mathfrak{a}(\gamma)}{2\pi}
\]
We refer to $\mathfrak{w}(\gamma)$ 
as the 
\emph{winding number} of $\gamma$.
By the construction one has
\[ 
\mathfrak{w}(\gamma)=
\frac{1}{2\pi}\int_0^T\,
\frac{x\dot y-y\dot x}{x^2+y^2}\cdot dt\tag{1.1}
\]


\noindent
{\bf Example.}
Let  $m$ be a positive integer and 
\[
x(t)=\text{cos}\, mt\quad\colon\quad
y(t)=\text{sin}\, mt
\]
Notice that $x^2+y^2=1$. It follows that
\[
\frac{x\dot y-y\dot x}{x^2+y^2}\cdot dt=
\text{cos}\,mt\cdot m\cdot \text{cos}\, mt-
\text{sin}\, mt\cdot (-m\cdot\text{sin}\, mt)=m
\]
Hence the winding number is equal to $m$.






\bigskip

\noindent
{\bf 1.4 Homotopy invariance.}
Consider a family of closed curves
\[
\{\gamma_s\colon\,0\leq s\leq 1\}\quad\colon\,\gamma_s(0)=\gamma_s(T)
\,\colon\quad\,0\leq s\leq 1
\]
Let $t\mapsto (x_s(t),y_s(t))$ be the parametrization of
$\gamma_s$. 
For  each fixed $s$ the curve
$t\mapsto \gamma_s(t)$ 
has a winding number $\mathfrak{w}(\gamma_s)$.
\emph{Assume} that
the two $C^1$-functions 
depend continuously upon $s$. Then
$s\mapsto\mathfrak{w}(\gamma_s)$ is a continuous function and
since it is  integer-valued it must be a constant.
Hence we have proved
\bigskip

\noindent {\bf 1.5 Theorem} \emph{Let $\{\gamma_s\}$
be a homotopic  family of closed
curves 
Then
they have the same winding number.}
\medskip

\noindent {\bf  Remark.} In topology one refers to
this by saying that the winding number is
the same in each homotopy class of closed 
parametrized curves which
surround the origin. A parametrized curve
$\gamma$ is defined by a map
$t\mapsto \gamma(t)$
which  need not be 1-1, i.e. we only assume that
$\gamma(0)=\gamma(T)$.
One may think of an insect which takes a walk on the
horisontal $(x,y)$-plane starting at point
$p$ at time $t=0$ and  returns to $p$ after a certain time
interval $T$. During this walk the
insect may
cross an earlier path several times and even walk in the same path
but in opposed direction for a while. The sole constraint is that the insect
never attains the origin, i.e. 
$x(t)^2+y(t)^2>0$ must hold in order to construct the winding number.







\bigskip


\centerline {\bf 1.6 Rouche's principle.}
\medskip

\noindent
Let $\gamma_*$ be a  parametrized closed curve. Suppose that
$\gamma$ is another closed curve such that
\[
|\gamma_*(t)-\gamma(t)|<|\gamma_*(t)|\quad
\colon\quad 0\leq t\leq T\tag{i}
\]
To each $0\leq s\leq 1$ we  obtain the closed curve
$\gamma_s(t)=s\cdot \gamma_*+(1-s)
(\gamma_*(t)-\gamma(t))$ which by (i)
also the  surrounds the origin.
This gives  a homotopic family and   Theorem 1.5 gives:
\[ 
\mathfrak{w}(\gamma_*)=
\mathfrak{w}(\gamma)\tag{*}
\]


\noindent
{\bf The case of non-closed curves.}
Let $p$ and $q$ be two points outside the origin.
Consider two curves
$\gamma_1$ and $\gamma_2$ where $p$ is the common initial point and $q$ the common end-point.
Now we get the closed curve
$\rho$ defined by
\[ 
\rho(s)=\gamma_1(2s)\quad\colon 0\leq s\leq T/2\text{and}\quad \rho(s)=
\gamma_2(2T-2s)
\quad\colon T/2\leq s\leq T
\]
Here we find that
\[
\mathfrak{w}(\rho)=
\mathfrak{a}(\gamma_1)-
\mathfrak{a}(\gamma_2)
\]
Next,
keeping $p$ and $q$ fixed we consider a continuous family of curves
$\{\gamma_s\}$ where
$\gamma_s(0)=p$ and $\gamma_s(T)=q$ for all $0\leq s\leq 1$.
To each $s$ we get the two curves
$\gamma_0$ and $\gamma_s$
and construct the closed curve
$\rho$ as above. Theorem 1.5  implies 
that the difference
\[
\mathfrak{a}(\gamma_0)-
\mathfrak{a}(\gamma_s)
\] 
is a constant function of $s$.
Since the difference obviously is zero when $s=0$
we conclude that
\[
\mathfrak{a}(\gamma_0)=
\mathfrak{a}(\gamma_s)\quad\colon\, 0\leq s\leq 1
\]
Thus, the angular variation is constant in a homotopic family of curves
which join a pair of
points $p$ and $q$.



\bigskip

\noindent
{\bf 1.7 Variation of vector-valued functions}
Let $\gamma$ be a  parametrized $C^1$-curve. Here we do not exclude
that
$\gamma(t)=(0,0)$ for some values of $t$, i.e. $\gamma$ is an arbitrary
$C^1$-curve.
Consider a pair of $C^1$-functions $u(x,y)$ and
$v(x,y)$ defined in some neighborhood of the compact
image set $\Gamma=\gamma([0,T])$.
If $u^2+v^2\neq0$ on $\Gamma$
we get a  curve $\gamma^*$ which surrounds the
origin
defined by
\[ 
t\mapsto (u(\gamma(t)),v(\gamma(t))\tag{i}
\]
Write $\gamma(t)=(x(t),y(t))$ and set
\[ 
\xi(t)=u(x(t),y(t))\quad\colon\quad
\eta(t)=v(x(t),y(t))
\]
Then we have
\[ 
\mathfrak{a}(\gamma^*)=
\int_0^T\,\frac{\xi\dot\eta-\eta\dot\xi}{\xi^2+\eta^2}\cdot dt\tag{ii}
\]
Now $\dot\xi= u_x\ddot x+u_y\dot y$ and similarly for
$\dot\eta$.
So the last integral becomes
\[
\int_0^T\,\frac{u(v_x\dot x+v_y\dot y)-v(u_x\dot x+u_y\dot y)}
{u^2+v^2}\cdot dt \tag{*}
\]
This yields an integer which we refer to as the
variation 
of the vector valued function $(u,v)$ along the closed curve
$\gamma$.
We denote this integer by a subscript notation and write
$\mathfrak{a}_{(u,v)}(\gamma)$. In the case when
$\gamma$ is a  closed curve we define the winding number
\[
\mathfrak{w}_{(u,v)}(\gamma)=
\frac{1}{2\pi}\cdot \mathfrak{a}_{(u,v)}(\gamma)
\]
Notice that this integer depends upon the pair $(u,v)$ while $\gamma$ is kept fixed. 



\medskip


\noindent{\bf 1.8 The case of CR-pairs}
Let $\gamma$ be a  curve
and $f(z)=u+iv$  an  analytic in a neighborhood of
$\gamma(T)$ where we assume that
$f(\gamma(t))\neq 0$ for all $t$.
Hence $u^2+v^2\neq 0$ on $\gamma$
so we can define
$\mathfrak{a}_{(u,v)}(\gamma)$.
Now $(u,v)$ satisfy the Cauchy-Riemann equations
which enables us to express  
$\mathfrak{a}_{(u,v)}(\gamma)$ 
in an elegant way.
Namely let $t\mapsto 
(x(t),y(t))$ be a parametrization of $\gamma$ and write
$z(t)=x(t)+iy(t))$. Then
\[ 
\dot z=\dot x+i\dot y
\]
Now we regard the function
\[
t\mapsto \mathfrak{Im}\,[\frac{f'(z(t))}{f(z(t))}\cdot\dot z(t)\,]\tag{i}
\]
Since the complex derivative
$f'(z)=u_x+iv_x$ we obtain
\[
\frac{f'(z(t))}{f(z(t))}\cdot\dot z(t)=
\frac{(u_x+iv_x)(u-iv)(\dot x+i\cdot \dot y)}{u^2+v^2}\tag{ii}
\]
The imaginary part becomes


\[
\frac{u_x u\dot y-u_xv\dot x+v_x u\dot x+v_xv\dot y}{u^2+v^2}=
\frac{u(u_x\dot y+v_x\dot x)-v(u_x\dot x-v_x\dot y)}{u^2+v^2}\tag{iii}
\]


\noindent
Next, we can apply the Cauchy-Riemann equations and replace
$u_x$ with $v_y$ and $-v_x$ by $u_y$. Then
we see that (iii) is equal to the integrand which appears
in (*) in 1.7.
Hence we have proved the following:.
\medskip

\noindent {\bf 1.9 Theorem} \emph{Let $f(z)=u+iv$ be holomorphic in a neighborhood of
$\gamma$ and set $\mathfrak{a}_f(\gamma)=\mathfrak{a}_{(u,v)}(\gamma)$.
Then}
\[ 
\mathfrak{a}_f(\gamma)=
\int_0^T\,
\mathfrak{Im}\,\bigl [\frac{f'(z(t))}{f(z(t))}\cdot\dot z(t)\,\bigr ]\cdot dt\tag{*}
\]
\medskip

\noindent
{\bf 1.10 Remark} By the construction of complex line integrals, 
(*) 
can  be written
as
\[
\frac{1}{i}\cdot\int_\gamma
\mathfrak{Im}\,[\frac{f'(z)}{f(z)}\,]\cdot dz
\]
This complex notation is often used.
When
$\gamma$ is a closed curve we get the winding number
\[
\mathfrak{w}_f(\gamma)=
\frac{1}{2\pi i}\cdot\int_\gamma
\mathfrak{Im}\,[\frac{f'(z)}{f(z)}\,]\cdot dz
\]
So this complex line integral always is an integer whenever
$f(z)$ is analytic and $\neq 0$ in some open neighborhood of
the compact set
$\gamma([0,T])$.






\medskip

\noindent
{\bf 1.11 Jordan's curve theorem  }
Let $\gamma$ be 
a closed $C^1$-curve and set $\Gamma=\gamma([0,T])$.
To each $a\in {\bf{ C}}\setminus \Gamma $
the closed curve 
\[
t\mapsto \frac{1}{\gamma(t)-a}
\]
surrounds the origin. Its   winding number
denoted by $\mathfrak{w}_a(\gamma)$.
From (*) in 1.4 we see
that
this winding number is constant in every connected
component if $ {\bf {C}}\setminus \Gamma$.

\bigskip


\noindent
{\bf 1.12 The case when $\gamma$ is 1-1}
Assume that $\gamma(t)$ is 1-1  except for the 
common end-values. This means that the image curve
$t\mapsto \gamma(t)$ is a 
\emph{closed Jordan curve}.
For each $a\in{\bf{ C}}\setminus\Gamma$ 
we notice that $t\mapsto \gamma(t)-a$
is  1-1.
In  the equation from XX which
determines the $\phi$-function for a given $a$
where we may take $\phi(0)=0$
as initial value shows that $t\mapsto\gamma(t)$ is 1-1 on the open interval $(0,T)$.
Hence $\phi(t)$ cannot be an integer multiple of $2\pi$ when
$0<t<T$.
Starting with $\phi(0)=0$ it  follows that
\[ -2\pi<\phi(t)<2\pi\quad\colon\quad
0<t<2\pi
\]
Hence $\phi(T)$ can only attain one of the values $-2\pi,0,2\pi$.
The \emph{Jordan curve theorem}  tells us that the value zero is never
attained. Moreover, 
the set of points $a$ for which the winding number equals 1 is a conneced
open set, called the Jordan domain bounded by $\Gamma$. The complementary set is also
connected and here $\mathfrak{w}_a(\gamma)=0$.
This can be expressed by saying
that the closed Jordan curve $\Gamma$ divides $\bf C$ into two component.
This topological  result was proved by   
Camille Jordan in 1850 and it is 
is actually  valid under the relaxed assumption that the $\gamma $-function is only continuous.
In that case the proof of Jordan's Curve Theorem
is  more demanding.
For a detailed proof 
of the continuous version of Jordan's Curve 
Theorem we refer  [Newmann] where methods of algebraic topology are used.
We remark that Jordan's theorem in the plane is subtle in view of a quite 
remarkable  discovery in dimension 3 due to 
X. Alexander who constructed 
a \emph{homeomorphic copy} of the unit sphere in $R^3$ where the analogue of Jordan's theorem is not valid. This goes beyond
the scope of these notes.
A recommended text-book in
algebraic topology is Alexander's classic text-book [Al] 
which gives an excellent introduction to the  subtle parts of the theory.
\medskip


\noindent
{\bf 1.13 The case of a simple polygon.}
Let $p_1,\ldots,p_N$ be distinct points in $\bf C$
where $N\geq 3$.
To each 
$1\leq\nu\leq N-1$ we get a line segment
$\ell_\nu=[p_\nu,p_{\nu+1}]$ and we also get the line segment 
$\ell_N=[p_N,p_1]$.
Assume that they do not intersect.Then
they give sides of a simple closed curve $\Gamma$ whose corner points are
$p_1,\ldots,p_N$. The circle $\Gamma$ is oriented where one travels in 
the positive direction from $p_\nu$ to $p_{\nu+1}$ 
when $1\leq\nu\leq N-1$ and makes the final positive travel from
$p_N$ to $p_1$.
We can imagine a narrow channel $\mathcal C_+$ which 
surrounds $\Gamma$ and from this one can
"escape" to the point at infinity. For example at a corner point
$p_\nu$ where $|p_\nu|$ is maximal the channel contains
points of absolute value $>1$.
From this picture it is clear the the \emph{outer component}
$\Omega_\infty$
of
$\Gamma$ is connected - and even simply connected if one
adds the point at infinity. Rouche's principle shows
that the winding number is zero for
all points in the exterior component.
If we instead construct a narrow channel $\mathcal C_*$
which moves "just inside" $\Gamma$ 
then the channel itself is obviously connected.
But their remains to see why the whole interior is connected and that
the common winding number is  equal to one.
This, if $\Omega_*$ is the open complement of
$\Gamma\cup\Omega_\infty$ we must first prove that
$\Omega_*$ is connected. Since the narrow channel $\mathcal C_*$
is connected it suffices to show that when $p\in\Omega_*$ then
there exists  some curve $\gamma$
from $p$ which reaches $\mathcal C_*$.
To obtain $\gamma$ we consider a point $p^*\in\Gamma$
such that $|p-p^*|$ is the distance of $p$ to $\Gamma$, i.e.
we pick a point nearest to $p$.
Now we draw the straight line $L$ through $p$ and $p^*$ and by a picture
the reader discovers that if we travel along $L$ from
$p$ towards $p^*$ then we  reach $\mathcal C_*$ prior to the
arrival at $p^*$.
This proves that $\Omega_*$ is connected. The proof that the
common winding number for points in $\Omega_*$ is equal to one
is left as an \emph{exercise} to the reader.



\newpage




\centerline{\bf\large  2. The argument principle}
\bigskip

\noindent
Let $\Omega\in \mathcal D(C^1)$ and $f(z)$ is an analytic function
in $\Omega$ which extends to a $C^1$-function on its closure.
Denote by $\mathcal N_\Omega(f)$ the number of zeros of $f$ in
$\Omega$. We also assume that $f\neq 0$ on $\partial\Omega$.
\medskip


\noindent {\bf 2.1 Theorem.} \emph{Let $\Omega\in\mathcal D(C^1)$
and let $\Gamma_1,\ldots,\Gamma_k$
be its simple and closed boundary curves.
Then}
\[
N_\Omega(f)=\sum_{\nu=1}^{\nu=k}\,\mathfrak{w}_f(\Gamma_\nu)
\]



\noindent 
\emph{Proof.}
By Theorem III.XX  we have
\[
N_\Omega(f)=\sum\,\frac{1}{2\pi i}\cdot
\int_{\Gamma_\nu}\, \frac{f'(z)dz}{f(z)}
\]
Since $N_\Omega(f)$ is an integer and hence a real nmber
this gives
\[
N_\Omega(f)=\sum\,\frac{1}{2\pi }\cdot
\int_{\Gamma_\nu}\, \mathfrak{Im}\,[\frac{f'(z)dz}{f(z)}\,]
\]
Theorem 1.9 expressed with  complex notations
shows each
term of the sum above is equal to
$\mathfrak{w}_f(\Gamma_\nu)$ and Theorem 2.1 follows.




\medskip



\noindent 
{\bf 2.2 Rouche's Theorem.}
\emph{Let $\Omega$ and $f$ be as above and let $g$ be another
holomorphic function in $\Omega$ which extends to be $C^1$�on the closure.
Then, if $|g|<|f|$ holds on $\partial\Omega$, it follows that}
\[
\mathcal N_{f+g}(\Omega)=\mathcal N_f(\Omega)
\]


\noindent 
\emph{Proof.} Apply the result in 1.6.


\bigskip


\noindent
{\bf 2.3 An application to trigonometric series}.
Let $1\leq m<n$ be a pair of positive integers. Consider a trigonometric polynomial
\[ 
P(\theta)=
\sum_{\nu=m}^{\nu=n}\,
a_\nu\text{cos}(\nu\theta)+
b_\nu\text{sin}(\nu\theta)\quad\colon\quad a_\nu,b_\nu\in\bf R
\]
We assume that at least one of the coefficients $a_m$ or $b_m$ is $\neq 0$, and similarly at least one of the numbers  $a_n$ or $b_n$ is $\neq 0$.
Then one has 
\medskip

\noindent {\bf 2.4 Theorem} 
\emph{$P$ has at least $2m$ zeros on $[0,2\pi]$ and at most $2n$ zeros.}
\medskip

\noindent
\emph{Proof}
Consider the polynomial 
\[
Q(z)=(a_m-ib-m)z^m+\ldots+(a_n-ib_n)z^n
\]
Notice that $\mathfrak{Re}(Q(e^{i\theta})=P(\theta)$.
The polynomial $Q$ has a zero of multiplicity $m$ at the origin.
Let us now consider some $r<1$ with $r\simeq 1$ and chosen so that
$Q\neq 0$ on the circle $T_r=\{|z|=r\}$.
Since $Q(z)$ has at least $m$ zeros counted with
multiplicity in the disc $D_r$, it follows from
Theorem 2.2that
\[
\mathfrak{w}_{T_r}(Q)\geq m
\]
Regarding a picture the reader discovers that  the curve
$\theta\mapsto Q(re^{i\theta})$ must intersect the real line at least $2m$
times.
This proves the lower bound. The upper bound $2n$ is easily
proved by elementary Calculus  and  left to the reader.

\bigskip

\noindent
{\bf 2.5 A special estimate.}
Theorem 2.1 can be used to give upper bounds for the counting function
$\mathcal N_\Omega(f)$.
Suppose  that $\Omega$ is a rectangle
\[ \{z=x+iy\colon\, a<x<b\colon\, 0<y<T\}
\]
Here $\partial\Omega$ contains the vertical line
$\ell=\{x=b\,\colon\,0<y<T\}$.
The line integral along $\ell$ contributes to the evaluation of
$\mathcal N_\Omega(f)$ by
\[
\frac{1}{2\pi}\cdot
\int_\ell\,\mathfrak{Im}\,\bigl(\frac{f'(z)dz}{f(z)}\,\bigr)
\]
\bigskip

\noindent
Now $dz=idy$ along $\ell$ and therefore
the integral above is equal to
\[
\frac{1}{2\pi}\cdot\int_0^T\,
\mathfrak{Re}[\frac{f'(b+iy)}{f(b+iy)}]\, dy
\]
\medskip

\noindent Assume that $\mathfrak{Re}\, f(b+iy)\geq c_0>0$
for all $0\leq y\leq T$ which gives
a single valued branch of the Log-function, i.e.
\[ 
\log\, f(b+iy)=
\log\, |f(b+iy)|+
i\cdot \text{arg}(f(b+iy))\,\,\colon\,
-\pi/2<\text{arg}(f(b+iy))<\pi/2
\]
Since $f'(z)=\frac{1}{i}\cdot\partial_y(f)$ it follows that
\[
\frac{f'(b+iy)}{fb+iy)}=
\frac{1}{i}\cdot [\partial_y(\log\, |f(b+iy)|)+
i\cdot\partial_y(\text{arg}(f(b+iy)))]
\]
Hence we obtain
\medskip

\noindent {\bf 2.6 Proposition.} \emph{One has the equality}
\[
\mathfrak{Re}\,\frac{f'(b+iy)}{f(b+iy)}=
\partial_y(\text{arg}(f(b+iy)))
\]


\noindent
{\bf 2.7 Remark.} Proposition 2.6  gives therefore
\[
\frac{1}{2\pi}\cdot
\int_\ell\,\mathfrak{Im}\,\bigl(\frac{f'(z)dz}{f(z)}\,\bigr)=
\frac{1}{2\pi}\cdot
\text{arg}(f(b+iT)))-\text{arg}(f(b))\tag{*}
\]
The right hand side is a real number in $(-1/4,1/4)$ and 
hence we get a small contribution from the line integral in the left
hand side when we regard  whole line integral over $\partial\Omega$
which
evaluates $\mathcal N_f(\Omega)$. This will be used to study the
zeros of Riemann's $\zeta$-function.
\bigskip


\noindent
{\bf{2.8 An application.}}
Let $m\geq 2$ and $g_2(z),\ldots,g_m(z)$
are analytic functions defined in an open disc
$D$ centered at $z=0$ and where $g_\nu(0)=0$ for every
$\nu$. Let also $\phi(z)$ be another analytic function in
$D$ with $\phi(0)=0$.
Consider the algebraic equation
\[ 
y+g_2(z)y^2+\ldots+g_m(z)y^m=\phi(z)\tag{*}
\]
Thus, we seek $y(z)$ so that (*) holds.
It turns out that there exists a unique analytic function
$y(z)$ defined in some open disc $D_*$ centered at $z=0$
where $y(0)=0$ and
(*) hold for every $z\in D_*$. 
To attain this we set
\[
P(y,z)=
y+g_2(z)y^2+\ldots+g_m(z)y^m
\]
Now we can find
$\delta>0$
such that if $z\in D(\delta)$ then
\[
|\phi(z)|<|P(e^{i\theta},z)|
\quad\text{for all}\quad 0\leq\theta\leq 2\pi\tag{i}
\]
Next, let us put
\[
P'_y(y,z)= 1+2g_2(z)y+\ldots+mg_m(z)y^{m-1}
\]
From (i)  there exists the integral
\[
\frac{1}{2\pi i}\cdot \int_{|y|=1}\,
\frac{P'_y(y,z)}{P(y,z)-\phi(z)}\cdot dy\quad
\colon\, z\in D(\delta)\tag{1}
\]
Rouche's Theorem shows that this integer-valued function
is constant as $z$ varies in $D(\delta)$.
When $z=0$ we see tha the integrand is $\frac{1}{y}$ and hence
the constant integer is one.
But this means precisely that when $z\in D(\delta)$ is fixed, then
the analytic function
\[ 
y\mapsto P(y,z)-\phi(z)
\] 
has exactly one simple zero in $|y|<1$.
Denote this zero by $y(z)$.
By the residue formula we get
\[ 
y(z)=\frac{1}{2\pi i}\cdot \int_{|y|=1}\,
\frac{y\cdot P'_y(y,z)}{P(y,z)-\phi(z)}\cdot dy\quad
\tag{2}
\]
It is clear that $y(z)$ is analytic in $D(\delta)$ and by the construction
$P(y(z),z)=0$.
Thus, $y(z)$ is the required solution. The proof of its uniqueness is left
as an exercise.













\bigskip









\newpage



\centerline{\bf\large  3. Multi-valued functions}
\bigskip

\noindent
Let $\Omega$ be an open connected subset of ${\bf{ C}}$
and 
$D\subset\Omega$ is an open disc of some radius $r$ centered at a point $z_0$.
The material about  power series in Chapter XX
shows that
$\mathcal O(D)$ is identified with  convergent power series
\[
\sum\,c_\nu(z-z_0)^\nu\quad\colon\quad
\text{radius of convergenece}\,\,\geq r
\]
So if $f\in\mathcal O(\Omega)$ its restriction to any disc $D\subset\Omega$
determines a convergent power series.
These power series must be matching when two discs have a
non-empty intersection.
This  observation is the starting point for a general construction
due to
Weierstrass.

\medskip

\noindent
{\bf 3.1 Analytic continuation along paths}
Let $s\mapsto \gamma(s)$ be a continuous and
complex valued function with values in
$\Omega$. We do not require that $\gamma(0)=\gamma(1)$
or that $\gamma$ is 1-1.
The points $p=\gamma(0)$ and $q=\gamma(1)$ are called the 
terminal points of $\gamma$. Let $f_0\in\mathcal O( D_r(p))$
for some $r>0$, i.e. $f$ is analytic in a small disc centered at $p$.
Consider a
strictly increasing sequence
$0=s_0<s_1<\ldots<s_N=1$ and to each $p_\nu=\gamma(s_\nu)$
we choose a small  disc $D_{p_\nu}(r_\nu)$ such that:
\[
D_{p_\nu}(r_\nu)\cap
D_{p_{\nu+1}}(r_{\nu+1})\neq\emptyset\quad\,\quad 0\leq\nu\leq N-1
\]
Assume that
for  each $1\leq\nu\leq N$
exists  $f_\nu\in\mathcal O( D_{r_\nu}(p_\nu)) $ such that
\[
f_\nu=f_{\nu+1}\,\,\text{holds in}\,\,\,
D_{p_\nu}(r_\nu)\cap
D_{p_{\nu+1}}(r_{\nu+1})
\]
After $N$ many   \emph{direct analytic continuations over  pairs of intersecting discs}
we arrive at
$f_N$ which is analytic in an open disc centered at $\gamma(1)$.
From the  uniqeness of each direct analytic continuation, it follows
that
the locally defined analytic function $f_N$
at $\gamma(1)$ is the same  if we instead have chosen
a \emph{refined} partition  of $[0,1]$.
Since two coverings of $\gamma$ via  finite families of discs
have a common refinement, we conclude
that  locally defined analytic function at the end-point is unique.
Thus, the  construction   yields a map
$T_\gamma$ map which sends an analytic function
$f$ defined in a disc around $\gamma(0)$ to an analytic
function $T_\gamma(f)$ defined is some disc centered at  $\gamma(1)$. Of course, here
$T_\gamma$ is only defined on those $f$ at $\gamma(0)$ which have an analytic continuation along $\gamma$ in the sense of Weierstrass.


\bigskip

\noindent
{\bf 3.2 Example}
Let $\gamma$ be the closed unit circle centered at the origin where
$\gamma(0)=\gamma(1)$.
At $\gamma(0)$ we start with a chosen branch of the complex
Log-function with  the   power series representation
\[ 
f_0(1+z)=z-z^2/2+z^3/3+\ldots
\]
This power series converges when $|z|=1$
So let the first disc $D_1$ be centered at $z=1$ and of radius 1.
Let $D_2$ be the disc centered at $i$ of radius $i$.
In this disc we have
the analytic function $f_1(z)$ given by the power series
\[ 
f_1(i+z)=i\frac{\pi}{2}+\sum_{n=1}^\infty\,
(-1)^{n+1}\cdot i^{-n}\cdot z^n/n
\]
Drawing a figure the two discs intersect and in particular $D_1\cap D_2$ contains an 
open arc
of the unit circle around $e^{i\pi/4}$.
With $\theta$ close to $\pi/4$ we get the two series  expansion
\[
f_0(e^{i\theta})=
\sum_{n=1}^{\infty}
(-1)^{n+1}\cdot (e^{i\theta}-1)^n\quad\colon\quad
f_1(e^{i\theta})=
i\frac{\pi}{2}+\sum_{n=1}^{\infty}\,(-1)^{n+1}
\cdot i^{-n}\cdot(e^{i\theta}-i)^n /n
\]
\medskip

\noindent
\emph{Exercise.} Verify that
the two series expansions give the same $\theta$-function, i.e. that one
has the equality
\bigskip
\[
\sum_{n=1}^{\infty}
(-1)^{n+1}\cdot (e^{i\theta}-1)^n=
i\frac{\pi}{2}+\sum_{n=1}^{\infty}\,(-1)^{n+1}
\cdot i^{-n}\cdot(e^{i\theta}-i)^n /n\quad
\colon \quad  |\theta-\frac{\pi}{4}|<\delta
\]
where $\delta>0$ is chosen so that $|e^{i\pi/4}-1|<1$.
This series expansion starts the analytic continuation of the complex Log-function along
the closed unit circle which after one turn around the origin gives
the
new local branch $f_0(z)+2\pi i$.



\bigskip




\centerline {\bf 3.3 The class $M \mathcal O(\Omega)$.}
\bigskip

\noindent
Let $\Omega$ be an open subset of $\bf C$. At each point
$z\in\Omega$ we denote by $\mathcal O(z_0)$
the germs of analytic functions at $z_0$ and recall that
this set is identified with  power series
$\sum\,c_\nu(z-z_0)^\nu$ which have some positive radius of
convergence. 
In  $\mathcal O(z_0)$ we can consider those germs which have
analytic continuation along \emph{every}
curve in $\Omega$ whose initial point is $z_0$ while the end-point is arbitrary.
This leads to:

\bigskip

\noindent
{\bf 3.4 Definition}
\emph{A germ $f\in\mathcal O(z_0)$ generates a multi-valued analytic function
in $\Omega$ if it can be extended in the sense of Weierstrass along every
curve $\gamma\subset\Omega$ which has $z_0$ as initial point.
The set of all these germs is denoted by
$M\mathcal O(\Omega)(z_0)$}.


\bigskip

\noindent{\bf Remark.} Notice that
$M\mathcal O(\Omega)(z_0)$ contains those germs at $z_0$ which
are induced by \emph{single-valued} analytic functions in
$\Omega$.
If $f\in M\mathcal O(\Omega)(z_0)$ and $\gamma$ is a curve in 
$\Omega$ with $z_0$ as initial point and $z_1$ as end-point, then
the germ $T_\gamma(f)$ at $z_1$ belongs to
$M\mathcal O(\Omega)(z_1)$.
This is obviuos since if $\gamma_1$
is a curve starting at $z_1$ with end-point at $z_2$, then $f$ extends along the composed curve $\gamma_1\circ \gamma_1$
and one has the composition formula:
\[ 
T_{\gamma_1}(T_\gamma(f))= T_{\gamma\circ\gamma}(f)\tag{*}
\]
\medskip

\noindent
{\bf 3.5 The total sheaf space $\mathcal {\widehat O}$.}
Let
us 
introduce a big topological space
$\mathcal {\widehat O}$ defined as follows: One has a map
$\rho$ from $\mathcal {\widehat O}$ onto $\bf C$. The inverse fiber
$\rho^{-1}(z)=\mathcal O(z)$  for each $z\in\bf C$.
An open neighborhood of a "point"  $f\in\rho^{-1}(z_0)$
consists of 
a pair $(f,D)$ where $D$ is a small disc centered
at $z_0$ such that the germ $f$ extends to an analytic function in $D$. 
Then its induced germ at a point $z\in D$ belongs to
$\rho^{-1}(z)$. The set of points in
$\mathcal {\widehat O}$ obtained in this way yields the subset $(f,D)$
and as $D$ shrinks to $z_0$ they give by definition a fundamental system of open
neighborhoods of the point $f$ in $\mathcal {\widehat O}$.
With this topology on $\mathcal{\widehat O}$ 
the map $\rho$ is a \emph{local homeomorphism} 
and each inverse fiber $\rho^{-1}(z_0)$ appears as a 
\emph{discrete} subset of
$\mathcal {\widehat O}$.
\medskip

\noindent
{\bf Remark} Above $\mathcal{\widehat O}$ is the first example of a sheaf
which  has
led
to the general construction of sheaves
which  is presented in elementary text on  topology.
The construction of the sheaf topology 
on $\mathcal{\widehat O}$ yields the following elegant
description of multi-valued functions.
\bigskip

\noindent
{\bf 3.6 Proposition.} \emph{Let $\Omega$ be an open and connected
subset of $\bf C$.
Let $z_0\in\Omega$ and  $f\in M\mathcal(\Omega)(z_0)$.
Then  $f$ appears in the inverse fiber $\rho^{-1}(z_0)$
of  an open and connected
set $\mathcal W(f)$ of $\rho^{-1}(\Omega)$ called 
Weierstrass Analytische Gebilde of the germ of this multi-valued
funtion. For each $z\in \Omega$ the set
$\mathcal W(f)\cap\rho^{-1}(z)$ consists of all germs at $z$ obtained by
analytic continuation of $f$ along some curve with end-point at $z$.}
\bigskip

\noindent
{\bf{Some notations.}}
Let $f$ be as above.
If $z\in\Omega$ we denote by $W(f:z)$ the set of germs at $z$ which arise via 
all analytic continuations  of $f$. Thus, $W(f:z)$ is equal to
$\mathcal W(f)\cap\rho^{-1}(z)$. In addition to this we can consider the
set of values at $z$ which are attained by these germs.
So we have also the set
\[ 
R_f(z)= \{T_\gamma(f)(z)\quad\colon\,\, T_\gamma(f)\in W(f:z)\}
\]
\medskip




\noindent
{\bf Example}
Let $\Omega=\bf C$ minus the origin, i.e. the punctured complex plane.
Then we have the multi-valued Log-function. At each point
$z\in\Omega$ it has an infinite set of local branches which differ by integer
multiples of $2\pi i$. The resulting connected set
$\mathcal W(\text{Log}(z))$ can be regarded as a 2-dimensional connected manifold.
In topology one learns that this is the \emph{universal covering space} of 
$\Omega$, so that
$\mathcal W(\text{Log}(z))$  is a  \emph{simply connected} manifold.
On the other hand, if $N\geq 2$ is an integer we have the multi-valued function 
$z^{\frac{1}{N}}$ Here $\mathcal W(z^{\frac{1}{N}})$ is an
$N$-fold unramified covering map of $\Omega$, i.e. the $\rho$-map is a local 
homeomorphism whose inverse fibers contain $N$ points.
\newpage


\centerline {\bf 3.7 Normal families.}
\medskip

\noindent
Let $\Omega$ be some connected open set in
${\bf{C}}$.  Let $x_0\in \Omega$ and consider
some germ $f\in M\mathcal O(\Omega)(x_0)$.
We say that $f$ yields a  bounded multi-valued function if there exists a constant
$K$ such that
\[
\bigl|T_\gamma(f)(x)\bigr |\leq K\tag{*}
\] 
holds for all pairs $x,\gamma$ where $x\in\Omega$ and
$\gamma$ is any curve from $x_0$ to $x$.
Suppose that
$\{f_\nu\}$ is a sequence of germs
in $M\mathcal O(\Omega)(x_0)$ which are uniformly bounded, i.e.
(*) above holds for some constant $K$ and every $\nu$.
If we to begin with consider a small open disc
$D$ centered at $x_0$ we get the unique single-valued branches of
each $f_\nu$ in
$\mathcal O(D)$. This family in
$\mathcal O(D)$ is normal by the results in
XXX. Passing to a subsequence we may assume that
there exists a limit function
$g\in\mathcal O(D)$, i.e. shrinking $D$ if necessary we may assume that
\[
\lim_{\nu\to\infty}\, ||f_\nu-g||_D\to 0\tag{i}
\]
Next, if $\gamma$ is a curve in
which starts at $x_0$ and has some end-point $x$
we cover $\gamma$ with a finite number of open discs and
each $f_\nu$ has its analytic continuation along $\gamma$
by the Weierstrass procedure.
From the material in XXX it is clear that
during these analytic continuations the local series expansions of
the sequence $\{f_\nu\}$ converge uniformly
and as a result we find that
$g$ has an analytic extension along
$\gamma$. Hence the germ of $g$ at $x_0$
belongs to $M\mathcal O(\Omega)(x_0)$.
Moreover, the uniform convergence "propagates".
For example, if $\gamma$ is a closed curve at
$x_0$ we get the sequence
$\{T_\gamma(f_\nu)\}$
after the analytic continuation along $\gamma$, and similarly
$T_\gamma(g)$. Then 
\[
\lim_{\nu\to\infty}\, ||T_\gamma(f_\nu)-T_\gamma(g)||_D\to 0\tag{ii}
\]
holds for a small disc $D$ centered at the end point of $\gamma$.






\bigskip


\centerline {\bf 3.8 Algebraic root functions}
\medskip

\noindent
Consider a polynomial $P(z,w)$ in two variables:
\[ 
P(z,w)= p_m(z)w^m+
p_{m-1}(z)w^{m-1}+\ldots + p_1(z)w+p_0(w)
\] 
Here $m\geq 2$ is assumed. 
The zeros of leading polynomial $p_m(z)$ is a finite subset of $\bf C$ 
which we denote by 
$\mathcal Z_P$.
If $z$ is outside this zero set we have an algebraic equation
$P(z_0,w)$ which by the Fundamental Theorem of algebra has $m$ roots
denoted by $\alpha_1(z),\ldots,\alpha_m(z)$.
By the  Newton formula from XXX in Chapter I
the
symmetric product

\[ 
D(z)=\prod_{i\neq\nu}\,
(\alpha_i(z)-\alpha_\nu(z)
\]
is a rational function of $z$, unless it happens to be identically zero. We shall 
exclude this case
and 
remark that elementary algebra shows that
the $D$-function is not
identically  zero if and only if
the polynomial $P(z,w)$ has no multiple factors 
in the unique factorisation domain $C[z,w]$. One
refers to $D(z)$ as the \emph{discrimant} of $P$.
Put

\[
\Sigma_P=\mathcal  Z _P\cup\, D^{-1}(0)
\]
This is a finite set in $\bf C$. If $\Omega$ is its open  complement
we get the multi-valued root functions.
Assuming that $P(z,w)$ from the start is an irreducible polynomial in
$C[z,w]$ these root functions are local branches of a
multi-valued function $f$ in $\Omega$.
The fibers of $W(f)$ above a point $z_0\in\Omega$
consists of the $k$-tuple of germs induced by
the $m$-many distinct $\alpha$-roots at $z_0$.
The \emph{connectivity} 
of $W(f)$ means precisely
that the polynomial $P(z,w)$ is irreducible.
\medskip

\noindent 
{\bf Remark} The realisation of roots to $P(z,w)$ as multi-valued
analytic functions is  classic.
Historically it led to the construction of the 
\emph{closed Riemann surface}
attached to the polynomial $P(z,w)$. We return to this in the
chapter about compact Riemann surfaces.
\newpage

\centerline
{\bf 4. The Monodromy Theorem}
\medskip

\noindent
Let $f\in M\mathcal O(\Omega)$.
If $z_0\in\Omega$ and $\gamma$ is a curve starting at $z_0$ we obtain 
the germ $T_\gamma(f)$ at the end-point $z_1$ of $\gamma$.
The analytic continuation is obtained by the Weierstrass procedure and
since $\gamma$ is a compact subset of $\Omega$
it can be covered by a finite set of discs $D_0,D_1,\ldots,D_N$ where 
$D_\nu\cap\,D_{\nu+1}$ are non-empty and the analytic continuation of 
$f$ is achieved by succesive direct continuations of analytic functions
$\{g_\nu\in\mathcal O(D_\nu)\}$
where $g_\nu=g_{\nu+1}$ holds in
$D_\nu\cap\,D_{\nu+1}$.
The discs are  chosen 
so small that they are 
relatively compact in $\Omega$.
If $\gamma_1$  is another curve from $z_0$ to $z_1$ which stays
so close to $\gamma$ that the discs $D_0,\ldots,D_N$
again can be used to perform the analytic continuation of $f$ along
$\gamma_1$, then it is clear that  $T_\gamma(f)=T_{\gamma_1}$.
This observation gives:
\medskip

\noindent {\bf 4.1 Theorem} \emph{Let $(z_0,z_1)$ be a pair in $\Omega$ and
$\Gamma(s,t)$ 
a continuous map from the unit square in the $(s,t)$-space into $\Omega$ where}
\[ 
\Gamma(s,0)=z_0\quad\colon,\gamma(s,1)=z_1\quad\colon\quad
0\leq s\leq 1 
\]
\emph{Then, if  $\{\gamma_s\}$ is the family of curves defined by
 $t\mapsto \Gamma(s,t)$, it follows that}
 \[ 
 T_{\gamma_s}(f)=T_{\gamma_0}(f)\quad\colon\,\quad 0\leq s\leq 1
 \]
 \medskip
 
 \noindent {\bf Remark} This is called the monodromy theorem and can be
 expressed by saying that  analytic continuation along a curve 
 which joins  a given pair of points only  depends on the
 \emph{homotopy} class of the curve, taken in the family of all
 curves which joint
 the two given points.
Of course, when we deal with some multi-valued function in an open set
$\Omega$ we are obliged to use curves inside $\Omega$ only.

\bigskip

\noindent
{\bf 4.2 The case of finite determination.}
Let $f\in M\mathcal(\Omega)$.
If $z_0\in\Omega$ we get the set of germs   $W(f:z_0)$ at $z_0$.
This is a subset of $\mathcal O(z_0)$ and we can regard the complex vector space
it generates. It is denoted by $\mathcal H_f(z_0)$.
Suppose that this complex vector space has a finite dimension $k$.
Then we can choose  a $k$-tuple of germs $g_1,\ldots,g_k$
in $W(f:z_0)$  which give a basis of
$\mathcal H_f(z_0)$. Thus, one has to begin with

\[ 
\mathcal H_f(z_0)=Cg_1+\ldots+Cg_k
\]
Let $z_1$ be another point in $\Omega$ and fix 
some curve $\gamma$ which joins $z_0$ and $z_1$.
At $z_1$ we get the germs $T_\gamma(g_1),\ldots,T_\gamma(g_k)$.
By the remark in XXX
$T_\gamma$ is a \emph{bijective map} from $W(f:z_0)$ 
to $W(f:z_1)$.
Moreover, if  $\phi=c_1g_1+\ldots+c_kg_k$ belongs to $\mathcal H_f(z_0)$ we have

\[
T_\gamma(\phi)=c_1T_\gamma(g_1)+\ldots+c_kT_\gamma(g_k)
\]
Hence the $k$-tuple $\{T_\gamma(g_\nu)\}$ generates the vector space 
$\mathcal H_f(z_1)$.
Since we also can use  the inverse map $T_{\gamma^{-1}}$
it follows that the $k$-tuple
$\{T_\gamma(g_\nu)\}$ yields a basis of $\mathcal H_f(z_1)$.
In particular the vector space $\mathcal H_f(z)$ have common dimension 
$k$ as $z$ varies in the connected open set
$\Omega$.
\emph{Summing up}, we can conclude the following:

\medskip


\noindent
{\bf 4.3 Proposition}
\emph{If $f\in M\mathcal (\Omega)$ has finite determination the complex vector spaces
$\mathcal H_f(z)$ have  common dimension. Moreover, one gets
a basis of these by  starting at any point $z_0$ and choose some $k$-tuple
of $C$-linearly germs $g_1,\ldots,g_k$ in $W(f:z_0)$. 
Then
we obtain a basis in $\mathcal H_f(z)$ for any point $z\in\Omega$ by a $k$-tuple
$\{T_\gamma(g_\nu)\}$ where $\gamma$ is any curve which joins  
$z_0$ and $z$.}
\bigskip


\noindent
{\bf 4.4 The case of a punctured disc}
Let $\dot D=\{0<|z|<R\}$
be a punctured disc centered at the origin.
Consider some $f\in\mathcal M\mathcal O(\dot D)$ of finite determination and let 
$k$ be its rank.
In a punctured open disc every closed curve is homotopic to
a closed circle parametrized by  $\theta\mapsto re^{i\theta}$.
Another way to express this is that the fundamental group
$\pi_1(\dot D)$ is isomorphic to the abelian group of
integers.
Thus, the multi-valuedness is determined by a sole $T$-operator
which arises when we let $\gamma$ be a circle surrounding the origin in
the positive sense.
Given $z_0\in\dot D$ we  consider the  ${\bf{C}}$-linear operator
\[
T_\gamma\colon\,\mathcal H_f(z_0)\mapsto\mathcal H_f(z_0)
\]
By Jordan's decomposition theorem we can choose a basis
in $\mathcal H_f(z_0)$ such that the matrix representing
$T_\gamma$ is of Jordan's  normal form.
This means that we have a direct sum
\[ 
\mathcal H_f(z_0)=\oplus\,
\mathcal K_\nu(z_0)
\]
where $\{\mathcal K_\nu(z_0)\}$ are $T_\gamma$-invariant subspaces and the
restriction of $T_\gamma$
$\mathcal K_\nu(z_0)$ is represented by an elementary
Jordan matrix $J(m,\lambda)$
for some complex number $\lambda$ and $m\geq 1$.
Given the pair $m,\lambda$  we consider
a local branch of the function
\[ 
f(z)=z^\alpha\cdot\,[ \text{Log}\,z\,]^{m-1}\quad\colon\,
e^{2\pi i\alpha}=\lambda
\]
which for example is defined close to $z=1$ where 
$f(1)=0$. After one turn around the origin we get a new local branch
of the form
\[ 
f_1(z)=\lambda\cdot  z^\alpha\cdot
[\text{Log}\,z\,+2\pi i]^{m-1}
\]
Continuing in this way $m$ times we see that 
the local branches of $f$ 
generate an $m$-dimensional complex vector space
whose  monodromy is  determined by the matrix $J(m,\lambda)$.
Using this fact it follows that if $f(z)$ is any local branch of a multi-valued function of
finite determination, then it can be expressed as:
\[ 
f(z)=\sum_{\nu=1}^k\sum_j\, \, g_{\alpha_\nu,j}(z)\cdot
z^{\alpha_\nu}\cdot\,[\text{Log}\, z\,]^j\tag{*}
\] 
Here 
$0\leq\mathfrak{Re}(\alpha_1)<\ldots<\mathfrak{Re}(\alpha_k)<1$
and $\{j\}$ is a finite set of  non-negative integers
and the $g$-functions 
are \emph{single-valued} in the punctured disc $\dot D$.
Moreover these $g$-functions
are
uniquely determined provided a specific local 
branch of the Log-function is chosen. 
For example, when $f$ is a local branch at some real point
$0<a<R$ where  $\text{Log}\,a$ is chosen to  be real.
\bigskip





\centerline { \bf\large 5. Homotopy and Covering spaces}
\bigskip

\noindent
First we  recall some facts
in  topology.
Let $X$ be a metric space, i.e. the topology is defined by some
distance function. By a  curve in $X$ we  mean a
continuous map $\gamma$ from the closed unit interval
$[0,1]$ into $X$.
In general $\gamma$ need not be 1-1.
The initial point is $\gamma(0)$ and the end point is $\gamma(1)$.
If $\gamma(0)=\gamma(1)$ we 
say that $\gamma$ is a \emph{closed} curve.
We say that $X$ is \emph{arcwise connected}
if there to each pair of points $x_0,x_1$ exists some curve
$\gamma$ with
$x_0=\gamma(0)$ and
$x_1=\gamma(1)$.

\medskip


\noindent
{\bf A notation.} Given a point $x_0\in X$ we denote by
$\mathcal C(x_0)$ the family of all closed curves
$\gamma$ where
$\gamma(0)=\gamma(1)=x_0$.
\medskip

\noindent
{\bf 5.1 Definition.}
\emph{A pair of closed curves
$\gamma_0$ and $\gamma_1$
in $\mathcal C(x_0)$ are homotopic if there
exists a continuos map $\Gamma$ from the
unit square
$\square=\{(t,s)\quad\colon\, 0\leq t,s\leq 1\}$ into $X$
such that}
\[ 
\Gamma(t,0)=\gamma_0(t)
\,\,\text{and}\,\, 
\Gamma(t,1)=\gamma_1(t)\quad
\, \Gamma(0,s)=\Gamma(1,s)=x_0\,\,\,
\colon\,\, 0\leq s\leq 1
\]
\medskip

\noindent
It is clear that homotopy yields an equivalence relation on 
$\mathcal C(x_0)$.
If $\gamma\in \mathcal C(x_0)$ then
$\{\gamma\}$ denotes its homotopy class.
Next, 
if
$\gamma_0$ and $\gamma_1$ are two closed curves
at $x_0$ we get a new closed curve
$\gamma_2$ defined by
\[ 
\gamma_2(t)=\gamma_1(2t)\,\colon\, 0\leq t\leq \frac{1}{2}\,
\,\text{and}\,\, 
\gamma_2(t)=\gamma_2(2t-1)\,\colon\, \frac{1}{2}\leq t\leq 1\,
\]
We refer to $\gamma_2$ as the composed curve
and it is denoted by $\gamma_1\circ\gamma_0$. One verifies easily that
the homotopy class of $\gamma_2$
depends upon $\{\gamma_1\}$
and $\{\gamma_1\}$ only.
In this way we obtain a composition law on the 
set of homotopy classes of closed curves at $x_0$ defined by
\[
\{\gamma_1\}\cdot \{\gamma_0\}=
\{\gamma_1\circ\gamma_0\}
\]
One verifies easily that this composition satisfies the associative law.
A neutral element is the closed curve
$\gamma_*$ for which $\gamma_*(t)=x_0$ for every $t$.
Finally, if
$\gamma(t)$ is any closed curve at
$x_0$ we get a new closed curve by reversing
the direction, i.e. set
\[ 
\gamma^{-1}(t)= \gamma(1-t)
\]


\noindent
{\bf Exercise.} Show that
the composed curve $\gamma^{-1}\circ\gamma$ is homotopic to $\gamma_*$.
\medskip

\noindent
{\bf 5.2 The fundamental group}. The construction of
composed closed curves and the exercise above show
that 
homotopy classes of closed curves at $x_0$ give elements of a group 
to be denoted
by $\pi_1(X:x_0)$.
\medskip

\noindent
{\bf Remark.} The group
$\pi_1(X:x_0)$ is intrinsic in  the sense that it does not depend upon the chosen point $x_0$. Namely, let $x_1$ be another point in $X$ and fix
a curve
$\lambda$ with $\lambda(0)=x_0$ and $\lambda(1)=x_1$.
Then we obtain a map from
$\mathcal C(x_1)$ to $\mathcal C(x_0)$ defined by
\[
\gamma\mapsto \lambda^{-1}\circ\gamma\circ\lambda\tag{i}
\]
One verifies that (i) sends homotopic curves to homotopic curves
and by considering homotopy classes we obtain an 
isomorphism between 
$\pi_1(X:x_0)$ and $\pi_1(X:x_1)$.
Hence there exists an intrinsically defined group
denoted by $\pi_1(X)$. It is called the fundamental group of
the metric space $X$. If 
$\pi_1(X)$ is reduced to a single element,  i.e. when all closed curves
in $\mathcal C(x_0)$ are homotopic we say that
$X$ is \emph{simply connected}.
\medskip

\noindent
{\bf Exercise.}
Let $x_0$ and $x_1$ be two distinct points in $X$. Denote by
$\mathcal C(x_0,x_1)$ the family of curves
$\gamma$ for which
$\gamma(0)=x_0$ and $\gamma(1)=x_1$.
Two such curves
$\gamma_0$ and $\gamma_1$ are homotopic if there exists
a continuos map $\Gamma$ from the square $\square$ such that
\[
\Gamma(t,0)=\gamma_0(t)
\,\,\text{and}\,\, 
\Gamma(t,1)=\gamma_1(t)\quad
\, \Gamma(0,s)=x_0\,\,\text{and}\,\, \Gamma(1,s)=x_1\,\,\,
\colon\,\, 0\leq s\leq 1
\]
\medskip
\noindent
Show that a pair $\gamma_0$ and $\gamma_1$ are homotopic in
$\mathcal C(x_0,x_1)$ if and only if the closed curve
$\gamma_1^{-1}\circ\gamma_0$ is homotopic to
$\gamma^*$ in $\mathcal C(x_0)$ where
\[ 
\gamma_1^{-1}=\gamma_1(1-t)
\]
In particular each pair of curves in $\mathcal C(x_0,x_1)$
are homotopic if $X$ is simply connected.
\bigskip

\noindent
{\bf 5.3 Covering spaces.}
Let $X$ and $Y$ be two
arcwise connected metric
spaces.
A continuous map $\phi$ from $X$ onto $Y$
is called a \emph{covering map} if the following hold:
For each $y_0\in Y$ there exists an open neighborhood
$U$ such that the inverse image
$\phi^{-1}(U)$ is a union of  pairwise disjoint open
sets $\{U_\alpha^*\}$ and the restriction of $\phi$ to each
$U_\alpha^*$
is a homeomorphism from
this set onto $U$.
\medskip

\noindent
{\bf 5.4 Lifting of curves.} Suppose that
$\phi\colon X\to Y$
is a covering map.
Let $\gamma$ be a curve in $Y$ defined by
a continuous map $t\to\gamma(t)$ from the closed unit interval $[0,1]$
into $Y$ with 
some initial point
$y_0=\gamma(0)$ and some end-point $y_1=\gamma(1)$.
The case  $y_0=y_1$ is not excluded, i.e. $\gamma$ may
be a closed curve.
Next, in $X$ we chose a point $x_0$ such that
$\phi(x_0)=y_0$.
By assumption there exists an open neighborhood $U$ of $y_0$
in $X$ a unique open neighborhood $U^*$ of $x_0$
such that
$\phi\colon\, U^*\to U$ is a homeomorphism.
Since $t\to\gamma(t)$is continuous there exists
some $\delta>0$ such that
\[
\gamma(t)\in U\,,\quad 0\leq t\leq \delta\tag{i}
\]
Then we get a \emph{unique} curve
$\gamma^*$ in $X$
defined for $0\leq t\leq\delta$ such that 
\[
\phi(\gamma^*(t))=\gamma(t)\,,\quad 0\leq t\leq \delta\quad
\text{and}\,\, \gamma^*(0)=x_0\,.\tag{ii}
\]
If this lifting process can continued for
all $0\leq t\leq 1$ we say that
$\gamma$ has a lifted curve
$\gamma^*$. This  means that there exists a curve
$t\mapsto\gamma^*(t)$ from $[0,1]$ into $X$
such that
\[
\phi(\gamma^*(t))=\gamma(t)\,,\quad 0\leq t\leq 1\quad
\text{and}\,\, \gamma^*(0)=x_0\,.\tag{*}
\]
\medskip

\noindent
{\bf Exercise.}
Show  that
the curve
$\gamma^*$ is unique if it exists.
The hint is to use that
$\phi$ is a local homeomorphism.
\medskip

\noindent
The whole discussion above leads to
\medskip

\noindent 
{\bf 5.5 Definition.}\emph{ A
covering
map $\phi\colon X\to Y$ is of class $\mathcal L$ 
if the following hold: For each pair of points
$y_0\in Y$ and 
$x_0\in \phi^{-1}(0)$, every 
curve $\gamma$ in $Y$ with initial 
point $y_0$ can be lifted to a curve in
$X$ with initial point $x_0$.}
\bigskip

\noindent
{\bf 5.6 The case when
$X$ is simply connected.}
Assume this and let
$\phi\colon X\to Y$ be a covering map of class
$\mathcal L$.
Let $y_0\in Y$ and choose some point $x_0\in\phi^{-1}(y_0)$.
Next, let
$\gamma$ be a closed curve in $Y$ with
$\gamma(0)=\gamma(1)=y_0$.
By assumption there exists a unique lifted curve
$\gamma^*$ in $X$ with
$\gamma^*(0)=x_0$.
Suppose  that
$\gamma^*(1)=x_0$, i.e.  the lifted curve is closed.
Since $X$ is simply connected it is homotopic to
the trivial curve which stays at $x_0$, i.e. there exists a continuos map
$\Gamma^*$ from $\square$ into $X$ 
such that
\[
\Gamma^*(t,0)=\gamma^*(t)\,
\text{and}\,\Gamma^*(t,1)=x_0\quad
\Gamma^*(0,s)=\Gamma(1,s)=x_0\,\,\colon\,0\leq s\leq 1\tag{i}
\]
Now $\Gamma(t,s)=\phi(\Gamma^*(t,s)$  is a continuous map from
$\square$ into $Y$ and from (i) we see that
$\Gamma$ yields a homotopy between
$\gamma_0$ and $\gamma_1$.
Using this observation we arrive at:
\medskip

\noindent
{\bf 5.7 Proposition.} \emph{Let
$\gamma_0$ and $\gamma_1$ be two closed curves at
$y_0$. Then  they are homotopic if and only if
$\gamma^*(1)=\gamma^*(1)$.}
\medskip

\noindent
\emph{Proof.}
We have already seen that if
$\gamma^*(1)=\gamma^*(1)$
then the two curves are homotopic.
Conversely, if they are homotopic we get a continus map
$\Gamma(s,t)$ from $\square$ into $Y$ and for each
$0\leq s\leq 1$ we have the closed curve
$\gamma_s(t)=\Gamma(t,s)$
at $y_0$.
Since the inverse fiber
$\phi^1{_1}(y_0)$ by assumption is a discrete set in
$X$, it follows by 
continuity and the unique path lifting
that $s\mapsto \gamma_s^*(1)$ 
is constant and hence
$\gamma_0^*(1)=\gamma^*_1(1)$.
\bigskip

\noindent
{\bf 5.8 Conclusion.}
Proposition 5.7 shows that homotopy classes of closed curves
$\gamma$ at $y_0$ are in a 1-1 correspondence with
their end-points in $X$.
Notice also that if $x$ belongs to $\phi^{-1}(y_0)$ then
the arc-wise connectivity of $X$ gives a curve
$\rho$ where
$\rho(0)=x_0$ and $\rho(1)=x$.
Now $\gamma(t)=\phi(\rho(t))$ is a closed curve at
$y_0$ and here $\gamma^*(t)=\rho(t)$
and hence $x$ appears as an end-point for at least one closed curve
at $y_0$. Identifying $\pi_1(Y)$ with homotopy classes of closed
curves at $y_0$ we have therefore proved the following:
\medskip

\noindent
{\bf 5.9 Theorem.}
\emph{The map $\gamma\to\gamma^*(1)$
yields a bijective correspondence between the fundamental group
$\pi_1(Y)$ and the inverse fiber
$\phi^{-1}(y_0)$.}

\medskip

\noindent
{\bf Exercise.}
Let $X$ and $Z$ be two simply connected metric spaces.
Suppose that  $\phi\colon X\to Y$ and $\psi\colon Z\to Y$
are two covering maps which both belong to the class $\mathcal L$.
Fix some $y_0\in Y$. Choose
$x_0\in \phi^{-1}(y_0)$ and 
$z_0\in \psi^{-1}(y_0)$.
Next, let $y\in Y$ and consider some curve
$\gamma$ in $Y$ with $\gamma(0)=y_0$ and $\gamma(1)=y$.
Its unique lifted curve to
$X$ is denoted by $\gamma^*$ and we get the end-point
\[
\gamma^*(1)\in \phi^{-1}(y)
\]
Similarly. we get a unique lifted curve
$\gamma^{**}$ in $Z$ and the end-point

\[
\gamma^{**}(1)\in \psi^{-1}(y)
\]
From the above these two end-points only depend on  the homotopy class
of $\gamma$. Use this to conclude that
we obtain a \emph{unique and  bijective} map
from the discrete fiber
$\phi^{-1}(y)$ to $\psi^{-1}(y)$. Moreover, as $y$ varies
in $Y$ this gives a unique homeomorphism $G$ from $X$ onto $Z$
with $G(x_0)=z_0$.
\medskip

\noindent
\centerline {\bf 5.10 The universal covering space.} 

\medskip

\noindent
The exercise above
shows that up to a homeomorphism the  metric space
$Y$ with a non-trivial fundamental group
has a unique simply connected covering space
$X$ where the map $\phi\colon X\to Y$ is of class $\mathcal L$.
In topology one refers to $X$ as the \emph{universal covering space}
of $Y$.
There remains the question   if there
exists  such a universal covering space.
We shall not
try to investigate this existence problem in full generality
but  consider the case when
$Y$ is an open and connected subset of
${\bf{C}}$.
In this case we can show that
$Y$ has a universal covering space by the following construction.
First, if
$y_0\in Y$ we  find an open disc
$D(y_0)$ centered at $y_0$ whose
radius is
equal to  $\text{dist}(y_0,\partial Y)$.
Next, let $G$ denote the fundamental group of
$Y$. To each $y_0\in Y$
we consider the product set
\[ 
D(y_0)\times G
\]
It is considered as a disjoint union
of copies of the discs given by
\[
D(y_0)\times\{g\}\quad\, g\in G
\]
Let us now consider two circles
$D(y_0)$ and $D(y_1)$ with a non-empty intersection.
Let $y\in D(y_1)\cap D(y_1)$.
If $g\in G$ we have the disc
$D(y)\times{g}$.
Let $g=\{\gamma\}$
for some  closed curve at $y$.
In the disc $D(y_0$ we have the line $\ell_0$ from
$y_0$ to $y$. Now we get
a closed curve  at $y_0$ defined by
\[
\gamma_0=\ell_0^{-1}\circ\gamma\circ\ell_0
\]
Identifying $G$ with $\pi_1(Y:y_0)$ we get the
$G$-element $g_0=\{\gamma_0\}$.
We say that $g_0$ is \emph{a neighbor} of $g$. In the same way
we use the straight line $\ell_1$ from $y_1$ to $y$
and construct a neighbor $g_1$
to $g$ when
$G$ is identified with
$\pi_1(Y:y_1)$.
Set
\[
W_0(g)=\{(y,g_0)\in D(y_0)\times\{g_0\}\quad\colon\, y\in
D(y_0)\cap D(y-1)\}
\]
Similarly we set
\[
W_1(g)=\{(y,g_0)\in D(y_1)\times\{g_1\}\quad\colon\, y\in
D(y_0)\cap D(y-1)\}
\]
Now we agree to
identify $W_0(G)$ and  $W_1(g)$. This identification 
takes place for each $g\in G$.
At this stage it is clear how one constructs the universal covering space over
$Y$ where the details are left to the reader.






\newpage


\centerline{\bf 6. The uniformisation theorem.}
\bigskip

\noindent
{\bf Introduction.}
Let $\Omega$ be a connected open subset
of
${\bf{C}}$. If
the closed complement contains at least two points
then the universal covering space
can be taken as the unit disc
$D$. Moreover  there exists a a covering map $f\colon\,D\to \Omega$
of $\mathcal L$-type given by an analytic function.
This will be proved in Chapter 6.
Here we  take this existence for granted and analyze some consequences.
In particular we  discuss some    properties of such an
analytic uniformisation.
More precisely,
in Chapter VI
we prove Riemann's mapping theorem for connected domains
which goes as follows:
\medskip

\noindent
{\bf 6.1 Theorem.} \emph{For every  $z_0\in\Omega$ 
there exists a unique analytic covering map
$f$ of $\mathcal L$-type where $f(0)=z_0$ and $f'(0)$ is real and positive.}
\bigskip


\noindent
{\bf 6.2 The multi-valued inverse to $f$.}
We take the theorem above for granted and discuss some consequences.
Let $f$ be a an analytic covering map as above.
To distinguish  the
$z$-coordinate in $\Omega$ from $D$ we let $w$ be the complex
coordinate in $D$.
To begin with $f$ yields a biholomorphic map from a small open
disc $D_*$ centered at the origin in $D$
to a small open neighborhood $U_0$ of
$z_0$. It gives the inverse analytic function
$F(z)$ defined in $U_0$ such that
\[ 
F(f(w))=w\quad\, w\in D\,.
\]
Next, let $\gamma$ be a curve in
$\Omega$ with
$\gamma(0)=z_0$.
Since $f$ is of $\mathcal L$-type there exists a unique lifted
curve
$\gamma^*$ in $D$ with
$\gamma^*(0)=0$.
Now the germ of $F$ at $z_0$ can be continued analytically along
$\gamma$ where
\[ 
T_{\gamma(t)}(F(\gamma(t))=\gamma^*(t)\quad\colon\,
0\leq t\leq 1\tag{i}
\]
Hence we get a multi-valued analytic function $F$ in $\Omega$.
It gives an inverse to $f$ in the following sense: Let
$w\in D$ and consider the curve
$t\mapsto t\cdot w$ in $D$. Now $t\mapsto f(t\cdot w)$
is a curve $\gamma$ in
$\Omega$ and by the construction (i) we have
\[
T_{\gamma(t)}(F(f(t\cdot w))=tw\quad\colon\,0\leq t\leq 1\tag{ii}
\]
We may express this by saying that the composed function
$F\circ f$ is the identity on $D$.
\medskip

\noindent
{\bf{6.3 Example.}}
To begin with the unit disc $D$ is conformally equivalent to the upper
half-plane so in Theorem we can just as well consider
an analytic covering map $f$ from $U_+$ to $\Omega$.
Suppose that $\Omega$ is the punctured unit disc
$\dot D= D\setminus \{0\}$.
in $U_+$ we have the analytic function
$f(z)= e^{iz}$
and it is clear that it
gives
an analytic covering map from $U_+$ onto $\dot D$
where we have
\[
f(i)= e^{-1}=z_0
\]
In $\dot D$ we have the multi-valued Log-function
\[
F(z)= -i\cdot \text{Log}(z)
\]
So here
\[ 
\mathfrak{Im}(F)=-\cdot \text{Log}|z|
\]
When $0<|z|<1$ it means that
the imaginary part is $>0$ and it is clear that  the $F$-image is $U_+$.
We also get
\[ 
-i\cdot \text{Log}(e^{iw})=w\quad\colon\, w\in U_+
\]





\medskip

\noindent
{\bf 6.4 Constructing single-valued functions.}
Return to the situation in Theorem 6.2, i.e. $f$ is a covering map from
$D$ into $\Omega$.
Let $g(w)$ be some analytic function in
$D$ whose range $g(D)\subset \Omega$ and $g(0)=0$.
We  use $F$ to construct another 
single-valued analytic function
$F\circ g$  in $D$. Namely, let $w\in D$
which gives  the curve $\gamma_w$
$t\mapsto g(t\cdot w)$ in $\Omega$
where $\gamma_w(0)=z_0$.
We can continue $F$ along
this curve
and when $t=1$ we 
get the value
\[ 
T_{\gamma_w(1)}(F(g(w))
\]
It is clear that this gives an analytic function in
$D$ defined by
\[ 
F\circ g(w)= T_{\gamma_w(1)}(F(g(w))
\]
This construction can be performed for every
$g\in\mathcal O(D)$ such that $g(0)=z_0$ and $g(D)\subset \Omega$.
Hence we have
proved the following:
\medskip

\noindent
{\bf 6.5 Proposition.}
\emph{Let $\mathcal O_*(D:\Omega)$  denote the family of analytic functions $g$
in
$D$ where $g(0)=z_0$ and $g(D)\subset \Omega$.
Then there exists
a map from $\mathcal O_*(D:\Omega)$ into $\mathcal O(D)$
given by:}
\[ 
g\mapsto F\circ g\,.
\] 
\emph{Here
$F\circ g(0)=0$ and the range
$(F\circ g)(D)\subset D$.}
\bigskip



\noindent
{\bf 6.6 M�bius transforms.}
Let $f$ be a covering map as in Theorem 6.1.
Identify the fundamental group
$\pi_1(\Omega)$ with
homotopy classes of closed curves at $z_0$.
Theorem xx gives a bijective map between
elements in the group $\pi_1(\Omega)$ and the discrete subset 
$f^{-1}(z_0)$ of $D$.
Consider a point $a$ in this inverse fiber, i.e.
here $f(a)=z_0$. For each
$0\leq\theta<2\pi$
we get a new covering map
$g$ defined by
\[ 
g(w)=f(e^{i\theta}\cdot \frac{w+a}{1+\bar a\cdot w})
\]
Here $g(0)=f(a)=z_0$ and
the complex derivative at $w=0$ becomes
\[ 
g'(0)=f'(a)\cdot  e^{i\theta}\cdot (1-|a|^2)
\]
We can choose $\theta$ so that
$f'(a)\cdot  e^{i\theta}$ is real and positive.
With this choice of $\theta$ it follows from
the uniqueness in Theorem 6.1
that $g=f$.
Hence the function $f$ satisfies

\[ 
f(w)=f(e^{i\theta}\cdot \frac{w+a}{1+\bar a\cdot w})\tag{*}
\]
\bigskip










\noindent
\centerline {\bf 6.7 Inverse multi-valued functions.}
\bigskip




\noindent
Let $\phi$ be an analytic function 
defined in some
open and connected subset $U$
of ${\bf{C}}$. We assume that the derivative
is $\neq 0$ at every point and get the open image domain
$\Omega=\phi(U)$.
Since $\phi$ is locally conformal it gives a covering map from
$\Omega$ onto $U$.
Consider some $\zeta_0\in\Omega$ and
put $x_0=\phi(\zeta_0)$.
We get a germ $f(x)$ of an analytic function at
$x_0$ using the local inverse of $\phi$, i.e. since
$\phi'(\zeta_0)\neq 0$
there exists a small open disc
$D_\delta(\zeta_0)$ such that
\[ 
f(\phi(\zeta)=\zeta\quad\colon\quad
|\zeta-\zeta_0|<\delta
\]
In fact, we simply find the  convergent power series
\[ 
f(x)=
\sum\, c_\nu(x-x_0)^\nu
\]
where $c_0,c_1,\ldots$ are determined so that
\[
\sum\, c_\nu\bigl (\phi(\zeta)-\phi(\zeta_0)\bigr)^\nu=\zeta
\]
Less obviousis the following
\medskip

\noindent
{\bf 6.8 Proposition.}
\emph{The germ $f$ at $x_0$ extends to a multi-valued analytic function in
$U$.}
\medskip

\noindent
\emph{Proof.} 
Let $\gamma$ be a curve in
$U$ having $x_0$ as initial point.
The lifting lemma gives a unique curve
$\gamma^*$ in $\Omega$.
The required analytic continuation of $f$
along
$\gamma$ now follows when we apply the Heine-Borel Lemma cover
the compact set $\gamma$ with a finite set of discs which
are homemorphic images of discs
in $\Omega$ whose consequtive union covers
$\gamma^*$. Then we use that
$\phi$ is everywhere analytic. The result is that
the germ $T_\gamma(f)$ at the end-point
$\zeta_1=\gamma(1)$ satisfies
\[
T_\gamma(f)(\phi(x))=x
\]
where $x$ is close to the point $\phi(\gamma^*(1))$
in $\Omega$. So in particular
\[ T_\gamma(f)(\gamma(1))= \gamma^*(1)
\]
which clarifies how to determine  values of
the multi-valued  analytic function.
\medskip

\noindent
{\bf 6.9 Remark.}
It is instructive to consider some specific cases.
Consider the entire function
$\phi(\zeta)=e^\zeta$. With $\Omega={\bf{C}}$
the image domain $U$
is the punctured complex plane. If we take
$x_0=1$ and $\zeta_0=0$
we find that $f$ is the multi-valued Log-function
where we start with the local branch at $x_0=1$ for which
$\log\,1=0$.
Next, let
us regard the polynomial $\phi(\zeta)= \zeta^2$.
in order to get a covering we must exclude the origin to
ensure that
$\phi'(\zeta)\neq 0$. So if $\Omega={\bf{C}}\setminus\{0\}$
we get a covering whose image set $U$
also becomes the punctured complex plane.
In this case the inverse fiber consists of two points and
the function $f(z)$ is the multi-valued
square-root of $z$.
More involved examples occur when
$\phi(\zeta)$ is a polynomial of degree
$\geq 3$.
Consider the case
\medskip

\[ 
\phi(\zeta)= \zeta^3-3\zeta-2
\]
Here $\phi'(\zeta)=3\zeta^2-3$ and to avoid zeros we take
$\Omega={\bf{C}}\setminus \{-1,1\}$.
In this case the multi-valued inverse function
is defined in the open set
given as the image under $\phi$.
Thus, we seek the domain:

\[ U=\{x\colon\quad x=
\zeta^3-3\zeta -2\quad\colon \zeta\neq -1,+1\}
\]
The reader is invited to find $U$. Next, notice that
$\phi$ has a simple zero when
$\zeta=2$.
Now we can start with the germ and 
$f_0$ at $x=0$
which satisfies
\[ 
f_0(\zeta^3-3\zeta +1)=\zeta\quad\colon\quad f_0(0)=0
\]
for a small disc centered at the $\zeta=2$  in the complex
$\zeta$-plane. 
After analytic continuation the total number
of local branches of $f$ at $x=0$ is 3, i.e. this follows since
the inverse fibers under the covering map $\phi$ contain
three points.
However, the determination of these local branches is not so easy.
The reason is that the fundamental group of
$U$ no longer is generated by a single loop around a point, i.e. above we have 
removed two
points. So one must study the analytic continuation of
$f$ along several closed curves. To begin with the two simple closed curves
in $U$ which surround +1 and -1 respectively.
It is clear from this "intuitive discussion" that
one needs a more systematic theory to analyze
analytic continuations.
The most efficient procedure is to use
$\mathcal D$-module theory where
one starts with  the polynomial map
\[ 
\phi\colon\zeta\mapsto x=\zeta^3-3\zeta+1
\]
and  regards not only the special multi-valued inverses
function $f$ but more objects which have
direct images and to achieve this the basic
role is played by the direct image of
$\mathcal D$-modules under $\phi$ which after
is used to determine the set of differential operators in
the Weyl algebra ${\bf{C}}\langle x,\partial_x\rangle$
which annihilate $f$.
Let us remark that $\mathcal D$-module theory has a rather recent origin.
The foundations of the theory
appeared for the first time in the 
thesis [Kash]
by Masaki Kashiwara from 1970.
We refer to Chapter for further comments on
$\mathcal D$-module theory and describe how we
use it to investigate the local branches of $f$ in the specific example above.

\medskip

\centerline{\bf 6.10 Constructing  single-valued functions.}

\medskip

\noindent
Let $\Omega$ be a connected open set and consider some
multi-valued analytic function
$F$ in $\Omega$.
Let $U$ be some open and 
\emph{simply connected} set.
Consider some $h\in\mathcal O(U)$ whose image set $h(U)$
is contained in $\Omega$. No further conditions on $h$ are imposed, i.e.
the inclusion
$h(U)\subset\Omega$ may be strict and
the derivative of $h$ may have zeros.
Using $F$ we produce single valued analytic functions
in $U$ by the following procedure.
Let us fix a point $\zeta_0\in U$
and put $x_0=h(\zeta_0)$.
At $x_0$ we haver the family of local branches of
$F$. Let $f_*$ be one such local branch.
Next, let $\gamma$
be a curve in $U$ where $x_0$ is the initial point
and $x=\gamma(1)$ denotes the end-point. 
In $\Omega$ we get the image curve
\[ 
t\mapsto  h(\gamma(t))\tag{i}
\]
Now $f_*$ has an analytic continuation along
the curve in (i).
When $t=1$ we arrive at the endpoint $\gamma(1)$
which we denote by $x$.
At $x$ 
we can evaluate the local branch $T_\gamma(f_*)$.
Next, let $\gamma_1(t)$ be another curve
in $U$ with  the same end-point $x$ as $\gamma$.
By assumption $U$is simply connected which means that
the tewo curves
$\gamma$ and $\gamma_1$ are homotopic.
It is clear that the homotopy in $U$ implies that
the two image curves
obtained via (i) are homotopic in  the curve family in
$\Omega$ which joint $x_0$ and $x$.
It follows that
the image curves constructed via (i) are homotopic 
The monodromy theorem applied to $F$ implies that
\[
T_\gamma(f_*)(x)=T_{\gamma_1}(f_*)(x)\tag{ii}
\]
\medskip

\noindent
Next, given an open and simply connected
set $U$ in ${\bf{C}}$
we denote by
$\mathcal O(U)_\Omega$
the family of analytic functions in
$U$ whose image is contained in
$\Omega$.
With these notations the discussion above gives:

\medskip

\noindent
{\bf 6.11 Proposition.}
For each point $\zeta_0\in U$ there exists a map
\[
\rho\colon\, \mathcal O(U)_\Omega\times M\mathcal O(\Omega)(x_0)
\to \mathcal O(U)
\]
where $x_0=h(\zeta_0)$
and  for a  pair
$h\in \mathcal O(U)_\Omega$ and $f_*\in M\mathcal O(\Omega)(x_0)$
the analytic function $\rho(h,f_*)$ 
satisfies
\[ 
\rho(h,f_*)(\zeta)= T_\gamma(f_*)(h(\zeta))\quad\colon\quad \zeta\in U
\]
where $\gamma$  is the $h$-image of any curve
in $U$ which joins
$\zeta_0$ with
$\zeta$.
\bigskip

\noindent
{\bf Remark.}
Keeping $h$ fixed we notice that
the map $f_*\to \rho(h,f_*)$ is a ${\bf{C}}$-algebra homomorphism from
the complex ${\bf{C}}$-algebra 
$M\mathcal O(\Omega)(x_0)$ into $\mathcal O(U)$.
\medskip


\noindent
{\bf Example.}
Let $h(z)$ be analytic in the open unit disc
$D$ and assume that $h(D)\subset U=\mathfrak{Im}\, z>0$.
Consider the multi-valued function
$\text{Log}\, z$. I has a single valued branch in
$U$ and we therefore get an analytic function
in $D$ defined by
\[ 
g(z)=\text{Log}\, h(z)
\]
which by the construction satisfies
\[ 
\mathfrak{Im}\, g(z)>0\quad\colon\quad z\in D
\]








\newpage


\centerline {\bf 7. The $p^*$-function.}
\medskip

\noindent
We  construct a special harmonic function which 
will be  used
to get solutions to the Dirichlet problem in
XXX.
Let $\Omega$ be an open and connected set in
${\bf{C}}$.
Its closed complement has connected components.
Let $E$ be
such a connected component.
To each $a\in E$ we get the winding number
$\mathfrak{w}_a(\gamma)$. If $b$  is another point in
$E$ which is sufficiently close to  $a$
it is clear that
\[
\bigl|\frac{1}{\gamma(t)-a}-\frac{1}{\gamma(t)-b}\bigr|<
\bigl|\frac{1}{\gamma(t)-a}\bigr |
\]


\noindent
Rouche's theorem from 1.4 implies that
$\mathfrak{w}_a(\gamma)=\mathfrak{w}_b(\gamma)$, i.e. 
for every closed curve $\gamma$ in $\Omega$,
the winding number
stays constant in each connected component of
${\bf{C}}\setminus\Omega$. 
This  enable us to construct single valued  Log-functions
in $\Omega$.
Namely, let $a\in E$ where $E$ is a connected
componen in the complement of
$\Omega$.
Consider 
$f=\text{Log}\,(z-a)$ and choose a single valued branch
$f_*$ at some point $z_0\in\Omega$. 
If $\gamma\subset\Omega $ is a closed curve with initial point at
$z_0$ the analytic continuation along $\gamma$ of the Log-function
gives:
\[
T_\gamma(f_*)= f_*+2\pi i\cdot \mathfrak{w}_a(\gamma)\tag {1}
\]


\noindent
Next, if $b$ is another point in $E$ we consider
$g_*=\text{Log}(z-b)$ and obtain
\[
T_\gamma(g_*)= g_*+2\pi i\cdot \mathfrak{w}_b(\gamma)\tag {2}
\]



\noindent
Since
$\mathfrak{w}_b(\gamma)=\mathfrak{w}_b(\gamma)$ it follows that
\[
T_\gamma(f_*)-T_\gamma(g_*)=f_*-g_*\tag{3}
\]
Hence the difference $\text{Log}(z-a)-\text{Log}(z-b)$
is a \emph{single valued} analytic function in $\Omega$. Taking  
is exponential we
find $\Psi(z)\in\mathcal O(\Omega)$ such that
\[
e^{\Psi(z)}=
\frac{z-a}{z-b}\tag{4}
\]


\noindent
Since $a\neq b$
we see that
$\Psi(z)\neq 0$
for all $z\in\Omega$. Next, we get the
harmonic function defined in 
$\Omega$
by
\[
p(z)=\mathfrak{Re}\bigl(\frac{1}{\Psi(z))}\bigr)=
\frac{\mathfrak{Re}(\Psi(z)}{|\Psi(z)|^2}\tag{*}
\]


\noindent
Notice that
$\mathfrak{Re}(\Psi(z))=\text{Log}\, |z-a|-\text{Log}|z-b|$
and since
$\text{Log}\, |z-a|\to-\infty$ as $z\to a$
we see from (*) that
\[
\lim_{z\to a}\, p(z)=0\tag{**}
\]


\noindent
Notice also that
$\Psi(z)$ exends to a continuous function on
$\bar\Omega\setminus (a,b)$ and we can
choose a small $\delta>0$ such that
\[ 
\text{Log}|z-a|-\text{Log}\,|z-b|<-1\quad\colon\quad
|z-a|\leq\delta\tag{ii}
\]

\noindent Then (i) and (ii) give

\medskip

\noindent {\bf 7.1 Theorem.}
\emph{Let $a\in\partial\Omega$ be such that
the connected component of ${\bf{C}}\setminus \Omega$
which contains $a$ is not reduced to the single point $a$.
Then there exists a harmonic function
$p^*(z)$
in $\Omega$
for which}
\[
\lim_{z\to a}\, p^*(z)=0
\]
\emph{and there exists $\delta>0$ such that}
\[
\max_{\{|z-a|=r\}\,\cap\Omega}\, p^*(z)<0\quad\colon\quad
z\in D_a(r)\,\cap\,\Omega
\]


\newpage


\centerline{\bf 8. Eisenstein's theorem.}
\bigskip

\medskip

\noindent
{\bf Introduction.}
The theorem  below was  announced by 
Eisenstein in 1852 and 
goes as follows: Let

\[ 
w(z)= c_1z+c_2z^2\ldots\tag{0.1}
\]
be a convergent power series in a disc centered at the origin
where the  coefficients 
$\{c_\nu\}$ are rational numbers. Assume also that 
$w$ satisfies an algebraic equation:
\[
q_m(z)w^m+\ldots+q_1(z)w+q_0(z)=0\quad\colon
q_0,\ldots,q_m\,\,\text{are polynomials}\tag{0.2}
\]
\medskip

\noindent
{\bf{Eisenstein's Theorem}}. 
\emph{Under the assumption above
there exists a positive integer $k$ such that}
\medskip
\[ 
k^\nu\cdot c_\nu\in {\bf{Z}}\tag{*}
\]
\medskip

\noindent
\emph{Or equivalently, we can find $k$ so that
$w(kz)$ has a power series expansion with integer coefficients.}
\medskip


\noindent
The complete proof of (*) was given by
Heine
from 1854. Since
${\bf{R}}$ is a vector space over the field of rational numbers
it is easily seen that (0.1) implies that
we can choose the $q$-polynomials in $Q[z]$, i.e. so that they all have
rational coefficients. Multiplying  these $q$-polynomials with
a positive integer we remove denominators and may  assume 
from the start that
they  have integer coefficients.
\medskip

\noindent
{\bf{0.1 A special case.}}
Suppose that
the $q$-polynomials satisfy:

\[ 
q_m(0=\ldots=q_2(0)=0\quad\text{where}\quad q_1(0)=k\neq 0\tag{**}
\]
When (**) holds
an easy induction over $\nu$ shows that
$k^\nu\cdot c_\nu$ are integers for all $k$.
The non-trivial part of the proof is the reduction  to the case (**).
Heine achieved this by the following construction: For every 
positive integer $p$ we break the series (0.1) and write

\[
w(z)=c_1z+\ldots+c_{p-1}z^{p-1}+\beta_p(z)\cdot z^p\tag{0.5}
\]
Heine showed that if $p$ is sufficiently large then
$\beta_p(z)$ satisfies an algebraic equation for which
(*) holds and after this  Eisenstein's theorem can be derived.
Heine's reduction is established in section 2. But first we
expose background about algebraic functions which is needed
for Heine's result
and in  section 3 we give the proof of
Eisenstein's theorem.
\bigskip

\noindent
\centerline {\bf 1. On algebraic functions.}
\medskip

\noindent
Let ${\bf{C}}[z,w]$ be the
polynomial ring in two
independent variables
Introducing the field
$K={\bf{C}}(z)$ of rational functions in $z$
we get the polynomial ring $K[w]$ of one variable.
This ring has important  properties. Namely,
if $Q(w)=\sum\, k_\nu(z)\cdot w^\nu$ is a $w$-polynomial with
of some degree $m\geq 1$, then
every other polynomial $S\in K[w]$ can be written in a unique way as
\[ S(w)= A(w)\cdot Q(w)+R(w)\quad\colon\,\text{degree of}\,\, R(w)\leq m-1\tag{1}
\]
This  division shows that 
$K[w]$ is an \emph{euclidian ring} where every ideal is principal and
every polynomial can be written in a unique way as a product of
irreducible polynomials. Of course, in such a factorisation multiple factors
can occur.
Suppose now that
$Q(w)$ is an irreducible polynomial on $K[w]$. Bringing out
the denominators in the coefficients from ${\bf{C}}(z)$
it follows that $Q(w)$ corresponds to
a polynomial
\[ 
q(w,z)=\rho(z)\cdot\, [g_m(z)w^m+\ldots+g_1(z)w+g_0(z)]\tag {2}
\]
where $g_0(x),\ldots,g_m(z)$ are polynomials in ${\bf{C}}[x]$ 
without a common zero. Moreover, this factorisation
is unique when we require that
the leading polynomial $g_m(x)$ is monic, i.e. if $k$ is its degree
then 
$g_m(x)= x^k+\text{lower order monomials}$.
Now (2) is used to construct root functions of
$q(z,w)$. Namely, let $\sigma$ be the finite set of zeros of
$g_m(z)$. For each fixed $z_*\in{\bf{C}}\setminus\sigma$
we get a $w$-polynomial of degree $m$:
\[
g_m(z_*)w^m+\ldots+g_1(z_*)w+g_0(z_*)
\]
By the fundamental theorem of algebra it has $m$ roots 
where eventual multiple roots are repeated. Denote this unordered
$m$-tuple of roots by
$\alpha_1(z_*),\ldots,\alpha_m(z_*)$.
Then we have
\[
q(w,z_*)=\prod_{\nu=1}^{\nu=m}\, (w-\alpha_\nu(z_*))\tag{3}
\]
To find out if multiple roots occur  we
set
\[
\Delta(z_*)= \prod_{j\neq\nu}\,
(\alpha_\nu(z_*)-\alpha_j(z_*))\tag{4}
\]
This is a product of $m(m-1)/2$ many factors 
and constructed in a symmetric fashion,
i.e. the product does not change when the $m$-tuple of roots is permuted.
Since $q(w,z)$ comes from the irreducible polynomial $Q(w)$
the $\Delta$-function is not  identically zero.
To see this we regard the $w$-derivative of $Q(w)$ which has degree
$m-1$ in $K[w]$.
Since $Q(w)$ is irreducible  there exist  polynomials
$A(w)$ and $B(w)$ in $K[w]$ such that
\[
A(w)Q'(w)+B(w)Q(w)=1\tag{5}
\]
Bringing out common denominators we get an equality in ${\bf{C}}[z,w]$:
 \[ 
a(w,z)\cdot \partial (Q(w,z)/\partial w+ b(w,z)\cdot Q(w,z)= h(z)\tag{6}
\]
where $a,b,h$ are some polynomials and $h(z)$ is not identically zero.
Then it is clear that if $z_*\in {\bf{C}}\setminus\sigma$ 
and $h(z_*)\neq 0$, then $\Delta(z_*)\neq 0$.
\medskip

\noindent
{\bf 1.1 The resolvent polynomial for $\Delta$.}
We have seen that $\Delta(z_*)$ is a symmetric product of the roots.
By a wellknown result in algebra this implies that
it can be expressed as a sum of elementary symmetric 
$\alpha$-polynomials. Applied to the present situation this gives
rational functions
$c_0(z)\,\ldots,c_{m(m-1)/2}(z)$ such that
\[ 
\Delta(z_*)=\sum\, c_j(z_*)\cdot[\,\alpha_1^j(z_*)+\ldots
,\alpha_m^j(z_*)]\quad\colon\, z_*\in{\bf{C}}\setminus\sigma.\tag{7}
\]


\noindent
Now residue calculus gives:
\[
\alpha_1^j(z_*)+\ldots+\alpha_m^j(z_*)=
\frac{1}{2\pi i}\cdot
\int_{|w|=R}\,
\frac{w^j\cdot\partial q/\partial w(w,z_*)\cdot dw}
{q(w,z_*)}\tag{8}
\]
where we for each fixed $z_*$ choose $R$ so large that the zeros of
$q(w,z_*)$ have absolute value $<R$.
From this we  conclude that
$\Delta(z_*)$ is a rational function
of the $z$-variable and since it stays bounded
when the leading polynomial $g_m(z)$ of $q(w,z)$ is non-zero, this
rational function has poles contained in $g_m^{-1}(0)$.
Finally, using (7-8) and the euclidian division in $K[w]$
we arrive at
\medskip



\noindent
{\bf 1.2 Proposition.}
\emph{There exists a unique $w$-polynomial}

\[ \mathcal R(w,z)= r_{m-1}(z)w^{m-1}+\ldots+r_0(z)\quad
\colon\, r_\nu(z)\in{\bf{C}}|z]
\] 
\emph{such that}
\[ 
\Delta(z_*)=
\frac{1}{2\pi i}\cdot
\int_{|w|=R}\,
\frac{\mathcal R(w,z_*)\cdot\partial q/\partial w(w,z_*)\cdot dw}
{q(w,z_*)}\quad\colon\quad z_*\in{\bf{C}}\setminus\sigma
\]
\emph{Moreover, the poles of the rational $r$-functions  are contained in
$g_m^{-1}(0)$.}


\medskip

\noindent 
\emph{Proof.}
The existence of $\mathcal R$ has already been settled. To see that
uniqueness holds we recall from
linear algebra that the \emph{van der Monde determinant}
constructed from the $m\times m $-matrix whose rows are
$1,\alpha_\nu(z_*),\ldots,\alpha_\nu^{m-1}(z_*)$
is $\neq 0$  when $z_*\in{\bf{C}}\setminus\sigma$.
\medskip


\noindent
{\bf 1.3 The multi-valued root functions.}
Put $\sigma^*=\sigma\cup\,h^{-1}(0)$.
Outside this set the $m$-tuple of roots are distinct and
at the same time we avoid zeros of the $g_m$-polynomial.
This implies that
the root functions yield single valued analytic functions
in every disc $D$ contained in $\Omega={\bf{C}}\setminus\sigma^*$.
Moreover, each single root is a germ of a multi-valued analytic function
defined in the whole connected set $\Omega$.
These multi-valued functions are special since 
we always  stay within  roots during an analytic continuation. So if $z_*$
is fixed in $\Omega$ and $\alpha_1(z)$ is a local branch of one root at
$z_*$, then $T_\gamma(\alpha_1)$ is again a root of $q(w,z_*)$
whenever $\gamma$ is a closed curve a $z_*$.
Hence the family $\{T_\gamma(\alpha_1)\}$
is a finite subset of $\mathcal O(z_*)$.
Moreover, this family consists of \emph{all the root functions}
at $z_*$.
For assume the contrary. Then all the analytic continuations of
$\alpha_1$ produce $k$ many root functions at $z_*$, say
$\alpha_1,\ldots,\alpha_k$.
This $k$-tuple is permuted under all  analytic continuations along  closed
curves at $z_*$. As explained in XX
the same holds at other points in
${\bf{C}}\setminus\sigma^*$. Hence their symmetric products
become single valued and
we get a function
\[ 
B(w,z)=
\prod_{\nu=1}^{\nu=k}\, [w-\alpha_\nu(z))
\] 
which is a polynomial in $w$ with rational coefficients in
${\bf{C}}$. In the same way get a polynomial $B_1(w,z)$ using the symmetric
product over the remaining $m-k$ roots functions. But then
the irreducible polynomial $Q(w)$ in $K[w]$ has a factorisation  $B\cdot B_1$
which is a contradiction.
Hence we have the proved:
\medskip

\noindent {\bf 1.4 Proposition.} \emph{For each single root-function
$\alpha$ at a point $z_*\in{\bf{C}})\setminus \sigma^*$
it follows that $\{T_\gamma(\alpha)\}$
consists of the whole $m$-tuple
of root functions at $z^*$.}
\bigskip


\noindent
{\bf{Remark}}.
This means that in the total  sheaf space
$\mathcal {\widehat O}$
there exists a connected set $W$ contained in
$\rho^{-1}({\bf{C}}\setminus\sigma^*)$
which corresponds to the root functions. In particular the projection
$\rho\colon\, W\to {\bf{C}}\setminus\sigma^*$ 
is an $m$-sheeted covering map.


\medskip

\noindent
{\bf 1.5 Associated algebraic functions.}
We have the multi valued root functions
$\{\alpha_\nu(z)\}$ in $\Omega={\bf{C}}\setminus \sigma_*$.
If $z_0\in\Omega$ each root function is analytic in a disc $D$
centered at $z_0$ and has some   series expansions:
\[ 
\alpha_\nu(z)=\sum_{j=0}^\infty\, c_{\nu,j}(z-z_0)^j
\quad\colon\, 1\leq \nu\leq m\tag{1}
\]


\noindent
Take $\nu=1$
and start the series expansion of $\alpha_1(z)$ . If $p\geq 1$
we can write
\[ 
\alpha_1(z)= c_{1,0}+\ldots+c_{1,p-1}\cdot (z-z_0)^{p-1}+
(z-z_0)^p\cdot \beta_p(z)\tag{2}
\]


\noindent 
Here $\beta_p(z)$ is a new germ of analytic function at $z_0$ and we can
write
\[
\beta_p(z)= \frac{\alpha_1(z)- c_{1,0}+
\ldots+c_{1,p-1}\cdot (z-z_0)^{p-1}}
{(z-z_0)^p}\tag{3}
\]


\noindent
If $\gamma$ is an curve in ${\bf{C}}\setminus\sigma^*$
we obtain an analytic continuation of $\beta$ where
\[
T_\gamma(\beta_p)= \frac{T_\gamma(\alpha_1)- c_{1,0}+
\ldots+c_{1,p-1}\cdot (z-z_0)^{p-1}}
{(z-z_0)^p}\tag{3}
\]
The resulting his multi-valued extension produces the same number of
different local branches a $z_0^*$ as the germ of $\alpha_1$, i.e. the number is equal to $m$. From this we conclude that 
$\beta_p$ is a root function associated to
an irreducible polynomial $S[w]$ of degree $m$ in
$K[w]$.
\newpage

\centerline {\bf 2.  Heine's reduction.}
\medskip


\noindent
We   establish a result due to Heine
which will be used
to prove
Eisenstein's theorem. With the notations from � 1.5 we have
$\beta_p(z) $  which is analytic at $z_0$. When we perform  analytic continuations, 
$\alpha_1$ changes  and it may occur that
we arrive to a root $\alpha_j$ where
the numerator in (3) from 1.5  does not vanish up to order $p$
at $z_0$, i.e. we may encounter a pole at $z_0$ when we regard
\[
 \frac{\alpha_j(z)- c_{1,0}+
\ldots+c_{1,p-1}\cdot (z-z_0)^{p-1}}
{(z-z_0)^p}\tag{4}
\]
In fact, this will occur for \emph{every} $2\leq j\leq k$
if $p$ is large enough.
To see  this we  notice that
the $m$-tuple of germs
$\alpha_1,\ldots,\alpha_m$ at $z_0$
are \emph{linearly independent}
in the complex vector space
$\mathcal O(z_0)$. This implies that there exists some
$p^*$ such that the power series of
$\alpha_1,\ldots,\alpha_m$ at $z_0$ cannot be identical up to order
$p^*$. So when $p\geq p^*$ it follows that
we get a pole at $z_0$ for each $2\leq j\leq m$ in (4).
\medskip


\noindent
{\bf 2.1 A consequence.}
Let $p$ be so large that we get poles at $z_0$
in (4) for every $2\leq j\leq m$.
Regard analytic continuations along
closed curves $\gamma^*$ which
stay in $\Omega^*$. Then (2) implies that
\[ 
T_{\gamma^*}(\alpha_1)=
c_{1,0}+\ldots+c_{1,p-1}\cdot (z-z_0)^{p-1}+
(z-z_0)^p\cdot T_{\gamma^*}(\beta_p)\tag{5}
\]


\noindent
Choose $\gamma^*$ so that
$T_{\gamma^*}(\beta_p)$ is another root of the algebraic equation satisfied by
$\beta_p$. Let us call it $\rho$ for the moment.
Since $\rho\neq \beta_p$ in $\mathcal O(z_0^*)$
it follows that
$T_{\gamma^*}(\alpha_1)\neq\alpha_1$ and hence
$T_{\gamma^*}(\alpha_1)=\alpha_j$ for some $j\geq 2$.
So at $z_0^*$ we have
\[
\alpha_j(z)=
 c_{1,0}+\ldots+c_{1,p-1}\cdot (z-z_0)^{p-1}+\rho(z)(z-z_0)^p\tag{6}
\]
But now (4) gives a pole at $z_0$
and therefore the germ $\rho$ \emph{cannot} extend to be analytic a $z_0$.
This holds for every root except $\beta_p$. Hence we have proved
\medskip

\noindent {\bf 2.2 Proposition.}
\emph{If $S[w)$ is the irreducible polynomial in $K[w]$ which produces
the root functions generated by $\beta_p$, then 
$\beta_p$ is the only root which can be extended analytically at
$z_0$. Moreover, if $\Omega$ is a simply connected sector in
$D$ and $\rho_2,\ldots,\rho_m$ the remaining roots then their single-valued
restrictions to $\Omega$ are unbounded when $z_0$ i approached.}

\bigskip

\noindent
{\bf 2.3 Structure of $S[w]$}
To $S[w]$ we get exactly as in (0.2) a polynomial
\[
s(w,z)=
\rho^*(z)\cdot s_m(z)w^m+\ldots+s_1(z)w+s_0(z)]\tag{i}
\]
where $\beta_p(z)$ appears as one root function.
By analyticity any germ $\phi\in\mathcal O(z_0)$
which satisfies
\[
s_m(z)\cdot \phi^m(z)+\ldots+s_1(z)\cdot \phi(z)+s_0(z)=0
\]
in a neighborhood of $z_0$ yields a root function of $S[w]$.
Now  Proposition 2.2 and the result in XX shows that
we must have
\[
s_m(0)=\ldots=s_2(0)=0\tag{**}
\]
This finishes the proof of Heine's reduction

\bigskip

\noindent
\centerline {\bf 3. Proof of Eisenstein's theorem.}
\bigskip

\noindent
Let $w(z)$ satisfy an algebraic equation $P(z,w)=0$
and assume that it has a rational series expansion (0.1) at some point.
Now $P(z,w)=\sum\, c_jk z^jw^ k$
with complex coefficients. But ${\bf{C}}$ is a vector space
over the field $Q$ of rational numbers.
So regarding the finite set of complex coefficients
they can be expanded in a basis, i.e. we can write
\[ 
c_{jk}=\sum\, q_{j,k;\alpha}\cdot \xi_\alpha\quad\colon\, q_{i,j;\alpha}\in
Q\quad,\colon\,m\{\xi_\alpha\} \,\,\text{ are linearly independent
over}\,\, Q
\]
If $P_\alpha(z,w)=\sum\, q_{j,k;\alpha}(z,w)$
it follows that
$\xi_\alpha\cdot \sum \, P_\alpha(z,w)=0$. Using the rational series for
$w$ and the $Q$-linear independence of the $\xi$-numbers, we see that
$P_\alpha(z,w)=0$ for eah $\alpha$. Now we simply use that at
least one of these polynomials is not identically zero.
hence $w$ satisfies t an algebraic equation defined by a polynomial in
$Q[z,w]$.
So now $P$ has rational coefficients. Next, in (0.5) the
expansion of $\beta_p(z)$ contains all 
coefficients in the expansion in (0.1) except for $c_0,\ldots,c_{p-1}$.
if $k\geq 1$ is found so that $\beta_p(kz)$ has an integer
expansion the same holds for $w(M\cdot z)$
when $M$ is an integer such that $M\cdot c_\nu\in{\bf{Z}}$ for
each $0\leq\nu\leq p-1$.
There remains only to show that $k$ exists for
the algebraic function $\beta_p(z)$ which satisfies
an equation $Q(z,\beta_p)=0$. Here $P$ has rational coefficients
and multiplying these by some integer we can assume that the polynomial has integer 
coefficients.
To simplyify notations we set $w=\beta_p$ and we are in the favourable case
described by (*) in (0.3).
The rational expansion of $w(z)$ is given
as in (0.1). 
\[
\sum_{\nu=2}^{\nu=m}\, q_\nu(z)\cdot w(z)^\nu+
q_1(z)w(z)+q_0(z)=0\tag{i}
\]
Let $2\leq \nu\leq m$ be given. Since $q_j(0)=0$
we  have
\[
q_\nu(z)\cdot w(z)^\nu=\sum_{j\geq 1}\,
q_{j,\nu}z^j\cdot(c_0+c_1z+\ldots)^\nu=\sum\,\rho_{\nu,N}\cdot z^N\tag{ii}
\]
By assumption $q_{j,\nu}\in{\bf{Z}}$ and since $j\geq 1$ always occurs
it is clear that if $N\geq 1$ then
the $\rho$-coefficient is expressed by
\[ 
\rho_{\nu,N}=\sum_{j\geq 1}\,q_{j,\nu}z^j\cdot B_j[c_0,\ldots,c_{N-j}]
\]
where each $B_j$ is a homogeneous polynomial of degree
$N-j$.
This implies that if $k$ is a positive integer such that
$k^j\cdot c_j\in {\bf{Z}}$ hold for $0\leq j\leq N-1$, then
$k^{N-1}\cdot\rho_{\nu,N}\in{\bf{Z}}$.
This conclusion holds for every $\nu\geq 2$.
Together with (i) we get the following
\medskip

\noindent
\emph{Sublemma.}
\emph{Let $N\geq 2$ and assume let $k$
be an integer $k$ is an integer such that
$k^jc_j\in{\bf{Z}}$ hold for each $0\leq j\leq N-1$.
Then }
\[ 
k^{N-1}\cdot\rho_{0,N}\in{\bf{Z}}
\]
\medskip

\noindent
The Sublemma  enable us to cary out an induction. For first we have the zero
coefficient for $z^N$ in (i) which gives:
\[ 
q_{0,1}\cdot c_N+\sum_{j\geq 1}\, q_{j,1}c_{N-j}+\sum_{\nu\geq 2}\,\rho_{\nu,N}
+q_{N,0}=0\tag{iii}
\]
\medskip

\noindent
Here all the doubly-indexed $q$-numbers are integers.
If $k=q_{0,1}$
and the induction holds up to $N_1$ we therefore get
$k^N\cdot c_N$ and hence Eisenstein's Theorem will follow. Well, of course
we must settle the start,  i.e. we must also show that
\[ 
q_{0,1}\cdot c_1\in{\bf{Z}}
\]
But this is clear for now we only have to identify the
$z$-coefficient in (0.1) and get
\[
\sum_{\nu\geq 2}\, q_{1,\nu}\cdot c_0+
q_{0,1}\cdot c_1+q_{1,0}=0
\]
Since we also may assume that
$c_0$  is an integer we get (x) as required.
So this finishes the proof of Eisenstein's Theorem.






\newpage

\centerline{\bf 9. Extensions by reflection}
\bigskip

\noindent
{\bf Introduction.} 
\emph{Das Spiegelungsprinzip} is due to H. Schwartz.
It is frequently  used to obtain analytic continuations.
First we describe the standard case.
Let $f(z)$ be an analytic function in the upper half plane
$U_+=\{\mathfrak{Im}\,z>0\}$.
Let $J(a,b)=\{a<x<b\}$ be an interval, on  the real axis.
Suppose that $f$ extends to a continuous function to this open interval
and takes real values. In the lower half-plane $U_-$
we get the analytic function
\[  
f_*(z)=\bar f(\bar z)\tag{i}
\]

\medskip
\noindent By the result in XX the two functions are analytic
continuations of each other over
$(a,b)$. So this means that $f$ itself has an analytic extension
to the open set $\Omega={\bf{C}}\setminus J$, where
$J_*=(-\infty,a]\cup\, [b,+\infty)$
is the closed complement of $J(a,b)$ on the $x$-axis.
Next, suppose that
$e^{i\theta}f(z)$ extends to a real-valued function on
$(a,b)$ for some $\theta$. 
After multiplication
with $e^{-i\theta}$ we get an extension of $f$. That is, one has only to require
that the argument of $f$ is constant to obtain an
analytic continuation.
Suppose now that the argument of $f$ is constant  over a family of
pairwise disjoint  intervals $\{J(a_\nu,b_\nu)\}$. Then we get analytic
continuations across each interval. In particular one has: 
\bigskip

\noindent
{\bf 9.1 Theorem.}
\emph{Let $a_1<\ldots<a_N$ be a finite set of real numbers
and assume that $f$ extends to a continuous function on each of the intervals}
\[ 
J_0=(-\infty,a_1)\quad\colon \,J_\nu=(a_\nu,a_{\nu+1}\,\colon
2\leq\nu\leq N_1\quad \colon\, J_N=(a_N,+\infty)
\]
\emph{and on every such interval the argument of $f$ is some constant.
By reflection over one such  interval
we obtain an analytic
of $f_*^\nu$ to the set
${\bf{C}}\setminus (a_1,\ldots,a_N)$.}
\medskip

\noindent
{\bf Example.} In the upper half-plane $U_+$
we consider the analytic function
\[ 
f(z)=\sqrt{z}\cdot\sqrt{1-z}
\]
The single-valued branches of the root functions are chosen so that
\[ 
\sqrt{z}=\sqrt{r}\cdot e^{i\theta/2}\quad \colon\quad
\sqrt{z-1})=\sqrt{1+r^2-2r\cdot\text{cos}\,\theta}\cdot e^{i\phi}
\quad\colon z= re^{i\theta}
\] 
where 
$0<\theta<\pi$ and $\phi$ is the outer angle of the triangle in figure xx.
So here $0<\phi<\pi$ holds.
As we approach a point $0<x<1$
we get the boundary value
\[ 
f(x)=\sqrt{x}\cdot i\cdot\sqrt{1-x}
\]
Now get the analytic continuation $f_*^1$
across the interval $J_1=(0,1)$ which becomes an analytic function
defined in the lower half-plane $U_-$ by
\[
f_*^1(z)=-\bar f(\bar z)
\]
Notice that the minus-sign appears in order that
$f(x)=f_*^1(x)$ holds for $0<x<1$.
Suppose now that $x>1$. Then we get
\[ 
\lim_{y\to 0}\,f_*^1(x-iy)=\lim_{y\to 0}\,-\bar f(1(x+iy)=-\sqrt{x}\cdot\sqrt{x-1}
\]
So $f$ and $f_*^1$ do not agree on
the real interval $(1,+\infty)$. At the same time
$f_*^1$ can be continued analytically across
$(1,+\infty$ and gives 
an analytic function
$f_+{**}$ defined in $U_+$ where we obtain
\[ 
f_+^{**}(z)= -f(z)\quad\colon\quad z\in U_+
\]
Another analytic continuation of
$f_*^1$ takes place across $(-\infty,0)$.
When $x<0$ we have
\[
\lim_{y\to 0}\,f_*^1(x-iy)=
\lim_{y\to 0}\,-f(1(x+iy)=
\]









After $f$ has been extended to the lower half-plane
where we get an analytic function denoted by
$f_*$ we notice that
$f_*$ by the construction also
has boundary values with a constant argument as
we approach points on the real axis form below.
So $f_*$ also extends to the upper half-plane where we encounter
a new analytic function $f^*(z)$.
Next, we can continue $f^*$ to the lower half-plane and so on.
The result is that
$f$ from the start extends to a multi-valued function in
${\bf{C}}\setminus (a_1,\ldots,a_N)$.

\medskip

\noindent
{\bf 9.2 The use of conformal maps.}
For local extensions there exists  a  general result.
Let $D$ be an  open disc and $\gamma$ a Jordan arc
which joins two points on $\partial D$ and separates
$D\setminus\gamma$ into a pair
of disjoint Jordan domains.
Let $f$ is analytic in one of the Jordan domains, say
$D^*$. Assume also that $f$ extends to a continuous and real-valued
function on $\gamma$.
Now there exists a conformal map from $D^*$ to
the upper half-plane and using this it follows that
$f$ extends analytically across $\gamma$. of course, the
extension $f_*$ will in general only exist in a small
domain close to $\gamma$, i.e. this is governed via the
conformal mapping. But here exists at least
a locally defined  analytic continuation across $\gamma$.
\medskip

\noindent
{\bf 9.3 Boundary values on circles}
Let $f(z)$ be as above and suppose it extends continuously to
$\gamma$ where the absolute value is constant, say 1.
Using a conformal map from $D_*$ to the unit disc we may
assume that $\gamma$ is an interval of the unit disc $D$
and
$f$ is analytic in a small region $U\subset D$ where
$\gamma$ appears as a relatively open subset of
$\partial U$.
By hypothesis  $f(\gamma)$
is a subset of another unit circle and using a conformal map
from the disc bordered by this unit circle we
get the situation in 7.2. and conclude that
$f$ continues analytically across $\gamma$.
More generally, using a locally defined conformal map
there exists an analytic
extension of $f$ across $\gamma$ if we only assume that
the continuous boundary values of $f$ on
$\gamma$ are contained in some locally defined 
\emph{real-analytic curve}.
Finally, by a two-fold application of conformal mappings
we get the following
quite general result:
\medskip

\noindent
{\bf {9.4 Theorem.}}
\emph{Let $f(z)$ be analytic in a Jordan domain
$\Omega$ and suppose that
$\gamma$is an open arc of $\partial\Omega$ such that
$f$ extends continuosly from
$\Omega$ to $\Omega\cup\gamma$
and the restriction $f|\gamma$has a range
$f(\gamma)$  contained in a 
simple real-analytic curve $\gamma^*$.
Then $f$ extends analytically across $\gamma$, i.e. there exists
an open and connected neighborhood $U$ of $\gamma$ such that
the original $f$-function extends to the connected domain
$\Omega\cup U$.}
\medskip

\noindent
{\bf{Remark.}}
Theorem 9.4 follows from the fact that if
$\Omega_2$ and $\Omega_2$
are two Jordan domains whose boundaries
both are \emph{real analytic}
closed Jordan curves, then a conformal map from
$\Omega_1$ to $\Omega_2$ extends to a conformal map 
from an open neighborhood of
$\bar\Omega_1$ to an open neighborhood
of $\bar\Omega_2$.




\newpage
\centerline{\bf 10. The elliptic modular function}

\bigskip

\noindent
{\bf Introduction.}
We   construct
an analytic function
$\lambda(z)$  in
the upper half-plane
$U_+=\mathfrak{Im}\, z>0$ whose
complex derivative of $\lambda$ 
is $\neq 0$ and the image set $\lambda(U_+)$ is equal to
the connected open set
$\Omega={\bf{C}}\setminus \{0,1\}$, i.e. the two points 0 and 1 are removed from
the complex plane.


\medskip

\noindent
{\bf 8.1 An initial construction.}
Consider the simply connected
open set
\[
V_0=U_+\,\cap {\ |z-1/2|>1/2}\cap\,
\{0<\mathfrak{Re}\, z<1\}
\]

\noindent
Here $\partial V_0$ consists of three pieces. One is
the vertical line $\ell_0$ on which $x=0$ and $y>0$. The second is
the line
$\ell_1$ on which $x=1$ and $y>0$. Finally we have the half-circle
\[
T_0=\{ 1/2+1/2\cdot e^{i\theta}\}\quad\colon\quad 0<\theta<\pi
\]


\noindent
We admit Riemann's mapping theorem for simply connected
domains and find a unique  conformal mapping $\lambda_0$ from
$V_0$ onto $U_+$ when we 
require that  $\lambda_0$
yields a 1-1 map from
$\ell_0$ to $(-\infty,0)$ and after
it maps $T_0$ onto $(0,1)$ and finally $\ell_1$ is mapped onto
$(1,+\infty)$. See Figure XXX.
\medskip

\noindent
{\bf 8.2 Reflection on vertical lines.}
The boundary values along $\ell_0$
of the mapping function $\lambda_0$ stay on the real interval
$(-\infty,0)$.
We can therefore apply the reflection principle to continue
$\lambda_0$ across $\ell_0$.
To see this we consider the domain
\[
V_{-1}= V_0-\{1\}
\]
That is, points in $V_{-1}$ are of the form
$z-1$ with $z\in V_0$.
Notice that we get a 1-1 from $V_{-1}$ onto $V_0$ by:
\[ 
z\mapsto -\bar z\quad\colon\quad z\in V_{-1}
\]
By the Schwarz reflection principle we get
an analytic function in 
$V_{-1}$ defined by
\[
g(z)=\bar\lambda_0(-\bar z)
\]
Since $-\bar z=z$ hold when $z=iy$
we conclude that
$\lambda_0$ extends to an analytic function
in the domain formed by the union 
$V_0\cup V_{-1}$ and the real line where $x=0$ and $y>0$.
Since conjugate values appear in (ii) this extended function
has an image given by ${\bf{C}}\setminus [0,+\infty)$
and by the construction this map is even conformal.
\medskip

\noindent
In exactly the same way we can extend $\lambda_0$ 
across $\ell_1$,. This gives  an analytic function
defined in the union of $V_1=V_0+\{1\}$ and $V_0$ 
which yields a conformal map from this open
set onto ${\bf{C}}\setminus(-\infty,0]$.
At this stage it is clear how one can proceed.
For example, we can regard the extended $\lambda$-function in
$V_1$ and extend it across the line $\ell_2$ where $x=2$ and $y>0$.
Simiarly we get extensions to the left.
As explained by Figure xx the result after all such
reflections across
vertical lines $\ell_\nu$ we arrive at the following.
Set
\[ V_*=\cup\, \{V_0+\nu\}
\] 
where the union is taken over all integers.
Then one has:
\medskip

\noindent
{\bf 8.3 Proposition.}
\emph{After  continuation we obtain
a single valued analytic function
$\lambda_*$ defined in the domain $V_*$
whose image set is ${\bf{C}}\setminus [0.1]$.}
\medskip

\noindent
{\bf Remark.} As explained by Figure x we have for each
integer $k$ the simply connected domain
$V_k=V_0+\{k\}$.
The restriction of $\lambda_*$
to $V_k$ gives a conformal map onto
the upper half-plane if
$k$ is an even integer and the lower half-plane if $k$ is odd.
Notice also that $\lambda_*$  is periodic:
\[ 
\lambda_*(z)=\lambda_*(z+2k)\quad\text{for each
integer}\quad  k.
\]

\medskip

\noindent
{\bf 8.4 Reflections over circles.}
Let us return to the function $\lambda_0$. The semi-circle $T_0$ borders $V_0$
and is mapped by $\lambda_0$ 
to the real interval  $(0,1)$.
So again we can apply the reflection principle. Namely,
introduce a new complex variable $w$
and represent points on $T_0$ by
\[
1/2+1/2\cdot e^{i\theta} 
\]
Here $\lambda_0(1/2+w/2)$ is defined in a portion
of the exterior disc
$|w|>1$, i.e. when
$1/2+w/2\in V_0$.
Set
\[ 
g(w)=\bar\lambda(1+\frac{1}{2\bar w})
\]
Then $g(w)=\lambda_0(1/2+w/2)$ when $|w|=1$
and $g$ extends to an analytic function in
$|w|<1$ as long as
\[
1+\frac{1}{2\bar w}\in V_0\tag{*}
\]
By drawing a figure and using euclidian geometry one sees 
that (*) holds if and only if the corresponding points
$z=1/2+w/2$ stay outside the union of the two half-discs
$|z-1/4|<1/4$ and $|z-3/4|<1/4$.
The conclusion is that
$\lambda_0$ extends from $V_0$ to
the larger domain
\[
V_0(1)=\{z\,\colon 0<\mathfrak{Re}(z)<1\}\setminus
\{|z-1/4|\leq 1/4\}\cup\, \{|z-3/4|\leq1/4\}
\]
\medskip

\noindent
The extended function gives a conformal map from
to $V_0(1)$ onto the simply connected
subset of
${\bf{C}}$ where
$(-\infty,0]$ and $(1,+\infty)$ are removed.
In the same way we apply the reflection principle in
each domain $V_k$ and obtain an analytic extension to the set
$V_k(1)=\{V_0(1)+k\}$. Taking the whole union we obtain an extension of
the  $\lambda_*$-function to the domain
$V_*(1)$ which is a simply connected subset of
the upper half-plane bordered by a sequence of half-circle of radius 1/4. See
Figure XX.
\medskip

\noindent
Let us denote the extended function by $\lambda_{**}$. 
The construction goes on where we now use the refection principle for
the extension of $\lambda_{**}$ across each of the half-circles which
borders $V_*(1)$ and as a result we get an extension to a domain
$V_*(2)$ which now is bordered by a sequence of half-circles of radius $1/8$
centered at integer multiples of 1/8.
We can continue and passing to the limit we obtain an analytic function
$\lambda$ defined in the whole upper half-plane with values in
${\bf{C}}\setminus \{0,1\}$.
This proves the existence of the required $\lambda$-function.
\bigskip

\noindent
{\bf{8.5 Invariance under a  M�bius group}}
By the construction
\noindent
the $\lambda$-function  is locally conformal but not 1-1.
It turns out that this function is
\emph{invariant} under a group of M�bius tranasformations
$U_+$. 
\medskip

\noindent
{\bf 8.6 Definition.} \emph{Denote by $G$ the group of
M�bius transformations of the form}
\[
z\mapsto \frac{az+b}{cz+d}\quad\colon\quad
ad-bc=1\quad\colon a,b,c,d\in{\bf{Z}}\quad
\colon\,a,d\,\,\,\text{are even}\,\colon b,c\,\,\,\text{are odd}
\]
\medskip

\noindent
One verifies easily that
$G$ is a group and that every 
$G$-element yields a conformal map of
$U_+$ onto itself.
By scrutinizing the construction of
the $\lambda$-function one verifies that
it is invariant under $G$. More precisely,
\[ 
\lambda(z)=\lambda(g(z))\quad\colon\quad
g\in G\quad\colon z\in U_+
\]
\medskip

\noindent
This means that
$\lambda$ is an automorphic function with respect to $G$.
Moreover,  for each
point $\zeta\in{\bf{C}}\setminus \{0,1\}$
the inverse the  fiber
$\lambda^{-1}(\zeta)$ is  an
orbit for $G$. More precisely,  we  get the  inverse
fiber by the following procedure: Pick any $z_0$ with $\lambda(z_0)=\zeta_0$.
Then the map
\[ 
g\mapsto g(z_0)\quad\colon\quad g\in G
\]
yields a \emph{bijective} correspondence between the elements of the group
$G$ and the inverse fiber
$\lambda^{-1}(\zeta_0)$.
\medskip

\noindent
{\bf Remark.}
For  a further account of the
assertions above the reader may consult
[Bieberbach : page 40-45].
\medskip

\noindent
{\bf{The multi-valued inverse}}.
Since $\lambda(U_+)={\bf{C}}\setminus \{0,1\}$ and 
$\lambda$ is locally conformal we can construct a multi-valued inverse function to be denoted by
$\mathfrak{m}$.
Namely, set $\Omega={\bf{C}}\setminus \{0,1\}$ and consider the point
$\zeta_0=i$.
We first find   the  unique point $z_0\in V_0$ such that $\lambda(z_0)=i$.
At $\zeta_0$ we get a unique germ $\mathfrak{m}_0(\zeta)
\in\mathcal O(\zeta_0)$
such that

\[ 
\mathfrak{m}_0(\lambda(z))= z
\] 
hold for $z$ close to $i$.
Next, let $\gamma$ be a curve in $\Omega$ which starts at $i$ and has some end-point
$\zeta_1$.
Since $\lambda$ is locally conformal there exists a unique curve
$\gamma^*$ in $U_+$ such that
\[
\lambda(\gamma^*(t))= \lambda(t)\quad\colon\, 0\leq t\leq 1
\]
Since $\lambda$ is locally conformal it is clear that we can construct 
an analytic extension of
$\mathfrak{m}_0$ along $\gamma$ which locally produces inverses of
the $\lambda$-function.
For the resulting multi-valued $\mathfrak{m}$-function
we get the sets of values  $W(\mathfrak{m},\zeta)$ for every 
$\zeta\in\Omega$. This set is in a 1-1 correspondence
with the inverse fiber
$\lambda^{-1}(\zeta)$ and hence also a bijective
with the unimodular  group $G$ above.



 


\newpage


\centerline
{\bf {11. Poincare's theory of Fuchsian groups}}
\bigskip

\noindent
The  theory of Fuchsian groups was
created by Poincar�.
His  two articles
\emph{Th�orie des groupes fuchsiens} and
\emph{Memoire sur les fonctions
fuchsiennes} were  published 1882
in the first
first volume of Acta Mathematica and
the article \emph{Memoire sur les groupes klein�ens}
appeared in  volume III.
The last  article  is  more advanced  and we shall not discuss
Kleinan groups here. Nor do we discuss
the article \emph{Memoire sur les fonctions
z�tafuchsiennes}. The connection to arithmetic was
presented in a later article \emph{Les fonctions
fuchsiennes et l'Arithm�tique} from 1887. One should also
mention the
article
\emph{Les fonctions
fuchsiennes et l'�quation $\Delta(u)=e^u$}
where Poincar� proved that this second order differential equation has a 
subharmonic solution with prescribed
singularities on every closed Riemann surface attached to an algebraic equation.
The last work started  potential theoretic analysis
on complex manifolds.
Here we  only discuss   material
from the first two cited articles.
\medskip

\noindent 
{\bf Remark.} Poincar� was  inspired by
earlier work, foremost by Bernhard Riemann, Hermann
Schwarz and  Karl Weierstrass.
For example, he  used the construction  of multi-valued analytic
extensions by Weierstrass which leads to the
\emph{Analytische Gebilde} of a multi-valued function
$f$ defined in some connected
open subset  $\Omega$ of 
${\bf{C}}$. This \emph{Analytische Gebilde} is a connected
complex manifold $X$ on which $f$
becomes a single valued analytic function $f^*$. More precisely, there
exists a locally biholomorphic map
\[
\pi\colon\, X\mapsto\Omega
\]
When $U\subset\Omega$ is simply connected
the inverse image $\pi^{-1}(U)$ is a union of pairwise
disjoint open sets
$U^*_\gamma$ where
the single-calued analytic function $f^*$
is determined by a branch $T_\gamma$ of $f$, i.e. one has
\[
T_\gamma(f)(\pi(x)=f^*(x)\quad\colon
x\in U^*_\gamma
\]

\medskip

\noindent
Major contributions are also due to Schwarz.
In 1869 he used the reflection principle
and calculus of variation to settle the Dirichlet problem
and used this to prove  the uniformisation  theorem for
connected 
domains bordered by $p$ many real analytic and closed Jordan curves where
$p$ in general is $\geq 2$.
Of special interest is the multi-valued $\mathfrak{m}$-function
defined in ${\bf{C}}\setminus\{0,1\}$
which is related to the elliptic integral of the first kind
and hence to Jacobi's $\mathfrak{sn}$-function
which appears in the equation of motion
when a rigid body rotates around a fixed point.
\medskip

\noindent
{\bf A comment.}
The theory  of Fuchsian functions was not restricted
to analytic function theory.
The  main concern for Poincar� was to develop the
theory of  differential systems, both linear and non-linear.
His research was
also directed towards to the general theory about abelian functions and their
integrals, inspired by Abel's pioneering work.
Hundreds of text-books have appeared after
Poincar�. Personally I find that  his own and often quite
personal
presentation 
superseeds most   text-books which individually only
treat
some  fraction from his great visions.
See in particular
the book \emph{Analyse de ses travaux scientifiques}
where  Poincar�
describes his 
research areas in  the period between
1880 until 1907. It contents  has the merit that
it 
not only contains a summary of results
but also explanations of the
the main
ideas and methods which led to the theories.
\medskip

\noindent
Of course  there exists more recent advancement   in function theory.
Here one should foremost
mention work by Lars Ahlfors. So in addition
to the cited reference above I recommend text-books by
Ahlfors, especially his book  \emph{Conformal Invariants} which 
contains material about the 
theory of \emph{extremal length}
created by Arne Beurling.
From a complex analytic   point of view the
discoveries by Ahlfors and Beurling
have a wider scope and has led to
many still unsolved problems in complex analysis.
including the study of  quasi-conformal mappings.
In addition  one should also  mention the book [A-S]
by Ahlfors and Sario about Riemann surfaces.



\newpage



\centerline
{\bf 12. Remarks about Fuchsian groups.}
\medskip

\noindent
They are constructed via M�bius transforms 
which give conformal mappings
of the unit disc $D$ onto itself.
The set up is as follows: To
each $a\in D$ we have
the M�bius transform
\[ 
M_a(z)=\frac{z+a}{1+\bar a\cdot z}\tag{i}
\]
If $a,b$ is a pair of points in $D$, a  computation shows that
the composed map
\[ 
M_b\circ M_a=M_c\quad\colon\quad c=\frac{a+b}{1+\bar a\cdot b}\tag{ii}
\]
In particular $M_{-a}$ is the inverse to $M_a$ and by
(ii) 
the points in the unit disc correspond to   elements in a group denoted by
$\mathcal M$.
Let $\mathcal F$ be a subgroup of
$\mathcal M$. To each $z\in D$ we get the orbit:
\[ 
\mathcal F_z=\{ M_a(z)\quad\colon\,M_a\in\mathcal F\}\tag{*}
\]



\noindent
Following Poincar�
we say that $\mathcal F$ is a discrete Fuchsian group if every
orbit is a discrete subset of $D$. It means that
if $r<1$ then $\mathcal F_z$ only contains a finite set of points
in the disc
$D_r$ of radius $r$. Notice that if $z=0$ is the origin then
the orbit $\mathcal F_0$ corresponds to the set of of points
$a\in D$ for which the M�bius transform
$M_a$ belongs to $\mathcal F$.
\medskip

\noindent
{\bf Fundamental domains.}
Let $\mathcal F$ be a discrete Fuchsian group.
We seek open subsets
of the unit disc
where every pair of distinct points are
\emph{non-equivalent}.
To find such domains we  use the hyperbolic  distance function on the unit disc
introduced by Hermann Schwarz.


\medskip


\noindent
{\bf 12.1 Definition.} \emph{The $\delta$-distance in the unit disc
$D$ is defined by}
\[
\delta(z_1,z_2)= \frac{|z_1-z_2|}{|1-z_1\bar z_2|}\quad\colon\quad
z_1,z_2\in D
\]
\medskip

\noindent
Let us see how a
M�bius transformation affects the $\delta$-function. 
If $a\in D$ 
a computation gives:
\[
\delta(M_a(z_1),M_a(z_1))= \frac{1}{1-|a|^2}\cdot\delta(z_1,z_2)
\]

\noindent
So when $|a|\to 1$
then
the M�bius transform
$M_a$ tends to increase the
$\delta$-distance. Given a Fuchsian group $\mathcal F$ as above we set
\[ 
\mathfrak{D}=\{z\in D\quad\colon\quad
\delta(z,0)<\delta(z,a)\quad\forall \, a\in\mathcal F_0\setminus\{0\}\}\tag{**}
\]
where $\mathcal F_0$ is the orbit which contains the origin.

\medskip
\noindent
{\bf 12.2 Proposition.}
\emph{Every $\mathcal F$-orbit intersects
$\mathfrak{D}$ in at most one point.}
\medskip

\noindent
\emph{Proof.}
Assume the contrary, i.e. there exists some
$b\in \mathfrak{D}$ and $0\neq a\in\mathcal F_0$ such that
both $b$ and $M_a(b)$ belong to
$\mathfrak{D}$. Since $\mathcal F$ is a group we also have
$-a\in\mathcal F_0$. Now
\[
\delta(b,0)<\delta(b,-a)
\]
From (iii) we get
\[
\delta(M_a(b),M_a(0))<\delta (M_a(b), M_a(-a))=
\delta(M_a(b),0)
\]
This gives a contradiction since
$M_a(0)=a$ and the inclusion
$M\uuu a(b)\in\mathfrak{D}$
means that we have the opposite inequality
\[
\delta(M_a(b),0)<\delta(M_a(b),M_a(0))
\]
\bigskip


\noindent
{\bf 12.3 The boundary of $\mathfrak{D}$.}
If $z$ is a boundary point of
$\mathfrak{D}$ it follows by continuity that
there exists at least
some
$0\neq a\in\mathcal F_0$ such that
\[
\delta(z,0)=\delta(z,a)\tag{*}
\]
The converse also holds, i.e. the reader
should verify
\medskip

\noindent
{\bf 12.4 Proposition.}
\emph{The set $\partial\mathfrak{D}\,\cap D$
is equal to the set of points
$z\in D$ for which there
exists some
$0\neq a\in\mathcal F_0$
for which (*) above holds.}


\bigskip 

\noindent
{\bf 12.5 The sets $K(a,b)$.}
Proposition 12.4  suggests that we consider sets of the form:
\[
K(a,b)=\{z\in D\quad\colon\quad
\delta(z,a)=\delta(z,b)\}\quad\colon\quad a,b \in D\tag{iv}
\]
\medskip

\noindent
{\bf A.5 Proposition.}
\emph{The set $K(a,b)$ is an arc of a circle which
intersects the unit circle at a right angle.
Moreover, $a$ and $0$ 
"liegen spiegelbildlich zueinander"}.
\medskip

\noindent
\emph{Proof.}
The assertion is invariant under a M�bius transform. So
it suffices to consider the case when
$b=-a$ with $0<a<1$, i.e. the pair are real and symmetrically placed
with respect to the origin. Then
\[ 
K(a,b)=\{z\,\colon\,\frac{|z-a|}{1-az|}=\frac{z+a|}{1+az|}
\]
With $z=x+iy$ an easy computation shows that (i) holds if and only if
\[ 
4ax(1-x^2-y^2)=0
\]
It follows that $K(A,-A)\cap D$ is the 
lie segment $(-i,i)$ on the imaginary axis. It is regarded
as a circle which has a  $\perp$-intersection with
the unit circle and  the tow real points
$a$ and $-a$ are mutually reflected with each other along the
imaginary axis.
\bigskip

\noindent
{\bf A.6 Remark.}
The reader should illustrate the results above
by suitable  pictures. For example,  
describe the sets $K(a,0)$ as $a$
varies where one seeks all $z\in D  $
such that

\[
|z|= \frac{|z-a|}{1-\bar az|}
\]
Up to rotation it is enough to treat the case when
$0<a<1$ is real and positive.
\bigskip

\noindent
{\bf A.7 The favourable case.}
In most applications
$\partial\mathfrak{D}$
consists of a finite union of circular
arcs which belong to
$K(0,a)$ for a finite set of points
$0\neq a\in\mathcal F_0$.
Moreover, every such
circular
arc has end-points on the unit circle.
So on $T$ there exists  a finite set of corner points
which appear as common end-points of two circular arcs
in the boundary of
$\mathfrak{D}$.
The simplest case is the Fuchsian group which
corresponds to the fundamental group
of ${\bf{C}}\setminus \{0,1\}$ which was already
described by Schwarz after his explicit construction of the
modular function with the aid of the reflection principle.
For further examples we refer to
the cited articles by Poincar�. Of course, the reader may also consult 
text-books of more recent origin   for further examples.

\bigskip

\noindent
{\bf {A.8. Automorphic functions.}}
Let $\mathcal F$ be a discrete Fuchsian group.
An analytic function $F(z)$ in $D$ is called
$\mathcal F$-automorphic if
\medskip

\[
 F(z)=F(M_a(z))\quad\colon\quad a\in\mathcal F
\]
\medskip

\noindent
The existence of such automorphic functions
can be established  when
$\mathcal F$ is defined via
a uniformisation of a connected
open domain in
${\bf{C}}$. 
The  construction of automorphic functions for 
an arbitrary Fuchsian group and further studies about these
is a topic which goes beyond the scope of these notes,
especially since we do not prove the uniformisation theorem in its full
generality.
In addition to the  cited literature above the reader may consult
other text-books. Personally I would recommend  [Ford].





\end{document}











