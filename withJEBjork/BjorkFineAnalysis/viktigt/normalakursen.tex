\documentclass[18pt]{amsart} 

\usepackage[applemac]{inputenc}

\addtolength{\hoffset}{-18mm}
\addtolength{\textwidth}{22mm}
\addtolength{\voffset}{-18mm}
\addtolength{\textheight}{20mm}

\def\uuu{_}

\def\vvv{-}

\begin{document} 

\centerline{\bf{The diffusion equation}}

\bigskip


\noindent
Random walk governed by
a Brownian motion in continuous time
leads to the PDE:
\[
\frac{\partial p}{\partial t}=
\frac{1}{2}\cdot \frac{\partial^2p}{\partial x^2}\tag{*}
\]
where $p(x,t)$ 
is a function of two variables  Here $t$ is a time variable which
is positve.
It  appears  in many
scientific areas, both in medicine
and biology, as well as physics and chemistry.
In general a diffusion constant $\sigma$ appers while one
solves the PDE:
\[
\frac{\partial p}{\partial t}=
\frac{\sigma^2}{2}\cdot \frac{\partial^2p}{\partial x^2}\tag{0.1}
\]
To describe situations where
solutions to (0.1) are needed under various boundary
value conditions would require
a very extensive discussion. The interested reader may
turno to numerous lectures by Nobel Price winners  in medicine, chemisiry and
physics during the last century. One can  for example mention
August Krough who  received the prize in 1920
for his expianation of
the penetration of blood through
very small walls which
border flowing blood  with
tissues in the body.
He gave  mathematical expansions of these
processes  which
have been applied to improve 
medical  treatment during the last 90 years. 
A kinetic  theory used upon solutions to (*) was 
established by Einstein in 1905 and  led to
a whole scientific  area in chemisity devoted to so called colloidal solutions.
A profound and very
interactive exposition of 
 Brownian  motion
related yo microscopic
phenomena  appear
in the famous text-book \emph{Les Atom} from 1913, written by
Jean Perrin who received the Nobel Prize in 1925
for his empirical verifications of Einstein's kinetic theory.

\medskip

\noindent
From a pour mathematical point  of view, the PDE is an example of
a parabolic equation.
It is usually solved when
one assigns
initial data, i.e. the function
\[
x\mapsto p(x,0)
\]
is supposed to be known.
At the same time the state  variable $x$
varies and  in many problems the state us  confined
to
intervals, either bounded or an 
unbounded interval $(0,+\infty)$, 
while one seeks solutions so that
\[
\lim_{x\to 0}\, p(x,t)=\phi(t)
\]
hold for a given time dependent $\phi$-function.
The heat equation was
considered long time ago and
specific solutions to (*) were 
established before 1800 by d'Alembert and Lagrange.
More recent construction of transforms due to Laplace and Fourier 
give a method to solve [*).
We shall expose a few of these, and  mention that
in
the so called "Mathematics of finance" these classic
solutions have become quite popular,  especially in connection with
option-pricing where various modlels have been introduced whose analytic solutions
are found via
the clasic d'Alembert-Fourier method.
\medskip


\noindent
{\bf{A. An example.}}
Here  we seek solutions to (*)
with the two boundary  conditions
\[
\lim_{x\to 0}\, p(x,t)=0
\quad\colon\quad t>0\quad\&\quad
\lim_{t\to 0}\, p(x,t)=f(x)
\quad\colon\quad x>0\tag{i}
\]
where $f(x)$ is some function defined for
$0<x<\infty$.
To solve this one employs a transform.
Namely, let $\xi$ be a new variable  and for
each function $\phi(\xi)$ we define  $p(x,t)$ by
\[
p(x,t)= \frac{1}{\pi}\cdot \int_0^\infty\, \phi(\xi)\cdot \sin(x\xi)\cdot
e^{-2\xi^2t}\, d\xi\tag{A.1}
\]
The reader can check that $p(x,t)$ satisfies (*) and 
the first boundary vale condition in (i) holds.
To attain the second boundary value one takes  $\phi$ as the so called sine-transform of
$f$.
More precisely, put
\[
\phi(\xi)=  \int_0^\infty\, \sin(x\xi)\cdot f(x)\, dx\tag{A.2}
\]
Then one has the inversion formula 
\[
f(x)=\frac{1}{\pi}\cdot \int_0^\infty\, \sin(x\xi)\cdot \phi(\xi)\, d\xi\tag{A.3}
\]

\medskip
\noindent
{\bf{Remark.}}
Readers who are not familiar  with
the inversion formula above can 
check the case when
$f(x)=1$ for $0<x<A$ while  $f(x)=0$ for $x\geq A$. Here
$A$�is an arbitrary constant and we have 
\[
\phi(\xi)= \int_0^A\, \sin(x\xi)\, dx
=\frac{1-\cos(A\xi)}{\xi}\tag{i}
\]
When $x>A$ one has
\[
\int_0^\infty\, \sin(x\xi)\cdot \phi(\xi)\, d\xi=0\tag{ii}
\]
To show (ii) one employs one of the most importent results in
mathematics,  namely the addition formula  for the sine-function
which in particular gives
\[
\sin(x\xi)\cdot \cos (A\xi)=
\frac{1}{2}\cdot [\sin(A+x)\xi-
\frac{1}{2}\cdot [\sin (A-x)\xi\,]
\]
Now the reader can check (ii) since
the integrals
\[
\int_0^\infty\, \frac{\sin(a\xi)}{\xi}\, d\xi =\frac{\pi}{2}\quad\colon\quad
\forall\,\, a>0\tag{iii}
\]
The reader should also check that
\[
\int_0^\infty\, \sin(x\xi)\cdot \phi(\xi)\, d\xi=\pi\quad 0< x <a
\]
and conclude that the inversion formula (A.3) holds with the special $f$ above.



\medskip

\noindent
{\bf{A second remark.}}
Above we  used that
\[
\int_0^\infty\, \frac{\sin\,\xi}{\xi}\,  d\xi=\frac{\pi}{2}\tag{*}
\]
The  magic equality  is proved via a detour to
the complex domain. 
Readers familiar with 
reside calculus knows how to prove (*).
Namely, for every $\epsilon$ we put
\[
J(\epsilon)=
\int_{\epsilon}^\infty\, 
\frac{\sin\,\xi}{\xi}\,  d\xi=
\frac{1}{2i}\cdot \int_{|\xi|\geq \epsilon}\,
\frac{e^{i\xi}}{\xi}\,  d\xi
\]
Using Stokes  Theorem and properties of the complex exponential function
the last integral above is equal to
\[
\frac{1}{2i}\cdot \int_0^\pi\,
e^{i\epsilon\cdot e^{i\theta}}\,i\cdot d\theta
\]
Finally,
\[
\lim_{z\to 0}\, e^z=1\implies
\lim_{\epsilon\to 0}\, J(\epsilon)=\frac{\pi}{2}
\]
which gives  (*).



\newpage


\noindent
{\bf{The d'Alembert-Fourier formula.}} 
The solution of  the diffusion equation  when
the boundary value function $f(x)=1$ for all $x>0$ becomes
\[
p(x,t)=\frac{2}{\pi}\cdot  \int_0^\infty\, \frac{\sin (x\xi)}{\xi}\cdot e^{-2\xi^2t}\, d\xi\tag{**}
\]
One refers to (*)  as the d'Alembert-Fourier formula for the $p$-function.
It  is used extensively in "mathematics of finance". For  example 
to
evaluate European options via the Black-.Sholes equation.
The point  is  here the  probabilistic interpretation.
Namey, the construction of the Brownian motion entails that
for every fixed $x>0$, the function
$t\mapsto p(x,t)$
express the survival probability that  a particle which
starting at $x$ at time zero, stays in $\{x>0\}$ up to time $t$, i.e.  during the time interval $[0,t]$
it has  not reached the barrier $\{x=0\}$.
Various numerical studies of this special $p$-function are
found in \emph{Lab 5} to this course. 
Keping $x>0$ e consider the negative time derivative
\[
-\frac{\partial p}{\partial t}=
\frac{4}{\pi}\cdot  \int_0^\infty\, \xi\cdot \sin (x\xi)\cdot e^{-2\xi^2t}\, d\xi\tag{**}
\]
which is the frequency function for the
first hitting time along the barrier $\{x=0\}$.
Plots of (**) for fixed numerical values of $x$
can be obtained via the computer.
The frequency function has a "fat tail". In fact, already the mean-value is
infinite.
\medskip


\noindent
Many interesting
probabilistic 
result can be attained via (**).
For example, let $a>0$ be given and fr a given $T>0$ 
we consider the number $p(a,T=$ which from the above is the
probability that a Brownian path starting at $a$
stays in $\{x>0\}$ up to time $T$.
In this restricted family of
paths
one can ask  how the stage $x$ varies at time $T$.
Another problem is for example to
investigate the maximum of the stage
during the time interval
$(0,T)$ of paths from this restricted family.
The reader is invited to try
to find equations for these probability densities.
\medskip


\noindent
{\bf{The geometric case.}}
So far we have discussed the additive Brownian motion.
We can also analyze the geometric case and
in
\emph{Lab 5} there appear
some numerical studies related to
the geometric case.
Suppose for example that we start at $x=1$ at time zero.
Under a geometric Brownian motion we seek the
the probability  that the geometric Brownian path
does not reach an upper barrier $\{x=A\}$
during a time interval $(0,t)$, where
$A>1$.
It turns out that this probability, denoted by
$p^*(A,T)$ can be found for all pairs  $T>0$ and $A>1$.
Moe precisely, one regards the logarithm of the
geometric Brownian motion denoted by 
$X^*_t$ which imlies that
\[
t\mapsto \log X^*_t
\]
is the usual additive motion.
The requested probability $p^*(A,T)$
is then equal to
$q(\log A,T)$ where
$\log A>=$ is a barrier while the additive Brownian motion
starts at $x=$ at time zoo and moves
in the interval $(-\infty,\log A)$. Hence one gets
\[
p^*(A,T)=
\frac{2}{\pi}\cdot  \int_0^\infty\, \frac{\sin (\log A\cdot \xi)}{\xi}\cdot e^{-2\xi^2T}\, d\xi
\]
Equations of  this kind show that
the use of computers is essential since it is not easy
to
grasp the $p^*$-function above !













 

















 






























\newpage





\centerline{\bf{The lognormal distribution}}

\bigskip


\noindent
For each $t>0$ the normal distribution with variance $t$
ha the  frequency  function 
\[
g_t(x)= \frac{1}{\sqrt{2\pi t}}\cdot e^{-x^2/2t}
\]
To be precise one has the equality
\[
t= \int_{-\infty}^\infty\, x^2\cdot g_t(x)\, dx\tag{0.1}
\]
The reader should check (0.1) where the hint is to employ
the variable substation $x\mapsto \sqrt{y}\cdot u$.
The distribution  function becomes
\[
G_t(x)= \int_{-\infty}^x\, g_t(u)\,du
\]
It   is instructive too ploy  $g_t$ and $G_t$ for some different values of $t$,
In particular to see how they change when $t$ is small respectively large,
\medskip

\noindent
{\bf{Exercise.}}
Take $t=1$ and seek the  higher even monuments
\[
{\bf{m}}_{2k}=\frac{1}{\sqrt{2\pi}}\cdot
 \int_{-\infty}^\infty\, x^{2k}\cdot e^{-x^2/2}\, dx \quad\colon\, k\geq 2
 \]
A hint is to use partial integration and verify the recursion formula
\[
{\bf{m}}_{2k}=(2k-1){\bf{m}}_{2k-2}
\]
\medskip


\noindent
{\bf{The central limit theorem.}}
It asserts that
suitably normalized sum variables of a sequence of independent random variables converge to
the normal distribution.
The classic version was established by de Moivre in 1733 who
considered a sequence oif independent Bernoulli variables
$B_1,B_2, \ldots$ and proved that the sum variables
\[
S_N=\frac{B_1+\ldots+B_N}{\sqrt{N}}
\]
converge to the nomal distribution.
More generally, if $X_1,X_2,\ldots$ are independent random variables, each with mean value zero and a finite variances
$\{\sigma_k\}$, which by defintionis
the square root of the second moment, then
the sum variables
\[
S_N=\frac{X_1+\ldots+X_N}{\sqrt{N}}
\]
converge to the normal distribution if one first has
\[
\lim_{N\to \infty}\, \frac{\sigma_1^2+\ldots+\sigma_N^2}{N}=1
\]
and in addition the tails of the these variables satisfy the Lindeberg condition.







\bigskip


\centerline{\bf{B. The log-normal distribution.}}


\bigskip

\noindent
For each $t>0$
we have the distribution function defined by
\[
L_t(x)=  \int_{-\infty}^{\log x}\, g_t(u)\,du\quad\colon\quad x>0\tag{B.1}
\]
Its frequency function $f_t(x)$ is zero when
$x<0$ and if $x>0$ we have
\[
f_t(x)= \frac{1}{\sqrt{2\pi t}}\cdot \frac{1}{x}\cdot
e^{-(\log x)^2/2t}\tag{B.2}
\]
The reader should check that even if $1/x$ appears as a term above,
the expontential log-function
\[
x\mapsto e^{-(\log x)^2/2t}
\]
tends to zero more quite rapidly as $x\to 0$ a
which implies that
\[
\lim_{x\to 0}\, f_t(x)=0
\]
It is instructive to plot the frequency function $f_t(x)$ for some different values of $t$.
\medskip


\noindent
{\bf{Moment formulas}}.
For each real number $a>0$ we consider the moment
\[
{\bf{m}}_a(t)=\frac{1}{\sqrt{2\pi t}}\cdot
\int_0^\infty\, x^a\cdot f_t(x)\, dx
\]
The variable substation $x\mapsto e^u$ identifies  the integral by
\[
\frac{1}{\sqrt{2\pi t}}\cdot 
\int_{-\infty}^\infty\, e^{ua}\cdot e^{-u^2/2t}\, du
\]
To compute this integral one
rewrites the integrand and get
\[
{\bf{m}}_a(t)=e^{a^2t/2}\cdot
\frac{1}{\sqrt{2\pi t}}\cdot 
\int_{-\infty}^\infty\,  e^{-(u-at)^2/2t}\, du=e^{a^2t/2}
\]
We find for example the mean value
\[
{\bf{m}}_1(t)=e^{t/2}
\]
The second moment becomes
$e^{2t}$ and hence the central variance $\sigma$ is given by
\[
\sigma^2={\bf{m}}_2(t)-{\bf{m}}_1(t)^2= e^{2t}-e^t= e^t(e^t-1)
\]
\medskip

\noindent
One has also the multiplicative version of the CLT.
Namely, let $t>0$ and for each $N$ we get the random variable
\[
Q_N(t)= \prod_{k=1}^{k=[tN]}\,
(1+\frac{B_k}{\sqrt{N}})
\]
Passing to logarithms we get
\[
\log L_N(t)=\sum_{k=1}^{k=[tN]}\,
\log(1+\frac{B_k}{\sqrt{N}})\tag{i}
\]
Taylor expansion of the log-function with a remainder shows that
(i) is equal to
\[
\sum_{k=1}^{k=[tN]}\,
\frac{B_k}{\sqrt{N}}-
\frac{1}{2N} \sum_{k=1}^{k=[tN]}+ O(N^{-1/2}
\]
De Moivres theorem entiuasls that this converges to
to the normal distribution with
variance $t$ minus $t/2$.
It follows that
\[
\lim _{N\to \infty}\, Q_N(t)=e^{-t/2}\cdot L(t)
\]
The right hand side is a log-normal distribution whose mean-value is zero.
\medskip



\noindent
Given a positive real number
$\mu$ we can also consider the sequence of random variables
\[
Q^\mu_N(t)== \prod_{k=1}^{k=[tN]}\,
(1+\frac{\mu}{N}+ \frac{B_k}{\sqrt{N}})
\]
The reader should confirm that
\[
\lim _{N\to \infty}\, Q^\mu_N(t)=e^{\mu t-t/2}\cdot L(t)
\]
So this time the mean-values
tend to $e^{\mu t}$.




















\newpage







\noindent
{\bf{Remark.}}
The PDE above implies that
the stochastic process in continuous time
with frequency functions $\{f_t\}$
is a diffusion process.
More precisely, recall that if
$\{X_t\}$�is a stochastic process with white  noise as 
increment, i.e.  satisfying  the stochastic PDE:
'\[
dX= a(t,X)\cdot dt+ b(t,X)\cdot dW
\]
then the frequency functions
$\{f_t(x)\}$ satisfy the PDE:
\[
\partial_t(f)=\partial_x(af)+\frac{1}{2} \cdot \partial_x^2(b^2\cdot f(x,t))
\]
So via (v) one has $a=x/2$ and $b=x$, i.e.the stochastic PDE becomes
\[
dX=\frac{X}{2}\cdot dt+ XdW
\]
Introducing the mean values $E(X_t)$  gives 
the ODE;
\[
\frac{d}{dt}(E(X_t)= \frac{X_t}{2}\implies
E(X_t)= e^{t/2}
\]
which is in accordance with the
mean-value formula found in (0.4)
\bigskip


\noindent
The solution to the parabolic PDE in (v)
expressed by the function $f(x,t)$ has been determined
via the initial condition that
$X_0=1$, i.e. 
\[
\lim_{t\to 0} f_t(x)= \delta(0)
\]
where $\delta(0)$ is the nit point mass at $x=0$.
In general one can consider an initial  condition expressed by
a frequency function $\phi_0(x)$
and then the PDE in (v) has the solution
\[
\phi(x,t)= \int\, f(x-y,t)\cdot \phi(y)\, dy\quad\colon\, t>0
\]
It is instructive to plot such $\phi$-functions
when $\phi_0(x)$ is specified.
Take as an example the case when
$\phi_0(x)= a/2$ on an interval $[-a,a]$ while
$\phi_0(x)=$ for $x|\geq a$.
Since we have found a closed formula for
the frequency functions $f_t(x)$ it is possible to
get accurate plots of the frequency functions
$\phi_t(x)$ when one assigns a numerical value $t>0$ 
Let us also remark that one can pass
to vector-valued PDE:s and also consider cases of
several $x$-variables. With $n=2$
one takes $x=(x_1,x_2)$ and  regards the PDE:
\[
\partial_t(f)= \frac{1}{2}\cdot \Delta(|x|^2\dot f)
\]
where $|x|^2=x_1^2+x_2^2$ and $\Delta$ is the Laplace  operator.
With initial condition $f(0,x)= \delta(0)$ the solution
is a radial function of $x$ which therefore admits a
plot when $t>9$.
The reader is invited to
obtain this, where the hint is to
express $\Delta$ in polar coordinates
which reduces to a second order operator in
$r$ with $r=|x|\geq 0$.
Of course one can try to solve more involved PDE:s.
Conider as an example with $n=1$ the equation
\[
\partial_t(f)= \frac{1}{2}\cdot \partial_x^2(x^4f)
\]
So here the stochastic PDE becomes
\[
dX= X^2\cdot dW\tag{i}
\]
In � xx we
use the Central Limit Theorem to
approximate this by a stochastic
process in discrete time
where Monte Carlo simulations provide approximate solutions.
Notice that the mean-values $E(X_t)=1$ for all $t>0$, i.e.
no drift appears.













 































\newpage



\centerline{\bf{The Central Limit Theorerm}}

\bigskip


\noindent
Let $\{f_n(x)\}$ be  sequence of
frequency functions, i.e. each function is non-negative and
\[
\int\, f_n(x)\, dx=1
\]
hold for every $n$,
We get the distribution functions
\[
F_n(x)= \int_{-\infty}^x\, f_n(s)\,ds
\]
Next, consider the Gaussian density function
\[
g(x)= \frac{1}{\sqrt{2\pi}} \cdot e^{-x^2/2}
\]
and let $G(x)$ be its distribution function.
We shall find a sufficient condition in order that
\[
\lim_{n\to\infty}\, \max_x\, |F_n(x)-G(x)|=0\tag{*}
\]
To achieve this we introduce the Fourier transforms
\[
\widehat{f_n}(\xi)=
\int\, e^{-ix\xi}\cdot f(x)\,dx
\]
and recall that the Fourier transform
\[
\widehat{g}(xi)= e^{-\xi^2/2}
\]
\medskip

\noindent
{\bf{Theorem.}}
\emph{Assume that for every $A>0$ one has}
\[
\lim_{n\to\infty}\, \int_{-A}^A\, |\widehat{f}_n(\xi)-e^{-\xi^2(2}|\, d\xi=0
\]
\emph{Then (*) follows.}
\bigskip

\noindent
First we prove Theirem A under the extra assumption that
there exists a constant $C$ such that
\[
|\widehat{f}_n(\xi)|\leq \frac{C}{\xi^2+1}\tag{1}
\]
hold for every $n$.
Name�ly when (1) holds and $A>=$, then Fouier's inversion formula
and the triangle inequality give
the inequality  below for every $x$:
\[
|f_n(x)-g(x)|\leq \frac{1}{2\pi}\cdot \int_{-A}^A\, |\widehat{f}_n(\xi)-e^{-\xi^2(2}|\, d\xi+
\frac{1}{2\pi}\cdot \int_{|\xi|>A}\,(|\widehat{f}_n(\xi)|+e^{-\xi^2(2})\, d\xi
\]
By (1) the last term is majored by
\[
\frac{1}{2\pi}\cdot \frac{2C}{A}+
\frac{1}{2\pi}\cdot \int_{|\xi|>A}\,e^{-\xi^2(2})\, d\xi\leq \frac{1}{2\pi A}\cdot (2C+2e^{-A^2/2})
\]
In particular the last term tends to zero as
$A\to +\infty$  and then it is clear that
(*) entails that
\[
\lim_{n\to \infty} \, \max_x |f_n(x)-g(x)|=0
\]
Thus, the sequence of frequency functions converge uniformly to
the Guassain density. Passing to the distribution functions ne easily checks that
(ii) gives true requested limit in Theorem A.,

\medskip

\noindent
{\bf{The use of convolutions.}}
There remains to see that
chart the specie case when (x) holds suffices to obtain Theorem A.,
The idea its to use suitable bubble functions. More preciously, for every
$\delta>0$ we find a non-negative $C^\infty$-function $\phi_\delta(x)$ 
with support in
$[-\delta,\delta]$ and the integral
$\int\, \phi_\delta(x)\, dx=1$.
Put
\[
f^\delta _n(x)=\int\, \phi_\delta(x-y)\cdot f_n(y)\, dy
\]
which gives 
\[
\widehat{f^\delta}_n(\xi)= \hat{f}_n(\xi)\cdot \hat{\phi}_\delta(\xi)
\]
Since the function
$\xi\mapsto \hat{\phi}_\delta(\xi)$ is rapifly decreasing we find in
particular a constant $C(\delta)$ such that
\[
|\hat{f}^\delta _n(\xi)|\leq \frac{C(\delta)}{\xi^2+1}
\]
Next, we also get the convoilutions
\[
g^\delta(x)= \phi_\delta)*g(x)
\]
where
\[
\widehat{g^\delta}(\xi)=
\hat{\phi_\delta}(\xi)\cdot e^{-\xi^2/2}
\]
From this we see that the special case applies and hence
\[
\lim_{n\to \infty}\, \max_x\, |F_n^\delta(x)-G^\delta(x)|=0
\]
Here (zz) hold for every $�delta>0$. Now we use that
$\phi_\delta$ is supported by $[-\delta,\delta]$ which gives
\[
F_n^\delta(x-\delta)\leq
F_n(x)\leq F_n^\delta(x+\delta)
\] 
for all $x$ sand every $n\geq 1$, and a similar inequality hold when we take
$G$.
So for every $x$ one has
\[
F_n(x)-G(x)\leq F^\delta_n(x+\delta)-G(x)\leq
|F_n^\delta(x+\delta)-G(x+\delta)|+ G(x+\delta)-G(x)
\]
and similarly
\[
F_n(x)-G(x)\geq F^\delta_n(x-\delta)-G(x)\geq
|F_n^\delta(x-\delta)-G(x-\delta)|+ G(x-\delta)-G(x)
\]
Finally, it is clear that
\[
\lim_{\delta|\to 0}\max_x\, \,G(x+\delta)-G(x-\delta)= 0
\]
and taking arbitrary  small,l $\delta$ we get the requested limit in Theorem A.
\bigskip


\noindent
{\bf{Applications.}}
Let $X_1,X_2,\ldots $ be a sequence oif ubsepoebdebt rabdoim variuabkes where
each $X_:k$ has mean value zero band a finite variance
$\sigma_k$, i.e.
\[
\int\, x^2\cdot f_k(x)\,dx=\sigma_k^2
\]
hold for every $k$.
To each $N\geq 1$ we get the sum variable
\[
S_N= \frac{X_1+\ldots +X_N}{\sqrt{N}}
\]
Let $\phi_N(x)$ be its frequency function.
Then we have
\[
\hat{\phi}_N(\xi)= \prod_{k=1}^{k=N}\, \widehat{f}_n(\xi/\sqrt{N})
\]
Theorem A entails that
$\{S_N\}$ converge to the normal distributions  in
the sense of (*), provided that
the integrals from Theorem A tend to zero for each fixed $A>0$.
A necessarty condiot is firt thst
one has a limit for the variance, i.e.
\[
\lim_{N\to \infty}, \frac{\sigma_1^2+\ldots \sigma_N^2}{N}=1
\]
must hold. The question arises if this condition also is sufficient.
Examples show that this is not so.
More precisely one
must impose a further condition which wads
introduced  Lindeberg in a famous work from
1919
which gives an almost optimal version of the
Central Limit Theorem.
We shall discuss Lindeberg's theorem in � x. But let
us first porove a classic versdioln of the CLt where
we impose the ext ran condition that
the 3rd-moments are
bounded by a fixed constant ,i.e, suppode that
yjrtr rcidyd as consdtsnt $C$  such that 
\[
\int\, |x|^3\cdot f_n(x)\, dx\leq C
\]
hold for every $n$.
Now we prove that (xx) implies that
the
convergence criterion from Theorerm A is valid.
Namely, with a fixed $A>0$
we notice that
$\xi/\sqrt{N}$ stays close to zero when
$-A\leq\xi\leq A$.
Hence we can use a ixck Taykoir exobasion fro
the funcvtyions $\widehat{f}_k(\xi)$.
More precisely, (xx) imply that
they are of class $C^3$ and
the maximum norms of the
derivsties up tom order
3 are uniofrmula bounded.
So when $N$ is large one has
via Tayklor�s formula with a rest-term
\[
\widehat{f}_k(\xi/\sqrt{N})= -\frac{\sigma_k^2}{2N}+
\rho_k(\xi)\cdot N^{-3/2}
\]
where there exists a constant $C(A))$ such that 
\[
|\rho_k(\xi)|\leq \frac{C(A)}{N^{3/2}}\quad\colon\quad -A\leq \xi\leq A
\]
for every $k$.
From (x-xx) and (x) it follows that
\[ 
\sum_{k=1}^{k=N}\, \log\, f_k(\xi/\sqrt{N})\to -\xi/2
\]
holds uniformly on the interval $[-A,A]$.
Passoing to the exponent ails we get
\[
\lim_{N\to \infty}\,
\widehat{\phi}_N(\xi)-e^{-\xi^2/2}=0
\]
holds uniformly in the interval $[-A,A]$.
This gives  in particular  (*) and hence Theorem A entails that
the random variables $\{S_N\}$ converge to the normal distribution in
the
strong sense expressed by (A.1).
\bigskip


\noindent
The CLT holds in particular when
$\{X_k\}$ are independent  Bernoulli variables
In this special case one cash also 
get a quite sharp upper bound for rthe rate of convergence.
More precisely, there exists a constant $C$such that
\[
\max_x\, |\Phi _N(x)- G(x)|\leq \frac{C}{\sqrt{N}}
\]
where $\{\Phi_N\}$ are rthe distribution funtyions of
the sum variables in
(x) when each $X_l$ is a Bernoulii variable.
let us remark that
estimates for $C$ were established by Gustav Essen in his 
doctor's  thesis  from Uppsala in 1942, with
Arne Beurllng as adviser.
With the power of today's  computer one can perform very extensive
Monte Carlo simulations and
in this way it is tempting to try to
predict some
value of the constant $C_*$ for which
(*) hold for very  large $N$.




































\newpage




\centerline{\bf{An optimization problem}}

\bigskip


\noindent
Let $[0,T]$ be a time interval
and $U$ is a utility function, i.e. it is defined
for $x\geq 0$ and striclty increasing and stricly concave.
If an invidual has a capital $K_0$ st time
$t=0$ and $t\mapsto c(t)$ is the rate of cunsuyption, then
t
we get the function $K(t)$ which satisfies the ODE:

\[
\dot K(t)= r\cdot K(t)-c(t)
\]
where $r>0$ is an interest rate.
We seek a $c$-function in order to maximise
\[
\int_0^T\, U(c(t))\, dt+e^{-rT}K(T)\tag{*}
\]
Above the last term is the discounted capital at time
$T$, i.e. it appears as a salvage value.
\medskip


\noindent
{\bf{Solution}}
The maximum is attained when
the time  dependent $c$-function satisfies  the equation
\[
U'(c(t))= e^{-rt}\quad\colon\quad 0\leq t\leq T(i)
\]
\medskip

\noindent
{\bf{Example.}}
Tajke
\[
U(c)=2A\cdot \sqrt{c}
\]
where $A$ is a positive constant.
Then (i) holds with
\[
c(t)=A^2\cdot e^{2rt}
\]
\medskip


\noindent
{\bf{Exerixe.}}
Find the maximum in (*) with $8t9$ given as above.
Next, in this solution it may occur thst
$K(T)<0$. This leads to a restriited optiimization problem
where one in addition resuires thst
$K(T)\geq 0$.
For example, the solution above gives
$k(T)<0$ if $K_0$ is rsther small.
The reader should find the solution to the OCT
in this restricted case.
Next, we cn also discount utiliity and consider ome
positive constant $\mu$ and seek the maximum of
\[
\int_0^T\, e^{-\mu t}\cdot U(c(t))\, dt+e^{-rT}K(T)\tag{**}
\]
Tge reader is invited to determine the optimal consmuption
$c(t)$.
Let us finally remark that
we later on will consider a stochastic situstion where the chnge of
$K(t)$ is not only governed by a fixed interest rate, but
may also be disposed in a risky asset.
In this stochastic optimization one seeks a strsgey of time dependent consumption over
the time
interval $[0,T]$ in order to maximise utility  plus the salvage value at
$t=T$.
Of course, here the analysis becomes
more involved.















\newpage








\centerline{\bf{About linear ODE:s}}

\bigskip


\noindent
We shall discuss linear ODE:s of order one or two.
The independent variable is denoted by $t$ and
in appplications  regarded as a time variable.
A first example is the exponential function
$y(t)= e^t$ which solves the ODE:
\[
\dot y= y\tag{*}
\]
where $\dot y$ denotes the time derivative
$\frac{dy}{dt}$.
It is  important to grasp  solution to
(*).
Without knowing about the expoential funtion one  solves (*) by a series expansion.
Namely, suppose thst $y(t)$ solves (*) with
the initial condition $y(0)=1$ and where the function has a series expansion
\[
y(t)=1+c_1t+c_2t^2+\ldots
\]
Since the time derivative of $t^k$ is $kt^{k-1}$ for every positive integer
we see that (*) gives the recursive fomulas:
\[
k\cdot c_k=c_{k-1}\quad\colon\, k\geq
1
\]
By an induction over $k$ this gives
\[
c_k= \frac{1}{k !}
\] which mesns that
the solution has the series expansion
\[
y(t)= 1+\sum_{k=1}^\infty\frac{t^k}{k !}\tag{0.1}
\]
This series represents  by definition the exponential  function
$e^t$m where $e$ is Neper's constant  found by 

\[
e= \lim_{n\to \infty}\, (1+1/n)^n\tag{0.2}
\]
\medskip


\noindent
{\bf{Remark.}}
Let us briefly recall  how Isaac Newton and his collegue John Wallis at
Oxford University
perceived the result above around 1680.
First  one defines a function
$L(x)$ when $x\geq 1$ by
\[
L(x)=\int_1^x\, \frac{ds}{s}
\]
Thus, following Newton we have taken the primitive function of
$1/x$ when $x\geq 1$.
Notice that $L(1)=0$
and the $L$-function is strictly increasing with the derivative
\[
L'(x)= \frac{1}{x}
\]
Next, let $x$ snd $y$ be $>1$.
Then
\[
L(xy)=\int_1^x\, \frac{ds}{s}+
\int_x^{xy}\, \frac{ds}{s}
\]
By a variable substituion the reader checks that the last integral is equal to
$L(y)$, Hence
\[
L(xy)=L(x)+L(y)\tag{i}
\]
hold for every pair $x,y>1$.
Next, one verifies easily that
\[
\lim_{x\to \infty}\, 
\int_1^x\, \frac{ds}{s}=+\infty
\]
This means that the range  of the $L$-function is
$[0,+\infty)$�and since it is strictly increasing we find
an inverse function denoted by $E$. So here
$[0,+\infty)$ is the domain of definition for
$E$ and
\[
E(L(x))= x\quad\colon\, x\geq 1\tag{ii}
\]
The $e$-number will   now be defined by
the equation
\[
1= \int_1^e\, \frac{ds}{s}=L(e)\tag{iii}
\]
Next, using (i) and (ii) the reader can check that
the $E$-function satisfies
\[
E(t+s)= E(t)E(s)\tag{iv}
\]
for every pair of non-negative resl numbers.
Keeping $t$ fixed while $s\to 0$
it follows that
\[
\dot E(t)= \lim_{s\to 0}\frac{E(t+s)-E(t)}{(s}=
E(t)\cdot  \lim_{s\to 0}\frac{E(s)-1}{(s}
\]
From the construction of $E$ via (ii) the reader can finally check that
\[
 \lim_{s\to 0}\frac{E(s)-1}{s}=
 \lim_{x\to 1}\,\frac{1}{1-x}\cdot \int_1^x\, \frac{ds}{s}=1
 \]
 Hence the $E$-function satsfies the ODE from (*)-
 \medskip


\noindent
Finally, from the construction of $e$ above
and the functional equation for the $E$-function expressed via
(iv) the reader should confirm that
\[
E(t)= e^t
\]
where the right hand side as a function of $t$
is found via  rules for exponential  of real numbers
with $e$ kept as a base while the
exponentials  are constructed for positive values of
$t$.
\medskip


\noindent
{\bf{Exercise.}}
Show via formulas for binomial expansions that
when
we define the function $y(t)$ via (0.1), then it satisfies the functional equation
\[
y(t+s)= y(t)y(s)
\]
\medskip


\noindent
{\bf{Exercise.}}
When $x\geq 1$ one puts
\[
\log(x)=L(x)
\]
i.e. the log-function is the primitive of $1/x$.
Show that $\log x$ has a series expansion
\[
\log (1+s)= s-s^2/2+s^3/3-s^4/4+\ldots
\]
which converges when $0<s<1$.
One can also define the log-function when
$0<x<$ by the equation
\[
\log x= -\log( x^{-1})
\]
Show that when  $0<s<1$ this gives
the series expansion
\[
\log (1-s)=-[ s+s^2/2+s^3/3+s^4/4+\ldots
\]
\medskip


\noindent
{\bf{Exercise.}}
Use the above to show thst
with $e$ defined via (iii)n above it is equal to Neper's limit in (0.2).
\bigskip


\centerline{\bf{1. First order ODE:s}}
\bigskip

\noindent
Let $a(t)$ be a real-valued function defined for
$t\geq 0$.
Now one seeks a function $y(t)$ where
$y(0)=1$ and
\[
\dot y= a\cdot y\tag{1.1}
\]
Suppose thst $a(t)$ has a series expansion
\[
a(t)= a_0+a_1t+a_2t^2+\ldots
\]
An example is when the series is finite, i.e. when
$a$ is a polynomial.
Now $y$ is found via a series
\[ 
y(t)= 1+c_1t+c_2t^2+\ldots
\]
Namely, the reader can check that
the $c$-coefficients satisfy the reursive equations
\[
k\cdot c_k=a_{k-1}+ a_{k-2}c_1+\ldots +a_0\cdot c_{k-1}\tag{1.2}
\]
Here the $c$-numbers become uniquely determined
which means that the solution to
the ODE in (1.1) is unique.
\medskip

\noindent
One can solve (1.1) in another way using
exponential functions.
Namely, if we seek $y$ in the form
\[
y(t)= e^{f(t)}
\] 
then (1.1) means that
\[
\dot f= a
\]
i.e. $f$ is a primitive function of
$a$. With the inital condition $y(0)=1$ this means that we take
the primitive
function
\[ 
A(t)= a_0\cdot t+a_2\cdot t^2/2+\ldots
\]
and the requested $y$-solution becomes
\[
y(t)= e^{A(t)}
\]
\bigskip


\noindent
{\bf{1.2 The equation $\dot y= af+b$.}}
Let $b(t)$ be another function of $t$
defined for $t\geq 0$.
One seeks $y$ which solves the equation above with
initial condition $y(0)=0$.
With $A(t)$ as above we try a solution of the form
\[
y(t)= e^{A(t)}\cdot f(t)
\]
Then
\[
\dot y= a\cdot y+ e^{A(T)}\cdot \dot
f
\]f
Hence $y$ gives the requested solution if
\[
e^{A(T)}\cdot \dot f= b
\]
We conclude that
a solution to the inhomogenous equation
is given by
\[
y(t) =e^{A(t)}\cdot \int_0^t\, e^{-A(s)}\cdot b(s)\, ds
\]


\bigskip


\centerline{\bf{2. Second order ODE:s}}
\bigskip

\noindent
Given a pair  of time dependent functions
$a,b$ we seek $y(t)$ which satisifes
\[
\ddot y=a\dot y+ by\tag{2.1}
\]
To solve (2.1) one proceeds as follows.
First we regard a first order system where one seeks a pair 
$f,g$ so that
\[
\dot f=g\quad\&\quad \dot g= ag+bf\tag{2.2}
\]
The reader can check that if $f,g$ satisfy the system
(2,2) then
$f$ alone satisfies (2.1).
Now the idea is to employ $2\times 2$-matrices
and get solutions to (2.2) via series expansions exactly
as 
for first order ODE:s.
Namely, let us introduce the $t$-dependent $2\times 2$-matrices
\[
A(t)= 
\begin{pmatrix}
0&1\\a&b
\end{pmatrix}
\]
Now we seek a $t$-dependent $2\times 2$-matrix
$\phi(t)$ where $\Phi(0)= E_2$ is the identity matrix
and
\[
\dot \Phi= A(t)\dot \Phi(t)
\]
We can find $\phi$ via
a reursion. Namely, expanding the functions
$a,b,c$ as a series of $t$ we can write
\[
A(t)=A_0+tA_1+t^2A_2+\ldots
\]
where $\{A_k\}$ are constant  $2\times 2$-matrices.
Then $\phi$ is found by a  series
\[
\Phi(t)= E_2+t\cdot \Phi_1+t2\cdot \Phi_2+\ldots
\]
where
the constant  matrices
$\{\Phi_k\}$ satisfy
\[
k\dot \Phi_k=A_{k-1}+ A_{k-2}\Phi_1+\ldots+A_0\Phi_{k-1}
\]
\medskip

\noindent
{\bf{Remark.}}
One refers to $\Phi(t)$ as the fundamental  solution to
the systrm defined via the matrix-valued function
$A(t)$.
\medskip


\noindent
{\bf{Exercise.}}
Introducing the elements of the $\Phi$-matrix we get four time dependent functions
where
\[
\Phi(t)= \begin{pmatrix}
\phi_{11}&\phi_{12}\\
\phi_{21}&\phi_{22}
\end{pmatrix}
\]
\medskip


\noindent
Show that the  function
\[
y_1(t)= \phi_{11}(t)
\]
satisfies  (2.1) where $y_1$ satisfies the  initial
conditions
\[
y_1(0)= 1\quad\&\quad \dot y(0)=0
\]
Show also that the function
\[
y_2(t)= \phi_{12}(t)
\]
satisfies  (2.1) where $y$ satisfies the  initial
conditions
\[
y_2(0)= 0\quad\&\quad \dot y_2(0)=1
\]
These two functions constitute a base for the 2-dimensional solution space to
the ODE in (2,1) where
a general solution $y$ with intial conditions
$y(0)=\alpha$ and $\dot y(0)=\beta$ is found
by
\[
y=\alpha\cdot y_1+\beta\cdot y_2
\]

\medskip

\noindent
{\bf{3. Inhomogeneous equations.}}
Given a third function $c$ we seek $y$ so that
\[
\ddot y= a\dot y+by+c\tag{3.1}
\]
It turns out that (3.1)
can be solved by a similar procedure as for 1st order ODE:s.
To achieve this one
first proves thst
the time dependent fundamental solution
give invertible matrices for every $t$.
To see this we use the product formula for
determinants. This implies that
\[
\frac{d}{dt}(\det \Phi)=\det A\cdot \det \Phi
\]
So the real-vslued function defined by
$g(t)=\det\Phi(t)$ satisfies the ODE:
\[
\dot g= \det A\cdot g
\]
Recall that $\phi_2(0)= E_2$ which gives
$\det\phi(O)=1$ and hence $g(0)=1$. It follws that

\[
\det\phi(t)= e^{\int_0^t\, \det A(s)\, ds}
\]
hold for every $t$. So every determinant is $\neq 0$
and hence the matrices $\Phi(t)$ are invertible.
\bigskip


\noindent
Now we can apply the inverse $2\times 2$-matrices
$\Phi^{-1}(t)$ to time dependent 2-vectors.
in this way we
get for example a vector valued funvtion defined by
\[
t\mapsto V(t)= \int_0^t\, \Phi^{-1}(s)\begin{pmatrix} c(s)\\0\end{pmatrix}
\, ds
\]
Now we  apply $\Phi(t)$ to this vector-valued function
and regsrd
\[
t\mapsto W(t) \Phi(t)(V(t))
\]
Rules for differentation give
\[
\frac{dW}{dt}= 
\dot  \Phi(t)(V(t))+\Phi(t)(\dot V(t))=
A(t)\Phi(t)+ \begin{pmatrix} c(t)\\0\end{pmatrix}
\]
\medskip

\noindent
{\bf{Exericse.}}
Use the above to express a solution to
(3.1) which satisfies $y(0)= \dot y(0)=0$ where
the two  solutions $y_1(t)$ and $y_2(t)$ can be used.
\medskip

\noindent
{\bf{A special case.}}
Cnosder the ODE
\[
\ddot y+y=c(t)
\]
where $c(t)$ is a givn function.
With $c=0$ the teo homogeneous solutions $y_1$ and $y_2$
are known, i.e. the reader can check that
\[ 
y_1(t)=\cos t\quad\&\quad y_2(t)=\sin t
\]
To solve (x) we should firdt find the fundamental solution
$\phi(t)$. To begin with we notice that the matrix $A(t)$ from (xx) 
is constant and becomes
\[
A(t)=A= \begin{pmatrix}0&1\\-1&0 \end{pmatrix}
\]
Now $\Phi(t)$ is given by
rhe exponetial  matrix-valued function
\[
t\mapsto e^{tA}= E_2+tA+t^2A^2/2 !+\ldots
\]
In order to evaluate the sum above we first notice that
\[
A^2=-E_2
\]
i.e. minus the identy matrix.
Recalling the poer series formulas for the cosine- and the sine funvtion
the reader can check thst
\[
\Phi(t)= \begin{pmatrix}\cos t&-\sin t  \\\sin t&\cos t \end{pmatrix}
\]
\medskip

\noindent
{\bf{Exercise.}} Compute the inverse matrix
$\phi^{1}(t)$ and use this to
find the solution to the inhomogeneous equation
(xx) where
$y(0)=\dot y(0)=0$.
















\bigskip


\centerline{\bf{Some further examples.}}

\bigskip


\noindent
Of course one can try to solve a second order ODE directly.
Let us give some examples. Consider the ODE:
\[
\ddot y= ty\tag{i}
\]
and seek a solution $y(t)$ with initial conditions
$y(0)=0$ and $\ddot y(0)=1$.
To find $y(t)$ we consider a power series expansion
\[
y(t)= t+c_2t^2+c_3t^3+\ldots
\]
Then (i) holds if the $c$-numbers satisfy
the recursive equations
\[
k(k-1)c_k=c_{k-3}\quad\colon\quad k\geq 3
\]
while the reader csn check that $c_2=c_3=0$.
We
find for example
\[
c_4=\frac{1}{2}\quad\&\quad c_7= \frac{1}{42} c_4
\]
and so on. The solution $y(t)4�is not expressed by
an "elementary function", i.e. not realised by
thod which occur in ordinary calculus where
a list of special funtions
appesr for historic  reasons.
Without knowing the exact doluion one csn st least
derive propeties of $y(t) from the ODE in (i). For example,
since $\dot y(0)=1$ it follows that
$\ddot y(t)>0$ and hence $y(t)$ is a strictly increasing and convex
funvtion of $t$. Using a computer one csn plot
$y(t)$ over an interval $[0,T]$ snd confirm this observation.
\medskip


\noindent
{\bf{Back-shooting.}}
Above one gets the eries soluion from the
two initial conditions
$y(0)=$ snd $\dot y(0)=1$.
Suppose now that we only impose  $y(0)=0$ snd
instead of prescribing $\dot y(0)$ we
fix some $T>0$ and require  thst
$y(T)=A$.
So on the  interval $[0,T]$ we prescribe end-values of 
the $y$-function.
It turns out that there exists a unique positive number
$a$ with
$\dot y(0)=a$ for which
$y(T)=A$.
To find - or st least approximate $a$ numerically i not
an easy affair.
Using a computer one can employ experiments to
approach the  solution. Namely, start with
some initial value $\alpha$ for $\dot y(0)=\alpha$ and plot this solution
and inspect if the value at $t=T$ is $>$�or $<$ than $A$. If it is
$<A$ one trie a new initial  value for $\dot y(0)$ which is larger than
$\alpha$, and so on.
In this way it is always possible to
attsin a solution where $y(T)$ is very close to $A$.


\newpage



\centerline{\bf{Non-linear ODE:s}}

\bigskip

\noindent
An example is the first order ODE:
\[
\dot y= y^2
\]
Even  though the right hand side is a polynomial of $y$, the
solution has a finite life-span. More precisely, let us find 
the  solution defined for $t\geq 0$ with initial  value $y(0)=1$.
The ODE csn be written as
\[
\frac{dy}{y^2}= dt\implies
-\frac{1}{y}= t+C
\] 
for some constant $C$.
With $y(0)=1$ it follows that
\[
y(t)=\frac{1}{1-t}
\]
This solution explodes when
$t\to 1$, for then $y(t)\to +\infty$.
More generally, consider an ODE of the form
\[
\dot y= P(y)
\]
where $P(y)$ is a polynomial of $y$
which is $>0$ when $y\geq 0$.
An example could  be
\[
\dot y= (y+1)(y+2)\tag{0.1}
\]
with  initial value $y(0)=0$.
Via numerical experiments on a computer one csn
approximate the life-span, i.e find the postive  number
$t^*$ for which $y(t)$ exists on
$[0,t^*)$ while $y(t)\to+\infty$ as $t\to t^*$.
At the same time one can solve the ODE. Namely,  use the
algebraic identity
\[
\frac{1}{(y+1)(y+2)}=
\frac{1}{y+1}-\frac{1}{y+2}
\]
Hence (0.2) has a
solution 
\[
\log (1+y)-\log(2+y)= t+C
\]
for a constant $C$.
Since $y(0)=0$ we get $C=-\log 2$
and get
\[
\log\frac{1+y}{2+y}= t-\log 2\implies t^*= \log 2
\]
It is instructive to check this theoretical discovery
via the computer  where plots approximate the life-span $t^*$.
\medskip


\noindent
{\bf{Another example.}}
Consider the ODE:
\[
\dot y= (y+1)(y^2+1)
\]
and seek the life span when
$y(0)=0$.
To get the solution one uses Newton's classic decomposition 
which gives constants $A;B;C$ so that
\[
\frac{1}{(y+1)(y^2+1)}=
\frac{A}{y+1}+ \frac{By+C}{y^2+1}
\]
A computation shows that 
\[
A=1/2\,\, \&\,\, B=-A\&\,\, C=A
\]
Friom this the reader can deduce thst the solution becomes
\[
\frac{1}{2}\dot \log (y+1)-
\frac{1}{4}\log (1+y^2)+\frac{1}{2}\cdot \arctan y=
t
\]
for a constant $c$.
Notice th�t the left hand side can be written 
as
\[
\frac{1}{4}\cdot \log\, \frac{y^2+2y+1}{y^2+1}+
+\frac{1}{2}\cdot \arctan y
\]
Now the $\arctan$-function tends to $\pi/2$ as
$y\to +\infty$. From this the reader should dduce that
the life span becomes
\[ 
t^*=\frac{\pi}{4}
\]
Again it is instructivle to check this result by numerical experiments  on
a computer, i.e. simply plot solutions to
the given ODE where the length of the time interval
is chosen in a careful way until
one gets an "explosive character" of solutions when
$t\to \pi/4$.
\medskip


\noindent
{\bf{Remark.}}
Above one gets formulas for $t^*$ since we have started
with relatively simple ODE:s.
in general one cannot get explciit solutions and must therefore rely
upon numerical experiments.
Consider as an example the ODE
\[
\dot y= y^2+1+3ty+t^2
\]
with
initial condition $y(0)=0$.
The reader is invited to
approximate the life span by numerical experiments.


\bigskip


\noindent
{\bf{A problem for Homework.}}
In general one can consider a polynomial $P(y)$ where we
assume that $P(y)>0$ for every $y\geq 0$.
So the zerios of $P$ are either strictly negative real numbers or
complex, where complex roots
$a+ib$ appear in conjugate pairs.
Under the assumption that
the zeros of $P$ are all simple one has a  decomposition
\[
\frac{1}{P(y)}=
\sum\ \frac{C_\nu}{y+c_\nu}+ \sum\, \frac{A_k+B_k y}{(y-a_k)^2+b_k^2)}
\]
where $-c_1,\ldots,   -c_m$ are the real negative roots
and the conjugate complex roots are
$a_k+ib_k$ and $a_k-ib_k$.
The solution to $\dot y= P(y)^{-1}$ with $y(0)=0$
becomes
\[
y(t)= \sum\, C_\nu\cdot \log (y-c_\nu)+
\sum \,B_k\cdot \log \sqrt{y-a_k)^2+b_k^2)}+
\sum\, A_k\cdot \arctan((y-a_k)^2+b_k^2)= t+D\tag{i}
\]
for a constant $D$ determined by $y(0)=0$.
Next, by a general fact when
$P^{1}$ is decomposed as above one has
\[\sum\, C:\nu+\sum \, B_k=0
\]
From this it folows that the sum in (i)
converges to
\[
\frac{\pi}{2}\cdot \sum A_k
\]
when $y=y(t)\to+\infty.$.
Hence the life span $t^*$ is given by 
\[
t^*= \frac{\pi}{2}\cdot \sum A_k-C\tag{ii}
\]
Using a comuter one csn perform epxeriments via
various polynomials $P(y)$.
The reader csn for example analyze the case when
\[
P(y)= y^5+y+1
\]
and use the computer to
find apprximate eros of $P$ and the constsnts which sppesr
in (i) . After one can check (ii) numerically, i.e. a first stsge is simply to plot solutiond to the ODE and recognize at which time value the
solution explodes.
Nticie also thst in a converse way the numerial solution to the ODE
gives insight sbout the algebraically determined numbers in
(i).
let us remark thst the polynomial $P(y)$ above was chosen to
illustrate a  deep discovery due to   Niels Henrik Abel.
In 1823 he demonstrated that the roots of
this equation of degree 5 cannot be found by extracting roots
and radicals as in the case for algebraic equation of
degree 2,3 or 4.
So whether the algebraically minded student  likes it or not, it is
necessary to employ numerical analysis in almost  all situations.















\newpage


\centerline{\bf{A general existence theorem for ODE:s}}
\bigskip


\noindent
Let $f(x,y)$ be a real-valued and continuous function  defined
on the closed rectangle $\square$:
\[
0\leq x\leq A\quad\&\quad -B\leq y\leq B
\]
Put
\[
M=\max_{x\in\square}\, |f(x,y)|
\]
We shall assume that
\[
AM\leq B\tag{0.0}
\]
Under these conditions there exists a solution
$y(x)$ to the ODE
\[ 
y'(x)= f(x,y(x))\tag{0.1}
\]
which is defined for $0\leq x\leq A$ and where
$y(0)=0$ is the initial condition.
Before we enter the proof we give an exampe which shows
thst such a solution in general is not unique.
Namely, consider the ODE;
\[
y'= y^{1/3}
\]
So here $f(x,y)= y^{1/3}$ onlyl dpends upon $y$.
Now we get a solution defined for $x\geq 0$ by
\[
y(x)=\frac{2}{3}x^{3/2}
\]
More genrally for every positive number
$a$ we find a solution $y_a(x)$  which is identically zero when
$0\leq x\leq a$ while
\[
y(x)= \frac{2}{3}(x-a)^{3/2}\quad\colon\quad x\geq a 
\]
To ensure uniqueness of a solution to (0.1) we impose
an extra condition. Namely, that there exists a constant
$C$ so that
\[
|f(x,y_1)-f(x,y_2)|\leq C\cdot |y_1-y_2|\quad\colon\quad
0\leq x\leq A\quad \&\,\, -B\leq y_1,y_2\leq B\tag{0.2}
\]
\medskip


\noindent
{\bf{Theorem.}}
\emph{When  (0.2) holds there
exists a unique solution $y(x)$ to the ODE
wirh $y(0)=0$ and for every $0\leq x\leq A$ one has} 
\[
|y(x)|\leq
\]
\medskip


\noindent
{\bf{Remark.}}
This general   result  is attributed
to Picard whose classic text-books 
contain a wealth of results about ODE:s.
Prior to Picard one treated the case when
$f(x,)$ is a polynomial, or more generlly an analytic
function of t $x$ snd $y$. Many specific situations, where 
the classic analyris from Newton's famous text-books from 1665 
confirmed  the existence and uniqueness in Picard's theorem.
Another, and for numerical applications important
study was performed by Euler who found approximate solutions
by decomposition  the $x$-interval and associated  
difference equations.
Euler's approximate solutions were  later been refined by
several authors. Conclusive.
- and for numerical approximations  - 
fundamental results were established by Germund Dahlquist
whose Phd-thesis from Stockholm university in 1958
is a veritable classic.
Here we shall not try to expose the
analysis by Dahlquiet. Today's  student can profit  upon
implemented
programs which solve ODE:s numerically via
fast algorithms.
\medskip


\noindent
{\bf{About the proof of Picard's theorem.}}
It will 
be presented during the lectures in the course and 
detsils will be included 
as Homework
where answers which  those  provide a "nice details  " gets credit for
the exam.
\bigskip

\noindent
{\bf{Pertubations.}}
Picard's theorem yieds both existence and uniqueness
and has several  consequences. Let us expose one of them.
Consider the fsmily $\mathcal F$ of all
functions $f(x,y)$ which satisfy  (0.0 )  in
the given rectsngel $\square$, and 
for  each $f$ in this family there also exists a Lipschitz constant 
$C(f)$ in (0.2) which  can depend on $f$.
To each $f\in\mathcal F$ we get the unique solution
denoted by $y_f(x)$.
Now
\[
f\mapsto y_f
\]
is a functional  whose domain of definition is
$\mathcal F$ where each  $f$ produces the unique solution
$y_f$.
One expects continuity propertites of this map.
The actual proof of Picard's Theorem   reveals 
why "nice properties" hold.
In paricular there exists a certain differential
which we begin to explain. Consider a fixed
$f\in\mathcal F$ where we assume that
its maximum norm $f$ is such that one has a strict inequslity
\[
AM<B
\]
Then, for every Lipshitz function $\phi$ on $\square$, it follows
that $f+\epsilon\cdot \phi$ belongs to
$\mathcal F$ when $\epsilon$ is sufficiently  small.
So for small $\epsilon$ we find
the solution
\[
y(f+\epsilon\cdot \phi)
\]
It turns out that there exists a function
$\rho(x)$ defined on $[0,A]$
so that
\[
\lim_{\epsilon  \to 0}\, \max_{0\leq x\leq A}\,
\bigl| \frac{y(f+\epsilon\cdot \phi)(x)-y(f)(x)\bigr|}{\epsilon}- \rho(x)
\bigr|=0
\]
Moreover, $\rho(x)$ is the unique  solution to the linear ODE
\[
\rho'(x)=\phi(x)\cdot f'_y(,y(x))
\]

\bigskip


\centerline{\bf{The ODE $\dot y=f(t,x,y)$.}}

\bigskip

\noindent
In many applications one
starts from a function $f(t,x,y)$ if 3 variables.
Here $t$ is regarded as a time parameter.
Given a time dependent function
$x(t)$ one seeks $y(t)$ which solves the ODE
\[
\dot y(t)= f(t,x(t),y(t))\tag{i}
\]
Keeping $f$ fixed one can   solve this ODE for 
different   time dependent $x$-functions.
Under suitable  Lipschitz cionditions one
gets unique solutions to (i), and again one can
analyze pertubed solutions when
$x(t)$ is replaced by
functions of the form 
\[
t\mapsto x(t)+\epsilon\cdot \phi(t)
\]
Keeping $\phi$ fied we get a solution
$y_\epsilon(t)$ which satisfies the ODE
\[
\dot y_\epsilon (t)= f(t,x(t)+\epsilon\cdot \phi(t), ),y_\epsilon(t))\tag{ii}
\]
It turns out that there exists a limit function
\[
\rho(t)= \lim_{\epsilon\to 0}\, 
\frac{y_\epsilon(t)-y(t)}{\epsilon}
\]
where $\rho$ satisfies the linear ODE:
\[
\dot\rho(t)= f'_x(t,x(t),y(t)\cdot \phi(t)+
f'_y(t,x(t),y(t))\cdot \rho(t)\tag{iii}
\]
This equation is  often used 
in optimal control theory.





























































































 
 


















\newpage





Let $n\geq 2$ and $A=(a_{pq})$ is an $n\times n$-matrix
whose elements in general are complex numbers.
The determinatn is defined by
\[
\det A= \sum\,\text{sign}(i_1\ldots i_n)\cdot a_{1i-1}\cdots
\]
To each pair $1\leq  p,q\leq n$
we obtain an $(n-1)\times(n-1)$-matrix by removing the $p$.th row and the $q$:th column.
it ios deboted by $A[p,q]$.
Now $\det(A)$ can be cpomuted via an exanbiusion along
rows. For every $1\leq p\leq n$ this entials that
\[
\det(A)=\sum_{q=1}^{q=n}\, a_{p,q}\cdot (-1)^{p+q}\cdot \det(A[p,q])
\]
We degbiun e the $n\times $n-matrix
$\mathfrak{C}_A$ with
elements
\[
c_{pq}= (-1)^{p+q}\cdot \det A[q,p]
\]
and refer to $\frak{C}_A$ as Cremer's matrix associated to $A$.
From (i-ii) one has
\[
A\cdot \frak{C}_A= \frak{C}_A\cdot A=
\det A\cdot E
'\]
where $E$ is the dientity matrix.
Next, let $s$ be a variable and wiuth $A$ kepts fixed we put
\[
C(s)= \frak(C)_{sE-A}
\]
QWe have also the characterstic piolynomial
\[
\phi(s)= \det(sE-A)
\]
which has degree $n$.
With these nottiuons
\[
(sE-A)\cdot C(s)= \phi(s)\cdot E
\]
Next, let
\[
 r^*=\max\, \lambda
\]
wiuth the maximum taken over the roots of $\phi$.
Then the inverse matrix $/(sE_a)^{-1}$ exists when $|s|>r^*$
and (xx) gives
\[
(sE-A)^{-1}= \frac{1}{\phi(s)}\cdot C(s)
\]
Letr us now assume that
$\phi$ has a simple root at $s=r^*$ while
the absolute values of the remaning roots are all $\geq r_*$
for some $r_*<r^*$.
We can write
\[
\phi(s)= (s-r^*)\psi(s)
\]
and newton'�s fractional series
gives
\[
\frac{1}{\phi(s)}= \frac{1}{(s-r)\cdot \phi'(r)}+\frac{g(s)}{\psi(s)}
\]
where $g(s)$ is a polynomial of degree $\leq n-2$.
It follows that
\[
(sE-A)^{-1}=\frac{1}{(s-r)\phi'(r)}\cdot C(s)+
\frac{g(s)}{\psi(s)}\cdot C(s)
\]
next, we have the neumann series expansion
\[
(sE-A)^{-1}= \frac{E}{s}+ \frac{A}{s^2}+
\]
At the same time one has the geometric series expansion
\[
\frac{1}{s-r}= \frac{1}{s}+\frac{r}{s^2}+\ldots
\]
Next, since the rooots of $\psi(s)$ stay in the disc
$|s|\leq r_*$, it follows that one has a expansion
\[
\frac{g(s)}{\psi(s)}\cdot C(s)= \sum\, \frac{B_k}{s^{k+1}}
\]
which converges in the exterior disc $|s|>r_*$.
Inserting $s=r$ in (xx) the expnaions abopve give
\[
A^k=r^k\cdot C(r)/\phi'(r)+ B_k
\]


















\newpage

\centerline{\bf{� 0. The normal distribution.}}

\bigskip

\noindent
The normal distribution
is defined by the function
\[
\mathcal N(x)= \frac{1}{\sqrt{2\pi}}\cdot \int_{-\infty}^x\, e^{-t^2/2}\, dt\tag{0.1}
\] 
The derivative becomes
\[
{\mathfrak{g}}(x)=  \frac{1}{\sqrt{2\pi}}\cdot e^{-x^2/2}
\]
This is an even function of $x$ and
one has
\[
\int_{-\infty}^\infty \, \mathfrak{g}(x)\, dx=1\tag{0.2}
\]
Recall the proof of (0.2). First, since $\mathfrak{g}$ is an even function
we have
\[
\int_{-\infty}^\infty \, \mathfrak{g}(x)\, dx=2\cdot \int_0^\infty \, \mathfrak{g}(x)\, dx
\]
The idea is now to consider the double integral
\[
J=\iint\, e^{-x^2/2-y^2/2}\, dxdy
\]
where integration takes place when $0\leq x,y<\infty$, i.e over the
non-negative quarter-plane ${\bf{R}}^2_+$.
Usng polar coordinates we see that
\[
J=\int_0^\infty\int_0^{\pi/2}\, e^{-r^2/2}\cdot r\cdot drd\theta=
\frac{\pi}{2}\cdot 
\int_0^\infty\, e^{-r^2/2}\cdot r\cdot dr=\frac{\pi}{2}\implies
\]
\[
\sqrt{\frac{\pi}{2}}=\int_0^\infty\, e^{-x^2/2}\, dx
\]
and from this the reader can check (0.2).
\medskip

\noindent
{\bf{0.3 Exercise.}}
Show that
\[
\int_{-\infty}^\infty \, x^2\cdot \mathfrak{g}(x)\, dx=1\tag{0.3.1}
\]
More generally, let $k\geq 2$ and put
\[
\mathfrak{m}_{2k}^2= 
\int_{-\infty}^\infty \, x^{2k}\cdot \mathfrak{g}(x)\, dx=
\frac{2}{\sqrt{2\pi}}\cdot \int_0^\infty \, x^{2k}\cdot e^{-x^2/2}\, dx\tag{0.3.2}
\]
The variable substituion $x^2=2u$ identifies the last integral by
\[
\frac{2}{\sqrt{2\pi}}\cdot \int_0^\infty \, u^k\cdot e^{-u}\, \frac{du}{\sqrt{u}}
\]
The integral above is
found via the
$\Gamma$-function which is defined by
\[
\Gamma(\lambda)=\int_0^\infty\, u^{\lambda}\cdot e^{-u}\, du\tag{0.3.3}
\]
for every non-negative real number $\lambda$.
In the special case when $\lambda$ is a positive integer $m$ the reader can check that
succesive partial integration gives
\[
\Gamma(m)= m\,!\tag{0.3.4}
\]
From the above one has
\[
\frak m_{2k}^2= \frac{2}{\sqrt{2\pi}}\cdot \Gamma(k-1/2)\tag{0.3.5}
\]
Hence (0.3.4) entails  that 
the sequence $\{\mathfrak{m}_{2k}\}$ increases like
$\{m\,!\}$ as $m\to +\infty$.













\medskip


\noindent
Later on we shall  need the following:

\medskip

\noindent
{\bf{0.4  Proposition.}} \emph{For every real number $a$ one has
the equaity}
\[
e^{a^2/2}\cdot \int_0^\infty\, \cos(ax)\cdot e^{-x^2/2}\, dx=\sqrt{2\pi}\tag{0.4.1}
\]
\medskip

\noindent
\emph{Proof.}
Put
\[
\psi(a)= \int_0^\infty\, \cos(ax)\cdot e^{-x^2/2}\,dx
\]
The derivative with respect to $a$ becomes
\[
\psi'(a)=  -\int_0^\infty\, \sin(ax)\cdot x\cdot e^{-x^2/2}\,dx
\]
Partial integration shows that the right hand side is equal to
\[
\sin(ax)\cdot  e^{-x^2/2}\bigl|_0^\infty-
a\cdot \int_0^\infty\, \cos(ax)\cdot e^{-x^2/2}\,dx=-a\cdot \psi(a)
\]
So the $\psi$-function satisfies the differential equation
\[
\psi'(a)=-a\cdot\psi(a)
\]
At the same time the $a-$derivative of $\phi(a)= e^{a^2/2}$
is equal to $a\cdot \phi(a)$.
Hence Leibniz's rule gives
\[
\frac{d}{da}(\phi(a)\cdot \psi(a)=0
\]
Hence (0.4.1) is a constant function of $a$ as $a>+$. Finally, since
$\cos 0=1$ we see that this constant function is equal to
$\sqrt{2\pi}$  by (0.2).
\medskip

\noindent
{\bf{0.4.2  Remark.}}
When $a\to+\infty $ we notice that $e^{a^2/2}$ gets large. At the same time the function
\[
x\mapsto \cos(ax)
\]
oscillates with values taken between $-1$ and $+1$
and as $a$ increases the
osciallations becomes rapid which  entails that
the integrals 
$\int_0^\infty\, \cos(ax)\cdot e^{-x^2/2}\, dx$
tend to zero as $a\to+\infty$.


\bigskip


\noindent
{\bf{0.5. The general normal distribution.}}
For each $\sigma>0$ and every real number $m$
we define the density function
\[
x\mapsto \frac{1}{\sigma\cdot \sqrt{2\pi}}\cdot e^{-(x-m)^2/2\sigma^2}\tag{0.5.1}
\]
By the variable subsituion $x\mapsto m+\sigma\cdot u$
one verifies that the integral  taken over the real $x$-line is one.
The distribution function 
\[
x\mapsto \frac{1}{\sigma\cdot \sqrt{2\pi}}\cdot \int_{-\infty}^x
e^{-(t-m)^2/2\sigma^2}\, dt
\]
is given by 
\[
x\mapsto \mathcal N(\frac{x-m}{\sigma})
\]
with $\mathcal N$ as in (0.1).
Here (0.5.1) is the  frequency) function of
a normally distributed stochastic variable whose mean value is $m$.
The variance is
defined by
\[
\frac{1}{\sigma\cdot \sqrt{2\pi}}\int\, (x-m)^2\cdot \cdot e^{-(x-m)^2/2\sigma^2}\,dx
\]
After a variable substitution the reader can check that
this integral is equal to 
$\sigma^2$.

\bigskip

\noindent
{\bf{0.6 Remark.}}
It  is instructive to plot graphs of the functions
\[
x\mapsto  \frac{1}{\sigma\cdot \sqrt{2\pi}}\cdot e^{-x^2/2\sigma^2}
\]
for various choice of $\sigma$.
When $\sigma$ is small 
the area below the graph is concentrated to small intervals around $x=0$.
For example, let $\epsilon>0$ be a small positive number.
Now
\[
\int_{|x|\geq \epsilon}\,  \frac{1}{\sigma}\cdot e^{-x^2/2\sigma^2}\,dx
=\int_{|x|\geq \epsilon/\sigma }\,  e^{-u^2/2}\,du
\]
Keeping $\epsilon>0$ fixed while $\sigma\to 0$
we have $\epsilon/\sigma\to +\infty$, and
when $R$ is a large positive number 
\[
\int_{x\geq R}\,  e^{-u^2/2}\,du
\leq \frac{1}{R}\cdot \int_R^{+\infty}\,
u\cdot e^{-u^2/2}\,du=\frac{1}{R}\cdot e^{-R^2/2}\tag{0.6.1}
\]
where the last term gets very small as $R\to+\infty$.
\bigskip

\noindent
\centerline{\bf{0.7 Log-normal distributions}}.
\medskip


\noindent
We define a
distribution funtion $\mathcal L$ which  is zero if
$x\leq 0$ and
\[
\mathcal L(x)= \mathcal N(\log x)\quad\colon\quad x>0
\]
The frequency function becomes
\[
\mathcal L'(x)=
\frac{1}{\sqrt{2\pi}\cdot x}\cdot  e^{-(\log x)^2/2}\tag{0.7.1}
\]
It is instructive to plot the graph of this function, and 
the reader should check that
(0.6.1) tends to zero as $x\to 0$ and that  the maximum is taken when $x=1$ where
\[
\mathcal L'(1)=
\frac{1}{\sqrt{2\pi}}\cdot  e^{-(\log 1)^2/2}=
\frac{1}{\sqrt{2\pi}}\tag{0.7.2}
\]
Next, the mean value
\[
E(\mathcal L)=
\int_0^\infty\, x\cdot \mathcal L'(x)\, dx=
\frac{1}{\sqrt{2\pi}}\cdot  \int_0^\infty e^{-(\log x)^2/2}\,dx
\]
The substitution  $x=e^u$ 
gives $dx=e^u\cdot du$
and the right hand side above  is equal to
\[
\frac{1}{\sqrt{2\pi}}\cdot  \int_{-\infty}^\infty e^{-u^2/2}\cdot e^u\, du\tag{0.7.3}
\]
The reader can check that (0.7.3) becomes
\[
\frac{\sqrt{e}}{\sqrt{2\pi}}\cdot  \int_{-\infty}^\infty\,
e^{-(u-1)^2/2}\, du=\sqrt{e}\tag{0.7.4}
\]
For the second moment one has  
\[
\mathfrak{m}_2(\mathcal L)=
\frac{1}{\sqrt{2\pi}}\cdot \int_0^\infty\,  s\cdot e^{-(\log s)^2/2}\,ds=
\frac{1}{\sqrt{2\pi}}\cdot  \int_{-\infty}^\infty e^{-u^2/2}\cdot e^{2u}\, du
\]
The reader can check that the last integral is equal to $e^2$.
It follows that the 
central variance 
\[
\sigma^2(\mathcal L)= 
\mathfrak{m}^2(\mathcal L)-E(\mathcal L)^ 2= e^2-e=(e-1)e\tag{0.7.5}
\]
\medskip

\noindent
If $k\geq 3$ the  moment
\[
\mathfrak{m}_k(\mathcal L)=
\frac{1}{\sqrt{2\pi}}\cdot  \int_{-\infty}^\infty e^{-u^2/2}\cdot e^{ku}\, du=
\]
\[
e^{k^2/2}\cdot \frac{1}{\sqrt{2\pi}}\cdot   \int_{-\infty}^\infty e^{-(u-k)^2/2}\, du
= e^{k^2/2}\tag{0.7.6}
\]

\bigskip


\noindent
{\bf{0.8 Other distribution functions.}}
In � xx we  study stochastic processes and encounter
situations where the distribution functions no longer are normal.
An example is the  stochastic variable $\chi$
defined by a sum:
\[
\chi=\chi_*+\mathcal N
\]
where $\chi_*$ and $\mathcal N$ are independent
stochastic variables and $\mathcal N$ the standard normal distribution.
Let  $f_*(x)$ be the frequency  function  of $\chi_*$. In � xx we  shall learn that
the frequency function of $\chi$ becomes:
\[
f(x)= \frac{1}{\sqrt{2\pi}}\cdot \int\, f_*(x+s)\cdot e^{-s^2/2}\, ds
\]
Consider as an example  the case when
$\chi_*$ takes the values +1 or -1 with probability 1/2. 
Then
\[
f(x)= \frac{1}{\sqrt{8\cdot \pi}}\cdot
[e^{-(s+1)^2/2}+e^{-(s-1)^2}]\tag{0.8.1}
\]




\end{document}














