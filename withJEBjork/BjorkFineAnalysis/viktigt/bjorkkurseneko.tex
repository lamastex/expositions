

\documentclass{amsart}
\usepackage[applemac]{inputenc}


\addtolength{\hoffset}{-12mm}
\addtolength{\textwidth}{22mm}
\addtolength{\voffset}{-10mm}
\addtolength{\textheight}{20mm}
\def\uuu{_}

\def\vvv{-}


\begin{document}


Let $U$ be a utulity function.
At time $t=0$ a consummer has a capital $K$
and if $c8t<)$ is the rate of consumption we have a differential equation
\[
\dot K(t)= rK(t)-c(t)
\]
where $r$ is a positive constant ,i.e. a. discount rate.
We seek
\[
V)K)= \max_c\,\int_0^\infty\, U(c(t))\cdot e^{-at}\, dt
\]
where $a>0$ is another constant.
For a general utiliity function we cannot
obtsan a "closed soluion". . But let us consider the "popular" case  
\[
U(c)= \sqrt{c}
\]
In this case it turn out that
the rate of consumption at every moment is proportional to
the capital, i.e. tghere exists a constsnt
$\beta>0$ such thst
\[
c(t)= \beta\cdot K(t)
\]
Moreover, the whole maximum probelm is only well-posed when
$a>2r$ and when it holds we have the equation
\[
V(K)=\frac{1}{\sqrt{2a-r}}\cdot \sqrt{K}\tag{*}
\]
It merans thst if the discount number $r$ is too large compred to
the "negstive attiitude" for futire utility expressed by
$e^{-at}$, then the
consumer tends to let cspitsl increase and eventuysl obtsin
an srbitrary large value $V$, even if the in itisl capitsl id
very small.
To prove (*) one preoceeds as follows. let
$c(t)= c$ be constsnt duriung a small initial time interval
$0\leq t\leq \Delta$.
Then
the value becomes
\[
\sqrt{c}\cdot \Delta+V(K+rK\delta-c\delta)e^{-a\Delta}=
\]
By a Tayklor expansion this becomes
\[
V(K)+ \Delta\sqrt{c}-V'(K)\cdot c-rK\cdot V'(K)- a\cdot V(K)]
+o(\delta)
\]
Now $c$ is determined whrn we tske the
maximum of
\[
\sqrt{c}--V'(K)\cdot c\implies \sqrt{c}=V'(K)/2
\] 
For this it follows thst
the value funtion satsfies the ODE:
\[
\frac(
{1}{4V'(K)}+rKV'(K)= aV(K)
\]
with initial coindition $V(0)=0$ snd from this the
reader can check (*) and after show that
(xx) hold for a constant $\beta$ which the reader should determine.
\bigskip


\noindent
{\bf{A more general case.}}
For an arbitary utili8iy function
$U(c)$
one first weeks
\[
\max_c\, U(c)-V'(K)\cdot c\implies U'(c)= V'(K)
\]
Since $c\mapsto U'(c)$ by assumption is strictly increasing
this determines $c$ via $V'(K)$,  i.e. if
$\gamma$ is the inverse of the function $U'$ one has
\[
c= \gamma\circ V'(K)
\]
and  $V$ satisfies the ODE:
\[
U\circ \gamma(V'(K))+rKV'(k)-\gamma\circ V'(K)\cdot V'(K)=aV(K)
\]
Here one needs a computer to
analyze the solution and also to  find
conditions on the pair $r,a$ in order
that the maxium problm is well-posed in
the sense thst the value function $V(K)$ is bounded.
\bigskip


\noindent
{\bf{Exercise.}}
Consider the utility function
\[
U(c)=\log\, (c+1)
\]
So here $U'(c)= \frac{1}{c+1}$
and from the above
\[
\frac{1}{1+c}=V'(K)\implies
\frac{1}{V'(K}-1=c
\]
Now the reader is invited to solve
the ODE and analyze for which pairs $r,a$ the
problem is well-posed.
Of course, here it is tempting to firt perform
some numericl exp�eriments when
vlues for $r$ snd $a$ are assigned from
the stsrt, ssy thst you take $r=0,1$ snd firdt try a rather large
$a$ to see if $K\mapsto V(K)$ stays bounded.
To fullfill this exercise gives considerable credit in the course !
Cases when we for example have
a iutility funvtion
\[
U(c)= \sqrt{c}+ c^{1/4}
\]
leads to quite involed computations where
even
the search for numercial solutions is not easy.
\bigskip

\centerline{\bf{The case of a riusky asset}}
\bigskip


\noindent
At time $t=0$ the consumerr puts  intial capital n in a risky asset which means that
$t\mapsto K(t)$ is a random vsriable, and
while consumption takes place one gets  the "stocastic equation"
\[
dK= \mu\cdot K-\cdot \sigma K\cdot dW
\]
Via Taylor's   fornula and a similar 
argument as in the case of a safe asset with  $U(c)= \sqrt{c}$
it follows that the $V$-function satisfies the ODE:
\[
\frac{1}{4V'(K)}+\mu KV'(K)+
\frac{\sigma^2 K^2}{2}\cdot V''(K)
= aV(K)
\]
The solution becomes
\[
V(K)= \rho\cdot \sqrt{K}
\]
where $\rho$ is a constant and
\[
\frac{1}{2\rho}+\frac{\mu\rho}{2}-\sigma^2\cdot \rho\cdot
\frac{1}{8}= a\rho\implies
\frac{1}{\rho}=(2a-r+\sigma^2/)\|cdot \rho
\]
Hence we find
the equation
\[
\rho= \frac{1}{2a-\mu+\sigma^2/4}
\]
Notice that $\rho$, and hence also $V(K)$ decreases with
$\sigma$ which reflects that the consumer has an aversion to
risk which follows from the concavity of
the utiliity function. So compared to a rsiky asset
the consumer only will prefer the risky asset if
the
positive rate $\mu>r$.
A risk neutral case occurs when
\[
\mu=\sigma^2/4+r\tag{1}
\]
A notable fact is that if the consumer at time zero
csn put a fraction $0<\gamma<1$ in a risky  
asset and the rest $(1-\gamma)K$ in a safe asset when
(1)  holds, then a clever choice of
$\gamma$ leads to a higher expected value $V(K)$.
Let us remark that
this optimal portfoiio has become popular in
text-books dealing with abstract economic models.
Here we remakt thst a more detailed study involves
the volatility when the risky asset is chosen, i.e.
once the consumer puts an amonunt in a risky asset
and after maximises the $V$-function, even the
time-dependent conumption is a random variable and
the outcome $V(K)$ is a random variable too, where
we so far only have discussed its mean-value,
Using a computer and eventual Monte-Carlo simulations one can
get  a numerical insight about the
random variable $V$, and 
get numerical numbers for its
volatility.























\newpage




\centerline{\bf{A system of ODE:s}}
\bigskip


\noindent
Consider  the following ODE-system
\[
\dot z_1=z_1^\alpha z_2^\beta-z_1\quad\&\quad
\dot z_2=z_1^\alpha z_2^\beta-z_2\tag{*}
\]
where $\alpha,\beta$ are positive constants and
$\alpha+\beta<1$.
If $z_1=z_2=1$ 
the solution to (*) with initial conditions
$z_1(0)=z_2(0)=p$ stays at $(1,1)$, i.e. $(1,1)$ is a stationary  point for
(*).
It  turns out that
every solution converges to this stationary point as
the time variable increases.
\medskip


\noindent
{\bf{1. Theorem.}} \emph{For every pair of positive numbers
$a,b$ the unique solution to (*) with
$z_1(0)=a$ and $z_2(0)=b$
tends to $(1,1)$ as $t\to +\infty$.}
\medskip


\noindent
{\bf{Exercise.}}
Prove this. The hint is as follows:
First (*) entails that
\[
\dot z_1+z_1=\dot z_2+z_2
\]
Check that this gives a function $f(t)$ defined for
$t\geq 0$ where $f(0)=0$ and
\[
z_1(t)=e^{-t}\cdot (f(t)+a)\quad\&\quad
z_2(t)=e^{-t}\cdot (f(t)+b)\tag{i}
\]
and $f$ satisfies the ODE:
\[
\dot f(t)= e^{t(1-\alpha-\beta)}\cdot (f(t)+a)^\alpha\cdot (f(t)+b)^\beta\tag{ii}i
\]
This non-linear first order ODE cannot be solved
in closed form. But in the speciai case  $a=b$
we find a solution.
Namely, let 
$a=b=M$ for some $M>0$ and put $\gamma=\alpha+\beta$.
Then 
\[
\frac{df}{(f+M)^\gamma}= e^{(1-\gamma)t}\implies
\]
\[
\frac{1}{1-\gamma}\cdot (f(t)+M)^{1-\gamma}=
\frac{1}{1-\gamma}\cdot [e^{(1-\gamma)t}+C]
\] 
where the reader should determine  the constant $C$ from the initial condition $f(0)=0$. It follows that
\[ 
f(t)+M= [e^{(1-\gamma)t}+C]^{\frac{1}{1-\gamma}}=
e^t\cdot [1+Ce^{(\gamma-1)t}]^ {\frac{1}{1-\gamma}}\tag{iii}
\]
Sinc $\gamma<1$
we see that
\[
\lim_{t\to\infty}\, e^{-t}\cdot f(t)=1
\]
and then (i) give 
the requested limits in the theorem.
Next, suppose that $a\neq b$ sand consider for example the case
$0<a<b$.
Now we consider the function
$\phi(t)$ which satisfies the ODE:
\[
\dot \phi(t)= e^{t(1-\alpha-\beta)}\cdot (\phi(t)+b)^\alpha\cdot (\phi(t)+b)^\beta\tag{ii}i
\]
with initial condition $\phi(0)=0$.
The reader can recognize that
the $\phi$-function increases faster than $f$ and from the special case
we  have
\[
\lim_{t\to\infty}\, e^{-t}\cdot \phi(t)=1\implies
\limsup_{t\to \infty}\, e^{-t}\cdot f(t)\leq 1\tag{1}
\]
In a similar fashion we regard the $\psi$-function which satisfies
$\psi(0)=0$�and
\[
\dot \psi(t)= e^{t(1-\alpha-\beta)}\cdot (\psi(t)+a)^\alpha\cdot (\psi(t)+a)^\beta\tag{ii}i
\]
This time $\psi\leq f$ and the special case applies with
$M=a$ so thast
\[
\lim_{t\to\infty}\, e^{-t}\cdot \psi(t)=1\implies
\liminf_{t\to \infty}\, e^{-t}\cdot f(t)\geq 1\tag{2}
\]
Together (1-2) proves the theorem.

\medskip

\noindent
{\bf{Remark.}}
One can confirm the theorem via numerical
experiments.
In practice one is also interested
in the rate towards the equilibrium point (1,1).
To be precise, take some small $\delta>0$ and 
with a given starting point $(a,b)$ one seeks the time
$T$ when
\[
|z_1(T]-1|\leq \delta\,\&\,
|z_2(T]-1|\leq \delta
\]
Here no analytic solutions are available so one must use the computer to get numeral
solutions.
For the special models in economy
where asymptotic stability
appears, it is of course
interesting - and in a sense very relevant - to
find approximations of time values as above.
The numeral examples show that
$T$ can be quite large even when
$\delta$ is not too small, say that
$\delta=0.1$.
Keeping $\alpha$ and $\beta$ fixed
the initial values $a$ and $b$
effect $T$.












\bigskip


\noindent
{\bf{2. A more general system.}}
Let $\phi(z_1,z_2)$ be a typical utlity funation
and consider the system
\[
\dot z_1=\phi(z_1,z_2)-z_1\quad\&\quad
\phi(z_1,z_2)-z_2\tag{**}
\]
One  gets $f$ as above which now satisfies the ODE
\[
\dot f=e^t\cdot \phi(e^{-t}(f(t)+a),e^{-t}(f(t)+b)\quad\&\quad f(0)=0\tag{2.1}
\]
When $\phi$ is a typical utility function of two variables
one can expect that
\[ 
\lim_{t\to +\infty}\, e^{-t}\cdot f(t)=1\tag{2.2}
\]
exactly as in the very special case above.
The reader is  invited to use a computer and perform
numerical experiments to
check if (2.2) hold for
some
non-trivial  $\phi$-functions apart from
$z_1^\alpha\cdot z_2^\beta$.
Let us notice  that with 
\[
f(t)= e^t\cdot y(t)
\]
 the $y$-function
satisfies the ODE
\[
\dot y+y= \phi(y(t)+ae^{-t},y(t)+be^{-t})
\]
which gives an indication that
$y(t)$ has a limit when $t\to +\infty$ because  $ae^{-t}$ and
$be^{-t}$ both tend to zero.
Of course, a numercial study using a more general $\phi$-function as in
(*) goes  far beyond standard text-books where
the popular choice $z_1^\alpha\cdot z_2^\beta$
appears  in hundreds of text-books exposing 
models in economy.
Let us only remark that studies of asymptotic 
limits to non-linear systems of ODE:s is an  extensive subject in mathematics where 
pioneering work was done by Picard and Poincar� around 1880.
But thanks to computers one can nowadsys at least
perform  numerical experiments
in order to analyze the eventual existence  of limits of solutions to
a stationary point of the system (**) where 
the stationary point would be $z_1=z_2=p$
with
\[
p=\phi(p,p)
\]




















\newpage





\centerline{\bf{Optimal copnsumption in a dynamic model}}


\bigskip

\noindent
Let $U$ be a utiliity function. During a time
interval $[0,T]$ a consumer
seeks maximal 
utlitiy expressed by a time dependent function
$c(t)$ where $c(t)$ is the rate of consumption.
We have also a budget constraint 
\[
\int_0^T\, c(t)=A\tag{*}
\]
and  seek
\[
V=\max_c\, \int_0^T\, U(c(t))e^{-rt}\, dt\tag{**}
\]
where $r>0$ is a discount factor.
Euler's equation shows that the maximum is attained  when
there exists a constant $B$ such that
\[
U'(c(t))= B\cdot e^{rt}\tag{1}
\]
for every $0\leq t\leq T$. Here  
$B$ is determined via (*).
Only in very special cases $c(t)$ can
be expressed in  a  closed form.
A popular example in text-books  is to take
$U(c)= c^\alpha$ for some
$0<\alpha<1$.
Consider as an example the case $\alpha=1/2$. Now (1) gives
\[
\frac{1}{2\sqrt{c(t)}}=Be^{rt}\implies
c(t)=\frac{1}{2B^2}e^{-2rt}
\]
Via (*) the reader can calculate $B$ when $A$ is given
and  find
the maximmum in (**), where the value function
$V$  depends on the 3 parameters $A,r,T$.
Using a computer one 
gets numeral solutioins when numerical values are assigned to
$A,T,r$.
\medskip


\noindent
{\bf{Exercise.}}
Let us consider the utility function
\[
U(c)= \sqrt{c}+ ac^{1/3}+ bc
\]
where $a$�and $b$ are  positive contants.
Here Euler's equation becomes
\[
\frac{1}{2}\cdot c^{-1/2}+ \frac{a}{3}\cdot c^{-2/3}+b= Ce^{rt}
\]
where the budget constraint determines $C$.
Even for readers who like to perform algebraic manipulations
it is quite hard to even vizualise the time dependent
solution $c(t)$ to this equation.
\medskip

\noindent
Let us consider the numerial example with $a=b=1$.
Taking the time derivative in Euler's equation (1)
we get he first order ODE:
\[
U''(c(t))\cdot \dot c(t)= Bre^{rt}\implies
\]
\[
-[\frac{1}{4}\cdot c^{-3/2}+\frac{2a}{9}c^{-5/3}]\cdot \dot c=
Bre^{rt}
\]
where $B$ is determined via the budget constraint
(*) where we take $A=1$.
Keeping $r$ and $T$ as "variable parameters" one may now try to
exhibit numerical values for the $V$-function.



\medskip


\centerline{\bf{2. Salvage values.}}
\medskip

\noindent
Above the consumer must obey the budget constraint (*).
We can also  imagine that
the consumer is allowed to take a loan which leads to determine 
\[
\max_c\, \int_0^T\, U(c(t))\cdot e^{-rt}\, dt  + G(\int_0^T\, c(t)\, dt -A)\tag{2.0}
\]
where $G$ is an increasing function. A special, case could be that
$G(s)= Ms$ for some positive constant  $M$.
So here we study  variational problem with
a salvage value,
Let us consider the  case
 $U(c)= \sqrt{c}$ which gives  an analytic soluttion.
Namely, the value function  $V$ from (**) for fixed $T$ and  $r$
while $A$ varies becomes
\[
V(A)=\rho(r,T)\cdot \sqrt{A}
\]
where the reader is invited to determine the
$\rho$-function.
Sdding the salvage value it is  profitble to
to
take a lone if
\[
V(A+\xi)-M\xi>=V(A)
\]
for some $\xi>0$.
To be precise, for $\xi\geq 0$ we consider the functioin
\[
\rho(T,r)\sqrt{A+\xi}- M\xi
\]
and one  finds
the maximum when $0\leq \xi<\infty$.
For example, the derivative at $\xi=0$
becomes
\[
\frac{\rho(T,r)}{\sqrt{A}}-M
\] 
and if it is $>0$ the cosnumer will
take a loan.
The reader is invited to give specific numerical examples
where
a loan will
give a maximum to (2.1).
\bigskip


\noindent
{\bf{2.1 Consumption from a capital stock.}}
A utility function $U(c)$ is given.
At time $t=0$ he consumer has a capital $K_0$.
Due to consumption the capital changes via the ODE
\[
\dot K(t)= \mu\cdot K(t)-c(t)\tag{i}
\]
where $\mu>0$ is a rate of interest.
Now we seek
\[
\max_c\, \int_0^T\, U(c(t))e^{-rt}\, dt+
K(T)\tag{2.1.1}
\]
Thus,   saved capital at the terminal time is
a salvage value.
To solve this optimization probelm we should first
express $K(T)$ via a chosen rate of consumption.
Here (i) gives
\[
\frac{d}{dt}(Ke^{-\mu t}=-c(t)\tag{ii}
\]
and from  this the reader should check that
\[
 K_0-K(T)= e^{\mu T}\cdot \int_0^T\, e^{-\mu t}c(t)\, dt\tag{iii}
\]
Hence 2.1.1 amounts to determine 
\[
V^*K_0+\max_c\, \int_0^T\, U(c(t))e^{-rt}\, dt-
e^{\mu T}\cdot \int_0^T\, e^{-\mu t}c(t)\, dt\tag{2.1.2}
\]
\medskip


\noindent
{\bf{Exercise.}}
Find the maximal value  $V^*$ in closed form
when
$U(c)= \sqrt{c}$ while $r,\mu,A,T$ are given parameters.
\bigskip



\noindent
{\bf{A problem with a risky asset.}}
Above a negative discount factor $e^{-rt}$ appears.
One may also consider a positive discount where the utility 
from consumption is expressed by
\[
\int\, U(c(t))\cdot e^{rt}\, dt
\]
An example could be when
the consumtion is used for medical care where
illness may increase over time and become more costy.
With a budget constraint 
as in (*) the maximum is now attainded by a function
$c(t)$ which increases eith $t$.
Let us now assume that the conumer can put a certain  initial capital in
a risky asset - like an obligation- which
may give extra capital at a random time.
More precisely, a profit is received via
a Poisson process. It  means that one has a positive number
$\lambda$ and the probability to gain during  a small time
interval $[t,t+\Delta]$ is equal to
$e^{-\lambda t}\lambda\cdot \Delta$.
If this happens - via a lucky  number from  the lottery - the 
consumer gets  extra 
capital $K$ which can be used for consumption up to time
$T$, and a part may also be kept as a salavage value.
At time  $t=0$ the consumer can  put
a fraction $a$ of the initial  capital $A$ to by 
an obligation whose  reward  becomes
$a\cdot K$ if a lucky number comes up during the lottery.
In practice  $K$ is quite large while $\lambda$ is small.
Now one seeks maximum of \emph{expected utility}.
Here  the outcome from the initial choice is random. So
we are confronted with  a \emph{stochastic optimization}. 
The solution to this optimiziation problem is far   more involved compared to
determinisitc models. The reason is that
before an  eventual profit in the lottery, the
usual Euler equation cannot be used to
determine  the optimal plan of consumption because future is random.
So we  shall  not try to
carty out the details which lead to a solution of the stochastic 
optimization above. But in the next section
we shal, consider
a case where the necessary stochastic  analysis can
be carried out explicitly.
\bigskip


\centerline{\bf{3. A random rescue  by a helicopter.}}\
\medskip

\noindent
First we
first regard a familiar problem  where one seeks
\[
\min \int _0^T\, \dot x(t)^2\, dt\tag{3.1}
\]
with fixed end-values $x(0)=0$ and $x(T)= A$.
The Euler equation shows that mimimum is attaiend when 
$\dot x$ is a constant, i.e 
$\dot x(t)= A/T$ and the minimum becomes
$A^2/T$.
A stochastic optimization  arises as follows:
Imagine  a person who starts
at a place in a dry desert and must arrive to
a lake at time $T$  to get water in order to  survive.
The person, call him or her "Silly Bill" or "Silly Mariy"  has a compass and knows the straight way to
the lake and the distance $A$.
The squared velociity $\dot x^2$ corresponds to kinetic
energy which Bill or Mary  try  to
minimize  
during the travel to the lake.
Next, society feels pity for persons who 
walk in the desert
and with no cost a helicopter
is looking around to search  Bill or Mary.
The chance   to find a person walking in the desert
is ruled  by a Poisson process with parameter
$\lambda$, i.e. the probability to discover the person at
a small  time interval�$[t,t+\Delta]$ is equal to 
\[
e^{-\lambda t}\lambda\cdot \Delta
\]
Once the pilot  in  has discovered a  person
we imagine  that the helicopeter comes to rescue
in a very short time and gives 
the person  free lift to the lake,
So if the discovery is made at some time
$0<t<T$ then
the  loss of energy only comes from walking up to
this time, i.e. given by
\[
\int_0^t\, \cdot x(t)^2\, dt
\]
The question arises how Bill or Mary  should behave before
an eventual discovery.
It is  tempting to just stay at rest in
the desert  and hope that the helicopeter comes.
But the probability that the pilot never discovers the person
�s $e^{-\lambda T}$. So  this strategy is  risky because
the person dies from thirst if the lake is not reached before time
$T$.
So Mary who has studed some mathematics
understands immediately thstit is necessary to
start  walking, but with a
speed which initiallly is smaller than
$A/T$ since  the helicopter might arrive  to rescue.
It turns out that expected energy is minimized when
the person  walks with a certain  slowly increasing pace
until the eventual arrival of 
the helicopter.
To solve this stochastic optimization one uses
a device which in the literature is referred to as
"dynamici programming" , though one should recall 
that methods  to solve various stochastic optimzation has been
known for a long time
in matheamticla physics, where can mention  deep work
Helmholtlz and Bolzmann more than a  century ago.
\medskip�

\noindent
{\bf{The solution.}}
Introduce  the function $V(A,T)$
defined as the minimum of \emph{expected cost of energy} when
the person has two wak the distance $A$ during a time interval $T$.
Notice that the $V$-function  is definec without knowing
the optimal  strategy in advance !
But we can find a PDE for $V$.
Namely, during a small time interval $[0,\Delta]$ we let the person walk with constant  speed $u$, and then 
expected energy becomes
\[
u^2\cdot \Delta+V(A-u\cdot \Delta,T-\Delta)\cdot e^{-\Delta\lambda}\tag{i}
\]
where we have used  that
$e^{-\Delta\lambda}$ is the probability that the helicopter does not arrive
during the small initial interval. Via Taylor expansion  (i) becomes
\[
V(A,T)+ (u^2- V_A\cdot u-V_T- \lambda\cdot V)\Delta+
o(\Delta)
\]
The right hand side is minimized when
$2u+V_A=0$ and  the definition of the $V$-function
gives the PDE:
\[
\frac{V_A^2}{4}+V_T+\lambda V=0\tag{ii}
\]
To solve (ii) we try
\[
V(A,T)= A^2\cdot g(T)
\]
and  (ii) corresponds to the ODE:
\[
g(T)^2+g'(T)+\lambda g(T)=0\tag{iii}
\]
To solve (iii) we put
\[
g(T)=e^{-\lambda T}\cdot \phi(T)\tag{iv}
\]
and get
the ODE:
\[
\phi'(T)=-e^{-\lambda T}\cdot \phi(T)^2\implies
\frac{d\phi}{\phi^2}=-e^{-\lambda T}dT
\]
It follows that
there is a constant $C$ such that
\[
\frac{1}{\phi(T)}= C-\frac{1}{\lambda}\cdot e^{-\lambda T}\tag{v}
\]
Above $T$ is a variable and
the reader should confirm that
\[
\lim_{T\to 0}\, g(T)=+\infty
\]
and deduce that
\[
C=\frac{1}{\lambda}\implies \phi(T)= \frac{\lambda}{1-e^{-\lambda T}}
\]
Hence we have proved that
\[
g(T) =\frac{\lambda\cdot e^{-\lambda T}}{1-e^{-\lambda T}}=
\frac{\lambda}{e^{\lambda T}-1}
\]
\medskip


\noindent
While Bill complains  that he
never has studied mathematics
and from his starting postion is  uncertain  how
to
start walking to save energy while he looks at the sky and hopes thst
the heliopter arrives, Mary
performs a clever computation and
determines the optimal time dependent velocity until
an eventual arrival of the helicopter.
To begin with she finds  from the above that
\[
V(A,T) =A^2\cdot \frac{\lambda}{e^{\lambda T}-1}
\]
So at time $t=0$ the intial velocity should be
\[
\dot x(0)=\frac{V_A}{2}=  A\cdot \frac{\lambda}{e^{\lambda T}-1}\tag{1}
\]
The reader should check that
(1) is $<\frac{A^2}{T}$ and  $\dot x(0)<\frac{A}{T}$.
The last inequality reflects the  intuitive fact that
the eventual   arrival of the helicopter
means that the clever Mary  initially moves with slower speed
compared to the case when no helicopter is present.

\medskip

\noindent
Next, when $0<t<T$
and the helicopter has not arrived
up to time $t$, 
the time dependent function
$t\mapsto x(t)$ satisfies the following ODE:
At time $t$ Mary has walked a distance $x(t)$ so
there remains the distance $A-x(t)$ to the lake which
must be reached before time $T-t$.
Now one has the definite result:
\medskip

\noindent
{\bf{1. Theorem.}}
\emph{When Mary  walks  in order to minimize
expected energy
the optimal  velocity prior to the eventual arrival of the helicopter
satisfies the differential equation}
\[
\dot x(t)=
\frac{\lambda\cdot(A-x(t))}{e^{\lambda (T-t)}-1}=
(A-x(t))\cdot\frac{\lambda\cdot e^{\lambda t}}{e^{\lambda T}-e^{\lambda t}}
\tag{1.1}
\]
\emph{and the intial condition $x(0)=0$
(1.1) has the solution}
\[
A-x(t)= \frac{A}{e^{\lambda T}-1}\cdot (e^{\lambda T}-e^{\lambda t})\implies
x(t)=A\cdot \frac{e^{\lambda t}-1}{e^{\lambda T}-1}\tag{1.2}
\]
\medskip


\noindent
{\bf{Remark.}}
So if the helicopter never arrives
then the velocity at time
$T$ becomes
\[
\dot x(T)= \frac{\lambda\cdot A\cdot e^{\lambda T}}{e^{\lambda T}-1}
=\frac{\lambda\cdot A}{1-e^{-\lambda T}}
\tag{1.3}
\]
The reader should check that (1.3) is strictly larger than
$\frac{A}{T}$.
\medskip


\noindent
Notice also that with $t$ and $A$ fixed, the function in (1.3) taken with respect to
$\lambda$ is striclty increasing where the limit value as $\lambda\to 0$ is equal
to $\frac{A}{T}$
On the other hand (1.3) becomes quite large when
$\lambda$ increases while $A$ and $T$ are kept fixed. This  is reflected by the intuitive
fact that with a large $\lambda$ the person
is optimistic and starts with a slow speed in the hope to  get a free lift
before time $T$, i.e. it  is only when
$t\to T$ that
the person must start  running with higher speed in order to
arrive at $x=A$ at time $T$.
\medskip

\noindent
{\bf{Exercise}}.
Give details of the proof of (1.1-1.2) in the Theorem above !
\bigskip


\noindent
{\bf{2. Optimization with a salvage funtion.}}
Above the peron is obliged to arrive
at $A$ when $t=T$. In addition to
the
probability of an arriving helicopter - where no charge occurs 
it it happens that the hrlicopter gives a free lift -
we suppose that  the person can  
rent a helicopter from a private company cost
and 
get an immediate rescue.
For example, if the person moves with a rather slow speed compared with (1.1)
in  Theorem 1, it may occur that
$x(T)<A$. But   the person has now the option to rent
a  helicopter
at a time $T-\delta$ with small $\delta$ and quickly arrive at
$x=A$ at time $T$, i.e. we imagine that a helicopter which
picks up the person moves very fast to the terminal point $x=A$
If  the person  rents the helicopter we suppose that
the cost is
$C\cdot (A-x(T)$ where $C$ is  a fairly large constaant.
To solve this new optimization problem
we regard the value function $V(A,T)$ which was computed above. 
Now 
the person can choose  a strategy with optimal speed $\dot x(t)$ before
the free helicoppter eventually arrives where the new end-value
$x(T)$ is equal to $a$ for some $a<A$ and  chosen to minimize the sum:
\[
V(a,T)+C\cdot (A-a)
\]
So it means that
\[
2a\lambda=(e^{\lambda T}-1)\cdot C\tag{2.1}
\]
Only when $a<A$ holds in (2.1 )  the person
is willing to rent a helicopter.
Thus, if
\[
2A\lambda\leq (e^{\lambda T}-1)\cdot C
\]
then the company who has a "saving helicopter" never gets a demand from
persons moving as above, while
the eventual resuce by the   helicopter which is free of charge  may appear.
This salvage prblem leads to
a further optimization problem.
Namely, the company which has a  helicopter which
can provide a free lift at any moment
should decide a price $C$ in order to maximize
\emph{expected profit} while clever 
persons are moving as above and choose
optimal strategies to minimize their expected cost of energy, plus eventual charge from
the company's helicopter when
$a<A$ holds in (2.1).
The analysis of this new stochastic optimization problem, which has to be solved
by the company which rents a helicopter is quite interesting.
Let us only remark that the solution
leads to a quite  involved calculation where computers help 
to provide numerical solutions in order to
get a feeling for this optimization problem.
It goes wothout sayung that
a solution to a problem of this nature wuold give
a top-rate in the present course.


\bigskip


\centerline{\bf{Another optimization problem.}}
\bigskip

\noindent
Imagine a company where research is essential to
discover  a new item.
A typical case would be to produce
a new medicine   which  requires extensive research
before it can be found and lead to profitable production.
Let $[0,T]$ be a time interval and suppose that
the company only gets
profit if the discovery is found 
during this time interval.
Let $t\mapsto \rho(t)$ be the intensity of research.
So
\[
R(t)= \int_0^t\, \rho(s)\, ds
\]
is  ackumulated research up to time
$t$.
We suppose that
the non-decreasing $R$-function affect a time dependent Poisson parameter
whose random process governs the probability that
a new item is discovered.
Thus, there exists a constant
$a>0$ such that
\[
\lambda(t)= a\cdot R(t)
\]
hold before an eventual discovery.
By Poisson's equation this means that
if the company invests in research with a given rate, then
the probability for a discovery during a small time
interval $[\tau,\tau+\Delta]$ becomes
\[
a\cdot e^{-a\cdot \int_0^\tau \, R(s)\, ds}\cdot R(\tau)\cdot \Delta
\]
If the discovery of the new item is found
at a time
$t$ the company recieves a profit
\[
\Pi(t)= A\cdot e^{-rt}
\]
where $r$ is some positive constant, i.e. the sooner
a discovery is found the higher is the reward.
At the same time
the company has a cost for performed research up to this time. So  net profit is 
given by
an equation
\[
\Pi_*(t)=\Pi(t)- B\cdot R(t)
\]
where $B$ is another positive constant.
The company now tries to maximize expected profit, i.e. the problem is to find
the optimal rate of research in order to maximize  expected net profit.
Above we   assumed that
the cost of research is constant, i.e.
during the time period we imagine that the company
already has
employed people whose salaries remain fixed.
It is  intuitively clear that
the company should invest in research as quick as possible in order
to
increase Poisson's $\lambda$-parameter.
For a company with large resources
the time derivative $\dot R=\rho$ can be made large.
However, in practice  limitations occur. One can imagine chemical processes 
or examinations of biological bodies  which imply
that the function $\rho(t)$ is bounded above by some constant.
So one is led to  consider  a constrained optimization problem where
one seeks  to maximize expected net profit when
\[
0\leq \rho(t)\leq M
\]
hold for every $t$.
Adding this constraint the optimization problem has a Bang-bang solution.
More precisely, during a time interval $[0,\tau]$
the company decides to choose $\rho(t)=M$
and after $\rho(t)=0$ when $t>\tau$.
If one regards such a strategy one can  compute expected profit as a function of
the switch time
$\tau$ and  solve the optimization problem. One also finds
$\Pi_*$ epxressed as  a function of the parameters
$A,B,a,r$ and $M$.
\medskip

\noindent
{\bf{Remark.}}
The formal proof that
$\Pi_*$ is maximized by a Bang-bang 
strategy relies upon
Pontryagin's general results in OCT.
\bigskip


\noindent
{\bf{A more involved case.}}
Tne model above is not always realistic. It may occur that
the input of research
at an early time
eventually does not affect the Poisson parameter.
One can imagine
fragile  items which are studied during the research and
if no disvovery has been made after a certain time, the
information
from early research becomes  diminished, or even lost.
Above  we assumed that
\[
\lambda(t)= a\cdot\int_0^t\, \rho(s)\,ds\tag{*}
\]
Let us replace this by an equation of the form
\[
\lambda(t)=a\cdot \int_0^t\, e^{\mu(s-t)}\cdot \rho(s)\, ds\tag{**}
\]
where $\mu $ is a positive constant. It means that research performed during a 
small early time
interval�$[s,s+\delta]$ will no
enlarge $\lambda(t)$ so much when $t>s$.
When (*) is replaced by (**) the optimization problem becomes more
involved
and it is no longer clear if the company will choose a Bang bang solution.
For example, if $\mu$ is large  one may expect that
the company will decide to perform some reseach at a late time value if
no discovery has appeared since invested early research has diminished
the Poisson parameter. For example,  one may get
a Bang-bang solution where
several switch times  appear.
So the   discussion shows that
more \emph{realistic optimization problems}  become quite complicated.
A further scenarium appears in a competitive
world where 
our company is not alone in the search for a new item.
But thanks to "clever spies" we suppose that  our  company knows
input of resarch at every moment performed by another company $C^*$
and
the resulting time dependent Poisson parsmrter which
$C^*$ has at every time.
If a reward for a discovery only is given once, this means that our company
is faced with a new optimization problem since
expected net profit from reseach decays while the
opponent is active.
Models of this kind appear in "real life" and  require  a considerable effort to
solve 
a  stochastic optimization of this kind.


















\newpage




\centerline{\bf{Optimizing consumption of two
commodities.}}

\bigskip

\noindent
A consumer can share a given capital
to buy two commodities.
Let $x$ snd $y$ be the amount
of them which gives a utility expressed by
$U(x,y)$. Here $U$ is a function defined for pairs
of positive numbers. We assume that
the first order partial derivatives 
$U_x$ and $U_y$ both are $>0$ for all points
$(x,y)\in{\bf{R}}_+^2$.
In addition 
$U$ is such
that for every
$A>0$ the level curve
$\{U=A\}$
can be expresed by a decreasing function
\[
x\mapsto y_A(x)
\]
whose  second derivative $\frac{d^2y_A}{dx^2}>0$
for all $x>0$.
Examples of such utilitiy functions arise as follows:
For every $m\geq 1$ and
a pair of $m$-tuples $\{0<a_k<1\}$ and $0<b_k<1\}$
and an arbitrary $m$-tuple
$\{c_k\}$ of positive numbers, we set
\[
U(x,y)= \sum\, c_k\cdot x^{a_k}\cdot y^{b_k}
\]
Next, for
each $A>0$ wee consider the open
set
$\{U>A\}$ and the reader should  verify that
this set is convex.
\medskip

\noindent
Given a utility function $U$ ss above we
consider the optimmzation problem
\[
\max_{xy}\, U(x,y)\tag{*}
\]
where the maximum is taken over pairs
$(x,y)$ which satisfy a  budget constraint
\[|
px+qy=A
\]
Here $p$ and $q$ are prices, i.e. both are $>0$.
The maximum problem has a unique
solution, i.e. for every triple $p,q,A$ there exists 
a unique pair $(x^*,y*)$ which maximises (*).
Keeping $p$ and $q$ fixed we denote the maximum in
(*) by $V(A)$ and then
$A\mapsto V(A)$ is an increasing functioin.
It turns out that the derivstive
$V'(A)$
coincides with Lagrange's multiplier. More precisely,
when (*) is maximized there exists $\lambda>0$ and
\[
U_x(x^*,y^*)=\lambda\cdot q\quad\&\quad
U_y(x^*,y^*)=\lambda\cdot p
\]
and then one has the equality
\[
\lambda= V'(A)\tag{**}
\]
The reader should verify (**)  and we remark that
(**) appears as a  popular equation in
text-books devoted to
economics since
it gives an interpretation of Lagrange's multiplier.
\medskip


\noindent
{\bf{Nujmerical experiments.}}
Explicit  solutions are easily found
when
\[
U(x,y)=x^ay^b
\]
for a pair $0<a,b<1$.
The reader is invited to treat the case
$a=b=1/2$ and analyze how
$(x^*,y*)$ changes with a price vector
$(p,q)$ while $A$ is kept fixed.
Passing to a more general  utilitiy function one
must rely upon
the computer which gives numerical solutions.
A specific case is when
\[
U(x,y)\sqrt{xy}+ ax+by
\]
where $a,b$ are teo positive constants.
With $A=1$ and fixed equal prices
$p=q=B$ the reader is invited to
see how $(x^*,y^*)$ changes with the pair $a,b$ and 
evaluate numerically the  maximal value
$V$ which with $A$ and $B$ kept fixed becomes a function of
$a$ and $b$.
\medskip


\noindent
{\bf{Remark.}} It goes without  saying that
the results above can be extended to the case
when $n\geq 3$ and the consumer
distributes the incomle to buy $n$
commodities  where a orice vector
$p_1,\ldots,p_n$ is given.
So here a function $U(x_1,\ldots,x_n)$ is
defined when the $n$-vector has $x_k>=$ for each $k$.
The optmisation problem for a given price vector and  income
has a unique  solution $x^*$
when $U$ is such that 
the first  order partial derivatives are  $>0$.
and the sets
$\{U>C\}$ are strictly convex for every $C>0$.



\newpage


\centerline{\bf{Composed ODE:s}}

\medskip

\noindent
Consider the  variational problem
\[
\min\, \int_0^T\, [e^{rt}x(t)^2+\dot x(t)^2]\, dt\tag{i}
\]
with fixed end-values $x(0)=0$ and $x(1)=1$, while $r$ is a constant which can be
$<0$ or $>0$.
The Euler equation becomes
\[
\ddot x(t)= e^{rt}x(t)\tag{i}
\]
Without solving (i) explicitly one cank keep trace of
this function and  introduce another ODE where
 a time dependent function $z(t)$ satsfies 
 \[
 \dot z(t)=e^{rt}x(t)^2+\dot x(t)^2\quad\&\quad z(0)=0
 \]
 Now one gets a numerial solution of the $z$-function and
 obtsan plot over given  time intervals $[0,T]$.
 In this way the minmal values in our variational problem
 can be found numerically while the time intevral $T$ varies.

\medskip

\noindent
The reader is invited to
perform numerical experiments and also 
confirm them after
having computed (by hand) the analytic solutions
to our  given problem.
To be precise, the Euler equstion has the solution
\[
x(t) C(e^{\sqrt{r}t}-1)
\]
where the constant $C$ is determined by the end-value $x(T)=1$,
and after this the reader can try to exhibit  a "closed formula"
for the $z$-function.
'\medskip


\noindent
{\bf{A secomd example.}}
Here one regsrds an optimization  with a salvage value:
\[
\max\, \int_0^T\, \sqrt{\dot x(t)}\cdot e^{-rt}\, dt+A(K-x(T))\tag{*}
\]
where $r$ and $A$ are $>0$ and
the maximum is found in the family  of
non-negative functions $c(t)$ defined on
$[0,T]$ with the  constraint 
\[
x(T)= \int_0^T\, \cdot x(t)\, dt\leq A
\]
The Euler equstion gives
the following ODE for an extremal $x$-function:
\[
\dot x(t)= C\cdot e^{-2rt}
\] 
where $C$ is a  constant which must satisfy
\[
C\cdot \int_=^T\, e^{-2rt}\, dt\leq A
\]
and then $C$ is  found via the salvage term while  one maximizes
(*). To carry out the computations and
get an equation for trhe maximal value $V$ when
$r,T,A$ are given is rather involved.
Using the computer  the
reader is invited to perform numerical experiments.
To be orecise, using the \emph{transversality equation}
when a salvage value is present, the
reader
can get hold of the constabnt $C$ and after
find a numerical vaklue fior $V$ by regarding the
$z$-function
which satisfies the ODE:
\[
\dot z= e^{rt}\cdot \dot x(t)^2
\]
This example  illustrates a general
situation where one can perform numerical investigations in
more general cases.
Let us also remakr that
apart from the standard device which for example is implemented in
Mathemacitca, it often requires extra  "organisation"
to
express the compound ODE:s which  evaluates $z(T)$
while $r,T,A$ are nuumerically assigned  from the start.









 
























\newpage


\centerline{\bf{ODE:s
and the use of computers.}}


\bigskip

\noindent
Very few ODE:s can be solved "nalayitcslly", i.e. via
expressions of established functions.
So in most applications one must rely upon numerical solutions.
At the same time experiments by computers
can suggest new theoretical results.
Let us illuminate this by discussing
a second order ODE of the form
\[
\ddot y=y+b(t)
\]
where we seek solutions defined for $t\geq 0$.
If we for example takee $b(t)= a$ for some constant
$a$
then we get the soluti
\[
y(t)= -a+a\cdot e^{-t}
\]
where $y(0)=0$ and
\[
\lim_{t\to +\infty}\, y(t)=-a
\]
So this function is decreasing but converges to a finite negative number.
Suppose now thst
$b(t)$ is defined on $[0,+\infty)$ where
\[
0\leq b(t)\leq 1
\]
hold for every $t$.
Now one can ask if the ODE:

\[
\ddot y=y+b(t)
\]
has a solution where  $y(0)=0$ and
a finite limit as in (xx) exist, i.e. 
\[
\lim_{t\to +\infty}\, y(t)=A
\]
hold for some real number $A$.
This problem is a challenge since
we cannot
solve the ODE explicitly.
It is therefore tempting to
perform  experiments by a computer.
One such experiment can be done as follows:
With a large positive integer, say $N=10^3$
we define $b(t)$ as follows.
on the interval $[1,N]$
in a random way , i.e. on  each
interval $[k,k+1]$ we let $b(t)$ be +1 or 0, where
the choice of the sign is random and picked form
a Monte Carlo box.
So here $b(t)$ is picewise constant with jumps at
the integers $1,2,\ldots,N$. When $t>N$ we declare  that
$b(t)=0$.
Fir each such $b$-function we find
a solution $y(t)$ where $y(t)$ is contonoud on
$[0,+\infty)$ with 
$y/(0)=0$ and 
\[
y(t)= y(N)e^{N-t}\quad\colon t\geq N
\]
















\newpage


\centerline {\bf{Calculus of variation.}}
\bigskip


\noindent
We are given a function
$f(x,z, t)$
of three variables.
For every $C^1$-function $x(t)$ of the variable $t$ we
assign its derivative $\dot x(t)$ and obtain the function
\[
 t\mapsto f(x(t),\dot x(t),t)
\]
Let $[0,T]$ be a given interval and $A$ is a constant. Denote by
$\mathcal C^1(A)$ the family of all $C^1$-functions
$x(t)$ on $[0,T]$ such that
$x(0)=0$ and $x(T)=A$.
To each such function we set
\[ 
\mathcal F(x)=\int_0^T\, f(x(t),\dot x(t),t)\cdot dt\tag{*}
\]
Here $x\mapsto \mathcal F(x)$ is a  function whose domain
of definition is $\mathcal C^1[A]$ and one refers to $\mathcal F$ as the 
evaluating functional.
Next, a \emph{bubble function} is  a $C^1$-function $\phi(t)$ with 
$\phi(0)=\phi(T)=0$.

\medskip

\noindent
{\bf{A. Exercise.}}
Use Taylor expansion to show
that
if $x\in\mathcal C^1[A]$ and $\phi$ is a  bubble function, then
\[
\frac{\mathcal F(x+\epsilon\cdot\phi)-\mathcal F(x)}{\epsilon}
=\int_0^T\, \bigl[\phi(t)\cdot f'_x(x(t),\dot x(t),t)+
\dot \phi(t)\cdot f'_{\dot x}(x(t),\dot x(t),t)\,\bigr]\cdot dt+o(\epsilon)\tag{i}
\] 
where $o(\epsilon)$ is small ordo, and show 
that partial integration gives
\[
\int_0^T\, 
\dot \phi(t)\cdot f'_{\dot x}(x(t),\dot x(t),t)\,\cdot dt=
\phi\cdot f'_{\dot x}|_0^T-\int_0^T\,
\phi(t)\cdot \frac{d}{dt}\bigl(f'_{\dot x}(x(t),\dot x(t),t)\bigr)\,\cdot dt\tag{ii}
\]
Since $\phi$ is a bubble function the first term above vanishes and (i\vvv ii)
give
\[
\frac{\mathcal F(x+\epsilon\cdot\phi)-\mathcal F(x)}{\epsilon}=
\]
\[
\int_0^T\, \bigl[\phi(t)\cdot [f'_x(x(t),\dot x(t),t)-
\frac{d}{dt}\bigl(f'_{\dot x}(x(t),\dot x(t),t)\,\bigr)\,\bigr]\cdot dt+o(\epsilon)\tag{iii}
\]
Passing to the limit  $\epsilon\to 0$
we can ignore the small ordo-term and obtain
\[
\lim_{\epsilon\to 0} \frac{\mathcal F(x+\epsilon\cdot\phi)-\mathcal F(x)}{\epsilon}=
\int_0^T\, \bigl[\phi(t)\cdot [f'_x(x,\dot x,t)-
\frac{d}{dt}\bigl(f'_{\dot x}(x,\dot x,t)\,\bigr)\,\bigr]\cdot dt\tag{iv}
\]
\medskip


\noindent
{\bf{A.1 Extremal $x$-functions.}}
Suppose that $x^*(t)$ is extremal in the sense that
\[ 
\mathcal F(x)\leq \mathcal F(x^*)
\] 
hold for all competing $\mathcal C^1(A)$-functions.
Then (iv)  must be zero for all bubble-functions
$\phi$, i.e.  the right hand side vanishes for all such $\phi$. This 
implies that the function
\[
t\mapsto f'_x(x^*(t),\dot x^*(t),t)-
\frac{d}{dt}\bigl(f'_{\dot x}(x^*(t),\dot x^*(t),t)=0\quad\text{for all}\quad 0\leq t\leq T
\tag{A.1.1}
\] 

\medskip

\noindent
One refers to (A.1.1) as the Euler equation which from the above
must  to be satisfied by every extremal 
$\mathcal F$-function $x^*(t)$.
\medskip

\noindent
Notice that
the same conclusion holds if we instead suppose that
$x_*(t)$ is an extremal function in the sense that
\[ 
\mathcal F(x)\geq \mathcal F(x^*)
\] 
hold for all competing functions. Thus, the Euler equation is
necessary for an extremal which yields a maximum or a minimum
to the variational problem.
\medskip


\noindent
{\bf{A.2 Exercise.}}
Consider the case when
$f=f(x,\dot x)$ is independent of $t$.
Use rules for differentiation to show the
equaluity below for every
function $x)t)$:
\[
\dot x(t)\cdot (f'_x(x(t),\dot x(t))-
\frac{d}{dt}\bigl(f'_{\dot x}(x(t),\dot x(t)))=
\]
\[
\frac{d}{dt}[(f(x(t),\dot x(t))-
\dot x(t)\cdot f'_{\dot x}(x(t),\dot x(t))]
\] 
So ignoring very special solutions when
$\dot x(t)=0$ on some intervsl, the EWuler equation for
an extremal solution is equivalent to
the existence of a constant $C$ such that
\[
f(x(t),\dot x(t)-
\dot x(t)\cdot f'_{\dot x}(x(t),\dot x(t))=C\tag{A.2.1}
\]









\bigskip


\noindent
{\bf{B. Examples.}}
Consider the variatioinal  problem with
\[
\mathcal F(x)= \int_0^T\, (\dot x^2+ax^2)\, dt\tag{B.1}
\]
where $a$ is a real number which can be $>0$ or $<0$, and
the end-values are $x(0)=0$ and $x(T)=1$.
It is tempting to seek a minimum.
To begin with the reader csn check thst the Euler equation becomes
\[
\ddot x=ax
\]
If $a>0$ this second order ODE has the solutlion
\[
x(t) =\frac{1}{e^T-e^{-T}}\cdot (e^t-e^{-t})
\]
which satisfies the end-value conditions.
The case when
$a<0$ leads to a more involved
analysis.
To begin eith
Euler's equation has the general solution
\[
x(t)=c_1\cdot \sin\,\sqrt{a}t+
c_2\cdot \cos\,\sqrt{a}t
\]
The conditiion that $x(0)=0$ gives $c_2=0$ and then
\[
1=x(T)= c_1\cdot  \sin\,\sqrt{a}T\tag{i}
\]
If we spoecify $a=-1$ and $T=\pi$
then (i) \emph{cannot be satsified}
because $\sin \,\pi=0$.
So this example shows that a variational problem can be ill-posed
and
fail to have extremal solutions.
\medskip


\noindent
{\bf{Exercise.}}
Take $a=-1$ and let $T=\alpha\cdot \pi$ for some
$0<\alpha<\pi$.
Kepinig the end-values $x(0)=0$ and $x(T)=1$
we find Euler's solution
\[
x_*(t)= \frac{\sin t}{\sin T}
\]
and the reader can check that
\[
\mathcal F(x_*)=\int_0^T\,( \cos^2t-\sin^2t)\, dt
\]
Now
\[
\int_0^T\,( \cos^2t-\sin^2t)\, dt=\int_0^T\, \cos 2t\, dt=
\frac{\sin 2T}{2}
\]
As an example we take
\[
T=\pi-\delta
\]
where $0<\delta<\pi/2$.
When $\delta\to 0$ the reader should verify that
\[
\lim_{\delta\to 0}\, \mathcal F(x_*)=-\infty
\]
From this one can conclude that
the varistional problem is ill-posed when
$T=\pi$ in the sense that
\[
\min_x\, \mathcal F(x)=-\infty
\]
when we take competing functions
$x(t)$ for which $x(0)=1$ and $x(\pi)=1$.



\newpage


\noindent
\centerline{\bf{Legendre's test.}}

\bigskip


\noindent
With $0<T<\pi$ we have found the Euler solution
$x_*(t)$ above.
Recall that  Euler*s  equation only asets a necessary condition for
an extremal. So there remains  to investigate if 
$x_*(t)$ is a global minimum, which amounts to show that
\[
\mathcal F(x_*)<\mathcal F(x_*+\phi)
\]
hold for every non-zero bubble function $\phi$ on $[0,T]$.
To prove that this indeed holds
the reader should fiurt verify the equation
\[
\mathcal F(x_*+\phi)=\mathcal F(x_*)+
\int_0^T\, (\dot \phi^2-\phi^2)\, dt
\]
Hence (x) holds  if 
\[
\int_0^T\, \phi^2\,dt<\int_0^T\, \dot \phi^2\, dt\tag{*}
\]
hold for every non-zero bubble function.
To prove  that (*) holds  
one can empoy series examnsiuons of bubble functions.
More precisely, every function
$\phi(t)$ defined on the interval $[0,\pi]$
for ehich
$\phi(0)= \phi(\pi)$ has a sine-series expansion
\[
\phi(t)= \sum_{k=1}^\infty\, c_k\cdot \sin kt
\]
where the $c$-numbers are determined by
\[
c_k=\frac{2}{\pi}\cdot \int_0^\pi\, \sin kt\cdot \phi(t)\, dt
\]
The reader is invited to deduce (*) from  this
wellknown fact from the theory about Fourier series.
One can also confirm (*) by experiments on a computer.
Namely, consider polynopmials $P(t)$ eith real coefficiets of
the form
\[
P(t)= t(t-1)\cdot p(t)
\]
where $p(t)$ is an arbitrary polynomial and via a Monte-Carlo box
one picks a fsmily of polynomials $\{p(t)\}$ and
check that (*) holds numerically.


\bigskip

\centerline{\bf{Legendre's conditions.}}


\bigskip


\noindent
The examples above show thst
in general one ust be careful while a variational problem
is studied, i.e. the Euler equstiongives only a necessary condition for an extremal, and it may even
occur that
no extremal solution exists which satisfied  prescribed end-values.
But in many applications there exists  a unique extremal solution
which satisfies Euler's equation.
Of course, one should distinguish between maxima or minima.
Follwing a classic stufy by Legendre we
impose the following   condition on
the function
$f(t,x,y)$ which
where $(x,y)$ vary in
${\bf{R}}^2$ while $0\leq t\leq T$.
\[
a^2\cdot f''_{xx}(t,x,y)+
2b\cdot f''_{xy}(x,y,t)+ c^2\cdot f''_{yy}(t,x,y)>0\tag{L.1}
\]
hold for every triple of real numbers
$a,b,c$ where at least one is $\neq 0$
and for all $(x,y,t)$.
\medskip


\noindent
{\bf{Exercise.}}
Suppose that $x_*(t)$ solve Euler's equation
and ket $\phi$ be a bubble function.
For every real number $s$ we put
\[
J(s)= \int_0^T\, f(t,x(t)+s\phi(t),\dot x(t)+s\dot \phi(t))|, dt
\]
Show that (L.1) entals that the second derivative
$J''8s)>0$ for every $0\leq s\leq 1$. At the same time
Euers equation for $x_*(t)$ entails  that
the derivative $J'(0)=0$.
Conclude via the general result below that
$x_*(t)$ yields a globsl minium.
\medskip

\noindent
{\bf{Exercise.}} Let $g(s)$ be a $C^2$-.function on
thr unit interval�$[0,1]$ with $g'(0)=0$ and $g'')s)>0$
for every $s$. Show that
\[
g(0)<g(1)
\]
\medskip


\noindent
{\bf{Remark.}}
If $>0$�is replaced by $<0$ we find exactly as above
a unique maximizing function $x^*(t)$.
Let us remark  that (L.1) means that the symmetric $2\times 2$-matrix
\[
xxx
\]
is positive definite  for all $t,x,y)$.
In the literature one often refers to
this as the positive Hessian condition for
the function $(x,y)\to f(t,x,y)$ which
is regarded for every fixed time-vslue.
Geometrically it means  that the function
\[
(x,y)\mapsto f(t,x,y)
\]
is stricly concave for every fixed $t$.
Notice  also that unique minimimal funcions
$x_*(T)$ csn exist even when
Legendre�s condition fails. We have seen such examples with
\[
f(t,x,y)= x^2-ay^2
\]
where $a>0$. 
But in most applicstions one of the teo legendre conditions
are satisfied which  gives
a unique minimizing .or maximizing - solution to the
variatinsl problem.
Excepy for a few cases one cannot find
$x^*$ or $X_*$ explicitly, i.e. one must rely upon
numerical experminetns. 
\bigskip

\noindent
{\bf{An example.}}
Let
\[
f(t,x,y)=y^2+x^4
\]
and with $T>0$ and $A>0$ we regard the variational probelm
\[
\min_x\, \int_=^T\, (\dot x^2+x^4)\, dt
\] 
with $x(0)=0$ and $x(T)= A$.
Euler's equation becomes
\[
\ddot x= 4x^3
\]
This is a non-linear second order ODE.
If one assigns a  numerical value for
the time
derivative, 
\[
\ddot x(0)=a
\]
for some $a>0$ then the computer  plots
a solution and one evaluates numerically $x(T)$.
usin the so called Back-Shooting one determines $a$ via
the end-value condition $x(T)=A$, and the reader is invited
to find numerical solutions 
for various pairs of $A$ and $T$.
\medskip


\noindent
{\bf{The value function
$V(T,A)$.}}
Keeping $f$ as above
one gets $x_*(t)$ and evaluate $\mathcal F(x_*)$ which
depends on $A$ and $T$.
In many spplications one os foremost interested in
this $V$-function.
So as an extra exercise the reader is invited to
evalue $V(T,A)$ numericslly with the aid of
solutions $x_*(t)$ while varoius numercaik vakues for
�$A$ snd $T$ are regarded.
A speciic case os to let $T=1$ be fixed while $A$ varies.
it is thrn vlerar thst the function
\[
A\mapsto V(1,A)
\]
increases with $A$ but its shape can only be revealed numerically.
\bigskip


\noindent
{\bf{A more special case.}}
it occurs if
$f(x,y) x^+y^2$, i.e with $T=1$ we consider
\[
\min_x\, \int_0^1\, (\dot x^2+x^2)\ dt
\]
Here we find the oinear Euler equation
\[
\ddot x= x
\]
and it can be solvled expocityl with
end-values $x(0)=0$ and $X(1)=A$ which after yields 
an explicit equation for
$V(1,A)$ as a function of $A$.


\bigskip

\noindent
{\bf{Exercise,}}
In the previous examples no discount factors  were
introduced.
A typcal example in economis is to regard
\[
\min_x\, \int_0^T\, e^{rt}\cdot \dot x^2\,dt
\]
with $x(t)=0$ and $x(T)=A$.
The Euler equation entails that
\[
t\mapsto e^{rt}\cdot x(t)
\]
is a constant function of $t$.
From this the reader can check that
\[
x(t)= C\cdot (1-e^{-rt})
\]
where the constant $C$ satisfies
\[
C(1-e^{-rT})=A
\]
Let us instad regard the variational problem
\[
\min_x\, \int_0^T\, (e^{rt}\cdot \dot x^2+ax)\,dt
\]
where $a>0$ with end-valued $x(0)=0$ snd $x(T)=A$.
The reader is invited to find the minimizing solution
$x_*(t)$ and investigate numerically the value function
$V$ where we take $T=1$ while $r,A$ and $a$ vary.
For example, hos is $V$ affected by $a$ ?



\bigskip

\centerline{\bf{C. Variation of end-values.}}
\medskip

\noindent
Assume  Legendre's concavity condition (l.1)  which gives
a unique
function $x_*(t)$  minimizing $\mathcal F$.
Now we shall  study  variations with respect to $A$ and   the time
horizon $T$ while the minimum value is taken.
So for  each pair $A,T$ Legendre's result gives
a unique  extremal function
$x^*_{A,T}$ defined on $[0,T]$ and
we 
define the
value function
\[ 
V(A,T)=\mathcal F(x^*_{A,T})
\]

\noindent
When   $\delta A$ is small,  pertubations of
solutions to an ODE of the Picard type give the existence
of  a
function $\xi(t)$ such that

\[
x^*_{A+\delta A,T}=
x^*_{A,T}+\delta A\cdot \xi(t)+o(\delta A)\tag{i}
\] 

\noindent
Using (i) a first order Taylor expansion gives:

\[ 
\frac{V(A+\delta A,T)-V(A,T)}{\delta A}= 
\]
\[
\int_0^T\, [\xi(t)\cdot f'_x(x^*,\dot x^*,t)+
\dot \xi(t)\cdot f'_{\dot x}(x^*,\dot x^*,t)]\cdot dt+o(\delta A)\tag{ii}
\]
A partial integration and using that $x^*$ satisfies the Euler's  equation
implies that the integral above becomes
\[
\xi(T)\cdot f'_{\dot x}(x^*(T),\dot x^*(T),T)\tag{iii}
\]
At the same time 
(i) and the equality
$x^*_{A+\delta A,T}= A+\delta A$ entail that
$\xi(T)=1$. Passing to the limit as $\delta A\to 0$ we have therefore proved:

\medskip

\noindent
{\bf{C.1 Theorem.}}
\emph{One has the equality}

\[
\frac{\partial V}{\partial A}(A,T)=
f'_{\dot x}(x^*_{A,T},\dot  x^*_{A,T}, T)
\]
\medskip

\noindent
{\bf{C.2 Exercise }} Use Theorem C.1 and similar 
arguments as above to show that

\[
\frac{\partial V}{\partial T}(A,T)=f(x^*_{A,T},\dot  x^*_{A,T}, T)-
\dot x^*_T(A,T)\cdot
f'_{\dot x}(x^*_{A,T},\dot  x^*_{A,T}, T)
\]
\bigskip


\noindent
{\bf{C.3 Exericise.}}
Consider the variational problem
\[ 
\min_x\, \int_ 0^1\, (x^2+\dot x^2)\, dt
\]
Let $x(0)=1$  while the end-value
$x(1)$ is free.
The Euler equation becomes
\[ 
\ddot x=x
\]
With $x(0)=1$ the general solution is
\[
x(t)= Ae^t+(1\vvv A)e^{\vvv t}\tag{i}
\]
Since $x(1)$ is free
the partial derivative
$\frac{\partial V}{\partial A}$ in
Theorem C.1 is zero a $T=1$. Since
$f\uuu{\dot x}= 2\dot x$ it follows that
$A$ should be chosen so that
\[
\dot x(1)= Ae+Ae^{\vvv 1}=0\implies A=0
\]
Hence the extremal function becomes 
\[ 
x_*(t)= e^{\vvv t}\tag{ii}
\]
If we try to prove (ii) without Theorem C.1 we
consider some function from
(i) and evaluate
\[
\int_0^1\, 
[(Ae^t+(1-A)e^{- t})^2+
(Ae^t+Ae^{-t})^2]\, dt
\]
This integral is a quadratic polynomial 
of $A$ and the reader can check that it takes a minimum when
$A=0$ which confirms the solution (ii) and at the same time
illustrates that Theorem C.1 simplifies computations in
situations  of a free terminal value.
\medskip



\centerline{\bf{D. Constrained Euler equations.}}
\medskip


\noindent
A new clas  of variational problems arise when
we in addition to
boundary values $x(0)=0$ and $x(T)=A$ impose a further constraint that
\[
\int_ 0^1\, g(x,\dot x,t)\, dt=B\tag{D.1}
\] 
for a constant $B$ and a given  $g$function of three variables.
An example is the isoperimetric problem  
where
$t$ is replaced by $x$ and we seek a curve
$y(x)$ with $y(-1)= y(1)=0$
and the  constraint 
\[ 
\int_0^1\, \sqrt{1+y'(x)^2}\, dx=L
\]
while $f=y$.
Before we treat this special case we consider the general 
problem where the pair $f,g$ are arbitrary.
Exactl  as before  one works with bubble-functions
$\phi$ and given some $x(t)$ for which (D.1) holds
we take
$x_\epsilon (t)+\epsilon\cdot \phi(t)$ . The new feature is that
(D.1) again should be satisfied.
With a small $\epsilon$ a Taylor expansion gives
\[
\int_0^1\, g(x\uuu\epsilon,\dot x\uuu\epsilon,t)\, dt=
B+
\int_0^1\, [\, 
g'\_x(x,\dot x,t)\cdot \phi+
g'\_{\dot x}(x,\dot x,t)\cdot\dot  \phi]\, dt
\]
After a partial integration of the last term we get the equation
\[
\int_0^1\, [\, 
g'_x(x,\dot x,t)\vvv\frac{d}{dt}(
g'_ {\dot x}(x,\dot x,t)]\cdot  \phi\, dt=0\tag{D.2}
\]
When this holds we say tha
$\phi$ is an admissable bubble-function with respec to 
$x(t)$.
If $x_*(t)$ is an extremal for the variational problem
one employs  admissable bubble-functions 
and repeating the previous arguments from  when Euler's equation is derived
it follows that
\[
\int_0^1\, [\, 
f'_x(x,\dot x,t)-\frac{d}{dt}(
f'_{\dot x}(x,\dot x,t)]\cdot  \phi\, dt=0\tag{D.3}
\]
So the vanishing in (D.2) implies that (D. 3) also vanishes
and then an elementary argument
which is left as an exercise to
the reader gives the  conclusive result:
\medskip

\noindent
{\bf{D.4 Theorem.}}
\emph{If $x^*(t)$ is an extremal there exists a constant
$c$ such that}
\[
f'_x(x^*,\dot x^*,t)\vvv\frac{d}{dt}(
f'_{\dot x}(x^*,\dot x^*,t)= c\cdot [
g'_x(x^*,\dot x^*,t)-\frac{d}{dt}(
g'_{\dot x}(x^*,\dot x^*,t)]
\]
\newpage



\centerline{\bf{Specific problems.}}

\bigskip


\noindent

{\bf{Problem A.}}
Consider the variational problem
\[
\min_x\, \int_0^1\, [\dot x^2(t)+ e^{\mu t}x(t)]\, dt
\]
with end-values $x(0)=0$ and $x(1)=1$.
Above $\mu$ is a real number.
Use Euler's equation
to find the minimizing $x$-function which amounts to
solve the ODE:
\[
2\cdot \ddot x=2e^{\mu t}x
\]
The minimium value is denoted by $V(\mu)$ which 
the reader should  express  as a funtion of
$\mu$, where $\mu$ can be either $>0$ or $<0$.
\medskip


\noindent
{\bf{Problem A.1}}
Do the same problem where we now
seek
\[
\min_x\, \int_0^1\, [\dot x^2(t)+  A\dot x\cdot x\cdot e^{\mu t}x(t)]\, dt
\]
where $A$ is another constant.
A hinit is that Euler's ODE for the differential equation is
the same as when
$A=0$. while the value function can depend on $A$.

\bigskip

\noindent
{\bf{Problem B.}}
Seek
\[
\max_c\, \int_0^\infty\, \sqrt{c(t)}\cdot e^{-rt}\,dt 
\]
where the competing $c$-functions are
$\geq 0$ and
\[
\int_0^\infty\, c(t)\,dt=A
\] 
hold for a positive constant $A$.
\bigskip


\noindent
{\bf{Problem C.}}
Here we regard an optimization with
a salvage value and seek
\[
\max_x\, \int_0^1\, \sqrt{\dot x}\, dt+A(K-x(T))
\] 
where $x(0)=0$ and
the end value satisfies $0\leq x(T)\leq K$, while
$A$ och $K$ are   positive  constants.
\bigskip


\noindent
{\bf{Problem D (harder)}}
Consider the variational problem
\[
\min_x\, \int_0^\pi\, [\dot x(t)^2-(\pi-t)x^2(t)]\, dt
\]
with end-value conditions $x(0)=0$ and $x(\pi)=1$.












\end{document}





