
\documentclass{amsart}
\usepackage[applemac]{inputenc}


\addtolength{\hoffset}{-12mm}
\addtolength{\textwidth}{22mm}
\addtolength{\voffset}{-10mm}
\addtolength{\textheight}{20mm}

\def\uuu{_}

\def\vvv{-}

\begin{document}

\centerline{\bf Bochner's moment theorem}

\bigskip

\noindent
In probability theory
the distribution of a 
stochastic variable is expressed
by a probability measure
$\mu$ on the real line where we take $t$
as the coordinate. The characteristic function
is by definition the Fourier transform
of $\mu$ and we set
\[ 
f(x)=\int\, e^{-ixt}\cdot d\mu(t)\tag{*}
\]
Since $\mu$ has total mass one we get
$f(0)=1$.
Moreover, let
$x_1,\ldots,x_N$ be some $N$-tuple of real numbers
and $\alpha_1,\ldots,\alpha_N$ some $N$-tuple of complex numbers.
Then 
\medskip
\[ 
\sum\sum\, f(x_p-x_q)\alpha_p\cdot\bar\alpha_q=
\int\,\bigl |\sum\, \alpha_p\cdot e^{-ix_p\cdot t}\bigr|^2d\mu(t)\tag{**}
\]
Since $\mu\geq 0$ the
right hand side is $\geq 0$.
It turns out that this inequality characterizes the family of
bounded continuous functions $f(x)$ which are Fourier
transforms of non-negative measures. To make it precise we give
\medskip

\noindent
{\bf Definition.}
\emph{Denote by $\mathcal{B}$ the class of
continuous  functions $f(x)$ on the real $x$-line such that}
\[
\sum\sum\, f(x_p-x_q)\alpha_p\cdot\bar\alpha_q\geq 0\quad\colon\quad
f(0)=1
\]
\emph{where the inequality holds for all pairs of $N$-tuples $x_\bullet$ and $\alpha_\bullet$
as above.}
\medskip

\noindent
{\bf Remark.}
Given $x>0$ we take $N=2$ with $x_1=0$ and $x_2=x$
and $\alpha_1=1$ while $\alpha-2=e^{i\theta}$.
Then Bochner's condition gives
\[
2\cdot f(0)+ e^{i\theta}f(x)+e^{-it\theta}f(-x)\geq 0\tag{i}
\]
With  $\theta=\pi/2$ it  follows that
$f(-x)=\bar f(x)$ and then the inequality (i)   gives
\[ 
|f(x)|\leq f(0)=1\tag{ii}
\]
So functions in $\mathcal B$ are automatically bounded.
Now we announce Bochner's  result.
\medskip

\noindent
{\bf 1. Theorem.} \emph{For each $f\in\mathcal{B}$ there exists
a unique non-negative measure $\mu$ such that}
\[
f(x)=\int e^{-ixt}d\mu(t)
\]



\noindent
{\bf Remark.} Theorem 1 
appears in the book [Boch]
\emph{Vorlesungen �ber Fouriersche Integrale} from 1932.
The essential ingredient in the proof
is a representation formula for positive harmonic functions
in the upper half-plane.
Prior to Bochner's
result the periodic version of Theorem 1
was established by
G. Herglotz who proved the following
in    [Herg] 
from 1911:
\medskip

\noindent
{\bf 2. Theorem.} \emph{Let $\{m_n\,\colon -\infty<n<\infty\}$
be a sequence of complex numbers.
In order that there exists a non-negative Riesz measure
$\mu$ on the interval $[0,2\pi]$ such that}
\[ m_n=\int_0^{2\pi}\, e^{in\theta}\cdot d\mu(\theta)
\]
\emph{it is necessary and sufficient that}
\[
\sum_{\nu=-N}^{\nu=N}\sum_{j=-N}^{j=N}\, m_{\nu-j}\cdot \alpha_\nu\cdot\bar\alpha_j\geq 0
\]
\emph{holds for and finite sequence of complex numbers}
$\alpha_{-N}\ldots,\alpha_N$.


\bigskip

\noindent
To prove Bochner's theorem we shall need
the following result:
\medskip

\noindent
{\bf 3. Proposition.}
\emph{For each pair of real numbers
$\xi,\eta$ with $\eta>0$
there exists a function $\phi(x)$ in
$L^1({\bf{R}})$
such that}
\[
e^{-i\xi x-\eta|x|}= \int_{-\infty}^\infty\,
\phi(x+y)\cdot\bar \phi(y)\cdot dy\quad\colon\, -\infty<x<\infty
\]
\medskip

\noindent
\emph {Proof}
Set
\[
\phi(x)=\sqrt{\frac{1}{2\pi}}\cdot \int\, e^{itx}\cdot 
\sqrt{\frac{1}{2\pi}}\cdot 
\sqrt{\frac{\eta}{\eta^2+(\xi+t)^2}}
\cdot dt\tag{ii}
\]


\noindent 
The reader can verify that
$\phi(x)\in L^1({\bf{R}})$
and that
the equality in Proposition 3 holds.
\medskip

\noindent
{\emph{ Proof of Theorem 1.}}
Let $f\in\mathcal B$ be given and put
\[ 
\Phi(\xi,\eta)=\int_{-\infty}^\infty\, 
e^{-i\xi x-\eta|x|}\cdot
f(x)\cdot dx\quad\colon\, \xi\in{\bf{R}}\quad\colon\eta>0\tag{1}
\]



\noindent
Proposition 3 gives:
\[
\Phi(\xi,\eta)=
\iint\,
\phi(x+y)\cdot\bar \phi(y)\cdot f(x)\cdot dxdy=
\]
\[
\iint\,
\phi(x)\cdot\bar \phi(y)\cdot f(x-y)\cdot dxdy
\]
Since both $f$ and $\phi$ belong to $L^1$ we can approximate
the last double integral by Riemann sums
which  are of the form
\[
\sum\, f(x_p-x_q)\cdot\alpha_p\bar \alpha_q
\]
The hypotehsis that
$f\in\mathcal B$ therefore
implies that
the $\Phi$-function is $\geq 0$.
Next, for each fixed $x$ we consider the function
\[ 
(\xi,\eta)\mapsto e^{-i\xi x-\eta|x|}\tag{*}
\]
Since $i^2=-1$ we see that this function is harmonic. Approximating
the integral (1) by Riemann sums we conclude that
$\Phi(\xi,\eta)$ is a harmonic function in the
upper half-plane
$\eta>0$.
Since $|f(x)|\leq 1$ for all $x$ and $|e^{-ix\xi}|=1$
the
the triangle inequality gives
\[ |\Phi(\xi,\eta)|\leq
\int_{-\infty}^\infty\, 
e^{-\eta|x|}\cdot
dx=
\frac{2}{\eta}\tag{i}
\]


\noindent
Now $\Phi$ is harmonic and $\geq 0$
in the upper half-plane. Hence  the inequality (i) and
the general result in � XX
gives a non-negative measure $\mu$ of finite total mass such that
\[ 
\Phi(\xi,\eta)=\frac{1}{\pi}\cdot
\int_{-\infty}^\infty\, \frac{\eta}{\eta^2+(\xi-t)^2}\cdot d\mu(t)\tag{2}
\]


\noindent
With $\eta>0$ kept fixed we notice that (1) means that the function
$\xi\mapsto \Phi(\xi,\eta)$ is the Fourier transform of
$e^{-\eta|x|}f(x)$. Hence  (2) and Fourier's inversion formula 
yield:
\[
e^{-\eta|x|}f(x)=
\frac{1}{2\pi^2}\cdot\int_{-\infty}^\infty\,\, 
e^{ix\xi}\cdot \bigr[ \int_{-\infty}^\infty\, \frac{\eta}{\eta^2+(\xi-t)^2}\cdot d\mu(t)
\bigl]\cdot d\xi=
\]
\[
\frac{1}{2\pi^2}\cdot\int\,
\bigr[ \int_{-\infty}^\infty\, \frac{\eta\cdot e^{ix\xi}}{\eta^2+(\xi-t)^2}\bigl]\,
d\mu(t)\quad\colon\quad \eta>0
\tag{iii}
\]


\noindent
Next,  we have  the limit formulas
\[
\frac{1}{\pi}\cdot \lim_{\eta\to 0}
\int_{-\infty}^ \infty e^{ix\xi}\cdot \frac{\eta}{\eta^2+(\xi-t)^2}\cdot d\xi=e^{ixt}
\quad\colon\,-\infty<t<\infty \tag{iv}
\]
\[
\lim_{\epsilon\to 0}\, e^{-\epsilon|x|}\cdot f(x)\to f(x)\tag{v}
\]


\noindent
So after the passage to the limit as
$\eta\to 0$ we  get the requested formula:
\[ 
f(x)=\frac{1}{2\pi}\cdot\, \int\, e^{ixt}\cdot d\mu(t)\tag{iv}
\]
\medskip


\newpage



\centerline{\bf Operational calculus on $L^1({\bf{R}})$}


\bigskip





\noindent
Let $f(x)$ be in $L^1({\bf{R}})$ and denote its Fourier transform
by $g(\xi)$, i.e.
\[ 
g(\xi)=\int\, e^{-ix\xi}f(x)dx\tag{*}
\]
Let $[a,b]$ be a closed interval on the real $\xi$-line.
Write  $w=g(\xi)$ which gives the compact subset
$g[a,b])$ of the complex $w$-plane.
Let 
$\Phi(w)$ be an analytic function
defined in some open neighborhood of
$g[a,b]$.
With these notations one has
\medskip

\noindent
{\bf Theorem.}
\emph{There exists a function
$\phi(x)\in L^1({\bf{R}})$ whose Fourier transform
satisfies}
\[ 
\hat\phi(\xi)=\Phi(g(\xi))\quad\colon\quad a\leq\xi\leq b
\]


\noindent
\emph{Proof.}
Consider  a point $a\leq \xi_*\leq b$ and put $w_*=g(\xi_*)$.
The analyticity of $\Phi$ gives
a  series expansion
\[ 
\Phi(w)=\Phi(w_*)+\sum_{\nu=1}^\infty\, c_\nu(w-w_*)^\nu\tag{*}
\]
which is convergent in some open disc centered at $w_*$. Hence there exist 
$\delta>0$ and a constant $M$
such that
\[
|c_\nu|\leq M\cdot\delta^{-\nu}\quad\colon\quad \nu=0,1,\ldots\tag{i}
\]
Next,  consider the function
\[
W(\xi)=1\quad\colon\quad |\xi|\leq1 \quad\colon
W(\xi)=2-|\xi|\quad\colon\quad 1\leq |\xi|\leq 2
\]

\noindent
Recall from the example in � XX that $W$ is the Fourier transform of an
$L^1$-function $P(x)$. Fourier's inversion formula gives:
\[ 
P(x)=\frac{1}{2\pi}\int e^{ix\xi}\cdot W(\xi)d\xi\tag{ii}
\]
Next, when $|g(\xi)-g(\xi_*)|<\delta$
it follows from (*) that
\[ 
\Phi(g(\xi))-\Phi(g(\xi_*))=\sum\, c_\nu(g(\xi)-g(\xi_*))^\nu\tag{iii}
\]
Let $k>0$ and put
\[ 
\psi_k(\xi)= W(k(\xi-\xi_*))\cdot \Phi(g(\xi_*))+
\sum\, c_\nu\cdot\bigl[ W(k(\xi-\xi_*))\cdot (g(\xi)-g(\xi_*)\,\bigr]^\nu\tag{iii}
\]


\noindent
Rules for dilation under the Fourier transform and (ii) give
\[
\frac{1}{k}\cdot e^{i\xi_*\cdot x}\cdot P(\frac{x}{k})=
\text{inverse Fourier transform of }\,\, W(k(\xi-\xi_*))\tag{iv}
\]


\noindent
More precisely, we have
\[
\frac{1}{k}\cdot e^{i\xi_*\cdot x}\cdot P(\frac{x}{k})=
\frac{1}{2\pi}\cdot\int e^{ix\xi}\cdot W(k(\xi-\xi_*))\cdot d\xi\tag{**}
\]


\noindent
Define the function $Q_k(x)$ by:
\[
Q_k(x)= \frac{1}{k}\int e^{i\xi_*(x-y)}\bigl [ P(\frac{x-y}{k}-P(\frac{x}{k})\,\bigr]
f(y) dy\tag{v}
\]
Then (**) and Fourier's inversion formula give:
\[
W(k(\xi-\xi_*))\cdot (g(\xi)-g(\xi_*)=
\int e^{-ix\xi} \cdot Q_k(x)dx\tag{vi}
\]


\noindent 
Next, the triangle inequality applied to the
right hand side in (vi) gives:
\[ \int\,|Q_k(x)|\cdot dx\leq 
\frac{1}{k} \iint
| P(\frac{x-y}{k}-P(\frac{x}{k})|\cdot |f(y)|\cdot dxdy=
||f||_1\cdot 
\int\, \bigl |P(x-\frac{y}{k})-P(\frac{x}{k})\bigr |\cdot dx\tag{***}
\]


\noindent
Since $P\in L^1({\bf{R}})$
the Riemann-Lebesgue theorem gives
\[ 
\lim_{k\to 0}\, 
\int\, \bigl |P(x-\frac{y}{k})-P(\frac{x}{k})\bigr |\cdot dx=0
\]
Together with the   inequality (***)
we therefore obtain
\[ 
||Q_k||_1=\lambda(k)\quad\text{where}\quad\lim_{k\to 0}\,\lambda(k)=0
\tag{****}
\]


\noindent
Next, for each
$\nu\geq 2$ we construct the $\nu$:th fold convolution of
$Q_k$ which we denote by
$Q_k^{(\nu)}$. The multiplicative inequality for
$L^1$-norms and (****)
give:
\[ 
||Q_k^{(\nu)}||_1\leq \lambda(k)^\nu\quad\colon\quad \nu=1,2,3,\ldots\tag{*****}
\]

\medskip

\noindent
Choose $k$ so large that
$\lambda(k)<\delta$. Then (*****) entails that
\[
G(x)= 
\sum_{\nu=1}^\infty\, c_\nu\cdot Q_k^{(\nu)}(x)\tag{vii}
\]
converges in the Banach space $L^1({\bf{R}})$.
Hence we obtain the $L^1({\bf{R}})$-function defined by
\[
G^*(x)= \frac{1}{k}\cdot\Phi(g(\xi_*))\cdot e^{i\xi_*\cdot x}
\cdot P\bigl(\frac{x}{k}\bigr)\tag{viii}
\]


\noindent
From the  constructions above it is clear that the
Fourier transform of $G^*(x)$ is equal to
the function $\psi_k(\xi)$ in (iii).
Moreover, 
the construction of the $W$-function and the  series expansion of
$\Phi$ in (*) give  the equality
\[
\psi_k(\xi)=\Phi(g(\xi))\quad\colon\quad
|\xi-\xi_*|\leq\frac{1}{k}\tag{ix}
\]
\medskip

\noindent
{\emph Final part of the proof.}
By (ix)
we find
$L^1$-functions whose Fourier transforms agrees with
$\Phi(g(\xi))$ on small intervals around every point $a\leq\xi_*\leq b$.
By  the Heine-Borel Lemma and a $C^\infty$-partition of the unit
we  finish the proof of Theorem 1. To be precise,  use that
if $h(\xi)$ is a test-functionon the real $\xi$-line then it is the
Fourier transform of some $L^1$-function.


\end{document}

