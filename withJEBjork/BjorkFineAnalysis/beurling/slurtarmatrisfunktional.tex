
\documentclass{amsart}
\usepackage[applemac]{inputenc}
\addtolength{\hoffset}{-12mm}
\addtolength{\textwidth}{22mm}
\addtolength{\voffset}{-10mm}
\addtolength{\textheight}{20mm}

\def\uuu{_}

\def\vvv{-}

\begin{document}



\newpage


\centerline {\bf \large I:C  Complex vector spaces}

\bigskip


\centerline{\emph{Contents}}
\bigskip


\noindent
0. Introduction
\medskip

\noindent
0.A The Sylvester\vvv Franke theorem
\medskip

\noindent
0.B Hankel determinants
\medskip

\noindent
0.C The Gram-Fredhom formula
\medskip

\noindent
0.D Resolvents of integral operators
\medskip

\noindent
0.D.1 Hilbert derterminants


\medskip

\noindent
0.D.2 Some results by Carleman


\medskip






\noindent
1. Wedderburn's Theorem
\medskip

\noindent
2. Resolvents


\medskip

\noindent
3. Jordan's normal form
\medskip

\noindent
4. Hermitian and normal operators
\medskip

\noindent
5. Fundamental solutions to ODE-equations

\medskip

\noindent
6. Carleman's  inequality for resolvents

\medskip

\noindent
7. Hadamad's  radius formula

\medskip

\noindent
8. On Positive definite quadratic forms
\medskip

\noindent
9. The Davies-Smith inequality
\medskip

\noindent
10. An application to integral equations























\bigskip


\centerline {\bf Introduction.} 
\bigskip

\noindent
The 
modern era about
matrices and determinants started around 1850
with major contributions by Hamilton, Sylvester  and Cayley.
An important result is the spectral theorem for symmetric
$n\times n$\vvv matrices with real elements, and its counterpart for
complex Hermitian matrices which was alreay discovere around 1810
by Cauchy in the case when   eigenvalues are distinct.
Recall that eigenvalues are found
by regarding maxima and minima of associated quadratic forms.
Weierstrass'  collected work contains a wealth of  results
related to the spectral theorem for hermitian matrices and
their interplay with quadratic forms.
Here is an elementary  result 
from Weierstrass' studies:
Let
$N\geq 2$ and $\{c_pq\}\,\colon 1\leq p,q\leq N$ is a doubly indexed sequence of
positive numbers which is symmetric, i.e.  $c_{qp}= c_{pq}$
hold for all pairs $1\leq p,q\leq N$.
Suppose that
\[
\sum_{q=1}^{q=N}\, c_{p,q}\leq 1\quad\colon 1\leq p\leq N
\]
Then it follows that
\[
\sum_{p=1}^{p=N}\, \bigr[\sum_{q=1}^{q=N}\,
c_{p,q}\cdot x_q\bigr]^2\leq 
\sum_{p=1}^{p=N}\, x_p^2
\]
for every $N$-tuple $\{x_p\}$ of non-negative real numbers.
The reader is invited to supply a proof
via
the spectral theorem for symmetric matrices.
\medskip



\noindent
Using Lagrange's interpolation formula 
Sylvester exhibited
extensive classes of matrix-valued functions by residue calculus and further
results were achieved by Frobenius who  treated
the general case when a characteristic
polynomial of a matrix has multiple roots.
The usefulness of 
matrices and their determinants in analysis was put forward 
by Fredholm in his studies of   integral 
equations. Here  estimates are  needed
to control determinants of matrices of large size  to
study  resolvents of linear operators acting on infinite dimensional vector spaces.
To handle cases where  
singular kernels appear in  an
integral operator, modified Fredholm
determinants were introduced by Hilbert whose  text\vvv book  \emph{Zur
Theorie der Integralgleichungen} from 1904  laid  the foundations
for  spectral theory  of
linear operators on infinite dimensional spaces.
A  systematic study of matrices with infinitely many elements was
done by Hellinger and Toeplitz in their joint article
\emph{Grundlagen f�r eine theorie der undendlichen matrizen} from 1910
and  applied to solve
integral equations of
the Fredholm-Hilbert type.
Carleman's  inequality for norms of resolvents
in � 6 serves  as a veritable cornerstone
for the spectral theory of linear operators since  it extends
to the infinite dimensional case.
See for example Chapter XX in [Dunford\vvv Schwartz].
\medskip


\noindent
{\bf{Hadamard's theorem.}}
A  high-light
in this section
appears in Theorem 7.1 whose
proof relies heavily upon calculus  with determinants.
Therefore we shall give
a rather extensive account  of these.
Of special relevance is the construction of Hankel determinants in � B
and the Gram-Fredholm formula in � C.
The less experienced reader may prefer to begin with the material in � 2-5 which is of a  
"soft-ware nature" compared to the more  involved results which rely upon 
computations of determinants in � 1.

\medskip


\noindent
{\bf{Integral operators and their entire functions.}}
A result which
motivates the material  in this section in regard to
functional analysis goes as follows:
Let $k(x,y)$ be a complex-valued continuous 
function on the unit square
$\{0\leq x,y\leq 1\}$. We do not assume that $k$ is symmetric,  i.e,
in general $k(x,y)\neq k(y,x)$.
If $f(x)$ is a   continuous function  on $[0,1]$
we get a sequence $\{f\uuu n\}$ where
$f\uuu 0(x)= f(x)$ and 
\[
f\uuu n(x)=\int\uuu 0^1\, k(x,y)\cdot f\uuu{n\vvv 1}(y)\cdot dy
\quad \colon\quad n\geq 1
\]
If $M=\max_x\, |f(x)|$ is the maximum norm of $f$ 
and $C=||k||_\infty$
it is clear that
the maximum norms
$|f\uuu n|_\infty\leq (CM)^n$.
Hence there exists  a power series:
\[
F\uuu\lambda(x)= \sum\uuu{n=0}^\infty\, f\uuu n(x)\cdot \lambda^n
\] 
which converges for every  $|\lambda|<1/CM$ and yields an analytic function 
With these notations the following result was established by Carleman in the article 
\emph{xxx}.

\medskip

\noindent
{\bf{Theorem.}}
\emph{The function $f\mapsto F\uuu\lambda$ with values in the Banach space
$B=C^0[0,1]$ extends to a meromorphic $B$\vvv valued 
function in the whole
$\lambda$\vvv plane.}
\bigskip

\medskip

\noindent
{\bf{Remark.}}
So in particular the series
\[
\sum\uuu{n=0}^\infty\, f\uuu n(x)\cdot \lambda^n
\] 
yields and entire function of $\lambda$ for each
$f$ and every $0\leq x\leq 1$.
The proof is given in � xx and relies upon properties of
the Hankel determinants
\[
\mathcal D_n^{(p)}(x)=
\det
\begin{pmatrix}
f_{n+1}(x)
&f_{n+2}(x)
&\ldots&\ldots
& f_{n+p}(x)\\
f_{n+2}(x)
&f_{n+3}(x)
&\ldots&\ldots
& f_{n+p+1}(x)\\

\ldots &\ldots &\ldots&\ldots \\
\ldots &\ldots&\ldots&\ldots\\
f_{n+p}(x)
&f_{n+p+1}(x)
&\ldots&\ldots
& f_{n+2p-1}(x)\\
\end{pmatrix}
\]
defined for all pairs $p,n\geq 1$ and each continuous function $f$.
The crucial result is the following a priori inequality:
\medskip


\noindent
{\bf{Proposition.}} \emph{If the maximum norms of $k$ and $f$ both are
$\leq 1$
then one has
the inequality}

\[
\bigl |\,  \mathcal D\uuu n^{(p)}(x))\,\bigr|\leq 
(p\, !)^{\vvv n}\cdot \bigl( p^{\frac{p}{2}}) ^n\cdot \frac{p^p}{p\,!}
\]
\emph{for every $p\geq 2$ and $0\leq x\leq 1$.}









\newpage

\centerline
{\bf{� 1. Determinants.}}
\bigskip


\noindent
{\bf{Introduction.}}
The material in this section is inspired by
the contents in the excellent    text\vvv book
\emph{Determinantenheorie} from 1909 by
Gerhard Kovalevski.
We give a rather detailed account
about some classic results
since
determinants are not always treated in detail in contemporary
text-books devoted to matrices and more "abstract linear algebra".


\bigskip

\centerline
{\bf{A.0 The Hilbert-Schmidt norm}}

\bigskip
\noindent
Let  $A$ be a matrix whose 
 elements $\{a_{pq}\}$ are complex numbers.
Its  Hilbert-Schmidt norm is defined
by
\[
||A||=\sqrt{ \sum\sum\, |a_{pq}|^2}
\]
where the doube sum extends over all pairs
$1\leq p,q\leq n$.
The operator norm is defined by:
\[
\text{Norm}(A)=
\max_{z_1,\ldots z_n}\,\sqrt{
\sum_{p=1}^{p=n}\, \bigl|\sum_{q=1}^{q=n}\, a_{pq}z_q|^2\,}\tag{*}
\]
with the maximum taken over $n$-tuples of complex numbers
such that
$\sum\, |z_p|^2=1$.
Introduce the Hermitian inner product on
${\bf{C}}^n$
and identify  $A$ with the linear operator which
sends a basis vector $e_q$ into
\[ 
A(e_q)= \sum_{p=1}^{p=n}\, a_{pq}\cdot e_p
\]
If $z$ and $w$ is a pair of complex $n$-vectors one gets:
\[ 
\langle Az,w\rangle=
\sum\sum a_{pq}z_q\bar w_p
\]
The Cauchy-Schwarz inequality
gives
\[
\bigl|\langle Az,w\rangle\bigr|^2\leq
\bigl(\sum_{p=1}^{p=n}\,  \bigl|\sum_{q=1}^{q=n}\, a_{pq}z_q|^2\,\bigr)
\cdot 
\sum_{p=1}^{p=n}\, |w_p|^2\tag{1}
\]
So if both $z$ and $w$ have length $\leq 1$
the definition of the operator norm entails that
\[
\max_{z,w}\, |\langle Az,w\rangle\bigr|=\text{Norm}(A)\tag{2}
\]
where the maximum is taken over
vectors  $z$ and $w$ of unit length.
Next, another application of the Cauchy-Schwarz inequality
shows that if $z$ has unit length, then

\[
\sum_{p=1}^{p=n}\,  \bigl|\sum_{q=1}^{q=n}\, a_{pq}z_q|^2
\leq 
\sum_{p=1}^{p=n}\sum_{q=1}^{q=n}\, |a_{pq}|^2
\]
This gives
in particular the inequality
\[
\text{Norm}(A)\leq ||A||\tag{3}
\]

\medskip

\noindent{\bf{Example.}}
Given a matrix $A$
we replace each element  by its absolute value
$|a_{pq}|$ and  get a matrix $A_*$ whose elements are non-negative real numbers
where
\[
\text{Norm}(A_*)=
\max_{x_1,\ldots x_n}\,\sqrt{
\sum_{p=1}^{p=n}\, \bigl|\sum_{q=1}^{q=n}\, |a_{pq}|x_q|^2\,}\tag{i}
\]
Above  it  suffices to compete with real $n$-vectors for which
$\sum\, x_p^2=1$ and every $x_p\geq 0$.
The triangle inequality gives
$\text{Norm}(A)\leq \text{Norm}(A_*)$.
The $A_*$-norm  is found
via Lagrange's multiplier i.e. one employs Lagrange's criterion for extremals
of quadratic forms which entails  that  (i) is maximized by a real non-negative $n$-vector $x$
which satsfies a linear system of equations
\[
\lambda\cdot x^*_j=\sum_{p=1}^{p=n}\, a_{pj}\cdot 
\sum_{q=1}^{q=n}\, a_{pq}x^*_q\tag{ii}
\]
Introducing the double indexed numbers
\[
\beta_{jq}= \sum_{p=1}^{p=n}\, a_{pj}a_{pq}
\]
Lagrange's equations corresponds to the system
\[ 
\lambda\cdot x^*_j=\sum_{q=1}^{q=n}\,\beta_{jq}\cdot x_q^*\tag{iii}
\]
In the "generic case" the
$n\times n$-matrix $\{\beta_{jq}\}$ is non-singular, i.e. its
determinant is $\neq 0$
and (iii) has a unique solution $x^*$ with
every $x^*_j\geq 0$ for
a uniquely determined multiplier $\lambda>0$.
As an  example 
we consider an 
$n\times n$-matrix of the form
\[ 
T_s=
\begin{pmatrix}
1&s&s&\ldots &s \\
0&1&\ldots &s&s\\
\ldots&\ldots&\ldots&\ldots&\ldots\\
\ldots&\ldots&\ldots&\ldots\\
0&0&\ldots &1& s\\
0&0&\ldots &0& 1\\
\end{pmatrix}
\]
Thus, the diagonal elements are all units and
$T$�is upper triangular with $t_{ij}=s$ for pair $i<j$
while the elements below the diagonal are zero.
In spite of the explicit
expression for $T$ the computation of its
operator norm is a bit involved.
The case $s=2$ is of special interest
and here one has
the classic formula
\[ 
\text{Norm}(T_2)= \cot\frac{\pi}{4n}\tag{*}
\]
\noindent{\bf{Exercise.}} Prove (*) and find  the
$x^*$-vector which maximizes (iii),
If necessary, consult the
literature and let us   remark that one can use numerical
experminents with a computer to settle (*).


\medskip




\centerline {\bf{A.1  The Sylvester-Franke theorem.}}
\medskip


\noindent
Let    $A$  be some
$n\times n$\vvv matrix with
elements $\{a\uuu{ik}\}$.
Put
\[
b\uuu{rs}=a\uuu {11}a\uuu {rs}\vvv a\uuu{r1}a\uuu{1s}
\quad\colon\quad 2\leq r,s\leq n
\]
These $b$\vvv numbers give an $(n\vvv 1)\times(n\vvv 1)$\vvv matrix
where $b\uuu{22}$ is put in position $(1,1)$ and so on.
The matrix is denoted by 
$\mathcal S^1(A)$ and called the first order
Sylvester matrix.
If $a\uuu{11}\neq 0$ one has
the equality
\[
a\uuu{11}^{n\vvv 2}\cdot \text{det}(A)=
\text{det}(\mathcal S^1(A))\tag{A.1.1}
\]
\medskip

\noindent
{\bf{Exercise.}} Prove this result or consult a text\vvv book
which
apart from "soft abstract notions"  
does not ignore to
treat  determinants. 
\medskip

\noindent
{\bf{A.1.2 Sylvester's equation.}}
For every
$1\leq h\leq n\vvv 1$ one constructs the
$(n\vvv h\times (n\vvv h)$\vvv matrix whose elements are

\[ b\uuu{rs}= \det\,
\begin{pmatrix}
a\uuu{11}&a\uuu{12}&\ldots &a\uuu{1h}& a\uuu{1s}\\
a\uuu{21}&a\uuu{22}&\ldots &a\uuu{2h}& a\uuu{2s}\\
\ldots&\ldots&\ldots&\ldots&\ldots\\
\ldots&\ldots&\ldots&\ldots&\ldots\\
a\uuu{h1}&a\uuu{h2}&\ldots &a\uuu{hh}& a\uuu{hs}\\
a\uuu{r1}&a\uuu{r2}&\ldots &a\uuu{rh}& a\uuu{rs}\\
\end{pmatrix}\quad\colon\quad h+1\leq r,s\leq n
\]
\medskip

\noindent
With these notation one has the Sylvester equation:

\[
\det
\begin{pmatrix}
b\uuu{h+1,h+1}&b\uuu{h+1,h+2}&\ldots &b\uuu{h+1,n}\\
b\uuu{h+2,h+1}&b\uuu{h+2,h+2}&\ldots &b\uuu{h+2,n}\\
\ldots&\ldots&\ldots&\ldots\\
\ldots&\ldots&\ldots&\ldots\\
b\uuu{n,h+1}&b\uuu{n,h+2}&\ldots &b\uuu{n,n}\\
\end{pmatrix}=
\bigl[\,\det\begin{pmatrix}
a\uuu{11}&a\uuu{12}&\ldots &a\uuu{1h}\\
a\uuu{21}&a\uuu{22}&\ldots &a\uuu{2h}\\
\ldots&\ldots&\ldots&\ldots\\
\ldots&\ldots&\ldots&\ldots\\
a\uuu{h1}&a\uuu{h2}&\ldots &a\uuu{hh}\\
\end{pmatrix}\,\bigr]^{n\vvv h\vvv 1}\cdot \det(A)\tag{*}
\]
\medskip


\noindent
For a proof of (*) we refer to original work by Sylvester or
[Kovalevski: page xx\vvv xx] which offers several different
proofs of (*).

\bigskip

\noindent
{\bf{A.1.3 The Sylvester\vvv Franke theorem.}}
Let  $n\geq 2$ and $A=\{a\uuu{ik}\}$ an
$n\times n$\vvv matrix.
Let $m<n$ and consider 
the family of minors of size $m$, i.e.
one picks $m$ columns and $m$ rows
which give  an $m\times m$\vvv matrix
whose determinant is called a minor of size $m$
of the given matrix $A$. The total number of
such minors is equal to
\[
N^2\quad\text{where}\quad N= \binom{n}{m}
\]
We have $N$ many strictly increasing sequences
$1\leq \gamma\uuu1<\ldots\gamma \uuu m\leq n$
where a $\gamma$\vvv sequence corresponds to preserved
columns when   a minor is constructed. Similarly we have
$N$ strictly increasing sequences which correspond to preserved rows.
With this in mind we get  for each pair $1\leq r,s\leq N$
a minor $\mathfrak{M}\uuu {rs}$
where the enumerated $r$:th $\gamma$\vvv  sequence preserve columns and similarly
$s$ corresponds to the enumerated sequence of rows.
Now we obtain the $N\times N$\vvv matrix

\[
\mathcal A\uuu m= \begin{pmatrix}
\mathfrak{M}\uuu{11}&\mathfrak{M}{12}&\ldots &\mathfrak{M}\uuu{1N}\\
\mathfrak{M}\uuu{21}&\mathfrak{M}{22}&\ldots &\mathfrak{M}\uuu{2N}\\
\ldots&\ldots&\ldots&\ldots\\
\ldots&\ldots&\ldots&\ldots\\
\mathfrak{M}\uuu{N1}&\mathfrak{M}{22}&\ldots &\mathfrak{M}\uuu{NN}\\
\end{pmatrix}
\]


\noindent
We refer to $\mathcal A\uuu m$ as the Franke\vvv Sylvester matrix of order
$m$. They  are defined for each $1\leq m\leq n\vvv 1$.

\medskip







\noindent
{\bf{A.1.4 Theorem.}}
\emph{For every $1\leq m<n$ one has
the equality}
\[
\mathcal A\uuu m= \text{det}(A)^{\binom{n\vvv 1}{m\vvv 1}}
\]


\medskip

\noindent
{\bf{Example.}} Consider the diagonal $3\times 3$\vvv matrix:

\[
A=\begin{pmatrix}
1&0&0\\
0&1&0\\
0&0&2\\
\end{pmatrix}
\]

\medskip

\noindent
With $m=2$
we have 9 minors of size 2 and the reader can recognize that
when they are arranged so that we begin to remove
the first column, respectively the first row, then 
the resulting $\mathfrak{M}$\vvv matrix becomes
\[
\begin{pmatrix}
2&0&0\\
0&2&0\\
0&0&1\\
\end{pmatrix}
\]
Its determinant is $4= 2^2$ which is in accordance with the general formula since
$n=3$ and $m=2$ give $\binom{n\vvv 1}{m\vvv 1}=2$.
For the proof of Theorem 0.A.1 the reader can consult 
[Kovalevski: page102\vvv 105].




\bigskip



\centerline {\bf{� B. Hankel determinants.}}
\bigskip


\noindent
Let $\{c\uuu 0,c\uuu 1,\ldots\}$
be a sequence of complex numbers.
For each  integer $p\geq 0$ and  every $n\geq 0$
we obtain the 
$(p+1)\times (p+1)$\vvv matrix:


\[
\mathcal C\uuu n^{(p)}=
\begin{pmatrix}
c\uuu{n}&c\uuu{n+1}&\ldots&c\uuu{n+p}\\
c\uuu{n+1}&c\uuu{n+2}&\ldots&c\uuu{n+p}\\
\ldots&\ldots&\ldots&\ldots\\
c\uuu{n+p}&c\uuu{n+p+1}&\ldots&c\uuu{n+2p}\\
\end{pmatrix}
\]

\medskip


\noindent
Let $\mathcal D\uuu n^{(p)}$ denote the determinant. One
refers to $\{\mathcal D\uuu n^{(p)}\}$
 as the recursive Hankel determinants.
They  are used to establish   various properties of the given
$c$\vvv sequence.
To begin with we define  the rank   $r^*$ 
of $\{c_n\}$
as follows:
To every non\vvv negative integer $n$
one has the  infinite vector
\[ 
\xi\uuu n=(c\uuu n,c\uuu{n+1},\ldots)
\]
We say that $\{c\uuu n\}$ has finite rank if
there exists a number $r^*$ such that
$r^*$ many $\xi$\vvv vectors
are linearly independent and the rest are linear combinations of these.
\medskip

\noindent
{\bf{B.1 Rational series expansions.}}
The sequence $\{c\uuu n\}$ gives the formal power series
\[
f(x)=\sum\uuu{\nu=0}^\infty\, c\uuu\nu x^\nu \tag{B.1.1}
\]
If $n\geq 1$ we set
\[
\phi\uuu n(x)= x^{\vvv n}\cdot(
f(x)\vvv \sum\uuu{\nu=0}^{n\vvv 1} c\uuu\nu x^\nu)=
\sum\uuu{\nu=0}^\infty c\uuu{n+\nu} x^\nu
\]
It is clear
that $\{c\uuu \nu\}$ has finite rank if and only if  the sequence
$\{\phi\uuu\nu(x)\}$
generates a finite dimensional complex subspace of the vector space 
${\bf{C}}[[x]]$ whose elements are formal power series.
If this dimension is finite we find a positive integer
$p$ and a 
non\vvv zero $(p+1)$\vvv tuple $(a\uuu 0,\ldots,a\uuu p)$ of complex numbers
such that the power series
\[ 
a\uuu 0\cdot  \phi\uuu 0(x)+\ldots+a\uuu p\cdot \phi\uuu p(x)=0
\]
Multiplying this equation with $x^p$ it follows that
\[
(a\uuu p+a\uuu{p\vvv 1} x+\ldots+a\uuu o x^p)\cdot f(x)=q(x)
\]
where $q(x)$ is a polynomial.
Hence the finite rank entails that the power series (B.1.1) 
represents a rational function.
\medskip


\noindent
{\bf{Exercise.}}
Conversely, assume that
\[
\sum\, c\uuu\nu x^\nu= \frac{q(x)}{g(x)}
\] 
for some pair of polynomials. Show that $\{c\uuu n\}$ has finite rank.
The next result is also left as an exercise to the reader.

\medskip


\noindent
{\bf{B.2 Proposition.}}
\emph{A sequence $\{c\uuu n\}$ has a finite rank if and only if
there exists an integer $p$ such that}
\[
\mathcal D\uuu 0^{(p)}\neq 0\quad
\text{and}\quad D\uuu 0^{(q)}=0\quad \colon\quad q>p\tag{4}
\]
\emph{Moreover, one has the equality $p$ is equal to the rank of
$\{c_n\}$.}


\medskip

\noindent
{\bf{B.3 A specific example.}}
Suppose that the degree of $q$ is strictly less than that of $g$ in the Exercise above  
and that the rational function $\frac{q}{g}$
is expressed by a sum of simple 
fractions:
\[ 
\sum\, c\uuu\nu x^\nu= \sum\uuu{k=1}^{k=p}\, \frac{d\uuu k}{1\vvv \alpha\uuu k x}
\] 
where $\alpha\uuu 1,\ldots,\alpha\uuu p$ are distinct and every $d\uuu k\neq 0$.
Then we see that
\[ 
c\uuu n=\sum\uuu{k=1}^{k=p}\, d\uuu k\cdot \alpha\uuu k^n\quad 
\text{where we have put}\quad
\alpha\uuu k^0=1\quad\text{ so that}\quad
c\uuu 0=\sum\, d\uuu k
\]
\medskip



\noindent
{\bf{B.4 The reduced rank.}}
Assume that $\{c\uuu n\}$ has a finite rank $r^*$. To each $k\geq 0$ we denote by $r\uuu k$
the dimension of the vector space generated by
$\xi\uuu k,\xi\uuu{k+1},\ldots$.
It is clear that $\{r\uuu k\}$ decrease and we find a non\vvv negative integer
$r\uuu *$ such that $r\uuu k=r\uuu *$ for large $k$ and
refer to $r\uuu *$ as the reduced rank. By the construction
$r\uuu *\leq r^*$. The relation between $r^*$ and $r\uuu *$
is related to the representation
\[
 f(x)= \frac{q(x)}{g(x)}
\]
where $q$ and $g$ are polynomials without common factor.
We shall not pursue this discussion any further but refer to the literature.
See in particular
the exercises
in [Polya\vvv Szeg� : Chapter VII:problems 17\vvv 34].




\bigskip

\noindent
{\bf{B.5 Hankel's formula for Laurent series.}}
Consider a rational function of the form
\[
R(z)= \frac{q(z)}{z^p\vvv [a\uuu 1z^{p\vvv 1}+\ldots
+a\uuu{p\vvv 1}z+ a\uuu p]}
\]
where the polynomial $q$ has degree $\leq p\vvv 1$.
At $\infty$ we have a Laurent series expansion

\[ 
R(z)= \frac{c\uuu 0}{z}+ 
\frac{c\uuu 1}{z^2}+\ldots
\]
Consider the $p\times p$\vvv matrix
\[
A=\begin{pmatrix}
0&0&&\ldots&0&a\uuu p\\
1&0&0&\ldots&0&a\uuu{p\vvv 1}\\
0&1&0&\ldots&\ldots&a\uuu{p\vvv 2}\\
\ldots&
\ldots&
\ldots&
\ldots&\ldots&\ldots
\\
0&0&0&\ldots&1&a\uuu 1\\
\end{pmatrix}
\]
\medskip

\noindent
{\bf{B.5.1 Theorem.}}
\emph{Let $\mathcal D\uuu n^{(p)}$ be the Hankel determinants of
$\{c_n\}$. Then 
the following  hold for every $n\geq 1$:}
\[
\mathcal D\uuu n^{(p)}=  \mathcal D^{(p)}\uuu 0\cdot
\bigl[\text{det}(\,A\bigr)\bigr ) ^n
\]
\medskip

\noindent
{\bf{Exercise. }} Prove this result.

\bigskip

\noindent

\noindent
{\bf{B.6 The Hadamard-Kronecker identity.}}
For all pairs
of positive integers $p$ and $n$ one has the equality:
\[
\mathcal D\uuu n^{(p+1)}\cdot
\mathcal D\uuu {n+2}^{(p-2)}=
\mathcal D\uuu n^{(p+1)}
\mathcal D\uuu {n+2}^{(p\vvv 1)}-
\bigl[\mathcal D\uuu {n+1}^{(p)}\,\bigr]^2\tag{5.6.1}
\]

\medskip


\noindent
\emph{Proof.}
The equality (5.6.1 ) is s special case of a determinant formula for symmetric matrices
which is due to Sylvester.  Namely,
let $N\geq 2$ and consider a symmetric matrix

\[
S=\begin{pmatrix}
s\uuu{11}&s\uuu{12}&\ldots &s\uuu{1N}\\
s\uuu{21}&s\uuu{22}&\ldots &s\uuu{2N}\\
\ldots&\ldots&\ldots&\ldots\\
\ldots&\ldots&\ldots&\ldots\\
s\uuu{N1}&a\uuu{N2}&\ldots &s\uuu{NN}\\
\end{pmatrix}
\]
Now we consider  the
$(N-1)\times (N-1)$-matrices

\[
S_1= 
\begin{pmatrix}
s\uuu{22}&s\uuu{23}&\ldots &s\uuu{2N}\\
s\uuu{32}&s\uuu{33}&\ldots &s\uuu{3N}\\
\ldots&\ldots&\ldots&\ldots\\
\ldots&\ldots&\ldots&\ldots\\
s\uuu{N2}&s\uuu{N3}&\ldots &s\uuu{NN}\\
\end{pmatrix}
\quad\colon\quad 
S_2= 
\begin{pmatrix}
s\uuu{12}&s\uuu{13}&\ldots &s\uuu{1N}\\
s\uuu{22}&s\uuu{23}&\ldots &s\uuu{2N}\\
\ldots&\ldots&\ldots&\ldots\\
\ldots&\ldots&\ldots&\ldots\\
s\uuu{N-1,2}&s\uuu{N-1,3}&\ldots &s\uuu{N-1,N}\\
\end{pmatrix}
\]
\medskip
\[
S_3= 
\begin{pmatrix}
s\uuu{11}&s\uuu{12}&\ldots &s\uuu{1,N-1}\\
s\uuu{21}&s\uuu{22}&\ldots &s\uuu{2,N-1}\\
\ldots&\ldots&\ldots&\ldots\\
\ldots&\ldots&\ldots&\ldots\\
s\uuu{N-1,1}&s\uuu{N-1,2}&\ldots &s\uuu{N-1,N-1}\\
\end{pmatrix}
\]
\medskip

\noindent
We have also the $(N-2)\times (N-2)$-matrix
when extremal rows and columns are removed:

\[ S_*=\begin{pmatrix}
s\uuu{22}&s\uuu{23}&\ldots &s\uuu{2,N-1}\\
s\uuu{32}&s\uuu{33}&\ldots &s\uuu{3,N-1}\\
\ldots&\ldots&\ldots&\ldots\\
\ldots&\ldots&\ldots&\ldots\\
s\uuu{2,N-1}&s\uuu{3,N-1}&\ldots &s\uuu{N-1,N-1}\\
\end{pmatrix}
\]
\medskip

\noindent
{\bf{B.7 Sylvester's identity.}}
\emph{One has the equation}
\[
\det(S)\cdot \det(S_*)=
\det S_1)\cdot \det S_3-\bigl(\det S_2\bigr)^2
\]
\medskip

\noindent{\bf{Exercise}}. Prove this result and deduce the 
Hadamard-Kronecker equation.














\newpage




\centerline{\bf{� C. The Gram\vvv Fredholm formula.}}
\medskip


\noindent
A result whose discrete version is due to Gram  was 
extended to integrals  by
Fredholm and goes as follows:
Let $\phi\uuu 1,\ldots,\phi\uuu p$
and $\psi\uuu 1,\ldots,\psi\uuu p$ be two
$p$\vvv tuples of continuous functions on
the unit interval.
We get the $p\times p$\vvv matrix with elements
\[
a\uuu {\nu k}= \int\uuu 0^1\, \phi\uuu \nu(x)\psi\uuu k(x)\cdot dx
\]
At the same time
we define the following  functions on $[0,1]^p$:

\[ 
\Phi(x\uuu 1,\ldots,x\uuu p)
=\text{det}
\begin{pmatrix}
\phi\uuu 1(x\uuu 1)&\cdots&\phi\uuu 1(x\uuu p)\\
\cdots &\cdots&\cdots \\
\cdots &\cdots&\cdots\\
\phi\uuu p(x\uuu 1)&\cdots &\phi\uuu 1(x\uuu p)\\
\end{pmatrix}\quad\colon\quad
\Psi(x\uuu 1,\ldots,x\uuu p)
=\text{det}
\begin{pmatrix}
\psi\uuu 1(x\uuu 1)&\cdots&\psi\uuu 1(x\uuu p)\\
\cdots &\cdots&\cdots \\
\cdots &\cdots&\cdots\\
\psi\uuu p(x\uuu 1)&\cdots &\psi\uuu 1(x\uuu p)\\
\end{pmatrix}
\]

\medskip

\noindent
Product rules for determinants give the Gram\vvv Fredholm  equation
\[
\text{det}(a\uuu{\nu k})=
\frac{1}{p\,!}\int\uuu{[0,1]^p}\, 
\Phi(x\uuu 1,\ldots,x\uuu p)\cdot
\Psi(x\uuu 1,\ldots,x\uuu p)\cdot
dx\uuu 1\ldots dx\uuu p\tag{*}
\]
\medskip

\noindent
{\bf{Exercise.}} Prove (*)
or consult the literature.
See for example  the excellent text-book
[Bocher]
which contains a detailed account about Fredholm
determinants and
their role for solutions to integral equations.





 
\bigskip

\noindent
\centerline
{\bf{� D. Resolvents of integral operators.}}
\medskip

\noindent
Fredholm studied integral equations of the form
\[
\phi(x)\vvv \lambda\cdot 
\int\uuu\Omega\,
K(x,y)\cdot \phi(y)\cdot dy=
f(x)\tag{*}
\]
where $\Omega$ is a bounded domain in some euclidian space
and the kernel function $K$ is complex\vvv valued. In general no
symmetry condition is imposed.
Various regularity conditions can be imposed upon the kernel. The simplest is when
$K(x,y)$ is a continuous function in
$\Omega\times\Omega$.
The situation becomes more involved when singularities occur, for example when
$K$ is $+\infty$ on the diagonal, i.e. $|K(x,x)|=+\infty$.
This occurs for example when
$K$�is derived from Green's functions
which yield fundamental solutions to
elliptic PDE\vvv equations
where corresponding boundary value problems are solved
via integral equations. 
To obtain square integrable solutions in (*) for less regular kernel
functions, 
the original determinants used by Fredholm were modified by
Hilbert which avoid the singularities
and lead to quite general 
formulas for resolvents of the integral operator $\mathcal K$ defined by
\[
\mathcal K(\phi)(x)=\int\uuu\Omega\,
K(x,y)\cdot \phi(y)\cdot dy
\]
One studies foremost the case when $K$ is square integrable,  i.e.
when
\[
\iint\uuu{\Omega\times\Omega}\, |K(x,y)|^2\, dxdy<\infty\tag{*}
\]
In this case one refers to a Hilbert-Schmidt operator.
An eigenvalue is a complex number
$\lambda\neq 0$ for which there exists a non\vvv zero 
$L^2$-function $\phi$ such that
\[
\mathcal K(\phi)= \lambda\cdot \phi
\]
It is not difficult to show that (*) entails  that 
the set of eigenvalues form a discrete set $\{\lambda\uuu n\}$.
In the article [Schur: 1909] Schur proved 
the inequality
\[
\sum\, \frac{1}{|\lambda\uuu n|^2}\leq
\iint\uuu{\Omega\times\Omega}\, |K(x,y)|^2\, dxdy<\infty\tag{**}
\]

\noindent
Notice that  one does not assume that
the kernel function is symmetric, i.e. in general $K(x,y)\neq K(y,x)$.

\bigskip

\noindent
{\bf{D.1 Hilbert's determinants.}}
Let $K$ be a kernel function for which  (*) is finite where
$K$ is singular on the diagonal subset
of $\Omega\times\Omega$ but continuous in its complement.
To each positive integer $m$ one associates a pair of matrices of size
$(m+1)\times m(+1)$ whose elements depend upon a pair 
$(\xi,\eta)
\in\Omega\times\Omega$ and an $m$\vvv tuple of distinct points
$x\uuu 1,\ldots,x\uuu m$ in $\Omega$: 

\[
C^*\uuu m=
\begin{pmatrix}
0&K(\xi,x\uuu 1)&K(\xi,x\uuu 2)&\ldots&\ldots &K(\xi, x\uuu m)\\
K(x\uuu 1,\eta)&0&K(x\uuu 1,x\uuu 2)&\ldots&\ldots &K(x\uuu 1,x\uuu m)\\
K(x\uuu 2,\eta)&K(x\uuu 2,x\uuu 1)&0
&\ldots&\ldots&0\\
\ldots&
\ldots&
\ldots&
\ldots&\ldots&\ldots
\\
\ldots&
\ldots&
\ldots&
\ldots&\ldots&\ldots
\\
K(x\uuu m,\eta)&K(x\uuu m,x\uuu 1)&K(x\uuu m,x\uuu 2)&\ldots &\ldots&0\\
\end{pmatrix}
\]

\bigskip


\[
C\uuu m=\begin{pmatrix}
0&K(x\uuu 1,x\uuu 2)&&\ldots&0&K(x\uuu 1, x\uuu m)\\
K(x\uuu 2,x\uuu 3)&0&K(x\uuu 2,x\uuu 3)
&\ldots&&K(x\uuu 2,x\uuu m)\\
\ldots&\ldots &\ldots&\ldots&\ldots&\ldots\\
\ldots&
\ldots&
\ldots&
\ldots&\ldots&\ldots
\\
K(x\uuu m,x\uuu 1)&K(x\uuu m,x\uuu 2)
&K(x\uuu m,x\uuu 3) &\ldots&K(x\uuu m,x\uuu{m\vvv 1})&0\\
\end{pmatrix}
\]


\noindent
Put:
\[
D^*\uuu m(\xi,\eta)=
\int\uuu{\Omega^m}\, 
C^*\uuu m(\xi,\eta: x\uuu 1,\ldots,x\uuu m)\cdot dx\uuu 1\cdots dx\uuu m\tag{i}
\]
\[
D\uuu m=
\int\uuu{\Omega^m}\, 
C\uuu m(x\uuu 1,\ldots,x\uuu m)\cdot dx\uuu 1\cdots dx\uuu m\tag{ii}
\]
Thus, we take the integral over the $m$\vvv fold product of
$\Omega$.
Next, let $\lambda$ be a new complex parameter and set
\[ 
\mathcal D^*(\xi,\eta,\lambda)=
\sum\uuu{m=1}^\infty\, \frac{(\vvv\lambda)^m}{m!}
\cdot C^{*}\uuu m(\xi,\eta)
\]

\[ 
\mathcal D(\lambda)=1+
\sum\uuu{m=1}^\infty\, \frac{(\vvv\lambda)^m}{m!}
\cdot D_m
\]



\bigskip

\centerline{\bf{� E.  Results by Carleman}}
\medskip


\noindent
Using the Fredholm-Hilbert determinants 
some
conclusive facts about integral operators
were established
by Carleman in the article \emph{Zur Theorie der Integralgleichungen}
from 1921  when the kernel
$K$ is of the Hilbert-Schmidt type, i.e. 
\[ 
\iint\, |K(x,y)|^2\, dxdy<\infty\tag{*}
\]
From � D we have the linear operator $\mathcal K$
on $L^2(\Omega$. When $\lambda$ is outside
$\sigma(\mathcal K)$ we get the resolvent operator
\[
R(\lambda)= (\lambda\cdot E-\mathcal K)^{-1}
\]
It is represented 
by a kernel  function $\Gamma(\xi,\eta;\lambda)$, i.e.
\[
R(\lambda)(\phi)(x)=
\int_\Omega\, \Gamma(\xi,\eta;\lambda)\cdot \phi(\xi)\, d\xi
\]


\medskip

\noindent
{\bf{E.1 Theorem.}} \emph{Outside the spectrum one has the equation}
\[
\Gamma(\xi,\eta;\lambda)=
K(\xi,\eta)+\frac{\mathcal D^*(\xi,\eta,\lambda)}{
\mathcal D(\lambda)}
\]


\noindent
{\bf{E.2  Remark.}} This equation is due to Fredholm and Hilbert and a detailed proof appears
in Hilbert's text-book \emph{Integralgleichungen}. The reader may try to discover a
proof via the Gram-Fredholm formula.
Next, since $\mathcal K$ is of Hiblert-Schmidt type it is a compact
operator on $L^(\Omega)$ wuth a discrete spectrum
$\{\lambda\uuu\nu\}$
where multiple
eigenvalues are repeated when the corresponding 
eigenspaces have dimension $\geq 2$. 
Theorem E.1 shows that
the  spectrum correspond to   zeros of the entire function
$\mathcal D(\lambda)$.
\medskip


\noindent
{\bf{E.3 Exercise.}}
Apply 
inequalities of the Fredholm\vvv Hadamard type for 
determinants to show that
\[
\int\uuu{\Omega}\, \Gamma (\xi,\xi; \lambda)\cdot d\xi=
\vvv \lambda\cdot \sum\uuu{\nu=1}^\infty\,
\frac{1}{\lambda\uuu\nu(\lambda\vvv\lambda\uuu\nu)}\tag{E.3}
\]
\medskip


\noindent
Now we announce a major result from Carleman's cited article.

\medskip

\noindent
{\bf{E.4 Theorem.}}
\emph{$\mathcal D(\lambda)$ is an entire function of the complex parameter
$\lambda$ given by a Hadamard product}
\[
\mathcal D(\lambda)=
\prod\,(1-\frac{\lambda}{\lambda_n})\cdot e^{\frac{\lambda}{\lambda_n}}\tag{1}
\]
\emph{where $\{\lambda_n\}$ satisfy}
\[
\sum\, |\lambda_n|^{-2}<\infty\tag{2}
\]



\noindent
{\bf{Remark.}}
The crucial
step in Carleman's  proof is based upon 
an  inequality for
determinants which goes as follows:
Let $q>p\geq 1$ be a pair of integers
and
$\{a\uuu{k,\nu}\}$ a doubly\vvv indexed sequence of complex numbers
which appear as elements in a $p+q$\vvv matrix of the form:
\[
\begin{pmatrix}
0&\ldots&0&a\uuu{1,p+1}&\ldots &a\uuu{1,p+q}\\
0&\ldots&0&a\uuu{2,p+1}&\ldots &a\uuu{1,p+q}\\
\ldots&
\ldots&\ldots&\ldots&\ldots&\ldots \\
0&\ldots&0&a\uuu{p,p+1}&\ldots &a\uuu{p,p+q}\\
a\uuu{p+1,1}&\ldots &a\uuu{p+1,p}&a\uuu{p+1,p+1}&\ldots&
a\uuu{p+1,p+q}\\
\ldots&
\ldots&\ldots&\ldots&\ldots&\ldots
\\
a\uuu{p+q,1}&\ldots &a\uuu{p+q,p}&a\uuu{p+q,p+1}&\ldots
&a\uuu{p+q,p+q}
\\
\end{pmatrix}\tag{*}
\]
\medskip



\noindent
For each  $1\leq m\leq p$ we put
\[
L\uuu m=\sum\uuu{\nu=1}^{\nu=q}\, |a\uuu{m,p+\nu}|^2
\quad\colon\quad 
S\uuu m=\sum\uuu{\nu=1}^{\nu=q}\, |a\uuu{p+\nu,m}|^2
\]
and finally
\[
N=\sum\uuu{j=1}^{j=q} \sum\uuu{\nu=1}^{\nu=q}\, |a\uuu{p+j,p+\nu}|^2
\]

\medskip




\noindent
{\bf{E.5 Theorem.}}\emph{
Let $D$ be the determinant of the matrix  (*). Then}
\[
|D|\leq (L\uuu 1\cdots L\uuu p) ^ {\frac{1}{p}}\cdot
\sqrt{M\uuu 1\cdots M\uuu p}\cdot
\frac{N^{\frac{q\vvv p}{2}}}{(q\vvv p)^{\frac{q\vvv p}{2}}}
\]


\noindent
\emph{Proof.}
After unitary transformations of the last $q$ rows and 
the last $q$ columns respectively, the proof is reduced to the case
when $a\uuu{jk}=0$ for pairs $(j.k)$ with  $j\leq p$ and $k>p+j$ or with
$k\leq p$ and $j>p+k$.
Here $L\uuu m, S\uuu m$ and $N$  are unchanged and we get
\[ 
D=(\vvv 1)^p\cdot \prod\uuu{j=1}^{j=p}\,
a\uuu{j,p+j}\cdot \prod\uuu{k=1}^{k=p}\,a\uuu {p+k,k}
\cdot 
\det \begin{pmatrix}
a\uuu{p+1,2p+1}&\ldots &a\uuu{p+1,p+q}\\
\ldots&
\ldots&\ldots
\\
a\uuu{p+q,p+1}&\ldots &a\uuu{p+q,p+q}\\
\end{pmatrix}
\]
\medskip


\noindent
The absolute value of the last determinant is
majorized by   Hadamard's inequality in � F.XX   and 
the requested inequality in Theorem E.5 follows.




\newpage


\centerline{\bf{2. Resolvent matrices}}
\bigskip

\noindent
Let $A$ be some matrix $n\times n$-matrix.
Its characteristic polynomial is defined by
\[ 
P_A(\lambda)=
\text{det}(\lambda\cdot E_n-A)\tag{*}
\]
By the fundamental theorem of
algebra $P\uuu A$
has $n$ roots
$\alpha_1,\ldots,\alpha_n$ where eventual
multiple roots are repeated.
The 
union of  distinct roots is denoted by
$\sigma(A)$ and called the spectrum of $A$.
Since matrices with non-zero determinants are invertible
we obtain a matrix valued function defined in
${\bf{C}}\setminus\sigma(A)$ by:
\bigskip
\[ 
R_A(\lambda)=
(\lambda\cdot E_n-A)^{-1}\quad\colon\quad
\lambda\in{\bf{C}}\setminus\sigma(A) \tag{**}
\]


\noindent
One refers to $R_A(\lambda)$ as the resolvent of $A$.
The map
\[ 
\lambda\mapsto R_A(\lambda)
\] 
yields a matrix-valued analytic function defined in
${\bf{C}}\setminus \sigma(A)$.
To see this we take some
$\lambda_*\in {\bf{C}}\setminus \sigma(A)$ and set
\[
R_*=(\lambda_*\cdot E_n-A)^{-1}
\]
Since
$R_*$ is a 2-sided inverse we have
the equality
\[
E_n=R_*(\lambda_*\cdot E_n-A)=
(\lambda_*\cdot E_n-A)\cdot R_*\implies
R_*A=AR_*
\] 
Hence the resolvent $R_*$ commutes with $A$.
Next,
construct the matrix-valued power series
\[
\sum_{\nu=1}^\infty (-1)^\nu\cdot \zeta^\nu\cdot (R_*A)^\nu\tag{1}
\]
which is convergent when $|\zeta|$ are small enough.
\medskip


\noindent
{\bf{2.1 Exercise.}}
Prove  the equality
\[
R_A(\lambda_*+\zeta)=R_*+\sum_{\nu=1}^\infty
(-1)^\nu\cdot \zeta^\nu\cdot R_*\cdot (R_*A)^\nu
\]
Thes local series expansions   show that
the resolvents yield a matrix-valued analytic function
in ${\bf{C}}\setminus\sigma(A)$.
We are going to use 
analytic function theory 
to establish results which after can be extended
to
an operational calculus for  linear operators on infinite dimensional 
vector spaces.
The  analytic constructions are     also useful to investigate
dependence  upon parameters. Here is 
an example.
Let
$A$ be an $n\times n$-matrix whose
characteristic polynomial $P_A(\lambda)$ has $n$ simple roots
$\alpha_1,\ldots,\alpha_n$. When
$\lambda$ is outside the spectrum $\sigma(A)$.
residue calculus  gives
the following   expression for
the resolvents:
\[
(\lambda\cdot E_n-A)^{-1}=
\sum_{k=1}^{k=n}
 \frac{1}{\lambda-\alpha_k}\cdot \mathcal C_k(A)\tag{*}
 \]
where each matrix $\mathcal C_k(A)$ is a polynomial in $A$ given by:
\[
 \mathcal C_k(A)=
 \frac{1}{\prod_{\nu\neq k}\, (\alpha_k-\alpha_\nu)}
\cdot\prod_{\nu\neq k}\,(A-\alpha_\nu E_n)
\]
The  formula (*)   goes back to work
by
Sylvester, Hamilton and Cayley.
The resolvent $R_A(\lambda)$
is also used to construct  the Cayley-Hamilton polynomial of $A$
which 
by definition this is the unique monic polynomial $P\uuu *(\lambda]$ in the 
polynomial ring ${\bf{C}}[\lambda]$
of smallest possible degree such that the associated
matrix
$p_*(A)=0$.
It is found as follows:
Let $\alpha_1,\ldots,\alpha_k$
be the distinct roots of $P_A(\lambda)$
so that
\[
P_A(\lambda)=
\prod_{\nu=1}^{\nu=k}\, 
(\lambda-\alpha_\nu)^{e_\nu}
\]
where $e_1+\ldots+e_k=n$.
Now the meromorphic and matrix-valued resolvent
$R_A(\lambda)$
has  poles at $\alpha_1,\ldots,\alpha_k$. If
the order of a pole at root $\alpha_j$ is denoted by
$\rho_j$ one has the inequality
$\rho_j\leq e(\alpha_j)$
which in general can be strict. The Cayley\vvv Hamilton polynomial
becomes:
\[
P_*(\lambda)=\prod_{\nu=1}^{\nu=k}\, 
(\lambda-\alpha_\nu)^{\rho_\nu}\tag{**}
\]
\medskip




\noindent
Now we begin to prove results in more detail.
To begin with one has the Neumann series expansion:

\medskip

\noindent
{\bf{Exercise.}}
Show that if $|\lambda|$ is 
strictly larger than the absolute values of the roots
of $P_A(\lambda)$, then the resolvent is given by the  series
\[ 
R_A(\lambda)=\frac{E_n}{\lambda}+ \sum_{\nu=1}^\infty
\,\lambda^{-\nu-1}\cdot A^\nu\tag{*}
\]

\noindent
{\bf{A differential equation.}}
Taking the complex derivative of $\lambda\cdot R_A(\lambda)$ 
in (*) we get
\[ \frac{d}{d\lambda}(\lambda R_A(\lambda))
=-\sum_{\nu=1}^\infty\,\nu\cdot\lambda^{-\nu-1}\cdot A^\nu\tag{1}
\]


\noindent
{\bf{Exercise.}}
Use (1) to prove that
if $|\lambda|$ is large then
$R_A(\lambda)$ satisfies the differential equation:
\[ 
\frac{d}{d\lambda}(\lambda R_A(\lambda))
+A[\lambda^2R_A(\lambda)-E_n-\lambda A]=0\tag{2}
\]


\noindent
Now
(2) and  the analyticity of the resolvent 
outside the spectrum of $A$ give:


\medskip

\noindent
{\bf 2.3 Theorem} \emph{Outside the spectrum
$\sigma(A)$
$R(\lambda)$
satisfies the differential equation}
\[
\lambda\cdot R_A'(\lambda)+R_A(\lambda)+\lambda^2\cdot
A\cdot R_A(\lambda)=
A+\lambda\cdot A^2
\]

\bigskip

\noindent
{\bf{2.4 Residue formulas.}}
Since the resolvent is analytic we can construct complex line integrals and
apply results in complex residue calculus.
Start from the Neumann series (*) above 
and  perform integrals over circles 
$|\lambda|=w$ where $w$ is large.
\medskip

\noindent
{\bf{2.5 Exercise.}}
Show that when $w$ is strictly 
larger than
the absolute value of every
root of
$P_A(\lambda)$ then
\[ A^k=\frac{1}{2\pi i}\int_{|\lambda|=w}\,
\lambda^k\cdot R_A(\lambda)\cdot d\lambda
\quad\colon\quad k=1,2,\ldots
\]
It follows that when $Q(\lambda)$ is an arbitrary polynomial then
\[ Q(A)=\frac{1}{2\pi i}\int_{|\lambda|=w}\,
Q(\lambda)\cdot R_A(\lambda)\cdot d\lambda\tag{*}
\]
In particular we take the identity $Q(\lambda)=1$ and obtain

\[ E_n=\frac{1}{2\pi i}\cdot\int_{|\lambda|=w}\,
R_A(\lambda)\cdot d\lambda\tag{**}
\]
Finally, show  that if $Q(\lambda)$ is a  polynomial
which has a zero of order
$\geq e(\alpha_\nu)$ at every root then
\[ 
Q(A)=0\tag{***}
\]

\bigskip

\noindent
{\bf{2.6 Residue matrices.}}
Let $\alpha_1,\ldots,\alpha_k$ be the distinct zeros
of $P_A(\lambda)$.
For a given root, say $\alpha_ 1$ of  multiplicity
$p\geq 1$ we have a local Laurent series expansion
\[
R_A(\alpha_1+\zeta)=
\frac{G_p}{\zeta^p}+\ldots+\frac{G_1}{\zeta}+
B_0+\zeta\cdot B_1+\ldots\tag{i}
\]
We refer to $G_1,\ldots,G_p$ as the residue matrices at $\alpha_1$.
Choose a polynomial $Q(\lambda)$ in ${\bf{C}}[\lambda]$
which vanishes up to the multiplicity at all the
the remaining roots
$\alpha_2,\ldots,\alpha_k$ while it has a zero of order $p-1$ at $\alpha_1$, i.e.
locally
\[
Q(\alpha_1+\zeta)= \zeta^{p-1}(1+q_1\zeta+\ldots)\tag{i}
\]

\noindent
{\bf{2.7 Exercise.}}
Use residue calculus and  (*) from Exercise 2.5 to show that:
\[ 
Q(A)=
\frac{1}{2\pi}\int_{|\lambda-\alpha_1|=\epsilon}\,
Q(\lambda)\cdot R_A(\lambda)\cdot d\lambda=
G_p\tag{*}
\]
Hence the matrix $G_p$ is a polynomial of $A$. In a similar way one proves that
every $G$-matrix in the Laurent series (i) is a polynomial in $A$.
\medskip


\noindent
{\bf{2.7 Some idempotent matrices.}}
Consider  a zero $\alpha_j$ and choose a polynomial
$Q_j$ such that
$Q_j(\lambda)-1$ has a zero of order $e(\alpha_j)$ at
$\alpha_j$ while $Q_j$ has a zero of order
$e(\alpha_\nu)$ at the remaining roots.
Set
\[ 
E_A(\alpha_j)=\frac{1}{2\pi i}\int_{|\lambda|=w}\,
Q_j(\lambda)\cdot R_A(\lambda)\cdot d\lambda\tag{1}
\]
where $w$ is large as in 2.5.
Since 
the polynomial 
$S=Q_j-Q_j^2$ vanishes up to the multiplicities
at all the roots of $P_A(\lambda$ we have $S(A)=0$ from (***) in 2.5
which entails that

\[
E_A(\alpha_j)=E_A(\alpha_j)\cdot E_A(\alpha_j)\tag{*}
\]
In other words, we have constructed an idempotent matrix.



 














\bigskip



\noindent
{\bf{2.8 The Cayley-Hamilton decomposition.}}
Recall the equality


\[ E_n=\frac{1}{2\pi i}\cdot\int_{|\lambda|=w}\,
R_A(\lambda)\cdot d\lambda
\]
where the radius $w$ is so large that the disc $D_w$ contains the zeros of 
$P_A(\lambda)$.
The previous construction of the $E$-matrices at
the roots of $P_A(\lambda)$  entail that
\[
 E_n=E_A(\alpha_1)+\ldots+E_A(\alpha_k)
\]


\noindent
Identifying $A$ with a ${\bf{C}}$-linear operator on
${\bf{C}}^n$ we obtain a direct sum decomposition
\[ 
{\bf{C}}^n= V_1\oplus\ldots\oplus V_k\tag{*}
\] 
where each $V_\nu$ is an $A$-invariant subspace given by 
the image of $E_A(\alpha_\nu)$.
Here $A-\alpha_\nu$ restricts to a \emph{nilpotent} linear operator on
$V_\nu$ and the dimension of this vector space is equal to the multiplicity
of the root $\alpha_\nu$ of the characteristic polynomial.
One refers to (*) as the \emph{Cayley-Hamilton decomposition} of
${\bf{C}}^n$.
\bigskip

\noindent
{\bf{2.9 The vanishing of $P_A(A)$}}.
Consider the characteristic polynomial $P_A(\lambda)$. By definition it vanishes up to the order
of multiplicity at every point in $\sigma(A)$ and hence (***) in 2.5
gives
$P_A(A)=0$.
Let us write:
\[
 P_A(\lambda)= 
 \lambda^n+c_{n-1}\lambda^{n-1}+\ldots+ c_1\lambda+c_0
\]

\noindent
Notice that
$c_0=(-1)^n\cdot\text{det}(A)$.
So if the determinant of $A$ is $\neq 0$
we get
\[ 
A\cdot [A^{n-1}+c_{n-1}A^{n-2}+\ldots+ c_1\bigr]=
(-1)^{n-1}
\text{det}(A)
\cdot E_n
\]



\noindent
Hence 
the inverse $A^{-1}$ is  expressed as a polynomial in $A$.
Concerning the equation
\[ 
P_A(A)=0
\]
it is in general not the minimal 
equation for $A$, i.e. it can occur that $A$ satisfies an equation of degree $<n$.
More precisely , if $\alpha_\nu$ is a  root of some multiplicity
$k\geq 2$ there exists a  Jordan decomposition which gives an 
integer $k_*(\alpha_\nu)$ for the largest Jordan block
attached to the nilpotent operator $A-\alpha_\nu$ on $V_{\alpha_\nu}$.
The \emph{reduced} polynomial $P_*(\lambda)$ is  the product
where the factor $(\lambda-\alpha_\nu)^{k_\nu}$ is replaced by
$(\lambda-\alpha_\nu)^{k_*(\alpha_\nu)}$
for every  $\alpha_\nu$ where $k_\nu<k_*(\alpha_\nu)$ occurs. 
Then $P_*$ is the polynomial of smallest possible degree such that
$P_*(A)=0$.
One  refers to $P_*$ as the \emph{Hamilton polynomial} attached to $A$.
This   result relies upon Jordan's result in � 3.

\medskip





\noindent
{\bf{2.10 Similarity of matrices.}}
Recall
that
the determinant of  a matrix $A$ does not  change when it is
replaced by $SAS^{-1}$
where $S$ is an arbitrary invertible matrix.
This implies  that
the coefficients of the
characterstic polynomial  $P_A(\lambda)$
are intrinsically defined via the
associated linear operator, i.e. if another basis is chosen in
${\bf{C}}^n$ the given $A$-linear operator is expressed by a matrix
$SAS^{-1}$ where $S$ effects the change of the basis.
Let us now draw an interesting consequence
of the previous operational calculus.
Let us give the following:
\medskip


\noindent {\bf 2.11 Definition.}
\emph{A pair of $n\times n$-matrices $A,B$ are
\emph{similar} if there exists some invertible matrix $S$ such that}
\[ 
B=SAS^{-1}
\]


\noindent
Since the product of two invertible matrices is invertible this yield
an equivalence relation on $M_n({\bf{C}})$ and from 2.2 above we conclude that
$P_A(\lambda)$�only depends on its equivalence class.
The question arises if  to matrices $A$ and $B$ whose characteristic 
polynomials are equal
also are similar in the sense of Definition 2.6.
This is not true in general.
More precisely,
\emph{Jordan normal form}
determines the eventual similarity between
a pair of matrices with the same characteristic polynomial.

\bigskip




\centerline {\bf\large 3.  Jordan's normal form}
\bigskip

\noindent
{\bf Introduction.}
Theorem 3.1 below is due to 
Camille Jordan. It plays
an important role when we discuss multi-valued analytic functions in punctured discs
and is also  used  in ODE-theory.
Jordan's theorem   says
that every equivalence class in $M_n({\bf{C}})$
contains a 
matrix which is built up by Jordan blocks which are defined below.


\medskip


\noindent
Before we enter Jordan's Theorem we
discuss some consequences of the material in the previous section.
The Cayley-Hamilton decomposition from 2.7.
shows that an arbitrary $n\times n$\vvv matrix $A$
has a similar matrix
$B=S^{-1}AS$ which is represented in a block form. More precisely, to every root
$\alpha_\nu$ of some multiplicity
$e(\alpha_\nu)$
there occurs a square matrix $B_\nu$
of size
$e(\alpha_\nu)$ and $\alpha_\nu$ is the only root of
$P_{B_\nu}(\lambda)$.
It follows that for every fixed $\nu$ one has
\[
B\uuu\nu= \alpha\cdot E\uuu {k\uuu \nu}+
S\uuu\nu
\]
where $E\uuu {k\uuu \nu}$ is an identity matrix of size
$k\uuu\nu$ and $S\uuu\nu$ is nilpotent, i.e. there exists an integer $m$ such that
$S\uuu\nu^m=0$.
Jordan's theorem gives a further
description of these nilpotent $S$\vvv matrices
which 
therefore yields a refinement of the Cayley-Hamilton decomposition.
\medskip

\noindent
{\bf{3.0 Jordan blocks.}}
An \emph{elementary} Jordan matrix of size $4$
is  matrix of the form
\[
\begin{pmatrix}
\lambda&0&0&0\\
1&\lambda&0&0\\
0&1&\lambda&0\\
0&0&1&\lambda\\
\end{pmatrix}
\]
where $\lambda$ is the eigenvalue. For $k\geq 5$ one has similar 
expressions. In general several elementary Jordan block matrices
build up a matrix which is said to be in Jordan's normal form.

\medskip

\noindent {\bf 3.1 Theorem}. \emph{For every matrix $A$ there
exists an invertible matrix $u$ such that
$UAU^{\vvv 1}$ is in Jordan's normal form.}



\bigskip

\noindent 
\emph{Proof.} By the remark after Proposition 2.12
it suffices to prove Jordan's result when $A$ has a single eigenvalue 
$\alpha$ and
replacing $A$ by $A-\alpha$
there remains only to consider the nilpotent case, i.e when
$P_A(\lambda)=\lambda^n$ so that $A^n=0$.
In this nilpotent case we must find a basis
where $A$ is represented in Jordan's normal form.
This is done below.

\bigskip



\noindent {\bf 3.2 The case of nilpotent operators.}
Let $S$ be a nilpotent ${\bf{C}}$-linear operator on some $n$-dimensional 
complex vector space $V$. 
So for  each non-zero vector in $v\in V$ there exists a unique integer
$m$ such that
\[ 
S^m(v)=0\quad\text{and}\quad
S^{m-1}(v)\neq 0
\]
The unique integer $m$ is denoted by
$\text{ord}(S,v)$. The case $m=1$ occurs if  
$S(v)=0$. If $v$ has  order $m\geq 2$ one 
verifies that the vectors $v,S(v),\ldots,S^{m-1}(v)$ are linearly independent.
The vector space generated by this $m$-tuple is denoted by
$\mathcal C(v)$ and  called a \emph{cyclic} subspace of $V$.
When $S(v)=$ the corresponding cyclic space
$\mathcal C(v)$ is reduced to ${\bf{C}}\cdot v$.
With these notations Jordan's theorem amounts to prove the following:
\bigskip

\noindent {\bf 3.3 Proposition}
\emph{Let $S$ be a nilpotent linear operator. Then $V$ is a direct sum of
cyclic subspaces.}
\medskip

\noindent\emph
{Proof.}
Set
\[ m^*=
\max_{v\in V}\, \text{ord}(S,v)
\]
Choose $v^*\in V$ such that
$\text{ord}(S,v^*)=m^*$ and
construct the quotient space
$W=\frac{V}{\mathcal C(v^*)}$ on which $S$ induces a linear operator
denoted by $\bar S$.
By  induction over  $\text{dim}(V)$ we may assume that
$W$ is a direct sum of cyclic subspaces. Hence
we can pick
a finite set of vectors $\{v_\alpha\}$ in $V$ such that if
$\{\bar v_\alpha\}$ are the images in $W$, then
\[
W=\oplus\,\mathcal C(\bar v_\alpha)\tag{1}
\]
To each
$\alpha$
we have  the integer
$k\uuu\alpha=\text{ord}(\bar S,\bar v_\alpha)$.
The construction of a quotient space  means that
\[
S^{k\uuu\alpha}(v_\alpha)\in\mathcal C(v^*)\tag{2}
\]
Hence there exists  some
$m^*$-tuple $c_0,\ldots,c_{m-1}$�in ${\bf{C}}$
such that 

\[ 
S^{k\uuu\alpha}(v_\alpha)=c_0\cdot v^*+c_1\cdot S(v^*)+ \ldots+c_{m-1}\cdot S^{m^*-1}(v^*)\tag{3}
\]
Next, put
\[
k^*\uuu\alpha= 
\text{ord}(S,v_\alpha)\tag{4}
\]
It is obvious that
$k^*\uuu\alpha\geq k\uuu \alpha$.
If  this inequality is
strict we use (3) and obtain

\[ 
0=S^{k^*\uuu\alpha}(v_\alpha)=
\sum\, 
c\uuu\nu\cdot
S^{k^*\uuu\alpha\vvv k\uuu\alpha+\nu}(v^*)
\]
The maximal choice of $m^*$ entails that $k^*\uuu\alpha\leq m^*$
which 
gives
\[
c\uuu 0=\ldots=c\uuu{k\uuu\alpha\vvv 1}=0\tag{5}
\]
Hence (3) enable us to find
$w\uuu\alpha\in \mathcal C(v^*)$
such that
\[
S^{k\uuu\alpha}(v\uuu \alpha)= S^{k\uuu\alpha}(w\uuu \alpha)\tag{6}
\]
Now the images of $v\uuu\alpha$ and $v\uuu\alpha\vvv w\uuu\alpha$
are equal in $\mathcal C(v^*)$. So if
$\{v\uuu \alpha\}$ are replaced by $\{\xi\uuu\alpha= v\uuu\alpha\vvv w\uuu\alpha\}$
one still has
\[
W=\oplus\,\mathcal C(\bar \xi_\alpha)\tag{7}
\]
Moreover, the construction of the $\xi$\vvv vectors entail that
\[
\text{ord}(\bar S,\bar\xi_\alpha)=\text{ord}(S,v_\alpha)\tag{8}
\]
hold for each $\alpha$. At this stage
an obvious counting of dimensions give the requested
direct sum decomposition
\[
V=\mathcal C(v^*)\,\oplus \mathcal C(\xi\uuu\alpha)
\]





\noindent
{\bf Remark.}
The proof was  bit cumbersome. The reason  is
that the direct sum decomposition  
in Jordan's Theorem is not unique.
Only the   individual \emph{dimensions}
of the cyclic subspaces which appear in a direct sum decomposition are unique.
It is instructive to perform Jordan decompositions using
an implemented program which for example
can be found in
\emph{Mathematica}.


\bigskip

\centerline{\bf {4. Hermitian and Normal operators.}}

\bigskip

\noindent
The $n$-dimensional vector space
${\bf{C}}^n$ is equipped with
the hermitian inner product:
\[ 
\langle x,y\rangle= x_1\bar y_1+\ldots+x_n\bar y_n
\]
A basis $e_1,\ldots,e_n$ is orthonormal if
$\langle e_i,e_k\rangle=\text{Kronecker's delta function}$.
A linear operator
$U$ is  unitary if
it preserves the inner product:
\[
\langle U(x),U(y)\rangle=
\langle x,y\rangle
\]
for all $x$ and $y$.
It is clear that a unitary operator $U$ 
sends an orthonormal basis to another 
orthonormal basis and the reader may verify
that a linear operator $U$ is unitary if and only if
\[
U^{-1}=U^*
\]

\medskip

\noindent
{\bf{4.0.1 Adjoint operators.}}
Let $A$ be a linear operator. Its adjoint $A^*$ is the linear operator for which
\[
\langle A(x),y\rangle=
\langle x,A^*(y)\rangle
\] 

\noindent
{\bf{4.0.2 Exercise.}}
Show that if  $e_1,\ldots,e_n$ is an arbitrary
orthonormal basis in the inner product space
${\bf{C}}^n$
where $ A$ is represented by a matrix with elements
$\{a_{p,q}\}$, then $A^*$ is represented by the matrix whose elements are
\[
 a^*_{pq}=\bar a_{qp}
\]

\medskip

\noindent
{\bf{4.0.3 Hermitian operators.}}
A linear operator $A$ is called Hermitian if
\[
\langle A(x),y\rangle=
\langle x,A(y)\rangle
\]
holds for all $x$ and $y$.
An equivalent condition is that $A$ is equal to its adjoint $A^*$. 
Therefore one also
refers to a self-adjoint operator, i.e the notion of a hermitian respectively
self-adjoint  matrix  is the same.
\medskip


\noindent
{\bf{4.0.4 Self-adjoint projections.}}
Let $V$ be a subspace of ${\bf{C}}^n$ of some dimension
$1\leq k\leq n-1$.
Its orthogonal complement is denoted by $V^\perp$ and we have 
the direct sum decomposition
\[
{\bf{C}}^n=V\oplus V^\perp
\]
To $V$ we associate the linear operator $E$ whose kernel is
$V^\perp$ while it restricts to the identity on $V$.
Here
\[
E=E^2\quad\text{and}\quad E=E^*
\]
One refers to $E$ as a self-adjoint projection.
\medskip


\noindent
{\bf{4.0.5 Exercise.}}
Show  that if $E$ is some $n\times n$-matrix
which is idempotent in $M_n({\bf{C}})$ and Hermitian
in the sense of 4.0.3 then $E$
is the self-adjoint projection attached to the subspace
$V=E({\bf{C}}^n)$.
\medskip


\noindent
{\bf{4.0.6 Orthonormal bases.}}
Let $V_1\subset V_2\subset \ldots V_n={\bf{C}}^n$
be a strictly increasing sequence of subspaces. So here each $V_k$
has dimension $k$.
The \emph{Gram-Schmidt orthogonalisation}
yields  an orthonormal basis $\xi_1,\ldots,\xi_n$
such that
\[
V_k={\bf{C}}\cdot\xi_1+\ldots+
{\bf{C}}\cdot\xi_k
\] 
hold for every $k$.
The verification of this wellknown construction is left to the reader.
Next, if $A$ is an arbitrary $n\times n$-matrix the fundamental theorem of algebra implies
that there exists a sequence
$\{V_k\}$ as above such that
every $V_k$ is $A$-invariant, i.e.
\[
 A(V_k)\subset V_k
\] 
hold for each $k$.
We find the orthonormal basis $\{\xi_k\}$
and construct the unitary operator $U$ which sends the standard basis in
${\bf{C}}^n$ onto this $\xi$-basis.
In this $\xi$-basis we see  that the linear operator $A$ is represented by an upper
triangular matrix. Hence we have

\medskip

\noindent
{\bf{4.0.7 Theorem.}}
\emph{For every $n\times n$-matrix $A$ there exists a unitary matrix
$U$ such that
$U^*AU$ is upper triangular.}


\newpage








\centerline {\bf{4.1 The spectral theorem.}}
\medskip

\noindent
This important result asserts the following:

\medskip

\noindent
{\bf{Theorem.}}
\emph{If $A$ is Hermitian  there exists
an orthonormal basis $e_1,\dots,e_n$ 
in ${\bf{C}}^n$ where each $e_k$ is an eigenvector 
to $A$ whose eigenvalue is a real number. Thus,
$A$ can be diagonalised in an
orthonormal basis and  expressed  by matrices this means that 
there exists
a unitary matrix $U$ such that}
\[ 
U^*AU= S\tag{*}
\] 
\emph{where $S$ is a diagonal  matrix and  every $s_{ii}$
is a real number. In particular the roots of the characteristic polynomial
$\text{det}(P_A(\lambda))$ are all real.}

\medskip

\noindent
\emph{Proof.}
Since $A$ is self-adjoint we have
a real-valued function on ${\bf{C}}^n$ defined by
\[ 
x\mapsto \langle Ax,x\rangle\tag{1}
\]
Let $m^*$ be the  maximum 
of (1) as $x$ varies over the compact unit sphere of unit vectors in
${\bf{C}}^n$.
The maximum is attained by some
complex vector $x_*$ of unit length. 
Suppose $y$ is a unit vector where
 that $y\perp x_*$ and let $\lambda$ be a complex number.
Since $A$ is self-adjoint we have:
\[
 \langle A(x_*+\lambda y),x_*+\lambda y\rangle=
 m^*+2\cdot\mathfrak{Re}
 \bigl(\lambda\cdot
\langle Ax_*,y\rangle\bigr)+|\lambda|^2\cdot
\langle Ay,y\rangle\tag{2}
\]
Now $x+\lambda y$ has norm $\sqrt{1+\lambda|^2}$ 
 and the maximality gives:
 \[
 m^*+2\cdot\mathfrak{Re}
 \bigl(\lambda\cdot
\langle Ax_*,y\rangle\bigr)+|\lambda|^2\cdot
\langle Ay,y\rangle\leq\sqrt{1+|\lambda|^2}\cdot m^*\tag{3}
\]
Suppose now that
$\langle Ax_*,y\rangle\neq 0$
and set
\[
\langle Ax_*,y\rangle= s\cdot e^{i\theta}\quad\colon\quad   s>0
\]
With $\delta>0$ we take
 $\lambda= \delta\cdot e^{-i\theta}$ and (3) entails
that
\[
2s\cdot\delta\leq (\sqrt{1+\delta^2}-1)\cdot m^*-\langle Ay,y\rangle\cdot\delta^2\tag{4}
\]
Next, by calculus one has
$2\cdot \sqrt{1+\delta^2}-1)\leq\delta^2$ so after division with $\delta$
we get
\[
2s\leq\delta\cdot\bigl(\frac{m_*}{2}- \langle Ay,y\rangle\bigr)\tag{5}
\]
But this is impossible for arbitrary small $\delta$
and hence we have proved that
\[ 
y\perp x_*\implies \langle Ax_*,y\rangle=0\tag{6}
\]

\noindent
This means that  $x_*^\perp$
is an invariant subspace for $A$
and the restricted operator remains self-adjoint. At this stage
the reader can finish the proof
to get a unitary matrix $U$ such that (*) holds.



\bigskip


\centerline {\bf{4.2 Normal operators.}}
\medskip

\noindent
An $n\times n$-matrix
$A$ is  normal if it commutes with its adjoint, i.e. 
\[ 
A^*A=AA^*\quad\text{holds in}\quad M_n({\bf{C}})\tag{*}
\]

\medskip

\noindent
{\bf{4.2.0 Exercise.}}
Let $A$ be a normal matrix. Show that every
equivalent
matrix is normal, i.e. if $S$ is invertible then
$SAS^{-1}$ is also normal. The hint is to use that
\[ 
(S^{-1})^*=(S^*)^{-1}
\] 
holds
for every invertible matrix.
Conclude from this that we can refer to normal linear operators
on ${\bf{C}}^n$.
\medskip

\noindent
{\bf{4.2.1 Exercise.}}
Let $A$ and $B$ be two Hermitian matrices which commute, 
i.e. $AB=BA$. Show that the matrix 
$A+iB$ is normal.





\medskip

\noindent
Next, let $R$ be normal and assume that
the its characteristic polynomial has simple roots.
This means that there exists  a basis $\xi_1,\ldots,\xi_n$
formed by eigenvectors to $R$ with eigenvalues
$\lambda_1,\ldots,\lambda_n$.
Thus:
\[ 
R(\xi_\nu)=\lambda_\nu\cdot \xi_\nu\quad\colon\quad 1\leq\nu\leq n\tag{*}
\] 
Notice that $R$ is invertible if and only if
al the eigenvalues are $\neq 0$. It turns out that the normality
gives a stronger conclusion.

\medskip

\noindent
{\bf{4.3 Proposition.}} \emph{Assume that  the eigenvalues
are $\neq 0$. Then
the $\xi$-vectors in (*) are orthogonal.}
\medskip

\noindent
\emph{Proof.}
Consider some  eigenvector, say $\xi_1$.
Now we get
\[ 
R(R^*(\xi_1))=
R^*(R(\xi_1))=\lambda_1\cdot 
R^*(\xi_1)\tag{i}
\]
Hence $R^*(\xi_1)$ is an eigenvector to $R$ with eigenvalue $\lambda_1$. By 
hypothesis this eigenspace is 1-dimensional which gives
\[ 
R^*(\xi_1)=\mu\cdot \xi_1\implies
\] 
\[
\lambda_1\cdot \langle \xi_1,\xi_1\rangle=
\langle R(\xi_1),\xi_1\rangle=
\langle \xi_1),R^*(\xi_1)\rangle=\bar\mu\cdot
\langle \xi_1,\xi_1\rangle
\]


\noindent
Hence  $\mu=\bar\lambda_1$ which shows that
the eigenvalues of $R^*$ are the complex conjugates of
the eigenvalues  of $R$. There remains to show that
the $\xi$-vectors are orthogonal.
Consider two eigenvectors, say $\xi_1,\xi_2$. Then
we obtain:
\[ 
\bar\lambda_2\lambda_1\cdot\langle \xi_1,\xi_2\rangle=
\langle R\xi_1,R\xi_2\rangle
=\langle \xi_1,R^*R\xi_2\rangle
\langle \xi_1,RR^*\xi_2\rangle=
\]
\[
\langle R^*\xi_1,R^*\xi_2\rangle=\bar\lambda_1\cdot \lambda_2\cdot
\langle \xi_1,\xi_2\rangle\implies
(\bar\lambda_2\lambda_1-\lambda_2\bar\lambda_1)\cdot\langle \xi_1,\xi_2\rangle=0\tag{ii}
\]

\noindent
By assumption   $\lambda_1\neq\lambda_2$ and both are $\neq 0$. It follows
that 
$\bar\lambda_2\lambda_1-\lambda_2\bar\lambda_1\neq 0$ and then (ii) gives
$\langle \xi_1,\xi_2\rangle=0$ as required.
\medskip

\noindent
{\bf{4.4 Remark.}}
Proposition 4.3  shows that if $R$ is an invertible normal operator with
$n$ distinct eigenvalues then
there exists a unitary matrix
$U$ such that
$U^*RU$ is a diagonal matrix. But in contrast to the Hermitian
case the eigenvalues can be  complex.
\medskip


\noindent
{\bf{4.5 Exercise.}}
Let $R$ as above be an invertible normal operator
with distinct eigenvalues.
Show that $R$ is a Hermitian matrix if and only if
the eigenvalues are real numbers.

\medskip


\noindent
{\bf{4.6 Theorem.}} \emph{Let $R$  be an invertible normal operator
with distinct eigenvalues. Then there exists a unique pair of 
Hermitian operators $A,B$ such that
$AB=BA$ and}
\[
 R=A+iB
\]


\noindent{\bf{4.7 Exercise.}} Prove Theorem 4.6.

\medskip

\noindent
{\bf{4.8 The operator $R^*R$}}.
Let $R$ as above be  an invertible  normal operator
with eigenvalues $\lambda_1,\ldots,\lambda_n$.
From Remark 4.4 it is clear that
$R^*R$ is a Hermitian operator whose eigenvalues all are
given by the positive numbers
$\{|\lambda_\nu|^2\}$ and if $A.B$ are the Hermitian operators in
Theorem 4.6 then we have
\[ 
R^*R= A^2+B^2
\] 
Thus, $R^*R$ is represented as a sum of squares of two pairwise commuting
Hermitian operators.
\medskip

\noindent{\bf{4.9 The normal operator
$(A+iE_n)^{-1}$}}.
Let $A$ be a arbitrary  Hermitian $n\times n$-matrix. 
We have already seen that
its eigenvalues are real . Let us denote them
by $r_1,\ldots,r_n$. The spectral theorem gives  
a unitary matrix $U$ such that
$U^*AU$ is diagonal
with elements $\{r_\nu\}$. It follows that the matrix
$A+iE_n$ is invertible and its inverse
\[
R=
(A+iE_N)^{-1}
\]
is a normal operator with eigenvalues $\{\frac{1}{r_\nu+i}\}$.

\bigskip

\centerline {\bf{4.10 The case of multiple roots}}
\medskip

\noindent
The assumption that the eigenvalues of a normal operator are all distinct can be 
relaxed. Thus, for every normal and invertible operator $R$ there  exists
a unitary operator $U$ such that $U^*RU$ is diagonal.


\medskip

\noindent{\bf{4.11 Exercise.}}
Prove the assertion above.
The hint is to establish the following
which has independent interest:
\medskip

\noindent {\bf{4.12 Proposition.}}
\emph{Let  $R$ be  normal and nilpotent. Then $R=0$}
\medskip

\noindent
\emph{Proof.}
By Jordan's Theorem it suffices to prove this when
$R$ is a single Jordan block
represented by a special $S$-matrix whose elements
below the diagonal, are 1 while all the other elements are zero.
If $n=2$ we have for example
\[
S=
\begin{matrix}
0&0\\1&0\end{matrix}\implies S^*=
\begin{matrix}
0&1\\0&0\end{matrix}
\]
The reader verifies that $S^*S\neq SS^*$ and a similar calculation gives
Proposition 4.12 for every $n\geq 3$.

\medskip

\noindent
{\bf{4.13 Remark.}} The  result above  means that if $R$ is normal
then there never   
appear Jordan blocks of size $>1$
and hence there exists an invertible matrix $S$ such that
$SRS^{-1}$ is diagonal.




\bigskip

\centerline{\bf\large{6. Carleman's inequality}}
\medskip

\noindent

\noindent
{\bf{Introduction }} Theorem 6.1 below was proved by
Carleman in the article
\emph{Sur le genre du denominateur $D(\lambda)$
de Fredholm}
from 1917. At that time the result was used 
to study non-singular integral equations of the Fredholm type.
For more recent applications of Theorem 6.1
we refer to Chapter XI
in [Dunford-Schwartz]. 
\medskip

\noindent
{\bf{The Hilbert-Schmidt norm. }} It is defined 
for an  
$n\times n$-matrix $A=\{a_{ik}\}$  by:
\[
||A||= \sqrt{ \sum\sum\, |a_{ik}|^2}
\]


\noindent
where the double sum extends over all pairs
$1\leq i,k\leq n$.
Notice that this norm is the same as
\[
||A||^2= \sum_{i=1}^{i=n}\, ||A(e_i)||^2
\] 
where $e_1,\ldots,e_n$ can be taken as an arbitrary orthogonal basis in
${\bf{C}}^n$.
Next, 
for a linear operator $S$
on ${\bf{C}}^n$
its \emph{operator norm} is defined by
\[
\text{Norm}[S]=
\max_x\, ||S(x)||\quad\text{with the maximum taken over unit vectors.}
\] 

\noindent
{\bf{6.1 Theorem.}}
\emph{Let $\lambda_1,\ldots,\lambda_n$ be the roots of 
$P_A(\lambda)$ and  $\lambda\neq 0$ is outside $\sigma(A)$. Then
one has the inequality:}
\[ 
\bigl|\prod_{i=1}^{i=n}\, 
\bigl[1-\frac{\lambda_i}{\lambda}\bigr ]
e^{\lambda_i/\lambda}\bigr|\cdot \text{Norm}\bigl[R_A(\lambda)\bigr]
\leq |\lambda|\cdot \text{exp}
\bigl(\frac{1}{2}+ \frac{||A||^2}{2\cdot |\lambda|^2}\bigr)
\]
\medskip

\noindent
The proof requires some preliminary results.
First we  need  inequality due to  Hadamard which
goes as follows:
\medskip

\noindent
{\bf {6.2 Hadamard's inequality.}}
\emph{For every matrix $A$ with a non-zero determinant one has the inequality}
\[
\bigl|\text{det}(A)\bigr|\cdot \text{Norm}(A^{-1})\leq
\frac{||A||^{n-1}}{(n-1)^{n-1)/2}}
\]


\noindent
{\bf{Exercise.}} Prove this  result. 
The hint is to  use
expansions of certain determinants while one
considers 
$\text {det}(A)\cdot \langle A^{-1}(x),y\rangle$ 
for all pairs of unit vectors $x$ and $y$.

\bigskip

\noindent
{\bf{6.3 Traceless matrices.}}
Let $A$ be an  $n\times n$-matrix. The trace is by definition given by:
\[
\text{Tr}(A)= b_{11}+\ldots+b_{nn}\tag{i}
\]
Recall  that 
$-\text{Tr}(A)$ is
equal to the sum of the roots of $P_A(\lambda)$.
In particular the trace of two equivalent matrices are equal.
This will be used to prove the following:
\medskip

\noindent
{\bf{6.4 Theorem.}}
\emph{Let $A$ be an $n\times n$-matrix whose trace is zero. Then there
exists a unitary matrix
$U$ such that  the diagonal elements of $U^*AU$ all are zero.}
\medskip


\noindent
\emph{Proof}. Consider
first consider the case $n=2$. By Theorem 4.0.7
it suffices to consider the case when the $2\times 2$-matrix $A$
is upper diagonal and since the trace is zero it has the form
\[ 
A=
\bigl(\,\begin{matrix} a&b\\0&-a
\end{matrix}\,\bigr)
\]
where $a,b$ is a pair of complex numbers.
If $a=0$ then the two diagonal elements are zero and wee can take $U=E_2$ to be the identity in Lemma 6.5. If $a\neq 0$ we consider a vector $\phi=(1,z)$ in ${\bf{C}}^2$.
Then $A(\phi)$ is the vector $(a+bz,-az)$ and hence the inner product becomes:
\[
\langle A(\phi),\phi\rangle=a+bz-a|z|^2\tag{i}
\]
We can write
\[
\frac{b}{a}= re^{i\theta}
\]
where $r>0$ and then (i) is zero if
\[
|z|^2=1+se^{i\theta}\cdot z\tag{ii}
\]
With $z=se^{-i\theta}$
it amounts to find a positive real number $s$ such that
$s^2=1+s$ which clearly exists.
Now we get the vector
\[ 
\phi_*=\frac{1}{1+s^2}(1,se^{-i\theta})
\]
which has unit length and 
\[
\langle A(\phi_*),\phi_*\rangle=0\tag{ii}
\]
By 4.0.6 we find another unit vector $\psi_*$ so that
$\phi_*,\psi_*$ is an orthonormal base in
${\bf{C}}^2$ and hence there exists a unitary matrix
$U$ such that $U(e_1)=\phi_*$ and $U(e_2)= \psi_*$.
If $B=U^*AB$ the vanishing in (ii) gives $b_{11}=0$. At the same time
the trace is unchanged, i.e. $\text{tr}(B)=0$ holds and hence
we also get
$b_{22}=0$. This means  that  the diagonal elements of $U^*AU$ 
are both zero as required.
\medskip

\noindent{\bf{The case $n\geq 3$}}.
For the
induction
the following is needed:
\medskip

\noindent
\emph{Sublemma.} \emph{Let $n\geq 3$
and assume as above that
$\text{Tr}(A)=0$. Then there exists some
non-zero vector $\phi\in{\bf{C}}^n$ such that}
\[
\langle A(\phi),\phi\rangle=0\tag{*}
\]

\noindent
\emph{Proof.}
If (*) does not hold we
get the positive number
\[
m_*=\min_\phi\, \bigl|\langle A(\phi),\phi\rangle\bigr|
\]
where the minimum is taken over unit vectors in
${\bf{C}}^n$.
The minimum is achieved by some unit vector $\phi_*$. Let
$\phi_*^\perp$ be its orthonormal complement
and $E$ the self-adjoint projection from
${\bf{C}}^n$ onto $\phi_*^\perp$.
On the $(n-1)$-dimensional inner product space
$\phi_*^\perp$ we get the linear operator
$B=EA$, i.e. 
\[ 
B(\xi)= E(A(\xi))\quad\colon\quad \xi\in \phi_*^\perp\tag{i}
\]
If $\psi_1,\ldots,\psi_{n-1}$ is an orthonormal basis in
$\phi_*^\perp$ then the $n$-tuple $\phi_*,\psi_1,\ldots,\psi_{n-1}$
is an orthonormal basis in ${\bf{C}}^n$ and since the trace of $A$ is zero we get
\[
0=\langle A(\phi_*),\phi_*\rangle+
\sum_{\nu=1}^{\nu=n-1}\, \langle A(\psi_\nu),\psi_\nu\rangle=
m+\sum_{\nu=1}^{\nu=n-1}\, \langle B(\psi_\nu),\psi_\nu\rangle\tag{ii}
\]
where we used that $E(\psi_\nu)=\psi_\nu$ for each $\nu$
and that $E$ is self-adjoint so that

\[
\langle A(\psi_\nu),\psi_\nu\rangle=
\langle A(\psi_\nu),E(\psi_\nu)\rangle=\langle E(A(\psi_\nu)),\psi_\nu\rangle
=\langle B(\psi_\nu),\psi_\nu\rangle
\]
Now (ii)  gives
\[ 
\text{Tr}(B)=-m
\]
Hence the $(n-1)\times(n-1)$-matrix which represents
$B+\frac{m}{n-1}\cdot E$
has trace zero. By an induction over $n$ we find a unit vector
$\psi\in \phi_*^\perp$
such that
\[
\langle B(\psi_*),\psi_*\rangle=-\frac{m}{n-1}
\]
Finally, since $E$ is self-adjoint we have already seen that
\[
\langle A(\psi_*),\psi_*\rangle=\langle B(\psi_*),\psi_*\rangle\implies
\bigl|\langle A(\psi_*),\psi_*\rangle\bigr |=\bigl|\frac{m}{n-1}\bigr |=
\frac{m_*}{n-1}
\]
Since $n\geq 3$ the last number is $<m_*$ which contradicts the minimal choice 
of $m_*$.
Hence we must have $m_*=0$ which proves lemma 6.5
\bigskip

\noindent
{\emph{Final part of the proof.}
Let $n\geq 3$. The Sublemma  gives unit vector $\phi$
such that
$\langle A(\phi),\phi\rangle=0$.
Consider the hyperplane
$\phi^\perp$ and the operator $B$ from the Sublemma  which now has trace
zero on this $(n-1)$-dimensional space. So by an induction over
$n$
there exists an orthonormal basis $\psi_1,\ldots,\psi_{n-1}$ in
$\phi^\perp$ such that
$\langle B(\psi_\nu),\psi_nu\rangle=0$ for every $\nu$.
Now $\phi,\psi_1,\ldots,\psi_{n-1}$
is an orthonormal basis in ${\bf{C}}^n$ and if $U$
is the unitary matrix which has this $n$-tuple as column vectors
it follows that the diagonal elements of $U^*AU$ all vanish.
This finishes the proof of Theorem 6.4.


 
\newpage

\centerline{\bf{Proof Theorem 6.1}}.



\noindent
Set $B=\lambda^{-1}A$ so that $\sigma(B)=\{ \lambda_i/\lambda\}$ and
$\text{Tr}(B)=\sum\,\frac{\lambda_i}{\lambda}$.
We also have 

\[
||B||^2=\frac{||A||^2||}{|\lambda|^2}\quad\text{and}\quad
\bigl |\lambda\bigr |\cdot \text{Norm}[R_A(\lambda)]=\text{Norm}[(E-B)^{-1}]
\]


\noindent Hence Theorem 6.1 follows if we prove the inequality
\[
\bigl |e^{\text{Tr}(B)}\bigr|\cdot
\bigl|\prod_{i=1}^{i=n}\, 
\bigl[1-\frac{\lambda}{\lambda_i}\bigr ]
\cdot \text{Norm}
\bigl [E-B)^{-1}\bigr]
\leq \text{exp}\bigl[\frac{1+ ||B||^2}{2}\bigr]\tag{*}
\]


\noindent
To prove (*) we choose an arbitrary integer $N$ such that
$N>\bigl |\text{Tr}(B)\bigr|$ and for each such $N$ we define the linear operator
$B_N$ on the $n+N$-dimensional complex space with points 
denoted by $(x,y)$ with  $y\in{\bf{C}}^N$
as follows:
\[
B_N(x,y)= (Bx\, , \, -\frac{\text{Tr}(B)}{N}\cdot y)\tag{**}
\]


\noindent
The
eigenvalues of the linear operator $E-B_N$ is the union of the $n$-tuple 
$\{1-\frac{\lambda_i}{\lambda}\}$ and 
the $N$-tuple of equal eigenvalues given by 
$1+\frac{\text{Tr}(B)}{N}$.
This gives the determinant formula
\[
\text{det}(E-B_N)=
\bigl(1+\frac{\text{Tr}(B)}{N}\bigr)^N
\cdot \prod_{i=1}^{i=n}\, (1-\frac{\lambda_i}{\lambda}\bigr)\tag{1}
\]
The choice of $N$ implies that (1) is $\neq 0$ so 
the inverse $(E-B_N)^{-1}$ exists.
Moreover, the construction of $B_N$ gives
for any pair $(x,y)$ in ${\bf{C}}^{N+n}$:
\medskip
\[
(E-B_N)^{-1}(x,y)=
\bigl (E-B)^{-1}(x), \frac{y}{
1+\frac{1}{N}\cdot \text{Tr}(B)}\bigr)
\]


\noindent
It follows that 
\[
\text{Norm}\bigl[(E-B)^{-1})\bigr]\leq
\text{Norm}\bigl[
(E-B_N)^{-1}\bigr]\implies
\]
\[
\bigl|\text{det}(E-B_N)\bigr|\cdot \text{Norm}\bigl[(E-B)^{-1})\bigr]\leq
\bigl|\text{det}(E-B_N)\bigr|\cdot
\text{Norm}\bigl[
(E-B_N)^{-1}\bigr]\tag{2}
\]


\noindent 
Hadarmard's inequality
estimates the
hand side in (2) by:
\[
\frac{||E-B_N||^{N+n-1}}{(N+n-1)^{N+n-1)/2}}\tag{3}
\]
\medskip

\noindent
Next, the construction of $B_N$ implies that its trace is zero.
So  by the result in 6.3 we can find
an orthonormal basis $\xi_1,\ldots,\xi_{n+N}$
in ${\bf{C}}^{n+N}$ such that
\[ 
\langle B_N(\xi_k),\xi_k\rangle=0\quad\colon 1\leq k\leq n+N
\]


\noindent
Relative to this basis the matrix of $E-B_N$ 
has 1 along the diagonal and the negative of the
elements of $B_N$ elsewhere. It follows that the Hilbert-Schmidt norm
satisfies the equality:
\[
||E-B_N||^2=
N+n+||B_N||^2=N+n+||B||^2+ N^{-1}\cdot
|\text{Tr}(B)|^2\tag{4}
\]

\medskip

\noindent Hence, (1) and the inequalities from (2-3) give:
\medskip

\[
\bigl(1+\frac{\text{Tr}(B)}{N}\bigr)^N
\cdot \prod_{i=1}^{i=n}\, (1-\frac{\lambda_i}{\lambda}\bigr)\cdot
\text{Norm}\bigl[
(E-B)^{-1}\bigr]\leq
\] 

\[
\frac{\bigl(N+n+||B||^2+
N^{-1}\cdot
|\text{Tr}(B)|^2\bigr)^{(N+n-1)(2}}{
\bigl(N+n-1\bigr)^{N+n-1/2}}=
\frac{\bigl(1+\frac{||B||^2}{N+n}+
\frac{|\text{Tr}(B)|^2}{N(N+n)}\bigr)^{(N+n-1)/2}}{
(1-\frac{1}{N+n}\bigr)^{N+n-1/2}}
\]
\bigskip

\noindent
This inequality holds for
arbitrary large $N$.
Passing to the limit as $N\to\infty$  the definition of Neper's constant $e$
give

\[
\lim_{N\to\infty}\, \bigl(1+\frac{\text{Tr}(B)}{N}\bigr)^N=
e^{\text{Tr}(B)}
\]
and the reader may also verify that
the limit of the last term above is equal to
$\text{exp}\bigl[\frac{1+ ||B||^2}{2}\bigr]$ which finishes the proof of
(*) above and hence also of Theorem 6.1.

\bigskip

\centerline {\bf{0.C.2 Hadamard's inequality.}}
\medskip

\noindent
The following result is due
Hadamard whose proof is left as an exercise.

\medskip

\noindent
{\bf{0.C.3  Theorem.}}
\emph{Let $A=\{a\uuu{\nu k}\}$ be some $p\times p$\vvv matrix whose elements are
complex numbers. To each $1\leq k\leq p$ we set}
\[
\ell\uuu p= \sqrt{|a\uuu{1k}|^2+\ldots+|a\uuu{p k}|^2}
\]
\emph{Then}
\[ 
\bigl |\text{det}(A)\bigr |\leq
\ell\uuu 1\cdots \ell\uuu p
\]
\newpage


\centerline{\bf{7. Hadamard's radius theorem.}}

\bigskip

\noindent
Hadamard's thesis \emph{Essais sur l'�tudes des fonctions donn�s par leur
d�velopment d Taylor} contains many interesting results.
Here we expose material from Section 2 in [ibid].
Consider a power series
\[
 f(z)=\sum\, c\uuu nz^n
\]
whose radius is a positive  number
$\rho$.
So $f$ is analytic in the open disc $\{|z|<\rho\}$
and has at least one singular point on the circle
$\{|z|=\rho\}$.
Hadamard found a condition in order that
these singularities consists of a finite set of poles only so that
$f$ extends to be meromorphic in some disc $\{|z|<\rho\uuu *\}$ with
$\rho\uuu * >\rho$. The condition is expressed via properties of
the  Hankel determinants 
$\{\mathcal D\uuu n^{(p)}\}$ from � 0.B.
For each $p\geq 1$ we set 
\[
\delta(p)=\, 
\limsup\uuu{n\to \infty}\, 
[\mathcal D\uuu n^{(p)}]^{\frac{1}{n}}
\]

\noindent
In the special case $p=0$ we have $\{\mathcal D\uuu n^{(0)}\}=\{c\uuu n\}$
and hence Hadarmard's special and very wellknowen formula for
the radius of convergence of a powrer series gives the equality:

\[
\delta(0)= \frac{1}{\rho}=\limsup\uuu{n\to \infty}\, |c\uuu n|^{\frac{1}{n}}
\]
This entails that for every  $\epsilon>0$  there exists a constant $C\uuu\epsilon$ 
such that
\[ 
|c\uuu n|\leq C\cdot (\rho \vvv  \epsilon)^{\vvv n}\quad\text{
hold for every}\quad  n
\]
It follows trivially that
\[
|\mathcal D\uuu n^{(p)}|\leq (p+1) !\cdot C^{p+1}(\rho\vvv \epsilon)^{\vvv (p+1)n}
\]
Passing to limes superior where  high $n$:th roots are taken
we conclude that:
\[
\delta(p)= \limsup\uuu{n\to \infty}\, 
\bigl[\mathcal D\uuu n^{(p)}\bigr]^{\frac{1}{n}}\leq \rho^{\vvv (p+1)}\tag{1}
\]

\medskip


\noindent
Suppose   there exists some $p\geq 1$ 
where a strict inequality occurs:
\[
\delta(p)<\rho^{\vvv(p+1)}\tag{2}
\]
If $p$ is the smallest integer  where (2) holds we get
a number $\rho\uuu *>\rho$ such that
\[
\delta(p)=\rho\uuu *^{\vvv 1}\cdot
\rho ^{\vvv p}\tag{3}
\]


\medskip

\noindent
{\bf{7.1 Theorem.}} \emph{With $p$ and $\rho_*$ as in (3),
it follows that $f(z)$ extends to a meromorphic function in the disc
of radius $\rho\uuu *$ where the number of poles counted with multiplicity
is at most  $p$.}
\bigskip


\noindent
The proof requires several steps. To begin with one has

\medskip

\noindent
{\bf{7.2 Lemma. }}\emph{When $p$ as above is minimal one has
the unrestricted limit formula:}
\[
\lim\uuu{n\to \infty}\, 
\bigl[\mathcal D\uuu n^{(p\vvv 1)}\bigr]^{\frac{1}{n}}=
\rho ^{\vvv p}\tag{*}
\]

\bigskip

TO BE GIVEN: Exercise power series+ Sylvesters equation.

\bigskip


\noindent
{\bf{7.3 The meromorphic extension
to $\{|z|<\rho\uuu *\}$.}} Lemma 7.2 entails that if $n$ is large
$\{\mathcal D\uuu n^{(p\vvv 1)}\}$
are  $\neq 0$.
So there exists some $n_*$ such that every  $n\geq n_*$
gives a    unique $p$\vvv vector
$(A\uuu n^{(1)},\ldots, A\uuu n^{(p)})$
which solves the inhomogeneous system
\[
\sum\uuu{k=0}^{k=p\vvv 1}
\, c\uuu {n+k+j}\cdot A\uuu n^{(p\vvv k)}
=\vvv c\uuu {n+p+j}\quad\colon\quad 0\leq j\leq p\vvv 1
\]
Or expressed in matrix notation:
\[
\begin{pmatrix}
c\uuu{n}&c\uuu{n+1}&\ldots&c\uuu{n+p-1}\\
c\uuu{n+1}&c\uuu{n+2}&\ldots&c\uuu{n+p}\\
\ldots&\ldots&\ldots&\ldots\\
c\uuu{n+p-1}&c\uuu{n+p}&\ldots&c\uuu{n+2p-2}\\
\end{pmatrix}\,
\begin{pmatrix}A_n^{(p)}\\\ldots\\\ldots\\\ldots\\
A_n^{(1)}\end{pmatrix}=-
\begin{pmatrix}c_{n+p} \\\ldots\\\ldots\\\ldots\\
c_{n+2p-1}\end{pmatrix}\tag{*}
\]

\medskip


\noindent
{\bf{7.4 Exercise.}}
Put
\[
H\uuu n=
c\uuu{n+2p}+ A\uuu n^{(1)}\cdot  c\uuu{n+2p\vvv 1}+
\ldots+ A\uuu n^{[(p)} \cdot c\uuu{n+p}
\]
Show that the evaluation of  $\mathcal D\uuu n^{(p)}$ 
via an expansion of the last column gives the equality:
\[ 
H\uuu n=
\frac{\mathcal D\uuu n^{(p)}}{\mathcal D\uuu n^{(p\vvv 1)}}\tag{i}
\]
\medskip

\noindent

\noindent
Next,  the  limit formula   (3) above Theorem 7.1 together with
Lemma 7.2  
give for every $\epsilon>0$
a constant $C\uuu\epsilon$ 
such that the following hold for all sufficiently large $n$:
\[ 
|H\uuu n|\leq C\uuu\epsilon \cdot 
\bigl(\frac{\rho+\epsilon}{\rho\uuu *\vvv \epsilon}\bigr)^n\tag{ii}
\]

\noindent
Next, 
put
\[ 
\delta\uuu n^{k}=A\uuu {n+1}^{(k)}\vvv A\uuu n^{(k)}
\quad\colon\quad 1\leq k\leq p\tag{iii}
\]

\medskip

\noindent
Solving (*) above for $n$ and $n+1$ a computation shows that
the $\delta$\vvv numbers satisfy the system
\[
\sum\uuu{k=0}^{k=p\vvv 1}
\, 
c\uuu {n+j+k+1}\cdot \delta \uuu n^{(p\vvv k)}=0
\quad\colon\quad 0\leq j\leq p\vvv 2
\]
\[
\sum\uuu{k=0}^{k=p\vvv 1}
\, 
c\uuu {n+p+k}\cdot \delta \uuu n^{(p\vvv k)}=
\vvv (c\uuu{n+2p}+ A\uuu n^{(1)}\cdot  c\uuu{n+2p\vvv 1}+
\ldots+ A\uuu n^{[(p)} \cdot c\uuu{n+p})
\tag{iv}
\]
\medskip



\noindent
The $\delta$\vvv numbers in the linear system ( iv)
are found  via Cramer's rule. 
The minors of degree $p\vvv 1$ in the Hankel matrices 
$\mathcal C\uuu {n+1}^{(p\vvv 1)}$ have elements from
the given
$c$\vvv sequence and  (7.0)  implies
that every such minor has an absolute value majorized by
\[
C\cdot (\rho\vvv \epsilon)^{\vvv (p\vvv 1)n}
\] 
where $C$ is a constant 
which is independent of $n$.
We conclude that the $\delta$\vvv numbers satisfy
\[
|\delta \uuu n^{(k)}|\leq |\mathcal D\uuu n^{(p\vvv 1)}|^{\vvv 1}\cdot 
C\cdot (\rho\vvv \epsilon)^{\vvv (p\vvv 1)n}\cdot |H\uuu n|\tag{v}
\]


\noindent
The unrestricted limit in Lemma 7.2
give  upper bounds for
 $|\mathcal D\uuu n^{(p\vvv 1)}|^{\vvv 1}$ so that  (iii) and (v) give:
 
\medskip
 
 \noindent
{\bf{7.5 Lemma}}
 \emph{To each $\epsilon>0$ there is a constant
 $C\uuu\epsilon$ such that}
 \[
|\delta \uuu n^{(k)}|\leq 
C\uuu\epsilon\cdot
 \bigl(\frac{\rho+\epsilon}{\rho\uuu *\vvv \epsilon}\bigr)^n
 \quad\colon\quad 1\leq k\leq p
\]
\medskip


\noindent
{\bf{7.6 The polynomial $Q(z)$}}.
Lemma 7.5  and (iii) entail that
the sequence $\{A\uuu n^{(k)}\,\colon\, n=1,2,\ldots\}$
converges for every $k$ and  we set
\[
A\uuu *^{(k)}=\lim_{n\to\infty}\, A\uuu n^{(k)}\,
\]
 Notice   that Lemma 7.5 after summations of geometric series gives
a constant $C_1$ such that
\[
|A\uuu *^{(k)}\vvv A\uuu n^{(k)}|\leq C_1\cdot 
\bigl(\frac{\rho+\epsilon}{\rho\uuu *\vvv \epsilon}\bigr)^n\tag{7.6.i}
\]
hold for every $1\leq k\leq p$ and every $n$.

\noindent
Now we consider the sequence
\[
b\uuu n=
 c\uuu{n+p}+ A\uuu *^{(1)}\cdot c\uuu {n+p\vvv 1}+
\ldots A\uuu *^{(p)}\cdot c\uuu n\tag{7.6.ii}
 \]
Equation (*) applied to   $j=0$  gives
\[
b\uuu n= 
(A\uuu *^{(1)}\vvv A\uuu n ^{(1)})
\cdot c\uuu {n+p\vvv 1}+
\ldots +(A\uuu *^{(p)}\vvv A\uuu n^{(p)})\cdot c\uuu n\tag{7.6.iii}
\]

\medskip

\noindent
Next, we have already seen that $|c\uuu n|\leq C\cdot(\rho\vvv \epsilon)^{\vvv n}$
hold for some constant $C$ which
together with (7.6.i)  gives:
\medskip

\noindent
{\bf{7.7 Lemma.}}
\emph{For every $\epsilon>0$ there exists a constant $C$ such that}
\[
|b\uuu n|\leq C\cdot \bigl(\frac{1+\epsilon}{\rho\uuu *}\bigr)^n
\]
\medskip

\noindent
Finally, consider the polynomial
\[
Q(z)=  1+ A\uuu *^{(1)}\cdot z+
\ldots A\uuu *^{(p)}\cdot z^p
\]

\noindent
Set $g(z)= Q(z)f(z)$ which has a power series
$\sum\, d_\nu z^\nu$
where 
\[
b_n=
c_n\cdot   A_*^{(p)}+\ldots
c_{n+p-1}A_*^{(1)} +c_{n+p}=d_{n+p}
\]
\medskip


\noindent
Above $p$ is fixed so Lemma 7.7 and the trivial spectral radius formula 
show that
$g(z)$ is analytic in the disc $|z|<\rho_*$. This
proves that $f$ extends and the poles are contained in
the zeros of the polynomial $Q$ which occur  in the annulus
$\rho\leq |z|<\rho_*$.







 
 











\newpage



\centerline{\bf{10. An application to integral equations.}}



\bigskip


\noindent
Let $k(x,y)$ be a complex-valued continuous function on the unit square
$\{0\leq x,y\leq 1\}$. We do not assume that $k$ is symmetric,  i.e,
in general $k(x,y)\neq k(y,x)$.
 Let $f(x)$ be another  continuousfunction  on $[0,1]$.
Assume that the maximum norms of $k$ and $f$ both are $<1$.
By induction over $n$ starting with $f\uuu 0(x)= f(x)$
we get a sequence $\{f\uuu n\}$ where
\[
f\uuu n(x)=\int\uuu 0^1\, k(x,y)\cdot f\uuu{n\vvv 1}(y)\cdot dy
\quad \colon\quad n\geq 1
\]
The hypothesis entails that each $f\uuu n$ has maximum norm
$<1$ and hence there exists  a power series:
\[
u\uuu\lambda(x)= \sum\uuu{n=0}^\infty\, f\uuu n(x)\cdot \lambda^n
\] 
which converges for every  $|\lambda|<1$ and yields a continuous function
$u\uuu\lambda(x)$  on $[0,1]$.

\medskip

\noindent
{\bf{10.1 Theorem.}}
\emph{The function $\lambda\mapsto u\uuu\lambda(x)$ with values in the Banach space
$B=C^0[0,1]$ extends to a meromorphic $B$\vvv valued 
function in the whole
$\lambda$\vvv plane.}
\bigskip

\noindent
To prove this we ontridice the recursive Hankel determinants for
each $0\leq x\leq 1$:


\[
\mathcal D_n^{(p)}(x)=
\det
\begin{pmatrix}
f_{n+1}(x)
&f_{n+2}(x)
&\ldots&\ldots
& f_{n+p}(x)\\
f_{n+2}(x)
&f_{n+3}(x)
&\ldots&\ldots
& f_{n+p+1}(x)\\

\ldots &\ldots &\ldots&\ldots \\
\ldots &\ldots&\ldots&\ldots\\
f_{n+p}(x)
&f_{n+p+1}(x)
&\ldots&\ldots
& f_{n+2p-1}(x)\\
\end{pmatrix}
\]
\medskip


\noindent
{\bf{Proposition 10.2}} \emph{For every $p\geq 2$ and $0\leq x\leq 1$ one has
the inequality}

\[
\bigl |\,  \mathcal D\uuu n^{(p)}(x))\,\bigr|\leq 
(p\, !)^{\vvv n}\cdot \bigl( p^{\frac{p}{2}}) ^n\cdot \frac{p^p}{p\,!}
\]
\medskip

\noindent
{\bf{10.3 Conclusion.}}
The inequality above  entails that
\[ 
\limsup\uuu{n\to \infty}\, \bigl| \mathcal D\uuu n^{(p)}(x))\,\bigr |^{1/n}
\leq 
\frac{p^{p/2}}{p\,!}
\]
Next, Stirling's formula gives:
\[
\lim\uuu{p\to \infty}\bigl[\frac{p^{1/2}}{p\,!}\,\bigr]^{\vvv 1/p}=0
\]
Hence  Hadamard's theorem gives
Theorem 10.1



\bigskip

\centerline{\emph{Proof of Proposition 10.2}}
\bigskip


\noindent
The proof requires several steps. 
First, 
we get the sequence $\{k^{(m)}(x)\}$
which starts with $k=k^{(1)}$ and:
\[
k^{(m)}(x)= \int\uuu 0^1\, k^{(m\vvv 1)}(x,s)\ddot k(s)\cdot ds
\quad\colon\quad m\geq 2
\]
It is easily seen that
\[
f\uuu{n+m}(x)= \int\uuu 0^1\, k^{m)}(x,s)\cdot f\uuu n(s)\cdot ds
\]
hold for all pairs $m\geq 1$ and $n\geq 0$.
\medskip

\noindent
{\bf{Determinant formulas.}}
Let $\phi\uuu 1(x),\ldots,\phi\uuu p(x)$ and
 $\psi\uuu x),\ldots,\psi\uuu p(x)$
be a pair of $p$\vvv tuples of continuous functions on
$[0,1]$.
For each point $(x\uuu 1,\ldots,x\uuu p)$ in
$[0,1]^p$ we put

\[ 
D_{\phi_1,\ldots,\phi_p}(x\uuu 1,\ldots,x\uuu p)
=\text{det}
\begin{pmatrix}
\phi\uuu 1(x\uuu 1)&\cdots&\phi\uuu 1(x\uuu p)\\
\cdots &\cdots&\cdots \\
\cdots &\cdots&\cdots\\
\phi\uuu p(x\uuu 1)&\cdots &\phi\uuu 1(x\uuu p)\\
\end{pmatrix}\quad\colon\quad
\]
In the same way we define $D_{\psi_1,\ldots,\psi_p}(x\uuu 1,\ldots,x\uuu p)$.
Next, define the  $p\times p$\vvv matrix with elements

\[
a\uuu{jk}= \int\uuu 0^1\, \phi\uuu j(s)\cdot \psi\uuu k(s)\, ds
\]

\medskip

\noindent {\bf{Lemma.}} \emph{One has the equality}
\[
\text{det}(a\uuu{jk})=
\frac{1}{p\,!}\int\uuu {[0,1] ^p}\,
\Phi(s\uuu 1,\ldots,s\uuu p)\cdot 
\Psi(s\uuu 1,\ldots,s\uuu p)\cdot ds\uuu 1\cdots ds\uuu p
\]

\medskip

\noindent
{\bf{Exercise.}} Prove this result using standard formulas for
determinants.

\medskip

\noindent
Next, for each $0\leq x\leq 1$ and every pair $n,p$ of
positive integers we consider the $p\times p$-matrix

\[
\begin{pmatrix}
\int_0^1\, k(x,s)f_n(s)\,ds
&\int_0^1\, k^{(2)}(x,s)f_n(s)\,ds
&\ldots&\ldots
&\int_0^1\, k^{(p)}(x,s)f_n(s)\\

\int_0^1\, k^{(2)}(x,s)f_{n+1}(s)\,ds
&\int_0^1\, k^{(2)}(x,s)f_{n+1}(s)\,ds
&\ldots&\ldots
&\int_0^1\, k^{(2)}(x,s)f_{n+1}(s)\\
\ldots &\ldots &\ldots&\ldots \\
\ldots &\ldots&\ldots&\ldots\\
\int_0^1\, k^{(p)}(x,s)f_{n+p-1}(s)\,ds
&\int_0^1\, k^{(p)}(x,s)f_{n+p-1}(s)\,ds
&\ldots&\ldots
&\int_0^1\, k^{(p)}(x,s)f_{n+p-1}(s)\\
\end{pmatrix}\quad\colon\quad
\]
\medskip

\noindent 
We also get the two determinant functions
\[
\mathcal K^{(p)}(x,s_1,\ldots,s_p)=
\det
\begin{pmatrix}
k^{(1)}(x,s_1)
&k^{(1)}(x,s_2)
&\ldots&\ldots
& k^{(1)}(x,s_p)\\
k^{(2)}(x,s_1)
&k^{(2)}(x,s_2)
&\ldots&\ldots
& k^{(2)}(x,s_p)\\
\ldots &\ldots &\ldots&\ldots \\
\ldots &\ldots&\ldots&\ldots\\
k^{(p)}(x,s_1)
&k^{(p)}(x,s_2)
&\ldots&\ldots
& k^{(p)}(x,s_p)\\
\end{pmatrix}
\]

\bigskip

\[
\mathcal F_n^{(p)}(s_1,\ldots,s_p)=
\det
\begin{pmatrix}
f_n(s_1)
&f_n(s_2)
&\ldots&\ldots
& f_n(s_p)\\
f_{n+1}(s_1)
&f_{n+1}(s_2)
&\ldots&\ldots
& f_{n+1}(s_p)\\

\ldots &\ldots &\ldots&\ldots \\
\ldots &\ldots&\ldots&\ldots\\
f_{n+p-1}(s_1)
&f_{n+p-1}(s_2)
&\ldots&\ldots
& f_{n+p-1}(s_p)\\
\end{pmatrix}
\]

\medskip


\noindent{\bf{Lemma}}.
Let 
$\mathcal D_n^{(p)}(x)$
denote the determinant of the matrix (x). Then
one has the equation





\[
\mathcal D_n^{(p)}(x)=
\frac{1}{p !}\cdot\int_{[0,1]^p}\,
\mathcal K^{(p)}(x,s_1\ldots,s_p)\cdot \mathcal F^{(p)}_n(s_1,\ldots,s_p)
\, ds_1\cdots ds_p
\]


PROOF: Apply previous lemma ....


\bigskip

\noindent
Next, using (xx) we have the equality





\bigskip

\noindent
{\bf{Exercise.}}
Use the formulas above to conclude that
the requested intequality in Proposition 10.2 holds.


\newpage 


\centerline{\bf{Continued fractions and there limit discs.}}
\bigskip


\noindent
Cnosioder a  pair of sequence of real numbers
$\{a_p\}$ and  $\{b_p\}$ where each $b_p$ is positive.
For the $a$-sequence it can occur that some - for even all -$a_p=0$ and
no conditions are imposed upon their
signs. For each postivre integer $n$
we have the symmetric $n\times n$-matrix
\[
A_n=xxxx
\]
If $\mu$ is a non-real complex number
there exists a unique $n$-vector
$\phi_1(\mu),\ldots,\phi_n(\mu)$
siuch that 
\[ 
A_n(\Phi)= \mu\cdot \Phi+e_1
\]
It means that the $n$-vector satisfies the system of inear equations
\[
 xxxx
\]
Set
\[
g_n(\mu)=\phi_1(\mu)
\] 
Form the system above we see thst
\[
 g_n(\mu)=\frac{P_n(\mu)}{Q_n(\mu)}
\]
Where $P$ and $Q$ are polynomials without a common zero: bottom line 192 gives formula.
and here $Q_n$ has degree $n$, while that of $P_n$ is $n-1$.
By an induction over $n$ one easily verifes
the equations
\[
xxxx=xxxx
\]
Now we can regard the Lureiunt series expnasion at
$\mu=\infty$ for each rational $g$-function.
Since the degree of $Q_n(\mu)\cdot Q_{n+}(\mu)= 2n+1$
for each $n$, it follows from (xx) that
the first $2n$ terms in the Laurent series expansion of
$g_{n+1}$ are equal to those of $g_n$.
This gives an asympttic series
\[
-\sum\, \frac{c_n}{\mu^{n+1}}
\]
hos truncated series up to degree $2n$ conicides with that
of
$g_n(\mu)$.
Next, for each $n$ we find a postive measure $\rho_N$ on the real $t$-line such that
\[
g_n(\mu)= |int\, \frac{d\rho_n(t)}{t-\mu}
\]
and from the above
we get
\[
c_\nu=\int\, t^\nu\cdot d\rho_n(t)
\]
when $0\leq \nu\leq 2n-1$.
After this one can pass to subsequences and get solutions to the moment problem.












We get an associated quadratic form in an inifinite number of variables
$x=(x_1,x_2,\ldots)$ defined by
\[
J(x)= \sum\, a_px_p+
\]
Consider also the symmetric matrix
\[
W=xxx=xxx
\]

\noindent
{\bf{1. The $\phi$-vectors.}}
For each pair of complex numbers $\phi_1$ and 
$\mu$ there exists  a unique vector $\phi(\mu)= (\phi_1,\phi_2(\mu),\ldots$
such that
\[
W-\mu\cdot E)(\phi(\mu))=0\tag{1.1}
\]
In fact, (1.1) holds if and only if
\[
b_1\phi_2(\mu)= (a_1-\mu)\phi_1\quad\colon\quad 
b_p\phi_{p+1}(\mu)=(a_p-\mu)\phi_p(\mu)-b_{p-1}\phi_{p-1}(\mu) \colon\,p\geq 3
\]
Since $\{b_p\}$ are postive this determines the
solution uniquely and it depends on $\phi_1$ in a linear fashion. So we can write
\[
 \phi_n(\mu)= \psi_n(\mu)\cdot \phi_1\quad\colon n\geq 2
\]
\medskip

\noindent
{\bf{2. Exercise}}.
Show that when $\mu$ varies then
$\mu\mapsto  \psi_n(\mu)$ is a polynomial of degree $n-1$ for every $n\geq 2$.
Moreover, $\psi_1(\mu)=1$ and
\[
b_1\psi_2(\mu)= (a_1-\mu)\psi_1(\mu)\quad\colon\quad 
b_n\psi_{n+1}(\mu)=(a_n-\mu)\psi_n(\mu)-b_{n-1}\psi_{n-1}(\mu) \colon\,n\geq 3
\]
\medskip

\noindent
{\bf{3. Exercise}}.
Multiply each of the equations above
by the complex conjugates
$\overline{\psi_1}(\mu),\ldots,\overline{\psi_n}(\mu)\}$
and take the sum to show that
\[
0=\mathfrak{Im} (\mu)\cdot \sum_{p=1}^{p=n}\,
|\psi_1(\mu)|^2+b_n\cdot 
\mathfrak{Im}(\overline{\psi_n}(\mu)\cdot \psi_{n+1}(\mu))\tag{3.1}
\]

\noindent
Suppose that 
$\mathfrak{Im} (\mu)>0$ and show via
the Cauchy-Schwarz inequslity that this gives

\[
\frac{1}{b_n}\leq\frac{1}{2\cdot \mathfrak{Im}(\mu)}\cdot 
|\psi_n(\mu)|^2+ |\psi_{n+1}(\mu)|^2
\]
THiese inequalities give:



\medskip

\noindent
{\bf{4. Proposition.}}
\emph{If the series
$\sum\, \psi_n(\mu)|^2<\infty$ for some non-real $\mu$, then
$\sum\, \frac{1}{b_n}<\infty$.}
\medskip

\noindent
{\bf{5. A conclusion.}}
Propostion 4 shows that
the quadratic form $J(x)$ is of Case II in the sense of � xx, then
we must have
\[
\sum\, \frac{1}{b_n}<\infty
\]
So conversely, if we have a divergent series
\[
\sum\, \frac{1}{b_n}=+\infty
\]
then $J(x)$ is of Case I.





\bigskip



\noindent
{\bf{6. The kernel function $K(x,y)$}}.
It is defined as zero outside the positive quadrant in ${|bf{R}}^2$, and if
$n$ is a positive integer we have $K(x,y)=a_p$ in the square
$\{p-1\leq x,y\leq p\}$, while $K(x,y)= b_p$
in the two squares
\[
\{p-1\leq x,y\leq p\} \times\{p<y\leq p+1\}\,\colon\, 
\{p-1\leq x,y\leq p\} \times\{p<y\leq p+1\}
\]


\medskip

\noindent
Next, For each integer $n$ we denote by $K_n(x,y)$
the kernel which is equal to�$K$ when both $x$ and $y$ are $\leq n$ and is otherwise zero.
Let $f_*(x)$ be the function which is 1 if $0<x<1$ and zero when $x>1$.
Since $K_n$ is symmetric the integral equation 
\[ 
\int\, K_n(x,y)\cdot G(y)\,dy=\mu\cdot G(x)+f_*(x)\tag{6.1}
\]
has a unique solution for each non-real $\mu$ and the reader can check
that
the $G$-function is  
constant over integer interval and vanishes when $x>n$.

\medskip

\noindent
{\bf{Excercise.}}
Let $g_p$ denote  the constant value on the interval 
$(p-1,p)$  of the $G$-function above.
Show that one has   the
system of equations
\[
xxxxx page 133\tag{6.2}
\]
In particular $g_1$ is eqaual to the value of the  finite 
continued  fraction
of degree $n$ as explained in � xx.
This clarifies the relation between
expansions in continued  fractions and the quadratic form $J(x)$ above.
\medskip

\noindent
{\bf{7. Some rational functions.}}
From (6.2 )  it is clear that
$g_1$ as a function of  $\mu$ is a rational function of the form
\[
\frac{P_n(\mu)}{Q_n(\mu)}
\]
where the polynomials  have no common factor. Moreover, 
(xx) gives the system of equations
\[
xxxx page 193\tag{7.1}
\]
\medskip

\noindent
{\bf{Exercise.}}
Show that (7.1 ) above and 
the equations for the $\psi$-polynomials in (� xx ) give the equations:
\[
\psi_{n+1}(\mu)=\frac{Q_n(\mu)}{b_1\cdots b_n}\quad\colon\, n\geq 1\tag{7.2}
\]
\medskip

\noindent
{\bf{8. The $\Psi$-function.}}
For each $\mu$ we get  the function $x\mapsto \Psi(x;\mu)$
defined on $x\geq 0$ by
\[
\Psi(x;\mu)=\psi_p(\mu)\quad\colon p-1<x<p)
\]
If $n\geq 1$ we get the truncated function
$\Psi_n(x;\mu)$ which is equal to
$\Psi(x;\mu)$ when $x<n$ and zero if $x>n$.
\medskip

\noindent
{\bf{Exercise.}}
Let $n\geq 1$ and suppose that $\psi_{n+1}(\mu)=0$ for some 
complex number $\mu$.
Apply the equation (xx)
and show that this
entails that
$\Psi_n(x;\mu)$ solves
the  equation
\[
\int_0^n\, K_n(x,y)\cdot \Psi_n(x;\mu)\, dx=
\mu\cdot \Psi_n(x;\mu)\tag{8.1}
\]


\medskip

\noindent
{\bf{9. Reality of zeros.}}
Since the kernel function $K_n$ is symmetric 
we cannot hve a non-trivial solution in (8.1) above
unless 
$\mu$ is real. Hence we conclude that   the zeros of all
the
$\psi$-polynomials  are real.
\bigskip


\centerline {\bf{10. Some Cauchy transforms.}}
\medskip

\noindent
Consider the finite continued fraction in (x) for each pair $n$ and $\mu$.
Keeping $n$ fixed we obtain  a unique probability measure s
on the real $t$-line such that
\[
g_n^*(\mu)=\int\,\frac{d\rho_n(t)}{t-\mu}
\]
\medskip

\noindent
Now $\{g_n^*(\mu)\}$ is a normal family of
analytic functions
in the upper half-plane $\mathfrak{Im}(\mu)>0$.
Montel's theorem  entails that
we can extract convergent subsequences.
Every analytic function
$g^*(\mu)$ obtained  as a limit  in the Frechet space
$\mathcal O(U_+)$ from a subsequence
$\{g_{n_1}^*(\mu),g_{n_2}^*(\mu)\ldots\}$
is called a limit function of the infinite continued fraction
from (xx).
\medskip

\noindent
{\bf{Exercise.}}
Use 
weak-star convergence in the space of probability measures on the real $t$-line
to prove that  each  limit function $g^*$ above is
the Cauchy transform of a unique
probability measure
$\rho$, i.e. 
\[
g^*(\mu)=\int\,\frac{d\rho^*(t)}{t-\mu}
\]
Moreover, check that  the weak-star convergence
and the orthogonal equations in (xx) give
\[
\int\, \psi_n(t)\cdot \psi_m(t)\, d\rho^*(t)=0\quad\colon n\neq m
\]
\medskip


\centerline {\bf{On the range of  limit functions.}}
\medskip

\noindent

 
















 








\end{document}








 
 
 
 
 
 
 
 
 
 
 
 
 
 
 
 
