
\documentclass{amsart}
\usepackage[applemac]{inputenc}


\addtolength{\hoffset}{-12mm}
\addtolength{\textwidth}{22mm}
\addtolength{\voffset}{-10mm}
\addtolength{\textheight}{20mm}
\def\uuu{_}

\def\vvv{-}


\begin{document}


\noindent
Consider an analytic function $f(x,y)$
of the form
\[
 f(x,y)= y^e+q_{e-1}(x)y^{e-1}+\ldots+ q_1(x)y+q_0(x)
\]
where $e\geq 2$ and $\{q_\nu(x)\}$ are holographic functions in
a disc
$\{x|<r\}$ with $q_\nu(0)=0$ for every $\nu$.
By the fundamental theorem of algebra we
get a factorization
\[
f(x,y)= \prod\, (y-\alpha_\nu(x))
\]
The unordered $e$-tuple of roots
give rise to multiple-valued functions
in the punctured disc.
Under analytic continuation they are permuted
and when $f$ is irreducible they matching each other.
Follows that
there exists an analytic function $A(\zeta)$
and
with $x= \zeta^e$
the roots become
\[ 
\alpha_\nu(x)= A(e^{2\pi i\nu/e}\cdot \zeta^e)
\]
It follows that
if $A(\zeta)$ has some order $k$
one has
\[
|\alpha_\nu(x)|= c\cdot |\zeta|^k(1+O(|\zeta|))
\]
So the absolute values are almost equal.
Compare with the order of $q_0$ which is the product of the roots.
Get
\[
e\cdot k=ord(q_0)\cdot e
\] 
and hence
we find exact formula.
Complete expansion for $A$:
\[
c_k\cdot \zeta^k+c_{k+1}\cdot \zeta^{k+1}+\ldots
\]
Interest follows.
roots must be separated.
Sp $e$ is not a common factor to
coefficients.
An optimal case occurs if $k$ 
and $e$ are relatively prime for  then difference of roots
like $|\zeta|^k$.
and product
over all distinct  roots
is estimated.




\newpage






































\centerline
{\bf{On complete intersections of two polynomials.}}

\bigskip

\[
\text{Jan-Erik Bj�rk}
\]

\bigskip



\centerline
{\bf{1. Algebraic function fields}}


\medskip

\noindent
An algebraic function field $K$ over ${\bf{C}}$ is an abstract field which contains
${\bf{C}}$ as a subfield with  
 degree of trancendency equal to  one and 
generated by a finite number of elements
 $k_1,\ldots,k_m$. When this holds we pick
 an arbitrary element $\xi\in K\setminus {\bf{C}}$,
Since the complex filed is algebraically closed this yields
a trancendental elnebnt and by the hypothesis the finite set of generators are
algebraic over the field ${\bf{C}}(\xi)$ of rational functions in the
$\xi$-variable.
each  $k\in K$ satisfies an equation
\[
k^m+q_{m-1}(\xi)k^{m-1}+\ldots+ q_1(\xi)k+q_0(\xi)=0\tag{*}
\] 
where the positive integer $m$ can be chosen to be minimal.
and $\{q_j(\xi))\}$ are  ratiuonal functions. One refers to (*) as the 
minimal funvion satisfies by $k$.
The field over
${\bf{C}}$ generated by $\xi$ and $k$ 
can then be identified wirth a vector space of dimension $m$ over the
field ${\bf{C}}(\xi)$. One can do this for each
of the orighimal generators and conclude that
$K$ is a finite dimensional vector space
over ${\bf{C}}(\xi)$.
\medskip

\noindent
{\bf{Primitive elements.}}
With $\xi$ as above we set
\[ 
m=\dim_{{\bf{C}}(\xi)}\, K
\]
An elment $\eta\in K$ is primitive with respect tothe chosen
trancendental elmenet if its minimal equation in (xx) has degree $m$
whuch then entais that
$K$ is generated by the to elements $|xi$ and $\eta$.
It turns out that primitive elements always exist
and they are even found in a "generic way".





 
Elementary algebra teaches that when
 $K$ is such a field and 
 $\xi\in K\setminus {\bf{C}}$,
then  there exists   $\eta\in K$
 such that
 $K$ is generated by $\eta$ and the field ${\bf{C}}(\xi)$ whose
 elements are rational functions in $\xi$ with complex coefficients.
 Moreover,  $\eta$ satisfies an equation
\[
 \eta^m+r_{m-1}(\xi)\eta^{m-1}+\ldots+ r_1(\xi)\eta+r_0(\xi)=0\tag{*}
 \]
 where $\{r_j(\xi)\}$ belong to ${\bf{C}}(\xi)$
 and each $k\in K$ can be written   as
 \[
  k=q_{m-1}(\xi)\eta^{m-1}+\ldots+ q_1(\xi)\eta+q_0(\xi)=0\tag{1}
 \] 
 where $\{q_j(\xi)\}$ is a unique  $m$-tuple in
 ${\bf{C}}(\xi)$. So $K$ is a vector space of dimension $m$
 over
 the subfield
 ${\bf{C}}(\xi)$.
 Moreover, 
 the polynomial 
 in the indeterminate variable $t$ given by
 \[
P(t,\xi)=t^m+r_{m-1}(\xi)t^{m-1}+\ldots+ r_1(\xi)t+r_0(\xi)\tag{2}
 \]
 is irreducible in the polynomial ring
 of   $t$ over the field
 ${\bf{C}}(\xi)$.
This is  expressed by saying that if
$K_*={\bf{C}}(z)$ is the standard field of rational functions in one
variable, then every  algebraic function field
is isomorphic
to a field
\[ 
\frac{K_*[t]}{(P)}
\] 
where $(P)$ denotes the principal ideal generated by an irreducible polynomial in
$K_*[t]$. Here one has  used  that 
$K_*[t]$ is an euclidian ring which 
implies that this $K_*$-algebra is a  unique factorisation  domain
and therefore   gives    a precise meaning in order that   a polynomial $P(t)$ is   irreducible.
\medskip

\noindent
{\bf{Remark.}} When an  algebraic function field $K$ is given
we can choose a  trancendental element $\xi$ in many ways.
Once $\xi$ is chosen,
${\bf{C}}(\xi)$ appears as a subfield  and $K$ becomes   a 
finite dimensional vector
pace over ${\bf{C}}(\xi)$.
More precisely one has the equality
\[
\dim_{{\bf{C}}(\xi)}\, (K)=
\text{deg}(\eta)
\] 
where $\eta$ is a primitive element whose minimal equation from (*) has degree $m$.
Let us  remark that when  a trancendental elment $\xi$ is chosen in $K$, 
then there is a whole family of primitive $\eta$-generators. More precisely, 
every
$\eta\in K$ whose minimal equation (*) has 
a degree which equals the dimension of $K$ as a vector space  over
${{\bf{C}}(\xi)}$ yields a primitive element 
in $K$ with respect to the chosen trancendental $\xi$-element.
Finally, two  algebraic function fields $K_1$ and $K_2$ 
are isomorphic if there exists a ${\bf{C}}$-algebra isomorphism
 between them.
With this kept in mind one has the following result predicted by Riemann and in 
the general version established by Weyl. 


 
 

\medskip

\noindent
{\bf{0.1  Theorem.}}
\emph{There is a 1-1 correspondence between the family of algebraic function fields in one variable and
the family of compact Riemann surfaces.}
\medskip

\noindent
{\bf{Remark about the proof.}}
A pair of 
Riemann surfaces are identified  when they are bi-holomorphic.
Several  steps are needed to prove Therorem 0.1.
First Weyl's theorem shows that  if $X$ is a compact Riemann surface,
then the field of meromorphic functions 
$\mathfrak{M}(X)$ is an algebraic function field.
Next, let $X_1,X_2$ be a pair of
biholomorphic Riemann surfaces and 
$\rho\colon X_1\to X_2$  a  biholomorphic mapping.
Then it follows that $\mathfrak{M}(X_1)\simeq \mathfrak{M}(X_2)$. In fact,
$\rho$ yields  an  algebra isomorphism between these two algebraic function fields
which sends
$f\in \mathfrak{M}(X_2)$ to the meromorphic function on $X_1$ defined by
\[ 
f^*(x_1)= f(\rho(x_1)\quad\colon\quad x_1\in X_1
\]
There remains to show
that if $X_1$ and $X_2$ are two Riemann surfaces whose 
algebraic function fields 
$\mathfrak{M}(X_1)$ and $\mathfrak{M}(X_2)$ are isomorphic,  then
the Riemann surfaces $X_1$ and $X_2$ are
biholomorphic.
Moreover, we must show 
that
for every algebraic function field $K$ is isomorphic to 
$\mathfrak{M}(X)$
for a compact Riemann surface
$X$.
\medskip

\noindent
To prove these  results one  considers  an arbitrary 
algebraic function field $K$.
A valuation map on $K$ is
an injective algebra homomorphism
\[
\rho\colon\ K\to {\bf{C}}\{ t\}[t^{-1}]\tag{1}
\]
where the right hand side is the  field of germs of meromorphic functions
at the origin with $t$ regarded as a complex variable. 
In addition one requires that $\rho$�is non-degenerated in the sense that there exists some 
$k\in K$ such that
\[
\rho(k)= t+\sum_{\nu=2}^\infty\, c_\nu \cdot t^\nu\tag{2}
\]
i.e. this germ is holomorphic and
its $t$-derivative is non-zero when $t=0$.
The  $\rho$-map defines a valuation on $K$ as follows:
Each 
non-zero element $k$ gives the meromorphic germ  $\rho(k)$ and we find
the unique integer $\rho_*(k)$ such that
\[
\rho(k)= t^{\rho_*(k)}\cdot \phi(t)
\] 
where $\phi(t)$�is a unit in the local ring
${\bf{C}}\{ t\}$, i.e its constant term is $\neq 0$.
Let $\mathcal V(K)$ denote the family of all valuations maps on $K$.
It turns out that $\mathcal V(K)$ corresponds to 
points in a compact Riemann surface $X$ and that
$K\simeq \mathfrak{M}(X)$
where
each point $x\in X$ yields
a valuation since
there exists  a local chart around $X$ with a coordinate $t$
so that
every  $f\in\mathfrak{M}$ has a series expansion at $x$ 
expressed by a an element in 
$\rho_f(t)\in{\bf{C}}\{ t\}[t^{-1}]$. By analyticity the map $f\to \rho_f(t)$ is injective which
clarifies the1-1  correspondence between points on $X$ and valuation maps on
$\mathfrak{M}(X)$.

\medskip

\noindent
So the main burden is to
prove  the  existence   of 
an ample family of valuations on a given  algebraic function field $K$ and 
explain how these 
valuations fabricate points on a Riemann surface.
This is done in the next sections
using  constrcutions by Puiseux from 1852, and put in a global form
by
Riemann a few years after.



\bigskip


\centerline{\bf{1.1 Algebraic curves.}}

\medskip

\noindent
Recall  that the polynomial ring
${\bf{C}}[x,y]$ in two variabes is a unique factorisation domain.
Let $n\geq 2$
and  consider an irreducible polynomial
\[ 
P(x,y)= y^n+q_1(x)y^{n\vvv 1}+\ldots+q_{n-1}(x)y+q_n(x)\tag{1}
\]
To $P$ corresponds the algebraic function field $K$ whose elements are 
\[ 
k=r_0(x)+r_1(x)y+\ldots+ r_{n-1}(x) y^{n-1}
\quad\text{where} \quad r_0,\ldots,r_{n-1}\in 
{\bf{C}}(x)\tag{2}
\]

\noindent
Next, we get  the algebraic curve $S$ in ${\bf{C}}^2$ defined by
the zero-set $\{P=0\}$. 
It has a closure in
the projective space ${\bf{P}}^2$
whose homogeneous coordinates are $(\zeta_0,\zeta_1,\zeta_2)$ and
points in the $(x,y)$-space are represented by $(1,x,y)$.
The hyperplane at infinity is
$\{\zeta_0=0\}$
and points $(x,y)\in S$
converge to this hyperplane
when $|x|+|y|\to +\infty$ which yields the closure $\bar S$ in
the compact manifold
${\bf{P}}^2$.
The boundary $\partial S=\bar S\setminus S$
is a finite set of at most 
$n$ points. 
\medskip

\noindent
{\bf{1.1.1  Example.}}
Suppose that $\deg q_\nu\leq \nu$
hold for the $q$-polynomials  in (1) and 
for  each $1\leq \nu\leq n$
we denote by $c_\nu$  the coefficient of $x^\nu$ in $q_\nu(x)$.
Now
\[ 
P^*(x,y) = y^n+c_1xy^{n-1}+\ldots +c_{n-1}yx^{n-1}+ c_nx^n
\]
is a homogeneous polynomial and
the fundamental theorem of algebra entails that
\[ 
P^*(x,y)=\prod_{k=1}^{k=m}\, (y-\beta_k x)^{e_k}
\] 
where $\{\beta_k\}$ are distinct complex numbers and
$e_1+\ldots+e_m=n$.
\medskip

\noindent
{\bf{1.1.2 Exercise.}}
Show that 
$\partial S$ consists of the $m$-tuple points 
$(0,1,\beta_1),\ldots,
(0,1,\beta_m)$.
The case where $\text{deg}(q_\nu)<n$ for each $\nu$ is not excluded.
Here   
$P^*(y)=y^n$ and $\partial S$ is reduced to the single point
$(0,1,0)$. The reason is that the conditions on the $q$-polynomials give a number 
$0<a<1$ and a constant $C$ such that
\[
|y|\leq C(1+|x|)^a
\]
for all points on $\{P=0\}$. Then
$(1,x,y)=(\frac{1}{x},1,\frac{y}{x})$
can only approach $(0,1,0)$ at infinity.


\medskip

\noindent
{\bf{1.1.3  Regular points on $S$}}.
Let us for a while restrict the attention to the affine curve $S$
and 
consider the polynomial 
\[ 
P'_y(x,y)= ny^{n-1}+(n-1)q_1(x) y^{n-2}+\ldots+ q_{n-1}(x)
\]
\medskip

\noindent
{\bf{1.1.4 The
discriminant polynomial.}}
By assumption $P$ is irreducible in
the polynomial ring
$K[y]$ in the single variable $y$ where
$k$ denotes the field ${\bf{C}}(x)$.
Euclidian divisions give
a unique pair $A(y),B(y)$ in $K[y]$ such that
\[ 
A(y)\cdot P(x,y)+ B(y)\cdot P'_y(x,y)=1
\]
where the degree of the $y$-polynomial $A$ is at most $n-1$ and that of
$B$ at most $(n-2)$.
Taking a common factor for the denominators in
the $K$-coefficients of these two $y$-polynomials gives
a unique monic polynomial $\delta(x)$
in ${\bf{C}}[x]$ such that
\[
A_*(x,y)\dot P(x,y)+ B_*(x,y)\cdot P'_y(x,y)=\delta(x)\tag{i}
\]
where $A_*$ and $B_*$ now are polynomials in $x$ and $y$.
For example, one has
\[ 
A_*(x,y)= a_{n-2}(x)y^{n-2}+\ldots a_1(x(y+a_0(x)
\]
where the $x$-polynomials $\{a_k(x)\}$ have no common
factor.
We refer to $\delta(x)$ as the discriminant polynomial of $P$.
\medskip

\noindent{\bf{1.1.5 Root functions.}}
For each fixed $n$ the fundamental theorem of algebra
yields an $n$-tuple of roots to the equation $P(y,x)=0$ and we can write
\[
P(y,x)= \prod_{k=1}^{k=n}\, (y-\alpha_k(x))\tag{*}
\]
From (i) in (1.1.4)  we see that
the roots are all simple if and only if
$\delta(x)\neq 0$.
The zero-set $\{\delta(x)=0\}$ is called the discriminant locus.
Each point $z_0$ in the open and connected set
${\bf{C}}\setminus \delta^{-1}(0)$ consists of 
an  unordered $n$-tuple of simple roots.
As explained in � XX they give rise to germs of analytic functions of
the complex variable $z$ and  extend to multi-valued analytic functions in
${\bf{C}}\setminus \delta^{-1}(0)$.
By analyticity each new local branch is again a root.
For example, start at some point $z_0\in{\bf{C}}\setminus \delta^{-1}(0)$
and pick one root
$\alpha_1(z_0)$ which to begin with  gives an analytic function
$\alpha_1(z)$ in a small open disc centered at $z_0$.
It is now extended in the sense of Weierstrass
and the multi-valued function produces a finite set of local branches
at $z_0$.
The fact that $P$ from the start is irreducible entails that
the local branches under all possible the analytic continuations of
$\alpha_1$
along closed curves in
${\bf{C}}\setminus \delta^{-1}(0)$
which start and finish at $z_0$
produce local branches of all�the roots at $z_0$.
The conclusion is that
the set
\[
 S_*= S\setminus \delta^{-1}(0)
\] 
is connected and the projection $\pi(x,y)=x$
restricts to an $n$-sheeted covering map from
$S_*$ onto ${\bf{C}}\setminus \delta^{-1}(0)$.
Moeover, since the root functions are analytic,
$S_*$ appears as a 1-dimensional complex submanifold of
${\bf{C}}^2\setminus \delta^{-1}(0)$, and by
continuity
of the roots which appear in (*), the closure of $S_*$ taken in
${\bf{C}}^2$ is equal to $S$.
\medskip

\noindent
{\bf{Remark.}}
The results  above was
the starting point when  Riemann  constructed
Riemann surfaces,  and later  Weierstrass extended the construction
to polynomials in $y$ which may depend upon
several $x$-variables. In this case the discrimant locus is
a hypersurface in a multi-dimensional complex vector space
so
the properties of the algebraic hypersurface
$P^{-1}(0)$ when
$P= P(x_1,\ldots ,x_n,y)$ is an irreducible polynomial of $n+1$
variables with $n\geq 2$ is more involved and will
not be treated here.
Let us only mention that Riemann's 
constructions of  curves were extended by
Zariski to the case of surfaces, i.e. when $n=2$.
For arbitrary $n\geq 3$ a major result is
the existence of desingularisation of 
algebraic hypersurfaces in every dimension which 
was established by Hironaka in  a famous work from
1962 which settled a conjecture posed by Zariski in lectures at Tokyo in 1954.



\newpage


\centerline{\bf{2. Construction of local charts.}}
\medskip

\noindent
Let  $p=(x_*,y_*)$ be a point in $S$.
Consider the local ring 
$\mathcal O={\bf{C}}\{x-x_*\}$
of germs of analytic functions in the complex $x$-variable at $x_*$.
Now $P(x,y)$ is an element in the polynomial ring
$\mathcal O[y]$ which has
a unique factorisation
\[
P(y,x)= q_*(x,y)\cdot \prod_{k=1}^{k=r}\,\phi_k(x,y)\tag{2.1}
\]
where $q_*(x_*,y_*)\neq 0$
while $\{\phi_k\}$ are 
irreducible Weierstrass polynomials in $y$ with coefficients 
$\mathcal O$.
So here each  $\phi_k$ is of the form
\[
\phi_k(y,x)= y^{e_k}+\rho_{1,k}(x)\cdot y^{e_k-1}+\ldots+
\rho_{e_k,k}(x)\tag{2.2}
\]
where the $\rho$-functions belong to $\mathcal O$ and
vanish at
$x=x_*$, and
in a neighborhood of $p=(x_*,y_*)$  the curve
$S$ is defined by the union of
the zero sets $\{\phi_\nu=0\}$.
\medskip


\noindent
{\bf{2.3 Puiseux charts.}}
Fix one $\phi$-polynomial say $\phi_1$.
In a small punctured disc centered at $x_*$ the $y$-polynomial $\phi_1(x,y)$
has $e_1$ many simple zeros
which occur among roots of $P(x,y)$.
As explained in Chapter 4 from my notes in analytic function theory,
we can introduce a new complex variable $\zeta$
and find
an analytic function $A_1(\zeta)$
in a disc of some radius $r_1>0$
centered at $\zeta=0$ such that
\[
q_1(x_*+\zeta^{e_1}, y_*+A_1(\zeta)=0\quad\colon\, |\zeta|<r_1\tag{2.3.1}
\]
Moreover, since $q_1$ was irreducible the Taylor series
\[
A_1(\zeta)= \sum_{\nu=1}^\infty  a_\nu\zeta^\nu
\]
is such that
the principal ideal in ${\bf{Z}}$
generated by those integers for which $a_\nu\neq 0$
does not contain any prime divisor of $e_1$.
This entails that the map
\[
\zeta\to  (x_*+\zeta^{e_1},y_*+A_1(\zeta))\tag{2.3.2}
\] 
is bijective,  i.e. the open $\zeta$-disc can be identified with
a subset of the given affine curve $S$
where $\zeta=0$ is mapped to $p=(x_*,y_*)$.
In this subset of $S$ we can wirte
\[ 
x=x_*+\zeta^{e_1}\quad\colon\, y=y_*+A_1(\zeta)\tag{2.3.3}
\]
This  means that one can take  $\zeta$ as a local coordinate
and
the image of the $\zeta$-disc constitues a chart in the Riemann surface
$X$  attached to the curve $S$.
One refers to  a Puiseux chart. They can be    constructed for
each $\phi$-function which appears in (2.1).
\medskip

\noindent
{\bf{Remark.}}
Conversely, let $A(\zeta)$ be an
analytic germ of  the $\zeta$-variable with a Taylor series
\[
A(\zeta)= \sum_{\nu=1}^\infty  a_\nu\zeta^\nu
\]
which converges in a disc $\{\zeta|<\delta\}$.
Let $e\geq 2$ be an integer and assume that
then integers $\nu$ for which $a_\nu\neq 0$
have no prime divisor in common with
$e$.
Set $x=\zeta^e$ and consider the map
\[ 
\zeta\mapsto (\zeta^e,A(\zeta))
\]
It sends $\zeta=0$ to the origin in
the $(x,y)$-space.
In a small puntured disc
centered at $\zeta=0$.
\medskip

\noindent
{\bf{Exercise.}}
Show that the map above is injectivin a sufficently small 
punctured disc $\{0<|\delta|<\delta_*\}$.
Take as an example $e=3$
and consider
the case 
\[
A(\zeta)= \zeta^m(1+c\zeta^k)
\]
where $c\neq 0$ is a constant, $m$ and $k$ are both $\geq 2$
and
at least one of the two integers $m$ and  $k$ do not have 3 as a prime factor.

\medskip

\noindent
{\bf{Warning.}}
If $A(\zeta)$ as above is an arbitraty
holomorphic germ it is not
always true that it comes from 
an equation in (2.3.1). Following the original studies by Heine around
1850, one
says that the germ
$A(\zeta)$ is of algebraic type if
there exist polynomials $\{q_\nu(zeta)\}$ such that
\[
q_m(\zeta)A(\zeta)^m+\ldots
q_1(\zeta)A(\zeta)+
q_0(\zeta)=0
\]
holds in a small disc.
The questin arises if one
can
find conditions on the given
Taylor coefficients $\{c_n\}$
in order that such an equation exists.


















\newpage


\centerline {\bf{2.4 Riemann's construction.}}
\medskip

\noindent
To obtain  the  Riemann surface $X$ associated to the curve
$S$
one must  separate the Puiseux charts above.
Riemann regarded  these charts as disjoint.
So if $r\geq 2$ then the Riemann surface
$X$ contains $r$ \emph{distinct}  points above
$(x_*,y_*)$.
In this way one gets a complex 1-dimensional manifold $X$
and a map $\rho\colon X\to S$
which is bijective except for those points $p=(x_*,y_*)$ in
$S$ where
more than one irreducible $\phi$-function appears in (2.1)
and local complex analytic cooordinates are found via
the Puiseux series expansions above.
\medskip


\noindent
{\bf{2.5  Example.}}
Consider the irreducible polynomial
\[
P(x,y)= y^4-x^2(x+1)
\]
At the point $p=(0,0)$ on t
$S=P^{-1}(0)$ we get a factorisation
\[
P=\phi_1\cdot \phi_2
\] 
where
we choose a local branch of $\sqrt{1+x}$ so that
\[ 
\phi_1(x,y)= y^2-x\cdot \sqrt{1+x}\quad\colon
\phi_2(x,y)= y^2+x\cdot \sqrt{1+x}
\]
This gives  two distinct Puiseux-Riemann charts. So the map
$\rho\colon X\to S$ has an inverse fiber above
the origin which contains two points.
In our special case we notice that
$y$ serves as a local coordinate in each of these charts. Passing to
the Riemann surfsce it means that $y$ has a simple zero in
each of the two Puiseux-Riemann charts. In addition one
finds that $y$ has a simple zero above  $x=1$. 

\medskip



\noindent
{\bf{2.6   The passage to infinity.}}
There remains to construct charts around points which belong to 
$\partial S$. 
Consider the case when the irreducible polynomial $P(x,y)$ has the form
\[
P(x,y)= y^m+q_1(x)y^{m-1}+ \ldots+q_{m-1}(x)y+q_m(x)
\]
where $m\geq 2$ and
\[
\deg q_\nu\leq \nu
\] 
hold for every $\nu$.
It is clear that there is a constant $C$ such that
the absolute  values of the roots satisfy
\[
|\alpha(x)|\leq C(1+|x|)
\]
Next, each $q_\nu$ is $c_\nu x^\nu+\rho_\nu(x)$ where
$\rho_\nu$ has degree $<\nu$ and we obtain the homogeneous polynomial
\[ 
P^*(x,y)=y^m+c_{m-1} xy^{m-1}+ \ldots+c_1x^{m-1}y+c_0
\]
which by  the fundamental  theorem of algebra  can be written as
\[
P^*(x,y)= \prod\, (y-a_\nu x)
\]
where $\{a_\nu\}$ is an $m$-tuple of complex numbers.
Suppose for example that
$a_1\neq 0$. Now one expects that if
$|x|$ is large, then
there exists a root $\alpha_1(x)$ which is asymptotically close to
$a_1x$ in the sense that
\[
|\alpha_1(x)-a_1x|\leq C|x|^\gamma
\]
for some $\gamma<1$ and a constant $C$.
This is indeed true and to establish the result
one
perform a change of variables.
Namely,  ${\bf{C}}^2$ is identified with an open subset of
${\bf{P}}^2$  whose points are
$(1,x,y)$ and when $x\neq 0$ they can be written as
\[
(\frac{1}{x},1,\frac{y}{x})
\]
With $\xi=\frac{1}{x}$ and
$\eta=\frac{y}{x}$ we have coordinates in
${\bf{P}}^2$  outside the set of points where $x=0$.
Now
\[
x^{-m}P(y,x)=
\eta^m+\rho_1(\xi)\eta^{m-1}+\ldots+\rho_{m-1}(\xi)\eta+\rho_m(\xi)
\]
where
\[
\rho_\nu(\xi)=\xi^\nu\cdot q_\nu(1/\xi)
\]
are polynomials in $\xi$.
Here we seek $\eta$-rooots while $\xi\to 0$.
Suppose for example that $a_1$ above is 
such that $a_\nu\neq a_1$ for the remaining $a$-numbers.
This corresponds to
the condition that $a-1$ is a simp,e root of the $\eta$-polynomial
\[
q_*(\eta)=
\eta^m+\rho_1(0)\eta^{m-1}+\ldots+\rho_{m-1}(0)\eta+\rho_m(0)
\]
With $|eta=a_1+s$
one is led to consider the function
\[
\phi(\xi,s)=(a_1+s)^m+\rho_1(\xi)(a_1+s)^{m-1}+\ldots+\rho_{m-1}(\xi)(a_1+s)+\rho_m(\xi)
\]
which now has a simple zero $s=0$ when $\xi=0$.
it follows that when $|xi$ is small then we find a unique
simple root $s=s(\xi)$ where $s(\xi)$ is close to
zero and by basic analytic function theory we get a
series expansion
\[
s(\xi) =c_1\xi+c_2\xi^2+\ldots
\]
It means that we get points in $S$ defined for $\xi\neq 0$ by
\[
(\xi,1,a_1+s(\xi))= (1,\xi^{-1},a_1/\xi+ s(\xi)/\xi)
\]
Retruring t the $x$-coordinates it means that
we recover points of the form
\[
(x,a_1x+xs_1(1/x))
\]
Here
\[
\lim_{x\to \infty}\, xs_1(1/x))=c_1
\]
So the asymptotic root $\alpha_1(x)$ deviattes from $a_1x$ by
a fixed constant as $|x|\to \infty$, i.e. we can even take $\gamma=0$ in (xx).
\medskip


\noindent
{\bf{The case of multie roots.}}
Suppose that $a-1$ is a double root of $P^*$.
In this case $s\mapsto \phi(0,s)$ has a double root at $s=0$
and this time
the pair of roots which arise for small non-zero $|xi$ have absolute va�lues
\[
|s(\xi)|\leq C\cdot |\xi|^{1/2}
\]
and returing to the affine $(x,y)$-coordinates
we find a constant $c$ such that
\[
|x\cdot s(1/x)|\leq C\cdot |x|^{1/2}
\]
and this time we can take $\gamma=1/2$ in (xx).
The case of a multple zero of $P^*$ of arbitrary order is treated in a similar fashion.
Let us remark that one always can take
\[
\gamma=1/m
\]
to achieve asymptotic deviations from the "true roots"
as $|x|\to  \infty$ while one moves along the lines
$y=a_\nu x$.




\bigskip


\centerline{\bf{Exampes.}}

\bigskip


\noindent
Consider the polynomial 
\[ 
P(x,y)= y^6-x^3-1
\]
When �$|x|$ is large we have $|y|\simeq |x|^{1/2}$
which means that the $x$-coordinate tends faster to
infinity than the $y$-coordinate. Here  $\partial S$ 
is reduced to the single point $p^*=(0,1,0)$.
In ${\bf{P}}^2$ we have local coorfionstes
$(\zeta,\eta)$ around $p^*$ which corresponds to points
$(\zeta,\eta)$. When $\zeta\neq 0$ 
we are outside the hyplerplane at infinity and have
\[
 x=\frac{1}{\zeta}\quad\colon\, y=\frac{\eta}{\zeta}
\]
The equation $P=0$ means that
\[
\zeta^{-3}=\frac{\eta^6}{\zeta^6}-1\implies
\zeta^3+\zeta^6=\eta^6
\]
\medskip


\noindent
{\bf{Exercise.}}
Show that  this gives   three Puiseux-Riemann charts
and in each chart
we can take $\eta$ as a local coordinate and that $x$ regarded as a meromrphic
function on the compact Riemann surface
has a double pole  in each of these charts.
At the same time we notice that in the finite affine part
the $x$-fuynction has six simple zeros which
appear when
$y$ solves the equation $y^6=1$.
So the number of poles counted with multiplicity is equal to the number of
zeros of $x$ as it should be.
Show by similar ca.cukations that $y$ has simple poles on the three
Puiseux charts. In the affine part
$y$ has three simple zeros
where the $x$-coordiantes are determined by the equation
$x^3=1$.
Let us then consider the function
\[
\phi=\frac{y^2}{x}
\]
From the above it is holomorphic and $\neq 0$ in the
Puiseux charts.
Passing to the affine part we notice that
$\phi$ gets sim+le poles when $x=0$ which occurs for
6 many $y$-values, i..e the number of poles is six.
Zeros occur when $x$ is a third  root of unity. At these points
$y$ serves as a local coordinate
so that
$\phi$ has three many double zeros.
\medskip

\noindent
{\bf{The 1-form
$\omega=\frac{dy}{x}$.}}
Since $x$ has a double pole and $y$ a simple pole in the charts around
$p^*$ we see that $\omega$ is holomorphic 
in these charts and also $\neq 0$.
Next, in the finite part $S$ we notice that
$dy=0$ can only occur when $x=0$ 
an this occurs at six points $\{j_\nu,0)\}$
where
$j_\nu^6 =1$. At each of these points$x$ is a local coordinste
and
$y$ has a zero of order three. Hence  the 1-form
$dy$ has a zero of order teo.
We conclude that $\omega$ has six simple zeros and
the Hurwitz-Riemann formula implies that $X$ has   genus four.
\medskip

\noindent {\bf{2.6.1 The curve
$y^3=x(x-1)^2$}}.
Here $|y|\simeq |x|$ and this time
$\partial S$ contains three points:
\[ 
p^*_k=(0,1,e^{2\pi i k/3})\quad\colon\, k=0,1,2
\]
We leave it to the reader to verify that
around each $p_k^*$ we get a single Puiseux-Riemann chart where both $x$ and $y$
have simple poles.
Next, in the finite part
$x$ has a triple zero at $(0,0)$ where $y$ serves as a local coorfinste.
At $(1,0)$ we have a cusp-like sungularity  where
the Pusiuex-Riemann chart has a local coordinate $\zeta$ with
\[ 
x=1+\zeta^3
\quad\colon \quad y= A(\zeta)
\] 
and  $A(\zeta)$ has a zero of order two.
Let us then consider the 1-form
\[ 
\omega=\frac{dx}{y^2}
\]
From the above it is holomorhic and $\neq 0$ at the $p^*$-points.
At $(0,0)$
$x$ has a triple zero so $dx$ has a double zero and since
$y$ has a simple zero we conclude that
$\omega$ is holomorphic and $\neq 0$ at $(0,0)$
At $(1,0)$
the reader may verify that
$\omega$ has a double pole.
The result is that the divisor $D$ for which $\mathcal O_D\cdot \omega=\Omega$
has degree
$-2$ and it follows from the Riemann-Hurwitz formula that the genus of $X$ is zero.
Set
\[
 f=\frac{x-1}{y}
\] 
Then $f$ is a meromorphic function and
from the above
we see that it is holomorphic and $\neq 0$ at the $p^*$-points
while it has a simple  pole at $(0,0)$ and a simple zero at $(1,0)$.
Therefore $\mathfrak{M}(X)$ is reduced to the field
${\bf{C}}(f)$ which means that both $x$ and $y$ can be expressed in this field.
That this holds can of course be verified directly.
The reader may for example show that
\[
 x= \frac{1}{1-f^3}
\]
\medskip

\noindent
{\bf{Remark.}}
The example above  illustrates
that even if one may be "lucky" to discover the existence of a
function like $f$, it is more systematic to proceed with a construction of charts in $X$ and
eventually discover $f$
via the positions of poles and zeros of $y$ and $x-1$.


\newpage





\centerline{\bf{3. Intersection numbers.}}
\bigskip


\noindent
A pair of different irreducible polynomials $P(x,y)$ and $Q(x,y)$  
produce, after taking the  projective closure of their affine zero-sets in ${\bf{C}}^2$,
a pair of projective curves $\{P=0\}$ and $\{Q=0\}$ in
${\bf{P}}^2$.
We are going to assign 
an intersection number which  takes into the account
eventual multiplicities.
A "nice case" occurs if the
points of intersection
appear in the regular parts of the two curves
and  the gradient vectors of $P$ and $Q$ are linearly
independent at every such point. Then one refers to
a simple transversal intersection and 
the intersection number is 
the number of  these transversal intersection points.
In general a  procedure to
assign an integer to the pair $P$ and $Q$ is the following:
Let $X$ be the Riemann surface associated to the projective curve defined
by $P$. Now $Q$ is a meromorphic function on $X$ and
we can  count its number of zeros with multiplicites which
yields an integer denoted by
\[
{\bf{i}}(P;Q)
\]
Reversing  the role, we start from  the Riemann surface $Y$ associated with
$\{Q=0\}$
and count the number of zeros for the meromorphic 
$P$-function on $Y$, which gives the  integer
${\bf{i}}(Q;P)$. It turns out that one has  the equality
\[
{\bf{i}}(P;Q)={\bf{i}}(P;Q)\tag{*}
\]


\noindent
The 
proof of (*)
using
Jacobi's residue for the pair $P$ and $Q$ is given
in � 4.
For the moment we admit (*).
Keeping $P$ fixed, the integer
${\bf{i}}(P;Q)$ is the degree of the meromorphic function $Q$ on $X$.
This entails that
the intersection number is unchanged
when $Q$ is replaced by $Q-\beta$
for a complex number as long as $Q-\beta$ is not reduced to
a constant function on $X$.
A similar invariance hold when we repace $P$ by $P-\alpha$.
Hence   intersection numbers enjoy
a robust property in the sense that
\[
{\bf{i}}(P;Q)={\bf{i}}(P-\alpha ;Q-\beta)\tag{3.1}
\]
hold for every pair of complex numbers. 
In addition to (*) one has  the result below which
essentially goes back  Abel and Jacobi, but in full generality
was established by Riemann and later explained in a pure algebraic context by Brill and
M. Noether (father of the eminent  mathematician Emmy Noether).
\medskip

\noindent
{\bf{3.1 Theorem.}}
\emph{The   intersection number in (*)
 is equal to the product of the degrees of the two polynomials.}
\medskip

\noindent
We prove this in � 4. But let us give the proof in a special case. Suppose that
$P$ has degree $m$ and is of the form
\[
P(x,y)=P_m(x,y)+P_*(x,y)
\] 
where $P_*$ has degree $\leq m-1$ and
$P_m$ is homogenous of degree $m$ given as a product
\[
P_m(x,y)= \prod_{\nu=1}^{\nu=m}\, 
(y-\alpha_\nu x)
\]
where we assume that
$\{\alpha_\nu \}$ is an $m$-tuple of distinct complex numbers.
Let $Q$ be of degree $n$ and given as
\[
Q(x,y)= Q_n(x,y)+Q_*(x,y)\quad\colon\,
Q_n(x,y)= \prod_{j=1}^{j=n}\, 
(y-\beta_j x)
\]
while $Q_*$ has degree $\leq n-1$.
Assume that
\[
\alpha_\nu\neq\beta_j\tag{i}
\]
 hold for all pairs $\nu,j$.
The construction of Puiseuex charts above $x=\infty$ on the Riemann surface
$X$ attached to $P$ shows that
one has an $m$-tuple of points
\[
p_\nu=(0,1,\alpha_\nu)
\]
above $x=\infty$.
Now (i) entails that the meromorphic function $Q$ on $X$ has
poles at each of these points of order $n$.
So the sum of poles counted with multiplicities is equal to $nm$.
By general facts about meromorphic functons on a compact Riemann surface it follows that
the number of zeros counted with multiplicities is equal to $nm$ and the equality in
Theorem 3.1 follows.
Next follow  some exampes which illustrate Theorem 3.1.










\medskip


\noindent
{\bf{3.2  Example.}}
Let $P(x,y)=x^2+a-y$ and $Q(x,y)= y^2-b-x^3$ where both
where $a$ and $b$ are complex numbers.
If $S$ is the closure of $\{P=0\}$ taken in
${\bf{P}}^2$ then
$\partial S$ is reduced to the single point
$(0,0,1)$  while 
the projective closure of $\{Q=0\}$
consists of $(0,1,0)$. Hence 
the curves only intersect in
${\bf{C}}^2$.
On the Riemann surface
$X$
attached to $\{P=0\}$
we see that the meromorphic $Q$-function becomes
\[
(x^2+a)^2-b-x^3
\] 
The number of zeros is 4 for all pairs $a,b$.
In the case $a=b=0$
we get $x^3(x-1)$ which means that a zero of order three occurs at
$(0,0)$ and geometrically
one verifies that the two curves $\{P=0\}$ and $\{Q=0\}$
do not intersect transversally at the origin but have a
contact of order three
while a transveral intersection occurs when $x=1$, i.e at the point
$(1,1-a)$ on $X$.
\[
x^2-a-y=0\quad\colon\quad y^2-b-x^3=0
\]
This gives
\[ 
(x^2+a)^2= b+x^3
\]
Let us reverse the role and with $b=0$ we consider the Riemann surface
$Y$ attached to $\{y^2-x^3\}$.
At the origin we get the local coordinate $\zeta$ with
\[
x=\zeta^2\quad\colon\quad y=\zeta^3
\]
Here 
$P=x^2-y=\zeta^4-\zeta^3$ has a zero of 
order three and at the point $(1,1)$ on $Y$ one verifies that
$P$ has a simple zero, i.e. the total number of zeros of
the meromorphic
function $P$ on $Y$ is equal to four as predicted by
Jacobi.

\medskip


\noindent
{\bf{3.3 Example.}}
Let $P(x,y)= y^2-x^2-1$ while $Q(x,y)= y^2-2x^2+L(x,y)$ where
$L(x,y)$ is some linear polynomial.
Here $\{P=0\}$ and
we notice that its boundary at infinity is reduced to the points
$(0,1,1)$ and $(0,1-1)$ while those of
$\{Q=0\}$ are $(0,1,\sqrt{2})$ and $(0,1,-\sqrt{2})$
To compute the number of zeros of $Q$ on the Riemann surface
$X$ we can equally well count the number of poles.
At the two points $p^*_1=(0,1,1)$ and $p^*_2=(0,1,-1)$
we notice that the meromorphic function $Q$ has poles of order two
and it follows that the number of zeros counted with multiplicities is equal to four.
\medskip


\noindent
{\bf{3.4  Example.}}
Let $P$ be as above but this time
$Q(x,y)= (y-x)(y-\alpha x)+ L(x,y)$ where
$\alpha$ differs from one and -1.
Again $Q$ has a double pole at $(0,1,-1)$ but 
at $(0,1,1)$
we must analyze the pole in more detail.
With $\zeta$ as a local coorniate on $X$ at $(0,1,1)$ we have

\[ 
x=\zeta^{-1}\quad\colon y=\zeta^{-1}\cdot \sqrt{1+\zeta^2}
\]
So with $L(x,y)= ax+by+c$ we get
\[ 
Q=(1- \sqrt{1+\zeta^2})\zeta^{-1}\cdot \zeta^{-1}(
\sqrt{1+\zeta^2}-\alpha)+a\zeta^{-1}+ b\zeta^{-1}\cdot \sqrt{1+\zeta^2}+c
\]
It follows that the coefficient of $\zeta^{-1}$ becomes
\[ 
(1-\alpha)+a+b\tag{i}
\]
So if this number is $\neq 0$ then $Q$ has a simple pole at
$(0,1,1)$ and if it is zero no pole at all.
This, if (i) is $\neq 0$ then the number of zeros is three and if (i)=0 then
$Q$ has two zeros on $X$.
\medskip

\noindent
{\bf{3.5 Exercise.}}
Find the equation which determines the three zeros of $Q$ when
(i) is $\neq 0$ and analyze under which conditions on
the numbers $\alpha,a,b,c$ we get three transversal intersection points.
Since the elimination to achieve these equations is rather cumbersome the
material in the next section 
illustrates
the efficiency
of 
Jacobi's counting method for the number of intersection points.



\newpage

\centerline {\bf{ 4. Jacobi's residue}}
\medskip

\noindent
{\bf{Introduction.}}
The pioneering work by Jacobi  
was devotoed to the case of two complex variabes.
An extensions to higher dimensions   
was given by Weil in the article \emph{L'Integrale
de Cauchy et les fonctions des plusieurs variables}.
Here we follow Weil's methods
applied to the  case of a pair of polynomials in two variables which
form a complete intersection.
The subsequent material in  gives in particular Theorem 3.1.
In � 4.B we include 
extra material which goes beyond our present
study of Riemann surfaces. It has been
inserted since it illustrates
calculus in
several complex variables and show how 
currents appear in a natural context.




\medskip

\centerline
{\bf{A.  The construction of residues.}}


\bigskip


\noindent
Let $P(x,y)$ 
and $Q(x,y)$ be a pair of polynomials.
We do not assume that they are irreducible and they may even have
multiple factors. But we suppose
that they have no common
factor in the unique factorisation domain
${\bf{C}}[x,y]$.
This entails that
the common zero set $\{P=0\}\cap \{Q=0\}$ is  a finite
subset of
${\bf{C}}^2$ and  one says that the pair $(P,Q)$ is a complete intersection.
For a while 
we shall work close to the origin and
choose  some
$r>0$ such that
\[
\min_{(x,y)\in B(r)}\, |P(x,y)|^2+|Q(x,y)|^2=\rho>0\tag{A.1}
\]
where $B(r)= \{|x|^2+|y|^2\leq r^2\}$
is a closed ball centered at the origin and 
\[
\{P=0\}\cap \{Q=0\}\cap B(r)=\{(0,0)\}
\]
With $\rho$ and $r$ kept fixed
we  consider
pairs $\alpha,\beta$ such that
$|\alpha|^2+|\beta|^2<\rho$
and in the  common zero set
$\{P=\alpha\}\cap \{Q=\beta\}$
we only 
pick points $(x,y)$ which belong to  $B(r)$.
Recall  from calculus that
the real-analytic function
$|P|^2$
only has a discrete set of critical values which entails that there exists
some $\epsilon^*>0$ such that the real hypersurfaces
$\{|P|^2=\epsilon\}$
are non-singular for every $0<\epsilon<\epsilon^*$.
One can therefore perform integrals on these.
With $x=u+iv$ and $y=\xi+i\eta$
we identify the 2-dimensional complex $(x,y)$-space with
the 4-dimensional real space where $(u,v,\xi,\eta)$ are coordinates.
Calculus teaches how to
integrate differential 3-forms
$\psi$   over
the
smooth hypersurfaces
$\{|P|^2=\epsilon\}$ which are embeeded into the $(u,v,\xi,\eta)$ and hence
oriented in a natural way. 
If $0<\delta_*<\delta^*$ we  set
\[
\square(\epsilon;\delta_*,\delta^*)=
\{|P|^2=\epsilon\}\cap\,
\{\delta_*<|Q|^2=\delta^*\}
\]
Stokes Theorem gives:
\[
\iiint_{\square(\epsilon;\delta_*,\delta^*)}
d\phi=
 \iint _{\partial \square(\epsilon;\delta_*,\delta^*)}
\phi\tag{A.2}
\] 
for every test-form $\phi$ of degree two.
When  $g(x,y)$ is a polynomial
we apply this starting from the 2-form
\[ 
\phi=\frac{g(x,y)\cdot dx\wedge dy}{P(x,y)\cdot Q(x,y)}\tag{A.3}
\]
Notice that  $\phi$ is $d$-closed since
we already have occupied the holomorphic 1-forms $dx$ and $dy$
while
the rational function $\frac{g}{P\cdot Q}$ is holomorphic in a neighborhood
of
$\square(\epsilon;\delta_*,\delta^*)$.
Hence (A.2) gives  the equality
 \[
 \iint _{\sigma(\epsilon;\delta_*)}\,\frac{g(x,y)\cdot dx\wedge dy}{P(x,y)\cdot Q(x,y)}
=
 \iint _{\sigma(\epsilon;\delta)}\,\frac{g(x,y)\cdot dx\wedge dy}{P(x,y)\cdot Q(x,y)}\tag{A.3}
\]
where  
\[
\sigma(\epsilon,\delta)= \{|P|^2=\epsilon\}\cap \{|Q|^2=\delta\}
\quad\colon\, \epsilon,\delta>0\tag{A.4}
\] 
To be precise this is okay provided that
the pair
$(\epsilon,\delta)$ are sufficiently small so that
integration only takes place over small compact sets close to the origin
in ${\bf{C}}^2$. One refers to the family
$\{\sigma(\epsilon,\delta)\}$ as integration chains of degree two.
In a similar fashion we can make a variation of  $\epsilon$ and
arrive at the following:

\bigskip


\noindent
{\bf{A.5. Proposition.}}
\emph{There exists a pair of positive numbers
$a,b$ such that
the integrals}
\[
 \iint _{\sigma(\epsilon;\delta)}\,\frac{g(x,y)\cdot dx\wedge dy}{P(x,y)\cdot Q(x,y)}
\quad\colon 0<\epsilon<a\,\colon\, 0<\delta<b\tag{A.5.1}
\]
\medskip

\noindent
\emph{are independent of $\epsilon,\delta$
as long as 
the 2-chain 
${\sigma(\epsilon;\delta)}$ stays in the ball $B(r)$.}
\medskip

\noindent
{\bf{A.6. Jacobi's residue.}}
The common value in (A.5.1)  is denoted by $\mathfrak{res}_{P,Q}(g)$
and  called the Jacobi residue of 
$g$ with respect to $P$ and $Q$.
Notice that Jacobi's residue   depends on the ordering of $P$ and $Q$
because
we started from  the 
oriented real hypersurface
$\{|P|^2=\epsilon\}$ which  induces a positive orientation on
the  $\sigma$-chains which determines the sign of the integrals.
If the role is changed so that we start
with a hypersuface
$|Q|^2=\delta$  then
\[
\mathfrak{res}_{Q,P}(g)=-\mathfrak{res}_{P,Q}(g)\tag{A.6.1}
\]

\medskip

\noindent
{\bf{A.7 A continuity property.}}
Keeping $\epsilon,\delta$
fixed it is clear that
\[
\lim_{(\alpha,\beta)\to (0,0}\,
 \iint _{\sigma(\epsilon;\delta)}\,\frac{g(x,y)\cdot dx\wedge dy}{(P(x,y)-\alpha)
 \cdot (Q(x,y)-\beta)}
= \iint _{\sigma(\epsilon;\delta)}\,\frac{g(x,y)\cdot dx\wedge dy}{P(x,y)\cdot Q(x,y)}
\]
This entails that
\[
\lim_{(\alpha,\beta)\to (0,0}\,\mathfrak{res}_{P-\alpha,Q-\beta}(g)=
\mathfrak{res}_{P,Q}(g)\tag{A.7.1}
\]



\medskip


\centerline {\bf{A.8. The Jacobian.}}

\medskip

\noindent
For a  pair $P,Q$ as above we set
\[ 
\mathcal J(x,y)=P'_x\cdot Q'_y-P'_y\cdot Q'_x\tag{A.8.1}
\]
As explained in � XX the polynomial
$\mathcal J$  is not
identically zero. 
\medskip

\noindent
{\bf{Admissable pairs.}}
A pair of complex numbers
$\alpha$ and $\beta$ 
which  are close to zero is admissable if 
\[
\{P=\alpha\}\cap \{Q=\beta\}\,\cap\, \mathcal J^{-1}(0)=\emptyset
\]
\medskip

\noindent
{\bf{Exercise.}}
Show that the family of non-admissable pairs is finite.
A hint
is that $\mathcal J$ is unchanged
when 
the pair $P,Q$ is replaced by  $P-\alpha$ and $Q-\beta$.
\bigskip

\noindent
{\bf{A.8.2 Jacobi's residue formula.}}
Let $\alpha,\beta$ be a pair of small complex numbers such that
$\mathcal J\neq 0$ at the common zeros of $P-\alpha$
and $Q-\beta$ which are close to the origin in
${\bf{C}}^2$. 

\medskip

\noindent
{\bf{Exercise.}}
Show 
by repeated use of Cauchy's residue formula that one has the equality
\[
\mathfrak{res}_{P-\alpha,Q-\beta}(g)=(2\pi i)^2\cdot 
\sum\,\frac{g(p_k)}{\mathcal J(p_k)}
\] 


\noindent
where
the sum extends over
the distinct points in
$\{P=\alpha\}\cap \{Q=\beta\}$.
which  belong to   $B(r)$.
\medskip

\noindent
{\bf{A.8.3 A special case.}}
Above we can take $g=\mathcal J$
in which case (2.1 ) is $-4\pi^2$ times an integer.
The continuity in (A.7)   shows that this integer
is constant as $\alpha,\beta$ varies 
in the set of admissable pairs. Passing to the limit it follows that
\[ 
\mathfrak{res}_{P,Q}(\mathcal J)=K
\] 
where $K$ is the set of  points when $\{P-\alpha\}$ and $\{Q-\beta\}$
have
transversal intersections.
The absolute value
of $K$ is called Jacobi's local  intersection number
and is denoted by
${\bf{J}}(P,Q)$.

\bigskip



\centerline {\bf{B. Further results.}}
\bigskip


\noindent
Above we defined the integers
$\mathfrak{res}_{P,Q}(g)$ when $g$ is a polynomial.
Keeping $P$ and $Q$ fixed this gives an additive  
map
\[ 
g\mapsto \mathfrak{res}_{P,Q}(g)\tag{*}
\]
\

\noindent
The kernel in (*) can be described in an algebraic fashion.
Namely, let $\mathcal O_2={\bf{C}}\{x,y\}$ 
be the local ring of convergent power series in two variables. 
Thus, the elements are germs of analytic functions in  $x$ and $y$.
Now $P$ and $Q$ are elements in $\mathcal O_2$ and generate an ideal denoted by
$(P,Q)$. Set
\[
\mathcal A=
\frac{\mathcal O_2}{(P,Q)}\tag{**}
\]
The assumption that the origin is an isolated
point in
the common seros of $P$ and $Q$ entails that
the ideal $(P,Q)$ contains a sufficiently high power of the maximal ideal 
$\mathfrak{m}$ of the local ring
$\mathcal O_2$. It follows that
$\mathcal A$ is a local and finite dimensional 
complex algebra.
In commutative algebra one refers to 
$\mathcal A$ as a local artinian ring.
If $g$ is a polynomial which belongs to the ideal $(P,Q)$ it follows easily from
Jacobi's residue formula in � 4  that
$\mathfrak{res}_{P,Q}(g)=0$.
Less obvious is the following:

\medskip

\noindent
{\bf{B.1 Theorem.}}
\emph{A polynomial $g$ belongs to the ideal $(P,Q)$ in
$\mathcal O_2$ if and only if}
\[
\mathfrak{res}_{P,Q}(h\cdot g)=0
\quad\text{hold for all polynomials}\,\, h
\]



\medskip

\noindent
{\bf{B.2 Noetherian operators.}}
Theorem B.1  entails that there exists  a ${\bf{C}}$-linear form
on the finite dimensional vector space
$\mathcal A$ defined by
\[ 
\bar g\mapsto 
\mathfrak{res}_{P,Q}(g)
\]
where $\bar g$ is the image in $\mathcal A$ of a polynomial $g$.
This linear functional can be expressed by a unique differential operator with
constant coefficients. More precisely, we have the polynomial ring
${\bf{C}}[\partial_x,\partial_y]$
of differential operators with constant coefficients
where
$\partial_x$ and $\partial_y$ are the holomorphic first order
operators defining partial derivatives with respect to $x$ and$�y$.
Then there exists a unique differential operator
$\mathcal N(\partial_x\partial_y)\in 
{\bf{C}}[\partial_x,\partial_y]$ such that
\[
\mathfrak{res}_{P,Q}(g)= \mathcal N(\partial_x,\partial_y)(g)(0)
\]
for every polynomial $g(x,y)$. Thus, in the right hand side one evaluates
the polynomial $\mathcal N(g)$ at the origin.
One refers to $\mathcal N$ as the noetherian operator attached to the pair $P,Q$. It
has the special property that
$\mathcal N(\phi)=0$ for every
$\phi$ in the ideal $(P,Q)$.
To avoid possible confusion we remark that
these differential operators were introduced by Max Noether, i.e. not by
his famous daughter Emmy whose name  is attributed to
the notion of noetherian rings
as well many other
deep results in algebra.




\medskip


\noindent
{\bf{B.2.1
The construction of Noetherian operators}}. They are  obtained
via  the localised Weyl algebra $A_2(*)$ whose elements
are
differential operators in whose  coefficients are rational functions with
no pole at the origin.
The crucial result is that if
${\bf{C}}[x,y]$ is identified with zero-order differential operators
then the right ideal in $A_2(*)$ generated by $P$ and $Q$ yields
a left  module
\[
\frac{A_2(*)}{A_2(*)\cdot P+A_2(*)\cdot Q}
\] 
which is isomorphic to $m$ copies of the simple
left $A_2$-module
\[
\frac{A_2(*)}{A_2(*)\cdot x+A_2(*)\cdot y}
\]
and $m$ is the integer which gives the Jordan-H�lder length of the artinian
local ring $\mathcal A$.
We shall not enter a discussion about this and remark only that the
result above belongs to  basic material in
$\mathcal D$-module theory.
Moreover, one can assign Noetherian operators in higher dimension.
The following result was proved by Palmodov in 1966, and
fir a further account including
a more algebraic construction I refer to my notes on
residue calculus from 1996. One starts with
a commutative field $K$ of characteristic zero
and the polynomial ring $K[x_1,\ldots,x_n]$ in $n$ variables where
$n$ is some positive integer.
Let $\mathfrak{q}$ be a primary ideal and 
$\mathfrak{p}= \sqrt{\mathfrak{q}}$
its prime radical.
A differential operator $Q$ in the Weyl algebra $A_n(K)$ is noetherian 
with respect to the primary ideal if one has the inclusion
\[
Q(\mathfrak{q})\subset \mathfrak{p}
\]
Now one proves that there exists a finite set of noetherian operators
$Q_1,\ldots,Q_e$  which determine when a polynomial $q$ 
belongs to the primary ideal in the sense that
\[
Q_\nu(q)\in \mathfrak{p}\implies q\in \mathfrak{q}
\]
We remark that one can construct such a family of noetherian operators in
a canonical fashion
and the minimal number for which the implication above holds
is the multiplicity of
the primary ideal $\mathfrak{q}$ with respect to its prime radical.






\medskip

\noindent
{\bf{B.3 The Gorenstein property.}}
The local algebra $\mathcal A$ from (**)  is special. Namely it is a  local Gorenstein 
ring which means that
the 
socle  consists of elements $a\in\mathcal A$  which
are annihilated by the maximal ideal $\mathfrak{m}$ in $\mathcal O_2$
is a 1-dimensional complex vector space.
This  is   easily be proved via a diagram chasing in homological algebra
where  the assumption that the pair $P,Q$ is a  complete
intersection gives an exact Koszul complex.
A notable fact is that the image of $\mathcal J$
in $\mathcal A$  generates the
1-dimensional socle.
This is a consequence of
the following vanishing property of residue integrals:
\[
\mathfrak{res}(g\cdot \mathcal J)=0\quad\colon\forall\,\, g\in\mathfrak {m}
\]
The reader may notice that this is an
immediate consequence of Jacobi's residue formula in � A.4.




\bigskip


\noindent
{\bf{B.4 The trace map.}}
Let us introduce two new complex variables $w$ and $u$.
In the 4-dimensional complex $(x,y,w,u)$-space
one has the non-singular analytic surface
defined by the equation
\[ S=\{w=P(x,y)\} \cap \{ u=Q(x,y)\}
\]
Let $g(x,y)$ be a polynomial.
For every test-form
$\psi^{0,2}$ of bi-degree $(0,2)$
in the $(x,y,w,u)$-space
we set
\[ 
\iint_S\, g(x,y)\cdot dx\wedge\wedge \psi^{0,2}
\]
This gives a current in ${\bf{C}}^4$ 
denoted by $g\cdot \square_S$
Consider the projection $\pi(x,y,w,u)= (w,u)$.
The hypothesis that $P$ and $Q$ is a complete intersection entails that
$\pi$ restricts to a proper map  on $S$
and hence there exists a direct image current
defined by
\[ 
\phi^{0,2}\mapsto
\iint_S\, g(x,y)\cdot dx\wedge\wedge \pi^*(\phi^{0,2})\tag{1}
\]
where 
$\pi^*(\phi^{0,2})$ is the  pull-back of the test-form
$\phi^{0,2}$ in the $(w,u)$-space.
On $S$ one has the equality
\[
\pi^*(dw\wedge du)= \mathcal J\cdot dx\wedge dy
\]
Next, in the $(w,u)$-space there exists
the set of admissable points $(w,u)$ for which
$\mathcal J(x,y)\neq 0$ for all $(x,y)$ with
$P(x,y)=w$ and $Q(x,y)=u$.
We notice that this is the same as the image set
$\pi(S\cap \mathcal J^{-1}(0)$. Here
$S\cap \mathcal J^{-1}(0)$ is an algebraic hypersurface in
$S$ and since
$\pi$ restrict to a proper mapping with finite fibers, it
follows that the image set is
a hypersurface in the $(w,u)$-space which we denote by
$\Delta$ and refer to as a discriminant locus.
The direct image current (1) can be described
in
the open complement of $\Delta$.
Namely,
$\pi$ restricts to an unramified covering map from
$S\setminus \pi^{-1}(\Delta)$ onto the open complement of
$\Delta$ in 
the $(w,u)$-space,. Let  $K$ be the number of points in every fiber
which is given by
an unordered $k$-tuple
$p_k(w,u)= (x_k(w,u),y_k(w,u)\}$,
Here the coordinates $\{x_k(w,u)\}$ and $\{y_k(w,u)\}$ are local branches of
multi-valued analytic functions in
the open complement of $\Delta$.
If $h(u,w)$ is a test-function in
the $(w,u)$-space whose compact support 
does not intersect $\Delta$ and we take  $\phi^{0,2}= h(w,u)\cdot dw\wedge du$
then one has the equality
\[ 
\iint_S\, g(x,y)\cdot dx\wedge\wedge \pi^*(\phi^{0,2})=
\iint_{D^2}\, \mathfrak{Tr}(\frac{g}{\mathcal J})(w,u)\cdot h(w,u)\cdot dw\wedge du
\]
where
\[
 \mathfrak{Tr}(\frac{g}{\mathcal J}) (w,u)
=\sum_{k=1}^{k=K}\, \frac{g(p_k(w,u)}{\mathcal J(p_k(w,u))}\tag{*}
\]
\medskip


\noindent
We refer to (*) as a trace function of
$\frac{g}{\mathcal J}$.
Next,  recall that the passage to direct image currents commute
with
differentials. Since $g(x,y)$ is holomorphic
the current (1) is $\bar\partial$-closed. Indeed, Stokes theorem entails that
\[
\iint_S\, g(x,y)\cdot dx\wedge\wedge dy\wedge
\bar\partial(\psi^{0,1})=0
\] 
hold for every test-form $\psi^{0,1}$ with compact support in
the $(x,y,w,u)$-space.
\medskip

\noindent
{\bf{B.4.1 Conclusion.}}
Denote by $\gamma$
the direct image current
from (1). By (*) its restriction to
the open complement of $\Delta$ is  the density
expressed by the $(2,0)$-form
\[
\mathfrak{Tr}(\frac{g}{\mathcal J})(w,u)\cdot dw\wedge du
\]
Next, in (*) the trace function is constructed by
a sum over fibers which entails that it extends to a meromorphic function in
the $(w,u)$-space with eventual  poles  confined to $\Delta$.
Let $G$ denote this meromorphic function so that
\[
G(w,u)= \mathfrak{Tr}(\frac{g}{\mathcal J})(w,u)\tag{i}
\]
holds when $(w,u)$ are outside $\Delta$.
By the above the current $\gamma$ is $\bar\partial$-closed, i.e.
the $(2,0)$-current defined outside $\Delta$ by
$G\cdot dw\wedge du$ can be extends via $\gamma$ to a
$\bar\partial$-closed current in the $(w,u)$-space.
But this can only hold if the meromorphic function $G$ has no poles at all. 
In fact, this follows
from Hartogs' extension result in
� XX. Hence we have proved:


\medskip

\noindent
{\bf{B.4.2 Theorem.}}
\emph{For every polynomial $g(x,y)$ 
the trace function
defined by (*)  in the
complement of $\Delta$ extends to 
a holomorphic function in the $(w,u)$-space.}

\bigskip

















\centerline {\bf{C. The local algebra ${\bf{C}}\{P,Q\}$}}.
\medskip


\noindent
Given the pair $P,Q$ there exists
a subalgebra 
of $\mathcal O_2$
whose elements
are germs of analytic functions
which
can be expanded into a power series in $P$ and $Q$.
More precisely, denote by
${\bf{C}}[P,Q]$
the set of entire functions of the form
\[
\phi_n(x,y)=\sum\, c_{jk}\cdot P^j\cdot Q^k
\]
where 
$\{c_{jk}\}$ is a finite set of doubly-indexed complex numbers.
A  germ $g$ belongs to 
${\bf{C}}\{P,Q\}$ if and only if there exists a small polydisc $D^2$ 
centered at the origin in ${\bf{C}}^2$ such that
$g$ is holomorphic in $D^2$ and there exists a sequence
$\{\phi_n\}$ of $(P,Q)$-polynomials as 
above which converge uniformly to $g$ in $D^2$.
Concerning the algebra
${\bf{C}}\{P,Q\}$ a wellknown result in several complex variables 
shows that it is isomorphic to
the local ring of convergent power series in two variables, i.e. the polynomials 
considered as germs in $\mathcal O_2$ are analytically independent.
Moreover, $\mathcal O_2$ is a finitely generated
module over
its subring
${\bf{C}}\{P,Q\}$.






\bigskip



\noindent
{\bf{C.1 The algebraic  trace.}}
By the above the quotient field of $\mathcal O_2$�is a 
finite algebraic extension of the quotient field of
${\bf{C}}\{P,Q\}$. 
If $m$ is the dimension we can choose an $m$-tuple
$\phi_1,\ldots,\phi_m$ in the quotient field of $\mathcal O_2$ whose images in
$\mathcal A$ yields a basis
for this finite algebraic extension.
Let us then consider a polynomial $g$.
For each $1\leq \nu\leq m$
we can write
\[
g\cdot \phi_\nu=\sum_{j=1}^{j=m}\, \rho_{\nu,j}\cdot \phi_j
\]
where $\{\rho_{\nu,j}\}$ belong to the quotient field of
${\bf{C}}\{P,Q\}$.
As explained in �� the trace defined by
\[
\mathfrak{Tr}(g)= \sum\,\rho_{\nu,\nu}
\]
does not  depend upon the chosen  $\phi$-basis.

\newpage


\centerline
{\bf{D. $\mathcal D$-module theory.}}

\bigskip
\noindent
A gateway to compute  intersection numbers  in  a complete intersection  
employs the 2-dimensional Weyl
algebra whose elements are differential operators with
coefficiets in ${\bf{C}}[x,y]$.
Namely, let $P,Q$ be a pair of polynomials in a compete interersection.
They generate the left ideal
\[
L=A_2\cdot P+A_2\cdot Q
\]
Results in Chapter 1 from my text-book
\emph{Rings of Differential Operators}[North Holland. Math.lib.ser, 21: 1979]
imply that
\[
 M=\frac{A_2}{L}
\] 
is a holonomic left $A_2$-module 
given as a finite sum
\[
\sum\, k_\nu\cdot \mathcal B(p_\nu)
\]
where  $\{p_\nu\}$ is 
the finite set common zeros to 
$P$ and $Q$ in ${\bf{C}}^2$
and $\{k_\nu\}$  the local intersection numbers at these points.
Finally, $\mathcal B(p_\nu)$ is the left $A_2$-module
where one has taken the quotient of the left ideal generated by
$x-x_\nu$ and $y-y_\nu$ with
$p_\nu=(x_\nu,y_\nu)$. So in this way one 
recoveres the local ontersection numbers
as well as the whole sum.

\medskip

\noindent
{\bf{Durect images.}}
A deeper insight about compete intersections using
direct images of
$\mathcal D$-modules. For their construction we refer to
Chapter 2 in my text-book
\emph{Analytic $\mathcal D$-modules} [Kluwer. 1989].
Here one considers another copy of
${\bf{C}}^2$ with coordinates $(\zeta,\eta)$
and the polynomial map
\[
\rho\colon (x,y)\mapsto (P(x,y),Q(x,y))
\]
With notations from [ibid] there exists the direct image
\[ 
\mathcal M= \rho_+(\mathcal O)
\] 
where $\mathcal O$ is the sheaf of holomorphic functions in
the
$(\zeta,\eta)$-space. General results from [ibid] entail  that
$\mathcal M$ is a regular holonomic
$\mathcal D$-module in the $(x,y)$-space whose
characteristic cycle
recaptures  the local 
intersection numbers from
� D. A more delicate
study arises when one considers
the points in Sato's
singular spectrum
$\text{SS}(\mathcal M)$
which are outside the zero section of the cotangent bundle over
${\bf{C}}^2$.
Again we refer to [ibid] for
these standard concepts
and notations
in
$\mathcal D$-module theory,
The polynomial map $\rho$
is special since it has finite fibers. To analyze situations
at the points of ramification
one 
employs
micro-localisations, i.e. one regards the regular holonomic
$\mathcal E$-module
\[
\mathcal E\times_{\pi^{-1}(\mathcal D}\, \pi^{-1}(\mathcal M)
\]
Now one can begin to  perform a micro-local study. Let us 
remark that during this investigation one should profit upon
"modern sheaf theory" from  the  text-book
by Kashiwara and Schapira
which gives  insight about
singularites in the base manifold after  the  passage to
micro-localisations of constructible sheaves. Here
the fundamental  notion of micro-support of sheaves
is essential, and my opinoion is that
this    should be one of the first issues i
when
one  teaches courses in general sheaf theory.
since  micro-loca considerations are needed in order
to fully grasp the 
notion of singularities.



\medskip

\noindent
\emph{Summing up}, a  more comprehensive study of
complete intersections with "generic parameters" arises via systematic use of 
$\mathcal D$-module theory which has the merit
that 
various facts become clear, while
they are not so easily understood if one restricts the attention to  
base manifolds only.
Let us also  remark that
when $P,Q$ is a  complete intersection, then
one can study more general direct images, i.e. replace $\mathcal O$
by an arbitrary
regular holonomic $\mathcal D$-module
and take its image under the direct image functor $\rho_+$.
Now  $\rho_+$ is
an exact covariant  functor from the 
abelian category of regular
 holonomic modules 
in the $(x,y)$-space to those in the $(\zeta,\eta)$ space.
So this functor is attached to the given complete intersection
and can be regarded as an "ultimate" object attached to the pair $P,Q$.
It goes without saying that these constructions can be extended to cover
the case of
a complete interersection in any number of variables.
One can also replace the affine space
${\bf{C}}^n$ by an arbitrary affine and non-singular algebraic variety
$V$ where the Weyl algebra is replaced by the ring
$\mathcal D(V)$ of globally defined differential operators as explained in
Chapter 3 from my book on rings of Differential Operators.




\bigskip


\centerline
{\bf{E. Residue calculus}}
\bigskip


\noindent
Above we have considered Jacobi's residues when  the
denominators are holomorphic functions.
More generally, if  $P$ and $Q$ are in a complete intersection we can
still
define the integrals
$J_g(\epsilon,\delta)$ from Proposition A.5.1
when $g(x,y)$ is a test-function with compact support
close to the origin.
But here the integral is no longer independent of the pair
$(\epsilon,\delta)$. This    leads to a  more
involved  situation  which was originally 
studied by Miguel Herrera. He  proved
that there exists certain limits  provided  that
one pays attention while $\epsilon$ and $\delta$ tend to zero.
We shall not enter a detailed discussiona about multi-residue calculus
but
recall that an example  discovered by Passare and Tsikh in
[P-T]  shows that
the   unrestricted limit of the integrals from Propostion A.5.1
does not exist in general.
In fact, such   examples are generic and one can even take
one polynpomial to be $x^M$ for suitable positive integer.
The reader may consult my article
[Bj�rk:Abel Legacy] for
further comments about the Passare-Tsikh example, as well about
Passare sectors which describe when
one can perform limits and arrive at the same residue  current as 
that of Herrera and  co-author Coleff.
On the positive side there exists a remarkable result  due to  H. Samuelsson
in his Ph.d-thesis at Chalmers University 2005.
From an analytic point of view Theorem D.1 below is
quite useful. Instead of taking "ugly residues"  one employs
regularisations. Namely, for each   polynomial $P$
there exists a smooth current
\[
\frac{\bar P}{|P|^2+\epsilon}
\] 
One can apply the $\bar\partial$-current and
get a smooth $(0,1)$-current
\[
\rho_P(\epsilon)=
\bar\partial(\frac{\bar P}{|P|^2+\epsilon})
\]
Similarly we construct smooth currents $\rho_\delta(Q)$.
With these notations one has:

\medskip

\noindent
{\bf{D.1 Theorem.}} \emph{In the space of $(0,2)$-currents there exists an unrestricted limit}
\[
\gamma_{P,Q}=
\lim_{(\epsilon,\delta)\to (0,0)}\,
\rho_P(\epsilon)\wedge \rho_\delta(Q)
\]
\emph{Moreover, this current is of the Coleff-Herrera type which means
that}
\[
\gamma_{P,Q}(\phi^{2,0})=0
\]
\emph{for every test-form
$\phi^{2,0}$ given as $[\bar x\cdot g_1(x,y)+\bar y\cdot g_2(x,y)]\ddot d\wedge dy$
where $g_1,g_2$ is a pair of test-functions.}
\medskip

\noindent
{\bf{Final remark.}}
It would bring us too far to discuss multi-valued residue theory in more detail. 
Above we have only exposed the so called absoute case in dimension two.
Let
is only say that in spite of  many impressive results which have been achieved
during the last four decades,   there 
remain many open questions.
Here one often regards problems related to analysis.
To give an example, consider a complete intersection $P$ and $Q$
where the origin in
${\bf{C}}^2$ is the sole common zero.
 Now there exists the distribution valued function of two complex parameters
 $s$ and $t$ defined by
 \[
 \mu_{s,t} (g)=\int_{{\bf{C}}^2}\,
 P^{-1}\cdot Q^{-1}\cdot
  |P|^s\cdot |Q|^t\cdot g \, dA
 \]
 where $dA$ is the Lebesgue measure in
 ${\bf{C}}^2$ and $g$ are test-functions.
$\mathcal D$-module theory entails that
$\mu$ extends to a meromorphic distribution-valued function in the
complex $(s,t)$-space.
Poles  occur and
the  complete intersection entails that
polar distributions under this meromorphic extension 
are supported by the origin.
To analyze all this leads to a quite involved study.
 We remark only that a number of results
 in connection with this have been obtained by
 Barlet, Sabbah and Yger
 who are among the invited speakers to
 the conference devoted to differential systems on the complex domain
 at Stockholm university in May 2010.
 
 
 
 
 

























\end{document}



















