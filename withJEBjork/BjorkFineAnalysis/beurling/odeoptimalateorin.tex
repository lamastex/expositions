




\documentclass{amsart}


\usepackage[applemac]{inputenc}

\addtolength{\hoffset}{-12mm}
\addtolength{\voffset}{-10mm}
\addtolength{\textheight}{20mm}

\def\uuu{_}

\def\vvv{-}

\begin{document}











\centerline {\bf{Special chapter: ODE-equations.}}
\medskip

\noindent
On the real $x$-line the space of distributions is denoted by
$\mathfrak{Db}({\bf{R}})$. Notice that we do not insist that
the distriobutions are tempered, i.e.
$\mathcal S^*$ appears as a proper subspace.
As a first  example we take  the first order differential operator
\[
\nabla=x\cdot \frac{\partial}{\partial x}
\]
When $x\neq 0$ the equation $\nabla(f)=0$ has
solutions given by constant functions.
To pass beyond $x=0$ we take  the Heaviside densities
$H^+$ and $H_-$, where $H^+(x)=1$ when $x>0$ and  zero if $x<0$, while
$H_-=1-H^+$. It turns out that
these two linearly independent distributions on the $x$-line
generate the vector space of all distribution solutions to
the equation $\nabla(\mu)=0$. See � xx below for details.
On the other hand, if $f(x)$�is a $C^1$-function, i.e. continuously
differentiable which  satisfies $\nabla(f)=0$, then it is
clear  that $f$ must be a constant.
So  the set of distribution solutions is more extensive.
Next, let $s$ be a complex number which is not an integer.
If  $\mathfrak{s}>-1$ then $x^s$ is  integrable 
on intervals $(0,a)$ with $a>0$ and there exists a distribution denoted
by $x_+^s$ acting as a linear functional on test-functions
$\phi(x)$ by
\[ 
x_+^s(\phi)= \int_0^\infty x^s\cdot \phi(x)\, dx
\]
On the open interval $(0,+\infty)$
we notice that
\[ 
(\nabla-s)(x^s)=0\quad\colon x>0
\]
It we apply 
$\nabla-s$ to the distribution $x_+^s$
the construction of distribution derivatives means that
$(\nabla-s)(x_+^s)$ acts on test-functions $\phi$ by
\[
\phi\mapsto \int_0^\infty x^s \cdot (-\partial(x\phi)-s\phi)\, dx\tag{i}
\]
When $\mathfrak{Re}\, s>-1$
a partial integration shows that
(i) is zero.
Hence $(\nabla-s)(x_+^s)=0$.
It turns out that there exist more distributions $\mu$ such that
$(\nabla-s)(\mu)=0$. Namely, there exists
the boundary
value distribution 
$\mu=(x+i0)^s$ which also satisfies the equation $(\nabla-s)(\mu)=0$.
Here we recall that $\mu$ is defined on test-functions $\phi(x)$ by
\[
\mu(\phi)= \lim_{\epsilon\to 0}\,
\int\, (x+i\epsilon)^s\cdot \phi(x)\, dx\tag{ii}
\]
Hence we have found two linearly independent 
distribution solutions to the equation $\nabla-s(\mu)=0$.
It turns out that they give a basis for the null solutions which gives the dimension
formula:
\[ 
\dim_{{\bf{C}}}\, 
(\text{Ker}_{\nabla-s}(
\mathfrak{Db}))=2
\]
\medskip

\noindent
{\bf{A special example.}}
Take $Q= \nabla+1$. Here
two boundary value distributions
$(x+i0)^{-1}$ and $(x-i0)^{-1}$ are null solutions.
Let us also recall that
the difference
\[
(x-i0)^{-1}-(x+i0)^{-1}=\pi i\cdot \delta_0
\]
So the Dirac measure at $x=0$ is also a null solution
which of course could have been verified directly.
By the general result in � xx the space of null solutions is
2-dimensional so above we have found a basis.
\medskip

\noindent
In � xx we consider general
differential operators with polynomial coefficients
\[ 
P(x,\partial)=p_m(x)\partial^m+\ldots+p_0(x)
\]
Under the assumption that the real zeros of the
leading polynomial $p_m(x)$
are simple and consists of some $k$-tuple
$\{a_1<\ldots a_k\}$ we show in � xx 
that the
$P$-kernel on
$\mathfrak{Db}$ has dimension $k+m$.
\medskip



\noindent
{\bf{0.1 A first order ODE-equation.}}
Let $p$ and $q$ be a pair of polynomials and set
\[
Q=q(x)\cdot \partial -p(x)
\] 
Assume that 
$q$ is a monic polynomial of some  degree $k\geq 2$ whose  zeros are real and simple and
arranged in strictly increasing order
$\{ a_1<a_2<\ldots <a_k\}$. The polynomial $p$ is such that
$p(a_\nu)\neq 0$  for every $\nu$ and in general it has complex coefficients and
no condition is imposed upon its degree.
Now we seek distributions $\mu$ on the $x$-line such that
$Q(\mu)=0$.
One such solution is found
as the boundary value of the analytic function
defined in the upper half-plane by
\[
f(z)= e^{\int_i^z\, \frac{p(\zeta)}{q(\zeta)}\, d\zeta}
\]
Tos see this we notice
that
when $\mathfrak{Im} z>0$ it is evident that
the complex derivative
\[
\frac{\partial f}{\partial z}=\frac{p(z)}{q(z)}\tag{i}
\]
Since
the passage to boundary value distributions commute
with
derivations it follows that
the boundary value distribution
 $f(x+i0)$ is a null solution to $Q$. Less obvious is that
 each simple and real zero 
$a_\nu$  of $q$ yields a null solution
$\mu_\nu$ supported by the half-line 
$[a_\nu.+\infty)$.This is a consequence of general results in � xx.
Let us remark 
that
without using boundary values of analytic functions it is not
easy to discover all this.



\bigskip






\noindent
{\bf{0.2 The equation $\nabla^2(\mu)=0$}}.
The Fuchsian operator is defined by
$\nabla =x\partial$.
It turns out that the space of distributions
$\mu$ satisfying $\nabla^2(\mu)$ is a 4-dimensional vector space.
One solution is the Heavisde function $H_+$
defined by the density 1 if $x>0$ and zero if
$x\leq 0$.
Here
\[ 
\partial(H_+)(g)=
-\int_0^\infty g'(x)\, dx= g(0)\quad\colon g\in C_0^\infty({\bf{R}})
\] 
This means that the distribution derivative
$\partial(H_+)=\delta_0$ and since
$x\cdot \delta_0=0$ we have 
$\nabla(H_+)=0$. 
Next, on $\{x>0\}$
we see that the density $\log x$ satisfies
$\nabla^2(\log x)=0$.
It is tempting to extend the locally integrable function
$\log x$ on the positive half-line to
${\bf{R}}$ by setting the value zero if $x\leq 0$.
Denote the resulting distribution by
$\log_+x$. Now
\[
\nabla(\log_+x)(g)=
-\int_0^\infty\, -\partial(xg)\cdot \log x\, dx=
\int_0^\infty\,xg\cdot \frac{1}{x}\, dx=
\int_0^\infty\,g dx
\]
Hence $\nabla(\log_+x)= H_+$
Since $\nabla(H_+)=0$
we get
$\nabla^2(\log_+ x)=0$.
Hence we have found two  linearly independent null solutions
given by the pair $(H_+,\log_+ x)$ which are supported by
$x\geq 0$.
In addition we find two other null solutions.
The first is the constant density 1, The second is
the boundary v alue distribution
$\log (x+i0)$.
By the gneral resu,t in � xx the space of null solutions is
4-dimensional so above we have found a basis for these.

\medskip


\noindent
{\bf{0.3. Higher order Fuchsian  equations.}}
Let $m\geq 2$ and consider an operator of the form
\[
Q= \nabla^m+q_{m-1}(x)\nabla^{m-1}+\ldots+q_1(x)\nabla+q_0(x)
\]
where $\{q_\nu(x)\}$ are polynomials.
With $\{c_\nu=q_\nu(0)\}$
we associate the polynomial
\[
Q^*(s)= s^m+c_{m-1}s^{m-1}+\ldots+ c_1s+c_0
\]
Under the assumption that
$Q^*(k)\neq 0$
for all non-negative integers
the solution space
$\mathcal S=\{\mu\colon\, Q(\mu)=0\}$
has dimension $2m$ and a basis is found as follows:
In
the upper half-plane 
the Picard-Fuchs theory about holomorphic differential equations
entails that 
there exists
an $m$-tuple of linearly independent analytic functions
$\{\phi_\nu(z)\}$ which solve $Q(z,\partial_z)(\phi_\nu)=0$.
Similarly one finds an $m$-tuple $\{\psi_\nu\}$ 
of linearly independent analytic functions in  the lower half-plane.
The boundary
value distributions
$\{\phi_\nu(x+i0)\}$ and
$\{\psi_\nu(x-i0)\}$
belong to $\mathcal S$ and  are linearly indepedent.
For if 
$\sum\, c_\nu\phi_\nu(x+i0)+ \sum d_\nu\psi_\nu(x-i0)=0$
where at least some
$c_\nu$ or $d_\nu$ is $\neq 0$
then \[
\phi_*(x+i0)=\psi_*(x-i0)=0\tag{i}
\]
where
$\phi_*=\sum\, c_\nu\phi(x+i0)\neq 0$ and
$\psi_*=-\sum\, d_\nu\psi(x-i0)\neq 0$.
Now (i) cannot hold. The reason is that 
the assumption about
$Q^*(s)$  entails that
the equation $Q(z,\partial_z)(f)=0$ has no holomorphic solutions
at $z=0$. This fact stems from local
$\mathcal D$-module theory and is exposed in � xx.
Now  the  reflection principle  for analytic functions entails
that the analytic wave front
sets of the distributions $\phi_*$ 
 and $\psi_*$ both are non-empty.
 On the other hand the material in � xx shows 
 that these non-empty wave fronts  have opposed directions and hence the 
 equality (i) cannot hold. This proves that 
 $\mathcal S$ is at least $2m$-dimensional
 and by the general results in � xx
 we have equality. So above we have constrcuted a basis for the null solutions.
 

\bigskip




\centerline{\bf{� 1. Fundamental solutions to ODE-equations with constant coefficients}}

\bigskip

\noindent
We  consider differential operators  with constant coefficients
acting on the real $x$-line. To
simplify the passage to Fourier transform we introduce the first
order operator
\[ 
D=\frac{1}{i}\cdot \frac{d}{dx}
\]
If $P(\xi)$ is a polynomial of the
$\xi$-variable and $\mu$ is a tempered distributionon
the $x$-line this gives  the equality:
\[
\widehat{P(D)\mu}(\xi)= P(\xi)\cdot \widehat\mu(\xi)\tag{*}
\]
By a tempered fundamental solution to
$P(D)$ we mean a distribution $\mu\in\mathcal S^*$ such that
\[
P(D)\mu=\delta_0
\] 
where $\delta_0$ is the Dirac measure at $x=0$.
Since
the Fourier transform
of $\delta_0$ is the identity on the $\xi$-line 
the Fourier transform
of a fundamental solution satisfies
\[
P(\xi)\cdot \widehat\mu(\xi)=1
\]
When
$P(\xi)$ has no zero on the
real $\xi$-line  there
exists a  fundamental
solution
given as the inverse Fourier transform of
the smooth density $P(\xi)^{-1}$.
If $P(\xi)$ has some real zeros we can write 
\[
P(\xi)=Q(\xi)\cdot R(\xi)
\]
where
$R$ has real zeros and the zeros of
$Q$ are all non-real.
The factorisation is unique when we
choose constants so that
$Q(\xi)$ is a monic polynomial.
The case $\deg Q=0$ is not excluded, i.e. this 
holds  when 
all zeros of $P(\xi)$ are real.
But in general one has a mixed case where
$n=\deg P$ and $1\leq \deg Q\leq n-1$.
\medskip

\noindent
{\bf{1. The case $\deg Q=0$}}.
When all zeros of $P(\xi)$  real there exists the boundary
value distribution on the $\xi$-line defined
by
\[
\gamma= \frac{1}{P(\xi-i0)}\tag{1.1}
\]
By the general results from � XX its inverse Fourier transform is supported by 
the half-line $\{x\geq 0\}$. Let $\mu_+$ denote this distribution.
Then
\[
\widehat{P(D)\mu_+}= P\xi)\cdot\gamma=1
\]
and hence $\mu_+$ is a fundamental
solution.
\medskip

\noindent
{\bf{2. The mixed case}}.
If $P=Q\cdot R$ where
$1\leq \deg Q\leq n-1$
we proceed as follows.
First
one has a bijective map on
the space of tempeted distributions on
the $\xi$-line defined by
\[
\gamma\mapsto
Q(\xi)^{-1}\cdot \gamma
\]
Fourier's inversion fomrula gives
a bijective linear operator
$T_Q$ on the space of tempered distribtutions on the $x$-line such that
\[
\widehat{ T_Q(\mu)}= Q(\xi)^{-1}\cdot\widehat\mu
\]
So if 
$\mu$ is a temepred distribution we get
\[
P(D)(T_Q(\mu))= R(D)(\mu)\tag{2.1}
\]
The zeros of $R(\xi)$ are real which gives
the
fundamental solution $\nu_+$  to $R(D)$
and now
\[ 
\mu=T_Q(\nu_+)\tag{2.2}
\] 
yields a fundamental solution to $P(D)$.
In this way we have constructed  a fundamental
solution in a canonical fashion.
In contrast to the real case where
$\deg Q=0$ the distribution $\mu$ above is in general not supported by
the half-line $\{x\geq 0\}$. We give  examples in � XX below.

\bigskip

\noindent
{\bf{3. The determination of $\mu_+$}}.
Consider the case when $\deg Q=0$ so that
the fundamental solution $\mu_+$ is
the inverse Fourier transform of (xx) above.
Let us for the moment assume that
the real zeros of $P(\xi)$ are all simple
and given by an $n$-tuple $\{\alpha_k\}$.
Define the distribution
$\rho$ on the real $x$-line by  the density
\[
\rho(x)= \sum\, \frac{1}{P'(\alpha_k)}\cdot e^{i\alpha_kx}\quad\colon x\geq
0
\] 
while $\rho(x)=0$ when $x<0$.
It is clear that the distribution 
$P(D)\rho$ vanishes when $x\neq 0$, i.e. 
supported by the singleton set $\{x=0\}$.
Newton's  formula from  � xx gives
\[ 
\sum_{k=1}^n\, \frac{1}{P'(\alpha_k)}\cdot \alpha_k^m=0\quad\colon\,\,
0\leq m\leq n-2
\]

\medskip

\noindent
This entails that
the derivatives up to order
$n-2$
of $\rho$ vanish at $x=0$.
Using this
we  show that $\rho$ up to a  constant gives a fundamental solution to $P(D)$.
For consider a test-function $f(x)$
and
let $P^*(D)$ be the adjoint of $P(D)$. 
The vanishing of the derivatives of $\rho$ at $x=0$ above 
gives after partial integration
\[
\int\, \rho(x)\cdot P^*(D)(f)(x)\, dx=(-1)^{xx}\cdot \rho^{(n-1)}(0)\cdot f(0)
\]
\medskip


\noindent
{\bf{4. Conclusion.}}
The fundamental solution $\mu_+$ supported by  $x\geq 0$ 
is given by the density
\[ 
\mu_+(x)=\frac{n}{xx}\cdot
\rho(x)= \sum\, \frac{1}{P'(\alpha_k)}\cdot e^{i\alpha_kx}
\]


\noindent
{\bf{5. Example.}}
Consider $P(D)= D^2-1$ so that $P(\xi)= \xi^2-1$.
Here 1 and $-1$ are the simple zeros and (xx) gives
\[ 
\mu_+(x)=XXX\cdot \sum\, \frac{1}{-2}\cdot e^{-ix}+\frac{1}{2}\cdot e^{ix}
=-\sin x\quad\colon\, x\geq 0
\]



\noindent
{\bf{6. An example in the mixed case.}}
Let $P(D)= (D^2+1)(D-a)$ 
where $a$ is some real number
$\neq 0$.
So here $Q(\xi)= \xi^2+1$
and the fundamental solution from � 2  becomes
\[
\mu=T_Q(\nu_+)\tag{6.1}
\]
where $\nu_+$ is the inverse Fourier transform
derived from the linear polynomial. $R(\xi)=\xi-a$.
This gives
\[
 \nu_+(x)= -e^{iax}\quad\colon x\geq 0\tag{6.2}
\]



\noindent
{\bf{7. The expression of $\mu$.}}
By the above $\mu$ 
is  the convolution of $\nu_+$ and
the continuous density 
\[ 
\phi(x)= \frac{1}{2\pi}\int\, \frac{e^{ix\xi}}{1+\xi^2}\, d\xi
\]
We leave it to the reader to verify that
\[ 
\phi(x)=
\frac{1}{2}\cdot e^{-|x|}
\]
Hence 
\[
\mu(x)=-\frac{1}{2}\cdot \int_0^\infty\ e^{-[x-y|}\cdot e^{-aiy}\, dy
\]
The reader is invited to analyze this function 
using
a computer to
plot  this function with different choice of $a$.


\newpage



\centerline {\bf{2.0. ODE-equations on the real line}}

\medskip

\noindent
To grasp the notion of distributions
it is natural to  
start  with a  study of distribution solutions to ordinary differential
operators which
leads to  more systematic results as compared to 
studies before  distribution theory was
established.
An example is the confluent hypergeometric function which
arises as a solution to a differential operator of the form
\[
P=x\partial^2+(\gamma -x)\partial -a
\]
where $\gamma$ is a non-zero complex number while
$a$ is arbitrary.
In the classic literature one solves this equation via the
Laplace method which involves a rather cumbersome
use of residue calculus.
More  information about the operator
$P$ arises when one determines its kernel on the  space
of distributions on the real $x$-line.
The  result in Theorem 0.0.1 below shows that
this $P$-kernel is a 3-dimensional subspace of 
$\mathfrak{Db}$. Moreover, there exists a fundamental 
solution supported by the 
half-line $\{x\geq 0\}$.
\medskip

\noindent
Let us now regard
a differential operator with
polynomial  coefficients
\[
P(x,\partial)=q_m(x)\cdot 
\partial^m+q_{m-1}(x)\partial^{m-1}+
\ldots+q_0(x)\tag{*}
\]
where $m\geq1 $ and  $q_0(x),\dots,q_m(x)$
are polynomials which in general  have complex coefficients.
Let $\mathfrak{Db}$ be the space of distributions on the real
$x$-line.
A first question is to determine the
$P$-kernel, i.e. one seeks all distributions $\mu$ such that
$P(\mu)=0$.
Following material from the thesis by Ismael (xxx - University of
xxx) we 
expose some general facts about null solutions to general operators
as above.
\bigskip

\noindent
{\bf{The local  Fuchsian condition.}}
We shall restrict the study to operators $P$
which are  \emph{locally Fuchsian} at every real zero of the leading polynomial
$q_m(x)$. This 
means the following:
Let $a$ be a real zero of
$q_m(x)$ with some multiplicity $e\geq 1$
so that $q_m(x)=q(x)(x-a)^e$ where the polynomial $q$ is
$\neq 0$ at $a$.
Then we can
write
\[
P(x,\partial)=q(x)\cdot 
[(x-a)^e\partial^m+r_{m-1}(x)\partial^{m-1}+
\ldots+r_0(x)]
\] 
where $\{ r_\nu=\frac{p_\nu}{q}\}$ are rational functions with
no pole at $a$ and therefore define analytic functions in a neighbrohood of $a$.
Hence
\[
 P_*(x,\partial)=(x-a)^e\partial^m+r_{m-1}(x)\partial^{m-1}+
\ldots+r_0(x)
\] 
can be identified with a germ of
a differential operator with coefficients in the local ring
$\mathcal O(a)$ of germs of analytic functions
at $a$.
The ring
$\mathcal D$ of such germs of differential operators
is
studied in � x where we  define the subfamily of
Fuchsian operators.
For example, if $a=0$ then
a Fuchsian operator in
$\mathcal D$ can be expressed as
\[
\rho(x)\cdot [\nabla^m+g_{m-1}(x)\nabla^{m-1}+\ldots+g_0(x)]
\] 
where $g_{m-1},\ldots,g_0$ belong to $\mathcal O$
and
$\nabla=x\partial$, while $\rho$ in general is a meromorphic
function, i.e. it may have a pole of some order at $x=0$.
A local study of null solutions to fuchsian operators in
$\mathcal D$ is carried out in � xx.
From this one can derive the
following conclusive result:
\medskip

\noindent
{\bf{0.1 Theorem }}
\emph{Let  $P(x,\partial)$ in (*) above be  locally Fuchsian
at the real zeros of $p_m$.
Then $\text{Ker}_P(\mathfrak{Db})$  is a complex vector space
of dimension 
$m+e_1+\ldots+e_k$,
where $\{e_\nu\}$ are the multiplicities
at
the real zeros of $p_m$. Moreover, for each real zero of
$p_m$ there exists a  fundamental solution $\mu$ supported by
$\{x\geq a\}$  such that
$P(\mu)= \delta_a$.}
\bigskip



\noindent


\noindent
{\bf{Remark.}}
In addition to this   the following supplement to Theorem 0.1 hold.
For   each real zero $a$ of $p_m(x)$ with some multiplicity
$e$ there exists
a distinguished $e$-dimensional subspace
$V_a$
of
$\text{Ker}_P(\mathfrak{Db})$  which consists of distributions
$\mu$ supported by  
the closed half-line $[a,+\infty]$ whose analytic wave front
sets satisfy the following: First, 
they contain  the whole fiber above $a$ and the remaining part
of the analytic wave front set
is either empty or a  union of half-lines
above some of the 
real zeros 
of $p_m$ which are $>a$.
Moreover, one has a direct sum decomposition 
\[
\text{Ker}_P(\mathfrak{Db})=\mathcal F_+
 \oplus\, V_{a_\nu}\tag{**}
\] 
where the last  direct sum is taken over the real zeros of $p_m$, and
$\mathcal F_+$ is an  $m$-dimensional subspace of
$\mathfrak{Db}$ with a basis given by
an $m$-tuple of     boundary value distributions
$\{\phi_k(x+i0)\}$.
Here  
$\{\phi_k(z)\}$ are analytic functions in
a strip domain 
$U=\{-\infty < x<+\infty\}\times \{0<y<A\}$
with $A>0$ chosen so that
the complex polynomial $p_m(z)$ is zero-free in this domain
and each  $\phi_k(z)$ satisfies the homogeneous equation
$P(z,\partial)(\phi)=0$ in $U$.
\medskip


\noindent
{\bf{Example.}}
Consider the first order differential operator 
\[ 
P=x\partial+1
\]
Outside $x=0$ the density $x^{-1}$ is a solution.
Now the Euler distribution $x_+^{-1}$
is suppored by $[0,+\infty)$.
The 1-dimensional
$\mathcal F_+$-space in (**)
is generated by the boundary value distribution
$(x+i0)^{-1}$ and in $V_0$ we find the Dirac measure $\delta_0$
which together with
$(x+i0)^{-1}$ is a basis for the null solutions. Next, 
an easy computation gives
\[
P(x_+^{-1})=\delta_0
\] 
and hence the Euler distribution $x_+^{-1}$ yields a fundamental solution.









\medskip

\noindent
{\bf{0.0.2 Tempered  solutions.}}
The $P$-kernel in  Theorem 0.0.1 need not consust of 
tempered distributions. 
The reason  is that we have not imposed
the condition that $P$ is locally Fuchsian at
infinity.
So if $\mathcal S^*$ denotes the space of tempered distributions, then
$\text{Ker}_P(\mathcal S^*)$ can have strictly smaller dimension than
$m+k$ and the   determination of the tempered solution space
leads to a more involved analysis.
Already the case $P=\partial-1$ illustrates the situation. Here
the $P$-kernel on
$\mathfrak{Db}$ is the 1-dimensional space given by the exponential density
$e^x$ which is not
tempered so the
$P$-kernel on $\mathcal S^*$ is reduced to zero.
During the search for tempered  fundamental  solutions  to $P$
supported by half-lines $\{x\geq a\}$
one can   use a  result due to Poincar�  under the extra assumption that 
$\deg p_k\leq \deg p_m$ hold for every $0\leq k<m$.
For in  this case there are 
series expansions when $x$ is large and positive:
\[
\frac{p_k(x)}{p_m(x)}=c_k+\sum_{\nu=1}^\infty\, c_{k\nu}x^{-\nu}
\quad\colon 0\leq k\leq m-1
\]
The leading coeficients $c_0,\ldots,c_{m-1}$
give a monic
polynomial
\[
 \phi(\alpha)=\alpha^m+c_{m-1}\alpha^{m-1}+\ldots c_0
 \] 
Let us also chose 
$A>0$  so  large that the leading polynomial $p_m$ has no
real zeros on
$[A,+\infty]$. This  gives an
$m$-dimensional  space of null solutions where a basis consists
of real-analytic densities $u_1(x),\ldots,u_m(x))$
on this interval.
 
\medskip

\noindent {\bf{0.0.3 Poincar�'s theorem.}}
\emph{Suppose that  $\phi$ has  simple
simple zeros 
$\alpha_1,\ldots,\alpha_m$. 
Then, with $A$ as above
one can arrange the $u$-basis so that}
\[
u_k(x)=e^{\alpha_k x}\cdot g_k(x)
\]
\emph{and  there exists a non-negative integer $w$ and a constant $C$ such that}
\[ 
|g_k(x)\leq C\cdot (1+x)^w\,\colon 1\leq k\leq m
\]
\emph{hold for all $x\geq A$.}


\medskip

\noindent
So for 
indices $k$ such that
$\mathfrak{Re}\,\alpha_k\leq 0$, it follows that
$u_k(x)$ has tempered
growth as $x\to +\infty$.
In particular Poincar�'s result entails
that if the real parts are all $\leq 0$, then the
fundamental solutions from Theorem 0.0.1 
are all tempered.


\medskip

\noindent
{\bf{0.0.4 Example.}}
Consider the operator
\[
 P=x\partial^2-x\partial-B
 \] 
 where  $B> 0$.
 In this case
 \[
 \phi(\alpha)=\alpha^2-\alpha=\alpha(\alpha-1)
\]
so one of the $u$-solutions above increse exponentially
while the other has tempered growth as $x\to +\infty$.
It is easily seen that
 the there exists   an entire solution
 \[ 
 f(x)= x+c_2x^2+\ldots\tag{i}
 \]
such that $P(f)=0$, whose
coefficients are found by the recursive formulas
\[ 
k(k-1)c_k= (k-1+B(c_{k-1}\quad \colon k\geq 2
\]
Hence $\{c_k\}$ are positive and it is clear that
$f$ has exponential growth as $x\to +\infty$.
In addition we have a solution on
$x>0$ of the form
\[ 
g(x)=  f(x)\cdot \log x+a(x)
\]
In  �� we explain that  $P(g_+)= a\cdot \delta_0$
hold for a non-zero constant
while $P(f_+)=0$.
Next, let et $u_1$ be the tempered solution and $u_2$ the non-tempered solution in
Poincar�'s theorem on the half-line $x>0$.
There are  constants $c_1,c_2$ such that
\[
f(x)= c_1u_1(x)+c_2u_2(x)
\]
Here   $c_2\neq 0$ because $f$ increases exponentially on $(0,+\infty)$.
At the same time
\[
g(x)= d_1u_1(x)+d_2u_2(x)
\]
hold for some constants $d_1,d_2$. Set
\[ 
\gamma=g_+-\frac{d_2}{c_2}\cdot f_+
\]
From the above $\gamma$ has tempered growth as
$x\to+\infty$ and 
$P(\gamma)= a\cdot \delta_0$
with $a\neq 0$. Hence $\mu= a^{-1}\cdot \gamma$ yields 
a tempered fundamental  solution supported by
$\{x\geq 0\}$.
\medskip

\noindent
In � xx we give further examples
of tempered fundamental solutions.

\medskip



\noindent{\bf{0.5 Another  example.}}
Here we take 
\[
P= \nabla^2+q(x)\tag{0.3.1}
\] 
where $q(x)$ is a polynomial  such that $q(0)=-1$ and $q'(0)=0$.
For example, if $q(x)=x^2-1$ we encounter 
a wellknown Bessel operator.
It is easily  seen that there exists a unique entire solution
$f(x)$  which satisfies $P(f)=0$ with 
a series expansion
\[ 
f(x)= x+c_3x^3+\ldots
\]
Moreover, one verifies easily that there exists another entire
function $g(x)$ with $g(0)=0$ such that
the multi-valued function
\[
\phi(z)= f(z)\cdot \log z+ g(z)\tag{i}
\]
satisfies $P(\phi)=0$.
Theorem 0.0.1 predics that the $P$-kernel on
$\mathfrak{Db}$ is 4-dimensional. To begin with
$f$ restricts to a real analytic densitiy on the
$x$-line and gives a null solution.
A second  solution is obtained by the boundary value distribution
\[
\gamma=\phi(x+i0)= f(x)\cdot \log (x+i0)+ g(x)
\]
Together they give a basis in the 2-dimensional space
$\mathcal F_+$ from (*) in the remark 
after Theorem 0.0.1.
There remains to find two linearly independent distributions in
$V_0$ since the leading polynomial of $P$ has a double zero at $x=0$.
To attain such a pair 
we first consider the boundary value distribution
\[
\gamma_*= f(x)\cdot \log (x-i0)+ g(x)
\]
which also is a null solution.
Here the multi-valuedness of the complex log-function entails that
\[
\gamma-\gamma_*=2\pi i\cdot f(x)\cdot H_-(x)
\]
where
$H_-(x)$ is the Heaviside distribution supported by
the negative half-line. Then
\[
\gamma^*= \gamma-\gamma_*-2\pi i\cdot f(x)
\]
is a null solution supportec by
the half-line $x\geq 0$ and hence belongs to 
the 2-dimensional space 
$V_0$. A second null solution in
$V_0$ is given by the 
 Dirac measure   $\delta_0$.
To see that $\delta_0$.
 is a null solution for $P$
we recall  that
in the non-commutative ring of differential operators one has the equality
$\nabla=\partial x\circ x-1$.
Since $x\cdot \delta_0=0$ we get 
the distribution equation
\[
\nabla(\delta_0)=-\delta_0\implies
\nabla^2(\delta_0)=\delta_0
\] 
Since $q(0)=-1$ is assumed in (0.3.1)
it follows that   $P(\delta_0)=0$.
Hence we have found four linearly independent null solutions
$f_+.f_-,\gamma^*,\delta_0$ in accordance with
Theolrem 0.0.1.
\medskip


\noindent
\emph{The fundamental solution.} 
A fundamental
solution $\mu$ supported by $x\geq 0$ is found as follows: From
(i) we have  the real-analytic density $\phi(x)$ on 
the open half-line $\{x>0\}$ which gives 
the  distribution
$\phi_+$  supported by $\{x\geq 0\}$ defined by
\[
\phi_+= f(x)\cdot (\log x)\cdot H_+ g(x)\cdot H_+
\]
In � xx we shall explain that
\[ 
\nabla(\log x\cdot H_+)=\delta_0
\]
and from this  deduce that
\[
P(\phi_+)=-\delta_0
\] 
Hence $\mu=-\phi_+$ gives the requested fundamental  solution.










\newpage


 


\centerline
{\bf{0.2 PDE-equations with constant coefficients.}}
\medskip


\noindent
The study of PDE-equations with contant coefficients in
${\bf{R}}^n$ for arbitrary $n\geq 2$
is a rich subject.
The interested reader may consult Chapter xx in
[H�rmander:Vol 2] for
an extensive study
of $PDE$-equations with constant coefficients.
Here we shall give 
a
construction from H�rmander's article [xxx]
which  
illustrates how analytic function theory can be used  with 
PDE-theory.
Fourier's inversion formula
for an arbitrary $n\geq 1$ asserts the following:
Let $f(x)= f(x_1,\ldots,x_n)$
be  a $C^\infty$-function which 
is rapidly decreasing as
$|x|= \sqrt{x_1^2+\ldots+x_n^2}$ tends to $+\infty$. Then
\[ 
f(x)= 
\frac{1}{(2\pi)^n}\cdot
\int\, e^{i\langle x,\xi\rangle}\cdot \widehat{f}(\xi)\, d\xi
\quad\,\text{where}\quad 
\widehat{f}(\xi)=\int\, e^{-i\langle x,\xi\rangle}\cdot f(x)\, dx\tag{*}
\] 
The inversion 
inversion formula (*) entails that the Fourier transform of
the partial derivative
$\frac{\partial f}{\partial x_j}(x)$ is equal to $i\xi_j\cdot
\widehat{f}(\xi)$.
In PDE-theory one  introduces the first order differential operators
\[ 
D_j=-i\cdot \frac{\partial}{\partial x_j}\quad\colon 1\leq j\leq n
\]
When $\alpha=(\alpha_1,\ldots,\alpha_n)$
is a multi-index we get the higher order differential operator
\[ 
D^\alpha=D_1^{\alpha_1}\cdots 
D_1^{\alpha_n}
\]
We can  take polynomials of these and get differential operatators with
constant coefficients
\[
 P(D)= \sum\, c_\alpha\cdot D^\alpha
\]
Fourier's inversion formula gives
\[ 
P(D)(f)(x)=
\frac{1}{2\pi^n}\cdot
\int\, e^{i\langle x,\xi\rangle}\cdot P(\xi)\cdot \widehat{f}(\xi)\, d\xi\tag{**}
\]
Thus, applying a differential operator with constant coefficients to $f$
corresponds to the product of
its Fourier transform with the polynomial $P(\xi)$
and 
(**) can be used to
construct solutions of the homogeneous equation $P(D)(f)=0$.
Following [H�rmander]
we  construct distributions
$\mu$ such that
$P(D)(\mu)=0$
for a suitable class of PDE-operators.
Let 
$\phi_1(s),\ldots,\phi_n(s)$ be some  $n$-tuple of analytic functions
of the complex variable $s$
which extend to continuous functions on the boundary of the domain
\[ 
\Omega=
\{\mathfrak{Im}(s)<0\}\cap \{|s|>M\}
\]
 where $M$ is some positive
number.
Assume that the $\phi$-functions satisfy
the growth conditions
\[
|\phi_k(s)|\leq C{|s|^a}\tag{i}
\] 
for a constant $C$ and some $0<a<1$.
If $a<\rho<1$ there exists
the analytic function in $\Omega$ defined by
\[
\psi(s)=e^{-(is)^\rho}
\]
As explained in � xx one has the estimate
\[
|\psi(s)|\leq e^{-\cos\frac {\pi\rho}{2}\cdot |s|^\rho }
\quad\colon \mathfrak{Im}\, s\leq 0
\]
The inequality $a<\rho$
and (i) entail that
the functions
\[ 
s\mapsto e^{ a_1\cdot \phi_1(s)+\dots+a_n\phi_n(s)}\cdot \psi(s)
\]
decrease like 
$e^{-\cos\frac {\pi\rho}{2}\cdot |s|^\rho }$ in $\Omega$.
\medskip

\noindent
{\bf{Exercise.}}
Verify 
that the complex line integrals
 below converge absolutely for every $s$-polynomial $Q(s)$ and
 every $n$-tuple of real numbers
 $x_1,\dots,x_n$:
 \[
 \frac{1}{(2\pi)^n}\cdot \int_{\partial\Omega}\,
 e^{x_1\phi_1(s)+\ldots+x_n\phi_n(s)}\cdot e^{ix_n s}\cdot
 Q(s)\cdot \psi(s)\, ds\tag{ii}
 \]
 and show  that when $x$ varies in ${\bf{R}}^n$
 this gives a $C^\infty$-function $f(x)$. If $1\leq j\leq n-1$
 one has for example
 \[
\frac{ \partial f}{\partial x_j}=
 \frac{1}{(2\pi)^n}\cdot \int_{\partial\Omega}\,\phi_j(s)\cdot 
 e^{x_1\phi_1(s)+\ldots+x_n\phi_n(s)}\cdot e^{ix_n s}\cdot
 Q(s)\cdot \psi(s)\, ds
\]
Less obvious  is that the
$C^\infty$-function $f(x)$ is supported by the half-space
$\{x_n\geq 0\}$.
To prove it one uses the analyticity of the integrand as a function of
$s$
which enable us to shift the contour of integration so that
(ii) is unchanged while we integrate on a horisontal line
$\mathfrak{Im}\, s=-N$ for every $N>M$.
With $s=u-iN$ we have
\[ 
|e^{ix_n\cdot s}|= e^{N\cdot x_n}
\]
If $x_n<0$ this term tends to zero as $N\to +\infty$
and from this the resder should confirm that
the $C^\infty$-function $f(x)$ is identically zero in
$\{x_n<0\}$.
\medskip

\noindent
Suppose now that we are given a PDE-operator $P(D)$
and  the $\phi$-functions  are chosen so
that
\[ 
s\mapsto P(\phi_1(s)\ldots \phi_{n-1}(s),\phi_n(s)+s)=0
\quad \colon s\in \Omega
\]
Then it is clear that
\[
P(D)( e^{x_1\phi_1(s)+\ldots+x_n\phi_n(s)}\cdot e^{ix_n s})=0
\]
hold for all $x\in{\bf{R}}^n$ and  $s\in\Omega$.
Hence 
$P(D)(f)=0$ where $f$ is a $C^\infty$-function supported by the half-space
$\{x_n\geq 0\}$.
In � xx we will show that the construction of solutions 
as above is not so special for
PDE-operators $P$
such that the hyperplane $\{x_n=0\}$ is non-characteristic.

































\newpage

\centerline {\bf{0.1 The distributions $x_+^s$}}
\medskip

\noindent
If $s$ is a complex number where
$\mathfrak{Re} s>-1$
the function defined by $x^s$ for $x>0$ and zero on
the half-line $x\leq 0$ is locally integrable and defines
a distribution denoted by $x_+^s$ acting on test-functions $g$ by
\[
x_+^s(g)= \int_0^\infty\, x^s\cdot g(x)\, dx
\]
The distribution-valued function
$s\mapsto x_+^s$ is analytic in
$\mathfrak{Re} s>-1$. Indeed
 if $x<0$ we have
 $\frac{d}{ds}(x^s=\log x\cdot x^s$
 which entails that
 the complex derivative
 of $x_+^s$ is the distribution defined by
 \[
 g\mapsto 
  \int_0^\infty\, \log x\cdot x^s\cdot g(x)\, dx
 \]
It turns out that $x_+^s$ extends to a meromorphic distribution-valued
function in the whole $s$-plane.
To prove this we perform a partial integration which
gives
\[
x_+^{s+1}(g')=
\int_0^\infty\, x^{s+1}\cdot g(x)\, dx=-(s+1)\cdot
 \int_0^\infty\, x^s\cdot g(x)\, dx\tag{0.0.1}
\]
 By the construction of distribution derivatives this means that
 \[
 \frac{d}{dx}(x_+^s+1)=(s+1)\cdot x_+^s
 \]


\noindent
{\bf{Euler's functional equation.}}
Set $\partial=\frac{d}{dx}$. We can iterate
(0.0.1) which  for every positive integer $m$ gives
\[
(s+1)\cdots(s+m)x_+^s=
\partial^m(x_+^{s+m})\tag{0.0.2}
\]
We refer to (0.0.2) as Euler's functional equation. It entails that
the distribution-valued function
$x_+^s$ extends to a meromorphic function with
at most simple poles at
negative integers.
Let us investigate the situation close to 
a
negative integer. With $s=-m+t$ and $t$ small one has
\[
t(t-1)\cdot (t-m+1)x_+^{-m+t}=\partial^m(x_+^t)
\]
When $x>0$ one has the expansion
\[ 
x^t=1+t\log x+\frac{t^2}{2}\cdot (\log x)^2+
\frac{t^3}{3 !}\cdot (\log x)^3+\ldots
\]
From this we obtain a series expansion
\[
x_+^{-m+t}=t^{-1}\cdot \rho_m+ \gamma_0+t\gamma_1+\ldots
\]
where $\rho_m$ and $\{\gamma_\nu\}$ are distributions.
In particular the reader may verify that
\[
(-1)^{m-1}(m-1)!\cdot \rho_m=\partial^m(H_+)
\]
Let us then consider the constant term $\gamma_0$.
The linear $t$-term in the expansion of
$t(t-1)\cdot (t-m+1)x_+^{-m+t}$
becomes
\[
(-1)^{m-1}(m-1)!\cdot \gamma_0+\frac{m(m-1)}{2}\cdot \rho_m
\]
If $x>0$ we notice that
\[
\partial^m(\log x)=(-1)^{m-1}\cdot (m-1)! \cdot x^{-m}
\]
From the above  $\gamma_0$ restricts to the density $x^{-m}$ when $x>0$.
At the same time $\gamma_0$ is a distribution defined
on the whole $x$--line supported by $\{x\geq 0\}$.
We set
\[ 
x_+^{-m}=\gamma_0\tag{*}
\]
and refer to this as Euler's extension of the density $x^{-m}$ which 
from the start is defined on
$\{x>0\}$.
So in (*) we have found
distributions for every positive integer $m$.
\medskip


\noindent
{\bf{0.1.2 Further formulas.}}
With $s=-1+z$ where $z$ is a small non-zero complex number
one has
\[
z\cdot x_+^{-1+z}=\partial(x_+^z)\tag{i}
\]
Next, if $x>0$ we have
\[
x^z= e^{z\log x}=
1+
\sum_{k=1}^\infty\, \frac{(\log x)^k}{k!}\cdot z^k
\]
Introducing the Heaviside distribution $H_+$ which is 1 on $x\geq 0$ and
zero on $x<0$
this means that
\[
\partial(x_+^z)=\partial(H_+)+
\sum_{k=1}^\infty\,
\partial(\frac{(\log x)^k}{k!}\cdot H_+(x))z^k\tag{ii}
\]
From this we get a Laurent expansion of
$x_+^s$ at $s=-1$. The crucial point
is that the distribution derivative
\[
\partial(H_+)=\delta_0\tag{iii}
\] 
where $\delta_0$ is the Dirac distribution at $x=0$.
It follows that
\[
x_+^{-1+z}= z^{-1}\cdot \delta_0+
\sum_{k=1}^\infty\,
\partial(\frac{(\log x)^k}{k!}\cdot H_+(x))z^{k-1}\tag{iv}
\]
In particular the constant term becomes
\[
\partial(\log x\cdot H_+(x))\tag{v}
\]
To find this distribution we take a test-function $g$
and  a partial integration gives
\[
-\int_0^\infty (\log x\cdot g'(x)\, dx
=\int_0^1\,\frac{g(x)-g(0)}{x}\, dx+
\int_1^\infty \,\frac{g(x)}{x}\, dx
\]
From this we conclude that
the  distribution $x_+^{-1}$ is defined on test-functions by
the formula:
\[
x_+^{-1}(g)=\frac{(-1)^{m-1}}{(m-1)!}\cdot
\int_0^1\,\frac{g(x)- g(0)}{x}\, dx+
\int_1^\infty \,\frac{g(x)}{x}\, dx
\]

\medskip

\noindent
{\bf{0.1.3 Exercise.}}
For each test-function $g$ and integer $m\geq 2$ we set
\[
T_{m-1}(g)(x)= g(0)+ g'(0)x\ldots \frac{g^{(m-1)}(0)}{(m-1)!}\cdot x^{m-1}
\]
Show from the above via  partial integrations
that 
\[
x_+^{-m}(g)=\frac{(-1)^{m-1}}{(m-1)!}\cdot
\int_0^1\,\frac{g(x)- T_{m-1}(g)(x)}{x^m}\, dx+
\int_1^\infty \,\frac{g(x)}{x^m}\, dx
\]
\medskip

\noindent
{\bf{0.1.4 The distributions $(x+i0)^\lambda$ and
$(x-i0)^\lambda$.}}
In the upper half plane there exists the single valued branch of
$\log z$ whose argument stays in $(0,\pi)$
and for every complex number $\lambda$ we have
\[ 
z^\lambda= e^{\lambda\cdot \log z}
\]
In � 3 we shall learn how to contruct boundary value
distributions  of analytic
functions defined in strip domains above or below the real $x$-line.
In particular there exists
the distribution  $(x+i0)^\lambda$
defined on test-functions $g(x)$ by the limit formula
\[
\lim_{\epsilon\to 0}\,
\int\, (x+i\epsilon)^\lambda\cdot g(x)\, dx
\]
Notice   that
this limit  exists for all complex $\lambda$,  i.e even when
the real part becomes very negative.
In the same way we have the single valued branch of $\log z$ in
the lower half-plane whose argument stays in
$(-\pi,0)$
and construct the distribution $(x-i0)^\lambda$
defined by 
\[
(x-i0)^\lambda(g)=
\lim_{\epsilon\to 0}\,
\int\, (x-i\epsilon)^\lambda\cdot g(x)\, dx
\]
Since $\lambda\mapsto e^{\lambda\cdot \log z}$
are entire in
$\lambda$, we get two entire distribution valued functions
by 
$(x-i0)^\lambda$ and $(x-i0)^\lambda$.
Regarding the choice of branches for the log-functions
we see that 
\[
(x-i0)^\lambda= e^{-2\pi i\lambda}\cdot (x+i0)^\lambda
\quad\colon x<0
\]
At the   same time
\[
(x+i0)^\lambda=(x-i0)^\lambda=x^\lambda
\quad\colon x>0
\]
From this we see that the distribution
\[
(x+i0)^\lambda-e^{2\pi i\lambda}\cdot (x-i0)^\lambda
\]
is supported by $x\geq 0$
and expressed by the density
$(1-e^{2\pi i\lambda})\cdot x^\lambda$.
The conclusion is that one has the equation
\[
\mu_\lambda=
\frac{(x+i0)^\lambda-e^{2\pi i\lambda}\cdot (x-i0)^\lambda}
{1-e^{2\pi i\lambda}}
\]

\medskip

\noindent
{\bf{Remark.}}
The equation (xx) is  more involved compared to the previous
description of the meromorphic $\mu$-function
found via Euler's functional equation.
But (xx) has the merit that
the denominator is an entire distribution valued function and
when one passes to Fourier transforms it turns out that
(xx) is quite useful.
\medskip

\noindent
{\bf{Principal value integrals.}}
If $g(x)$ is a test-function there exists a limit
\[
\lim_{\epsilon\to 0}
\int_{|x|>\epsilon}\, \frac{g(x)}{x}\, dx
\]
This yields a distribution denoted by $\text{VP}(x^{-1})$.
Outside $\{x=0\}$ it is given bt the density
$x^{-1}$
where it agrees with $(x+i0)^{-1}$ and
hence the difference
 \[ 
 \mu=
 \text{VP}(x^{-1})-(x+i0)^{.-1}
 \]
 is supported by $\{x=0\}$.
\medskip

\noindent
{\bf{Exercise.}} Notoice that
\[
\lim_{\epsilon\to 0}
\int_{|x|>\epsilon}\, \frac{1}{x+i\epsilon}\, dx
=\log (1+i\epsilon)-\log(-1+i\epsilon)=-\pi i
\]
and use this to show that
\[ 
\mu=-\pi i\cdot \delta_0
\]


 












 
 









\end{document}









